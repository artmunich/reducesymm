% siminos/cgang/def2modes.tex for 2modes.tex
% $Author$ $Date$

%%%%%%%%%%%%%%% DasBuch MACROS %%%%%%%%%%%%%%%%%%%%%%

\ifpaper % prepare for B&W paper printing:
       \newcommand{\HREF}[2]{{#2}}
       \renewcommand{\color}[1]{}       % B&W
       \newcommand{\wwwcb}[1]{{\tt ChaosBook.org#1}}
%       \newcommand{\weblink}[1]{{\tt #1}}
       \newcommand{\arXiv}[1]{ {\tt arXiv:#1}}
\else % prepare hyperlinked pdf
        \newcommand{\wwwcb}[1]{       % keep homepage flexible:
                  {\tt \href{http://ChaosBook.org#1}
              {ChaosBook.org#1}}}
%       \newcommand{\weblink}[1]{{\tt \href{http://#1}{#1}}}
       \newcommand{\HREF}[2]{
              {\href{#1}{#2}}}
       \newcommand{\arXiv}[1]{
              {\tt \href{http://arXiv.org/abs/#1}{arXiv:#1}}}
\fi

\ifpaper % prepare for B&W paper printing:
\else % prepare hyperlinked pdf
\fi %%%% prepare for B&W paper printing END %%%%%%


%%%%%%%%%%%%%%% EQUATIONS %%%%%%%%%%%%%%%%%%%%%%%%%%%%%%%
\newcommand{\beq}{\begin{equation}}
\newcommand{\continue}{\nonumber \\ }
\newcommand{\nnu}{\nonumber}
\newcommand{\eeq}{\end{equation}}
\newcommand{\ee}[1] {\label{#1} \end{equation}}
\newcommand{\bea}{\begin{eqnarray}}
\newcommand{\ceq}{\nonumber \\ & & }
\newcommand{\eea}{\end{eqnarray}}

\newcommand{\rf}     [1] {~\cite{#1}}
\newcommand{\refref} [1] {ref.~\cite{#1}}
\newcommand{\refRef} [1] {Ref.~\cite{#1}}
\newcommand{\refrefs}[1] {refs.~\cite{#1}}
\newcommand{\refRefs}[1] {Refs.~\cite{#1}}
\newcommand{\refeq}  [1] {(\ref{#1})}
\newcommand{\refeqs} [2]{(\ref{#1}--\ref{#2})}
\newcommand{\reffig} [1] {figure~\ref{#1}}
\newcommand{\reffigs} [2] {figures~\ref{#1} and~\ref{#2}}
\newcommand{\refFig} [1] {Figure~\ref{#1}}
\newcommand{\refFigs} [2] {Figures~\ref{#1} and~\ref{#2}}
\newcommand{\reftab} [1] {table~\ref{#1}}
\newcommand{\refTab} [1] {Table~\ref{#1}}
\newcommand{\reftabs}[2] {tables~\ref{#1} and~\ref{#2}}
\newcommand{\refsect}[1] {sect.~\ref{#1}}
\newcommand{\refsects}[2] {sects.~\ref{#1} and \ref{#2}}
\newcommand{\refSect}[1] {Sect.~\ref{#1}}
\newcommand{\refSects}[2] {Sects.~\ref{#1} and \ref{#2}}
\newcommand{\refsecttosect}[2] {Sects.~\ref{#1} to~\ref{#2}}
\newcommand{\refappe}[1] {appendix~\ref{#1}}
\newcommand{\refappes}[2] {appendices~\ref{#1} and~\ref{#2}}
\newcommand{\refAppe}[1] {Appendix~\ref{#1}}

%%%%%%%%%%%%%%%%%%%%%% CHAOS J SPECIFIC %%%%%%%%%%%%%%%%%%%%%%%%%%%%
\renewcommand{\reffig} [1] {Fig.~\ref{#1}}
\renewcommand{\reffigs} [2] {Figs.~\ref{#1} and~\ref{#2}}
\renewcommand{\refFig} [1] {Fig.~\ref{#1}}
\renewcommand{\refFigs} [2] {Figs.~\ref{#1} and~\ref{#2}}
\renewcommand{\refref} [1] {Ref.~\onlinecite{#1}}
\renewcommand{\refRef} [1] {Ref.~\onlinecite{#1}}
\renewcommand{\refrefs}[1] {Refs.~\onlinecite{#1}}
\renewcommand{\refRefs}[1] {Refs.~\onlinecite{#1}}
\newcommand{\refsec}[1] {Sect.~\onlinecite{#1}}

%%%%%%%%%%%%%%  Abbreviations %%%%%%%%%%%%%%%%%%%%%%%%%%%%%%%%%%%%%%%%
%%% APS (American Physiology Society, it seems) style:
%%%     Latin or foreign words or phrases should be roman, not italic.

\newcommand{\etc}{{etc.}}       % APS
\newcommand{\etal}{{\em et al.}}    % etal in italics, APS too
\newcommand{\ie}{{i.e.}}        % APS
\newcommand{\cf}{{\em cf.\ }}     % APS
\newcommand{\eg}{{e.g.\ }}        % APS, OUP, hard space '\eg\ NextWord'

%%%%%%%%%%%%%%% DasBuch MACROS %%%%%%%%%%%%%%%%%%%%%%

\newcommand{\po}{periodic orbit}
\newcommand{\Po}{Periodic orbit}
\newcommand{\rpo}{rela\-ti\-ve periodic orbit}
\newcommand{\Rpo}{Rela\-ti\-ve periodic orbit}
\newcommand{\eqv}{equi\-lib\-rium}
\newcommand{\Eqv}{Equi\-lib\-rium}
\newcommand{\eqva}{equi\-lib\-ria}
\newcommand{\Eqva}{Equi\-lib\-ria}
\newcommand{\reqv}{rela\-ti\-ve equi\-lib\-rium}
\newcommand{\Reqv}{Rela\-ti\-ve equi\-lib\-rium}
\newcommand{\reqva}{rela\-ti\-ve equi\-lib\-ria}
\newcommand{\Reqva}{Rela\-ti\-ve equi\-lib\-ria}
\newcommand{\reducedsp}{reduced state space}
\newcommand{\Reducedsp}{Reduced state space}
\newcommand{\nws}{non--wandering set}

\newcommand{\cycForm}{cycle averaging formula}
\newcommand{\CycForm}{Cycle averaging formula}

\newcommand{\evOper}{evolution oper\-ator}
\newcommand{\EvOper}{Evolution oper\-ator}
\newcommand{\FPoper}{Perron-Frobenius oper\-ator} % Pesin's ordering
\newcommand{\FP}{Perron-Frobenius}
\newcommand{\statesp}{state space}
\newcommand{\Statesp}{State space}
\newcommand{\jacobianM}{Jacobian matrix}  % back to Predrag's name 20oct2009
\newcommand{\jacobianMs}{Jacobian matrices}   % matrices
\newcommand{\JacobianM}{Jacobian matrix} %
\newcommand{\JacobianMs}{Jacobian matrices}  %
\newcommand{\FloquetM}{Floquet matrix} % specialized to periodic orb
\newcommand{\FloquetMs}{Floquet matrices}  %
\newcommand{\stabmat}{stability matrix}     % stability matrix, velocity gradients
\newcommand{\Stabmat}{Stability matrix}     % Stability matrix
\newcommand{\stabmats}{stability matrices}
\newcommand{\dzeta}{dyn\-am\-ic\-al zeta func\-tion}
\newcommand{\Dzeta}{Dyn\-am\-ic\-al zeta func\-tion}
\newcommand{\Fd}{spec\-tral det\-er\-min\-ant}
\newcommand{\dmn}{-dimensional}  %  experimental 220ct2009
\newcommand{\KS}{Kuramoto-Siva\-shin\-sky}
\newcommand{\KSe}{Kuramoto-Siva\-shin\-sky equation}
\newcommand{\pCf}{plane Couette flow}
\newcommand{\PCf}{Plane Couette flow}
\newcommand{\cLe}{complex Lorenz equations}
\newcommand{\cLf}{complex Lorenz flow}
\newcommand{\CLe}{Complex Lorenz equations}
\newcommand{\CLf}{Complex Lorenz flow}
\newcommand{\slice}{slice}
\newcommand{\Slice}{Slice}
\newcommand{\mslices}{method of slices}
\newcommand{\Mslices}{Method of slices}
\newcommand{\mframes}{method of moving frames}
\newcommand{\Mframes}{Method of moving frames}
\newcommand{\chartBord}{chart border}
\newcommand{\ChartBord}{Chart border}
\newcommand{\poincBord}{section border}
\newcommand{\PoincBord}{Section border}
\newcommand{\template}{template} % {slice-fixing point} % {reference state}
\newcommand{\sliceBord}{slice border}
\newcommand{\SliceBord}{Slice border}

\newcommand{\Un}[1]{\ensuremath{\textrm{U}(#1)}}         % in DasBuch
\newcommand{\SUn}[1]{\ensuremath{\textrm{SU}(#1)}}         % in DasBuch
\newcommand{\On}[1]{\ensuremath{\textrm{O}(#1)}}
\newcommand{\SOn}[1]{\ensuremath{\textrm{SO}(#1)}}         % in DasBuch

\newcommand{\braket}[2]
		   {\langle{#1}\vphantom{#2}|\vphantom{#1}{#2}\rangle}
\newcommand{\transp}[1]{{#1}{}^\top}
\newcommand{\norm}[1]{\left\Arrowvert \, #1 \, \right\Arrowvert}
\renewcommand{\det}{\mbox{\rm det}\,}
\newcommand{\tr}{\mbox{\rm tr}\,}
\newcommand{\pS}{\ensuremath{{\cal M}}}          % symbol for state space
\newcommand{\ssp}{\ensuremath{x}}                % state space point
\newcommand{\tildeL}{\ensuremath{\tilde{L}}}
\newcommand{\EQV}[1]{\ensuremath{EQ_{#1}}} %experimental
\newcommand{\REQV}[2]{\ensuremath{TW_{#1#2}}} % #1 is + or -
% TW_1^{+,-} for 1-wave traveling waves (positive and negative velocity).
\newcommand{\PO}[1]{\ensuremath{PO_{#1}}}
\newcommand{\RPO}[1]{\ensuremath{RPO_{#1}}}
\newcommand\stagn{q}      %equilibrium/stagnation point suffix
\newcommand{\rpprime}{{\tilde{p}}}  % relative periodic prime orbit
\newcommand{\shift}{\ensuremath{d}}
\newcommand{\Lop}{\ensuremath{{\cal L}}}       % evolution operator
\newcommand{\LopFP}{\ensuremath{{\cal L}_{PF}}} % Perron-Frob operator Dec 2012
\newcommand{\Aop}{\ensuremath{{\cal A}}}       % evolution generator
\newcommand{\matId}{\ensuremath{{\bf 1}}}      % matrix identity

%%%%%%%%%%%%%%% Sundry symbols within math eviron.: %%%%%%%%%%%%

\newcommand{\obser}{\ensuremath{a}}     % an observable from phase space to R^n
\newcommand{\Obser}{\ensuremath{A}}     % time integral of an observable
\newcommand{\pde}{\partial}
\newcommand{\reals}{\mathbb{R}}
\newcommand{\complex}{\mathbb{C}}

     %%%%%%%%%% periods: %%%%%%%%%%%%%%%%%%%%%%%%%%%%
\newcommand\period[1]{{\ensuremath{T_{#1}}}}         %continuous cycle period
\newcommand{\cl}[1]{{\ensuremath{n_{#1}}}}   % discrete length of a cycle, Predrag

     %%%%%%%%%% flows: %%%%%%%%%%%%%%%%%%%%%%%%%%%%
\newcommand\map{f}                  % other people like \phi's etc
\newcommand\flow[2]{{f^{#1}(#2)}}
\newcommand{\vel}{\ensuremath{v}}   % state space velocity
\newcommand{\Mvar}{\ensuremath{A}}  % stability matrix
\newcommand{\jMps}{\ensuremath{J}}   % jacobian matrix, phase space/state space
\newcommand{\ExpaEig}{\ensuremath{\Lambda}}
\newcommand{\Lyap}{\ensuremath{\lambda}}            %Lyapunov exponent
\newcommand{\monodromy}{\ensuremath{M}}   % monodromy matrix, full Poincare cut
\newcommand{\oneMinJ}[1]
           {\left|\det\!\left(\matId-\monodromy_p^{#1}\right)\right|}
\newcommand{\jEigvec}[1][]{\ensuremath{{\bf e}^{(#1)}}} % right jacobiam eigenvector
\newcommand{\jEigvecT}[1][]{\ensuremath{{\bf e}_{(#1)}}} % left jacobiam eigenvector
\newcommand{\eigExp}[1][]{
     \ifthenelse{\equal{#1}{}}{\ensuremath{\lambda}}{\ensuremath{\lambda^{(#1)}}}}
\newcommand{\eigRe}[1][]{
     \ifthenelse{\equal{#1}{}}{\ensuremath{\mu}}{\ensuremath{\mu^{(#1)}}}}
\newcommand{\eigIm}[1][]{
     \ifthenelse{\equal{#1}{}}{\ensuremath{\omega}}{\ensuremath{\omega^{(#1)}}}}

\newcommand{\cycle}[1]{{\ensuremath{\overline{#1}}}}
\newcommand{\prune}[1]{\ensuremath{\_{#1}\_}}        % fits into math env.

%%%%%%%%%%%%%%% LIE GROUP PARAMETRIZATIONS %%%%%%%%%%%%%%%%%%%%%%
\newcommand{\gSpace}{\ensuremath{{\bf \phi}}}   % MA group rotation parameters
\newcommand{\velRel}{\ensuremath{c}}    % relative state or phase velocity
\newcommand{\phaseVel}{phase velocity}      % pipe slicing
\newcommand{\phaseVels}{phase velocities}   % pipe slicing
\newcommand{\PhaseVel}{Phase velocity}      % pipe slicing
\newcommand{\PhaseVels}{Phase velocities}   % pipe slicing


%%%%%%%%%%%%%%%%%%%%%% TWO-MODE SPECIFIC %%%%%%%%%%%%%%%%%%%%%%%%%%%%%%%

%%% This in Daniel's 05.15.2014 attempt at aligning notation for two mode paper
%%% with BudCvi14 and Atlas
\renewcommand{\ssp}{\ensuremath{a}} % Real valued statespace variable
\renewcommand{\rpprime}{\ensuremath{p}}
\newcommand{\pd}{\ensuremath{P}}
\newcommand{\invpol}{\ensuremath{p}} % Invariant polynomials
% Coordinate systems
\newcommand{\cartpt}[1]{$\left( #1 \right)$} % Points in full Cartesian 4D space
\newcommand{\polpt}[1]{$\left\{ #1 \right\}$} % Points in polar coordinates
\newcommand{\invpt}[1]{$\left[ #1 \right]$} % Points in invariant polynomial basis

\renewcommand{\shift}{\ensuremath{\ell}}
\newcommand{\Template}{Template}
\newcommand\mapRed{\ensuremath{\hat{f}}}     % other people like \phi's etc
\newcommand\flowRed[2]{\ensuremath{\hat{f}^{#1}(#2)}}
\newcommand{\wurst}{wurst}
\newcommand{\Wurst}{Wurst}
%\newcommand{\twomode}{Porter-Knobloch}
%\newcommand{\twoMode}{Porter-Knobloch}
\newcommand{\twoMode}{Two-mode}
\newcommand{\twomode}{two-mode}
\newcommand{\slicePlane}{slice hyperplane}
\newcommand{\SlicePlane}{Slice hyperplane}
\newcommand{\comovframe}{comoving frame}
\newcommand{\comovFrame}{Comoving frame}
% \newcommand{\mconn}{method of connections}
% \newcommand{\Mconn}{Method of connections}
\newcommand{\mconn}{method of \comovframe s}
\newcommand{\Mconn}{Method of \comovframe s}

\newcommand{\pSRed}{\ensuremath{\hat{\cal M}}} % reduced state space Jan 2012
\newcommand{\sspRed}{\ensuremath{\hat{\ssp}}}    % reduced state space point Jan 2012
\newcommand{\velRed}{\ensuremath{\hat{\vel}}}    % ES reduced state space velocity Jan 2012
\newcommand{\slicep}{{\ensuremath{\sspRed'}}}   % slice-fixing point Jan 2012
\newcommand{\sliceTan}[1]{\ensuremath{t'_{#1}}}    % group orbit tangent at slice-fixing
\newcommand{\groupTan}{\ensuremath{t}}    % group orbit tangent
\newcommand{\Group}{\ensuremath{G}}         % Predrag Lie or discrete group
\newcommand{\Lg}{\ensuremath{\mathbf{T}}}   % Predrag Lie algebra generator
\newcommand{\LieEl}{\ensuremath{g}}  %  Lie group element

\newcommand{\zeit}{\ensuremath{\tau}}  %time variable
\newcommand{\sspRSing}{\ensuremath{\sspRed^\ast}} 	% inflection point, reduced space
\newcommand{\sspC}{\ensuremath{z}} %Complex valued state space variable
\newcommand{\sspRedC}{\ensuremath{\hat{\sspC}}}
\newcommand{\conf}{\ensuremath{x}} %Configuration space coordinate


\ifdraft    %%%%%%%%%%% display comments in text %%%%%%%%%%%%%%%%%%
   \newcommand{\PublicPrivate}[2]
       {\marginpar{\color{blue}$\Downarrow$}%
       {\color{blue}#2}%
       \marginpar{\color{blue}$\Uparrow$}}
   \newcommand{\toCB}{$\footnotemark\footnotetext{2CB}$}  % to compare with ChaosBook
   \newcommand{\PC}[1]{$\footnotemark\footnotetext{PC: {\color{blue}#1}}$}
   \newcommand{\PCedit}[1]{{\color{blue}#1}}
   \newcommand{\MA}[1]{$\footnotemark\footnotetext{MA: #1}$}
   \newcommand{\MAedit}[1]{{\color{green}#1}}
   \newcommand{\APW}[1]{$\footnotemark\footnotetext{APW: #1}$}
   \newcommand{\APWedit}[1]{{\color{green}#1}}
   \newcommand{\DB}[2]{$\footnotemark\footnotetext{DB #1: {\color{red}#2}}$} %date, comment
   \newcommand{\DBedit}[1]{{\color{red}#1}}
   \newcommand{\KC}[2]{$\footnotemark\footnotetext{KC #1: #2}$} %date, comment
   \newcommand{\KCedit}[1]{{\color{green}#1}}
   \newcommand{\LZ}[2]{$\footnotemark\footnotetext{LZ #1: #2}$} %date, comment
   \newcommand{\LZedit}[1]{{\color{green}#1}}
   \newcommand{\ES}[2]{$\footnotemark\footnotetext{ES #1: {\color{magenta}#2}}$} %date, comment
   \newcommand{\ESedit}[1]{{\color{magenta}#1}}
   \newcommand{\BB}[2]{$\footnotemark\footnotetext{BB #1: {\color{green}#2}}$} %date, comment
   \newcommand{\BBedit}[1]{{\color{green}#1}}
   \newcommand{\mycomment}[2]{\noindent \textbf{\underline{#1}}: \emph{#2}}
   \newcommand{\edit}[1]{{\color{blue}#1}} % for referees
\else   % drop comments
   \newcommand{\PublicPrivate}[2]{#1}
   \newcommand{\toCB}{}
   \newcommand{\PC}[1]{}
   \newcommand{\MA}[1]{}
   \newcommand{\APW}[1]{}
%   \newcommand{\PCedit}[1]{{\color{magenta}#1}}  % for referees
   \newcommand{\PCedit}[1]{#1}
   \newcommand{\MAedit}[1]{#1}
%   \newcommand{\APWedit}[1]{{\color{green}#1}} % for referees
   \newcommand{\APWedit}[1]{#1}
   \newcommand{\DB}[2]{}{}
   \newcommand{\DBedit}[1]{#1}
   \newcommand{\KC}[2]{}{}
   \newcommand{\KCedit}[1]{#1}
   \newcommand{\LZ}[2]{}{}
   \newcommand{\LZedit}[1]{#1}
   \newcommand{\ES}[2]{}{}
   \newcommand{\ESedit}[1]{#1}
   \newcommand{\BB}[2]{}{}
   \newcommand{\BBedit}[1]{#1}
   \newcommand{\mycomment}[2]{}
   \newcommand{\edit}[1]{{\color{blue}#1}} % for referees
%   \newcommand{\edit}[1]{#1}               % for the journal

\fi %%%%% COMMENTS END %%%%%%%%%%%%%%%

\newcommand{\stateDsp}{state-space}
\newcommand{\StateDsp}{State-space}
%\newcommand{\reqvD}{traveling-wave}
%\newcommand{\ReqvD}{Traveling-wave}
%\newcommand{\reqvaD}{traveling-waves}
%\newcommand{\ReqvaD}{Traveling-waves}
%\newcommand{\REqvaD}{Traveling-Waves}

%%%% 3D physical flow
\newcommand{\NS}{Navier-Stokes}
\newcommand{\NSe}{Navier-Stokes equations}
\newcommand{\Reynolds}{\textit{Re}}  % Reynolds number
\newcommand{\cohStr}{coherent structure}
\newcommand{\recurrStr}{recurrent coherent structure}
\newcommand{\RecurrStr}{Recurrent coherent structure}
\newcommand{\bnabla}{\ensuremath{\bf \nabla}}
\newcommand{\Gpipe}{\ensuremath{\Gamma}} % Hoyle notation, equivariant symmetry group
\newcommand{\bCell}{\ensuremath{\Omega}}
\newcommand{\normVec}{\ensuremath{\mathbf{n}}}    % group orbit curvature normal

%%%%%%%%%%%%%%%%%%%%%%%%%%%%%%%%%%%%%%%%%%%%%%%%%%%%%%%%%%%
\renewcommand{\shift}{\ensuremath{\ell}}
\newcommand{\bu}{\ensuremath{{\bf u}}}
\newcommand{\be}{{\bf e}}
\newcommand{\bx}{{\bf x}}
\newcommand{\Norm}[1]{\|{#1}\|}
\newcommand{\ii}{\ensuremath{\mathrm{i}}} % sqrt{-1}

%%%%%%%%%%%%%%% DasBuch MACROS %%%%%%%%%%%%%%%%%%%%%%

%  \SFIG{#1}    % f_name.png (or .pdf)
%       {#2}    % short caption text
%       {#3}    % full caption text
%       {#4}    % f-figure-label
\newcommand{\SFIG}[4]{\begin{figure}[h]
              %\hspace*{-0.10\textwidth}
              \hspace*{0.10\textwidth}
              \begin{minipage}[b]{0.55\textwidth}
                      \caption[#2]{#3}
                      \label{#4}
              \end{minipage}~~~~~%
              \begin{minipage}[b]{0.40\textwidth}
                      \includegraphics[width=1.00\textwidth]{#1}
              \end{minipage}
              %\hfill
              \end{figure} }

\newcommand{\CostFct}{Cost function}    % functional to minimize
\newcommand{\costFct}{cost function}    % functional to minimize
\newcommand{\pVeloc}{v}         % phase-space velocity
