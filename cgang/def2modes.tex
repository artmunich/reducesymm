% siminos/cgang/def2modes.tex for 2modes.tex
% $Author$ $Date$

%%%%%%%%%%%%%%%%%%%%%% CHAOS J SPECIFIC %%%%%%%%%%%%%%%%%%%%%%%%%%%%
\renewcommand{\reffig} [1] {Fig.~\ref{#1}}
\renewcommand{\reffigs} [2] {Figs.~\ref{#1} and~\ref{#2}}
\renewcommand{\refFig} [1] {Fig.~\ref{#1}}
\renewcommand{\refFigs} [2] {Figs.~\ref{#1} and~\ref{#2}}
\renewcommand{\refref} [1] {Ref.~\onlinecite{#1}}
\renewcommand{\refRef} [1] {Ref.~\onlinecite{#1}}
\renewcommand{\refrefs}[1] {Refs.~\onlinecite{#1}}
\renewcommand{\refRefs}[1] {Refs.~\onlinecite{#1}}
\newcommand{\refsec}[1] {Sect.~\onlinecite{#1}}

%%%%%%%%%%%%%%%%%%%%%% TWO-MODE SPECIFIC %%%%%%%%%%%%%%%%%%%%%%%%%%%%%%%

%%% This in Daniel's 05.15.2014 attempt at aligning notation for two mode paper
%%% with BudCvi14 and Atlas
\renewcommand{\ssp}{\ensuremath{a}}
\renewcommand{\rpprime}{\ensuremath{p}}
\newcommand{\pd}{\ensuremath{P}}
% Coordinate systems
\newcommand{\cartpt}[1]{$\left( #1 \right)$} % Points in full Cartesian 4D space
\newcommand{\polpt}[1]{$\left\{ #1 \right\}$} % Points in polar coordinates
\newcommand{\invpt}[1]{$\left[ #1 \right]$} % Points in invariant polynomial basis

\renewcommand{\shift}{\ensuremath{\ell}}
\newcommand{\Template}{Template}
\newcommand\mapRed{\ensuremath{\hat{f}}}     % other people like \phi's etc
\newcommand\flowRed[2]{\ensuremath{\hat{f}^{#1}(#2)}}
\newcommand{\wurst}{wurst}
\newcommand{\Wurst}{Wurst}
%\newcommand{\twomode}{Porter-Knobloch}
%\newcommand{\twoMode}{Porter-Knobloch}
\newcommand{\twoMode}{Two-mode}
\newcommand{\twomode}{two-mode}
\newcommand{\slicePlane}{slice hyperplane}
\newcommand{\SlicePlane}{Slice hyperplane}
\newcommand{\comovframe}{comoving frame}
\newcommand{\comovFrame}{Comoving frame}
% \newcommand{\mconn}{method of connections}
% \newcommand{\Mconn}{Method of connections}
\newcommand{\mconn}{method of \comovframe s}
\newcommand{\Mconn}{Method of \comovframe s}

\renewcommand{\zeit}{\ensuremath{\tau}}  %time variable Predrag

\newcommand{\sspC}{\ensuremath{z}}
\newcommand{\sspRedC}{\ensuremath{\hat{z}}}

\ifdraft    %%%%%%%%%%% display comments in text %%%%%%%%%%%%%%%%%%
   \newcommand{\PublicPrivate}[2]
       {\marginpar{\color{blue}$\Downarrow$}%
       {\color{blue}#2}%
       \marginpar{\color{blue}$\Uparrow$}}
   \newcommand{\toCB}{$\footnotemark\footnotetext{2CB}$}  % to compare with ChaosBook
   \newcommand{\PC}[1]{$\footnotemark\footnotetext{PC: {\color{blue}#1}}$}
   \newcommand{\PCedit}[1]{{\color{blue}#1}}
   \newcommand{\MA}[1]{$\footnotemark\footnotetext{MA: #1}$}
   \newcommand{\MAedit}[1]{{\color{green}#1}}
   \newcommand{\APW}[1]{$\footnotemark\footnotetext{APW: #1}$}
   \newcommand{\APWedit}[1]{{\color{green}#1}}
   \newcommand{\DB}[2]{$\footnotemark\footnotetext{DB #1: {\color{red}#2}}$} %date, comment
   \newcommand{\DBedit}[1]{{\color{red}#1}}
   \newcommand{\KC}[2]{$\footnotemark\footnotetext{KC #1: #2}$} %date, comment
   \newcommand{\KCedit}[1]{{\color{green}#1}}
   \newcommand{\LZ}[2]{$\footnotemark\footnotetext{LZ #1: #2}$} %date, comment
   \newcommand{\LZedit}[1]{{\color{green}#1}}
   \newcommand{\ES}[2]{$\footnotemark\footnotetext{ES #1: {\color{magenta}#2}}$} %date, comment
   \newcommand{\ESedit}[1]{{\color{magenta}#1}}
   \newcommand{\BB}[2]{$\footnotemark\footnotetext{BB #1: {\color{green}#2}}$} %date, comment
   \newcommand{\BBedit}[1]{{\color{green}#1}}
   \newcommand{\mycomment}[2]{\noindent \textbf{\underline{#1}}: \emph{#2}}
   \newcommand{\edit}[1]{{\color{blue}#1}} % for referees
\else   % drop comments
   \newcommand{\PublicPrivate}[2]{#1}
   \newcommand{\toCB}{}
   \newcommand{\PC}[1]{}
   \newcommand{\MA}[1]{}
   \newcommand{\APW}[1]{}
%   \newcommand{\PCedit}[1]{{\color{magenta}#1}}  % for referees
   \newcommand{\PCedit}[1]{#1}
   \newcommand{\MAedit}[1]{#1}
%   \newcommand{\APWedit}[1]{{\color{green}#1}} % for referees
   \newcommand{\APWedit}[1]{#1}
   \newcommand{\DB}[2]{}{}
   \newcommand{\DBedit}[1]{#1}
   \newcommand{\KC}[2]{}{}
   \newcommand{\KCedit}[1]{#1}
   \newcommand{\LZ}[2]{}{}
   \newcommand{\LZedit}[1]{#1}
   \newcommand{\ES}[2]{}{}
   \newcommand{\ESedit}[1]{#1}
   \newcommand{\BB}[2]{}{}
   \newcommand{\BBedit}[1]{#1}
   \newcommand{\mycomment}[2]{}
   \newcommand{\edit}[1]{{\color{blue}#1}} % for referees
%   \newcommand{\edit}[1]{#1}               % for the journal

\fi %%%%% COMMENTS END %%%%%%%%%%%%%%%

\ifpaper % prepare for B&W paper printing:
       \renewcommand{\arXiv}[1]{ {\tt arXiv:#1}}
       \renewcommand{\mpArc}[1]{{\tt mp\_arc~#1}}
\else % prepare hyperlinked pdf
       \renewcommand{\mpArc}[1]{
              {\tt \href{http://www.ma.utexas.edu/mp_arc-bin/mpa?yn=#1}
                   {mp\_arc~#1}}}
       \renewcommand{\arXiv}[1]{
              {\tt \href{http://arXiv.org/abs/#1}{arXiv:#1}}}
\fi %%%% prepare for B&W paper printing END %%%%%%

\newcommand{\stateDsp}{state-space}
\newcommand{\StateDsp}{State-space}
%\newcommand{\reqvD}{traveling-wave}
%\newcommand{\ReqvD}{Traveling-wave}
%\newcommand{\reqvaD}{traveling-waves}
%\newcommand{\ReqvaD}{Traveling-waves}
%\newcommand{\REqvaD}{Traveling-Waves}

%%%% 3D physical flow
\newcommand{\NS}{Navier-Stokes}
\newcommand{\NSe}{Navier-Stokes equations}
\newcommand{\Reynolds}{\textit{Re}}  % Reynolds number
\newcommand{\cohStr}{coherent structure}
\newcommand{\recurrStr}{recurrent coherent structure}
\newcommand{\RecurrStr}{Recurrent coherent structure}
\newcommand{\bnabla}{\ensuremath{\bf \nabla}}
\newcommand{\Gpipe}{\ensuremath{\Gamma}} % Hoyle notation, equivariant symmetry group
\newcommand{\bCell}{\ensuremath{\Omega}}
\newcommand{\normVec}{\ensuremath{\mathbf{n}}}    % group orbit curvature normal

%%%%%%%%%%%%%%%%%%%%%%%%%%%%%%%%%%%%%%%%%%%%%%%%%%%%%%%%%%%
\renewcommand{\shift}{\ensuremath{\ell}}
\newcommand{\bu}{\ensuremath{{\bf u}}}
\newcommand{\be}{{\bf e}}
\newcommand{\bx}{{\bf x}}
\newcommand{\Norm}[1]{\|{#1}\|}
\newcommand{\ii}{\ensuremath{\mathrm{i}}} % sqrt{-1}

%%%%%%%%%%%%%%% DasBuch MACROS %%%%%%%%%%%%%%%%%%%%%%

%  \SFIG{#1}    % f_name.png (or .pdf)
%       {#2}    % short caption text
%       {#3}    % full caption text
%       {#4}    % f-figure-label
\newcommand{\SFIG}[4]{\begin{figure}[h]
              %\hspace*{-0.10\textwidth}
              \hspace*{0.10\textwidth}
              \begin{minipage}[b]{0.55\textwidth}
                      \caption[#2]{#3}
                      \label{#4}
              \end{minipage}~~~~~%
              \begin{minipage}[b]{0.40\textwidth}
                      \includegraphics[width=1.00\textwidth]{#1}
              \end{minipage}
              %\hfill
              \end{figure} }
