\section{Multiple shooting method for finding \rpo s}
\label{s:newton}

Let us assume that we have a set of good guesses
for a set of \statesp\ points,
flight times and
$1D$ symmetry group parameter increments
$\{\ssp_i^{(0)}\,,\,\zeit_i^{(0)}\,,\,\gSpace_i^{(0)}\}$
such that the points
$\{\ssp_i^{(0)}\}$
lie close to the \rpo\ $p$,
\beq
	\ssp_{i+1}^{(0)}
\approx
    \matrixRep(\gSpace_i^{(0)}) \flow{{\zeit_i^{(0)}}}{\ssp_i^{(0)}}
\quad
    \mbox{cyclic in $i = 1, ..., n$}
\,.
\eeq
Here the period and the shift of the \rpo\ $p$ are
$\period{p} \approx \sum \zeit_i\,,$
$\gSpace_p \approx \sum \gSpace_i \,.$
and the Lagrangian description of the flow is
$\ssp(\zeit) = \flow{\zeit}{\ssp(0)}$
We want to determine
$(\Delta\ssp_i\,,\,\Delta\zeit_i\,,\,\Delta\gSpace_i)$
corresponding to the exact \rpo,
\bea
	\ssp_{i+1} + \Delta \ssp_{i+1} &=& \matrixRep(\gSpace_i + \Delta \gSpace_i)
		\flow{\zeit_i + \Delta \zeit_i}{\ssp_i + \Delta \ssp_i} \continue
		&&  \mbox{cyclic in } i = 1, ..., n
\,.
\eea
To linear order in
\bea
&& (\Delta\ssp_i^{(m+1)}\,,\, \Delta\zeit_i^{(m+1)}\,,\,\Delta\gSpace_i^{(m+1)}) \\
&&=
(\ssp_i^{(m+1)}-\ssp_i^{(m)}\,,\,
 \zeit_i^{(m+1)}-\zeit_i^{(m)}\,,\,
 \gSpace_i^{(m+1)}-\gSpace_i^{(m)}) \nonumber
\eea
the improved Newton guess
$
(\ssp_i^{(m+1)}\,,\,\zeit_i^{(m+1)}\,,\,\gSpace_i^{(m+1)} )
$
is obtained by minimizing the effect of perturbations along
the spatial, time and phase directions,
    \PC{2014-02-02 I have changed Burak's \refeq{PCnetwonStep} by
    $\Lg \flow{\zeit_i}{\ssp_i}
        \to
    \Lg \matrixRep(\gSpace_i)\flow{\zeit_i}{\ssp_i}$.
    The rest of the formulas need to be changed accordingly.}
    \PC{2014-02-02 to Burak: please write up the detailed
    derivation, with what $O(\Delta^2)$'s are dropped
    as a problem / solution set for \texttt{cycles.tex}.}
\beq
	\ssp_{i+1}^{'} - \matrixRep_{i+1} \flow{{\zeit_i}}{\ssp_i}
= \matrixRep_{i+1}\left(
  \groupTan_{i+1} \Delta \gSpace_i
+ \vel_{i+1} \Delta \zeit_i
+ \jMps_{i+1} \Delta \ssp_i
    \right)
\,,
\ee{PCnetwonStep}
where, for brevity,
$\ssp_{i}^{(m+1)} = \ssp_{i}^{(m)} + \Delta \ssp_{i}^{(m)}
   = \ssp_{i}^{'}$,
$\ssp_{i}^{(m)} = \ssp_{i}$,
$\matrixRep(\gSpace_i) = \matrixRep_{i+1}$,
$\vel (\ssp_{i}(\zeit_{i})) = \vel_{i+1}$,
$\jMps^{\zeit_i}(\ssp_i) = \jMps_{i+1}$,
$\groupTan(\ssp_{i}(\zeit_{i})) = \Lg \ssp_{i}(\zeit_{i}) = \groupTan_{i+1}$,
\etc.
For sufficiently good initial guesses,
the improved values converge under Newton iterations to
the exact values
$(\Delta\ssp_i\,,\,\Delta\zeit_i\,,\,\Delta\gSpace_i)$
=$(\Delta\ssp_i^{(\infty)}\,,\,\Delta\zeit_i^{(\infty)}\,,\,\Delta\gSpace_i^{(\infty)})$
at a super-exponential rate.
    \PC{2014-02-02 I am often running into the situation that the linear
    operators are better marked by the final rather than by the initial
    points of trajectory segments, as in $\jMps^{\zeit_i}(\ssp_i) =
    \jMps_{i+1}$. Implementing this requires a huge rewrite of the
    ChaosBook}
%
%Regrouping terms we get a form that we can turn to a matrix equation:
%\beq
%	 \LieEl (\gSpace_i) \jMps^{\zeit_i}(\ssp_i) \Delta \ssp_i
%	+ \LieEl (\gSpace_i) \vel (\flow{\zeit_i}{\ssp_i}) \Delta \zeit_i
%	+\Lg \flow{\zeit_i}{\ssp_i} \Delta \gSpace_i
%	- \Delta \ssp_{i+1}
%	= \ssp_{i+1} - \LieEl (\gSpace_i) \flow{\zeit_i}{\ssp_i}
%\eeq
In order to deal with the marginal multipliers along the time and group
orbit directions, one needs to apply a pair of constraints, which
eliminate variations along the marginal directions on the \rpo s\ $2D$
torus: a local Poincar\'e section orthogonal to the flow, and a local slice
orthogonal to the group orbit at each point along the orbit,
\beq
   \vel(\ssp_i ) \cdot \Delta \ssp_i = 0
\,,\qquad
   \groupTan(\ssp_i ) \cdot \Delta \ssp_i = 0
\,.
\ee{RPOConstrsLocal}
We can rewrite everything as one matrix equation:
\beq \label{eq:multishootmatrix}
	A \Delta = E, \quad \mbox{where,}
\eeq
\begin{widetext}
\bea 
	A &=& \left(
	\begin{array}{ccccccccccc}	
	  \matrixRep_{2} \jMps_{2} &
	  \matrixRep_{2} \vel_2 &
	  \Lg \LieEl_{2} \flow{\zeit_1}{\ssp_1} &
	  - \matId & 0 & 0 & 0 & \cdots & 0 & 0 & 0 \\
	  \vel(\ssp_1) & 0 & 0 & 0 & 0 & 0 & 0 & \cdots & 0 & 0 & 0 \\
	  \groupTan(\ssp_1) & 0 & 0 & 0 & 0 & 0 & 0 & \cdots & 0 & 0 & 0 \\
	  0 & 0 & 0 &
	  \matrixRep_{3} \jMps_{3} &
	  \matrixRep_{3} \vel_3 &
	  \Lg \LieEl_{3} \flow{\zeit_2}{\ssp_2}   &
	  - \matId & \cdots & 0 & 0 & 0\\
	  0 & 0 & 0 & \vel(\ssp_2) & 0 & 0 & 0 & \cdots & 0 & 0 & 0 \\
	  0 & 0 & 0 & \groupTan(\ssp_2) & 0 & 0 & 0 & \cdots & 0 & 0 & 0 \\
	  \vdots & \vdots & \vdots & \vdots & \vdots & \vdots & \vdots & \ddots & \vdots & \vdots & \vdots \\
	  - \matId & 0 & 0 & 0 & 0 & 0 & 0 & \cdots &
	  \matrixRep_{1} \jMps_{1} &
	  \matrixRep_{1} \vel_1 &
	  \Lg \matrixRep_{1} \flow{\zeit_1}{\ssp_1} \\
	  0 & 0 & 0 & 0 & 0 & 0 & 0 & \cdots & \vel(\ssp_n) & 0 & 0 \\
	  0 & 0 & 0 & 0 & 0 & 0 & 0 & \cdots & \groupTan(\ssp_n) & 0 & 0
	\end{array} \right) \label{eq:AforNewton} \\
	\Delta &=&
	 (
	  \Delta \ssp_1, \,
	  \Delta \zeit_1, \,
	  \Delta \gSpace_1, \,
	  \Delta \ssp_2, \,
	  \Delta \zeit_2, \,
	  \Delta \gSpace_2, \,
	  \ldots , \,
	  \Delta \ssp_n, \,
	  \Delta \zeit_n, \,
	  \Delta \gSpace_n
	 )^T 
	 \continue	 
	 E &=&
	 (
	  \ssp_{2} - \matrixRep_2 \flow{\zeit_1}{\ssp_1} , \,
	   0 	, \,
	   0 	, \,
	  \ssp_{3} - \matrixRep_3 \flow{\zeit_2}{\ssp_2} , \,
	  0 	, \,
	  0 	, \,
	  \ldots , \,
	  \ssp_{1} - \matrixRep_1 \flow{\zeit_n}{\ssp_n} , \,
	  0 	, \,
	  0 	
	  )^T \label{eq:DeltaandE}	 
\eea
\end{widetext}
%\beq \label{eq:DeltaandE}
%	\Delta =
%	 \begin{pmatrix}
%	  \Delta \ssp_1 \\
%	  \Delta \zeit_1 	\\
%	  \Delta \gSpace_1 \\
%	  \Delta \ssp_2 \\
%	  \Delta \zeit_2 	\\
%	  \Delta \gSpace_2 \\
%	  \vdots \\
%	  \Delta \ssp_n \\
%	  \Delta \zeit_n 	\\
%	  \Delta \gSpace_n \\
%	 \end{pmatrix}  	
%	 \ \mbox{and} \
%	 E =
%	 \begin{pmatrix}
%	  \ssp_{2} - \LieEl_2 \flow{\zeit_1}{\ssp_1} \\
%	   0 	\\
%	   0 	\\
%	  \ssp_{3} - \LieEl_3 \flow{\zeit_2}{\ssp_2} \\
%	  0 	\\
%	  0 \\
%	  \vdots \\
%	  \ssp_{1} - \LieEl_1 \flow{\zeit_n}{\ssp_n} \\
%	  0 	\\
%	  0 	\\
%	 \end{pmatrix}
%\eeq
We than solve \refeq{eq:multishootmatrix} for $\Delta$ and update our initial
guess by adding the vector of the computed $\Delta$ values, and iterate.
