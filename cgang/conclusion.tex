\section{Conclusions and discussion}
\label{s:concl}

In this tutorial, we have studied \revision{a simple} dynamical 
system which exhibits chaos and is equivariant under a continuous 
symmetry transformation. We have shown that reducing this symmetry 
simplifies the qualitative dynamics to a great extent and enables 
one to find all \rpo s of the systems via standard techniques such 
as Poincar\'e sections and return maps. In addition, we have shown 
that one can extract quantitative information from the \rpo s by 
computing cycle averages.

We motivated our study of the \twomode\ system by the resemblance of its
symmetry structure to that of the spatially extended systems; and the
steps we outlined here are, in principle, applicable to physical systems
that are described by $N$-Fourier mode truncations of PDEs such as $1D$
\KS\rf{SCD07}, $3D$ pipe flows\rf{WiShCv14}, \etc.

We showed in \refsect{s:numerics} that \twomode\ dynamics can be 
completely described by a unimodal return map on the Poincar\'e 
section we constructed after continuous symmetry reduction. In a 
high-dimensional system, finding such an easy symbolic dynamics, 
or any symbolic dynamics at all is a challenging problem on its 
own. For desymmetrized (confined in the odd subspace) $1$D 
spatio-temporally chaotic \KS\ system \refref{lanCvit07} finds a 
bimodal return map after reducing the discrete symmetry of the 
problem, however, for turbulent fluids, we do not know any study 
that simplifies the flow to such an extent.

In \refsect{s:DynAvers}, we have shown that the symbolic dynamics and the
associated grammar rules greatly affect the convergence of \cycForm s.
In general, finding a finite symbolic description of a flow is
rarely as easy as it is in our model system.
There exist other methods of ordering cycle
expansion terms, for example, ordering pseudo-cycles by their stability and discarding terms
that are above a threshold\rf{DM97}; one expects the remaining terms to form
shadowing combinations and converge exponentially.
Whichever method of term ordering is deployed, the cycle expansions are only as good
as the least unstable cycle that one fails to find. Symbolic dynamics solves both
of problems at once since it puts the cycles into a topological order so that
one cannot miss any accessible cycle and shadowing combinations naturally occur
when the expansion is ordered in topological length. The question one might ask
is: When there is no symbolic dynamics, how can we make sure that we find all
periodic orbits of a flow up to some cycle period?

In the searches of the cycles of high-dimensional flows, one usually looks at
the near recurrences of the ergodic flow, and then runs Newton searches
starting nearby these recurrences to find if they are influenced by an exact
recurrence. Such an approach does not answer the question we just asked to full
confidence, however, one may argue that the dynamically important cycles
influence recurrences of the ergodic flow, and hence cycles found this way are
those that are relevant for computing averages.

To sum up, we have shown that the periodic orbit theory successfully extends to
the systems with continuous symmetries. One still needs to think about
remaining challenges, discussed above, associated with high
dimensional systems. Once these are overcome, we would then be able to extract
quantitative information from turbulence by using its exact unstable solutions,
which would be big news.
