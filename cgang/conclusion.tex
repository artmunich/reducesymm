\section{Conclusions and discussion}
\label{s:concl}

In this tutorial, we have studied what is probably one of the simplest dynamical systems that
exhibits chaos and is equivariant under a continuous symmetry transformation.
We have shown that reducing this symmetry simplifies the qualitative dynamics
to a great extent and enables one to find all \rpo s of the systems via
standard techniques such as Poincar\'e sections and return maps. In
addition, we have shown that one can extract quantitative information
from the \rpo s by computing cycle averages by using a modified version of
classical spectral determinants.

We motivated our study of the \twomode\ system by the resemblance of its
symmetry structure to that of the spatially extended systems. The
steps outlined here are, in principle, applicable to physical systems
that are described by $N$-Fourier mode truncations of PDEs such as $1D$
\KS\rf{SCD07}, $3D$ pipe flows\rf{WiShCv14}, \etc.

We showed in \refsect{s:numerics} that \twomode\ dynamics can be accurately
described by a unimodal return map on the Poincar\'e section after continuous 
symmetry reduction. In a high-dimensional system, finding such an easy symbolic 
dynamics, or any symbolic dynamics at all is a challenging problem on its own. 
Progress in this direction was made by Lan and Cvitanovi\'{c}, who found a bimodal 
return map for the desymmetrized (confined in the odd subspace) $1$D spatio-temporally \KS\ 
equation \refref{lanCvit07} after reducing the discrete symmetry of the problem. To our
knowledge, no study has achieved such a simplification for turbulent flows.

In \refsect{s:DynAvers}, we showed that the symbolic dynamics and the
associated grammar rules greatly affects the convergence of the \cycForm s.
In general, finding a finite symbolic description of a flow is rarely as easy as 
it is for our model system. There exists other methods of ordering cycle expansion 
terms, for example, ordering pseudo-cycles by their stability and discarding terms
that are above a threshold; one expects the remaining terms to form shadowing 
combinations and converge exponentially. Whichever method of term ordering is 
deployed, cycle expansions are only as good as the least unstable cycle that 
one fails to find. Symbolic dynamics solves both of problems at once since it puts 
the cycles into a topological order so that one cannot miss any accessible cycle 
and shadowing combinations naturally occur when the expansion is ordered in 
topological length. The question one might ask is: When there is no symbolic dynamics, 
how can we make sure that we find all periodic orbits of a flow up to some cycle 
period?

In the searches of the cycles of high-dimensional flows, one usually looks at
the near recurrences of the ergodic flow and then runs Newton searches
starting near these recurrences to determine if they correspond to near passes
to an exactly recurring solution. This approach does not answer the question 
we just asked with full confidence. However, one may argue that dynamically 
important cycles will lead to observable near recurrences, while less
dynamically important ones will have less influence on the ergodic flow. Hence, 
cycles found this way are those that are relevant for computing averages.

To sum up, we have shown that periodic orbit theory can be successfully extended to
the systems with continuous symmetries. When dealing with high-dimensional systems,
one must still keep in mind the remaining challenges, discussed above. However, once 
these are overcome, it should be possible to extract quantitative information 
from turbulence by using the exact unstable solutions embedded within the chaotic 
attractor.