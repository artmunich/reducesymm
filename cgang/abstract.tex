\begin{abstract}
In nonlinear flows drifts along symmetry directions often obscure the
physical, `shape changing' dynamics. Symmetry reduction eliminates such
drifts through a coordinate transformation by which motions along
symmetry directions are parameterized by `phase' coordinates and
collections of \statesp\ points related by symmetry are mapped to single
points in a symmetry-reduced \statesp. Here, we illustrate and compare
three different symmetry reduction methods: polar coordinates, invariant
polynomial bases, and the `{method of slices}' by applying them to a
simple dynamical system \DBedit-{a chaotic flow based on the normal form of the 1:2 spatial resonance 
with broken reflection symmetry}\DB{Is there anything called a normal form flow? I don't think so but I could be wrong. I know `normal forms' and I know `flows' but not `normal form flows'. Refering to 1:2 resonance might also peak a certain crowd's interest, even though we don't really do anything in a regime that they would care about. Talking about broken symmetry is sexier than talking about SO(2).} - that can be thought of as a truncation of a spatially extended PDE. We find that the {method
of slices} offers the most insight.  Because the flow is four dimensional,
the slice hyperplane for this system is three dimensional, which allows
the visualization of each step of the symmetry-reduction process without having to
project the dynamics onto lower-dimensional submanifolds. A Poincar\'e return
map within the slice hyperplane enables further reduction of the dynamics to what is effectively a unimodal map and the determination (in principle) of all relative periodic orbits of the system.
\end{abstract}
