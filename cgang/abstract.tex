\begin{abstract}
In nonlinear flows drifts along symmetry directions often obscure the
physical, `shape changing' dynamics. Symmetry reduction eliminates such
drifts through a coordinate transformation by which motions along
symmetry directions are parameterized by `phase' coordinates, and a whole
orbit of \statesp\ points related by the symmetry is mapped to a single
point in a symmetry-reduced \statesp. Here we illustrate and compare
three different symmetry reduction methods: polar coordinates, invariant
polynomial bases, and the `{method of slices}', by applying them to a
simple dynamical system -a two Fourier-mode ODE normal form flow
equivariant under \SOn{2} transformations- that can be thought of as a
truncation of a spatially extended PDE. We find that the {method
of slices} offers most insight.  As the flow is only four dimensional,
the slice hyperplane for this system is three dimensional, and one can
visualize each step of a symmetry-reduction process without having to
project dynamics onto lower-dimensional submanifolds. A Poincar\'e return
map within the slice hyperplane enables us to reduce the dynamics
further, to what is essentially a unimodal map, and determine, in
principle, all relative periodic orbits of the system.
\end{abstract}
