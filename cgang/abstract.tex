\begin{abstract}
Dynamical systems with \SOn{2} symmetry arise in the
study of nonlinear PDEs, such as
\NS\ equations, under periodic boundary conditions; when one truncates
the Fourier expansion at a finite number of modes.
We study a `\twomode' model of this type,
the smallest possible truncation with a 4\dmn\ \statesp,
which has the same continuous symmetry structure as the $1D$ PDE
while being just high-dimensional enough to allow for chaotic dynamics.
The
crucial step in analysis of such systems is symmetry reduction, here a
coordinate transformation that separates physical, `shape changing'
dynamics from the drifts along the symmetry direction. We start by
reviewing continuous symmetries and symmetry reduction methods with a
focus on the `\mslices ', which, to the best of our knowledge, is the
only symmetry reduction method that can be applied to the infinite
dimensional problems. We then define our \twomode\ \SOn{2}-equivariant
model, compare different symmetry-reduction schemes, and determine its
\reqva\ using an invariant polynomial basis. We show that a Poincar\'e section
within the `\slice ' can be used to further reduce this flow to what is
for all practical purposes a
unimodal map, which enables us to find all \rpo s and their binary symbolic
dynamics up to any desired period.
We then compute dynamical averages using \rpo s, and discuss
convergence of the spectral determinants.
\end{abstract}
