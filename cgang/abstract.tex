\begin{abstract}
Dynamical systems with \SOn{2} symmetry arise naturally in the
study of nonlinear partial differential equations (PDEs), such as the
\NS\ equations, in domains with periodic boundary conditions. In these cases, 
it is natural to express the solution as a Fourier series truncated
at a finite number of modes. We study a `\twomode' model of this type,
the smallest possible truncation with \SOn{2} symmetry that is high-dimensional enough 
to allow for chaotic dynamics. A crucial step in analysis of such a system is symmetry reduction. Here, this takes the form of a 
coordinate transformation that separates physically relevant, `shape changing'
dynamics from the drifts along the symmetry direction. We start by
reviewing continuous symmetries and symmetry reduction methods with a
focus on the `\mslices ', a symmetry reduction method that is applicable to very high-dimensional problems. We then define our \twomode\ \SOn{2}-equivariant
model, compare different symmetry-reduction schemes, and determine its
\reqva\ by rewriting the dynamics in terms of a symmetry-invariant polynomial basis. We show that a Poincar\'e section
within the `\slice ' can be used to further reduce this flow to what is
for all practical purposes a
unimodal map. This enables us to find all \rpo s and their binary symbolic
dynamics up to any desired period.
We then compute dynamical averages using \rpo s and discuss
convergence of the spectral determinants.
\end{abstract}
