\section{Cycle Averages}
\label{s:DynAvers}

So far, we have explained how we find the \rpo s of the \twomode\ system in
its \reducedsp\ and how to compute their stability. However, we have not yet
said anything about what to do with these numbers. We begin this section with
an overview of the main results of the periodic orbit theory, referring the reader
to \refref{DasBuch} for a detailed introduction to the subject. Our discussion
closely follows the presentation of \refref{DasBuch} with the addition of how
the theory is modified in the presence of continuous symmetries in
\refsect{s-ContFac}. In \refsect{s-CycExp}, we present cycle expansions and
explain how to approximate the Poincar\'e section in
\reffig{fig:psectandretmap} (d), in order to obtain a better convergence of
the spectral determinants. We finish this section with the numerical results in
\refsect{s-NumResults}

\subsection{Classical trace formula}
Consider the {\evOper}, the action of which evolves a density
$\rho_0(\ssp)$ in the \statesp :\DB{2014-11-11}{There is some notational nastiness here since there 
are $\ssp'$ s floating around. We have previously used $\ssp'$ for the template and used primes in general
mark template related things}
\bea
    \rho(\zeit ,\ssp) &=& [\Lop^\zeit \rho_0 ] (\ssp) \, , \continue
    &=& \int d \ssp' \delta (\ssp - \flow{\zeit}{\ssp'})
        e^{\beta \Obser^\zeit (\ssp' )} \rho_0(\ssp') \, .
        \label{e-EvOper}
\eea
Here, $\beta$ is an auxiliary variable and $\Obser^\zeit (\ssp')$ is the
integrated value of an `additive' observable $\obser$ along the orbit
$\flow{\zeit}{\ssp'}$:
\beq
    \Obser^\zeit (\ssp' ) = \int_0^{\zeit} d \zeit'
                              \obser(\flow{\zeit'}{\ssp'}) \, .
\eeq
Notice that when $\beta = 0$, the \evOper\ \refeq{e-EvOper} simply evolves
the density of \statesp\ points to its new form after time $\zeit$. As we
shall see, attaching\DB{2014-11-3}{What does ``attaching'' mean in this context... might be lacking technical rigor here.}
the integrated observables to this operator enables us to
study values of these observables averaged over the invariant measures.

Since we required our observable to be additive along an orbit and
exponentiated its integrated value in the construction of the \evOper\
\refeq{e-EvOper}; the evolution operator itself is multiplicative:
\beq
    \Lop^{\zeit_1 + \zeit_2} = \Lop^{\zeit_2} \Lop^{\zeit_1} \, .
    \label{eq-SemiGroup}
\eeq
For the kernel of the evolution integral, which we will refer to as
$\Lop^\zeit (\ssp, \ssp')$ with explicit arguments, we can write this relation
as:
\beq
	\Lop^{\zeit_1 + \zeit_2} (\ssp,\ssp') =
    \int d\ssp'' \Lop^{\zeit_2} (\ssp, \ssp'')
                   \Lop^{\zeit_1} (\ssp'', \ssp) \, .
	\label{eq-SemiGroupKernel}
\eeq
This `semigroup property' \refeq{eq-SemiGroup} of the {\evOper} allows us to
define the {\evOper} as the formal exponential of its infinitesimal generator
\Aop :
\beq
	\Lop^t = e^{\Aop t} \, .
	\label{eq-EvOpExp}
\eeq
By definition \refeq{e-EvOper}, the eigenvalues and eigenfunctions of $\Lop^t$ (and
thus \Aop ) are functions of $\beta$. Let us define $\rho_{\beta} (x)$ as the
eigenfunction of \refeq{e-EvOper} corresponding to the leading eigenvalue (i.e., the one with the
largest real part); we can write the action of \refeq{e-EvOper} on this density
explicitly as follows:
\beq
    \left[ \Lop^t \rho_{\beta} \right] (x) = e^{t s(\beta )} \rho_{\beta} (x)
    \, .
    \label{eq-EigenvalueRel}
\eeq
Here, $s(\beta)$ is the eigenvalue of $\Aop$. As stated earlier, when
$\beta = 0$, the {\evOper} simply evolves densities; this form of the evolution
operator is known as the {\FPoper}. If we assume that the system under study is ergodic,
then an `invariant measure' $\rho_0(\ssp)$ exists with eigenvalue
$s(0) = 0$ exists. The long time spectrum of any observable is going to be dominated by
its average over such a density, hence we define the average of an observable
as its average over the invariant measure:
\beq
    \langle \obser \rangle = \int d \ssp \, \obser(\ssp) \rho_0 (\ssp) \, .
    \label{e-obserAvg}
\eeq
By evaluating the action of the {\evOper} \refeq{e-EvOper} for infinitesimal
times and after some algebra, which we skip here, one finds that the
averages of observables, as well as their higher moments, can be generated from the
derivatives of $s(\beta)$:
\beq
    \langle \obser \rangle =
        \left. \frac{d s}{d \beta} \right|_{\beta = 0} \, , \quad
    \langle (\obser - \langle \obser \rangle )^2 \rangle =
        \left. \frac{d^2 s}{d \beta^2} \right|_{\beta = 0} \,, ...
    \label{eq-moments}
\eeq
In order to obtain $s(\beta)$, we construct the resolvent of \Aop , by taking
the Laplace transform of \refeq{eq-EvOpExp}:
\beq
	\int_0^{\infty} d\zeit e^{-s\zeit} \Lop^\zeit = (s-\Aop)^{-1} \, ,
	\label{eq-ResolventA}
\eeq
the trace of which peaks at the eigenvalues of \Aop. By taking the
Laplace transform of $\Lop^\zeit$ and computing its trace
by $\tr \Lop^\zeit = \int d\ssp \Lop^\zeit (\ssp,\ssp)$, one obtains the
classical trace formula:
\beq
\sum_{\alpha=0}^{\infty} \frac{1}{s-s_{\alpha}} = \sum_p T_p
\sum_{r=1}^{\infty} \frac{e^{r(\beta \Obser_p - s T_p)}}{\oneMinJ{r}}
\ee{e-ClassicalTraceFormula}
that relates the spectrum of the {\evOper} to the spectrum of the periodic
orbits. Here,  $s$ is the auxiliary
variable of the Laplace transform and $s_{\alpha}$ are the eigenvalues of \Aop . The
outer sum on the right hand side runs over the `prime cycles' $p$ of the system,
which have periods $T_p$. $\Obser_p$ is the value of
the observable integrated along the prime cycle and $\monodromy_p$ is the transverse
monodromy matrix, the eigenvalues of which are the Floquet multipliers of $p$
excluding the marginal ones ($|\Lambda| \neq 1$). In the derivation of
\refeq{e-ClassicalTraceFormula}, one assumes that the flow has a single marginal direction,
namely the direction that is parallel to the periodic orbit at all times, and evaluates the
contribution of each \po\ to the trace integral by transforming to a local coordinate
system where one of the coordinates is
parallel to the flow while the rest is transverse. Integration along the
parallel direction is what contributes the factors of $T_p$. The transverse integral
over the delta function contributes the factor of $\oneMinJ{r}$.

\subsection{Continuous factorization}
\label{s-ContFac}

The classical trace formula \refeq{e-ClassicalTraceFormula} accounts for contributions from \po s
to long time dynamical averages. However, \rpo s of equivariant systems are almost never
periodic in the full \statesp. In order to compute the contributions of \rpo s
to the trace of the \evOper, one has to factorize
\DB{}{factor or factorize... not sure which sounds better...
are they different?}
the \evOper\
into the irreducible subspaces of the symmetry group. For discrete symmetries,
this procedure is studied in \refref{CvitaEckardt}. For the quantum systems
with continuous symmetries (Abelian and 3D rotations), the factorization of
the semiclassical Green's operator is carried out in \refref{Creagh93}.
\refRef{Cvi07} addresses the continuous factorization of the \evOper\ and its
trace; we provide a sketch of this treatment here. We start by stating, without
proof, that a square-integrable field $\psi (\ssp)$ over a vector space can be
factorized into its projections over the irreducible subspaces of a group
$\Group$:
\beq
    \psi (\ssp) = \sum_m \mathbb{P}_m \psi (\ssp) \, ,
\eeq
where the sum runs over the irreducible representations of $\Group$ and
the projection operator onto the $m$th irreducible subspace, for a continuous
group, is:
    \PC{2014-11-10: From now on, a problem - if I redefine $D(\cdots)$ as \matrixRep,
    cannot easily revert to the concise $g$ notation...}
\beq
    \mathbb{P}_m = d_m \int_\Group d \mu(\LieEl) \chi_m (\LieEl(\theta))
                            \mathbb{D}(\theta)
\,.
\ee{e-ProjectionOperator}
Here, $d_m$ is the dimension of the representation, $d \mu(g)$ is the
normalized `Haar measure', $\chi_m (\LieEl)$ is the `character' of $m$th
irreducible representation and $\mathbb{D}(\theta)$ is the operator that
transforms a scalar field defined on the \statesp\ according to $\matrixRep(\theta)$,
namely, $\mathbb{D}(\theta) \rho (\ssp) = \rho(\matrixRep(\theta)^{-1} \ssp)$.
    \DB{2014-11-10}{Added missing parenthesis back on 2014-11-3 to
    this expression. Did I put it in the right place? I think so, but please double check and then delete this comment.}
For
our specific case of a single $\SOn{2}$ symmetry,
\bea
d_m &\rightarrow& 1\, , \\
\int_G d \mu(g) &\rightarrow& \oint \frac{d \theta} {2 \pi} \, , \\
\chi_m (\LieEl(\theta)) &\rightarrow& e^{- \ii m \theta } \, .
\eea
Because the projection
operator \refeq{e-ProjectionOperator} factorizes scalar fields in the \statesp\
into their projections onto irreducible subspaces of $\Group$, it can be used to
factorize the \evOper\ since the \evOper\ both acts on and returns scalar
fields (densities) on the \statesp . Thus, the kernel
of the \evOper\ transforms under the action of $\mathbb{D}(\theta)$ as:
\bea
    \mathbb{D}(\theta) \Lop^t (\ssp', \ssp) &=&
        \Lop^t (\matrixRep(\theta)^{-1} \ssp', \ssp)\,,
    \continue
    &=& \Lop^t (\ssp', \matrixRep(\theta) \ssp) \,, \continue
    &=& \delta (\ssp' - \matrixRep(\theta) f^t (\ssp)) e^{\beta \Obser^t(\ssp)}\, ,
    \label{e-gEvOper}
\eea
where the second step follows from the equivariance of the system under
consideration. \Rpo s contribute to $\mathbb{P}_m \Lop^t = \Lop_m^t$ since when its
kernel is modified as in \refeq{e-gEvOper}, the projection involves an integral
over the group parameters that is non-zero when $\theta=\theta_{\rpprime}$, the phase shifts of the
\rpo s. By computing the trace of $\Lop_m^t$, which in addition to the integral
over \statesp , now involves another integral over the group parameters, one
obtains the $m$th irreducible subspace contribution to the classical trace as
\beq
\sum_{\alpha=0}^{\infty} \frac{1}{s-s_{m, \alpha}} = \sum_p T_{\rpprime}
\sum_{r=1}^{\infty} \frac{\chi_m (\LieEl^r(\theta_{\rpprime}))
            e^{r(\beta \Obser_{\rpprime} - s T_{\rpprime})}}{\oneMinJred{r}} .
\ee{e-ReducedTraceFormula}
The reduced trace formula \refeq{e-ReducedTraceFormula} differs from the
classical trace formula \refeq{e-ClassicalTraceFormula} by the group character
term, which is evaluated at the \rpo\ phase shifts, and the reduced monodromy
matrix $\monodromyRed$, which is the $(d-N-1)\times(d-N-1)$ reduced Jacobian
for the \rpo\ evaluated on a Poincar\'e section in the \reducedsp . The eigenvalues
of $\monodromyRed$ are those of the \rpo\ Jacobian \refeq{e-rpoJacobian}
excluding the marginal ones, i.e., the ones corresponding to time evolution and evolution
along the continuous symmetry directions.

Since we are only interested in the leading eigenvalue of the \evOper , we
only consider contributions to the
trace \refeq{e-ClassicalTraceFormula} from the projections
\refeq{e-ReducedTraceFormula} of the $0$th irreducible subspace. For the $\SOn{2}$ case at hand, these can be written
explicitly as 
\beq
\sum_{\alpha=0}^{\infty} \frac{1}{s-s_{0, \alpha}} = \sum_p T_p
\sum_{r=1}^{\infty} \frac{e^{r(\beta \Obser_p - s T_p)}}{\oneMinJred{r}} \, .
\ee{e-tracem0}
This form differs from the classical trace formula
\refeq{e-ClassicalTraceFormula} only by the use of the reduced monodromy matrix 
\DBedit{instead of the full monodromy matrix}\DB{2014-11-10}{What is the non-reduced monodromy matrix called? 
I put down full but don't know if this is the correct nomenclature} since
the $0$th irreducible representation of $\SOn{2}$ has character $1$. For this
reason, cycle expansions \rf{AACI}, which we cover next, are applicable
to \refeq{e-tracem0} after the replacement
$\monodromy \rightarrow \monodromyRed$.

\subsection{Cycle expansions}
\label{s-CycExp}

While the classical trace formula \refeq{e-ClassicalTraceFormula} and its
factorization for systems with continuous symmetry \refeq{e-ReducedTraceFormula} manifest
the essential duality between the spectrum of an observable and that of
the \po s and \rpo s, in practice, they are hard to work with since the
eigenvalues are located at the poles of \refeq{e-ClassicalTraceFormula} and
\refeq{e-ReducedTraceFormula}. The dynamical zeta function
\refeq{e-DynamicalZeta}, which we derive below, provides a perturbative expansion form that 
enables us to order terms in decreasing importance while computing
spectra for the \twomode\ system. As stated earlier, \refeq{e-tracem0}
is equivalent to the \refeq{e-ClassicalTraceFormula} via substitution
$\monodromy \rightarrow \monodromyRed$. We start by defining the
`spectral determinant':
\beq
  \det (s-\Aop) = \exp \left( - \sum_p \sum_{r=1}^{\infty}
      \frac{1}{r} \frac{e^{r(\beta \Obser_p - s T_p)}}{\oneMinJ{r}} \right)\, ,
\ee{e-SpectralDeterminant}
whose logarithmic derivative ($(d/ds) \ln \det(s - \Aop)$) gives
the classical trace formula \refeq{e-ClassicalTraceFormula}.
The spectral determinant \refeq{e-SpectralDeterminant} \DB{2014-11-10}{What's up with G65?} is easier to work
with since the spectrum of $\mathcal{A}$ is now located at the zeros of
\refeq{e-SpectralDeterminant}. The convergence of \refeq{e-SpectralDeterminant}
is, however, still not obvious. More insight is gained by approximating
$\oneMinJ{r}$ by the product of expanding Floquet multipliers and then
carrying out the sum over $r$ in \refeq{e-SpectralDeterminant}. This
approximation yields
\bea
\oneMinJ{} &=& | (1 - \ExpaEig_{e,1})(1 - \ExpaEig_{e,2})... \continue
			&&(1 - \ExpaEig_{c,1}) (1 - \ExpaEig_{c,2}) ... | \nonumber \\
			&\approx& \prod_e |\ExpaEig_e| \equiv |\ExpaEig_p|,
    \label{e-LambdapApprox}
\eea
where $|\ExpaEig_{e,i}| > 1$ and $|\ExpaEig_{c,i}| < 1$ are expanding and
contracting Floquet multipliers respectively. By making this approximation, the sum over $r$ in
\refeq{e-SpectralDeterminant} becomes the Taylor expansion of natural logarithm. Carrying out this sum, brings the
spectral determinant \refeq{e-SpectralDeterminant} to a product (over prime
cycles) known as the dynamical zeta function:
\beq
1 / \zeta = \prod_p (1 - t_p) \, \mbox{where}, \, t_p = \frac{1}{|\ExpaEig_p|}
            e^{\beta \Obser_p - s T_p} z^{n_p} .
\ee{e-DynamicalZeta}
Each `cycle weight' $t_p$ is multiplied by the `order tracking term' $z^{n_p}$,
where $n_p$ is the topological length of the $p$th prime cycle. This polynomial
ordering arises naturally in the study of discrete time systems where the
Laplace transform is replaced by $z$-transform. Here, we insert the powers of
$z$ by hand \DBedit{to keep track of the ordering and then} set its value to $1$ at the 
end of calculation. Doing so allows us to write the dynamical zeta function
\refeq{e-DynamicalZeta} in the `cycle expansion' form by grouping its
terms in powers of $z$. For complete binary symbolic dynamics, where every binary symbol
sequence is accessible, the cycle expansion reads
\bea
1 / \zeta &=& 1 - t_0 - t_1 - (t_{01} - t_0 t_1 )  \label{e-CycleExpansion} \\
		  && - [(t_{011} - t_{01}t_1) + (t_{001} - t_{01} t_0)] - ... \continue
		  &=& 1 - \sum_f t_f - \sum_n \hat{c}_n \label{e-CurvatureExpansion},
\eea
\DBedit{where we labeled each prime cycle by its binary symbol sequence. In
\refeq{e-CurvatureExpansion} we grouped the contributions to the zeta function
into two groups: `fundamental' contributions $t_f$ and `curvature' corrections $c_n$.
The curvature correction terms are denoted explicitly by parentheses in \refeq{e-CycleExpansion} and
correspond to `shadowing' combinations where combinations of
shorter cycle weights, also known as `pseudocycle' weights, are subtracted from the weights of longer
prime cycles. Since the cycle weights in \refeq{e-DynamicalZeta} already
decrease exponentially with increasing cycle period, the cycle expansion
\refeq{e-CycleExpansion} converges even faster than exponentially when the
terms corresponding to longer prime cycles are shadowed.}\DB{2014-11-10}{Tried to make this clearer. Think I didn't change the content, but somebody more erudite than me please double check this.}

For complete binary symbolic dynamics, the only fundamental contributions to
the dynamical zeta function are from the cycles with topological length $1$. All 
the longer cycles appear in the shadowing combinations. This is not the case for 
any unimodal map\DB{2014-11-11}{Are we trying to say that there is no unimodal map 
for which this is true or that this is not true for a generic, but could be true for 
some subset of unimodal maps?} since some symbol sequences might be inaccessible and so that 
terms corresponding to pseudocycles that include them fail to appear in the cycle expansion 
\refeq{e-CycleExpansion}. However, if the symbolic dynamics of a system can be obtained 
from a complete binary set by finite number of replacements, then another cycle expansion 
form can be obtained is guaranteed to converge super-exponentially. Such a
duality is be obtained when there exists a finite set of `grammar rules'
to the symbolic dynamics that makes some symbol sequences inaccessible, or
`pruned'. We argued in \refsect{s:numerics} that the Poincar\'e return map for
the \twomode\ system (\reffig{fig:psectandretmap} (d)) diverges at 
$s \approx 0.98$ and approximated it as if its tip was located at the
furthest point visited by an ergodic trajectory. Here, we ask the question: Can we
reasonably approximate the map in \reffig{fig:psectandretmap} (d) in such a way
that corresponding symbolic dynamics has a finite grammar of prunning rules?
The answer, fortunately, is yes.

As shown in \reffig{fig:psectandretmap} (d) the cycles \cycle{001}
and \cycle{011} pass quite close to the tip of the cusp. Approximating the
map as if its tip located exactly at the point where \cycle{001} cuts gives us
exactly what we are looking for: a single grammar rule, which says that the symbol
sequence `00' is inaccessible. This can be made rigorous by the help of
kneading theory, however, the simple result is easy to see from the return map
in \reffig{fig:psectandretmap} (d): Cover the parts of the return map, which
are outside the borders set by the red dashed lines, the cycle \cycle{001} and
then start any point to the left of the tip and look at images. You will always
land on a point to the right of the tip, unless you start at the lower left
corner, exactly on the cycle \cycle{001}. As we will show, this `finite grammar
approximation' is reasonable since the orbits that visit outside
the borders set by \cycle{001} are very unstable, and hence, less
important for the description of invariant dynamics.

The binary grammar with only rule that forbids repeats of one of the symbols is
known as the `golden mean' shift, named after its topological entropy which is
$\ln (1 + \sqrt{5})/2$. Binary itineraries of golden mean cycles can be easiliy
obtained from the complete binary symbolic dynamics by substitution
$0 \rightarrow 01$ in  the latter. Thus, we can write the dynamical zeta
function for the golden mean pruned symbolic dynamics by replacing $0$s in
\refeq{e-CycleExpansion} by $01$:
\bea
1 / \zeta &=& 1 - t_{01} - t_1 - (t_{011} - t_{01} t_1 )
              \label{e-GoldenMeanCycleExpansion}\\
		  && - [(t_{0111} - t_{011}t_1) + (t_{01011} - t_{01} t_{011} ) ] - ...
          \nonumber
\eea
Note that all the contributions longer than topological length $2$ to the
golden mean dynamical zeta function are in form of shadowing combinations. In \refsect{s-NumResults},
we will compare the convergence of the cycle averages with and without the
finite grammar approximation, but before moving on to numerical results, 
we explain the remaining details of computation.

While dynamical zeta functions are useful for investigating the convergence
properties, they are not exact, and their computational cost is same as that of 
exact spectral determinants. For this reason, we expand the
spectral determinant \refeq{e-SpectralDeterminant} ordered in the topological
length of cycles and pseudocycles. We start with the following form of the
spectral determinant \refeq{e-SpectralDeterminant}:
\beq
    \det (s-\Aop) =   \prod_p \exp \left( - \sum_{r=1}^{n_p r < N}
                             \frac{1}{r} \frac{e^{r(\beta \Obser_p - s T_p)}
                                          }{\oneMinJ{r}} z^{n_p r} \right) \, ,
\ee{e-SpectralDeterminantExp}
where the sum over the prime cycles in the exponential becomes a
product. We also inserted the order tracking term $z$ and truncated the sum over cycle
repeats at the expansion order $N$. For each prime cycle, we compute the sum in
\refeq{e-SpectralDeterminantExp} and expand the exponential up to order
$N$. We then multiply this expansion with the contributions from previous cycles
and drop terms with order greater than $N$. This way, after setting $z=1$,
we obtain the $N^{th}$ order spectral determinant, which we denote as
\beq
    F_N(\beta , s) = 1 - \sum_{n=1}^{N} Q_n(s, \beta ) \, .
    \label{e-NthOrderSpectDet}
\eeq
Remember that we are searching for the eigenvalues $s ( \beta)$ of the \Aop ,
more specifically, we would like to compute the moments \refeq{eq-moments}.
$s ( \beta)$ are located at the zeros of the spectral determinant, hence they
satisfy the implicit equation:
\beq
    F_N(\beta, s(\beta )) = 0 \, .
    \label{e-FNimplicit}
\eeq
By taking derivative of \refeq{e-FNimplicit} with respect to $\beta$ and
applying chain rule we obtain
\beq
    \frac{d s}{d \beta} = - \left. \frac{\partial F}{\partial \beta} \right/
                                     \frac{\partial F}{\partial s}\, .
\eeq
Higher order derivatives yield higher can also be obtained similarly, and
finally, we define \DB{2014-11-11}{What is $T$ here... at this point in the paper $T$ has been used to represent both period of cycles and Lie group generator. Okay... reading further it's clearly a period, but we may want to fix this notational ambiguity}
\beq
	\langle T \rangle_N = \left. \partial F_N / \partial s
                          \right|_{\beta=0, s=s (0)} \, ,
	\label{eq-Tavg}
\eeq
and write the \cycForm s as
\bea
    \langle \obser \rangle_N &=& - \frac{1}{\langle T \rangle_N} \left.
                              \frac{\partial F_N}{\partial \beta}
                              \right|_{\beta=0, s=s (0)} \, , \label{e-Avga} \\
    \langle (\obser - \langle \obser \rangle )^2 \rangle_N
    &=& - \frac{1}{\langle T \rangle_N} \left. \frac{\partial^2 F_N}{
                        \partial \beta^2} \right|_{\beta=0, s=s (0)} \,
                        \label{e-Avgsigma} .
\eea
As mentioned earlier, we expect that for an invariant measure  $\rho_0(\ssp)$,
the eigenvalue $s(0)$ is $0$. However, we did not make this substitution in \cycForm s since, in practice,
our approximation to the spectral determinant is always of a finite precision, so that 
the solution of $F_N(0, s(0)) = 0$ is small, but not exactly $0$. This
eigenvalue has a special meaning: It indicates how well the \po s cover the
strange attractor. Following this interpretation, we define $\gamma = - s(0)$
as the `escape rate': the rate at which the dynamics escape the region that is
covered by the \po s. Specifically for our finite grammar approximation; the
escape rate tells us how frequently the flow visits the part of the
Poincar\'e map that we cut off by applying our finite grammar approximation.

We defined $\langle T \rangle$ in \refeq{eq-Tavg} as a shorthand for a partial
derivative, however, we can also develop and interpretation for it by looking
at the definitions of the dynamical zeta function \refeq{e-DynamicalZeta} and the
spectral determinant \refeq{e-SpectralDeterminant}. In both series, the partial
derivative with respect to $s$ turns them into weighted sum of the cycle
periods; with this intuition, we define $\langle T \rangle$ as the `mean cycle
period'.

These remarks conclude our review of the periodic theory and its
extension to the equivariant dynamical systems. We are now ready to present
our numerical results and discuss their quality.

\subsection{Numerical results}
\label{s-NumResults}

We constructed the spectral determinant \refeq{e-NthOrderSpectDet} to different
orders for two observables: phase velocity $\dot{\theta}$ and the leading
Lyapunov exponent. Remember that $\Obser_p$ appearing in
\refeq{e-SpectralDeterminantExp} is the integrated observable, so in order to
obtain the moments of phase velocity and the leading Lyapunov exponent from
\refeq{e-Avga} and \refeq{e-Avgsigma}, we respectively input
$\Obser_p = \theta_p$ phase shift of the prime cycle, and
$\Obser_p = \ln |\Lambda_{p,e}|$ logarithm the expanding Floquet multiplier of
the prime cycle.

In \refsect{s:visual}, we explained that \SOn{2} phase shifts correspond to
the drifts in the configuration space. With this in mind, we can relate the
variance of phase velocity to the diffusion constant for these drifts. We define the
diffusion constant as:
\beq
    D = \frac{1}{2 d} \sigma_{\dot{\theta}}^2
      = \frac{\langle (\dot{\theta} - \langle \dot{\theta} \rangle)^2
              \rangle}{2} ,
\eeq
where $d=1$ since our configuration space is one dimensional.

\refTab{t-DynamicalAverages} and \reftab{t-DynamicalAveragesNoGrammar} shows
the cycle averages of the escape rate $\gamma$, mean period
$\langle T \rangle$, leading Lyapunov exponent $\Lyap$, mean phase velocity
$\langle \dot{\theta} \rangle$ and the diffusion constant $D$ respectively
with and without the finite grammar approximation. In the latter, we input
all the \rpo s we have found into the expansion
\refeq{e-SpectralDeterminantExp}, whereas in the former, we discarded the
cycles with symbol sequence `00'.

\begin{table}
	\begin{tabular}{c|c|c|c|c|c}
	 $N$ & $\gamma$ & $\langle T \rangle$ & $\lambda$ & $\langle \dot{\phi} \rangle$ & $D$ \\ 
	\hline
	1 & 0.24983007 & 3.64151210 & 0.10834935 & 0.02223518 & 0.00090019 \\ 
 	2 & -0.01159743 & 5.89675999 & 0.10302904 & -0.14242535 & 0.22615040 \\ 
 	3 & 0.02744646 & 4.72713874 & 0.11849781 & -0.14916163 & 0.22541882 \\ 
 	4 & -0.00445545 & 6.23866097 & 0.10631075 & -0.22271184 & 0.54199883 \\ 
 	5 & 0.00068106 & 5.89674866 & 0.11842709 & -0.19506985 & 0.47091930 \\ 
 	6 & 0.00068491 & 5.89688492 & 0.11820055 & -0.21660997 & 0.64918118 \\ 
 	7 & 0.00063044 & 5.90316812 & 0.11835165 & -0.18083296 & 0.51935080 \\ 
 	8 & 0.00071488 & 5.89189180 & 0.11827587 & -0.21129228 & 0.78673581 \\ 
 	9 & 0.00072867 & 5.88975993 & 0.11826879 & -0.19203222 & 0.70977900 \\ 
 	10 & 0.00072808 & 5.88986409 & 0.11826793 & -0.20302879 & 0.90493496 \\ 
 	11 & 0.00072790 & 5.88989901 & 0.11826783 & -0.19453187 & 0.86683629 \\ 
 	\end{tabular}
	\caption{Cyle expansion estimates of the escape rate $\gamma$, average cycle period $\langle T \rangle$, Lyapunov exponent $\lambda$, average phase velocity $\langle \dot{\phi} \rangle$ and the diffusion coefficient $D$ with respect to the expansion order $N$ .}
	\label{t-DynamicalAverages}
\end{table}
\begin{table}
	\begin{tabular}{c|c|c|c|c|c}
	 $N$ & $\gamma$ & $\langle T \rangle$ & $\lambda$ & $\langle \dot{\phi} \rangle$ & $D$ \\ 
	\hline
	1 & 0.24982996 & 3.64151221 & 0.10834917 & 0.02223518 & 0.00090019 \\ 
 	2 & -0.01159761 & 5.89676053 & 0.10302891 & -0.14242516 & 0.22615014 \\ 
 	3 & 0.02261469 & 4.88995874 & 0.13055574 & -0.16698925 & 0.28645997 \\ 
 	4 & -0.00606560 & 6.24822611 & 0.11086469 & -0.22623282 & 0.55402655 \\ 
 	5 & 0.00091264 & 5.77716415 & 0.11812034 & -0.19839015 & 0.47840111 \\ 
 	6 & 0.00026210 & 5.83645342 & 0.11948918 & -0.21788301 & 0.65393430 \\ 
 	7 & 0.00001771 & 5.86382098 & 0.12058951 & -0.18565119 & 0.55022944 \\ 
 	8 & 0.00011328 & 5.85110445 & 0.12028459 & -0.21480977 & 0.80768111 \\ 
 	9 & 0.00017071 & 5.84222727 & 0.12010579 & -0.19376222 & 0.72685211 \\ 
 	10 & 0.00015683 & 5.84466944 & 0.12014022 & -0.20501316 & 0.92174228 \\ 
 	\end{tabular}
	\caption{Cyle expansion estimates of the escape rate $\gamma$, average cycle period $\langle T \rangle$, Lyapunov exponent $\lambda$, average phase velocity $\langle \dot{\phi} \rangle$ and the diffusion coefficient $D$ with respect to the expansion order $N$ .}
	\label{t-DynamicalAveragesNoGrammar}
\end{table}


In \refsect{s-CycExp}, we motivated the finite grammar approximation by expecting a faster convergence
due to the nearly exact shadowing combinations of the golden mean zeta function
\refeq{e-GoldenMeanCycleExpansion}\DB{2014-11-11}{What's with $G69$ here}. This claim is clearly supported by the
data in \reftab{t-DynamicalAverages}\DB{2014-11-11}{Table numbering is all screwy} and
\reftab{t-DynamicalAveragesNoGrammar}. Take, for example, the Lyapunov exponent
which converges to $7$ digits for the $12^{th}$ order expansion when using the finite
grammar approximation
\reftab{t-DynamicalAverages}, only converges to $4$ digits at this order in
\reftab{t-DynamicalAveragesNoGrammar}. Other observables compare similarly in
terms of their convergence in both cases. Note, however, that the escape rate
in \reftab{t-DynamicalAverages} converges to $\gamma = 0.000727889$, whereas
in \reftab{t-DynamicalAveragesNoGrammar} it gets smaller and smaller with an
oscillatory behavior. This is due to the fact that in the finite grammar
approximation, we threw out a the part of attractor that corresponds to the
cusp of the return map in \reffig{fig:psectandretmap} (d) above the point that
is cut by \cycle{001}.

In order to compare with the cycle averages, we numerically estimated the
leading Lyapunov exponent of the \twomode\ system using the method of
Wolf \etal\rf{WolfSwift85}. This procedure was repeated 100 times for
different initial conditions, yielding a mean estimate of
$\Lyap_{Numerical} = 0.1198 \pm 0.0008$. While the finite grammar
estimate $\Lyap_{FG} = 0.1183$ is within $0.6\%$ range of this value,
the full cycle expansion agrees with the numerical estimate. This is not
surprising, since in the finite grammar approximation, we discard the
most unstable cycles, thus, obtaining a slightly smaller Lyapunov
exponent while obtaining a significantly better convergence.
