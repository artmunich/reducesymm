\section{Introduction}
\label{s:intro}

Recent experimental observations of travelling waves in pipe flows have confirmed predictions based on dynamical systems theory that coherent structures play an important role in shaping the \statesp\ of turbulent flows\rf{science04}. When one formulates a turbulent fluid flow as a dynamical system, the outcome is an infinite dimensional system usually symmetric under various transformations such as translations, reflections, rotations etc. For example when a periodic boundary condition along the stream direction is imposed, pipe flow is symmetric under streamwise translations, azimuthal rotations and reflections about the central axis. In other words, it is symmetric under the action of $\SOn{2} \times \On{2}$. Symmetric structure of the configuration space suggests the formulation of these problems in Fourier spaces, however, due to the nonlinear nature of the problem, Fourier modes mix as the system evolves. This mode mixing phenomenon complicates the dynamics of the system, and gives rise to the higher dimensional coherent solutions such as \reqva\ and \rpo s. Our goal in this paper is to show that \reqva\ and \rpo s of the dynamical systems with continuous symmetries, respectively take the roles of \eqva\ and \po s of the flows without such symmetries. Furthermore, we will demonstrate an extension of the \po\ theory of nonlinear flows by computing cycle averages using \rpo s.

Our objective is to provide a prototype of the study that should ultimately applied to the turbulent problems. For this purpose, we study the \twoMode\ \SOn{2} equivariant flow, probably the simplest dynamical dynamical system that exhibits chaos, and is symmetric under a \SOn{2} rotations.

We start by establishing the terminology that we will use in the rest of the paper in \refSect{s:symm}, and we provide a short review of the symmetry reduction literature. We finish this section with the formulation of \mslices\ as proposed by Budanur \etal\rf{BudCvi14}. In \refSect{s:twoMode}, We introduce the \twomode\ system and its symmetry-equviariant and different symmetry-reduced representations, and show that symmetry-reduced representations can be used to find the \reqva\ of the system. We then show how the \mslices\ enables the use a Poincar\'e section within the \slice\ to find  \rpo s and construct symbolic dynamics.
Finally, \refSect{s:concl} discusses possible applications of the \mslices\ to various spatially extended systems.
