\section{Introduction}
\label{s:intro}
Experimental observation of the travelling waves in pipe flows \rf{science04} 
confirmed the anticipation based on the theories of nonlinear dynamics that 
the coherent structures shapes the \statesp s of turbulent flows \rf{REFERENCE}. 
Finding such solutions can lead to a better understanding of transition to 
turbulence in \NS\ and in the long term, computation of dynamical averages
over the periodic orbits. In order to find the coherent solutions such as
\reqva\ and \rpo s an essential step to take, as we shall demonstrate, is
continuous symmetry reduction.

Symmetry reduction by casting solutions on to an invariant polynomial basis 
is well studied\rf{GL-Gil07b} for low-dimensional systems such as Lorenz 
and R\"{o}ssler, however, invariants of the continuous symmetries becomes 
practically incomputable as the system dimension exceeds 12\rf{REFERENCE}. 
\Mslices  
\rf{rowley_reconstruction_2000,BeTh04,SiCvi10,FrCv11,atlas12,ACHKW11,BudCvi14}
which we study in detail in this paper, offers a symmetry reduction scheme
for high dimensional flows, such as \NS .

While the history of the method traces back to Cartan\rf{CartanMF}, best  
to our knowledge, Rowley and Marsden\rf{rowley_reconstruction_2000} were 
first to apply \mslices\ in the context of nonlinear flows. Rowley \etal 
\rf{rowley_reduction_2003} generalized the reconstruction equations in\rf{rowley_reconstruction_2000} 
and formulated them using a ``rescaled time'' variable where they demonstrated 
the method on 1 \dmn\ \KS\ system in a periodic domain in the neighborhood 
of a travelling wave, picking the travelling wave solution as the \slice\ 
``\template ''.  Beyn and Th\"{u}mmler \rf{BeTh04} applied the method 
slices to ``freeze'' the spiral waves in reaction-diffusion systems. Siminos 
and Cvitanov\'{c} \rf{SiCvi10} used \mslices\ to reduce the \SOn{2} symmetry 
of complex Lorenz system's chaotic solutions where they showed the slice-dependent 
singularity of reconstruction equation causes the reduced flow to make discontinuous 
jumps. This singularity is studied in detail in \refref{FrCv11}, and the 
solution of using multiple ``charts'' of connected slices was proposed, 
and this approach was applied to Complex Lorenz system in \refref{atlas12} 
and to pipe flows in \refref{ACHKW11}. In a recent paper, Budanur \etal 
\rf{BudCvi14} followed a geometrical approach and showed, for \SOn{2}, 
that for a specific choice of a \slice\ \template , one can move the \slice\ 
singularity to the edge of the symmetry-reduced sub-manifold and regulate 
the close-passages to the singularity by means of a time rescaling. Here
we follow this approach and apply the method to a 2-mode \SOn{2} ODE normal
form which perhaps is the simplest system that exhibits chaos with evolution
equations equivariant under continuous \SOn{2} transformations.

In the following section, we start by setting the terminology for the rest 
of the paper and formulate the \mslices\ as in \refref{BudCvi14}. In 
section \ref{s:twoMode}, we introduce the \twoMode\ system and its representations
in its real valued \statesp , invariant polynomials, polar coordinates and
on a \slice . We then show that one can use either symmetry reduced representation
to compute all \reqva of the system. We then find the \rpo s and construct
symbolic dynamics by the help of Poincar\'e sections within the \slice . 
We conclude by discussing the possible applications of this method in various 
spatially extended systems.
