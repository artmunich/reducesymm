\section{Introduction}
\label{s:intro}

Recent experimental observations of \reqva\ in pipe flows have confirmed
predictions based on dynamical systems theory that coherent structures shape 
the \statesp\ of turbulent flows\rf{science04}. Identifying such solutions can lead to a better
understanding of the transition to turbulence in fluid flows and may, in the long-term,
allow the computation of dynamical averages using periodic orbits. In order to
find coherent solutions such as \reqva\ and \rpo s it is essential, as we shall demonstrate, 
to use continuous symmetry reduction.

One well-studied technique for symmetry reduction, which works well for low-dimensional dynamical systems such 
as Lorenz and R\"{o}ssler flows, is to recast the dynamical equations in terms of invariant polynomials\rf{GL-Gil07b}. 
However, establishing an invariant polynomial basis becomes impractical 
for systems with more than 12 dimensions\rf{gatermannHab}\ES{2014-05-27}{\refref{gatermannHab}
is somewhat dated and probably nowdays people can find bases for $d>12$, 
but I have not kept up with the literature. Do you have any
newer reference?}.
The \mslices%
\rf{rowley_reconstruction_2000,BeTh04,SiCvi10,FrCv11,atlas12,ACHKW11,BudCvi14},
which we study in detail here, offers a symmetry reduction
scheme applicable to high-dimensional flows like the \NS\ equations.

While the \mslices\ goes back to Cartan\rf{CartanMF}, to the
best to our knowledge, Rowley and Marsden\rf{rowley_reconstruction_2000}
were first to apply it to spatially extended nonlinear flows. 
They demonstrated their method by applying it to the one-dimensional \KS\ equation
in a periodic domain and used it to study its dynamics in the neighborhood of 
a \reqv\ by using the \reqv\ itself as the \slice\ `\template'. Later on, 
Rowley \etal~\rf{rowley_reduction_2003} generalized the method in order
to handle self-similar solutions. Beyn and Th\"{u}mmler\rf{BeTh04} applied 
the \mslices\ to `freeze' spiral waves in reaction-diffusion systems.

These early studies applied the \mslices\ to a single solution at a time, but
studying the nonlinear dynamics of extended systems requires symmetry reduction 
of global objects, such as strange attractors or invariant manifolds.
In this spirit, Siminos and Cvitanovi\'{c}\rf{SiCvi10} used the \mslices\ to 
quotient the \SOn{2} symmetry of chaotic dynamics in \cLf\ and showed that the 
slice-dependent singularity of the reconstruction equation 
causes the reduced flow to make discontinuous jumps. This singularity was studied 
in detail by Froehlich and Cvitanovi\'{c}\rf{FrCv11}.

Two strategies have been proposed in order to handle this problem: either to
try to identify a template such that slice singularities are not visited
by dynamics\rf{SiCvi10} or to use multiple `charts' of connected
slices\rf{rowley_reconstruction_2000,FrCv11}.
The latter approach was applied to \cLf\ by Cvitanovi\'{c} \etal~\rf{atlas12} and 
to pipe flow by Willis, Cvitanovi\'{c}, and Avila\rf{ACHKW11}. 
\ESedit{However, neither approach is straightforward to apply, particularly in
high-dimensional dynamical systems.
More recently, Budanur \etal\rf{BudCvi14} considered Fourier space discretizations of
PDEs, in which translational symmetry naturally appears as the rotation group
\SOn{2}. They showed that a simple choice of \slice, associated to the first Fourier mode, 
results in a \slice\ singularity that generic dynamics is extremely improbable to visit.}
\ES{2014-06-03}{rephrased this to make easier to follow for the casual reader:
followed a geometrical approach
and showed that for \SOn{2} a specific choice of a \slice\
\template\ can be used to move the \slice\ singularity to the border of the
symmetry-reduced submanifold.
}
Close-passages to the
singularity were regularized by means of a time rescaling. Here, we follow this approach
and apply \ESedit{this so-called ``first Fourier mode slice''} method 
to a 2-mode ODE normal form, which may be
the simplest system with \SOn{2} equivariant dynamics
that exhibits chaos.

In \refSect{s:symm}, we establish the terminology that will be used in the rest
of the paper and formulate the \mslices\ as proposed by Budanur \etal\rf{BudCvi14}.
\refSect{s:twoMode} introduces the \twoMode\ system and its representation
in terms of four real-valued variables. We then present its symmetry-reduced representations
in terms of invariant polynomials and polar coordinates and discuss its dynamics on a \slice. 
In \refSect{s:numerics}, we show how any of these symmetry-reduced representations can be used
to compute the \reqva\ of the system. We then show how the \mslices\ enables the use a Poincar\'e section 
within the \slice\ to find  \rpo s and construct symbolic dynamics. 
Finally, \refSect{s:concl} discusses possible applications of the \mslices\ to various spatially extended systems.
