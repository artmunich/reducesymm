\section{Introduction}
\label{s:intro}

Recent experimental observations of travelling waves in pipe flows have
confirmed dynamical systems theory predictions that the invariant
solutions of \NSe\ play an important role in shaping the \statesp\ of turbulent
flows\rf{science04}. When one formulates a turbulent flow as a
dynamical system, the outcome is an infinite dimensional system that is often
symmetric under transformations such as translations,
reflections and rotations. For example, when a periodic boundary
condition is imposed along the streamwise direction, pipe flow is symmetric
under streamwise translations, azimuthal rotations and reflections about
the central axis, \ie, it is equivariant under the actions of
$\SOn{2} \times \On{2}$. This suggests that the problem should be formulated in
terms of a Fourier series. However, the nonlinear nature of the problem
leads Fourier modes to mix as the system evolves, and thus the state of the
system moves in both symmetry directions and the others that are transverse to 
symmetries. This phenomenon in dynamical systems with continuous symmetries 
complicates their \statesp , and gives rise to the high dimensional coherent 
solutions such as \reqva\ and \rpo s, which take on the roles played by 
\eqva\ and \po s in flows without symmetry.

The goal of this paper is to provide an example of the analysis that should
ultimately be applied to turbulent flows. For this purpose, we study
a \twomode\ \SOn{2} equivariant flow, with the minimal dimensionality required 
for chaotic dynamics. The rest of the paper is organized as follows:
In \refsect{s:symm}, we define basic concepts and briefly review the relevant 
symmetry reduction literature. In \refsect{s:twoMode}, we introduce the \twomode\ 
model system, describe several of its representations, and
utilize a symmetry-reduced polynomial representation to find the only \reqv\
of the system. In \refsect{s:numerics}, we show how the \mslices\ can be used to quotient 
the symmetry and reduce the dynamics onto a symmetry-reduced \statesp\ called a slice. Taking a Poincar\'e 
section within the \slice\ allows the reduction of the 4\dmn\ chaotic dynamics to a unimodal Poincar\'e
return map. This return map is then used to construct a finite grammar symbolic dynamics for
the flow and determine {\em all} \rpo s up to a given period. In \refsect{s:DynAvers},
these orbits are used as input to various {\cycForm s} to calculate estimates
of dynamically interesting observables. Finally, in \refsect{s:concl}, we discuss possible 
applications of the \mslices\ to various spatially extended systems. 

The main text is supplemented with two appendices. \refAppe{s:newton} describes the multi-shooting 
method used to calculate the \rpo s
\DB{9/16/2014}{Replaced cycles with \rpo s. This IS what we are refering to right?}. 
\BB{9/17/2014}{I confirm}
\refAppe{s:schur} discusses how periodic Schur decomposition can be used to determine their Floquet 
multipliers, which can easily differ by 100s of orders of magnitude even in a model as simple as 
the \twomode\ system.
