\section{Introduction}
\label{s:intro}

Recent experimental observations of travelling waves in pipe flows have
confirmed predictions based on dynamical systems theory that coherent
structures play an important role in shaping the \statesp\ of turbulent
flows\rf{science04}. When one formulates a turbulent fluid flow as a
dynamical system, the outcome is an infinite dimensional system usually
symmetric under various transformations such as translations,
reflections, rotations etc. For example when a periodic boundary
condition along the stream direction is imposed, pipe flow is symmetric
under streamwise translations, azimuthal rotations and reflections about
the central axis. In other words, it is symmetric under the action of
$\SOn{2} \times \On{2}$. Symmetric structure of the configuration space
suggests the formulation of these problems in Fourier spaces, however,
due to the nonlinear nature of the problem, Fourier modes mix as the
system evolves. This mode mixing phenomenon complicates the dynamics of
the system, and gives rise to the higher dimensional coherent solutions
such as \reqva\ and \rpo s. Our goal in this paper is to show that
\reqva\ and \rpo s of the dynamical systems with continuous symmetries,
respectively take the roles of \eqva\ and \po s of the flows without such
symmetries. Furthermore, we will demonstrate an extension of the \po\
theory of nonlinear flows by computing cycle averages using \rpo s.

Our objective is to provide a prototype of the study that should
ultimately applied to the turbulent problems. For this purpose, we study
the \twoMode\ \SOn{2} equivariant flow, probably the simplest dynamical
dynamical system that exhibits chaos, and is symmetric under a \SOn{2}
rotations.

In \refsect{s:symm} we define the basic terms used in the paper, and
provide a brief review of the symmetry reduction literature.
In \refsect{s:twoMode} we introduce our \twomode\ model system, its
symmetry-equviariant and different symmetry-reduced representations, and
utilize a symmetry-reduced polynomial representation to find the \reqva\
of the system.
In \refsect{s:numerics} we show how the \mslices\ combined with a
Poincar\'e section within the \slice\ enables us to reduce this 4\dmn\
chaotic flow to what is for all practical purposes a unimodal Poincar\'e
return map which enables us to  and construct a symbolic dynamics for
this flow and determine {\em all} \rpo s up to a given period, which
then are deployed
in \refsect{s:DynAvers} as input to various {\cycForm s} for estimates
of several dynamically interesting observables.
A student who gets this far in the tutorial has essentially covered the
first 500 pages of \HREF{http://ChaosBook.org} {ChaosBook.org}.
Finally, in \refsect{s:concl} we discuss possible applications of the
\mslices\ to various spatially extended systems.
