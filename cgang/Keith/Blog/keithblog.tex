
\subsection{Blog}
\begin{description}
\item[2012-04-18 Keith] Starting Project.  Here is what I have learned/done over the past few days/weeks/semester.
    \begin{itemize}
    \item Return maps are really useful, their usefulness was not apparent to me at the beginning of the course, but now, I have a better idea of how things work.
    \item Asides fixing some issues with the R\"ossler system, I have the return map for it.
    \item Am slightly confused by doing the return map for the complex lorenz.  When you run the Euclidean distance along the unstable manifold, how does one easily parameterize the curve?  In other words, it seems simple to me for the R\"ossler system as long as you choose a wise direction, where the curve can be fitted as $z = f(r)$ where $r$ is the usual polar coordinate.  Then once I find this, I can find the curvilinear distance along the curve (of course taking into account the fact that we need to start at the equilibrium). I am wondering if parameterizing the curve is even that necessary, in other words, couldn't I just arrange the unstable crossings through the Poincar\'e sections by nearest neighbor and then piece the curvilinear distance together from that?  I think this will work.  I have read E. Siminos' thesis on this, but he does it through invariant polynomials.  Their another work they have in Physica D; it seems they suggest just doing this idea, but ordering the distances seems extremely difficult at the moment, perhaps it is the late night.
        \end{itemize}

\item[2012-04-18 Keith] I have successfully made a general program to make return maps.  I tested it on the R\"ossler system and it returns what appears to be a good return map (haven't checked the periodic solutions).  I also went on the try it on the Complex Lorenz; I now see why symm red is necessary.  I seemed to have gotten the tent like map that Siminos has in his thesis and paper, but either his scaling is off (his goes to like 500) or I am completely wrong (mine has the same dimensions as the Lorenz equations, which makes me less skeptical of my dimensions).  Still thinking about the phase stuff for finding RPOs.  I may just need a night's rest to digest.


\end{description}
