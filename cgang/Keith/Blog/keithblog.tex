% siminos/cgang/Keith/Blog/keithblog.tex  pdflatex ../kcchaosproj
% $Author$
% $Date$

\section{Vita Brevis Blog: \\ So many schemes, so little time}
\begin{description}
\item[2012-04-18 Keith] Starting Project.  Here is what I have
learned/done over the past few days/weeks/semester.
    \begin{itemize}
    \item Return maps are really useful, their usefulness was not
    apparent to me at the beginning of the course, but now, I have a
    better idea of how things work.

    \item Asides fixing some issues with the R\"ossler system, I have the
    return map for it.

    \item Am slightly confused by doing the return map for the \cLe.
    When you run the Euclidean distance along the unstable manifold, how
    does one easily parameterize the curve?  In other words, it seems
    simple to me for the R\"ossler system as long as you choose a wise
    direction, where the curve can be fitted as $z = f(r)$ where $r$ is
    the usual polar coordinate.  Then once I find this, I can find the
    curvilinear distance along the curve (of course taking into account
    the fact that we need to start at the equilibrium). I am wondering if
    parameterizing the curve is even that necessary, in other words,
    couldn't I just arrange the unstable crossings through the Poincar\'e
    sections by nearest neighbor and then piece the curvilinear distance
    together from that?  I think this will work.  I have read E. Siminos'
    thesis on this, but he does it through invariant polynomials.  Their
    another work they have in Physica D; it seems they suggest just doing
    this idea, but ordering the distances seems extremely difficult at
    the moment, perhaps it is the late night.
        \end{itemize}

\item[2012-04-18 Keith] I have successfully made a general program to
make return maps.  I tested it on the R\"ossler system and it returns
what appears to be a good return map (haven't checked the periodic
solutions).  I also went on the try it on the \cLf; I now see
why symm red is necessary.  I seemed to have gotten the tent like map
that Siminos has in his thesis and paper, but either his scaling is off
(his goes to like 500) or I am completely wrong (mine has the same
dimensions as the Lorenz equations, which makes me less skeptical of my
dimensions).  Still thinking about the phase stuff for finding RPOs.  I
may just need a night's rest to digest.

\item[2012-04-20 Predrag] Maybe you should refer to Fig number or page,
I'm too lazy to go fishing, but if his return map in for invariant
polynomials, they are quadratic in coordinates, so 500 might be just
about right.

\item[2012-04-20 Predrag] I would put a priority on the Dangelmayr 2-mode
model, as \cLe\ seems too simple. Lei is MIA (last blog entry on 2-mode
was [2012-04-10]), you can perturb his \reqva\ and see what happens in
Cartesian coordinates), so maybe you can just play with it yourself -
it's no harder than \cLe, maybe even easier, it is in 4 dimensions. If
there are two distinct \reqva\ and chaos that hops between them, I hope
there is a robust \chartBord\ separating them, with 2-chart atlas ridging
it. That would be much more persuasive. Besides, the Dangelmayr 2-mode
model is known by many more people than \cLe, would have more of an
audience.

It's not guaranteed to work, I'm worried about the role that the
invariant subspace plays - might still turn out not to be too
interesting, even though is should have more \reqva\ than \cLe.

\item[2012-04-20 Keith to Predrag]  Yes I was also going to take a look
at the 2-mode this weekend.  I'll see what I can make out. The return map
is on page 30 of Siminos' thesis.  And yes it was computed for the
invariant polynomial set, so you are possibly right.  I did my initial
one as the post processing.  Perhaps I can try the invariant polynomial
and compare.

\item[2012-04-20 Predrag] Ignore invariant polynomials - it's only a
distraction, they are useless for problems we want to tackle. There is a
wonderful paper explaining precisely
\HREF{http://www.cns.gatech.edu/~predrag/papers/preprints.html\#atlas12}{that}.
(If the paper does not get through, I might have to resort to a pickaxe.)

\end{description}
