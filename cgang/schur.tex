\section{Periodic Schur decomposition}
\label{s:schur}

Here we briefly summarize the periodic eigendecomposition\rf{DingCvit14}
needed for evaluation of Floquet multipliers for \twomode\ \po s. Due to
the non-hyperbolicity of the return map of
\reffig{fig:psectandretmap}\,(d), Floquet multipliers can easily differ
by 100s of orders of magnitude even in a model as simple as the \twomode\
system.
    \PC{2014-07-14: cannot find anyplace in the blog numerical value of
    any of the allegedly very large unstable multipliers. $\ExpaEig
    \approx 80,000$ does not seem so large compared to the numerical
    precision? I guess it shows I did not have to compute them myself
    :)}

We obtain the Jacobian of the \rpo\ as the following multiplication of short-time
Jacobians from the multiple shooting computation of the previous section:
\bea
    \jMpsRed &=& \LieEl_n \jMps_n  \LieEl_{n-1} \jMps_{n-1} \, ... \, \LieEl_1 \jMps_1  \continue
                 &=& \hat{\jMps}_n \hat{\jMps}_{n-1} \, ... \, \hat{\jMps}_1 \label{e-JacobianProduct} \\
                 && \mbox{where,}\, \hat{\jMps_i} = \LieEl_i \jMps_i \in
                    \mathbb{R}^{4 \times 4}, i = 1,2,...,n \, . \nonumber
\eea
This Jacobian is equivalent to our previous definition in \refeq{e-rpoJacobian}
since $J_i$ and $g_i$ commte with each other, and are multiplicative respectively
in time and phase. In order to determine the eigenvalues of $\hat{\jMps}$, we
bring each term appearing in the product \refeq{e-JacobianProduct} into periodic
real Schur form as follows:
\beq
    \jMpsRed_i = Q_i R_i Q_{i-1}^T \, ,
\eeq
where $Q_i$ is an orthogonal matrix and they satisfy the cyclic property: $Q_0 = Q_n$ .
After this similarity transformation, we can define $R = R_k R_{k-1} ... R_1$ and
re-write the Jacobian as:
\beq
    \jMpsRed = Q_n R Q_n^T \, .
\eeq
The matrix $R$ is block-diagonal, in general, with $1 \times 1$ blocks for real
eigenvalues and $2 \times 2$ blocks for the complex pairs; and it has the same
eigenvalues with $\hat{\jMps}$. In our case, it is diagonal since all Floquet multipliers
are real in the \twomode\ system \rpo s. For each \rpo , we have two marginal Floquet
multiplier corresponding to the time evolution direction and continuous symmetry direction,
in addition to one expanding and one contracting eigenvalue.
