%% start of file `template_en.tex'.
%% Copyright 2007 Xavier Danaux (xdanaux@gmail.com).
%
% This work may be distributed and/or modified under the
% conditions of the LaTeX Project Public License version 1.3c,
% available at http://www.latex-project.org/lppl/.


\documentclass[11pt,a4paper,final]{moderncv}

% moderncv themes
% \moderncvtheme[blue]{casual}                 % optional argument are 'blue' (default), 'orange', 'red', 'green', 'grey' and 'roman' (for roman fonts, instead of sans serif fonts)
\moderncvtheme[grey]{classic}                % idem

% character encoding
\usepackage[utf8]{inputenc}                   % replace by the encoding you are using

% 
% \usepackage{hyperref} %ES
\usepackage{url} %ES
\usepackage{ifthen}
\usepackage{multicol} %ES
\usepackage{textcomp} %ES
\usepackage{moderncvcompatibility}

\newif\ifpaper 

% \papertrue % For committees, usually printed out and handed to members.
% \paperfalse % For individuals, use [doi] and [pdf] links to publications

% adjust the page margins
\usepackage[scale=0.8]{geometry}
% \setlength{\hintscolumnwidth}{1cm}			% if you want to change the width of the column with the dates
\AtBeginDocument{\setlength{\maketitlenamewidth}{7cm}}  % only for the classic theme, if you want to change the width of your name placeholder (to leave more space for your address details
\AtBeginDocument{\recomputelengths}                     % required when changes are made to page layout lengths


\input defs



% personal data
\firstname{\Huge{Evangelos}}
\familyname{\Huge{Siminos}}
 \title{curriculum vitae}               % optional, remove the line if not wanted
\address{Max Planck Institute for the Physics of Complex Systems\\
	  N\"{o}thnitzer Str. 38}
	{01187 Dresden, Germany}% optional, remove the line if not wanted
% \mobile{mobile (optional)}                    % optional, remove the line if not wanted
\phone{+49 351 871 2412}                      % optional, remove the line if not wanted
% \fax{fax (optional)}                          % optional, remove the line if not wanted
\email{siminos@gatech.edu}% optional, remove the line if not wanted
% \email{evangelos.siminos@cea.fr}% optional, remove the line if not wanted
% \extrainfo{\href{http://www.cns.gatech.edu/~siminos}{\url{www.cns.gatech.edu/~siminos}}} % Does not work
 \extrainfo{\httplink[http://www.cns.gatech.edu/$\sim$siminos]{www.cns.gatech.edu/~siminos}} % optional, remove the line if not wanted
% \photo[80pt]{siminos_small.jpg}                         % '64pt' is the height the picture must be resized to and 'picture' is the name of the picture file; optional, remove the line if not wanted
% \quote{Some quote (optional)}                 % optional, remove the line if not wanted

% \nopagenumbers{}                             % uncomment to suppress automatic page numbering for CVs longer than one page

%----------------------------------------------------------------------------------
%            content
%----------------------------------------------------------------------------------
\begin{document}
\maketitle

% \section{\textsc{Personal Details}}
% \cvitemSC{Date of Birth}{August 3, 1979}
% \cvitemSC{Place of Birth}{Thessaloniki, Greece}
% \cvitemSC{Nationality}{Greek}
% \cvitemSC{Family status}{Married, two children}

\section{\textsc{Education}}
\cventryAlt{2009}{PhD in Physics}{Georgia Institute of Technology}{Atlanta, GA, USA}{}{adviser: Prof. P. Cvitanovi\'{c}}
\cventryAlt{2005}{MS in Physics}{Georgia Institute of Technology}{Atlanta, GA, USA}{}{}%{Grade Point Average: 3.56/4} % arguments 3 to 6 are optional
\cventryAlt{2003}{BS in Physics}{University of Thessaloniki}{Thessaloniki, Greece}{}{}%{Average Grade: 8.17/10}
\cventryAlt{fall 2001}{Exchange Student}{Max Planck Institut f\"{u}r Plasmaphysik}{Greifswald, Germany}{}{}


\section{\textsc{Employment}}
\cventryAlt{2011 -- now}{Visiting Scientist}{Max Planck Institute for the Physics of Complex Systems}{}{}{Dresden, Germany}
\cventryAlt{2009 -- 2011}{Postdoctoral Fellow}{Commissariat \`{a} l' \'Energie Atomique (CEA), DAM, DIF}{}{}{Bruy\`{e}res-le-Ch\^{a}tel (Paris), France}
\cventryAlt{2008 -- 2009}{Research Assistant}{Center for Nonlinear Science}{School of Physics}{Georgia Tech}
			{support: NSF grant DMS-0807574 \& G.~Robinson~Fund}
\cventryAlt{summer 2005}{Research Assistant}{Center for Nonlinear Science}{School of Physics}{Georgia Tech}
			{support: G.~Robinson~Fund}
\cventryAlt{2003 -- 2008}{Teaching Assistant}{School of Physics, Georgia Tech}{}{}{}{}

\section{\textsc{Research Experience}}
\cventryAlt{2011 -- now}{Max Planck Inst. for the Physics of Complex Systems}{Germany}{}{}{}{}
\cventryDescr{}{\textit{Ultra-intense laser pulse propagation in solid density targets}}{with}{M. Grech, V.\,T.~Tikhonchuk \etal}{}{}
\cventryDescr{area}{Relativistic intensity laser-plasma interaction}
 		{tools}{Particle-in-Cell codes, cold fluid-plasma theory, nonlinear stability methods}
 		{main results}{Analysis of electron heating effect on self-induced transparency threshold for
				relativistic intensity pulses interacting with overdense plasmas
 				}

\cventryAlt{2009 -- 2011}{D\'{e}p. Physique Th\'{e}orique et Appliqu\'{e}e}{CEA, DAM, DIF}{France}{}{}{}
\cventryDescr{Project I}{\textit{Kinetic Description of Stimulated Raman Scattering}}{adviser}{D. B\'enisti}
		{area}{Basic plasma physics, inertial confinement fusion, nonlinear dynamics}
\cventryDescr{tools}{Kinetic theory, Galerkin projection methods, sparse eigenproblems, Arnoldi iteration, spectral deformation, Vlasov codes}%{Kinetic theory of plasmas, spectral methods, sparse eigenproblems, spectral deformation.}
		{main results}{A fast converging semi-analytic method for the computation of stability of nonlinear {Vlasov-Poisson} waves.
				Application to vortex fusion instabilities of electrostatic plasma waves.}
		{\sep in progress}{Application to the modeling and control of stimulated Raman scattering}
\cventryDescr{Project II}{\textit{Relativistic Solitary Waves in Plasmas}}{with}{G. S\'anchez-Arriaga, E. Lefebvre}
		{area}{Relativistic intensity laser-plasma interaction}
\cventryDescr{tools}{Plasma-fluid models, Hamiltonian dynamical systems, spectral methods}
		{main results}{Identification and classification of new families of solitary waves}
		{in progress}{Stability, excitation, physical relevance of solutions}{}{}
% \clearpage

\cventryAlt{2004 -- 2009}{Center for Nonlinear Science}{School of Physics, Georgia Tech}{USA}{}{}{}
\cventryDescr{PhD thesis}{\textit{Recurrent Spatio-temporal Structures in Presence of Continuous Symmetries}}{adviser}{Prof. P. Cvitanovi\'{c}}{}{}
\cventryDescr{area}{Spatially extended systems, chaos and turbulence}
		{tools}{Dynamical systems theory, symmetry reduction, state-space visualization, numerical integration of stiff partial differential equations, periodic orbit searches}
		{main results}{Efficient continuous symmetry reduction methods for systems with a high-dimensional state space.
				Geometric description of symmetry reduced Kuramoto-Sivashinsky and complex Lorenz attractors 
				in terms of the unstable manifolds of traveling waves.
				}

\cventryAlt{2002 -- 2003}{Department of Physics}{University of Thessaloniki}{Greece}{}{}{}
\cventryDescr{diploma thesis}{\textit{Lattice-gas modeling of anomalous diffusion}}{adviser}{Prof. L. Vlahos}
{description}{Numerical study of anomalous diffusion of passive tracers in a turbulent enviroment modeled by
	      a lattice-gas cellular automaton}

\cventryAlt{Fall 2001}{Max Planck Institut f\"{u}r Plasmaphysik}{Greifswald}{Germany}{}{}{}
\cventryDescr{project}{\textit{Asymptotic study of toroidal and helical MHD equilibria of magnetic confinement devices}}
{adviser}{Prof. J. N\"{u}hrenberg}
{description}{Perturbative study of the effect of magnetic field geometry in steady-state confinement
	      properties of tokamaks and stellarators}

\section{Teaching Experience}
\cventryAlt{fall 2008}{Symmetry in dynamical systems}{School of Physics}{Georgia Tech}{USA}
{Series of three lectures for the advanced graduate course {\em Nonlinear Dynamics} (PHYS 7224)}
\cventryAlt{2003--2008}{Teaching Assistant}{School of Physics}{Georgia Tech}{USA}{}
\cventryDescr{courses}{Undergraduate Physics I \& II, Physics Laboratory I \& II, Classical Mechanics I \& II, Electromagnetism, Special Relativity, Quantum Mechanics I}
	     {duties}{lab sessions, recitation sessions, office hours, preparation and grading of homework \& exams\sep}{}{}
\cventryAlt{1999-2000}{Teaching Assistant}{Department of Physics, University of Thessaloniki}{Greece}{}{}{}
\cventryDescr{fall 1999}{Lab assistant for Introductory Computer Lab}
	      {spring 2000}{Grader for course Calculus II}{}{}

\section{\textsc{Fellowships}}
	\cvitem{2007}{Gerondelis Foundation Graduate Student Fellowship, USA}
	\cvitem{2001}{Erasmus Fellowship, European Union}

\section{\textsc{Computer skills}}
% {environments}{Linux, Windows}
\cvcomputer{programming}{C/C++, Fortran, Python}{scripting}{Perl, bash}
\cvcomputer{markup}{\LaTeX, \textsc{html}}{other}{Mathematica, matplotlib, PETSc} %channelflow

\section{\textsc{Other Activities}}
\cventryDescr{2008}{Organized informal seminar for Center for Nonlinear Science, Georgia Tech.}{}{}{}{}

% \section{\textsc{Languages}}
% \cvcomputer{English}{fluent}{Greek}{native speaker}
% \cvcomputer{German}{fair}{French}{elementary}

% \\ \\ \\ \\ %\\ \\ \\ \\ \\ \\ \\ \\ \\
\section{\textsc{Seminar Talks}}
\cvitem{March 2011}{ETH Zurich, Department of Materials\newline
\emph{Stability of nonlinear waves in rarefied plasmas}}% March 30 2009
\cvitem{May 2011}{Max Planck Inst. for the Physics of Complex Systems, Dresden\newline
\emph{Stability of nonlinear waves in collisionless plasmas}}% May 30 2009


\section{\textsc{Recent Conferences}}
\cvitem{June 2011\\ poster}{EPS Conference on Plasma Physics, Strasbourg, France\newline
	\siminos,  D. B\'enisti and L. Gremillet, {\emtitle A spectral method for the stability of BGK modes and application to vortex-fusion instabilities}
}
\cvitem{May 2011\\ talk}{Chaos, Complexity and Transport, Marseilles, France\newline
	\siminos,  D. B\'enisti and L. Gremillet, {\emtitle A spectral method for the stability of nonlinear Vlasov-Poisson equilibria}
}
\cvitem{Nov. 2010\\ talk}{Annual Meeting of the APS Division of Plasma Physics, Chicago, IL, USA\newline
	\siminos,  D. B\'enisti and L. Gremillet, {\emtitle Stability of nonlinear Vlasov-Poisson equilibria through spectral deformation and Fourier-Hermite expansion}
}
\cvitem{Sept. 2010\\ poster}{International Workshop on Laser-Matter Interaction,  Porquerolles, France\newline
	\siminos,  D. B\'enisti and L. Gremillet, {\emtitle Stability of nonlinear Vlasov-Poisson equilibria through Fourier-Hermite expansion}
% 13-17 September 2010,
}
\cvitem{June 2009\\ poster}{Modern Challenges in Nonlinear Plasma Physics, Sani, Halkidiki, Greece\newline
 	\siminos, P. Cvitanovi\'c and R.\,L. Davidchack,
 	{\emtitle State-space geometry of a continuous symmetry reduced Kuramoto-Sivashinsky flow}
}
\cvitem{May 2009\\ talk}{SIAM Conference on Applications of Dynamical Systems, Snowbird, UT, USA\newline
	\siminos, P. Cvitanovi\'c and R.\,L. Davidchack,
	{\emtitle State-space geometry of a Kuramoto-Sivashinsky flow in
	terms of relative periodic orbits}\newline
	in Minisymposium: {\em Dynamical systems and turbulence: unstable periodic orbits}
}
\cvitem{Jan. 2009\\ poster}{Dynamics Days, San Diego, CA, USA\newline 
 	\siminos\ and  P.\ Cvitanovi\'c, {\emtitle Continuous symmetry reduction for high dimensional flows}
}
% \cvitem{January 2008}{Dynamics Days, Knoxville, TN, USA\newline
%  	\siminos, P.\ Cvitanovi\'c, and R.\,L.\ Davidchack, {\emtitle State space geometry of a spatio-temporally chaotic Kuramoto-Sivashinsky flow}
% }


\section{\textsc{References}}
% % \cvitem{}{Available upon request.}
\begin{minipage}{\textwidth}
\begin{multicols*}{2} % ES: * means all white space at last column.
\textbf{Prof. Predrag Cvitanovi\'c}\vspace{1pt}\\ 
Glen Robinson Chair in Nonlinear Sciences\\
School of Physics\\
Georgia Institute of Technology\\
Atlanta, GA 30332-0430, USA\\
\phonesymbol\ +1 404 385 2502\\
\emaillink[\emailsymbol\ predrag.cvitanovic@physics.gatech.edu]{predrag.cvitanovic@physics.gatech.edu}\\

\textbf{Dr. Didier B\'enisti}\vspace{1pt}\\
% Commissariat \`a l' \'Energie Atomique\\
CEA, DAM, DIF\\
F-91297 Arpajon, France\\
\phonesymbol\ +33 169 26 7396\\
\emaillink[\emailsymbol\ didier.benisti@cea.fr]{didier.benisti@cea.fr}\\


\columnbreak % ES: prevent breaking references accross entries
% 

\textbf{Dr. Erik Lefebvre}\vspace{1pt}\\
% Commissariat \`a l' \'Energie Atomique\\
CEA, DAM, DIF\\
F-91297 Arpajon, France\\
\phonesymbol\ +33 169 26 7378\\
\emaillink[\emailsymbol\ erik.lefebvre@cea.fr]{erik.lefebvre@cea.fr}\\


% \textbf{Prof. Loukas Vlahos}\vspace{1pt}\\
% Department of Physics\\
% Aristotle University of Thessaloniki\\
% Thessaloniki, 54006 Greece\\
% \phonesymbol\ +30 2310 998044\\ 
% \emaillink[\emailsymbol\ vlahos@astro.auth.gr]{vlahos@astro.auth.gr}\vspace{0.3cm}\\
% 
% \textbf{Dr. Ruslan L. Davidchack}\vspace{1pt}\\
% Department of Mathematics\\
% University of Leicester\\
% University Road\\
% Leicester LE1 7RH, UK\\
% \phonesymbol\ +44 116 252 3884\\
% \emaillink[\emailsymbol\ rld8@mcs.le.ac.uk]{rld8@mcs.le.ac.uk}\vspace{0.3cm}\\

\end{multicols*}
\end{minipage}



\clearpage
\input inset_publ





% Publications from a BibTeX file
% \nocite{siminos11,SiCvi10,SSL10,SCD07,benisti10-1,benisti10-2,benisti10-3}
% \bibliographystyle{unsrt}
% \bibliography{publications}       % 'publications' is the name of a BibTeX file

\closesection{}
% \vfill{\hfill{\scriptsize \today.}}

\end{document}


%% end of file `template_en.tex'.
