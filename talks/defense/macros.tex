% defSteady.tex for steady.tex
% $Author: predrag $ $Date: 2007-05-22 01:08:51 -0400 (Tue, 22 May 2007) $

% before you change `v' into `b' in a macro just for kicks,
%          STOP! in the name of love!
% you are editing this one paper - Predrag's is editing a triptych of related
% PDE papers + ChaosBook.org + n student repositories, all of which need similar
% text and identical macros.

% Predrag           Dec  3 2006
% Predrag           Apr 21 2006
% based on defs.tex     FRG05 proposal
% Predrag           sep 14 2005
% based on Waleffe macros   sep  3 2005

%%%%%%%%%%%%%%%%%%%%%% Weblinks in PDF %%%%%%%%%%%%%%%%%%%
\newcommand{\weblink}[1]{\href{http://#1}{{\tt #1}}}
\newcommand{\arXiv}[1]{\href{http://arXiv.org/abs/#1}{{\tt #1}}}

%%%%%%%%%%%%%%%%%%%%%% COMMENTS %%%%%%%%%%%%%%%%%%%

\newcommand{\rf}     [1] {~\cite{#1}}
\newcommand{\refref} [1] {ref.~\cite{#1}}
\newcommand{\refRef} [1] {Ref.~\cite{#1}}
\newcommand{\refrefs}[1] {refs.~\cite{#1}}
\newcommand{\refRefs}[1] {Refs.~\cite{#1}}
\newcommand{\refeq}  [1] {(\ref{#1})}
\newcommand{\refeqs} [2]{(\ref{#1}--\ref{#2})}
\newcommand{\eqn}[1]{eqn.\ {\ref{#1}}}
\newcommand{\Eqn}[1]{Eqn.\ {\ref{#1}}}
\newcommand{\refpage}[1] {p.~\pageref{#1}}
\newcommand{\reffig} [1] {figure~\ref{#1}}
\newcommand{\reffigs} [2] {figures~\ref{#1} and~\ref{#2}}
\newcommand{\refFig} [1] {Figure~\ref{#1}}
\newcommand{\refFigs} [2] {Figures~\ref{#1} and~\ref{#2}}
\newcommand{\reftab} [1] {table~\ref{#1}}
\newcommand{\refTab} [1] {Table~\ref{#1}}
\newcommand{\reftabs}[2] {tables~\ref{#1} and~\ref{#2}}
\newcommand{\refsect}[1] {\S\,\ref{#1}}
\newcommand{\refsects}[2] {\S\,\ref{#1} and \S\,\ref{#2}}
\newcommand{\refSect}[1] {\S\,\ref{#1}}
\newcommand{\refappe}[1] {appendix~\ref{#1}}
\newcommand{\refappes}[2] {appendices~\ref{#1} and \ref{#2}}
\newcommand{\refAppe}[1] {Appendix~\ref{#1}}

%%%%%%%%%%%%%%% EQUATIONS %%%%%%%%%%%%%%%%%%%%%%%%%%%%%%%
\newcommand{\continue}{\nonumber \\ }
\newcommand{\nnu}{\nonumber}
\newcommand{\bea}{\begin{eqnarray}}
\newcommand{\eea}{\end{eqnarray}}
\newcommand{\barr}{\begin{array}}
\newcommand{\earr}{\end{array}}
% Unfortunately these don't work. Latex doesn't recognize the \end{align}.
%\newcommand{\bal}{\begin{align}}
%\newcommand{\eal}{\end{align}}

%%%%%%%%%%%%%%% DIVAKAR"S FAVORITE MACROS %%%%%%%%%%%%%%%%%%%%%%
\newcommand {\abs} [1] {\left| #1 \right|}
\newcommand {\Abs} [1] {\bigl\lvert #1 \bigr\rvert}

%%%%%%%%%%%%    PREDRAG'S FAVORITE MACROS %%%%%%%%%%%%%
\newcommand{\NS}{Navier-Stokes}
\newcommand{\NSe}{Navier-Stokes equation}
\newcommand{\KS}{Kuramoto-Sivashinsky}
\newcommand{\KSe}{Kuramoto-Sivashinsky equation}
\newcommand{\Reynolds}{\textit{Re}}  % Reynolds number
\newcommand{\pCf}{plane Couette flow}
\newcommand{\PCf}{Plane Couette flow}
\newcommand{\ubranch}{upper-branch}
\newcommand{\Ubranch}{Upper-branch}
\newcommand{\lbranch}{lower-branch}
\newcommand{\Lbranch}{Lower-branch}

\newcommand{\ew}{eigen\-value}
\newcommand{\ev}{eigen\-vector}
\newcommand{\ef}{eigen\-function}


%%%%%%%%%%%%%%% GIBSON FAVORITE MACROS %%%%%%%%%%%%%%%%%%%%%%

\newcommand{\qb}{\ensuremath{\quad \bullet \;}}
\newcommand{\qqc}{\ensuremath{\qquad \quad \circ \;}}
\newcommand{\phx}{\phantom{+}}
\newcommand{\bu}{\ensuremath{{\bf u}}}
\newcommand{\bv}{\ensuremath{{\bf v}}}
\newcommand{\bff}{\ensuremath{{\bf f}}}
\newcommand{\dbu}{\delta {\bf u}}
\newcommand{\dbv}{\delta {\bf v}}
\newcommand{\hbu}{\hat{{\bf u}}}
\newcommand{\hbv}{\hat{{\bf v}}}
\newcommand{\hu}{\hat{u}}
\newcommand{\hv}{\hat{v}}
\newcommand{\hw}{\hat{w}}
%\newcommand{\bnabla}{{\boldmath \nabla}} % what's wrong with this?
\newcommand{\be}{{\bf e}}
\newcommand{\bx}{{\bf x}}
\newcommand{\ex}{{\hat{\bf x}}} % unit vectors
\newcommand{\ey}{{\hat{\bf y}}}
\newcommand{\ez}{{\hat{\bf z}}}
\newcommand{\Om}{\Omega}    % JFG mantra
\newcommand{\tny}[1]{{\text{\tiny {#1}}}}
\newcommand{\bPhi}{{\bf \Phi}}
\newcommand{\bphi}{{\bf \phi}}
\newcommand{\bhphi}{{\bf \hat{\phi}}}
\newcommand{\bU}{{\bf U}}
\newcommand{\bW}{{\bf W}}
\newcommand{\lapl}{\nabla^2}
\newcommand{\tEQ}{\ensuremath{{\text{EQ}}}}
\newcommand{\tNB}{\ensuremath{{\text{NB}}}}
\newcommand{\tLB}{\ensuremath{{\text{LB}}}}
\newcommand{\tUB}{\ensuremath{{\text{UB}}}}
\newcommand{\tLM}{\ensuremath{{\text{LM}}}}
\newcommand{\tNS}{\ensuremath{{\text{NS}}}}
\newcommand{\tCFD}{\ensuremath{{\text{CFD}}}}
% \newcommand{\tODE}{\text{ODE}}    % JFG school of decoration
\newcommand{\tODE}{}        % PC: ODE is only one we use
\newcommand{\stagn}{\ensuremath{\text{\tiny EQ}}}% JFG equilib/stagnation point
%\newcommand{\aEQ}{\ensuremath{a_{\text{\tiny EQ}}}}
\newcommand{\aEQ}{\ensuremath{a}}
\newcommand{\uEQ}{\ensuremath{\bu_{\text{\tiny EQ}}}}
\newcommand{\vEQ}{\ensuremath{\bv_{\text{\tiny EQ}}}}
\newcommand{\uLM}{\ensuremath{\bu_{\text{\tiny LM}}}}
\newcommand{\uLB}{\ensuremath{\bu_{\text{\tiny LB}}}}
\newcommand{\uNB}{\ensuremath{\bu_{\text{\tiny NB}}}}
\newcommand{\uUB}{\ensuremath{\bu_{\text{\tiny UB}}}}
\newcommand{\vLM}{\ensuremath{\bv_{\text{\tiny LM}}}}
\newcommand{\vLB}{\ensuremath{\bv_{\text{\tiny LB}}}}
\newcommand{\vNB}{\ensuremath{\bv_{\text{\tiny NB}}}}
\newcommand{\vUB}{\ensuremath{\bv_{\text{\tiny UB}}}}
\newcommand{\huUB}{\ensuremath{{\bf \hat{u}}_{\text{\tiny UB}}}}
\newcommand{\bbR}{\mathbb{R}}
\newcommand{\bbU}{\mathbb{U}}
\newcommand{\bbUsymm}{\bbU_{S}}
\newcommand{\half}{\frac{1}{2}}
\newcommand{\pd}[2]{\frac{\partial #1}{\partial #2}}
\newcommand{\Norm}[1]{\|{#1}\|}
\newcommand{\grad}{\boldsymbol{\nabla}}
\newcommand{\eUB}[1]{\be_{\tny{UB},#1}}

%%%%%multiletter symbols
\newcommand\Real{\mbox{Re}} % cf plain TeX's \Re, not Reynolds number
\newcommand\Imag{\mbox{Im}} % cf plain TeX's \Im

%%%%%%%%%%%%%%% Sundry symbols within math eviron.: %%%%%%%%%%%%
\renewcommand\Im{{\rm Im\,}}
\renewcommand\Re{{\rm Re\,}}
\newcommand{\Det}{\mbox{\rm Det}\,}
\newcommand{\tr}{\mbox{\rm tr}\,}
\newcommand{\Tr}{\mbox{\rm tr}\,}

%%       optional parameter comes in [\ldots], for example
%%       \newcommand\eigRe[1][ ]{\ensuremath{\mu_{#1}}}
%%       no subscript: \eigRe\
%%       with subscript j: \eigRe[j]
%%
%%      Guckenheimer-Holmes:  lambda = alpha + i beta
%%      Hirsch-Smale:         lambda = a     + i b
%%      Boyce-di Prima:       lambda = mu    + i nu
%%      Gibson:        lambda = mu    + i omega (best of the bunch!)
%
% Re eigen-exponent superscripting
% Getting into the ChaosBookie groove... awesome!
\newcommand{\eigExp}[1][]{
\ifthenelse{\equal{#1}{}}{\ensuremath{\lambda}}{\ensuremath{\lambda^{(#1)}}}
                        }
\newcommand{\eigRe}[1][]{
\ifthenelse{\equal{#1}{}}{\ensuremath{\mu}}{\ensuremath{\mu^{(#1)}}}
                        }
\newcommand{\eigIm}[1][]{
  \ifthenelse{\equal{#1}{}}{\ensuremath{\omega}}{\ensuremath{\omega^{(#1)}}}
                        }
    % Guck & Holmes use $W^s$, $W^u$ for stable, unstable manifolds.
    % usage: \Wmnfld{u}{} unstable
    % \Wmnfld{s,k}{EQ} k-th stable, equilibrium EQ
\newcommand{\Wmnfld}[2]{
\ifthenelse{\equal{#2}{}}{\ensuremath{W^{#1}}}{\ensuremath{W^{#1}_{\text{\tiny #2}}}}
                        }


%%%%%%%%% Flows

\newcommand{\Le}{Lorenz equations}
\newcommand{\CLe}{Complex Lorenz equations}
\newcommand{\CLf}{Complex Lorenz flow}
\newcommand{\AGHe}{Armbruster-Guckenheimer-Holmes flow}

%%%%%%%% Symmetries
\newcommand{\On}[1]{\ensuremath{O(#1)}}
\newcommand{\Rg}[1]{\Rls{#1}}
\newcommand{\Idg}{\ensuremath{\mathbf{1}}}
\newcommand{\SOn}[1]{\ensuremath{SO(#1)}}
\newcommand{\Dn}[1]{\ensuremath{D_{#1}}}
\newcommand{\Cn}[1]{\ensuremath{C_{#1}}}
\newcommand{\Zn}[1]{\ensuremath{Z_{#1}}}
\newcommand{\En}[1]{\ensuremath{E(#1)}}
%\newcommand{\Zn}[1]{\ensuremath{C_#1}}         % in DasBuch
%\newcommand{\Ztwo}{\ensuremath{\mathbf{Z}_2}}   % in thesis (obsolete)
\newcommand{\Ztwo}{\ensuremath{D_1}}           % in DasBuch & thesis
\newcommand{\Refl}{\ensuremath{\kappa}}         %%%% Changed this, use R for rotations.
\newcommand{\Shift}{\ensuremath{\tau}}
\newcommand{\shift}{\ensuremath{\ell}}
\newcommand{\Rot}[1]{\ensuremath{R(#1)}}
\newcommand{\Rotn}[1]{\ensuremath{R_{#1}}}
\newcommand{\Drot}{\ensuremath{\zeta}}
\newcommand{\Lg}{\mathfrak{a}}
\newcommand{\stab}[1]{\ensuremath{\Sigma_{#1}}}
\newcommand{\globstab}[1]{\ensuremath{\Sigma_{#1}}} % Change to be the same as stab. Was \Sigma^\ast_{#1}
\newcommand{\Str}[1]{\ensuremath{\mathcal{S}_{#1}}} % Stratum
\newcommand{\Fix}[1]{\ensuremath{\mathrm{Fix}\left(#1\right)}}
\newcommand{\Nlz}[1]{\ensuremath{N(#1)}}
\newcommand{\doubleperiod}[1]{{\ensuremath{\mathcal{T}_{#1}}}}
%%%%%%%%%%%%%%%%%%%%%%


%%%%%%%%%%%% MACROS, Siminos thesis specific %%%%%%%%%%

\newcommand{\vf}{v}	%%% keep notation for vector field flexible. For the time being follow Das Buch.
\newcommand{\Lint}[1]{\frac{1}{L}\!\oint d#1\,}
\newcommand{\ode}{ODE}
\newcommand{\Rls}[1]{\ensuremath{\mathbb{R}^{#1}}}
\newcommand{\Clx}[1]{\ensuremath{\mathbb{C}^{#1}}}
\newcommand{\conj}[1]{\ensuremath{\bar{#1}}}
\newcommand{\trace}{\mbox{\rm trace}\,}
\newcommand{\Manif}{\ensuremath{\mathcal{M}}}
\newcommand{\Order}[1]{\mathrm{O}(#1)}
\newcommand{\tildeL}{\ensuremath{\tilde{L}}}


%%%%%%%%%%%%%% TO DELETE, EVENTUALLY: %%%%%%%%%%%%%%%%%%%%%
%%
