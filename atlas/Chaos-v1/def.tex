
        \newcommand{\wwwcb}[1]{       % keep homepage flexible:
                  {\tt \href{http://ChaosBook.org#1}
              {ChaosBook.org#1}}}
       \newcommand{\wwwgt}{{\tt \href{http://birtracks.eu}
              {birtracks.eu}}}
       \newcommand{\wwwQFT}[1]{
                  {\tt \href{http://ChaosBook.org/FieldTheory#1}
              {ChaosBook.org/\-Field\-Theory#1}}}
       \newcommand{\wwwcnsQFT}[1]{
                  {\tt \href{http://ChaosBook.org/FieldTheory#1}
              {ChaosBook.org/\-Field\-Theory#1}}}
       \newcommand{\weblink}[1]{{\tt \href{http://#1}{#1}}}
       \newcommand{\HREF}[2]{
              {\href{#1}{#2}}}
       \newcommand{\mpArc}[1]{
              {\tt \href{http://www.ma.utexas.edu/mp_arc-bin/mpa?yn=#1}
                   {\goodbreak mp\_arc~#1}}}
       \newcommand{\arXiv}[1]{
              {\tt \href{http://arXiv.org/abs/#1}{\goodbreak arXiv:#1}}}

%%%%%%%%%%%%%%% EQUATIONS %%%%%%%%%%%%%%%%%%%%%%%%%%%%%%%
\newcommand{\beq}{\begin{equation}}
\newcommand{\continue}{\nonumber \\ }
\newcommand{\nnu}{\nonumber}
\newcommand{\eeq}{\end{equation}}
\newcommand{\ee}[1] {\label{#1} \end{equation}}
\newcommand{\bea}{\begin{eqnarray}}
\newcommand{\ceq}{\nonumber \\ & & }
\newcommand{\eea}{\end{eqnarray}}
\newcommand{\barr}{\begin{array}}
\newcommand{\earr}{\end{array}}

%%%%%%%%%%%%%%% REFERENCING EQUATIONS ETC. %%%%%%%%%%%%%%%%%%%%%%%%%%%%%%%
\newcommand{\rf}     [1] {~\cite{#1}}
\newcommand{\refref} [1] {ref.~\cite{#1}}
\newcommand{\refRef} [1] {Ref.~\cite{#1}}
\newcommand{\refrefs}[1] {refs.~\cite{#1}}
\newcommand{\refRefs}[1] {Refs.~\cite{#1}}
\newcommand{\refeq}  [1] {(\ref{#1})}
\newcommand{\refeqs} [2]{(\ref{#1}--\ref{#2})}
\newcommand{\refpage}[1] {page~\pageref{#1}}
\newcommand{\reffig} [1] {figure~\ref{#1}}
\newcommand{\reffigs} [2] {figures~\ref{#1} and~\ref{#2}}
\newcommand{\refFig} [1] {Figure~\ref{#1}}
\newcommand{\refFigs} [2] {Figures~\ref{#1} and~\ref{#2}}
\newcommand{\reftab} [1] {table~\ref{#1}}
\newcommand{\refTab} [1] {Table~\ref{#1}}
\newcommand{\reftabs}[2] {tables~\ref{#1} and~\ref{#2}}
\newcommand{\refsect}[1] {sect.~\ref{#1}}
\newcommand{\refsects}[2] {sects.~\ref{#1} and \ref{#2}}
\newcommand{\refSect}[1] {Sect.~\ref{#1}}
\newcommand{\refSects}[2] {Sects.~\ref{#1} and \ref{#2}}
\newcommand{\refchap}[1] {chapter~\ref{#1}}
\newcommand{\refChap}[1] {Chapter~\ref{#1}}
\newcommand{\refchaps}[2] {chapters~\ref{#1} and~\ref{#2}}
\newcommand{\refchaptochap}[2] {chapters~\ref{#1} to~\ref{#2}}
\newcommand{\refappe}[1] {appendix~\ref{#1}}
\newcommand{\refappes}[2] {appendices~\ref{#1} and~\ref{#2}}
\newcommand{\refAppe}[1] {Appendix~\ref{#1}}
\newcommand{\refrem} [1] {remark~\ref{#1}}
\newcommand{\refexam}[1] {example~\ref{#1}}
\newcommand{\refExam}[1] {Example~\ref{#1}}
\newcommand{\refexer}[1] {exercise~\ref{#1}}
\newcommand{\refExer}[1] {Exercise~\ref{#1}}
\newcommand{\refsolu}[1] {solution~\ref{#1}}
\newcommand{\refSolu}[1] {Solution~\ref{#1}}

%%%%%%%%%%%%%%  Abbreviations %%%%%%%%%%%%%%%%%%%%%%%%%%%%%%%%%%%%%%%%
%%% APS (American Physiology Society, it seems) style:
%%%     Latin or foreign words or phrases should be roman, not italic.
%%%     Insert a `hard' space after full points
%%%                                         that do not end sentences.

\newcommand{\etc}{{etc.}}       % APS
\newcommand{\etal}{{\em et al.}}    % etal in italics, APS too
\newcommand{\ie}{{i.e.}}        % APS
\newcommand{\cf}{{\em cf.\ }}     % APS
\newcommand{\eg}{{e.g.\ }}        % APS, OUP, hard space '\eg\ NextWord'

%%%%%%%%%%%%%%% ChaosBook Abbreviations %%%%%%%%%%%%%%%%%%%%%%%%

\newcommand{\evOper}{evolution oper\-ator}
\newcommand{\EvOper}{Evolution oper\-ator}
 %% \newcommand{\evOp}{Ruelle operator} %could be ``evolution'' instead?
%\newcommand{\FPoper}{Frobenius-Perron oper\-ator}
\newcommand{\FPoper}{Perron-Frobenius oper\-ator} % Pesin's ordering
\newcommand{\FP}{Perron-Frobenius}
\newcommand{\statesp}{state space}
\newcommand{\Statesp}{State space}
\newcommand{\fixedpnt}{fixed point}
\newcommand{\Fixedpnt}{fixed point}
\newcommand{\maslov}{topological}
\newcommand{\Maslov}{Topological}
%\newcommand{\Maslov}{Keller-Maslov}
\newcommand{\jacobian}{Jacobian}        % determinant
\newcommand{\jacobianM}{Jacobian matrix}  % back to Predrag's name 20oct2009
\newcommand{\jacobianMs}{Jacobian matrices}   % matrices
\newcommand{\JacobianM}{Jacobian matrix} %
\newcommand{\JacobianMs}{Jacobian matrices}  %
\newcommand{\FloquetM}{Floquet matrix} % specialized to periodic orb
\newcommand{\FloquetMs}{Floquet matrices}  %
% \newcommand{\stabmat}{matrix of variations}   % Arnold, says Vattay
\newcommand{\stabmat}{stability matrix}     % stability matrix, velocity gradients
\newcommand{\Stabmat}{Stability matrix}     % Stability matrix
\newcommand{\stabmats}{stability matrices}
\newcommand{\monodromyM}{monodromy matrix} % monodromy matrix, Poincare cut
\newcommand{\MonodromyM}{Monodromy matrix} % monodromy matrix, Poincare cut
\newcommand{\dzeta}{dyn\-am\-ic\-al zeta func\-tion}
\newcommand{\Dzeta}{Dyn\-am\-ic\-al zeta func\-tion}
\newcommand{\tzeta}{top\-o\-lo\-gi\-cal zeta func\-tion}
\newcommand{\Tzeta}{Top\-o\-lo\-gi\-cal zeta func\-tion}
\newcommand{\BERzeta}{BER zeta func\-tion}
%\newcommand{\tzeta}{Artin-Mazur zeta func\-tion} %alternative to topological
\newcommand{\qS}{semi\-classical zeta func\-tion}
%\newcommand{\qS}{Gutz\-willer-Voros zeta func\-tion}
\newcommand{\Gt}{Gutz\-willer trace formula}
\newcommand{\Fd}{spec\-tral det\-er\-min\-ant}
%\newcommand{\fd}{spec\-tral det\-er\-min\-ant} %in many articles
\newcommand{\FD}{Spec\-tral det\-er\-min\-ant}
\newcommand{\cFd}{semiclass\-ic\-al spec\-tral det\-er\-mi\-nant}
\newcommand{\cFD}{Semiclass\-ic\-al spec\-tral det\-er\-mi\-nant}
% \newcommand{\cFd}{semiclass\-ic\-al Fred\-holm det\-er\-mi\-nant}
\newcommand{\Vd}{Vattay det\-er\-mi\-nant}
\newcommand{\cycForm}{cycle averaging formula}
\newcommand{\CycForm}{Cycle averaging formula}
\newcommand{\freeFlight}{mean free flight time}
\newcommand{\FreeFlight}{Mean free flight time}
\newcommand{\pdes}{partial differential equations}
\newcommand{\Pdes}{Partial differential equations}
\newcommand{\dof}{dof}         % Hamiltonian deegree of freedom
% \newcommand{\dof}{deegree of freedom}

%%%%%%%%%%%%%%% VECTORS, MATRICES, NORMS %%%%%%%%%%%%%%%%%%%%%%%%%%%%%%%%%

\newcommand{\braket}[2]
		   {\langle{#1}\vphantom{#2}|\vphantom{#1}{#2}\rangle}
\newcommand{\bra}[1]{\langle{#1}\vphantom{ }|}
\newcommand{\ket}[1]{|\vphantom{}{#1}\rangle}
%                        {#2}\right\rangle}

\newcommand{\dual}[1]{{#1}^\ast}

%%%%%%%%%%%%%%% Sundry symbols within math eviron.: %%%%%%%%%%%%

\newcommand{\obser}{\ensuremath{a}}     % an observable from phase space to R^n
\newcommand{\Obser}{\ensuremath{A}}     % time integral of an observable
\newcommand{\onefun}{\iota} % the function that returns one no matter what
\newcommand{\defeq}{=}      % the different equal for a definition
\newcommand {\deff}{\stackrel{\rm def}{=}}
\newcommand{\reals}{\mathbb{R}}
\newcommand{\complex}{\mathbb{C}}
\newcommand{\integers}{\mathbb{Z}}
\newcommand{\rationals}{\mathbb{Q}}
\newcommand{\naturals}{\mathbb{N}}
\newcommand{\LieD}{{{\cal L}\!\!\llap{-}\,\,}}  % {{\pound}} % Lie Derivative
\newcommand{\half}{{\scriptstyle{1\over2}}}
\newcommand{\pde}{\partial}
\newcommand{\pdfrac}[2]{{\partial #1 \over \partial #2}}
\renewcommand\Im{\ensuremath{{\rm Im}\,}}
\renewcommand\Re{\ensuremath{{\rm Re}\,}}
\renewcommand{\det}{\mbox{\rm det}\,}
\newcommand{\Det}{\mbox{\rm Det}\,}
\newcommand{\tr}{\mbox{\rm tr}\,}
\newcommand{\Tr}{\mbox{\rm tr}\,}
%\newcommand{\Tr}{\mbox{Tr}\,}
\newcommand{\sign}[1]{\sigma_{#1}}
%\newcommand{\sign}[1]{{\rm sign}(#1)}
\newcommand{\mInv}{{I}}                 % material invariant
\newcommand{\msr}{\ensuremath{\rho}}                % measure
\newcommand{\Msr}{{\mu}}                % coarse measure
\newcommand{\dMsr}{{d\mu}}              % measure infinitesimal
\newcommand{\SRB}{{\rho_0}}             % natural measure
\newcommand{\vol}{{V}}                  % volume of i-th tile
\newcommand{\prpgtr}[1]{\delta\negthinspace\left( {#1} \right)}
%\newcommand{\Zqm}{\ensuremath{Z_{qm}}}         % Gutz-Voros zeta function
\newcommand{\Zqm}{\ensuremath{\det(\hat{H} - E)_{sc} }} % semicls spectr. det:
\newcommand{\Fqm}{\ensuremath{F_{qm}}}
\newcommand{\zfct}[1]{\zeta ^{-1}_{#1}}
\newcommand{\zetaInv}{\ensuremath{1/\zeta}}
% \newcommand{\zetaInv}{{\zeta^{-1}}}
\newcommand{\zetatop}{\ensuremath{1/\zeta_{\mbox{\footnotesize top}} }}
\newcommand{\zetaInvBER}[1]{1/\zeta_{\mbox{\footnotesize BER}}(#1)}
\newcommand{\BER}[1]{{\mbox{\footnotesize BER}}} % Baladi-Ruelle-Eckmann
\newcommand{\eigCond}{\ensuremath{F}}           % eigenvalue cond. function
\newcommand{\expct}    [1]{\left\langle {#1} \right\rangle}
\newcommand{\spaceAver}[1]{\left\langle {#1} \right\rangle}
\newcommand{\timeAver} [1]{\overline{#1}}
\newcommand{\norm}[1]{\left\Arrowvert \, #1 \, \right\Arrowvert}
\newcommand{\pS}{\ensuremath{{\cal M}}}          % symbol for state space
\newcommand{\ssp}{\ensuremath{x}}                % state space point
\newcommand{\tissp}{\tilde{\Delta\ssp}} % Rytis \CostFct
\newcommand{\pSpace}{x}       % Hamiltonian phase space x=(q,p) coordinate
\newcommand{\coord}{q}        % configuration space p coordinate
\newcommand{\DOF}{\ensuremath{D}}          % Hamiltonian deegree of freedom
\newcommand{\NWS}{\ensuremath{\Omega}}     % symbol for the non--wandering set
\newcommand{\AdmItnr}{\Sigma}      % set of admissible itineraries
\newcommand{\intM}[1]{{\int_\pS{\!d #1}\:}} %phase space integral
\newcommand{\Cint}[1]{\oint\frac{d#1}{2 \pi i}\;} %Cauchy contour integral
\newcommand{\PoincS}{\ensuremath{{\cal P}}}  % symbol for Poincare section
\newcommand{\PoincM}{\ensuremath{P}}       % symbol for Poincare map
\newcommand{\PoincC}{\ensuremath{U}}       % symbol for Poincare constraint function
\newcommand{\arc}{\ensuremath{s}}          % symbol for billiard wall arc
\newcommand{\mompar}{\ensuremath{p}}       % billiard wall parall. momentum
\newcommand{\restCoeff}{\ensuremath{\gamma}}  % billiard wall restitution coeff
\newcommand{\timeIn}[1]{{t^{-}_{#1}}} % billiard wall time of arrival
\newcommand{\timeOut}[1]{{t^{+}_{#1}}}   % billiard wall time of departure
%\newcommand{\PoincS}{\partial{\cal M}}          % billiard Poincare section
\newcommand{\Lop}{\ensuremath{{\cal L}}}       % evolution operator
\newcommand{\Uop}{\ensuremath{{\cal K}}}       % Koopman operator, Driebe notation
\newcommand{\Aop}{\ensuremath{{\cal A}}}       % evolution generator
\newcommand{\TrOp}{\ensuremath{{\cal T}}}       % transfer operator, like in statmech
\newcommand{\matId}{\ensuremath{{\bf 1}}}      % matrix identity
\newcommand{\eigenvL}{\ensuremath{s}}      % evolution operator eigenvalue
\newcommand{\eigenvG}{\ensuremath{m}}      % compact group eigenvalues
\newcommand{\inFix}[1]{{\in \mbox{\footnotesize Fix}f^{#1}}}
\newcommand{\inZero}[1]{{\in \mbox{\footnotesize Zero} \, f^{#1} }}
\newcommand{\xzero}[1]{{x_{#1}^\ast}}
\newcommand{\fractal}{{{\cal F}}}
\newcommand{\contract}{F}
% \newcommand{\presentation}{P} % PC commented out 7sep2008
\newcommand{\orderof}[1]{o(#1)} % Rytis 22mar2005

     %%%%%%%%%% flows: %%%%%%%%%%%%%%%%%%%%%%%%%%%%
\newcommand\map{f}                  % other people like \phi's etc
\newcommand\flow[2]{{f^{#1}(#2)}}
\newcommand{\vel}{\ensuremath{v}}   % state space velocity

   %%%%%%%% Siminos thesis %%%%%%%%%%%%%%%%%%%%%%%%%%%%
\newcommand{\Le}{Lorenz equations}
\newcommand{\rLor}{\rho}    % parameter r in Lorenz paper
\newcommand{\cLe}{complex Lorenz equations}
\newcommand{\cLf}{complex Lorenz flow}
\newcommand{\CLe}{Complex Lorenz equations}
\newcommand{\CLf}{Complex Lorenz flow}
\newcommand{\RerCLor}{\rho_1}    % real      part of parameter r, CLe
\newcommand{\ImrCLor}{\rho_2}    % imaginary part of parameter r, CLe
% \newcommand{\AGHe}{Armbruster-Guckenheimer-Holmes flow}

     %%%%%%%%%% periods: %%%%%%%%%%%%%%%%%%%%%%%%%%%%
\newcommand\period[1]{{\ensuremath{T_{#1}}}}         %continuous cycle period
%\newcommand\period[1]{{\tau_{#1}}}
\newcommand{\cl}[1]{{\ensuremath{n_{#1}}}}   % discrete length of a cycle, Predrag
%\newcommand{\cl}[1]{|#1|}  % the length of a periodic orbit, Ronnie
\newcommand{\nCutoff}{N}    % maximal cycle length
                % maximal stability cutoff:
\newcommand{\stabCutoff}{\ExpaEig_{\mbox{\footnotesize max}}}
\newcommand{\timeSegm}[1]{{\tau_{#1}}}      %billiard segment time period
\newcommand{\timeStep}{\ensuremath{{\delta \tau}}}  %integration step
\newcommand{\deltaX}{\ensuremath{{\delta x}}}       %trajectory displacement
\newcommand{\unitVec}{\ensuremath{\hat{n}}}     %unit vector

\newcommand{\Mvar}{\ensuremath{A}}  % stability matrix
\newcommand{\derF}[1]{\ensuremath{A(#1)}}   % Predrag stability matrix
 %\newcommand{\derF}[1]{{DF |_{#1}}}        % Gibson stability matrix
\newcommand{\jMps}{\ensuremath{J}}   % jacobian matrix, phase space/state space
% \newcommand{\jMps}{\ensuremath{{\bf J}}}  % bold fundamental matrix phase space
\newcommand{\derf}[2]{\ensuremath{{J}^{#1}(#2)}}    % Predrag fundamental matrix
% \newcommand{\derf}[2]{\ensuremath{{\bf J}^{#1}(#2)}}  % Predrag bold fundamental matrix
 % \newcommand{\derf}[2]{{Df^{#1}|_{#2}}}   % Gibson fundamental matrix
\newcommand{\jMConfig}{\ensuremath{{\bf j}}}    % fundamental matrix, configuration space
\newcommand{\jConfig}{\ensuremath{j}}      % jacobian, configuration space
\newcommand{\jMP}{\ensuremath{\hat{J}}}   % jacobian matrix, Poincare return
% \newcommand{\jMP}{\ensuremath{{\bf \hat{J}}}}   % bold jacobian matrix, Poincare return
\newcommand{\monodromy}{\ensuremath{M}}   % monodromy matrix, full Poincare cut
% \newcommand{\monodromy}{\ensuremath{{\bf M}}}   % bold monodromy matrix, full Poincare cut
                   % Fredholm det jacobian weight:
%\newcommand{\jEigvec}[1]{\ensuremath{{\bf e}^{(#1)}}}  % right jacobiam eigenvector
%\newcommand{\jEigvecT}[1]{\ensuremath{{\bf e}_{(#1)}}}  % left jacobiam eigenvector
\newcommand{\jEigvec}[1][]{\ensuremath{{\bf e}^{(#1)}}} % right jacobiam eigenvector
\newcommand{\jEigvecT}[1][]{\ensuremath{{\bf e}_{(#1)}}} % left jacobiam eigenvector
\newcommand{\oneMinJ}[1]
           {\left|\det\!\left(\matId-\monodromy_p^{#1}\right)\right|}
\newcommand{\maslovInd}{\ensuremath{m}}        % Maslov index
\newcommand{\ExpaEig}{\ensuremath{\Lambda}}
\newcommand{\Lyap}{\ensuremath{\lambda}}            %Lyapunov exponent

%%   optional parameter comes in [\ldots], for example
%%   \newcommand\eigRe[1][ ]{\ensuremath{\mu_{#1}}}
%%   no subscript: \eigRe\
%%   with subscript j: \eigRe[j]
%%
% \newcommand{\eigExp}[1][ ]{\ensuremath{\lambda_{#1}}}   % complex eigenexponent
%%  Guckenheimer&Holmes:  lambda = alpha + i beta
%%  Hirsch-Smale:         lambda = a     + i b
%%  Boyce-di Prima:       lambda = mu    + i nu
% \newcommand{\eigRe}[1][ ]{\ensuremath{\mu_{#1}}}    % Re eigenexponent
% \newcommand{\eigIm}[1][ ]{\ensuremath{\nu_{#1}}}    % Im eigenexponent

\newcommand{\cycle}[1]{\ensuremath{\overline{#1}}}

%%%%%%%%%%%%%%% relative periodic orbits: %%%%%%%%%%%%%%%%%%%%%%%%%%%%
\newcommand{\po}{periodic orbit}
\newcommand{\Po}{Periodic orbit}
\newcommand{\rpo}{rela\-ti\-ve periodic orbit}
%   \newcommand{\rpo}{equi\-vari\-ant periodic orbit}
\newcommand{\Rpo}{Rela\-ti\-ve periodic orbit}
%   \newcommand{\Rpo}{Equi\-vari\-ant periodic orbit}
\newcommand{\eqv}{equi\-lib\-rium}
\newcommand{\Eqv}{Equi\-lib\-rium}
\newcommand{\eqva}{equi\-lib\-ria}
\newcommand{\Eqva}{Equi\-lib\-ria}
\newcommand{\reqv}{rela\-ti\-ve equi\-lib\-rium}
%   \newcommand{\reqv}{equi\-vari\-ant equilibrium}
%   \newcommand{\reqv}{travelling wave}
\newcommand{\Reqv}{Rela\-ti\-ve equi\-lib\-rium}
%   \newcommand{\Reqv}{Equi\-variant equi\-librium}
%   \newcommand{\Reqv}{travelling wave}
\newcommand{\reqva}{rela\-ti\-ve equi\-lib\-ria}
%   \newcommand{\reqva}{equivariant equilibria}
\newcommand{\Reqva}{Rela\-ti\-ve equi\-lib\-ria}
%   \newcommand{\Reqva}{Equivariant equilibria}
\newcommand{\equilibrium}{equi\-lib\-rium}
\newcommand{\equilibria}{equi\-lib\-ria}
\newcommand{\Equilibria}{Equi\-lib\-ria}
% \newcommand{\equilibrium}{steady state}
% \newcommand{\equilibria}{steady states}
% \newcommand{\Equilibria}{Steady states}
\newcommand{\Hec}{Hetero\-clinic connect\-ion}
\newcommand{\hec}{hetero\-clinic connect\-ion}
\newcommand{\HeC}{Hetero\-clinic Connect\-ion}

%%%%%%%%%%%%%%% SECTIONS, SLICES %%%%%%%%%%%%%%%%%%%%%%%%%%%%%%%%%

\newcommand{\Poincare}{Poincar\'e }
\newcommand{\PoincSec}{Poincar\'e section}
% \newcommand{\reducedsp}{orbit space}
% \newcommand{\Reducedsp}{Orbit space}
\newcommand{\reducedsp}{reduced state space}
\newcommand{\Reducedsp}{Reduced state space}
\newcommand{\fixedsp}{fixed-point subspace}
\newcommand{\Fixedsp}{Fixed-point subspace}
\newcommand{\csection}{cross-section} % eventually eliminate
\newcommand{\Csection}{Cross-section} % eventually eliminate
\newcommand{\slice}{slice}
\newcommand{\Slice}{Slice}
\newcommand{\mslices}{method of slices}
\newcommand{\Mslices}{Method of slices}
\newcommand{\mframes}{method of moving frames}
\newcommand{\Mframes}{Method of moving frames}
\newcommand{\chartBord}{chart border}
\newcommand{\ChartBord}{Chart border}
\newcommand{\poincBord}{section border}
\newcommand{\PoincBord}{Section border}
% \newcommand{\poincBord}{\PoincSec\ border}
% \newcommand{\PoincBord}{\PoincSec\ border}
% \newcommand{\poincBord}{border of transversality}
\newcommand{\template}{template} % {slice-fixing point} % {reference state}
\newcommand{\sliceBord}{slice border}
\newcommand{\SliceBord}{Slice border}
\newcommand{\Sset}{Inflection hyperplane}
\newcommand{\sset}{inflection hyperplane} 	% {singularity hyperplane}
											% {singular set}
\newcommand{\pSRed}{\ensuremath{\hat{\cal M}}} % reduced state space Jan 2012
%\newcommand{\pSRed}{\ensuremath{\bar{\cal M}}} % reduced state space
\newcommand{\sspRed}{\ensuremath{\hat{\ssp}}}    % reduced state space point Jan 2012
% \newcommand{\sspRed}{\ensuremath{y}}    % reduced state space point, experiment
% \newcommand{\sspRed}{\ensuremath{\bar{x}}}    % reduced state space point
\newcommand{\velRed}{\ensuremath{\hat{\vel}}}    % ES reduced state space velocity Jan 2012
% \newcommand{\velRed}{\ensuremath{\bar{v}}}    % PC reduced state space velocity
% \newcommand{\velRed}{\ensuremath{u}}    % ES reduced state space velocity

\newcommand{\slicep}{{\ensuremath{\sspRed'}}}   % slice-fixing point Jan 2012
% \newcommand{\slicep}{{\ensuremath{y'}}}   % slice-fixing point, experimental
% \newcommand{\slicep}{\ensuremath{\ssp'}}   % slice-fixing point
%\newcommand{\sliceTan}[1]{\ensuremath{t_{#1}(y')}}    % tangent at slice-fixing, experimental
\newcommand{\sliceTan}[1]{\ensuremath{t'_{#1}}}    % group orbit tangent at slice-fixing
\newcommand{\groupTan}{\ensuremath{t}}    % group orbit tangent
%\newcommand{\Group}{\ensuremath{\Gamma}}    % Siminos Lie group
\newcommand{\Group}{\ensuremath{G}}         % Predrag Lie or discrete group
%\newcommand{\Lg}{\mathfrak{a}}             % Siminos Lie algebra generator
\newcommand{\Lg}{\ensuremath{\mathbf{T}}}   % Predrag Lie algebra generator
%\newcommand{\LieEl}{\ensuremath{\mathbb{G}}}  % Wiczek project Lie group element
\newcommand{\LieEl}{\ensuremath{g}}  % Predrag Lie group element

\newcommand{\zeit}{\ensuremath{t}}  %time variable Ashley
\newcommand{\sspSing}{\ensuremath{\ssp^\ast}} 	% inflection point
\newcommand{\sspRSing}{\ensuremath{\sspRed^\ast}} 	% inflection point, reduced space

%%%%%%%%%%%%%%% LIE GROUP PARAMETRIZATIONS %%%%%%%%%%%%%%%%%%%%%%
\newcommand{\gSpace}{\ensuremath{{\bf \phi}}}   % MA group rotation parameters
% \newcommand{\gSpace}{\ensuremath{{\bf \theta}}}   % PC group rotation parameters
\newcommand{\velRel}{\ensuremath{c}}    % relative state or phase velocity
\newcommand{\angVel}{angular velocity}      % Froehlich
\newcommand{\angVels}{angular velocities}   % Froehlich
\newcommand{\phaseVel}{phase velocity}      % pipe slicing
\newcommand{\phaseVels}{phase velocities}   % pipe slicing
\newcommand{\PhaseVel}{Phase velocity}      % pipe slicing
\newcommand{\PhaseVels}{Phase velocities}   % pipe slicing

%%%%%%%% Siminos macros %%%%%%%%%%%%%%%%%%%%%%%%%%%%%%
\newcommand{\Rls}[1]{\ensuremath{\mathbb{R}^{#1}}}
%\newcommand{\Idg}{\ensuremath{\mathbf{1}}}
%\newcommand{\Clx}[1]{\ensuremath{\mathbb{C}^{#1}}}
%\newcommand{\conj}[1]{\ensuremath{\bar{#1}}}
%\newcommand{\trace}{\mbox{\rm trace}\,}
%\newcommand{\On}[1]{\ensuremath{\mathbf{O}(#1)}}
\newcommand{\Un}[1]{\ensuremath{\textrm{U}(#1)}}         % in DasBuch
\newcommand{\On}[1]{\ensuremath{\textrm{O}(#1)}}
%\newcommand{\SOn}[1]{\ensuremath{\mathbf{SO}(#1)}} % in Siminos thesis
\newcommand{\SOn}[1]{\ensuremath{\textrm{SO}(#1)}}         % in DasBuch
\newcommand{\SUn}[1]{\ensuremath{\textrm{SU}(#1)}}         % in DasBuch
\newcommand{\Spn}[1]{\ensuremath{\textrm{Sp}(#1)}}         % in DasBuch
%\newcommand{\Dn}[1]{\ensuremath{\mathbf{D}_{#1}}    % in Siminos thesis
\newcommand{\Dn}[1]{\ensuremath{\textrm{D}_{#1}}}              % in DasBuch
%\newcommand{\Zn}[1]{\ensuremath{\mathbf{Z}_{#1}}}    % in Siminos thesis
\newcommand{\Zn}[1]{\ensuremath{\textrm{C}_{#1}}}              % in DasBuch
%\newcommand{\Ztwo}{\ensuremath{\mathbf{Z}_2}}      % in Siminos thesis
\newcommand{\Ztwo}{\ensuremath{\textrm{C}_2}}                % in DasBuch
%\newcommand{\Refl}{\ensuremath{\kappa}}            % Siminos uses R for rotations.
\newcommand{\Refl}{\ensuremath{\sigma}}             % in DasBuch
%\newcommand{\Shift}{\ensuremath{\tau}}
\newcommand{\Rot}[1]{\ensuremath{C^{#1}}}           % in DasBuch, e.g. C^{1/3}
%\newcommand{\Rot}[1]{\ensuremath{R(#1)}}           % Siminos uses R for rotations.
%\newcommand{\Drot}{\ensuremath{\zeta}}
%\newcommand{\Lg}{\mathcal{G}}
%\newcommand{\stab}[1]{\ensuremath{\Sigma_{#1}}}
\newcommand{\stab}[1]{\ensuremath{G_{#1}}}
\newcommand{\shift}{\ensuremath{d}}
\newcommand{\Fix}[1]{\ensuremath{\mathrm{Fix}\left(#1\right)}}

%%%%%%%%%%%% loopDef.tex, defCrete.tex specific %%%%%%%%%%%%%
% Predrag   defCrete.tex             4mar2003
% Predrag   loopDefs.tex            10jul2003
\newcommand{\descent}{Newton descent}
\newcommand{\Descent}{Newton Descent}
\newcommand{\CostFct}{Cost function}    % functional to minimize
\newcommand{\costFct}{cost function}    % functional to minimize
\newcommand{\costF}{F^2}        % cost function,
\newcommand{\Loop}{L}
\newcommand{\pVeloc}{v}         % phase-space velocity
\newcommand{\lSpace}{\tilde{x}}     % a point on a loop
\newcommand{\lVeloc}{\tilde{v}}     % loop tangent
\newcommand{\damp}{\Delta\tau}      % descrete fictitous time step
% \newcommand{\pSpaceDer}[1]{x^{(#1)}}
% \newcommand{\lSpaceDer}[1]{\tilde{x}^{(#1)}}

%%%%%%%%%%%%%% ks.tex specific %%%%%%%%%%%%%%%%%%%%%%%%%%%%
\newcommand{\KS}{Kuramoto-Siva\-shin\-sky}
\newcommand{\KSe}{Kuramoto-Siva\-shin\-sky equation}
\newcommand{\pCf}{plane Couette flow}
\newcommand{\PCf}{Plane Couette flow}
\newcommand{\dmn}{-dimensional}  %  experimental 220ct2009
%\newcommand{\dmn}{\ensuremath{d}}  %  n-dimensional
%\newcommand{\dmn}{\ensuremath{\!-\!d}}  %  n-dimensional
\newcommand{\expctE}{\ensuremath{E}}    % E space averaged
\newcommand{\tildeL}{\ensuremath{\tilde{L}}}
\newcommand{\EQV}[1]{\ensuremath{EQ_{#1}}} %experimental
% \newcommand{\EQV}[1]{\ensuremath{q_{#1}}} %ChaosBook
% \newcommand{\EQV}[1]{\ensuremath{E_{#1}}} %Ruslan
% E_0: u = 0 - trivial equilibrium
% E_1,E_2,E_3, for 1,2,3-wave equilibria
\newcommand{\REQV}[2]{\ensuremath{TW_{#1#2}}} % #1 is + or -
% TW_1^{+,-} for 1-wave traveling waves (positive and negative velocity).
\newcommand{\PO}[1]{\ensuremath{PO_{#1}}}
% PO_{period to 2-4 significant digits} - periodic orbits
\newcommand{\RPO}[1]{\ensuremath{RPO_{#1}}}
% RPO_{period to 2-4 significant digits} - relative PO.  We use ^{+,-}
% to distinguish between members of a reflection-symmetric pair.
% Gibson likes:
\newcommand{\tEQ}{\ensuremath{{EQ}}}

%%%%%%%%%%%%%%%%%%%%%%%%%%%%%%%%%%%%%%%%%%%%%%%%%%%%
