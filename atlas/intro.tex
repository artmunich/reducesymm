% siminos/atlas/intro.tex  pdflatex atlas
% $Author$ $Date$

\begin{quotation}
    \ifdraft\color{blue}
The ``lead paragraph'' is formatted as a single paragraph before the first
section heading. Numbered references are allowed.
        \PC{
    The first paragraph of the article should be a Lead Paragraph and
    will be highlighted in the journal in boldface type. This paragraph,
    which essentially advertises the main points of the article, must
    describe in terms accessible to the nonspecialist reader the context
    and significance of the research problem studied and the importance
    of the results. The Editors will pay special attention to the clarity
    and accessibility of this paragraph, and in many cases may rewrite it
        }
    \color{black}\fi
\end{quotation}

    \PC{2012-01-03 experiment with
    \ensuremath{\hat{\ssp}}, \ensuremath{\bar{\ssp}} or \ensuremath{\tilde{\ssp}}
    as the \reducedsp\ coordinate.
    }

\section{Introduction}
\label{s:intro}
% former siminos/atlas/intro.tex

    \ifdraft\color{blue}
    \PC{
{2012-03-12} A putative outline of the paper is in
\refsect{chap:outline}.
    }
Goal: chart the \statesp\ explored by chaotic dynamics,
a curved manifold embedded in a high-dimensional \statesp.

Key notion: recurrence.
To quantify `near' need the notion of distance, or 'norm'.

Problem: evolution in time decomposes \statesp\ into spaghetti of time
orbits or trajectories. Symmetries stratify it into layers of an onion.
Need to pick a single point for each trajectory (section it) and each group orbit
(slice it).

(template)

Cover the curved manifold by the shortest-distance sections (for time
recurrence) and \slice s (for continuous shifts). In the limit of longer
and longer cycles this leads to the usual curved manifold geometry,
measured locally by Euclidean distances.
    \color{black}\fi


    \DB{2012-03-28}{
    at this point the writing abruptly jumps from goody-goody intro for
    the everyman to technical without sufficient warning. Need to segway
    into the group theory stuff
    }
This problem is here resolved by the {\mslices}\rf{rowley_reconstruction_2000,BeTh04,SiCvi10,FrCv11}, in
which the group orbit of any full-flow structure is represented by a
single point (see \reffig{fig:BeThTraj}), the group orbit's intersection
with a fixed hypersurface, or the \emph{`\slice'}. A
\slice\ fixes only the symmetry group phases: a continuous time full space
orbit is reduced to a continuous time orbit in the symmetry-\reducedsp,
as in \reffig{f:MeanVelocityFrame}.

Our goals here are two-fold:
(i) First, we review the method of Poincar\'e sections,
    in order to motivate the need for for symmetry reduction, explain
    what it is, and how with it the geometry of \statesp\ dynamics is revealed;
(ii) Next, we demonstrate that this tool enables us to commence a systematic
exploration of the hierarchy of dynamically important invariant solutions
of flows with continuous symmetries. The $\infty$-dimensional \stateDsp\
representation\rf{GHCW07} of PDEs, such as \reffig{f:MeanVelocityFrame},
enables us to track the unstable manifolds of invariant
solutions, the heteroclinic connections between them\rf{GHCV08}, and
{provides us with} new insights into the nonlinear \statesp\ geometry and
dynamics of moderate \Reynolds\ wall-bounded flows.

We review  ??? flows, their visualization, and their symmetries in
\refsect{s:review}. The {\mslices}  is described in \refsect{s:slice},
and the computation of invariant solutions and their stability
eigenvalues and eigenvectors in \refsects {sect:TimeOrb}{s:algorithm}.
The main advances reported in this paper are the symmetry \reducedsp\
visualization and [...], (\refsect{s:rpos}). Outstanding challenges are
discussed in \refsect{s:concl}.
