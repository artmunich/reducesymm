% siminos/atlas/symm.tex  pdflatex atlas
% $Author$ $Date$


\subsection{Symmetry-induced coordinate frames}
\label{s:symm}

So far we have not offered any advice as to the choice of basis vectors
in constructing \statesp\ coordinates \refeq{intrSspTraj}. We now show
that at least locally, presence of a continuous symmetry suggests two
natural mutually orthogonal basis vectors, the group action tangent and
curvature vectors.

Consider the one-parameter rotation group \SOn{2} acting on a smooth
periodic function $u(\gSpace + 2\pi) = u(\gSpace)$ defined on domain
$\gSpace \in [0,2\pi)$, expanded in the Fourier basis
\[ %beq
   u(\gSpace) = \sum \ssp_m \mathrm{e}^{\mathrm{i}m\gSpace},
\] %ee{FourierExp1}
where, as $u$ is real, $\ssp_m=\ssp_{-m}^*$. Parametrize the forward
translation by the continuous parameter $\phi$,
\(
    \LieEl(\phi)\,u(\gSpace) = u(\gSpace-\phi)
\,,
\)
or, in the Fourier basis,
\(
   \LieEl(\phi) \,\ssp = \mathrm{diag}\{ \mathrm{e}^{-\mathrm{i}m\phi} \} \,\ssp
\,.
\)
The tangent to the group orbit at the point $\ssp$ is then given by
the first derivative with respect to the group parameter,
The tangent to the group orbit at the point $\ssp$ is then given by
the first derivative with respect to the group parameter,
and the direction of curvature by the second derivative,
\bea
   {\bf t}(\ssp) &=&
   \lim_{\gSpace\to 0}
   \left(\LieEl(\gSpace)\,\ssp - \ssp\right)/\gSpace
   = \mathrm{diag}\{ -\mathrm{i}m \} \, \ssp = \Lg \ssp,
\label{eq:tang}\\
   \kappa(\ssp)\, {\bf n}(\ssp) &=& \Lg^2 \ssp  = - \mathrm{diag}\{m^2\} \, \ssp
   \,,
\label{eq:curv}
\eea
where $\normVec$ is a unit vector normal to the tangent and
$1/\kappa$ is the radius of curvature. The pair of unit vectors
    \PC{2011-10-28
    ``As $\Norm{\LieEl(\gSpace)\slicep}$ is a constant, for the group tangent
    vector $\Lg_\gSpace \slicep$ evaluated at $\slicep$ \refeq{eq:tang}
    %GroupTangField} is normal to $\slicep$, and the term
    $\braket{\slicep}{\Lg_\theta\,\slicep}$ vanishes ($\Lg_{\theta}$ is
    antisymmetric).''
The state vector $\ssp$ is not normal to \normVec(\ssp), as $\braket{\ssp
\Lg^2}{\ssp} = - \Norm{\groupTan(\ssp)}^2 \neq 0$, but can one use it to
produce from $\ssp$ the 3. local eigenbasis unit vector? Have not thought
that through. If we do that here, need to rewrite text leading to
\refeq{PCsectQ0}.
    }
\beq
\{{\be_n},{\be_{n+1}}\} =
\{\groupTan(\ssp)/\Norm{\groupTan(\ssp)},\normVec(\ssp)\}
\ee{FrenetFrame}
forms a local orthogonal Frenet-Serret frame at $\ssp$, and can be useful
in constructing the \statesp\ basis vector set \refeq{intrSspTraj}. For
example, in \reffig{f:MeanVelocityFrame} the \rpo\ $\RPO{36.92}$ is
projected onto the $3$\dmn\ orthogonal frame
\beq
\{{\be}_1,{\be}_2,{\be}_3\}
=
\left\{{\groupTan(\sspRed_0)}/{\Norm{\groupTan(\sspRed_0)}},
\,\normVec(\sspRed_0),\,
(\sspRed_d-\sspRed_0)_{GS} \right\}
\ee{FrenetFrame1}
where $\sspRed_0=\sspRed(0)$ is a point on the \rpo, $\sspRed_d$ is the
most distant point  from $\sspRed_0$ along its symmetry-\reducedsp\ orbit
$\sspRed(\zeit)$, measured in the energy norm \refeq{innerproduct}, and
$(\sspRed_d-\sspRed_0)_{GS}$ is their separation vector, Gram-Schmidt
orthogonalized to $\{\tilde{\be}_1,\tilde{\be}_2\}$.

We note also for later use that, for a time-dependent group parameter
$\gSpace$, the phase speed $\dot{\gSpace}$ along the group tangent
evaluated at the \statesp\ point $\ssp$ (the `Cartan derivative') is
given by
\beq
\LieEl^{-1}\dot{\LieEl} \,\ssp % =e^{-\gSpace \cdot \Lg} \,
     =\mathrm{e}^{-\gSpace \Lg} \,
\left(\frac{\mathrm{d} ~~}{\mathrm{d} \, \zeit} \, % e^{\gSpace \cdot \Lg}\ssp
                             \mathrm{e}^{\gSpace \Lg}\right)\ssp
    =\dot{\gSpace} \, \groupTan(\ssp)
%    =\dot{\gSpace}\cdot \groupTan(\ssp)
\,.
\ee{CartanDer}

\subsection{Relative invariant solutions}
\label{s:RelInvSol}

Along with continuous symmetries come important classes of invariant
solutions referred to as `relative' or `equivariant'
\rf{Huyg1673,Poinc1896}. For pipe flows one expects to find relative
equilibria and \rpo s\rf{Rand82}, associated with the translational
and rotational symmetries of the flow.

{\em \Reqv} (labeled here $\REQV{}{}$) is a dynamical
orbit whose velocity field \refeq{symbolicNS} lies within the group
tangent space, with a constant phase speed $c$,
% $c=(c_1,\cdots,c_N)$,
and whose time evolution is thus confined to the group orbit,
    \PC{recheck, I might have it normalized wrong; I think it is OK,
        $c$ is angular speed, so $c\,\ssp$ is $\vel$}
\bea
\vel(\ssp) &=& c \, \groupTan(\ssp) % c \cdot \groupTan(\ssp)
\label{phaseVel}\\
\ssp(\zeit) &=& \LieEl(\gSpace(\zeit)) \, \ssp(0)
%          = \mathrm{e}^{ \zeit  c \Lg} \,  \ssp(0)
%           = \mathrm{e}^{ \zeit\,  c \cdot \Lg} \,  \ssp(0)
\,,\qquad
\ssp(\zeit) \in \pS_{\REQV{}{}}
\nnu
\,.
\eea
While in the case of \SOn{2} symmetry a \reqv\ traces out a loop in the
full \statesp, for a higher-dimensional continuous symmetry it explores
the group orbit $\pS_{\REQV{}{}}$ quasi-periodically, so \reqv\ is
\emph{not} a \po. Rather, as all states in this group orbit are
physically the same state, this is a generalized \eqv\ state. Relative
equilibria can propagate in the stream-wise direction $z$ (travelling
waves), in azimuthal $\theta$ direction (rotational waves), or both. As
noted in \refsect{s:intro}, for pipe flows many stream-wise
\reqva\ have been discovered, satisfying \refeq{phaseVel}
\beq
   \LieEl(0,-ct)\,\vec{u}(\zeit) - \vec{u}(0) = \vec{0}
\,,
\ee{pipeAxTW}
where $c$ is the stream-wise phase speed.

A {\rpo} $p$ is an orbit in {\statesp} $\pS$ which exactly recurs
\[ %beq
\ssp(\zeit) = \LieEl_p \, \ssp(\zeit + \period{p} )
    \,,\qquad
\ssp(\zeit) \in \pS_p
\] %ee{RPOrelper1}
after a fixed {relative period} $\period{}$, but shifted by a fixed group
action ${\LieEl}$ that maps the endpoint $\ssp (\period{} ) $ back
into the initial point cycle point $\ssp (0) $.
In the pipe flow cylindrical coordinate system
a \rpo\ is a time-dependent velocity field
\beq
\bu_p(r,\theta,z,t)=\bu_p(r,\theta+\phi_p,z+ \shift_p,t+\period{p})
\ee{RPOpipe}
that recurs after time $\period{p}$, rotated and shifted by $\phi_p$ and
$\shift_p$.
In our Newton search for a \rpo\ $p$, we seek the zeros of
\beq
   \vec{f}(\vec{u}(0),\period{},\shift)
\,=\, \LieEl(0,-\shift)\,\vec{u}(\period{}) - \vec{u}(0) \,=\, \vec{0}
\,,
\ee{eq:fRPO}
for some $\vec{u}_p$ with period $\period{p}$ and shift $\shift_p$.

Continuous symmetry parameters (`phases' or `shifts')
$\{\gSpace_j\}=\{\phi_p,\shift_p\}$
% $\{\gSpace_1,\cdots,\gSpace_N\}$
are real numbers, ratios $\pi/\gSpace_j$ are almost never rational, and
\rpo s are almost never eventually periodic; for pipe the time evolution
of a relative periodic point thus sweeps out quasi-periodically the
$3$\dmn\ group orbit $\pS_p$ without ever closing into a \po, unless the
dynamics is restricted to a discrete-symmetry invariant subspace.
