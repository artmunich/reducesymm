%\Problems{exerContinuous}{10oct2009}
% $Author$ $Date$

\exercise{Visualizations of the 5-dimensional \cLf:}{
\label{exer:PlotCLf}
%from  wilczak/blog/exerFlow.tex 	Predrag 2009-08-29
Plot \cLf\ projected on any three of the five $\{x_1, x_2,
y_1, y_2, z\}$ axes. Experiment with different
visualizations.
    %from  wilczak/blog/exerFlow.tex 	Predrag 2009-08-29
    } % end \exercise{Visualizations of the 5-dimensional \cLf

\exercise{An $\SOn{2}$-equivariant flow with two Fourier modes:}{
    \label{exer:2mR}
    \index{SO(2)@\SOn{2}}
\CLe\ \refeq{eq:CLe} of Gibbon and
Mc\-Guin\-ness\rf{GibMcCLE82} have a degenerate
4-dimensional subspace, with \SOn{2} acting only in its
lowest non-trivial representation. Here is a possible
model, still 5-dimensional, but with \SOn{2} acting in
the two lowest representations. Such models arise as
truncations of Fourier-basis representations of PDEs on
periodic domains. In the complex form, the simplest such
modification of \cLe\ may be the ``2-mode'' system
\bea
 \dot{x} &=&-\sigma x+ \sigma x^\ast y \continue
 \dot{y} &=&(r-z)x^2-a y \continue
 \dot{z} &=& \frac{1}{2}\left(x^2 y^\ast+x^{\ast 2} y\right)-b z
 \,,
\label{eq:2me}
\eea
where $x,y$, $r=r_1+ i\,r_2$, $a=1+i\,e$ are complex and $z$,
$b$, $\sigma$ are real. Rewritten in terms of real variables
$x=x_1+ i\, x_2\,,\ y=y_1+i\, y_2$ this is a 5-dimensional
first order ODE system
    \PC{complete the rewrite here}
\bea
	\dot{x}_1 &=& -\sigma x_1 + \sigma y_1\continue
	\dot{x}_2 &=& -\sigma x_2 + \sigma y_2\continue
	\dot{y}_1 &=& (\rho_1-z) x_1^2 - r_2 x_2 -y_1-e y_2 \continue
	\dot{y}_2 &=& \continue
	\dot{z} &=& -b z + x_1 y_1 + x_2 y_2\,.
	\label{eq:2meR}
\eea
Verify \refeq{eq:2meR} by substituting $x=x_1+ i\, x_2\,,\
y=y_1+i\, y_2$, $r=r_1+ i\,r_2$, $a=1+i\,e$ into the complex
2-mode equations \refeq{eq:2me}.
% Wilczek blog/open.tex has a series of related exercises
%                       none of them tried
% \authorPC %Jul 9 2009
    }

\exercise{\SOn{2} rotations in a plane:}{ \label{exer:FinRot2d}
\index{SO(2)@\SOn{2}}
Show by exponentiation \refeq{FiniteRot} that the \SOn{2} Lie algebra
element $\Lg$ generates rotation $\LieEl$ in a plane,
\bea
\LieEl(\theta) &=& e^{\Lg \theta}
 = \cos\theta
   \left(\barr{cc}
    1  &  0   \\
    0  &  1
    \earr\right)
 + \sin\theta
   \left(\barr{cc}
    0  &  1   \\
   -1  &  0
    \earr\right)
                \continue
 &=&   \left(\barr{cc}
    \cos\theta  &  \sin\theta   \\
   -\sin\theta  &  \cos\theta
    \earr\right)
 \,.
\label{SO2gener2d}
\eea
    %from  wilczak/blog/exerFlow.tex 	Predrag 2009-08-29
    } %end \exercise{\SOn{2} rotations in a plane}{

\exercise{Rotational equivariance, infinitesimal angles.}
   { \label{exer:InfinRotInvari}
Show that \cLe\ are equivariant under infinitesimal \SOn{2} rotations.
    %from  wilczak/blog/exerFlow.tex 	Predrag 2009-08-29
    }

\exercise{\CLe\ in a Hilbert basis.}{\label{exer:CLeHilbertBas}
\index{invariant!polynomial basis}
(continuation of \refexam{exam:SO2HilbertBas})~~
Derive \cLe\ in terms of invariant polynomials \refeq{eq:CLEip}, plot the
strange attractor in projections you find illuminating (one example is
\reffig{fig:CLEip}).
    } %end \exercise{\CLe\ in a Hilbert basis.}



%    \ProblemsEnd
