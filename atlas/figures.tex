% siminos/atlas/figures.tex    master file: atlas.tex
% $Author: predrag $ $Date: 2012-04-06 17:44:20 -0400 (Fri, 06 Apr 2012) $

\section{Figures}
\label{s:figures}

\subsection{Figures: ToDo}
\label{s:figsToDo}

The most important: good (and not large) figures illustrating the key
steps. Save the source files in siminos/figsSrc/...

Check {\bf [~]} once finalized
\begin{description}

\item[{[x]}] Some pointers to figure drawing are in
             {siminos/figSrc/00ReadMe.txt}
\item[{[ ]}] all figures: apply \emph{tight crop}, leave no white acreage around the figure
\item[{[ ]}] all figures: make section and slice planes rectangular, no funny corner clips
\item[{[ ]}] all figures: make section and slice planes semi-transparent
\item[{[ ]}] [2012-04-11 Predrag] experiment with tasteful grays for hyperplanes,
    something like \reffig{fig:LorenzSect} (not sure about this one...)
\item[{[ ]}] [2012-04-11 Predrag] prefix names to all Roess figures by Roess*.*
\item[{[ ]}] [2012-04-11 Predrag] prefix names to all \cLe\ figures by CLE*.*

\item[{[ ]}] all R\"ossler figures: do not squish them, make them proud
        and tall as in \reffig{f:RosslSect}; use the same absolute scale
        in $(x,y,z)$ directions
\item[{[ ]}] all R\"ossler figures: the same perspective, scale. Indicate origin, $(x,y,z)$ axes.
\item[{[ ]}] \reffig{fig:RoessTrjs},
        R\"ossler \eqva\ and their invariant manifolds
    \begin{itemize}
        \item[{[ ]}] [2012-04-11 Predrag] apply \emph{tight crop}, leave no white acreage
        \item[{[ ]}] [2012-04-07 Predrag] should have the same axes
            (including their scale) as \reffig{fig:RoessBothEq}
        \item[{[ ]}] [2012-04-07 Predrag] Draw (un)stable eigenvectors as
                in \reffig{fig:AmLeAg06Im1}.
        \item[{[ ]}] [2012-04-07 Predrag] start the unstable spiral(s)
            for the outer \eqv\ \emph{on the complex eigenvectors plane}
            so it spirals out, sufficiently far out to get sense that it
            traces out the basin boundary. Currently it is off the stable
            manifold, so it is drifting along the unstable eigenvector
    \end{itemize}

\item[{[ ]}]
    illustrate \poincBord s, and a two-chart atlas for R\"ossler
    \begin{itemize}
        \item[{[ ]}] [2012-04-07 Predrag] should be small in size, and laid out like
            \reffig{f:RosslSect}, with sensible $z$-axis; no huge green,
            red acres much beyond the scale set by the attractor and
            outer $\ssp_{+}$.
        \item[{[ ]}] [2012-04-08 Predrag] layout something like \reffig{fig:LorenzSect}
        \item[{[ ]}] [2012-04-07 Predrag] all 3 should have the same axes
            (including their scale)
        \item[{[ ]}] [2012-04-07 Predrag] all 3 should optimally be viewed from the same
            perspective viewpoint.
        \item[{[ ]}] [2012-04-11 Predrag] the section rectangles should
            not be much bigger than the strange attractor + height set by
            outer  $\ssp_{+}$ \eqv.
        \item[{[ ]}] \reffig{fig:RoessNearEq} plot the ridge; omit the
            outer $\ssp_{+}$ transient
        \item[{[ ]}] \reffig{fig:RoessFarEq} Plot
            the ridge, both the $\ssp_{-}$ and the $\ssp_{+}$ transient
        \item[{[ ]}] \reffig{fig:RoessSctAtlas} RoessBothEq
        \item[{[ ]}] does it print clearly in black \&\ white?
    \end{itemize}

\item[{[ ]}] \reffig{fig:CLf01group}, \CLf: $\cycle{01}$ {\rpo} group orbit.
    \begin{itemize}
        \item[{[x]}]  [2012-04-06 Predrag] to Daniel:
            Put the generating program into siminos/figSrc/matlab/ , maybe
            somebody else will play with it.
        \item[{[x]}]  [2012-04-06 Predrag] extract a bitmap image of the
            group orbit, insert axes and labels in Inkscape.
        \item[{[ ]}] Okay replacing \reffig{fig:CLf01group} with new and improved \reffig{fig:CLEWurst}
		\end{itemize}

\end{description}

\subsection{Figures: Done}
\label{s:figsDone}

Check {\bf [~]} once finalized
\begin{description}

\item[{[x]}] moved \reffig{fig:AmLeAg06Im1} to flotsam,
\item[{[x]}] \reffig{fig:RoessSct1} to flotsam, replaced by \reffig{fig:RoessNearEq}
\item[{[x]}] \reffig{fig:RoessSct2} to flotsam, replaced by \reffig{fig:RoessFarEq}
\item[{[x]}] \reffig{fig:RoessSctAtlas} to flotsam, replaced by \reffig{fig:RoessBothEq}

\item[{[x]}] [2012-03-28 Predrag]
        \reffig{fig:robbins3-7} by Bryce robbins3-7.nb, now superseded by
        \reffig{fig:RoessSctAtlas}.

\item[{[x]}] [2012-04-08 Predrag] gave up on \reffig{fig:A29PoincBad},
    good and bad templates / sections
    \begin{itemize}
        \item[{[ ]}] [2012-04-07 Predrag] \reffig{fig:A29PoincBad}\,({\it a})
        \item[{[ ]}] [2012-04-07 Predrag] \reffig{fig:A29PoincBad}\,({\it b})
            (including their scale) as
    \end{itemize}

\item[{[x]}] [2012-04-11 Predrag]  gave up on \reffig{fig:RoessRetMap}
    R\"ossler section, return map, moved it to flotsam
    \begin{itemize}
        \item[{[ ]}] [2012-04-07 Predrag] need {RoessSct1}.png
        \item[{[ ]}] [2012-04-07 Keith] {RoessRetMap},png
    \end{itemize}

\end{description}
\newpage

\subsection{Gang of Chaos article outline}
\label{chap:outline}
% Predrag: was siminos/blog/outline.tex         2012-03-28

\noindent
{\bf [2012-03-12 Predrag]}
\\
How to write this paper? Start writing freely - nothing is lost, earlier text can be recovered
from earlier versions. Try to remove unnecessary text - clip  it and paste it into flotsam.com.
I have clipped and pasted in far more text than what we will need.

I propose that we write the
paper in the same sequence as in the three chaos course lectures
the no professor with any self-respect would attend:

\begin{itemize}
  \item \statesp\ visualization
    \begin{itemize}
      \item experimentalist description: a video 1D to 3D arrays of pixels
      \item theorist description: $\infty$-\dmn\ \statesp -DB
      \item time orbit: point is a point, line is a line in all dimensions
      \item physical dimension: covariant Lyapunov vectors
      \item norms - distances between states
    \end{itemize}
  \item Poincar\'e sections, R\"ossler
    \begin{itemize}
      \item local chart
      \item R\"ossler {\poincBord} -Bryce
      \item R\"ossler two-chart atlas -Bryce
      \item R\"ossler unstable manifold curvilinear distance -Keith
      \item R\"ossler return map - Keith
      \item $N$-chart atlas, forward maps - Lei
    \end{itemize}
  \item what is a symmetry?
    \begin{itemize}
      \item pipes
      \item \cLe
    \end{itemize}
  \item group action
    \begin{itemize}
      \item group orbit
      \item finite
      \item eqs of motion
      \item infinitesimal, Jacobian derivative
      \item moving frame
    \end{itemize}
  \item symmetry reduction
    \begin{itemize}
      \item the goal, 3 ideas:
      \item Hilbert
      \item method of connections
      \item \mslices
    \end{itemize}
  \item what it is not
    \begin{itemize}
      \item co-moving frame
      \item an average over the group phase variables
      \item reduced-dimensionality model
    \end{itemize}
  \item \mslices, a local chart
    \begin{itemize}
      \item closest point on a group orbit
      \item variation $\to$ \slice\ hyperplane
    \end{itemize}
  \item \mslices, a global atlas
    \begin{itemize}
      \item flow in a \slice: \cLf\ example
      \item {\chartBord}: \cLf\ example
      \item 2-chart atlas, ridges:  \cLf\ example
      \item gauge fixing, geometric phase?
    \end{itemize}
  \item conclusions and summary
      \begin{itemize}
      \item theorist must work in \statesp, symmetry-reduce
      \item experimentalist must work in \statesp,
            symmetry-reduce the data
      \end{itemize}
\end{itemize}
