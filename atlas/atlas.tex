% siminos/atlas/atlas.tex    pdflatex atlas
%     then:    pdflatex atlas; bibtex atlas; pdflatex atlas; pdflatex atlas
%
% $Author$ $Date$
% http://www.cns.gatech.edu/~predrag/papers/Cvit12.pdf

%\documentclass{jfm}
\documentclass[twocolumn,
secnumarabic,
nofootinbib, tightenlines,
nobibnotes, showkeys, aps,
%preprint,%
%reprint,%
%author-year,%
%author-numerical,%
cha]{revtex4-1}

\newcommand{\version}{atlas ver. 0.5, Mar 20 2011}
% Predrag 					ver. 0.4, Mar 10 2011}
% Predrag 					ver. 0.3, Mar 10 2011}
% Predrag 					ver. 0.2, Dec 10 2011}
% Predrag 					ver. 0.1, Apr 28 2011}

        \input setupAtlas
        \input ../inputs/def
        \input defAtlas

\begin{document}

\title[How to cut and slice a symmetry]
{Have a symmetry? How to cut and slice it}

%PC, until 2012-03-19: "Continuous symmetries, and how to slice them"

\author{Predrag Cvitanovi{\'c}}
\email{predrag@gatech.edu.}
\author{Daniel Borrero-Echeverry}
\author{Keith Carroll}
\author{Bryce Robbins}
\author{Lei Zhang}
\affiliation{
 Center for Nonlinear Science and School of Physics,
 Georgia Institute of Technology,
 837 State Street, Atlanta, GA  30332, USA
}

\date{\today}
%\date{18 December 2011}
%\setcounter{page}{1}

\begin{abstract}
Symmetry reduction by the `method of slices'
quotients the continuous symmetries of chaotic flows. Within the
symmetry reduced state space, \reqva\ reduce to
equilibria, and relative periodic orbits reduce to periodic orbits.
Projections of these solutions and their unstable manifolds from their
high-dimensional symmetry reduced state space onto suitably chosen 2-
or 3-dimensional subspaces reveal their interrelations and the role they
play in organizing chaos.

%
Valid PACS numbers may be entered using the \verb+\pacs{#1}+ command.
\end{abstract}

\pacs{Valid PACS appear here}% PACS, the Physics and Astronomy
                             % Classification Scheme.
\keywords{Suggested keywords}%Use showkeys class option if keyword
                              %display desired
\maketitle

\input intro
\input cut
\input symm
\input slice
\input chart
\input concl

% \bibliographystyle{jfm}
% \nocite{*}
\bibliography{../bibtex/siminos}

\ifdraft
    \onecolumngrid

    \newpage
\input figures
    \newpage
\input flotsam
    \newpage
    \section{Daily blog, point by point}
    \label{chap:atlas}
\input ../blog/atlas
\fi

\end{document}
