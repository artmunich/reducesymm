% siminos/atlas/atlas.tex    pdflatex atlas
%     then:    pdflatex atlas; bibtex atlas; pdflatex atlas; pdflatex atlas
% for public version, toggle \draftfalse in setupAtlas.tex
%     (that removes all comments, the blog)

% $Author$ $Date$
% http://www.cns.gatech.edu/~predrag/papers/Cvit12.pdf

%\documentclass{jfm}
\documentclass[twocolumn,
secnumarabic,
nofootinbib, tightenlines,
nobibnotes, showkeys, aps, showpacs,
%preprint,%
%reprint,%
%author-year,%
%author-numerical,%
cha]{revtex4-1}

\newcommand{\version}{atlas ver. 0.5, Apr 11 2012}
% Predrag 					ver. 0.5, Apr 11 2012}
% Predrag 					ver. 0.4, Mar 10 2012}
% Predrag 					ver. 0.3, Mar 10 2012}
% Predrag 					ver. 0.2, Dec 10 2011}
% Predrag 					ver. 0.1, Apr 28 2011}

        \input setupAtlas
        \input ../inputs/def
        \input defAtlas

\begin{document}

\title[High-dimensional cartography]
{Cartography of high-dimensional flows: section \& slice visual guide}

%Predrag 2012-04-12: Continuous symmetry reduction of high-dimensional flows
%                    by the method of slices
%Predrag 2012-04-11: How to cut and slice a symmetry
%Predrag 2012-04-11: Have a symmetry? How to cut and slice it
%PC, until 2012-03-19: "Continuous symmetries, and how to slice them"

\author{Predrag Cvitanovi{\'c}}
\email{predrag@gatech.edu.}
\author{Daniel Borrero-Echeverry}
\author{Keith Carroll}
\author{Bryce Robbins}
\author{Lei Zhang}
\affiliation{
 Center for Nonlinear Science and School of Physics,
 Georgia Institute of Technology,
 837 State Street, Atlanta, GA  30332, USA
}

\date{\today}
%\date{18 December 2011}
%\setcounter{page}{1}

\begin{abstract}
Symmetry reduction by the method of slices quotients the continuous
symmetries of chaotic flows. Within the symmetry reduced state space,
\reqva\ reduce to equilibria, and relative periodic orbits reduce to
periodic orbits. Projections of these solutions and their unstable
manifolds from their high-dimensional symmetry reduced state space onto
suitably chosen 2- or 3-dimensional subspaces reveal their interrelations
and the role they play in organizing chaos.

\end{abstract}

\pacs{02.20.-a, 05.45.-a, 05.45.Jn, 47.27.ed, 47.52.+j, 83.60.Wc}
%DB  04/12/2012: Turned PACS them on.
%\pacs{Valid PACS appear here}% PACS, the Physics and Astronomy
                             % Classification Scheme.
%% copied from siminos/blog/strategy.tex
% \PACS 02.20.-a \sep 05.45.-a \sep 05.45.Jn \sep 47.27.ed \sep 47.52.+j
% 02.20.-a      Group theory, mathematics
% 05.45.-a      Nonlinear dynamics and chaos
% 05.45.Jn      High-dimensional chaos
% 47.27.ed      Dynamical systems approaches (turbulent flows)
% 47.52.+j      Chaos in fluid dynamics
% 83.60.Wc      Flow instabilities
% 95.10.Fh      Chaotic dynamics

\keywords{
symmetry reduction,
equivariant dynamics,
relative equilibria,
relative periodic orbits,
slices,
moving frames
}%Use showkeys class option if keyword
                              %display desired
\maketitle

\input intro
\input cut
\input symm
\input slice
\input chart
\input bridge
\input concl

% \bibliographystyle{jfm}
% \nocite{*}
\bibliography{../bibtex/siminos}

\ifdraft
    \onecolumngrid

    \newpage
\input figures
    \newpage
\input flotsam
    \newpage
    \section{Daily blog, point by point}
    \label{chap:atlas}
\input ../blog/atlas
\fi

\end{document}
