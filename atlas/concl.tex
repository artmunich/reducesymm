% siminos/atlas/concl.tex  pdflatex atlas
% $Author$ $Date$

\section{Conclusion and perspectives}
\label{s:concl}
% former siminos/atlas/concl.tex


As a turbulent flow evolves, every so often we catch a glimpse of a
familiar structure. For any finite spatial resolution, the flow
follows for a finite time an unstable {\cohStr} belonging to an
alphabet of admissible states, represented here by a set of \reqva\
and \rpo s.
In presence of symmetries, these recurrences can be shifted both in time
(period \period{p}) and space (phase shift $\shift_p$).
The role of invariant solutions is
to partition the $\infty$-dimensional \statesp\ into a finite set of
neighborhoods visited by a typical long-time turbulent fluid state.

The main message of this paper is that if a problem has a continuous
symmetry, the symmetry \emph{must} be used to simplify it. Ignore it at
your own peril; the invariant solutions found by restricting
searches to the discrete-symmetry invariant subspaces have little if
anything to do with the full \statesp\ explored by turbulence, not more
than the \eqv\ points of the Lorenz equation have to do with its strange
attractor.
Symmetry reduction
is here achieved, and now all pipe flow solutions can be plotted
together, as one happy family: all points equivalent by symmetries are
represented by a single point, families of solutions are mapped to a
single solution, \reqva\ become \eqva, and \rpo s become \po s.
    %
    \PC{2012-03-14 this is a repeat of an earlier sentence}
    \PC{2011-10-18 incorporate this paragraph:``
Note also that the rotation of a fluid flow into a \slice\ {\em is not}
an average over the 3D pipe azimuthal angle, it is the full snapshot of
the flow embedded in the $\infty$-dimensional \statesp. Symmetry
reduction is not a dimensional-reduction scheme, or flow modeling by
fewer degrees of freedom: the \reducedsp\ is also $\infty$-dimensional
and no information is lost, one can go freely between solutions in the
full and reduced \statesp s by integrating the associated
\emph{reconstruction equations}.
''}

\begin{acknowledgments}
We are indebted to
% M.~Avila,
% D.~Barkley,
% R.~L.~Davidchack,
S.~Froehlich,
E.~Siminos,
S.A.~Solla,
% L.~S.~Tuckerman,
% and
% A.~P.~Willis
% D.~Viswanath
and
R.~Wilczak,
for inspiring discussions.
P.~C.\ thanks G.~Robinson,~Jr.\ for support, and
Max-Planck-Institut f\"ur Dynamik und Selbstorganisation,
G\"ottingen for hospitality.
P.~C.\ was partly supported by NSF grant DMS-0807574
and
2009 Forschungspreis der Alexander von Humboldt-Stiftung.
\end{acknowledgments}
