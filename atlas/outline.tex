% siminos/blog/outline.tex
% $Author$ $Date$

%    \section{Outline}
%   \label{chap:outline}

\noindent
{\bf [2012-03-12 Predrag]}
\\
How to write this paper? Start writing freely - nothing is lost, earlier text can be recovered
from earlier versions. Try to remove unnecessary text - clip  it and past it into flotsam.com.
I have clipped and pasted more text in than what we will need.

The most important: good (and not large) figures illustrating the key steps. Save the source files
in FigSrc/...

I propose that we write the
paper in the same sequence as in the three chaos course lectures
the no professor with any self-respect would attend:

\begin{itemize}
  \item \statesp\ visualization
    \begin{itemize}
      \item \cLe
      \item Ring of Fire
      \item experimentalist description: a video 1D to 3D arrays of pixels
      \item theorist description: $\infty$-\dmn\ \statesp
      \item time orbit: point is a point, line is a line in all dimensions
      \item physical dimension: covariant Lyapunov vectors
      \item norms - distances between states
    \end{itemize}
  \item Poincar\'e sections, R\"ossler
    \begin{itemize}
      \item local chart
      \item R\"ossler {\poincBord}
      \item R\"ossler two-chart atlas
      \item R\"ossler unstable manifold curvilinear distance
      \item R\"ossler return map
      \item $N$-chart atlas, forward maps
      \item Ring of Fire return map\rf{lanCvit07}
    \end{itemize}
  \item what is a symmetry?
    \begin{itemize}
      \item pipes
      \item Ring of Fire
      \item \cLe
    \end{itemize}
  \item group action
    \begin{itemize}
      \item group orbit
      \item finite
      \item eqs of motion
      \item infinitesimal, Jacobian derivative
      \item moving frame
    \end{itemize}
  \item symmetry reduction
    \begin{itemize}
      \item the goal, 3 ideas:
      \item Hilbert
      \item method of connections
      \item \mslices
    \end{itemize}
  \item what it is not
    \begin{itemize}
      \item comoving frame
      \item an average over the group phase variables
      \item reduced-dimensionality model
    \end{itemize}
  \item \mslices, a local chart
    \begin{itemize}
      \item closest point on a group orbit
      \item variation $\to$ \slice\ hyperplane
    \end{itemize}
  \item \mslices, a global atlas
    \begin{itemize}
      \item flow in a \slice: \cLf\ example
      \item {\chartBord}: \cLf\ example
      \item 2-chart atlas, ridges:  \cLf\ example
      \item gauge fixing, geometric phase?
    \end{itemize}
  \item conclusions and summary
      \begin{itemize}
      \item theorist must work in \statesp, symmetry-reduce
      \item experimentalist must work in \statesp,
            symmetry-reduce the data
      \end{itemize}
\end{itemize}
