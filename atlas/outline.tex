% siminos/blog/outline.tex
% $Author$ $Date$

%    \section{Outline}
%   \label{chap:outline}

\noindent
{\bf [2012-03-12 Predrag]}
\\
How to write this paper? Start writing freely - nothing is lost, earlier text can be recovered
from earlier versions. Try to remove unnecessary text - clip  it and past it into flotsam.com.
I have clipped and pasted more text in than what we will need.

The most important: good (and not large) figures illustrating the key steps. Save the source files
in FigSrc/...

I propose that we write the
paper in the same sequence as in the three chaos course lectures
the no professor with any self-respect would attend:

\begin{itemize}
  \item \statesp\ visualization
    \begin{itemize}
      \item \cLe
      \item Ring of Fire
      \item experimentalist description: a video 1D to 3D arrays of pixels
      \item theorist description: $\infty$-\dmn\ \statesp
      \item time orbit: point is a point, line is a line in all dimensions
      \item physical dimension: covariant Lyapunov vectors
      \item norms - distances between states
    \end{itemize}
  \item Poincar\'e sections, R\"ossler
    \begin{itemize}
      \item local chart
      \item R\"ossler {\poincBord}
      \item R\"ossler two-chart atlas
      \item R\"ossler unstable manifold curvilinear distance
      \item R\"ossler return map
      \item $N$-chart atlas, forward maps
      \item Ring of Fire return map\rf{lanCvit07}
    \end{itemize}
  \item what is a symmetry?
    \begin{itemize}
      \item pipes
      \item Ring of Fire
      \item \cLe
    \end{itemize}
  \item group action
    \begin{itemize}
      \item group orbit
      \item finite
      \item eqs of motion
      \item infinitesimal, Jacobian derivative
      \item moving frame
    \end{itemize}
  \item symmetry reduction
    \begin{itemize}
      \item the goal, 3 ideas:
      \item Hilbert
      \item method of connections
      \item \mslices
    \end{itemize}
  \item what it is not
    \begin{itemize}
      \item comoving frame
      \item an average over the group phase variables
      \item reduced-dimensionality model
    \end{itemize}
  \item \mslices, a local chart
    \begin{itemize}
      \item closest point on a group orbit
      \item variation $\to$ \slice\ hyperplane
    \end{itemize}
  \item \mslices, a global atlas
    \begin{itemize}
      \item flow in a \slice: \cLf\ example
      \item {\chartBord}: \cLf\ example
      \item 2-chart atlas, ridges:  \cLf\ example
      \item gauge fixing, geometric phase?
    \end{itemize}
  \item conclusions and summary
      \begin{itemize}
      \item theorist must work in \statesp, symmetry-reduce
      \item experimentalist must work in \statesp,
            symmetry-reduce the data
      \end{itemize}
\end{itemize}

\newpage


\section{How to read me}

Here is a novice's guide to desymmetrization bloggery:
\begin{itemize}
  \item
How to read the running blog: go first to the latest blog post, end
of \refsect{chap:atlasBlog}.
  \item
Guys writing the ultimate guide to slicing for the woman on the street,
\texttt{siminos/atlas/}, blog in \refchap{chap:atlas}{\em Atlas}.
  \item
If you are reading an article of common interest (which does not fit into
one of the specialized topics), enter your notes into \refsect{chap:atlasBlog}.
  \item
Comments to ChaosBook.org go into ChaosBook.tex section of siminos/blog/.
  \item
Plumbers who ponder how to slice experimental data also blog there,  in
 {\em Symmetry reduction of experimental data}
  \item
Save all figures (pdf or png, not eps) used in the blog in siminos/figs/
sv rm the figures no longer used (they all exist on the repository, if
you need to recover them any time later
  \item
Save useful source programs for figures used in the blog in siminos/figSrc/
  \item
Enter all bibliography items into siminos/bibtex/siminos.bib alphabetically by
first author, create no other *.bib file
  \item
Save useful simulation programs  in (for example) siminos/matlab/
  \item
Never commit large data sets, movies, blog.pdf or anything large that you
can easily recreate.


\end{itemize}

\section{How to share literature}
\label{s:HowToLit}

\begin{itemize}

\item[2012-03-28 PC] \emph{Saved articles.} If you put an article into
Dropbox, I can put it into \HREF{http://ChaosBook.org/library/}
{ChaosBook.org/library}. You can fetch it from there by clicking on
ChaosBook.org/library/BibTexName.pdf (login in as: student Lautrup),
where BibTexName is the name you gave it in siminos/bibtex/siminos.bib.

\item[2012-01-01 PC] In the long run that is a very awkward system, as we
do not have manpower for entering the articles into the
ChaosBook.org/library/index.html, so it is hit and miss finding out
whether a paper you want is there, unless linked it to this blog, like
this:

``Porter and E. Knobloch\rf{PoKno05}
(\HREF{http://ChaosBook.org/library/PoKno05.pdf}{click here})
.''

Much better solution is to join \HREF{http://www.zotero.org/groups/cns}
{www.zotero.org/groups/cns} in order to be able to access the papers we
are saving there, and save your own downloads there of copyright
protected publications. Search for zotero in siminos/blog/blog.pdf to
find a bit more info about it. There you can see what each article is,
use BibTeX to describe it, etc. Ask Gable to help you get started.

\item[2012-03-28 PC]
Put 'our' articles into zotero CNS [library] folder [symmetry] unless the
fit in another section better, enter them into
siminos/bibtex/siminos.bib, and note somewhere in this blog that you have
added them (and why?)


\end{itemize}
