% siminos/atlas/defAtlas.tex for atlas.tex
% $Author$ $Date$

%%%%%%%%%%%%%%%%%%%%%% CHAOS J SPECIFIC %%%%%%%%%%%%%%%%%%%%%%%%%%%%
\renewcommand{\reffig} [1] {Fig.~\ref{#1}}
\renewcommand{\reffigs} [2] {Figs.~\ref{#1} and~\ref{#2}}
\renewcommand{\refFig} [1] {Fig.~\ref{#1}}
\renewcommand{\refFigs} [2] {Figs.~\ref{#1} and~\ref{#2}}
\renewcommand{\refref} [1] {Ref.~\onlinecite{#1}}
\renewcommand{\refRef} [1] {Ref.~\onlinecite{#1}}
\renewcommand{\refrefs}[1] {Refs.~\onlinecite{#1}}
\renewcommand{\refRefs}[1] {Refs.~\onlinecite{#1}}

%%%%%%%%%%%%%%%%%%%%%% Atlas SPECIFIC %%%%%%%%%%%%%%%%%%%%%%%%%%%%%%%
\newcommand{\Template}{Template}
\newcommand\mapRed{\ensuremath{\hat{f}}}     % other people like \phi's etc
\newcommand\flowRed[2]{\ensuremath{\hat{f}^{#1}(#2)}}
\newcommand{\twoMode}{Porter-Knobloch}

\renewcommand{\zeit}{\ensuremath{\tau}}  %time variable Predrag
\renewcommand{\poincBord}{chart border}  % for chaos book - no need to distinguish
\renewcommand{\PoincBord}{Chart border}  %

\ifdraft    %%%%%%%%%%% display comments in text %%%%%%%%%%%%%%%%%%
   \newcommand{\PublicPrivate}[2]
       {\marginpar{\color{blue}$\Downarrow$}%
       {\color{blue}#2}%
       \marginpar{\color{blue}$\Uparrow$}}
   \newcommand{\toCB}{$\footnotemark\footnotetext{2CB}$}  % to compare with ChaosBook
   \newcommand{\PC}[1]{$\footnotemark\footnotetext{PC: #1}$}
   \newcommand{\PCedit}[1]{{\color{magenta}#1}}
   \newcommand{\MA}[1]{$\footnotemark\footnotetext{MA: #1}$}
   \newcommand{\MAedit}[1]{{\color{green}#1}}
   \newcommand{\APW}[1]{$\footnotemark\footnotetext{APW: #1}$}
   \newcommand{\APWedit}[1]{{\color{green}#1}}
   \newcommand{\DB}[2]{$\footnotemark\footnotetext{DB #1: {\color{red}#2}}$} %date, comment
   \newcommand{\DBedit}[1]{{\color{red}#1}}
   \newcommand{\KC}[2]{$\footnotemark\footnotetext{KC #1: {\color{cyan}#2}}$} %date, comment
   \newcommand{\KCedit}[1]{{\color{cyan}#1}}
   \newcommand{\LZ}[2]{$\footnotemark\footnotetext{LZ #1: #2}$} %date, comment
   \newcommand{\LZedit}[1]{{\color{green}#1}}
   \newcommand{\ES}[2]{$\footnotemark\footnotetext{ES #1: #2}$} %date, comment
   \newcommand{\ESedit}[1]{{\color{green}#1}}
   \newcommand{\mycomment}[2]{\noindent \textbf{\underline{#1}}: \emph{#2}}
   \newcommand{\edit}[1]{{\color{blue}#1}} % for referees
\else   % drop comments
   \newcommand{\PublicPrivate}[2]{#1}
   \newcommand{\toCB}{}
   \newcommand{\PC}[1]{}
   \newcommand{\MA}[1]{}
   \newcommand{\APW}[1]{}
%   \newcommand{\PCedit}[1]{{\color{magenta}#1}}  % for referees
   \newcommand{\PCedit}[1]{#1}
   \newcommand{\MAedit}[1]{#1}
%   \newcommand{\APWedit}[1]{{\color{green}#1}} % for referees
   \newcommand{\APWedit}[1]{#1}
   \newcommand{\DB}[2]{}{}
   \newcommand{\DBedit}[1]{#1}
   \newcommand{\KC}[2]{}{}
   \newcommand{\KCedit}[1]{#1}
   \newcommand{\LZ}[2]{}{}
   \newcommand{\LZedit}[1]{#1}
   \newcommand{\ES}[2]{}{}
   \newcommand{\ESedit}[1]{#1}
   \newcommand{\mycomment}[2]{}
   \newcommand{\edit}[1]{{\color{blue}#1}} % for referees
%   \newcommand{\edit}[1]{#1}               % for the journal

\fi %%%%% COMMENTS END %%%%%%%%%%%%%%%

\ifpaper % prepare for B&W paper printing:
       \renewcommand{\arXiv}[1]{ {\tt arXiv:#1}}
       \renewcommand{\mpArc}[1]{{\tt mp\_arc~#1}}
\else % prepare hyperlinked pdf
       \renewcommand{\mpArc}[1]{
              {\tt \href{http://www.ma.utexas.edu/mp_arc-bin/mpa?yn=#1}
                   {mp\_arc~#1}}}
       \renewcommand{\arXiv}[1]{
              {\tt \href{http://arXiv.org/abs/#1}{arXiv:#1}}}
\fi %%%% prepare for B&W paper printing END %%%%%%

\newcommand{\stateDsp}{state-space}
\newcommand{\StateDsp}{State-space}
\newcommand{\wurst}{wurst}
\newcommand{\Wurst}{Wurst}
%\newcommand{\reqvD}{traveling-wave}
%\newcommand{\ReqvD}{Traveling-wave}
%\newcommand{\reqvaD}{traveling-waves}
%\newcommand{\ReqvaD}{Traveling-waves}
%\newcommand{\REqvaD}{Traveling-Waves}

%%%% 3D physical flow
\newcommand{\NS}{Navier-Stokes}
\newcommand{\NSe}{Navier-Stokes equations}
\newcommand{\Reynolds}{\textit{Re}}  % Reynolds number
\newcommand{\cohStr}{coherent structure}
\newcommand{\recurrStr}{recurrent coherent structure}
\newcommand{\RecurrStr}{Recurrent coherent structure}
\newcommand{\bnabla}{\ensuremath{\bf \nabla}}
\newcommand{\Gpipe}{\ensuremath{\Gamma}} % Hoyle notation, equivariant symmetry group
\newcommand{\bCell}{\ensuremath{\Omega}}
\newcommand{\normVec}{\ensuremath{\mathbf{n}}}    % group orbit curvature normal

%%%%%%%%%%%%%%%%%%%%%%%%%%%%%%%%%%%%%%%%%%%%%%%%%%%%%%%%%%%
\renewcommand{\shift}{\ensuremath{\ell}}
\newcommand{\bu}{\ensuremath{{\bf u}}}
\newcommand{\be}{{\bf e}}
\newcommand{\bx}{{\bf x}}
\newcommand{\Norm}[1]{\|{#1}\|}
\newcommand{\ii}{\ensuremath{\mathrm{i}}} % sqrt{-1}
