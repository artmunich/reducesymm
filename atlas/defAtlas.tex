% siminos/atlas/defAtlas.tex for slice.tex
% $Author: predrag $ $Date: 2011-10-14 12:56:36 -0400 (Fri, 14 Oct 2011) $

% JFG: version history can be read with ``svn log''

%%%%%%%%%%%%%%%%%%%%%% sliced pipe SPECIFIC %%%%%%%%%%%%%%%%%%%%%%%%%%%%%%%

\newcommand{\zeit}{\ensuremath{t}}  %time variable Ashley
%\newcommand{\zeit}{\ensuremath{\tau}}  %time variable Predrag
\newcommand{\normVec}{\ensuremath{\mathbf{n}}}    % group orbit curvature normal

\newcommand{\sliceTan}[1]{\ensuremath{\mathbf{t}{}'_{#1}}}    % group orbit tangent at slice-fixing
%\newcommand{\sliceTan}[1]{\ensuremath{t{}'_{#1}}}    % group orbit tangent at slice-fixing
\newcommand{\groupTan}{\ensuremath{\mathbf{t}}}    % group orbit tangent
%\newcommand{\groupTan}{\ensuremath{t}}    % group orbit tangent
\newcommand{\Lg}{\ensuremath{\mathbf{T}}}   % FrCv11.tex Lie algebra generator
%\newcommand{\Lg}{\ensuremath{T}}   % FrCv11.tex Lie algebra generator
\newcommand{\LieEl}{\ensuremath{g}}  % Predrag Lie group element


\ifdraft    % display comments in text
   \newcommand{\PublicPrivate}[2]
       {\marginpar{\color{blue}$\Downarrow$\footnotesize PRIVATE}%
       {\color{blue}#2}%
       \marginpar{\color{blue}$\Uparrow$\footnotesize PRIVATE}}
   \newcommand{\toCB}{\marginpar{\footnotesize 2CB}}  % to compare with ChaosBook
   \newcommand{\PC}[1]{$\footnotemark\footnotetext{PC: #1}$}
   \newcommand{\PCedit}[1]{{\color{magenta}#1}}
   \newcommand{\MA}[1]{$\footnotemark\footnotetext{MA: #1}$}
   \newcommand{\MAedit}[1]{{\color{green}#1}}
   \newcommand{\APW}[1]{$\footnotemark\footnotetext{APW: #1}$}
   \newcommand{\APWedit}[1]{{\color{green}#1}}
   \newcommand{\file}[1]{$\footnotemark\footnotetext{{\bf file} #1}$}
   \newcommand{\mycomment}[2]{\noindent \textbf{\underline{#1}}: \emph{#2}}
   \newcommand{\edit}[1]{{\color{blue}#1}} % for referees
\else   % drop comments
   \newcommand{\PublicPrivate}[2]{#1}
   \newcommand{\toCB}{}
   \newcommand{\PC}[1]{}
   \newcommand{\MA}[1]{}
   \newcommand{\APW}[1]{}
%   \newcommand{\PCedit}[1]{{\color{magenta}#1}}  % for referees
   \newcommand{\PCedit}[1]{#1}
   \newcommand{\MAedit}[1]{#1}
%   \newcommand{\APWedit}[1]{{\color{green}#1}} % for referees
   \newcommand{\APWedit}[1]{#1}
   \newcommand{\RG}{}
   \newcommand{\file}[1]{}
   \newcommand{\mycomment}[2]{}
   \newcommand{\edit}[1]{{\color{blue}#1}} % for referees
%   \newcommand{\edit}[1]{#1}               % for the journal

\fi %%%%% COMMENTS END %%%%%%%%%%%%%%%

\ifcolorfigs            %% switch for color/BW figure captions
  \newcommand{\colorcomm}[2]{#1}
  %%%%%%%%%%%%%%%%%%%%%% Weblinks in PDF %%%%%%%%%%%%%%%%%%%
  % keep homepages flexible, if hyperlinked:
  %\newcommand{\weblink}[1]{{\tt #1}}
  %% RESTORE \newcommand{\weblink}[1]{\href{http://#1}{{\tt #1}}}
  %\newcommand{\arXiv}[1]{arXiv.org:#1}
  %% RESTORE \newcommand{\arXiv}[1]{\href{http://arXiv.org/abs/#1}{{\tt #1}}}
  %\newcommand{\wwwcb}[1]{{\tt ChaosBook.org#1}}
  %\newcommand{\HREF}[2]{{#2}}
  %% RESTORE \newcommand{\HREF}[2]{{\href{#1}{#2}}}
  \newcommand{\wwwcb}[1]{{\tt \href{http://ChaosBook.org#1}
         {ChaosBook.org#1}}}
  \newcommand{\wwwQFT}[1]{
         {\tt \href{http://ChaosBook.org/FieldTheory#1}
         {ChaosBook.org/\-Field\-Theory#1}}}
  \newcommand{\wwwnsQFT}[1]{
         {\tt \href{http://ChaosBook.org/FieldTheory#1}
         {ChaosBook.org/\-Field\-Theory#1}}}
  \newcommand{\weblink}[1]{{\tt \href{http://#1}{#1}}}
  \newcommand{\HREF}[2]{{\href{#1}{#2}}}
  \newcommand{\mpArc}[1]{
         {\tt \href{http://www.ma.utexas.edu/mp_arc-bin/mpa?yn=#1}
         {\goodbreak mp\_arc~#1}}}
  \newcommand{\arXiv}[1]{
         {\tt \href{http://arXiv.org/abs/#1}{\goodbreak #1}}}
\else
  \newcommand{\colorcomm}[2]{#2}
  %%%%%%%%%%%%%%%%%%%%%% No weblinks in PDF %%%%%%%%%%%%%%%%%%%
  \newcommand{\wwwcb}[1]{{\tt ChaosBook.org#1}}
  \newcommand{\wwwQFT}[1]{{\tt ChaosBook.org/\-Field\-Theory#1}}
  \newcommand{\wwwnsQFT}[1]{{\tt ChaosBook.org/\-Field\-Theory#1}}
  \newcommand{\weblink}[1]{{\tt #1}}
  \newcommand{\HREF}[2]{{#2}}
  \newcommand{\mpArc}[1]{{\tt \goodbreak mp\_arc~#1}}
  \newcommand{\arXiv}[1]{{\tt \goodbreak #1}}
\fi

\ifJFM
 \newcommand{\citecomm}[2]{#2}  % J Fluid Mech style citation - no names
\else
 \newcommand{\citecomm}[2]{#1}  % physics style citation - with names
\fi

\newcommand{\rf}     [1] {~\cite{#1}}
%   Phys Rev format:
%\newcommand{\refref} [1] {ref.~\cite{#1}}
%\newcommand{\refRef} [1] {Ref.~\cite{#1}}
%\newcommand{\refrefs}[1] {refs.~\cite{#1}}
%\newcommand{\refRefs}[1] {Refs.~\cite{#1}}
%   JFM format:
\newcommand{\refref} [1] {\cite{#1}}
\newcommand{\refRef} [1] {\cite{#1}}
\newcommand{\refrefs}[1] {\cite{#1}}
\newcommand{\refRefs}[1] {\cite{#1}}

\newcommand{\refeq}  [1] {(\ref{#1})}
\newcommand{\refeqs} [2]{(\ref{#1}--\ref{#2})}
\newcommand{\eqn}[1]{eqn.\ {\ref{#1}}}
\newcommand{\Eqn}[1]{Eqn.\ {\ref{#1}}}
\newcommand{\refpage}[1] {p.~\pageref{#1}}
\newcommand{\reffig} [1] {figure~\ref{#1}}
\newcommand{\reffigs} [2] {figures~\ref{#1} and~\ref{#2}}
\newcommand{\refFig} [1] {Figure~\ref{#1}}
\newcommand{\refFigs} [2] {Figures~\ref{#1} and~\ref{#2}}
\newcommand{\reftab} [1] {table~\ref{#1}}
\newcommand{\refTab} [1] {Table~\ref{#1}}
\newcommand{\reftabs}[2] {tables~\ref{#1} and~\ref{#2}}
\newcommand{\refsect}[1] {\S\,\ref{#1}}
\newcommand{\refsects}[2] {\S\,\ref{#1} and \S\,\ref{#2}}
\newcommand{\refSect}[1] {\S\,\ref{#1}}
\newcommand{\refchap}[1] {chapter~\ref{#1}}
\newcommand{\refChap}[1] {Chapter~\ref{#1}}
\newcommand{\refchaps}[2] {chapters~\ref{#1} and \ref{#2}}
\newcommand{\refchaptochap}[2] {chapters~\ref{#1} to \ref{#2}}
\newcommand{\refappe}[1] {appendix~\ref{#1}}
\newcommand{\refappes}[2] {appendices~\ref{#1} and \ref{#2}}
\newcommand{\refAppe}[1] {Appendix~\ref{#1}}
\newcommand{\refrem} [1] {remark~\ref{#1}}
\newcommand{\refexam}[1] {example~\ref{#1}}
\newcommand{\refExam}[1] {Example~\ref{#1}}
\newcommand{\refexer}[1] {exercise~\ref{#1}}
\newcommand{\refExer}[1] {Exercise~\ref{#1}}
\newcommand{\refsolu}[1] {solution~\ref{#1}}

%%%%%%%%%%%%%%% EQUATIONS %%%%%%%%%%%%%%%%%%%%%%%%%%%%%%%
\newcommand{\beq}{\begin{equation}}
\newcommand{\continue}{\nonumber \\ }
\newcommand{\nnu}{\nonumber}
\newcommand{\eeq}{\end{equation}}
\newcommand{\ee}[1] {\label{#1} \end{equation}}
\newcommand{\ceq}{\nonumber \\ & & }
\newcommand{\bea}{\begin{eqnarray}}
\newcommand{\eea}{\end{eqnarray}}
\newcommand{\barr}{\begin{array}}
\newcommand{\earr}{\end{array}}

%%%%%%%%%%%%%%%%%%%%%% QUOTATIONS %%%%%%%%%%%%%%%%%%%%%%%%%%%%%%%%%%%%%%
%
%  the learned/witty quotes at the chapter and section headings
%  (liberated from Das Buch defs.tex)
\newsavebox{\bartName}
\newcommand{\bauthor}[1]{\sbox{\bartName}{\parbox{\textwidth}{\vspace*{0.8ex}
       %\hspace*{\fill}
       \small\noindent #1}}}
\newenvironment{bartlett}{\hfill\begin{minipage}[t]{0.65\textwidth}\small}%
{\hspace*{\fill}\nolinebreak[1]\usebox{\bartName}\vspace*{1ex}\end{minipage}}

%
%  a quotation inserted into the text
%
\newenvironment{txtquote}{\begin{quotation} \small}{\end{quotation}}

\newcommand{\newfp}{\ensuremath{\bu_{\text{\tiny NB}}}}
\newcommand{\Newfp}{\ensuremath{\bu_{\text{\tiny NB}}}}

%%%%%%%%%%%%%%%%%%%%%% birdtracks SPECIFIC %%%%%%%%%%%%%%%%%%%%%%%%%%%%%%%
\newcommand{\PP}{{\mathbf P}}                   % projection operator

%%%%%%%%%%%%%%  Abbreviations %%%%%%%%%%%%%%%%%%%%%%%%%%%%%%%%%%%%%%%%
%%% APS (American Physiology Society, it seems) style:
%%%     Latin or foreign words or phrases should be roman, not italic.

\newcommand{\etc}{{etc.}}       % APS
\newcommand{\etal}{{\em et al.}}    % etal in italics, APS too
\newcommand{\ie}{{i.e.}}        % APS
\newcommand{\cf}{{\em cf.}}     % APS
\newcommand{\eg}{{e.g.}}        % APS

\ifJFM                          % J Fluid Mech macros
\renewcommand\etal{\mbox{\textit{et al.}}}
\renewcommand\etc{etc.\ }
\renewcommand\eg{e.g.\ }
\else
\fi

%%%%%%%%%%%% Froehlich's FAVORITE MACROS %%%%%%%%%%

\newcommand\PoincSec{Poincar\'e section}

	% without large brackets:
\newcommand{\braket}[2]
		   {\langle{#1}\vphantom{#2}|\vphantom{#1}{#2}\rangle}
\newcommand{\bra}[1]{\langle{#1}\vphantom{ }|}
\newcommand{\ket}[1]{|\vphantom{}{#1}\rangle}

\newcommand{\dual}[1]{{#1}^T}		% SO(n) case
\newcommand{\Sset}{Inflection hyperplane}
\newcommand{\sset}{inflection hyperplane} % {singularity hyperplane}
\newcommand{\sspSing}{\ensuremath{\ssp^*}} 	% inflection point
\newcommand{\sspRSing}{\ensuremath{\sspRed^*}} 	% inflection point, reduced space
\newcommand{\template}{template} % {slice-fixing point} % {reference state}
\newcommand{\angVel}{angular velocity}
\newcommand{\angVels}{angular velocities}

%%%%%%%%%%%% SIMINOS' FAVORITE MACROS %%%%%%%%%%

\newcommand{\po}{periodic orbit}
\newcommand{\Po}{Periodic orbit}
\newcommand{\rpo}{rela\-ti\-ve periodic orbit}
%   \newcommand{\rpo}{equivariant periodic orbit}
\newcommand{\Rpo}{Rela\-ti\-ve periodic orbit}
%   \newcommand{\Rpo}{Equivariant periodic orbit}
\newcommand{\UPO}{unstable periodic orbit}
\newcommand{\reducedsp}{reduced state space}
\newcommand{\Reducedsp}{Reduced state space}
\newcommand{\fixedsp}{fixed-point subspace}
\newcommand{\Fixedsp}{Fixed-point subspace}
\newcommand{\slice}{slice}
\newcommand{\Slice}{Slice}
\newcommand{\mslices}{method of slices}
\newcommand{\Mslices}{Method of slices}
\newcommand{\mframes}{method of moving frames}
\newcommand{\Mframes}{Method of moving frames}

\newcommand{\timeStep}{\ensuremath{{\delta \tau}}}  %integration step
\newcommand{\id}{{\ \hbox{{\rm 1}\kern-.6em\hbox{\rm 1}}}}

\newcommand{\On}[1]{\ensuremath{\textrm{O}(#1)}}
\newcommand{\SOn}[1]{\ensuremath{\textrm{SO}(#1)}}         % in DasBuch
%\newcommand{\Dn}[1]{\ensuremath{\mathbf{D}_{#1}}    % in Siminos thesis
\newcommand{\Dn}[1]{\ensuremath{\textrm{D}_{#1}}}              % in DasBuch
%\newcommand{\Zn}[1]{\ensuremath{\mathbf{Z}_{#1}}}    % in Siminos thesis
\newcommand{\Zn}[1]{\ensuremath{\textrm{C}_{#1}}}              % in DasBuch

\newcommand{\pSRed}{\ensuremath{\bar{\cal M}}} % reduced state space
\newcommand{\sspRed}{\ensuremath{\bar{\ssp}}}    % reduced state space point, experiment
\newcommand{\velRed}{\ensuremath{\bar{\vel}}}    % ES reduced state space velocity
\newcommand{\slicep}{\ensuremath{\bar{\ssp}'}}   % slice-fixing point, experimental
\newcommand{\Group}{\ensuremath{G}}         % Predrag Lie or discrete group
\newcommand{\Fix}[1]{\ensuremath{\mathrm{Fix}\left(#1\right)}}
\newcommand{\gSpace}{\ensuremath{{\bf \theta}}}   % group rotation parameters
\newcommand{\velRel}{\ensuremath{c}}    % relative state velocity

%%%%%%%%%%%%%%% WALLY's FAVORITE MACROS %%%%%%%%%%%%%%%%%%%%%%
\newcommand{\bvec}[1]{\boldsymbol{#1}}
\newcommand{\Pv}{{\mathcal{P}_v}}
\newcommand{\Pe}{{\mathcal{P}_\eta}}
\newcommand{\vv}{\bvec{v}}      % \newcommand{vv{{\mbox {\boldmath $v$}}}
\newcommand{\vx}{\bvec{x}}      % \newcommand{vx{{\mbox {\boldmath $x$}}}
\newcommand{\vk}{\bvec{k}}      % \newcommand{vk{{\mbox {\boldmath $k$}}}
\newcommand{\uh}{\hat{u}} \newcommand{\vh}{\hat{v}}
\newcommand{\wh}{\hat{w}} \newcommand{\eh}{\hat{\eta}}

%%%%%%%%%%%%%%% GIBSON FAVORITE MACROS %%%%%%%%%%%%%%%%%%%%%%
\defcitealias{GHCW07}{GHC}

%%%%%%%%%%%%%%% DIVAKAR"S FAVORITE MACROS %%%%%%%%%%%%%%%%%%%%%%
\newcommand {\abs} [1] {\left| #1 \right|}
\newcommand {\Abs} [1] {\bigl\lvert #1 \bigr\rvert}

%%%%%%%%%%%%    PREDRAG'S FAVORITE MACROS %%%%%%%%%%%%%

\newcommand{\NS}{Navier-Stokes}
\newcommand{\NSe}{Navier-Stokes equations}
\newcommand{\NSE}{Navier-Stokes Equations}
\newcommand{\KS}{Kuramoto-Sivashinsky}
\newcommand{\KSe}{Kuramoto-Sivashinsky equation}
% \newcommand{\KS}{KS}
% \newcommand{\KSe}{KS equation}
\newcommand{\Reynolds}{\textit{Re}}  % Reynolds number
% \newcommand\Rey{\mbox{\textit{Re}}}  % Reynolds number ELIMINATE?
\newcommand{\pC}{plane Couette}
\newcommand{\Pc}{Plane Couette}
\newcommand{\pCf}{plane Couette flow}
\newcommand{\PCf}{Plane Couette flow}
\newcommand{\ubranch}{upper-branch}
\newcommand{\Ubranch}{Upper-branch}
\newcommand{\lbranch}{lower-branch}
\newcommand{\Lbranch}{Lower-branch}

\newcommand{\eqv}{equilib\-rium}
\newcommand{\Eqv}{Equilib\-rium}
\newcommand{\eqva}{equilib\-ria}
\newcommand{\Eqva}{Equilib\-ria}
\newcommand{\eqpoint}{equilibrium point}
\newcommand{\Eqpoint}{Equilibrium point}
\newcommand{\equilibrium}{equilib\-rium}
\newcommand{\equilibria}{equilib\-ria}
\newcommand{\Equilibria}{Equilib\-ria}
%\newcommand{\reqv}{rela\-ti\-ve equilib\-rium}
%\newcommand{\Reqv}{Rela\-ti\-ve equilib\-rium}
%\newcommand{\reqva}{rela\-ti\-ve equilib\-ria}
%\newcommand{\Reqva}{Rela\-ti\-ve equilib\-ria}
\newcommand{\reqv}{travelling wave}
\newcommand{\Reqv}{Travelling wave}
\newcommand{\reqva}{travelling waves}
\newcommand{\Reqva}{Travelling waves}
\newcommand{\reqvD}{travelling-wave}
\newcommand{\ReqvD}{Travelling-wave}
\newcommand{\reqvaD}{travelling-waves}
\newcommand{\ReqvaD}{Travelling-waves}
\newcommand{\REqvaD}{Travelling-Waves}

%%% 3D physical flow
\newcommand{\stagp}{stagnation point}
\newcommand{\Stagp}{Stagnation point}
\newcommand{\relstagp}{traveling stagnation point}
\newcommand{\Relstagp}{Traveling stagnation point}

\newcommand{\Hec}{Heteroclinic connection}
\newcommand{\hec}{heteroclinic connection}
\newcommand{\HeC}{Heteroclinic Connection}

\newcommand{\cohStr}{coherent structure}
\newcommand{\recurrStr}{recurrent coherent structure}
\newcommand{\RecurrStr}{Recurrent coherent structure}
%\newcommand{\cohStr}{coherent state}
%\newcommand{\recurrStr}{recurrent coherent state}
%\newcommand{\RecurrStr}{Recurrent coherent state}

\newcommand{\stateDsp}{state-space}
\newcommand{\StateDsp}{State-space}
\newcommand{\Statesp}{State space}
\newcommand{\statesp}{state space}
\newcommand{\nameit}{E}         % equilibrium label
\newcommand{\SIS}{non-wondering set}
\newcommand{\descent}{Newton descent}
\newcommand{\Descent}{Newton Descent}
\newcommand{\stabmat}{stability matrix}     % stability matrix
\newcommand{\Stabmat}{Stability matrix}     % Stability matrix
\newcommand{\jacobianM}{fundamental matrix}     % standard name
\newcommand{\jacobianMs}{fundamental matrices}  %
\newcommand{\JacobianM}{Fundamental matrix}     %
\newcommand{\JacobianMs}{Fundamental matrices}  %
%\newcommand{\jacobian}{Jacobian}                % determinant
%\newcommand{\jacobianM}{Jacobian matrix}        % matrix
%\newcommand{\jacobianMs}{Jacobian matrices}     % matrices

\newcommand{\ew}{eigen\-value}
\newcommand{\ev}{eigen\-vector}
\newcommand{\ef}{eigen\-function}

\newcommand{\steady}{\marginpar{{\color{green}\textdollar}}}

%%%%%%%%%%%%%%%%%%%%%%%%%%%%%%%%%%%%%%%%%%%%%%%%%%%%%%%%%%%
% JFG favorite macros
\newcommand{\utot}{\ensuremath{\bu_{\text{\tiny tot}}}} % total velocity
\newcommand{\bu}{\ensuremath{{\bf u}}}
\newcommand{\bv}{\ensuremath{{\bf v}}}
\newcommand{\bff}{\ensuremath{{\bf f}}}
\newcommand{\dbu}{\delta {\bf u}}
\newcommand{\dbv}{\delta {\bf v}}
\newcommand{\hbu}{\tilde{{\bf u}}}
\newcommand{\hbv}{\tilde{{\bf v}}}
\newcommand{\hu}{\tidle{u}}
\newcommand{\hv}{\tidle{v}}
\newcommand{\hw}{\tidle{w}}
\newcommand{\be}{{\bf e}}
\newcommand{\bx}{{\bf x}}
\newcommand{\ex}{{\hat{\bf x}}} % unit vectors
\newcommand{\ey}{{\hat{\bf y}}}
\newcommand{\ez}{{\hat{\bf z}}}
\newcommand{\Om}{\Omega}    % JFG mantra
\newcommand{\bnabla}{\ensuremath{\bf \nabla}}
\newcommand{\GPCF}{\ensuremath{\Gamma}} % Hoyle notation, equivariant symmetry group
\newcommand{\Gpipe}{\ensuremath{\Gamma}} % Hoyle notation, equivariant symmetry group

\newcommand{\bPhi}{{\bf \Phi}}
\newcommand{\bphi}{{\bf \phi}}
\newcommand{\bhphi}{{\bf \hat{\phi}}}
\newcommand{\bU}{{\bf U}}
\newcommand{\bW}{{\bf W}}
\newcommand{\lapl}{\nabla^2}
\newcommand{\tNS}{\ensuremath{{\text{NS}}}}
\newcommand{\tCFD}{\ensuremath{{\text{CFD}}}}
\newcommand{\stagn}{\ensuremath{\text{\tiny EQ}}}% JFG equilib/stagnation point

%%%%%%%%%%%%%%%%%%%%%%%%%%%%%%%%%%%%%%%%%%%%%%%%%%%%%%%%%%%

%John F. Gibson - Set in stone:                   May 9, 2008
% PC \eqva/\reqva directory: \tAA is the name, \sAA is the  symbol
% \newcommand{\tEQ}[1]{\ensuremath{{\text{EQ}#1}}}
\newcommand{\tEQ}{\ensuremath{{\text{EQ}}}}
\newcommand{\tLM}{\ensuremath{{\text{EQ}_0}}}
\newcommand{\tLB}{\ensuremath{{\text{EQ}_1}}}
\newcommand{\tUB}{\ensuremath{{\text{EQ}_2}}}
\newcommand{\tNNB}{\ensuremath{{\text{EQ}_3}}}
\newcommand{\tNB}{\ensuremath{{\text{EQ}_4}}}

\newcommand{\tEQzero}{\ensuremath{{\text{EQ}_0}}}
\newcommand{\tEQone}{\ensuremath{{\text{EQ}_1}}}
\newcommand{\tEQtwo}{\ensuremath{{\text{EQ}_2}}}
\newcommand{\tEQthree}{\ensuremath{{\text{EQ}_3}}}
\newcommand{\tEQfour}{\ensuremath{{\text{EQ}_4}}}
\newcommand{\tEQfive}{\ensuremath{{\text{EQ}_5}}}
\newcommand{\tEQsix}{\ensuremath{{\text{EQ}_6}}}
\newcommand{\tEQsev}{\ensuremath{{\text{EQ}_7}}}
\newcommand{\tEQeight}{\ensuremath{{\text{EQ}_8}}}
\newcommand{\tEQnine}{\ensuremath{{\text{EQ}_9}}}
\newcommand{\tEQnineb}{\ensuremath{{\text{EQ}_{9b}}}}
\newcommand{\tEQten}{\ensuremath{{\text{EQ}_{10}}}}
\newcommand{\tEQelev}{\ensuremath{{\text{EQ}_{11}}}}
\newcommand{\tEQtwel}{\ensuremath{{\text{EQ}_{12}}}}
\newcommand{\tEQthirt}{\ensuremath{{\text{EQ}_{13}}}}

\newcommand{\tTW}[1]{\ensuremath{{\text{TW}_{#1}}}}
\newcommand{\tTWone}{\ensuremath{{\text{TW}_1}}}  % spanwise
\newcommand{\tTWtwo}{\ensuremath{{\text{TW}_2}}} % Divakar D1, lower streamwise
\newcommand{\tTWDone}{\ensuremath{{\text{TW}_2}}} % Divakar D1, lower streamwise
\newcommand{\tTWthree}{\ensuremath{{\text{TW}_3}}}  % upper streamwise

% PC \eqva labeling symbols for all figures: halcrow/figsSrc/drawsyms.tex

\colorcomm{\definecolor{orangina}{rgb}{0.9,0.7,0}}{}
\newcommand{\sLM}{\ensuremath{\odot}}

\newcommand{\sLB}{\colorcomm{{\Large \ensuremath{\color{blue}\circ}}}
                            {{\Large \ensuremath{\circ}}}}
\newcommand{\sUB}{\colorcomm{{\Large \ensuremath{\color{blue}\bullet}}}
                            {{\Large \ensuremath{\bullet}}}}

\newcommand{\sNNB}{\colorcomm{{\scriptsize \ensuremath{\color{red}\square}}}
                            {{\scriptsize \ensuremath{\square}}}}
\newcommand{\sNB}{\colorcomm{{\scriptsize \ensuremath{\color{red}\blacksquare}}}
                            {{\scriptsize \ensuremath{\blacksquare}}}}

\newcommand{\sEQzero}{\ensuremath{\odot}}

\newcommand{\sEQone}{\colorcomm{{\Large \ensuremath{\color{blue}\circ}}}
                            {{\Large \ensuremath{\circ}}}}
\newcommand{\sEQtwo}{\colorcomm{{\Large \ensuremath{\color{blue}\bullet}}}
                            {{\Large \ensuremath{\bullet}}}}

\newcommand{\sEQthree}{\colorcomm{{\scriptsize \ensuremath{\color{red}\square}}}
                            {{\scriptsize \ensuremath{\square}}}}
\newcommand{\sEQfour}{\colorcomm{{\scriptsize \ensuremath{\color{red}\blacksquare}}}
                            {{\scriptsize \ensuremath{\blacksquare}}}}

\newcommand{\sEQfive}{\colorcomm{\ensuremath{\color{green}\lozenge}}
                            {\ensuremath{\lozenge}}}
\newcommand{\sEQsix}{\colorcomm{\ensuremath{\color{green}\blacklozenge}}
                            {\ensuremath{\blacklozenge}}}

\newcommand{\sEQsev}{\colorcomm{\ensuremath{\triangleleft}}
                            {\ensuremath{\triangleleft}}}
\newcommand{\sEQeight}{\colorcomm{\ensuremath{\blacktriangleleft}}
                            {\ensuremath{\blacktriangleleft}}}

\newcommand{\sEQnine}{\colorcomm{\ensuremath{\color{orangina}\bigstar}}
                            {\ensuremath{\bigstar}}}

\newcommand{\sEQten}{\colorcomm{\ensuremath{\color{magenta}\triangledown}}
                            {\ensuremath{\triangledown}}}
\newcommand{\sEQelev}{\colorcomm{\ensuremath{\color{magenta}\blacktriangledown}}
                            {\ensuremath{\blacktriangledown}}}

\newcommand{\sEQtwel}{\colorcomm{\ensuremath{\color{magenta}\triangle}}
                            {\ensuremath{\triangle}}}
\newcommand{\sEQthirt}{\colorcomm{\ensuremath{\color{magenta}\blacktriangle}}
                            {\ensuremath{\blacktriangle}}}


\newcommand{\sTWone}{\colorcomm{{\large \ensuremath{\color{blue}\triangleright}}}
                            {{\large \ensuremath{\triangleright}}}}
\newcommand{\sTWDone}{\colorcomm{\ensuremath{\color{red}\vartriangle}}
                            {\ensuremath{\vartriangle}}}
\newcommand{\sTWtwo}{\colorcomm{\ensuremath{\color{red}\vartriangle}}
                            {\ensuremath{\vartriangle}}}
\newcommand{\sTWthree}{\colorcomm{\ensuremath{\color{green}\triangle}}
                            {\ensuremath{\triangle}}}

% PC \eqva velocity field naming conventions
\newcommand{\uEQ}{\ensuremath{\bu_{\text{\tiny EQ}}}}
\newcommand{\uLM}{\ensuremath{\bu_{\text{\tiny EQ0}}}}
\newcommand{\uLB}{\ensuremath{\bu_{\text{\tiny EQ1}}}}
\newcommand{\uUB}{\ensuremath{\bu_{\text{\tiny EQ2}}}}
\newcommand{\uNNB}{\ensuremath{\bu_{\text{\tiny EQ3}}}}
\newcommand{\uNB}{\ensuremath{\bu_{\text{\tiny EQ4}}}}
\newcommand{\uEQfive}{\ensuremath{\bu_{\text{\tiny EQ5}}}}
\newcommand{\uEQsix}{\ensuremath{\bu_{\text{\tiny EQ6}}}}
\newcommand{\uEQsev}{\ensuremath{\bu_{\text{\tiny EQ7}}}}
\newcommand{\uEQeight}{\ensuremath{\bu_{\text{\tiny EQ8}}}}
\newcommand{\uEQnine}{\ensuremath{\bu_{\text{\tiny EQ9}}}}
\newcommand{\uEQten}{\ensuremath{\bu_{\text{\tiny EQ10}}}}
\newcommand{\uEQelev}{\ensuremath{\bu_{\text{\tiny EQ11}}}}

\newcommand{\huEQ}{\ensuremath{\hbu_{\text{\tiny EQ}}}}
\newcommand{\huLM}{\ensuremath{\hbu_{\text{\tiny EQ0}}}}
\newcommand{\huLB}{\ensuremath{\hbu_{\text{\tiny EQ1}}}}
\newcommand{\huUB}{\ensuremath{\hbu_{\text{\tiny EQ2}}}}
\newcommand{\huNNB}{\ensuremath{\hbu_{\text{\tiny EQ3}}}}
\newcommand{\huNB}{\ensuremath{\hbu_{\text{\tiny EQ4}}}}
\newcommand{\huEQfive}{\ensuremath{\hbu_{\text{\tiny EQ5}}}}
\newcommand{\huEQsix}{\ensuremath{\hbu_{\text{\tiny EQ6}}}}
\newcommand{\huEQsev}{\ensuremath{\hbu_{\text{\tiny EQ7}}}}
\newcommand{\huEQeight}{\ensuremath{\hbu_{\text{\tiny EQ8}}}}
\newcommand{\huEQnine}{\ensuremath{\hbu_{\text{\tiny EQ9}}}}
\newcommand{\huEQten}{\ensuremath{\hbu_{\text{\tiny EQ10}}}}
\newcommand{\huEQelev}{\ensuremath{\hbu_{\text{\tiny EQ11}}}}

\newcommand{\uREQV}{\ensuremath{\bu_{\text{\tiny TW}}}}
\newcommand{\uTW}{\ensuremath{\bu_{\text{\tiny TW}}}}
\newcommand{\uTWone}{\ensuremath{\bu_{\text{\tiny TW1}}}}
\newcommand{\uTWDone}{\ensuremath{\bu_{\text{\tiny TW2}}}}
\newcommand{\uTWthree}{\ensuremath{\bu_{\text{\tiny TW3}}}}

% PC \eqva stability eigenfunction naming conventions
\newcommand{\vEQ}{\ensuremath{\bv_{\text{\tiny EQ}}}}
\newcommand{\vLM}{\ensuremath{\bv_{\text{\tiny EQ0}}}}
\newcommand{\vLB}{\ensuremath{\bv_{\text{\tiny EQ1}}}}
\newcommand{\vUB}{\ensuremath{\bv_{\text{\tiny EQ2}}}}
\newcommand{\vNNB}{\ensuremath{\bv_{\text{\tiny EQ3}}}}
\newcommand{\vNB}{\ensuremath{\bv_{\text{\tiny EQ4}}}}

% to distinguish between members of a reflection-symmetric pair.

% Redefine using mathrm, it is a label not a math symbol
\newcommand{\EQV}[1]{\ensuremath{\mathrm{E}_{#1}}}
% \newcommand{\EQV}[1]{\ensuremath{\text{EQ}_{#1}}} % ELIMINATE
% E_0: u = 0 - trivial equilibrium
% E_1,E_2,E_3, for 1,2,3-wave equilibria
\newcommand{\REQV}[2]{\ensuremath{\mathrm{TW}_{#1#2}}} % #1 is + or -
% \newcommand{\REQV}[2]{{\ensuremath{\text{TW}_{#1#2}}}}
% TW_1^{+,-} for 1-wave traveling waves (positive and negative velocity).
\newcommand{\PO}[1]{\ensuremath{\mathrm{PO}_{#1}}}
% PO_{period to 2-4 significant digits} - periodic orbits
% \newcommand{\PO}[1]{\ensuremath{\text{P$#1$}}}
\newcommand{\RPO}[1]{\ensuremath{\mathrm{RPO}_{#1}}}
% \newcommand{\RPO}[1]{\ensuremath{\text{RP$#1$}}}
% RPO_{period to 2-4 significant digits} - relative PO.  We use ^{+,-}
% to distinguish between members of a reflection-symmetric pair.
% Gibson likes:
\renewcommand{\tEQ}{\ensuremath{\mathrm{EQ}}}


% isotropy subgroup $H \incl G$:
\newcommand{\isotropyG}[1]{\ensuremath{H_{\text{\tiny #1}}}}

\newcommand{\bCell}{\ensuremath{\Omega}}
\newcommand{\bNarrow}{\ensuremath{\Omega_{\text{\tiny GHC}}}}
    % PC: W for Waleffe
    % JG: W02 for Waleffe Tokyo proceedings 2002, where this cell first appears,
    % ok'd by wally, barely
\newcommand{\bHKW}{\ensuremath{\Omega_{\text{\tiny{HKW}}}}}
\newcommand{\bbR}{\mathbb{R}}
\newcommand{\bbU}{\mathbb{U}}
\newcommand{\hbbU}{\hat{\mathbb{U}}}
\newcommand{\bbUsymm}{\ensuremath{\bbU_{S}}}
% \newcommand{\bbUS}{\ensuremath{\bbU_{S}}}
\newcommand{\bbUone}{\ensuremath{\bbU_{S1}}}
\newcommand{\bbUtwo}{\ensuremath{\bbU_{S2}}}
\newcommand{\bbUthree}{\ensuremath{\bbU_{S3}}}
\newcommand{\half}{\frac{1}{2}}
\newcommand{\pd}[2]{\frac{\partial #1}{\partial #2}}
\newcommand{\Norm}[1]{\|{#1}\|}
\newcommand{\grad}{\boldsymbol{\nabla}}
\newcommand{\Ssym}[1]{\ensuremath{s_{#1}}}  % from dasbuch symbolic dynamics

\newcommand{\evOper}{evolution oper\-ator}
\newcommand{\EvOper}{Evolution oper\-ator}
\newcommand{\Fd}{spec\-tral det\-er\-min\-ant}
\newcommand{\fd}{spec\-tral det\-er\-min\-ant}
\newcommand{\FD}{Spec\-tral det\-er\-min\-ant}

%%%%%multiletter symbols
\newcommand\Real{\mbox{Re}} % cf plain TeX's \Re, not Reynolds number
\newcommand\Imag{\mbox{Im}} % cf plain TeX's \Im

%%%%%%%%%%%%%%% Sundry symbols within math eviron.: %%%%%%%%%%%%
\newcommand\flow[2]{{f^{#1}(#2)}}
\newcommand\timeflow{{f^t}}
\newcommand{\reals}{\mathbb{R}}
\newcommand{\PoincS}{{\cal P}}     % symbol for Poincare section
\newcommand{\PoincM}{{P}}      % symbol for Poincare map
\newcommand{\PoincC}{{U}}      % symbol for Poincare constraint function
\newcommand{\pde}{\partial}
\newcommand{\jMP}{{\bf \hat{J}}}   % jacobiam matrix, Poincare return
\newcommand{\monodromy}{{\bf J}}   % monodromy matrix, full Poincare cut
                   % Fredholm det jacobian weight:
\newcommand{\oneMinJ}[1]
           {\left|\det\!\left(\matId-\monodromy_p^{#1}\right)\right|}

\newcommand{\dmn}{-dimensional} %{\!-\!dimensional}

\newcommand{\obser}{a}      % an observable from state space to R^n
\newcommand{\Obser}{A}      % time integral of an observable
\newcommand{\expct}    [1]{\left\langle {#1} \right\rangle}
\newcommand{\spaceAver}[1]{\left\langle {#1} \right\rangle}
\newcommand{\timeAver} [1]{\overline{#1}}
\newcommand{\Lop}{{\cal L}}    % evolution operator
\renewcommand\Im{{\rm Im\,}}
\renewcommand\Re{{\rm Re\,}}
\renewcommand{\det}{\mbox{\rm det}\,}
\newcommand{\Det}{\mbox{\rm Det}\,}
\newcommand{\tr}{\mbox{\rm tr}\,}
\newcommand{\Tr}{\mbox{\rm tr}\,}
\newcommand{\sign}[1]{\sigma_{#1}}

\newcommand{\Refl}{\ensuremath{\mathbf{R}}}
\newcommand{\Shift}{\ensuremath{\mathbf{S}}}
\newcommand{\shift}{\ensuremath{\ell}}
\newcommand{\trHalf}[1]{\ensuremath{\tau_{#1}}}    % 1/2 cell translation
% \newcommand{\trHalf}[1]{\tau_{#1}^{1/2}}    % 1/2 cell translation
\newcommand{\trDiscr}[2]{\tau_{#1}^{#2}}    % discrete cell translation 1/4, ...
\newcommand\period[1]{{T_{#1}}}         %continuous cycle period
\newcommand{\cl}[1]{{n_{#1}}}   % discrete length of a cycle, Predrag
\newcommand{\pS}{{\cal M}}          % symbol for state space
% \newcommand\pSpace{x}     % phase space x=(q,p) coordinate
\newcommand{\ssp}{a}            % state space point
\newcommand{\vel}{\ensuremath{v}}   % state space velocity
\newcommand\velField[1]{{F(#1)}}    % Gibson statespace velocity field
\newcommand\vField{\ensuremath{{\bf F}}} % yet another Gibson statesp vel field
\newcommand\vCM{mean velocity}  % or `center-of-mass velocity?' `bulk momentum?'
\newcommand\xInit{{a_0}}        %initial x
\newcommand{\deltaX}{{\delta a}}                %trajectory displacement

\newcommand{\costFct}{cost function}    % functional to minimize
\newcommand{\costF}{F^2}        % cost function,
\newcommand{\Loop}{L}
\newcommand{\pVeloc}{v}         % phase-space velocity
\newcommand{\lSpace}{\tilde{x}}     % a point on a loop
\newcommand{\lVeloc}{\tilde{v}}     % loop tangent
\newcommand{\damp}{\Delta\tau}      % descrete fictitous time step
\newcommand{\prpgtr}[1]{\delta\negthinspace\left( {#1} \right)}
\newcommand{\matId}{{\bf 1}}       % matrix identity
\newcommand{\inertM}{{\mathcal M}}          % inertial manifold
\newcommand{\Mvar}{\ensuremath{A}}  % stability matrix
% \newcommand{\derF}[1]{\ensuremath{A(#1)}}   % Predrag stability matrix
\newcommand{\jEigvec}[1]{\ensuremath{{\bf e}^{(#1)}}}   % jacobiam eigenvector
\newcommand{\jEigvecT}[1]{\ensuremath{{\bf e}_{(#1)}}}  % eigenvec transposed
\newcommand{\derF}[1]{{DF |_{#1}}}        % Gibson stability matrix
\newcommand{\jMps}{\ensuremath{J}}   % fundamental matrix, phase space
% \newcommand{\jMps}{\ensuremath{{\bf J}}}  % bold fundamental matrix phase space
\newcommand{\derf}[2]{\ensuremath{{J}^{#1}(#2)}}    % Predrag fundamental matrix
% \newcommand{\derf}[2]{\ensuremath{{\bf J}^{#1}(#2)}}  % Predrag bold fundamental matrix
% \newcommand{\derf}[2]{{Df^{#1}|_{#2}}}   % Gibson fundamental matrix
\newcommand{\ExpaEig}{\Lambda}
\newcommand{\PredragsGarlic}{e}
\newcommand\Lyap{\lambda}                       %Lyapunov exponent
\newcommand{\eigenvL}{{s}}
\newcommand{\eigenvG}{{m}}         % compact group eigenvalues

%%       optional parameter comes in [\ldots], for example
%%       \newcommand\eigRe[1][ ]{\ensuremath{\mu_{#1}}}
%%       no subscript: \eigRe\
%%       with subscript j: \eigRe[j]
%%
%%      Guckenheimer-Holmes:  lambda = alpha + i beta
%%      Hirsch-Smale:         lambda = a     + i b
%%      Boyce-di Prima:       lambda = mu    + i nu
%%      Gibson:        lambda = mu    + i omega (best of the bunch!)
%
% Re eigen-exponent superscripting
% Getting into the ChaosBookie groove... awesome!
% The groove is groovy when the macros reduce typing...

\newcommand{\eigExp}[1][]{
\ifthenelse{\equal{#1}{}}{\ensuremath{\lambda}}{\ensuremath{\lambda^{(#1)}}}
                        }
\newcommand{\eigRe}[1][]{
\ifthenelse{\equal{#1}{}}{\ensuremath{\mu}}{\ensuremath{\mu^{(#1)}}}
                        }
\newcommand{\eigIm}[1][]{
  \ifthenelse{\equal{#1}{}}{\ensuremath{\omega}}{\ensuremath{\omega^{(#1)}}}
            }

\newcommand{\tny}[1]{{\text{\tiny {#1}}}}

% Guck & Holmes use $W^s$, $W^u$ for stable, unstable manifolds.
% usage: \Wmnfld{u (n)}{NB} unstable manifold of NB's nth eigenvalue.

\newcommand{\Wmnfld}[2]{%
\ifthenelse{\equal{#2}{}}{\ensuremath{W_{#1}}\!}
                         {\ensuremath{W^{#1}_{\text{\tiny #2}}}\!} %Negative space is screwing up spacing in text
                        }

\newcommand{\ben}[1]{{\be}_{#1}}
\newcommand{\beUBg}[1]{\ensuremath{\be_{#1}}}
\newcommand{\beUBl}[1]{\ensuremath{\be_{#1}^{\lambda}}}
\newcommand{\sspn}[1]{a_{#1}}               % state space point
\newcommand{\sspg}[1]{\ssp_{#1}}        % state space point global
\newcommand{\sspl}[1]{\ssp_{#1}^{\lambda}}  % state space point local

\ifblog
%   \FIG{#1}    % \includegraphics[width=0.40\textwidth]{../figs/f_name.ps}
%   {#2}    % short caption text
%   {#3}    % full caption text
%   {#4}    % f-figure-label
%       defined here:
\newcommand{\FIG}[4]{\begin{figure}
              \hspace*{0.10\textwidth}%
              \begin{minipage}[b]{1.00\textwidth}
              \noindent{#1}
              %\centering{#1}
                      \caption[#2]{#3}
                      \label{#4}
              \end{minipage}
              \end{figure} }

%  \SFIG{#1}    % f_name.eps
%       {#2}    % short caption text
%       {#3}    % full caption text
%       {#4}    % f-figure-label
\newcommand{\SFIG}[4]{\begin{figure}
              %\hspace*{-0.10\textwidth}
              %\hspace*{0.05\textwidth}
              \centering
              \begin{minipage}[b]{0.55\textwidth}
                      \caption[#2]{#3}
                      \label{#4}
              \end{minipage}~~~~~%
              \begin{minipage}[b]{0.40\textwidth}
                      \includegraphics[width=1.00\textwidth]{#1}
              \end{minipage}
              %\hfill
              \end{figure} }


\newenvironment{offset}
               {\list{}{\listparindent 3em%
                        \advance\rightmargin -3em}%
                \item\relax}
               {\endlist}
\newtheorem{exmple}{\noindent\small\textsf{\textbf{Example}}}[chapter]
\newcommand{\example}[2]{
    \vskip -13mm
        \begin{offset}
        \begin{exmple}
           \noindent\small
           \textsf{\textbf{#1}} ~
       \slshape\sffamily{#2}
       % \textsl would not work...
    \end{exmple}
    \end{offset}
        \vskip -1mm
             }

\newcommand{\Remarks}{
        \noindent{\textsf{\large\textbf{Commentary}}}

        %\section*{\textsf{\textbf{Commentary}}}
        %\section*{Historical remarks}
        \addcontentsline{toc}{subsection}{{~~~~Historical remarks}}
                        }
\newtheorem{rmark}{{\small\textsf{\textbf{Remark}}}}[chapter]
\newcommand{\remark}[2]{
        % \begin{quotation}
        \begin{rmark}
        {\small\em\noindent {\small\sf \underline{ #1} ~} #2 }
    \end{rmark}
    % \end{quotation}
              }

\newtheorem{exerc}{\textsf{\textbf{Exercise}}}[chapter]
% \newtheorem{exerc}{}[chapter]
% \newtheorem{exerc}{{$\bullet$}}[chapter]
 \newcommand{\exercise}[2]{
%       \vskip -13mm
         \noindent
         \begin{exerc}{
\renewcommand{\theenumi}{\alph{enumi}}
\renewcommand{\labelenumi}{\textbf{(\alph{enumi})\ }}
    {\noindent\small
         ~~\textsf{\textbf{#1}} ~
           \slshape\sffamily{#2}  } % \textsl would not work...
    }
         \vskip -1mm
% removed the line: % \noindent\rule[.1mm]{\linewidth}{.5mm}
         \end{exerc}
                          }
\newcommand{\solution}[3]{
%         \vskip -4mm
        {\noindent\small
         \textsf{\textbf{Solution \ref{#1}~-~#2}} %LABEL - TITLE
         \slshape\sffamily{#3}                    %TEXT
         }
         \vskip  1ex  %4mm
% removed the line: % \noindent\rule[.1mm]{\linewidth}{.5mm}

\fi % end of blog switch

%%%% TEMPORARY, ELIMINATE EVENTUALLY                    }
%\newcommand{\EQ}[1]{\ensuremath{\bu_{\text{\tiny #1}}}}


%%%%%%%%%%%%%%% ASH'S JFM-FRIENDLY MACROS %%%%%%%%%%%%

%\newcommand{\eg}{{e.g.\ }}
%\newcommand{\etc}{{etc.\ }}
%\newcommand{\etal}{\mbox{\it et al.\ }}
\newcommand{\Rey}{\mbox{\it Re}}
%\newcommand{\Rm}{\Rey_m}
%\newcommand{\Pm}{P_m}
%\newcommand{\upartial}{\partial}
%\newcommand{\upi}{\pi}
\renewcommand{\bnabla}{\mbox{\boldmath $\nabla$}}
\renewcommand{\vec}[1]{\mbox{\boldmath $#1$}}
\newcommand{\vechat}[1]{{\skew3\hat{\vec{#1}}}}
\newcommand{\mat}[1]{\mbox{\sls #1}}
