% siminos/atlas/cut.tex  pdflatex atlas
% $Author$ $Date$

\section{\Statesp\ visualization}

\subsection{Group orbit}
    \PublicPrivate{}{
%%%%%%%%%%%%%%%%%%%%%%%%%%%%%%%%%%%%%%%%%%%%%%%%%%%%%%%%%%%%%%%%%%%%%
\begin{figure}
   \centering
   %\includegraphics[width=0.45\textwidth]{???}
   \caption{\label{fig:tangents}
3 tangents: one $\vel(\ssp)$  and two group tangents
$\groupTan(\ssp)^{(1)}$, $\groupTan(\ssp)^{(2)}$.
}
\end{figure}
%%%%%%%%%%%%%%%%%%%%%%%%%%%%%%%%%%%%%%%%%%%%%%%%%%%%%%%%%%%%%%%%%%%%%

define
\begin{itemize}
  \item dynamical system $\{\pS,f^t,\Group\}$
        vs reduced dynamics $\{\PoincS/\Group,f\}$
  \item \statesp\ vs tangent space, see \reffig{fig:tangents}
  \item trajectory vs orbit
  \item template
        \\
        there is always tension between mathematics - linear problem eigenmodes
        (Fourier for translations and rotations) and physics - the fact that
        nonlinear dynamics states are far away from such axes, as they
        always involve a number of such linear modes strongly entangled.
  \item section vs slice
  \item
\end{itemize}
will explain later on
    }

The \emph{group orbit} $\pS_\ssp $ of a \statesp\ point $\ssp \in \pS$ is
traced out by the set of all group actions
\beq
\pS_\ssp = \{\LieEl\,\ssp \mid \LieEl \in {\Group}\}
% \,,\qquad \pS_\ssp \subset \pS
\,.
\ee{sspOrbit}
Any state in the  group orbit set $\pS_{\ssp}$
is physically equivalent to any other. The action of a symmetry group
thus foliates the \statesp\ into a union of group orbits,
\reffig{fig:BeThTraj}\,(a).

\subsection{\cLe}
\subsection{Ring of Fire}
\subsection{Experimentalist description: a video 1D to 3D arrays of pixels}
\subsection{Theorist description: $\infty$-\dmn\ \statesp}
\subsection{Time orbit: point is a point, line is a line in all dimensions}
\label{sect:TimeOrb}

\subsection{Physical dimension: covariant Lyapunov vectors}

\section{Poincar\'e sections, R\"ossler}
\label{s:cut}

In the \mslices\ the symmetry reduction is achieved by cutting the group
orbits with a finite set of hyperplanes, one for each continuous group
parameter, with each group orbit of
symmetry-equivalent points represented by a single point, its
intersection with the \slice.


\subsection{Local chart}
After some experimentation and observations of turbulence in a given
flow, one can identify a set of dynamically important unstable
{\recurrStr s}.

\subsection{R\"ossler {\poincBord}}
\subsection{R\"ossler two-chart atlas}
\subsection{R\"ossler unstable manifold curvilinear distance}
\subsection{R\"ossler return map}
\subsection{$N$-chart atlas, forward maps}
\subsection{Ring of Fire return map\rf{lanCvit07}}
