% siminos/blog/rpo.tex
% $Author: predrag $ $Date: 2007-06-13 18:28:58 +0100 (Wed, 13 Jun 2007) $

\documentclass[letter,10pt]{article}

\title{\KS\ blog: \rpo s}
\author{Evangelos Siminos}
    % and Predrag Cvitanovi\'{c}, }

\input ../inputs/setupBlog     % all bloggy usepackage, formating, ...
\input ../inputs/def           % do not edit this file
\input ../inputs/defsBlog      % bloggy macros


\begin{document}

\maketitle

\setcounter{page}{1}
\tableofcontents
\newpage

Throughout:  {\textdollar} on the margin
{\steady}
indicates that the text has been transferred to
article siminos/rpo\_ks/ .

% Enter details, like dates in/out, dated comments into \subsubsection{--project--}
% enter live PDF links to blogs etc where ongoing work is described or forgotten
% continuously reorder items by priority


\input strategy
\input bluesky

\newpage

\section{May 17, 2007 -- $s_1$, $s_2$, $\tau_x$, $\tau_z$ generate 16 irreps?}

{\bf PC}: My main problem is - we are currently using only 4 irreducible reps
of $C_2 \times C_2$ = $D_2$ dihedral group generated by $\tau_x, \tau_z$,
but why not 16 irreps of
the $D_2 \times D_2$ generated by $s_1, s_2, \tau_x, \tau_z$?
They all commute, each one splits the space of reps into 2.
Why stop at $U_S$ subspace?
There should be 16 discrete copies of any
general solution, not just 4.
However, there would still be 4 copies of UB, as UB is within the
fully symmetric irrep $A_1$ od $s_1, s_2$ $D_4$.

\section{Jun 13, 2007 Reading for Ruslan?}

Armbruster {\em et. al} showed that four complex Fourier
modes suffice to exhibit most
of the qualitative features of the dynamics,
for a wide range of system sizes\rf{AGHks89}.

\section{Symmetry-Reduced Representation (SRR) for KSE}

\subsection{RLD Jan 17, 2008 -- How to quotient the SO(2) symmetry}

In order to quotient the SO(2) symmetry we need to be able to define,
for any state of the KS system $u(x)$
($a = (a_1, a_2, \ldots)^\mathsf{T}$ in Fourier space),
a 'shift' parameter, $s(a) \in S^1 $, such that
\[ \tau_{-s(a)} a \in M/\mathrm{SO}(2). \]
This parameter must satisfy the following {\em monotonicity condition} with
respect to the translation of the state $a$:
\[ s(\tau_{\ell/L}a) = s(a) + \phi(\ell/L) \]
where $\phi(x): S^1 \mapsto S^1$ should be a continuous strictly monotonic function.
This condition is necessary to avoid any ambiguity in the definition of
the shift parameter.

This condition is clearly satisfied when $s(a)$ is proportional to
the first Fourier mode (provided that $|a_1| > 0$)
\begin{equation}
  s(a) =  \theta_1/(2\pi) = \arg(\hat{e}_1^\dagger\,a)/(2\pi)
\label{eq:shift1} \end{equation}
where $\hat{e}_1 = (1+0i, 0, 0, \ldots)^\mathsf{T}$ is the basis vector corresponding
to the first Fourier mode and $\dagger$ denotes Hermitian transpose.
In this case
\[ \arg (e_1^\dagger\,\tau_{\ell/L}a)/(2\pi) = \theta_1/(2\pi) + \ell/L\,, \]
and so $\phi(x) = x$.

It is also possible to get a well-defined shift parameter by
using the difference between phases of Fourier modes $k$ and $k+1$ (provided
that $|a_k|, |a_{k+1}| > 0$):
\begin{equation}
  s(a) = \arg(\hat{e}_{k+1}^\dagger\,a)/(2\pi) -
  \arg(\hat{e}_{k}^\dagger\,a)/(2\pi)\,.
  \label{eq:shiftk} \end{equation}
Maybe for $L = 22$, where dominant Fourier modes are 2 and 3, it is better to use
this shift parameter with $k = 2$?  This needs to be explored.

Of course, as suggested by Predrag, we can define in a similar
fashion the shift parameter with respect to any other
state (e.g. an equilibrium state $a_q$):
\[ s(a) = \arg(a_q^\dagger\, a)/(2\pi)\,, \]
but this shift cannot be guaranteed to satisfy the
monotonicity condition for all $a$.

Since equilibria E1, E2, and E3 for $L = 22$ have dominant
1st, 2nd, and 3rd Fourier modes, respectively, fixing the modes by
Eqs.~(\ref{eq:shift1}) or (\ref{eq:shiftk}) also fixes the equilibria.


\subsection{PC Jan 17, 2008 -- Monotonicity?}

Not sure about need for monotonicity - one needs it for the 1-dimensional
Lie group of time evolution, parameterized by a continuous parameter $t$
which can be conjugated to any other parameter $u = u(t,\ssp)$ as long
as it is monotone, but rotations can go whichever way they want, modulo
$2\pi$ (or $L$). Once we look at a problem with $SO(3)$ symmetry, what's the
need for monotonicity in Euler angles, \etc.?







\input flotsam
\input bronski-2005

\bibliographystyle{plain}
\bibliography{../bibtex/siminos}
\end{document}
