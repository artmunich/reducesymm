% siminos/blog/exp.tex
% $Author$ $Date$

\chapter{Symmetry reduction of experimental data}
\label{c-exp}

\begin{description}

\item[2011-08-12 Predrag] Woods Hole Geophysical Fluid Dynamics Program
report: After deriving in
\HREF{http://chaosbook.org/overheads/continuous/index.html}{my seminar}
the \mslices, I attempted to talk experimentalists
\HREF{http://www.mcnd.manchester.ac.uk/mullin.html}{Tom Mullin} and
\HREF{https://www.irphe.fr/~legal/}{Patrice Le Gal} (studies instabilities
and transitions of rotar-stator flows) into implementing numerical the
`nearest group orbit point' distance for experimental video images of
flows. In fluids community everybody obsesses about `norms', especially
the control engineers who have to optimize perturbations in order to
attain desired ends. Collecting the full 3D velocity vector field information is
not feasible, but as the inertial manifold for transitional boundary-shear flows
is only 100 to 1,000-dimensional, most projections from 100,000 to
1,000,000-dimensional \statesp s will capture the topology and correctly
identify the nearest neighbors. People agree that there is no compelling
reason to use the energy norm, other than that velocity fields is what is
given in a numerical computation. In practice people use ``dissipation
norms'', ``compensatory norms'', \etc, \etc.

My first thought has been to define a Gaussian-smeared out norm to
compare Hamming (pixel by pixel) distance between two images, smearing
being necessary to account for the sensor noise. Problem is that for
nonlinear flows this noise should not be isotropic\rf{LipCvi08,LipCvi07}
- it matters whether points compared are on unstable or stable manifolds,
etc. But as this is a pattern recognition problem, why not use the
accumulated wisdom of the pattern compression and pattern recognition
community?

So here is my proposal: jpeg and gif algorithms are optimized to reduce
the information content of an image. Optimization is for human vision,
but that might be OK for our purposes. So use not the raw, but the jpeg
compressed images, their [pixel$\times$pixel] size reduced to the
resolution realistically needed, each pixel a coordinate in the \statesp\
(if the proposal works, one would actually go into the jpeg algorithm and
start tweaking its parameters).
The group orbit is traced in this \statesp\ by translating a given image.
For example, in Le Gal's French washing machine experiments, or Mullin's
expanding pipe ones, the group orbit is traced out by rotating a given
image around the rotation axis, and the group tangent at a given image is
obtained by a minute angular rotation of a pattern. Pick a state of the
fluid that seems important and typical of the flow studies, and us it a
template. Its group tangent defines the slice hyperplane, and its group
curvature (the second derivative of the group orbit with respect to the
angle) both defines the \sset\ (the outer edges of the neighborhood that can be
associated with a given template), and an optimal coordinate to project
the group orbit and visualize it in a 2\dmn\ projection. The dynamics in
the slice is determined by step-by-step post-processing, \ie, by finding
the angle that satisfies the slice-group tangent orthogonality, and
rotating the full \statesp\ trajectory into the slice. This symmetry reduction is
lossless: it \emph{loses no information}, other than what was lost in
going from the raw image to the jpeg compressed image.

Reduced dynamics can be now visualized in two ways:
\begin{enumerate}
    \item Make a video from the slice reduced video frames; in such video a radially
symmetric spiral
pattern becomes stationary \reqv\ (again, no information lost - \emph{this is
not averaging over the azimuthal angle}), and
turbulence reduces to turbulence with azimuthal drifts removed.

    \item Project the \statesp\ trajectory on 2-3 coordinates formed from
dynamically important states of the fluid\rf{GHCW07}. There things like
bifurcation of a spiral in two spirals, or bifurcation of a torus into
two tori will be reduced to the garden variety dynamical systems
bifurcations.

    \item More importantly, experimental dynamics will reveal swirls and
close passages to invariant solutions that cannot be discerned from the
unreduced flow.

    \item The connection to numerics is now made by generating jpegs of
numerical flows visualized from the same vantage point as that of the camera
shooting the experimental flow. Once a nearby image is identified in the
numerical flow, one can run numerics to predict the future evolution, and
compare it with experimental flow, using Kalman filter data assimilation
to keep the numerical prediction close to the observed state of the
fluid.
\end{enumerate}

\end{description}
