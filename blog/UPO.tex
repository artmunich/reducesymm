% siminos/blog/UPO.tex
% $Author$ $Date$

\chapter{Periodic orbit theory}
\label{chap:UPO}

This part of the blog collects references to periodic orbit theory.
Some of the articles are in the CNS Zotero collection.

As material is written up, parts of it will migrate from its
current placement into coherent sections, suitable for
inclusion into ChaosBook.org, or, God Forbid, actual {\em
publications}.

\begin{description}

\item[2011-05-15 Predrag]
\HREF{PubMed.gov}{PubMed.gov} is scary. Find a paper you
want, click on "Related citations" on the right, and you get more stuff than you
can ever digest... Zotero helps you generate (imperfect) BibTeX entries.

\item[2011-05-15 Predrag]
Creagh\rf{Creagh94},
{\em Quantum zeta function for perturbed cat maps}, says: ``
The behavior of semiclassical approximations to the spectra of perturbed
quantum cat maps is examined as the perturbation parameter brings the
corresponding classical system into the nonhyperbolic regime. The
approximations are initially accurate but large errors are found to
appear in the traces and in the coefficients of the characteristic
polynomial after nonhyperbolic structures appear. Nevertheless, the
eigenvalues obtained from them remain accurate up to large perturbations.
''
(see CNS Zotero collection)

\item[2011-05-15 Predrag]
Wisniacki \etal\rf{WiVeBeBo04},
{\em Classical invariants and the quantization of chaotic systems}, says: ``
Due to their exponential proliferation, long periodic orbits constitute a
serious drawback in Gutzwiller's theory of chaotic systems. Therefore, it
would be desirable that other classical invariants, not suffering from
the same problem, could be used in alternative semiclassical quantization
schemes. In this Rapid Communication, we demonstrate how a suitable
dynamical analysis of chaotic quantum spectra unveils the role played, in
this respect, by classical invariant areas related to the stable and
unstable manifolds of short periodic orbits.
''

\item[2011-05-15 Predrag]
Heusler \etal\rf{HeMuAlBrHa07},
{\em Periodic-orbit theory of level correlations}, says: ``
We present a semiclassical explanation of the so-called
Bohigas-Giannoni-Schmit conjecture which asserts universality of spectral
fluctuations in chaotic dynamics. We work with a generating function
whose semiclassical limit is determined by quadruplets of sets of
periodic orbits. The asymptotic expansions of both the nonoscillatory and
the oscillatory part of the universal spectral correlator are obtained.
Borel summation of the series reproduces the exact correlator of
random-matrix theory.
''

\item[2011-05-15 Predrag]
Ando \etal\rf{AnBoAi07},
{\em Automatic control and tracking of periodic orbits in chaotic systems}, says: ``
Based on an automatic feedback adjustment of an additional parameter of a
dynamical system, we propose a strategy for controlling periodic orbits
of desired periods in chaotic dynamics and tracking them toward the set
of unstable periodic orbits embedded within the original chaotic
attractor. The method does not require information on the system to be
controlled, nor on any reference states for the targets, and it overcomes
some of the difficulties encountered by other techniques. Assessments of
the method's effectiveness and robustness are given by means of the
application of the technique to the stabilization of unstable periodic
orbits in both discrete- and continuous-time systems.
''

\item[2011-05-15 Predrag]
Brack \etal\rf{BrOgYuRe05},
{\em Uniform semiclassical trace formula for
U(3) $\to$ SO(3) symmetry breaking}, says: ``
We develop a uniform semiclassical trace formula for the density of
states of a three-dimensional isotropic harmonic oscillator (HO),
perturbed by a term which term breaks the U (3) symmetry of the HO,
resulting in a spherical system with SO (3) symmetry.
We obtain an analytical uniform trace formula
which in the limit of strong perturbations (or high energy)
asymptotically goes over into the correct trace formula of the full
anharmonic system with SO (3) symmetry, and in the other limit
restores the HO trace formula with U (3) symmetry.
''

This might be useful to us in leading us to physically interesting systems
with non-abelian symmetries. However, have no time today to continue...
%
%Bengtsson I, Br�annlund J and � Zyczkowski K 2002 Int. J. Mod. Phys. A 17 4675
%give example of

\item[2011-05-16 Predrag]
Mark Srednicki has several interesting recent papers on Riemann zeros.
In {\em Nonclasssical Degrees of Freedom in the Riemann Hamiltonian},
\arXiv{1105.2342},
he says: ``
The Hilbert-Polya conjecture states that the imaginary parts of the zeros of
the Riemann zeta function are eigenvalues of a quantum hamiltonian. If so,
conjectures by Katz and Sarnak put this hamiltonian in Altland and Zirnbauer's
universality class C. This implies that the system must have a nonclassical
two-valued degree of freedom. In such a system, the dominant primitive periodic
orbits contribute to the density of states with a phase factor of -1, which
partially resolves a previously mysterious sign problem for oscillatory
contributions to the density of the Riemann zeros.
''

\item[2011-05-16 Predrag] I believe that Ruelle's linear response
theory is fundamentally wrong, as deterministic diffusion transport
coefficients are nowhere differentiable.
Lucarini \etal\ do like it. In
{\em Relevance of sampling schemes in light of Ruelle's linear response
 theory}, \arXiv{1105.2527}, they say: ``
We reconsider the theory of the linear response of non-equilibrium steady
states to perturbations. We first show that by using a general functional
decomposition for space-time dependent forcings, we can define elementary
susceptibilities that allow to construct the response of the system to general
perturbations. Starting from the definition of SRB measure, we then study the
consequence of taking different sampling schemes for analyzing the response of
the system. We show that only a specific choice of the time horizon for
evaluating the response of the system to a general time-dependent perturbation
allows to obtain the formula first presented by Ruelle. We also discuss the
special case of periodic perturbations, showing that when they are taken into
consideration the sampling can be fine-tuned to make the definition of the
correct time horizon immaterial. Finally, we discuss the implications of our
results in terms of strategies for analyzing the outputs of numerical
experiments by providing a critical review of a formula proposed by Reick.
''

\item[2011-07-22 Predrag]
Gao, Xie and Lan\rf{GaXiLa11}
accelerate convergence of cycle expansions by dynamical conjugacies.

The key idea of this paper, of replacing the stability of an unstable
fixed-point or periodic orbit that a critical point is preperiodic to, by
the root corresponding to the order of the critical point was developed
in detail in a careful study of convergence by Artuso \etal\rf{AACII},
where it is shown that the change in convergence is due to a single fixed
point whose preimage is the critical point. It is shown how to modify the
cycle expansion to fix the convergence. The original $1/\zeta$ is kept,
but the pole induced by the critical point singularity is explicitly
factored out. The method is essentially a quadratic conjugacy restricted
to the critical point (Ulam map to tent map being the trivial example).

The innovation of this paper that goes beyond \refref{AACII} is the
explicit study of natural measures of such maps and use of conjugacies to
excise the singularities in the Ulam map (and its Misiurowicz family
generalization) settings. While \refref{AACII} motivates excision of the
singularity by a detailed study of many families of periodic orbits, the
authors accomplish this more elegantly, by a simple, well designed
conjugacy.

They are looking at the series of generalized Ulam maps, or what Ruelle
calls ``Misiurewicz maps'', where the critical point is preperiodic to,
ie  mapped onto an unstable cycle and thus rendered non-contracting.
Period 2 below is the next in sequence. The interesting new one is the
``Golden mean'' map, see for example exercise 11.6 in ChaosBook.org.
There the critical point is a part of the 3-cycle, so you know (at least
numerically) where the 3 measure singularities are. The answer is
somewhere in the literature. Perhaps in L.~Billings and E.~M.~Bollt,
``Invariant densities for skew tent maps,'' Chaos Solitons and Fractals
12, 365 (2001).

The step in $\rho(x)$ at $x_f$ is suspicious - perhaps the problem is that
the prefactor of the $1/\sqrt{|x-x_f|}$ singularity is different on the two
sides of $x_f$. How can a half-singularity at $x=0$ map into both sides of
$x_f$ neighborhood? Presumably the conjugacy (which they do not explain)
should be piecewise analytic, not smooth as in their figures.

Their $g(x)$ is a bad news - they seem to have introduced infinite slope
at an arbitrary point of the map. It's probably an artifact of the
unexplained method for constructing conjugacies - nothing interesting
happens here in the original dynamics. If they can argue that any periodic
orbit that includes critical point has this problem, that is interesting.
Again, the good conjugacy is probably piecewise analytic - the natural
measure they get has worrisome steps.

Here (and in all finite grammar cases they study) working out the
symbolic dynamics and Markov graphs of this map would help - they have to
understand which cycles form the fundamental set and which (families of)
shadowed cycles are causing wild oscillations in their figures, before
embarking on constructing a conjugacy. Original map has no cycles of
infinite instability, so their subsequent troubles presumably come from a
badly chosen conjugacy.

There can no be analytic conjugacy in higher dimensions - measure
singularities always sit on fractal sets, just observe pictures of
natural measure on the H\'enon attractor. Still, if you one find a
conjugacy that excises the neighborhood of the nearly-attracting
13-cycle, that would deal with the main impediment to zeta function
convergence in this case. The method cannot be generalized to higher
dimensions. For private amusement, just try constructing a 2-d conjugacy
for something like a H\'enon map $\to$ Lozi map. Good luck.

There is immense literature on measures of 1-d maps (A. Boyarski? A.
Lasota and M.C. Mackey\rf{LM94}? G. Froyland? E. Bollt? J. M.
Aguirregabiria, Robust chaos with prescribed natural invariant measure
and Lyapunov exponent; \arXiv{0907.3790}? D.J. Driebe, Fully Chaotic Map
and Broken Time Symmetry (Kluwer, 1999)? ... and authors would profit
from using some of that work to illustrate their ideas. They should do a
literature search on measures of 1-d maps. This is reminiscent of work on
natural measures published since 1980's by Hungarian school (Szapfalusy,
Tel, ...), Bollt, and many others.

1-d maps with a single critical point are very special, and unfortunately
little of this is useful in higher dimensions - already for the H\'enon
attractor there is a fractal set of critical points (ie, stable-unstable
manifold tangencies) and their images. No conjugacy or a finite set of
conjugacies can help there...


\item[2011-07-26 Predrag]
Katsanikas, Patsis and Pinotsis% \etal\rf{XXX},
{\em Chains of rotational tori and filamentary structures close to
  high multiplicity periodic orbits in a 3D galactic potential},
  \arXiv{1103.3981}, might be of interest - it struggles with visualization
of tori in 4\dmn.

\item[2011-08-04 Predrag]                                       \toCB
Qihuai Liu and Dingbian Qian% \etal\rf{XXX},
{\em Modulated amplitude waves with nonzero phases in
     Bose-Einstein condensates},
\arXiv{1103.5277}, is unintelligible to me, but I would like to
understand what is the relation of MAWs to \reqva, \etc, so this is a
start. They say:

  ``
We consider uniformly propagating coherent structures of form
ansatz
\beq\label{LiQi11:2}
\psi(t,x)=R(x)\exp(i[\Theta(x)-\mu t]),
\eeq
where $R(x)\in\mathbb{R}$ gives the amplitude dynamics of the condensate
wave function, $\theta(x)$ determines the phase dynamics, and the
``chemical potential'' $\mu$, defined as the energy which takes to add
one more particle to the system, is proportional to the number of atoms
trapped in the condensate. When the (temporally periodic) coherent
structure is also spatially periodic, it is called a modulated amplitude
wave (MAW).
% \cite{brusch2000modulated,brusch2001modulated}
Similarly, a solution of the equation with the (temporally periodic)
coherent structure is called a quasi-periodic modulated amplitude wave
(QMAW) if it is also spatially quasi-periodic. If the phase of the
condensate wave function is trivial the MAW is a standing wave. Even the
amplitude $R(x)$ is $L$-periodic, the corresponding condensate wave
function $\psi(x,t)$ may be not periodic with respect to the spatial
variable $x$. If $2\pi/\nu$ and $L$ are rationally related, then
$\psi(x, t)$ is a MAW; otherwise $2\pi/\nu$ and $L$ are incommensurate,
and $\psi(x, t)$ is not periodic but quasi-periodic QMAW with the
frequency $\omega = \langle 2\pi/\nu,L\rangle$.
  ''

They copy their definitions from L. Brusch, M. G. Zimmermann, M. Van
Hecke, M. B\"ar and A. Torcini, ``Modulated amplitude waves and the
transition from phase to defect chaos'', {\em Phys. Rev. Lett. \bf 85},
86--89 (2000); L. Brusch, A. Torcini, M. Van Hecke, M. G. Zimmermann and
M. B\"ar, ``Modulated amplitude waves and defect formation in the
one-dimensional complex Ginzburg-Landau equation'', Physica D 160,
127--148 (2001).

\item[2011-08-24 Predrag]                                       \toCB
Liao\rf{Liao09,WaLiLi11,Liao11,Liao11a} is able to integrate Lorenz
equations with the 800-digit precision, using Mathematica with the
400th-order Taylor expansion for continuous functions. He gets ``clean''
numerical simulation of chaotic solution of Lorenz equation in a long
interval $0\leq t \leq 1000$ LTU (Lorenz time unit) with negligible
truncation and round-off error.   He found that,  to gain a solution in
$0\leq t \leq T_c$,  the initial conditions must be at least in the
accuracy of $10^{-2T_c/5}$.   Thus,  when $T_c = 1000$ LTU,  the initial
condition must be  in the accuracy of 400-digit precision at least. The
averaged velocity fluctuation in  a cube meter of fluid is $3.722 \times
10^{-30}$ m/s, so he starts simulations with initial standard deviation
$\sigma=10^{-30}$ (not clear why Lorenz origins in fluid dynamics justify
precisely $\sigma=10^{-30}$, but never mid) and integrate
deterministically (that is wrong too - integration should be
Fokker-Planck, not deterministic). The effects kicks in at 80 LTU, and by
120 LTU initial conditions are forgotten. ``This strongly suggests that
chaos builds a bridge from the micro-level uncertainty to   macroscopic
randomness, and  thus  is an origin of macroscopic randomness and  time's
arrow.''

\item[2011-09-15 Predrag]
Gutkin and Osipov, %\rf{XXX},
\arXiv{1109.3329}, say: ``
By considering symbolic dynamics of the system one can introduce a
natural ultrametric distance between periodic orbits and organize them
into clusters. Each cluster consists of orbits approaching closely each
other in the phase space. We study the distribution of cluster sizes for
the baker's map in the asymptotic limit of long trajectories. This
problem is equivalent to the one of counting degeneracies in the length
spectrum of the {\it de Bruijn} graphs. Based on this fact, we derive the
probability $\P_k$ that $k$ randomly chosen periodic orbits belong to the
same cluster, find asymptotic behaviour of the largest
cluster size $|Cl_{\max}|$ and derive the probability $P(t)$ that a
random periodic orbit belongs to a cluster of the size smaller than
$t|Cl_{\max}|$, $t\in[0,1]$.
''

\item[2011-10-09 Predrag]
Klebanoff and E. Bollt\rf{KlBo011} \emph{Convergence analysis of
{Davidchack and Lai's} algorithm for finding periodic orbits} might be
worth a read. Here are also two Guckenheimer\rf{ChoGuck99,GM00aut} on
computing periodic orbits.

This might be of interest to Chandrites: Olvera and Vargas\rf{OlVa94},
\emph{A continuation method to study periodic orbits of the Froeschl\'e
map} write: `` The dynamics of many Hamiltonian systems with three
degrees of freedom is represented by the Froeschl\'e map which is
symplectic and four-dimensional. In this paper we study sequences of
periodic orbits approaching the invariant tori.''

%\item[2011-XX-XX Predrag]
%XXX \etal\rf{XXX},
%{\em XXX}, says: ``
%XXX
%''
% (see CNS Zotero collection)


\end{description}
