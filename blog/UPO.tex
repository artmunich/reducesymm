% siminos/blog/UPO.tex
% $Author$ $Date$

\chapter{Periodic orbit theory}
\label{chap:UPO}

%\begin{center}
%{\Huge INTERNAL, NOT FOR DISTRIBUTION}
%\end{center}

This part of the blog collects references to periodic orbit theory.
Some of the articles are in the CNS Zotero collection.

As material is written up, parts of it will migrate from its
current placement into coherent sections, suitable for
inclusion into ChaosBook.org, or, God Forbid, actual {\em
publications}.


\section{Symmetries imply possible existence of \rpo s}
\label{sec:SymRPO}
% Ruslan edited                                 2007-03-11
% Predrag  eliminated siminos/blog/symRLD.tex   2009-10-09

In a dynamical system $\dot{a} = v(a)$ with a strange
invariant set, there exist infinitely many periodic orbits
\[ f^\period{}(a) = a \]
characterized by period $\period{}$, which are dense within
the invariant set. Here $f^t$ is the flow map of the flow
$v$, \ie, $a(t) = f^t(a)$ is the solution of the flow $v$
with initial condition $a(0) = a$.

Let the dynamical system have symmetries represented by the operators
$\Sigma_{k,s}$, where $k \in {\cal K} \subset \mathbb{Z}^p$ are
parameters of discrete symmetries and $s \in {\cal S} \subset \mathbb{R}^q$
are parameters of continuous symmetries.  In other words,
\[ f^t(\Sigma_{k,s} a) = \Sigma_{k,s} f^t(a)\,. \]
In this case it is likely that, in addition to \po s, the dynamical system also
has \rpo s, characterized by the condition
\[ \Sigma_{k,s}f^\period{}(a) = a\,, \]
where, in addition to the period $\period{}$, the \rpo\ is also characterized by
the symmetry parameters $k$ and/or $s$.

In fact, if the symmetry is continuous, then it is much more likely
to find \rpo s, than it is to find exact \po s, since $s = 0$,
corresponding to the \po , is only one specific value in the
continuum of possible values of parameter $s$.

In the case of KS equation, which has continuous symmetry
$\Shift_{\shift/L}$ and discrete symmetry $\Refl$, it is possible to
find \rpo s that satisfy one of the following conditions
\[
  \Shift_{\shift/L}f^\period{}(a) = a\,,
\quad\mbox{or}\quad
  \Refl\Shift_{\shift/L}f^\period{}(a) = a\,.
\]
The first condition is satisfied by \rpo s with shift $\shift$,
while \rpo s that satisfy the second condition are exactly periodic
($\shift = 0$) with period $2\period{}$.

\section{Complex Ginzburg-Landau}
\label{sec:CGL}
% Predrag  collected notes on CGLe here  2012-01-27

\begin{description}

\item[{\Rpo s} of the complex Ginzburg-Landau]
V. L{\'o}pez\rf{lop05rel}, as far as the discussion about ``drift" is
concerned: In presence of continuous symmetries (for PC, streamwise and
spanwise translations) one should be searching for {\reqva} and {\rpo s}
- they are most likely more important than the \eqva\ and spatially
periodic orbits.

                                        \toCB
I like {\rpo s} paper\rf{lop05rel} by Vanessa L{\'o}pez %\rf{lopezLink},
%        http://www.cse.uiuc.edu/$\sim$vlopez:
``\emph{Relative periodic solutions of the complex Ginzburg-Landau equation}.''
There is much clear math text in it that we should reread when editing the
\rpo s chapter of ChaosBook. They write:

% [From siads\@siam.org  Dec 22 2004]:
% willing to review this manuscript for Dwight Barkley.

A method of finding {\rpo s} for differential equations with continuous
symmetries is described and its utility demonstrated by computing {\rpo
s} for the one-dimensional complex Ginzburg-Landau equation (CGLE) with
periodic boundary conditions.  A {\rpo} is a solution that is periodic in
time, up to a transformation by an element of the equation's symmetry
group.  With the method used, {\rpo s} are represented by a space-time
Fourier series modified to include the symmetry group element and are
sought as solutions to a system of nonlinear algebraic equations for the
Fourier coefficients, group element, and time period. {\Rpo s} found for
the CGLE exhibit a wide variety of temporal dynamics,
with the % sum of their positive Lyapunov exponents varying from 5.19 to
% 60.35 and their
unstable dimensions from 3 to 8.

\item[2004-09-15 Predrag]
All of your {\Rpo s} have many unstable eigendirections. Do you know that
there exist no other solutions with a lesser number of unstable
directions (let's say only one)?

--vanessa:
I do not know of any proof that shows that there are no solutions with a
lesser number of unstable directions.  I did not find any solutions with
just one or two unstable directions, but at this point I cannot say
whether that is an indication that there are none or simply that
the procedure I used only identified solutions with 3+ unstable
directions.

\item[2004-09-15 Predrag]
Periodic solutions are useful if they are embedded into a chaotic
attractor. Do you have any measure of whether the typical solutions of
CGL, your parameter values, come close to any of the solutions that you
have found? If the periodic orbit is embedded into a chaotic attractor,
typical solution visits it infinitely often, infinitesimally close.

--vanessa:
This is something that I have not examined in detail.  I have observed (using
the ``naked eye") that the time evolution of typical trajectories (when viewed
by plotting the real versus imaginary part of A(x,t) at different times, as in
Figures 3,5,7 from the paper) sometimes resembles that of the first solution
found (i.e., the patterns displayed in Figure 3). (The first solution
found also happened to be the solution to which the solver I used
converged to most frequently.)  But I do not have a
quantitative measure of this ``resemblance".

\item[2004-09-15 Predrag] After talking to Lopez at SIAM
DS05 meeting: I do not think her \rpo s are the dynamically important
ones, so the problem remains wide open.

I read through Vanessa L{\'o}pez's paper\rf{lop05rel} (clearly a summary of a PhD
thesis), and while I a like {\rpo s}, I am very worried about the general
drift of the paper.

As I cannot tell how exhaustive is her set of numerical solutions, I do
not know what to make of them - like Zoldi and Greenside\rf{ZG96}, she
gets all of them with a large number unstable dimensions. Presumably none
of them are close to the asymptotic inertial manifold. More worrisome
still, she uses ZG `escape-time weighting' to ``average" over periodic orbits.

Here are some snippets from \refref{lop05rel}:

We work with the CGLE with cubic nonlinearity in one spatial dimension,
%\end{eqnarray}
\begin{equation}   \label{eq:cgle_pde}
    \frac {\partial A}{\partial t}  =
    R A + (1 + \mathrm{i} \nu) \frac{\partial^2 A}{\partial x^2} - (1 + \mathrm{i} \mu) A|A|^2,
\end{equation}
with periodic boundary conditions
\begin{equation}   \label{eq:cgle_bcs}
    A(x,t)  =  A(x+2\pi,t),
\end{equation}
and, as a convention, when we refer to \textit{the CGLE} we mean
equation~\refeq{eq:cgle_pde} with the boundary conditions
\refeq{eq:cgle_bcs}, unless otherwise noted. Equation~\refeq{eq:cgle_pde}
describes the evolution of a complex-valued field $A(x,t)$.

The CGLE has a three-parameter group ${\mathbb T}^2 \times \mathbb{R}$ of
continuous symmetries generated by space-time translations and a rotation
of the complex field $A$.  That is, if $A(x, t)$ is a solution of the
CGLE, then so is $\mathrm{e}^{\mathrm{i}\eta_1} A(x+\eta_2,  t+\tau)$ for any
element $(\eta_1, \eta_2)$ of the two-torus ${\mathbb T}^2$ and $\tau \in
\reals$.  Thus, we focus our study on time-periodic solutions of the CGLE
relative to the ${\mathbb T}^2$-symmetry, namely, solutions of the CGLE that
satisfy
\begin{equation}  \label{eq:Atpsol}
    A(x,t)  =  \mathrm{e}^{\mathrm{i}\varphi}A(x+S, t+T)
\end{equation}
for some $(\varphi, S) \in {\mathbb T}^2$ and $T > 0$.   Note that after a
period of time $T$, such a solution returns not to itself, but rather to
an element in its ${\mathbb T}^2$-orbit.  Such \emph{relative periodic
solutions} represent invariant three-tori in the CGLE flow.  Previous
studies of time-periodic solutions of the CGLE~\refeq{eq:cgle_pde} have
centered on two types of solutions.  The first are single-frequency
solutions of the form $A(x,t) = B(x) \mathrm{e}^{\mathrm{i}\omega t}$ (see, for
example, \refref{DaVuDel00,doelman89,KaMaPaa96}); these are referred to as
\emph{stationary solutions}.
\toCB

The second type are generalized traveling waves, also called
\emph{coherent structures}, for which $A(x,t) = \rho(x-vt)
\mathrm{e}^{\mathrm{i}\phi(x-vt)} \mathrm{e}^{\mathrm{i}\omega t}$, where $\rho$ and
$\phi$ are real-valued functions and $\omega$ is some frequency (see, for
example, \refref{AKcgl02,BTHmaw01, Mielke02,Mcoh03}).  These coherent
structures reduce to single-frequency solutions by a change to a moving
frame, that is, by a change of variables $x \rightarrow x+vt$, $t
\rightarrow t$.

The relative periodic solutions of interest in this study are different
in that they exhibit more complicated temporal behavior than the
single-frequency solutions just described. Studies of bifurcations to
(stable) invariant two and three-tori for the CGLE are described in
\refref{takac98} and references therein.

The \rpo s we compute, which represent invariant three-tori, are obtained
by working with fixed parameter values for the CGLE, not via a
bifurcation study.

\item[2010-06-02 Predrag] Just heard a talk by
\HREF{http://win.ua.ac.be/~nschloe/}{N. Schl\"omer},
\HREF{http://win.ua.ac.be/~wvroose/}{W. Vanroose}
 and
\HREF{http://personal.maths.surrey.ac.uk/st/D.Avitabile/index.html}{D. Avitabile}
 on
\HREF{http://www.dspdes2010.org/admin/files/fileabstract/A452.pdf}
{Numerical continuation and symmetries} in mesoscopic superconductors.
and
N. Schl\"omer, D. Avitabile and W. Vanroose\rf{SchlAvVa11},
\emph{Numerical bifurcation study of superconducting patterns on a square.}
They fix the overall $U(1)=\SOn{2}$ phase in a BEC problem, do not
reduce the remaining $D_4$ discrete symmetry. Only stable solutions,
we might be able to do this better.
													\toCB
More precisely: numerical methods break the symmetry of any given
problem, and the symmetry-breaking errors can be unpleasantly large. If
symmetry is reduced {\em prior} to numerical work, this does not happen.
Yet another argument why one should always reduce the symmetries prior to
the analysis of any given problem.

\item[2010-06-04 Predrag] to nico.schloemer@ua.ac.be,
Wim.Vanroose@ua.ac.be,
D.Avitabile@bristol.ac.uk, on
\emph{Slice \& dice} $\SOn{2} \times D_4$

Nico et al, I enjoyed your talk. Feel free to ignore the rest of this
email, but if you know literature or have some ideas that might help me
write up methods of symmetry reduction better, let me know.

You probably have not heard Ashley Willis and various spiraling patterns
talks, but in fluid and cardiac dynamics one has to deal with various
(rather simple) continuous and discrete symmetries. All I have found
useful on the subject so far is described "pedagogically" in Chapter 10
Relativity for cyclists (another version is given in Continuous symmetry
reduction and return maps for high-dimensional flows).

I also always quotient discrete symmetries, such as $D_2$ or $D_4$, as
described in Chapter 9 World in a mirror. That kind of thing you
appreciate only after you do it yourself.

As long as you look at a single stable equilibrium or travelling wave,
getting rid of $\SOn{2}$ seems not a problem. It becomes a problem once when
you start looking at sets of unstable solutions.

Visualization of all relative periodic orbits as periodic orbits we
attain only by global symmetry reductions. We do it by "slicing", ie
finding hypersurfaces that cut group orbits just like Poincare sections
cut time-evolution orbits. The problem is that the natural choice; a
hyperplane normal to a group orbit tangent vector (Lia algebra generator)
is only good locally, globally it runs into singularities.

Calling overall complex phase ``gauge invariance'' is a waste of an ugly
but otherwise quite precise word. In quantum field theory one means by
that the local (a very deep concept) not the relatively trivial global
overall complex phase. But AMO physicists do indeed do that.

\item[2010-06-05 Daniele Avitabile]
Our problem has, as you point out, $\SOn{2} \times D_4$ symmetry. What we would
like to achieve is exploring the symmetry-breaking bifurcation scenario
pertaining to $D_4$, in the case of stationary solutions. In other words,
we do not want to simulate the time-dependent solutions of the
Ginzburg-Landau system. We concentrate instead on stationary solutions,
and we want to see how, by breaking $D_4$, we can get new, hopefully
observable, steady states.

\item[2010-06-05 Predrag]
I think that you get equilibrium solutions only within subspaces fixed by
$D_4$ symmetries. The should be also $\SOn{2}$ travelling wave solutions,
perhaps also important. They would come in two flavors - with irrational
velocities, and pre-periodic, belonging to a discrete $C_m$ subgroup of
$\SOn{2}$. And you can bifurcate into Hopf cycles, traveling waves, periodic
orbits, relative periodic orbits etc., unless for some reason the physics
does not like anything time-dependent.

\item[2010-06-05 Daniele Avitabile]
In order to do that we need to factor out the continuous
group symmetry $\SOn{2}$. As far as I understood from your comments, we
should be careful in doing that, as the the phase condition that we chose
now  (seeking a solution in a hyperplane normal to a group-orbit tangent
vector) is valid only locally. Am I correct in interpreting your
statement? If so, I should say that we explore our stationary solutions
only by means of numerical continuation. With this method, the best we
can do *is to search locally*. What we need to make sure is that, once we
have a solution and perturb it, we can still invert our problem and find
a new steady solution. In other words, seeking for a solution locally is
the best we can do, as far as I can tell. Do you think we could be
missing some patterns by using our phase condition? I think your
observation would be absolutely crucial if our continuous symmetry group
was acting on the time-variable as well, and we wanted to simulate them.
A typical case would be continuing time-dependent patterns like spiral
waves or scroll waves. A method that comes to my mind in this context is
the ``Freezing method'' by Tuemmler and Beyn, which you will probably know. I
am pretty sure that this differs from the slicing method that you told us
about. A recent article by Hermann and Gottwald recaps the method and
contains an extensive set of references about it.

Now, let us assume that we manage to factor out $\SOn{2}$, either locally
with our phase condition, or globally via the slicing method. You mention
that you always quotient discrete symmetries. Would you recommend this in
our case? What is the danger in not factoring them out? My gut feeling is
that, by factoring $D_4$, for instance, we will make sure that we will find
just one solution on the group orbit. I don't think this is bad in
general, and I would be pleased to learn what are the problems associated
with it.

\item[2010-06-05 Predrag]
I have read Tuemmler and Beyn and believe ``freezing'' = ``slicing''.
Siminos and I have Hermann and Gottwald on our reading list, but have not
studied it yet. Tuemmler and Beyn impose a slice by adding a new
dimension (the phase parameter) and a condition (a Lagrange multiplier),
while we go one dimension down by restricting the dynamics into the
slice. Both approaches are standard in imposing Poincar\' sections; Rytis
Paskauskas has written that up for Chapter 3 - Discrete time dynamics
(not yet on the public version), and Chapter 13 - Fixed points, and how
to get them, Section 13.4 Flows. My feeling is that Tuemmler and Beyn is
better, but we have not implemented it. If you do use it, let me know how
it works.

[I moved the discussion to Channelflow.org blog, and that killed it -
never heard from them again]

\item[2011-08-04 Predrag]                                       \toCB
Qihuai Liu and Dingbian Qian% \etal\rf{XXX},
{\em Modulated amplitude waves with nonzero phases in
     Bose-Einstein condensates},
\arXiv{1103.5277}, is unintelligible to me, but I would like to
understand what is the relation of MAWs to \reqva, \etc, so this is a
start. They say:

  ``
We consider uniformly propagating coherent structures of form
ansatz
\beq\label{LiQi11:2}
\psi(t,x)=R(x)\exp(i[\Theta(x)-\mu t]),
\eeq
where $R(x)\in\mathbb{R}$ gives the amplitude dynamics of the condensate
wave function, $\theta(x)$ determines the phase dynamics, and the
``chemical potential'' $\mu$, defined as the energy which takes to add
one more particle to the system, is proportional to the number of atoms
trapped in the condensate. When the (temporally periodic) coherent
structure is also spatially periodic, it is called a modulated amplitude
wave (MAW).
% \cite{brusch2000modulated,brusch2001modulated}
Similarly, a solution of the equation with the (temporally periodic)
coherent structure is called a quasi-periodic modulated amplitude wave
(QMAW) if it is also spatially quasi-periodic. If the phase of the
condensate wave function is trivial the MAW is a standing wave. Even the
amplitude $R(x)$ is $L$-periodic, the corresponding condensate wave
function $\psi(x,t)$ may be not periodic with respect to the spatial
variable $x$. If $2\pi/\nu$ and $L$ are rationally related, then
$\psi(x, t)$ is a MAW; otherwise $2\pi/\nu$ and $L$ are incommensurate,
and $\psi(x, t)$ is not periodic but quasi-periodic QMAW with the
frequency $\omega = \langle 2\pi/\nu,L\rangle$.
  ''

They copy their definitions from L. Brusch, M. G. Zimmermann, M. Van
Hecke, M. B\"ar and A. Torcini, ``Modulated amplitude waves and the
transition from phase to defect chaos'', {\em Phys. Rev. Lett. \bf 85},
86--89 (2000); L. Brusch, A. Torcini, M. Van Hecke, M. G. Zimmermann and
M. B\"ar, ``Modulated amplitude waves and defect formation in the
one-dimensional complex Ginzburg-Landau equation'', Physica D 160,
127--148 (2001).

\item[2012-01-15 Predrag] Read Scheuer and Malomed\rf{SMcgl02},
\emph{``Stable and chaotic solutions of the complex {G}inzburg-{L}andau
equation with periodic boundary conditions''}: ``The behavior of the
solutions of the CGLe without diffusion is studied with a periodic
boundary condition.''

\item[2012-01-15 Predrag] Read Doering \etal\rf{doercgl88,doercgl87},
``Low-dimensional behaviour in the complex {G}inzburg-{L}andau
equation'': ``The stability of plane waves of the CGLe is analyzed in a
periodic domain. The mass and energy integrals are estimated to give an
upper bound of the magnitude of the field variable. After that, cone
condition is established for the CGLe which indicates the existence of
the inertial manifold and proves the existence of finite dimensionality
of the attractor in this infinite-dimensional system. The Lyapunov
dimension is estimated.''

\item[2012-01-15 Predrag] Read Torcini\rf{torc96},
\emph{``Order Parameter for the Transition from Phase to
                 Amplitude Turbulence''}:
``The maximal consersed phase gradient is introduced as
                 an order parameter to characterize the transition from
                 phase to defect turbulence in the CGLe. It has a finite
                 value in the PT regime and decreases to zero when the
                 transition to defect turbulence is approached. A
                 modified KSe is able to reproduce the main feature of
                 the stable waves and to explain their origin.''

\item[2012-01-15 Predrag] Read van Hecke and Howard\rf{MvHhole01},
\emph{``Ordered and Self-Disordered Dynamics of Holes and Defects in the
One-Dimensional Complex {G}inzburg-{L}andau Equation.'' }

\item[2012-01-15 Predrag] Read Torcini \etal{TFGphase97}, \emph{``Studies
of Phase Turbulence in the One Dimensional Complex {G}inzburg-{L}andau
Equation''}, and Aranson and L. Kramer\rf{AKcgl02}, and there are a ton
more CGL references in the bib file, inherited from
Lan\rf{LanThesis,lanmaw03}.

\item[2012-03-28 Daniel, Predrag]
Rodriguez and Schell\rf{Rodriguez1990} are using a truncation of the
Ginsberg-Landau equation, which results in a 4D system of ODE's. More
notes on this in \refchap{chap:atlas}~{\em Atlas}.

``[...] Chaos has previously
been studied in this LG equation\rf{SiRo87}. %[5,6] - Keefe?.

Looks like one should have a look at \refrefs{SiRo87,SPScgl92,GoSi94}.


\end{description}

% \newpage
\section{Alternative periodic orbit `theories'}
\label{sec:alternCycl}
% Predrag  collected notes on Zoldi here  2012-09-17
\renewcommand{\ssp}{x}             % finite-dimensional state space point

{\bf [2012-09-13 Y.-Mee Predrag]}   %\toCB
Kerswell\rf{ChaKer12} cites both the correct and the erroneous references on `Periodic
Orbit Theory', so I have reread Zoldi, Lopez and Kazantsev. It's like having to
read papers on alternative `Quantum Theories' by people who lack patience
to take intro QM and learn the Schr\"odinger equation.
% Instead, they strenuously argue that their home-cooked delirious
% `Theory' is superior.
I had assumed that with time erroneous formulas would disappear, but no. We
have to explain clearly and in print why some of the published formulas
are wrong, other suboptimal, because they keep getting cited by reputable
researchers.

My recommendation to Kerswell is to remove all computations over
`averages' from their paper\rf{ChaKer12}, and state instead that while
large sets of \NS\ unstable \po s have been determined in their work (as
well as in \refrefs{CviGib10,KreEck12}), we still lack the understanding
of how they are interrelated (\statesp\ partitions, \markGraph s,
symbolic dynamics) required before periodic orbit
theory\rf{DasBuch,Lan10} can be applied to evaluation of dynamical
averages.

Here are some preparatory notes for a remark for ChaosBook.org (or in
some publication?). I would be grateful for any corrections, comments,
clarifications, pointers to important missed references.

\vfill


\begin{description}
\item[2013-01-30 Predrag]
Remark {\em Alternative 'Periodic Orbit Theories'} moved to
ChaosBook chapter recycle.tex
\end{description}

\subsection{`Escape-time weighting'}
\label{s:ZoldiSums}


\begin{description}

% call ZG `escape-time weighting'.

\item[1997-05-23 Predrag]
Zoldi and Greenside\rf{ZG96} write:
``
[...]
Theory that expresses the natural measure of an attractor in terms of the
set of \po s requires the assumption of hyperbolicity\rf{yorke2}, which
fails for most dynamical systems because of tangencies of stable and
unstable manifolds\rf{AACI} and because of unstable dimension
variability\rf{kostelich97}.
    \PC{2004-09-20  Its subtle. Depending on the observable, marginal
    cycles might lead to power laws, or might not matter at all. That's
    why we have a theory of phase transitions on strange sets\rf{ACK89}.}
The powerful cycle expansion method that
expresses averages in terms of a moderate number of \po s\rf{AACI} is
practical only if a symbolic dynamics unique labeling of each
\po \rf{AACI} is explicitly known and it is widely believed that most
dynamical systems lack a symbolic dynamics.
    \PC{2004-09-20 Stability ordering helps\rf{DM97,carl97int}.}
Even if a system is
hyperbolic and has an explicit symbolic dynamics, it is still nontrivial
to calculate a relatively complete set of \po s from specified equations
or from measured time series to apply a cycle expansion.
    \PC{2004-09-20 Tough luck. Nobody promised you a rose garden.}
In the absence of hyperbolicity and of a symbolic dynamics, it is not
known how to weight a given set of \po s so as to approximate a given
statistical average in a high-dimensional regime.
[... Our]
third achievement is an empirical discovery that a weighting of \po s
based on escape times can approximate several statistical averages
accurately.
    \PC{2004-09-20 Really?}

[...] The number of [\KS]
positive Lyapunov exponents for each \po\  varied between 3 and 8 and the
largest Lyapunov exponent of the \po s varied between 0.02 and 0.34.
        \PC{ It is extremely unlikely that \po s with 8 expanding
        eigendirections are a part of this \KS\ strange attractor. Blind
        Newton routine searches find whatever they find, isolated or part
        of an attractor. Weighted by Floquet \emph{multipliers}, very
        unstable orbits contribute very little; weighted by Lyapunov
        exponents they can contribute as much as the dominant short
        orbits. Plain wrong. }
[...] The convergence
of the escape time estimate to the Lyapunov dimension
is statistical in that the error decreases approximately as
$1/\sqrt{N}$, where N is the number of \po s contributing to the
weighted sum.
    \PC{2004-09-19 To make this into escape time one would have to
    specify the size of a ball of initial conditions around an initial
    point on the \po\ $p$. Were an ergodic orbit to enter a neighborhood
    of a \po\ the time it would linger there is presumably determined by
    the least contracting stable multipliers (the time of approach) no
    less than by the least expanding unstable multipliers (the time of
    escape). }
    \PC{2004-09-15  One is averaging over local Lyapunov exponents by
    essentially random samples of \statesp\ points, so $1/\sqrt{N}$ is
    about right. Garbage in, garbage out.}
Other previously published weightings of the \po s were tried
\rf{PawSchu91,yorke2} but were found not to give results as accurate as
our escape-time weighting, with relative errors larger than 10\%.
''


In \refref{Zoldi98} Zoldi writes
``
The aim of this paper is to examine the relative success and limitations
of trace formulas and escape-time weighting~(ETW) of \po s in predicting
the structure commonly found in histograms of chaotic time series data.
To this end, the Lorenz equations are well suited as they can model
a single-mode laser [11] allowing us to make qualitative comparisons with
recent chaotic multimode laser experiments that have intensity histograms
that exhibit sharp reproducible peaks [2].
[...]
We demonstrate that the structure of histograms of the chaotic $\ssp$
variable can be predicted in terms of the \po s of the Lorenz equations
using both trace formulas and ETW in the Axiom-A regime, and further we
demonstrate that ETW is more effective in the non-Axiom-A regime than
trace formulas.

In the unpublished \refref{Zoldi99} ``Escape time weighting of unstable
stationary solutions of spatiotemporal chaos'' Zoldi incorporates
unstable \eqva, and decides, for reasons known only to him, to throw away
the asymmetric ones; \eqva\ that he keeps are precisely the symmetric
\eqva\ that \refrefs{KeTu06,ACHKW11,atlas12} find unphysical. If he adds
these symmetric, not isolated \eqva, he gets a better plot for
time-average of the Kuramoto-Sivashinsky chaotic trajectory~$ \langle
u(t,x) \rangle_T$ estimated by the escape-time weighting of 80 symmetry
pruned unstable stationary solution and 127 unstable periodic orbit
time-average patterns, than from 127 unstable periodic orbits alone.

``
Our goal in this report is to demonstrate that the convergence of the ETW
average of \po s in the spatiotemporal chaotic Kuramoto-Sivashinsky
equation can be improved using unstable \eqva\
    \PC{Zoldi calls them `unstable stationary solutions' (USS). }
[...]
We demonstrate that ETW converges as O(1/N) with N computed \po s and \eqva.
[...]
Past work computing \eqva\ in the Kuramoto-Sivashinsky equation with
periodic boundary conditions demonstrated that the \eqva\ consisted of an
unexpectedly large class of stationary solutions, which include laminar
states, N-cell states (periodic patterns), long-wave modulated N-cell
states, giant states (large amplitude solutions), and traveling wave
states [18-20].
[...]
Many unstable stationary solutions were found to be isolated from the
chaotic attractor not resembling any chaotic solution. Other \eqva\ did
resemble chaotic solutions and exhibited small values of the difference
function,
\[
G(t) = \|u(t,x) - s(x)\|,
\]
where $u(t,x)$ denotes
the spatiotemporal chaotic field, $s(x)$ denotes the stationary
solution, and $\| \cdots \|$ denotes the infinity norm.
\PC{2012-09-20  the sup norm.} Unfortunately
finding the minimum value of~$G(t)$ depends on recurrence times
of the spatiotemporal chaos, so we prune the set of computed \eqva\ by
requiring that stationary solutions have the same symmetry as the
time-average pattern of the chaotic attractor.
\PC{2012-09-20  what the ??? Why?} We impose the
constraint that the stationary solutions satisfy the condition that
$\partial_x^2 u(0) < 0$ and $\partial_x^2 u(L) > 0$.  This symmetry
pruning results in a set of 80 symmetry related \eqva.  We note that
this criteria does not remove all isolated \eqva.
''
    %
\PC{2012-09-18  Incorporating unstable \eqva\ is not crazy, as
that is the only situation for which \refeq{ZoldiAv} might make sense.
Contribution of an \eqv\ is evaluated in the boy scout version of
ChaosBook.org det.tex, and in Sect.~3.7 of Gaspard\rf{PG97}, but its
inclusion into cycle expansions and expectation values has never been
carried through.}

\item[2004-09-15 Predrag]
Vanessa L{\'o}pez\rf{lop05rel} used the escape-time weighting
formulas of Zoldi\rf{Zoldi98,ZG96} to ``average'' over periodic orbits,
otherwise the paper is a good paper.
One cannot blame a computer science graduate student for trying a formula
published in Phys. Rev. Letters.
% (where were faculty advisers?), but
% citing it will just keep confusing future readers.
I had hoped I would
never see that stuff again. Dettmann and I tried to get Zoldi to derive
`escape-time weighting' for us, and in 1999 Mainieri hired him at Los
Alamos to learn how this works. To the best of my knowledge, none of us could
make any sense of it whatsoever.

\item[2012-09-13 Predrag]   %\toCB
Kazantsev\rf{Kazantsev98} `weighted averages' \refeq{KazantsevAv} over \po s
are a variant of Zoldi's escape-time weighting.

I cannot even guess where these formulas come from. Perhaps Kazantsev
writes them because initial work was on the stability of \eqva, where
stability exponents $\Lyap_j$ might make sense? Reading the 'seminal'
Charney and DeVore paper\rf{ChaDeVore79} on this is not cheerful (see
{\bf [2012-09-14 Predrag]} below).

He cites Zoldi\rf{Zoldi98,ZG96} and then writes: `reasoning for this is
clear; less unstable orbits must be weighted more heavily, so the
attractor dimension is approximated as \refeq{ZoldiAv}, $\sum^{e}
\Lyap_{i}$ has dimension of $1/t$, so he interprets it as ``a priori
estimate of the mean time spent by the system in vicinity of \po.''
That's it: the `derivation'. The weight is obviously wrong, as any
periodic orbit, no matter how long and unstable, has the same weight
$\sum^{e} \Lyap_{p,i}$ as long as it's Lyapunovs (instability per unit time)
is the same. That Kazantsev fixes by saying ``it is reasonable also to
suppose that orbits with longer periods must be weighted more heavily as
they are longer and should provide a greater contribution to the total
sum.'' Thus he modifies it to \refeq{KazantsevAv}, which does not help at
all.

\item[2012-09-13 Predrag]
Kazantsev\rf{Kazantsev98} defines some arbitrary Euclidean \statesp\
distance $r=0.05$ as `vicinity' and finds 90\% (!) correlation with
episodes in which `barotropic ocean model' ergodic trajectory approaches
his \po, whose periods range from 40 to 250 days. That cannot possibly be
right - the duration of a visit to a neighborhood surely also depends on
the contracting eigenvalues - if they are weakly negative, you longer in
the neighborhood. Would not see that for Lorenz, as contracting
eigenvalues are strongly contractive. This is not even a Kaplan-Yorke
type formula, and that is easy to understand, even though his `dimension'
of a \po\ is Kaplan-Yorke dimension... Go figure. His `weighted formula'
for `dimension' is in the same ballpark as what he gets from dimension
$D=5.8$ computed over 500 years of model integration.

\item[2009-05-01 Predrag]
(entry in
\HREF{http://www.channelflow.org/dokuwiki/doku.php?id=chaosbook:pcf}
{Channelflow.org}) Andrey Gritsun work on unstable periodic orbits in
atmospheric science sounds potentially interesting, but I have not been
able to find anything to read about it. They have had an
\HREF{http://www.checkout.org.cn/awardsearch/showAward.do?AwardNumber=0530868}
{NSF grant}, a number of identical conference abstracts for different
conferences in 2007, 2008 and SIAM DS09, but at most one
publication\rf{Gritsun08}. I am not able to get the article through
GaTech library, but from the abstract I do not think he is working with
high-dimensional PDEs. There is Kazantsev work\rf{Kazantsev98} from 1998,
Gritsun work seems to be a continuation. Kazantsev might be worth a read
for understanding what the problem is and his methodology of finding
periodic orbits.
\HREF{http://arxiv.org/find/all/1/all:+AND+Eugene+Kazantsev/0/1/0/all/0/1}
{Does not do this} stuff any more. See {\bf [2012-09-21 Predrag]} below.

\item[2009-05-02 John Gibson]
Gritsun is working in a 200-d model of barotropic flow.

\item[2012-09-21 Predrag]
OK, now I did find this Gritsun \refrefs{Gritsun10,Gritsun11}
and saved them in ChaosBook.org/library.
Now have to also read Selten and Branstator\rf{SelBran04}
as well as \refrefs{GrBrMa08,HaMaj10,AbrMaj08,Abramov09}.

\item[2007-11-28 Predrag: Japanese heresy]
We do not want to refer to wrong papers, but here it is, for
the internal record, so we do not forget not to cite it
(from Physical Review Letters, 16 Sep 2004 request to referee,
which I ignored):

Mitsuhiro Kawasaki and Shin-ichi Sasa\rf{KaSa05},
    ``Statistics of unstable periodic orbits of a chaotic dynamical system
    with a large number of degrees of freedom.''

\item[2011-10-06 Predrag]
The famous
Japanese Heresy: There is NO such single \po\ -
instead of this there is perfectly well developed theory that says how
you use \po s and how many do you need to capture the hyperbolic parts of
the {\nws}. It's as elegant and systematic as Stat Mech and Quantum Field
Theory. Read \HREF{http://chaosbook.org/}{The Book}. But who reads books
nowadays? BTW, there is no need to attach prefix U to \po s; there are a
few or no stable orbits in chaotic dynamics, and exponentially many
unstable ones, it's some dumb Soviet style abbreviation that must have
come from Maryland or somewhere.

\item[2012-05-13  Predrag] [...] until the first unstable periodic solutions of
Navier-Stokes were computed by Kawahara and Kida\rf{KawKida01} in 2001,
determining such solutions seemed utterly out of reach. Their \pCf\
`upper' periodic orbit appears embedded in the turbulent sea, and
captures statistics so well that it lead to the `Heresy', a belief of the
innocent that there exists a \emph{single} periodic orbit (!) that
captures turbulent statistics; % we do not cite these papers, as
that was a vain hope of those too busy to read ChaosBook.org.

\item[2012-08-12 PC]
Goldobin\rf{Goldobin12} \emph{Limit distribution of averages over
unstable periodic orbits forming chaotic attractor}, \arXiv{1208.1691},
is much weirder still: it is motivated by the Japanese Heresy and cites
only the Maryland non-theory paper as the source on the periodic orbit
theory.  Remind him to cite ChaosBook.

\item[2012-09-18 Predrag]
Still to read:

`Unstable dimension variability'\rf{kostelich97}. I do not even
comprehend that there is a problem; our \po s have varying number of
unstable Floquet multipliers. What's the problem with that?

Pawelzik and Schuster\rf{PawSchu91}

{\em Statistical mechanics of deterministic chaos},
by Kai and Tomita\rf{KaiTom80}

\newpage

\subsection{A brief history of cycle expansions}
% Predrag 2012-09-17 transferred from
% channelflow.org/dokuwiki/doku.php?id=chaosbook:appendhist

%\Chapter{appendHist}{19aug2008}{A brief history of chaos}
{\bf [2012-09-19 Predrag]} moved temporarily to here from ChaosBook.org
appendHist.tex, remember to return this back.

\bigskip

\noindent
If one is reluctant to refer to people with long, weird surnames, one can
always refer to Ruelle (not Bowen, he did not do that) for deriving the
dynamical (or Ruelle) zeta function, and Gutzwiller for formulating
semiclassical quantization as a Zeta function over unstable \po s.
Sinai-Bowen-Ruelle work was much smarter and more profound than vast
majority of physicists' publications from 1980s on. However, continuous
time flow traces weighted by cycle periods were introduced by Bowen who
treated them as Poincar� section suspensions weighted by the `time
ceiling' function. But I do not see cycle expansions that we use in
Bowen's work (see, for example, the description in
\HREF{Scholarpedia.org/article/Bowen-Margulis_measure}
{Scholarpedia.org}).

%Predrag 2009-12-01
Aurell, Artuso, Gunaratne and Cvitanovi\'c derived `cycle expansions' in
1986-87 (earliest reference to `cycle expansion' in my notes is from 29
October 1987).
    \PC{
    pruning fronts symbolic dynamics is coded by \MarkGraph s;
    topological entropy is given by roots of their determinants.
    That lead us to the spectral determinant for smooth flows.}
They arose not from dynamics, but
from a study of the scalings in period-doubling and cycle-map
renormalizations\rf{CCR,AACII,CGV}. I then applied the same technique to
study of the topology of the H\'enon attractor, and developed my pruning
front theory\rf{pre88top}.
    \PC{My notes say it was from 17 July 1986 to 9 October 1987.}
This work was done at Chalmers, Gothenburg
1985-1986 and at Cornell January-May 1985.
    \PC{Notebook \#41, 10 October 1986: Thinking about the cycle-map
    renormalization understood that the same cycles are involved as in
    the H\'enon map, but now in the renormalizatio `time'.}

Our group tends to publish only once a topic is understand in some depth (cycle expansions
were developed in  PhD theses of Artuso and Aurell, which we wrote up in
\refrefs{AACI,AACII}, as well as PhD theses of Christiansen, Rugh and
Putkaradze. The pruning front theory was developed in Hansen's PhD
thesis\rf{hansen}.  I gave many conference and seminar talks about cycle
expansions. Inter alia, I talked about the periodic-orbit topology of
Lozi and H\'enon attractors, and their cycle expansions at the
``\emph{Chaos and Related Nonlinear Phenomena: Where do we go from
here?}'', a meeting organized by Moshe Shapiro and Itamar Procaccia,
held December 13--19, 1986 at Kiryat Anavim kibbutz in Israel. A great
meeting, and Celso Grebogi was in the audience.

Grebogi never cites me or Kiryat Anavim or cycle expansions.
\PC{ A number of
people do have proceedings papers listed in their CVs as published in
Proc. Fritz Haber Int. Symposium ``Chaos and Related Nonlinear Phenomena:
Where do we go from here?'' edited by I. Procaccia (Plenum, New York,
1987). That's funny, as no proceedings were ever published.}

The first published paper was Auerbach \etal\rf{pchaot} (submitted 13
March 1987). (But it might be fair to say it is written down first in
\HREF{http://www.scholarpedia.org/article/Unstable_periodic_orbits} {Kandanoff (1984)}.)
We had agreed that Procaccia would not show me the
manuscript before the paper was published, as I surely would have gone
through the roof had I read it (we had not developed cycle
expansions yet), so only a `level sum' approximation \refeq{GOY:levl-sum},
    \PC{read ChaosBook.org \refsect{rem_lev_sums}}
an $n$th order estimate $\eigenvL_{(n)}$ of the leading \FP\ eigenvalue
$\eigenvL$ , \ie, an approximation
to the trace formula is
presented, and applied to the H\'enon attractor (Eq.~(4) of the paper).
Mysteriously, this paper seems to be the most cited periodic orbits
paper, at least of the papers with my name on it. My first attempt to
make cycle expansions accessible was the Phys. Rev. Letter\rf{inv}
\HREF{http://www.cns.gatech.edu/~predrag/papers/preprints.html\#Cycling}
{Invariant measurement} of strange sets in terms of cycles (submitted
March 1988). The two long papers with Artuso and Aurell\rf{AACI,AACII}
and the paper with Eckhardt\rf{pexp} for continuous time flows are
better. Developing the theory in detail did not do much good.

The \cycForm s %\rf{AACI}
for the expectation value of the observable $\obser$ is given by
\refeq{GOY:16.16a}.

After that meeting Grebogi and the prodigiously prolific Maryland group
wrote a long series of articles about `UPOs'. In the first one\rf{yorke2}
\emph{Unstable periodic orbits and the dimensions of multifractal chaotic
attractor} (submitted September 1987), they presented a formula for the
natural measure in terms of the unstable periodic orbits (UPOs), an
approximate trace formula in which periodic points are  weighted by
products of the unstable Floquet multipliers, their Eq.~(14).  That paper
does cite Auerbach \etal\rf{pchaot}, in which the same formula seems to
have been published for the first time.

More frequently cited is the Grebogi, Ott and Yorke\rf{yorke2}. They proved
that for mixing hyperbolic attractors the SRB or natural measure $\SRB$
can be described by the unstable \po s, \ie, the natural probability measure
of the attractor contained in some closed subset $\pS_{\cal S}$ of the $d$\dmn\
\statesp\ $\pS$, is given by the sum over the periodic points $j$ of period
$n$ in $\pS_{\cal S}$, weighted by the inverse of the product of their
expanding Floquet multipliers $\ExpaEig_j$,
\beq
\SRB(\pS_S) = \lim_{\cl{}\to \infty}
\sum_{j\inFix{\cl{}}}
\frac{1}{\ExpaEig_j}
\,,\qquad \ssp_j \in \pS_{\cal S}
\,.
\ee{GOY-POnatMes}
This formula for natural measure is the special case of
\refeq{GOY:levl-sum}, with leading \FP\ eigenvalue $\eigenvL =0$ (no
escape), and trivial observable $\beta =0$; 1 in \refeq{GOY:levl-sum} is
the statement of probability conservation. In semi-classical quantization
of chaos this statement that total probability is conserved even
has a name: it is called `Hannay-Ozorio de Almeida sum rule'. The 1987
derivation of \refeq{GOY:levl-sum} was heuristic, so Grebogi, Ott and
Yorke proved \refeq{GOY-POnatMes} by taking the ${\cl{} \to \infty}$
limit. However, there is no need for an ${\cl{} \to \infty}$ limit,
as the exact {\Fd} is not a long-time limit of anything, it
is the exact sum over all \po s, just as its statistical mechanics
cousin, the partition function, is the full sum of all states, not a limit
of some approximate subsums.

% Grebogi, Ott, and Yorke\rf{yorke2} write in
% Partition function formulation of the
% multifractal properties of chaotic attractors:
%    \label{s:MarylandSums}
% see Eqs. (4.5) and (4.10)

In actual computations of dynamical averages it would be madness to take
an ${\cl{} \to \infty}$ limit, as very quickly computations are undoable:
longer and longer period orbits are exponentially more and more unstable,
their number grows exponentially, the natural measure is everywhere
singular, and its limit uncomputable. So why would one prefer limit of a
heuristic sum, like \refeq{GOY-POnatMes}, when one has the exact {\Fd},
convergent exact \po s sums, and exact \po s formulas for dynamical
averages of observables? It is not even wrong.
Grebogi, Ott and Yorke paper about unstable periodic orbits focused on
the fractal dimensions of chaotic attractors, as was the fashion in the
end of the 1980's. They proved that for mixing hyperbolic attractors the
SRB or natural measure $\SRB$ is described by the unstable \po s, \ie,
the natural probability measure of the attractor contained in some closed
subset $\pS_S$ of the $d$\dmn\ \statesp\ $\pS$ is given by
\refeq{GOY-POnatMes},
the sum over
all the fixed points $k$ of the $n$th iterate of map $f$, weighted by the
inverse of the product of its expanding Floquet multipliers $\ExpaEig_k$.
In mathematics one is extraordinarily fond of taking limits; but in
statistical mechanics a partition function is not a limit of anything, it
is the full sum of all states, and its ergodic theory cousin, the
spectral determinant is not a long-time limit, it is the exact sum over
all \po s. It would be madness to take an ${n\to \infty}$ as longer and
longer periodic points are exponentially more and more unstable,
exponentially growing in number, and uncomputable; and the natural
measure everywhere singular and its limit even more uncomputable. So why
would one insist on limits like \refeq{GOY-POnatMes} when one has the
exact and convergent \po s sums, and exact \po s formulas for dynamical
averages of observables? Go figure...

We cite Zoldi and Greenside\rf{ZG96} for being the second paper
to determine unstable \po s for \KS, on a domain larger than what was
studied in \refref{Christiansen97}; L{\'o}pez\rf{lop05rel} for being first
to determine \rpo s in a spatio-temporal PDE, and
Kazantsev\rf{Kazantsev98} for being first to determine \po s in \po s in
a weather model, and for his variational method for finding periodic
orbits. We love them, but not because of their `escape-time weighting'.

Maryland group is extraordinarily fond of
baker's maps. Baker's maps are linear, they have no cycle expansion
curvature terms, and no associated shadowing, which might be the reason
why the group developed no appreciation of spectral determinants and
their cycle expansions. The Maryland approach was and remained heuristic,
and there is no need for that. These heuristic formulas are
approximations to the exact trace formulas (that are derived in ChaosBook
and Gaspard monograph\rf{PG97} with no more effort than the heuristic
approximations), but they are not smart for computations; faster
convergence and the correct spectrum is obtained by utilizing the
shadowing that is built into  the exact cycle expansions of dynamical
zeta functions and Fredholm determinants. Cycle expansions are not
heuristic, they are exact expansions in the unstable periodic orbits for
classical dynamics, and semi-classically exact for quantum
mechanics\rf{inv,AACI,AACII}.

Ever since the Maryland school
cites only Maryland papers and some of the mathematicians of the 1970's,
never the Copenhagen papers. In mathematics that is the custom; after
long discomfort of not understanding a physics paper, one relabels all
symbols in all formulas, proves something that was expected to be true,
and mentions once to 'physicist Blah has conjectured that', and then
never refer to poor Blah in any subsequent papers. But a 2nd physics
paper rarely is so much better than the original idea that we should
practice this standard.

So much meaningless busyness, and so little effort to learn the
original and better formulation instead, and use the time to apply it
to new problems.


% \item[2012-09-26 Predrag]
Lai\rf{Lai97} \emph{Characterization of the natural measure by unstable
periodic orbits in nonhyperbolic chaotic systems} is a bit exasperating,
too. We did this for H\'enon (like Lai does) in 1990\rf{AACII}, showed
when non-hyperbolicity does not matter, and developed the theory of phase
transitions on strange sets\rf{ACK89} to explain it. Why not cite any of
the Copenhagen papers. Though in the sequel, Dhamala and Lai\rf{DhaLai99}
do cite them.

The most recent in the Maryland school series is
``\HREF{http://arxiv.org/abs/0912.0596} {\emph{Time-averaged properties}}
\emph{of unstable periodic orbits and chaotic orbits} in ordinary
differential equation systems'' by Michael A. Zaks and Denis S.
Goldobin\rf{ZakGol10}, and the sequel\rf{Goldobin12}. It would not hurt
to recycle these kinds of computations in terms of spectral determinants,
instead of sticking to evaluations of approximate trace formulas. There
is no extra work involved, except reading relevant bits of ChaosBook
and/or Gaspard monograph\rf{PG97}, and inserting the same set of UPOs of
the Lorenz equations into cycle expansions. They should converge faster -
symbolic dynamics is a problem here, so perhaps one would need to use the
stability cutoffs.

Goldobin\rf{Goldobin12} writes: ``
We address the question of representativeness of a single long unstable
periodic orbit for properties of the chaotic attractor it is embedded in.
Y. Saiki and M. Yamada\rf{SaiYam98} have recently suggested the
hypothesis that there exist a limit distribution of averages over
unstable periodic orbits with given number of loops, N, which is not a
Dirac $\delta$-function for infinitely long orbits. In this paper we show
that the limit distribution is actually a $\delta$-function.
[...]
    \PC{He only refers to Maryland}
Notice, the evaluation of average values for the chaotic attractor
requires this distribution to be weighted with the distribution of the
natural measure over the set of UPOs; in \refref{yorke2} the natural
measure of an UPO was shown to be inverse proportional to its
multiplier.Meanwhile, we restrict our consideration to the question of
convergence of the limit distribution and do not consider the
distribution of the natural measure.
''


\newpage

\item[Historical postscript:] As of 2012 Scott M. Zoldi was Vice President,
Analytic Science, Fair Isaac Corporation (leading the Transaction
Analytics team in the development of transaction analtyics [sic] products
associated with fraud, credit risk, and various custom models), board
member of www.sdsic.org, and had a number of patents.

\item[Historical postscript:] Henry S. Greenside is a professor in
Physics Department, Duke University. Web of Science shows no journal publication
more recent than 2006.

\item[Predrag] I've never gotten research funding in US for
work on periodic orbit theory. Here is why, from Sept. 2003:
\HREF{http://www.cns.gatech.edu/~predrag/research.html} {an NSF
evaluation} from  anonymous reviewer (italics are mine).

RATING: Fair

The idea that chaotic dynamics is built upon unstable periodic orbits is
hardly new. In classical chaos theory, it has been known for many years
that for hyperbolic systems, ergodic averages associated with natural
invariant measures can be expressed as weighted summations of the
corresponding averages about the infinite set of unstable periodic orbits
embedded in the underlying chaotic set. For nonhyperbolic systems, there
is no rigorous assurance of the validity of the periodic-orbit theory,
although recent success on explicit enumerations of unstable periodic
orbits in low-dimensional maps leads to confidence in the applicability
of the theory. In semiclassical quantum mechanics, Gutzwiller taught us
more than thirty years ago that the density of states, upon which most
quantities of physical interests build, can be expressed as an infinite
sum in terms of classical periodic orbits.

The PI's idea to apply the periodic-theory to spatiotemporal chaotic
systems may be interesting \emph{but clearly not original}.
    \PC{Christiansen \etal\rf{Christiansen97} submitted June 1996.}
The key questions
are thus what new understanding would one possibly gain about
spatiotemporal chaotic systems and turbulence, and how useful and
feasible such an approach could be.

First of all, what new insights into spatiotemporal chaos or turbulence
can the periodic-orbit theory provide? For instance, in turbulence there
is already vast knowledge about the various scaling laws dealing with
energy, velocity, vorticity, and wavenumber statistics. Are the
predictions of the PI's ``chaotic field theory'' consistent with the
existing understanding and more importantly, what NEW results can one
expect from such a theory? Unfortunately these issues were not addressed
clearly in the proposal.

The second issue concerns the difficulty to compute unstable periodic
orbits of chaotic systems (the PI should be very well aware of this).
Although, in recent ten years or so there were progresses in enumerating
periodic orbits of low-dimensional chaotic systems (mostly
two-dimensional invertible maps or three-dimensional flows), how to
obtain a relatively complete set of orbits for systems in higher
dimensions remains to be an open problem. One difficulty concerns the
high values of the topological entropies typically seen in
high-dimensional chaotic systems: the number of unstable periodic orbits
increases so extremely rapidly with the period that not many orbits of
even low periods can be computed reliably. In the case of spatiotemporal
systems which are much higher-dimensional than, say, two-dimensional
maps, it is not clear how unstable periodic orbits can be computed in
general (in some specific setting, there was success to compute a small
set of these orbits a few years ago by Greenside and Zoldi).
    \PC{Zoldi and Greenside\rf{ZG96} submitted April 1997.}
To this
reviewer the issue of ACTUALLY computing unstable periodic orbits is more
important than applying known formulae for statistical averages to
spatiotemporal systems. A detailed description of the method to compute
these orbits and justification for its feasibility for spatiotemporal
chaotic systems would be much more useful than a \emph{show-off of fancy
formulae or diagrams from existing papers}. Unfortunately the PI just
mentioned very briefly that ``variational methods for determining
recurrent patterns are currently under development.''

For people who practice and/or apply nonlinear dynamics, the burning
issue in terms of any unstable-periodic-orbit based theory is how to
compute the orbits from measurements, when the system equations are not
known. For researchers in physical or biomedical sciences, systems of
interest are usually such that explicit descriptions of their equations
are not available. Periodic orbits \emph{are interesting only when they
can be computed from time series}, typically under noise. Works and
partial success in the past demonstrate how difficult this task can be.
There is no mentioning of this important issue in the proposal. \emph{It
is difficult to see how the proposed research could be of interest even
to a researcher in nonlinear dynamics, let alone people from other
disciplines.}

\emph{The narrow scope and the possible lack of appeals of the proposal
even to researchers indicate that it may be difficult to attract talented
students or postdocs to the proposed research.}
        \PC{Huh?  Go figure...}


\end{description}

\newpage


\section{\Po s literature}
\label{sec:UPOlit}

\begin{description}

\item[2011-05-15 Predrag]
\HREF{PubMed.gov}{PubMed.gov} is scary. Find a paper you
want, click on "Related citations" on the right, and you get more stuff than you
can ever digest... Zotero helps you generate (imperfect) BibTeX entries.

\item[2010-03-12 PC] have a look at Morita \etal\rf{MFKM10}:
    {\em Scytale decodes chaos: {A} method for estimating
    unstable symmetric solutions}.
{\bf 2010-03-12 PC:} Had a look myself, and also uploaded a copy to
\HREF{http://www.zotero.org/groups/cns/items/collection/VTVKWDCZ}
{zotero}. They say: ``
A method for estimating a period of unstable periodic solutions
is suggested in continuous dissipative chaotic dynamical
systems. The measurement of a minimum distance between a
reference state and an image of transformation of it exhibits a
characteristic structure of the system, and the local minima of
the structure give candidates of period and state of
corresponding symmetric solutions. Appropriate periods and
initial states for the Newton method are chosen efficiently by
setting a threshold to the range of the minimum distance and
the period.
	''

											\toCB
They cite Dennis and Schnabel\rf{Dennis96} and Mees\rf{Mees81} for Newton
methods, which they call ``Newton-Raphson-Mees.'' Their ``Scytale''
method they test on the Lorenz equations and \KSe\ (you can safely ignore
the Spartan cypher  ``scytale,'' it's only of minor interest).
(See also {\bf [2010-03-12 PC]} below.)

As a desymmetrization paper, this is a bit naive: ``From the viewpoint of
symmetry and the corresponding transformation, UPOs can be considered
invariant solutions under discrete time translation.'' By using the usual
definition of an \rpo\ with a given symmetry, they can ``select the
symmetric orbit fragment.'' They only study $C_2$-equivariant examples
and observe (surprise) that \rpo s are easier to find than the
corresponding full \po s. They call ``symmetry'' a ``generalized
symmetry.'' Who ordered that?

											\toCB
According to them, my \refref{pchaot} is the most standard
method but it relies on reducing the problem from a flow in \statesp\ to
the \Poincare\ return map by taking an appropriate section, which is hard
to select in high-dimensional flows and/or experimental data analysis. I
did not realize that \Poincare\ return map was needed, but then I have
not read that paper (the deal was that I was not allowed to see it until
it got published - seemed trivial to me, hence it's one of my more cited
papers). Gibson\rf{CviGib10} emphatically does not use any \Poincare\
sections. They cite \refref{EKR87,MRTK07} for recurrence plots, but
that's because there is a
\HREF{http://en.wikipedia.org/wiki/Recurrence_plot}{wiki}, presumably
created by Marwan\rf{Marw08} (as well as bunch of other
\HREF{http://www.scitopics.com/Recurrence_Plot.html}{wikis} and
\HREF{http://www.recurrence-plot.tk/}{sites}, all referring to -guess
who?- Marwan). It looks like now one has to go edit wikis if one wants to
get cited for something. They are of unpredictable quality -
for example, whoever wrote
\HREF{http://en.wikipedia.org/wiki/Poincar\%C3\%A9_map}{\Poincare\ map wiki}
believes that Poincar\'e (return) map has to do with periodic orbits.
Morita \etal\ credit Grassberger and Procaccia\rf{GraPro83} with the idea
of the measurement of the distance between the reference state and the
delayed state, in that case via the correlation integral function.

Lathrop and Kostelich\rf{LaKo89} devised a method for detecting UPOs in a
chaotic time series by measuring the distance between a current and a
past state with delayed coordinates. They focus on the distribution of
states whose distance from the past state is smaller than a threshold,
and they extract periods of UPOs from the distribution. This is the
method used by Morita \etal. They search for the minimum distance between
a reference state and the image of ``generalized transformation of the
state'' (\ie, a discrete group transformation), and estimate the period
of unstable symmetric solutions (\ie, \rpo s or preperiodic orbits).

When they test it on \KSe, they first cite Rempel
\etal\rf{CRMRF02,ReCi05} but do cite us as well thereafter. Rempel,
however, has a huge production on spatio-temporal matters related to our
work\rf{RCMR04,ReChMi07,ReChMi07,ReCi07,CMRSY10}, and probably should be
cites for some of that in the future. He is especially fond of interior
crises.

They measure the distance in the usual $L_2$ or `energy' norm. For Lorenz
system, they use the phase of the fixed point eigenvalue to guess the
recurrence time, just as it is done in the ChaosBook.org (DasBuch is
never referred to by anyone, even when they clearly use the notation from
it, as in this paper). Of course, they do not know how to compute this
turn-around time for KS, even though they could have read it off from
Lan\rf{lanCvit07}.They show no recurrence plots; they find only one \po\
(have not checked which one against Lan's
calculations\rf{LanThesis,lanCvit07}), and for that one they show a near
recurrence in a 1\dmn\ time delay plot; out of 4 dips (``valleys''), only
one gives them an initial guess that converges. They find no \rpo s and
waffle about it ``For the case of unstable symmetric solutions apart from
the strange attractor, it is less expected that they will be detected''
(?). Very poor by Lan or Ruslan's standards, and adds nothing to what we
already had published prior to this paper. Wasted at least 4 hours on
reading this. Please find me a really good paper to read...

\item[2011-05-15 Predrag]
Creagh\rf{Creagh94},
{\em Quantum zeta function for perturbed cat maps}, says: ``
The behavior of semiclassical approximations to the spectra of perturbed
quantum cat maps is examined as the perturbation parameter brings the
corresponding classical system into the nonhyperbolic regime. The
approximations are initially accurate but large errors are found to
appear in the traces and in the coefficients of the characteristic
polynomial after nonhyperbolic structures appear. Nevertheless, the
eigenvalues obtained from them remain accurate up to large perturbations.
''
(see CNS Zotero collection)

\item[Recursive Projection Method] (RPM) by
Shroff and Keller\rf{shroff1099} might be of interest to us in numerical
work. RPM stabilizes fixed-point iterative procedures by computing a
projection onto the unstable subspace. On this subspace a Newton or
special Newton iteration is performed, and the fixed-point iteration is
used on the complement. The method is effective when the dimension of the
unstable subspace is small compared to the dimension of the system.
Examples are presented for computing unstable steady states. The RPM can
also be used to accelerate iterative procedures when slow convergence is
due to a few slowly decaying modes. PC also has 2 hard-copy, rather
readable old proceedings reprints by Keller\rf{Keller77,Keller79}. There
are some readable words about homotopy methods in \refref{AllgGeorg88}.


        Here we discuss these continuous symmetries as
a small-step limit of discrete symmetries:

\begin{enumerate}
\item
        symmetries of 3-disk
\item
        $C_n$ symmetry of $n$-slice pizza without reflection symmetry
\item
        Fourier analysis of periodic lattices
\item
        running modes in periodic lattice deterministic
           diffusion
\item
    celestial {\rpo s}
\end{enumerate}

The reading material for 1)-3) is on
http://www.cns.gatech.edu/courses/chaosSpring06

In my \KS\ work with Christiansen, Putkaradze and Lan we
had excluded them ``by hand", by concentrating only on the space of
antisymmetric solutions. That is not good physics, as perturbations off
them mix into the full space of asymmetric solutions.


\item[{\Rpo s} in point vortex systems]

Laurent-Polz\rf{Laurent-Polz04} says: `` We give a method to determine
{\rpo s} in point vortex systems.

perform a symplectic reduction on a fixed point submanifold in
order to obtain a two-dimensional reduced \statesp. The method
is applied to point vortices systems on a sphere and on the
plane, but works for other surfaces with isotropy (cylinder,
ellipsoid, ...). The method permits also to determine some
{\reqva} and heteroclinic cycles connecting these {\reqva}.

{\Reqva} are orbits of the symmetry group action which are in-
variant under the flow, this corresponds here to motions of
point vortices which are stationary in a steadily rotating
frame. The existence and nonlinear stability of {\reqva} formed
of three vortices have been studied respectively in [KN98] and
[PM98].

Periodic orbits on the sphere were determined in [ST,To01]
thanks to the following method: they reduced the system to
two-dimensional systems by a symmetric reduction (using some
finite subgroups of $SO(3)$); the computation of the dynamics
on the reduced spaces permits then to determine periodic
orbits. Our paper is devoted to transpose that method to
determine {\rpo s}. To this end, we combine a symmetric
reduction together with a symplectic reduction.


Relative periodic orbits (RPOs for short) are the analogous of relative
equilibria concerning periodic orbits, this corresponds here to motions
which are periodic in a steadily rotating frame (a precise definition is
given in Section  ref~{methode}).

Let $F_t$ the flow of $X_H$. A point $p\in\p$ is said to be
\emph{periodic} if there exists a constant $T>0$ such that for all time
$t$, $F_{t+T}(p)=F_t(p)$. The \emph{period} is the smallest $T>0$ which
satisfies that condition. The set $\gamma=\{F_t(p)\mid t\in\reals\}$ is
called a \emph{periodic orbit}. Every point of $\gamma$ is periodic with
the same period, hence we can define \emph{the} period of a periodic
orbit.


A point $p\in\p$ is said to be a \emph{relative periodic point} if there
exist $g\in G^o$ and $T>0$ such that $F_{t+T}(p)=g\cdot F_t(p)$ for all
time $t$. The set $\gamma=\{F_t(p)\mid t\in\reals\}$ is called a
\emph{relative periodic orbit} (RPO), and every point of $\gamma$ is a
relative periodic point. In particular, a periodic orbit which is not a
\reqv\ is a RPO.

Note that there exist other equivalent definitions of a relative periodic
orbit  cite~{Or98}.

Typically a RPO is a solution which, in a suitably moving frame, looks time-periodic.

heteroclinic orbit  $\Longrightarrow$  heteroclinic orbit between
relative equilibria
''

Predrag: \Rpo s of this paper seem to be all of finite symmetry group
type, ie they are
all eventually periodic.



[Third Referee's Report Dr Newton]:
The manuscript considers {\rpo s} in point vortex systems that
arise from splitting {\reqv} configurations. It more
comprehensively treats problems of the type considered in
Souliere and Tokieda (2002) and Tokieda (2001). It is a very
nice paper.

Angel Duran, Numerical Integration of Hamiltonian Relative
Periodic Solutions. A First Approach
http://www.math.human.nagoya-u.ac.jp/scicade05/
% angel@mac.uva.es

    [1] B. Cano and A. Duran, A technique to improve the error propagation when inte-
grating relative equilibria, BIT 44(2004) 215-235.
    [2] A. Duran and J.M. Sanz-Serna, The numerical integration of relative equilibrium
solutions. Geometric theory, Nonlinearity 11(1998) 1547-1567.
    [3] E. Hairer, Ch. Lubich and G. Wanner, Geometric Numerical Integration. Structure-
Preserving Algorithms for Ordinary Differential Equations, Springer Series in Comput.
Mathematics, Vol. 31, Springer-Verlag 2002.
	\\
\PC{probably cite:
\rf{Creagh91}
    % author = {Creagh, Stephen C. and Littlejohn, Robert G.},
    % title = {Semiclassical trace formulas
    % in the presence of continuous symmetries},
\rf{Creagh92}
    % author = {Creagh, Stephen C. and Littlejohn, Robert G.},
    % title = {Semiclassical trace formulas
    % for systems with non-Abelian symmetry},
\rf{robb}
    % author = {Robbins, Jonathan M.},
    % title = {Discrete symmetries in periodic-orbit theory},
\rf{mta}
    % author = "Abraham, Ralph and Marsden, Jerrold E. and Ratiu, Tudor",
    % title =  "Manifolds, Tensor Analysis, and Applications",
\rf{Visw07b}
    % author = {Viswanath, Divakar},
    % title = {Recurrent motions within plane Couette turbulence},
    }


\item[2012-02-23 PC]                                            \toCB
I should write to Chris Joyner, Sebastian M\"uller, and Martin Sieber,
alert them to our papers. In
{\em Semiclassical approach to discrete symmetries in quantum chaos},
\arXiv{1202.4998},
they say: ``
We use semiclassical methods to evaluate the spectral two-point
correlation function of quantum chaotic systems with discrete geometrical
symmetries. The energy spectra of these systems can be divided into
subspectra that are associated to irreducible representations of the
corresponding symmetry group. We show that for (spinless) time reversal
invariant systems the statistics inside these subspectra depend on the
type of irreducible representation. For real representations the spectral
statistics agree with those of the Gaussian Orthogonal Ensemble (GOE) of
Random Matrix Theory (RMT), whereas complex representations correspond to
the Gaussian Unitary Ensemble (GUE). For systems without time reversal
invariance all subspectra show GUE statistics. There are no correlations
between non-degenerate subspectra. Our techniques generalize recent
developments in the semiclassical approach to quantum chaos allowing one
to obtain full agreement with the two-point correlation function
predicted by RMT, including oscillatory contributions.
''

It looks different, but parts of it are probably related to my
un-understood and unpublished \refref{Cvi07}.

\item[A. Hramov and A. Koronovskii],
``Detecting unstable periodic spatio-temporal states of spatial
extended chaotic systems,''
\arXiv.org{0708.4349}.

\item[2011-05-15 Predrag]
Wisniacki \etal\rf{WiVeBeBo04},
{\em Classical invariants and the quantization of chaotic systems}, says: ``
Due to their exponential proliferation, long periodic orbits constitute a
serious drawback in Gutzwiller's theory of chaotic systems. Therefore, it
would be desirable that other classical invariants, not suffering from
the same problem, could be used in alternative semiclassical quantization
schemes. In this Rapid Communication, we demonstrate how a suitable
dynamical analysis of chaotic quantum spectra unveils the role played, in
this respect, by classical invariant areas related to the stable and
unstable manifolds of short periodic orbits.
''

\item[2011-05-15 Predrag]
Heusler \etal\rf{HeMuAlBrHa07},
{\em Periodic-orbit theory of level correlations}, says: ``
We present a semiclassical explanation of the so-called
Bohigas-Giannoni-Schmit conjecture which asserts universality of spectral
fluctuations in chaotic dynamics. We work with a generating function
whose semiclassical limit is determined by quadruplets of sets of
periodic orbits. The asymptotic expansions of both the nonoscillatory and
the oscillatory part of the universal spectral correlator are obtained.
Borel summation of the series reproduces the exact correlator of
random-matrix theory.
''

\item[2011-05-15 Predrag]
Ando \etal\rf{AnBoAi07},
{\em Automatic control and tracking of periodic orbits in chaotic systems}, says: ``
Based on an automatic feedback adjustment of an additional parameter of a
dynamical system, we propose a strategy for controlling periodic orbits
of desired periods in chaotic dynamics and tracking them toward the set
of unstable periodic orbits embedded within the original chaotic
attractor. The method does not require information on the system to be
controlled, nor on any reference states for the targets, and it overcomes
some of the difficulties encountered by other techniques. Assessments of
the method's effectiveness and robustness are given by means of the
application of the technique to the stabilization of unstable periodic
orbits in both discrete- and continuous-time systems.
''

\item[2011-05-15 Predrag]
Brack \etal\rf{BrOgYuRe05},
{\em Uniform semiclassical trace formula for
U(3) $\to$ SO(3) symmetry breaking}, says: ``
We develop a uniform semiclassical trace formula for the density of
states of a three-dimensional isotropic harmonic oscillator (HO),
perturbed by a term which term breaks the U (3) symmetry of the HO,
resulting in a spherical system with SO (3) symmetry.
We obtain an analytical uniform trace formula
which in the limit of strong perturbations (or high energy)
asymptotically goes over into the correct trace formula of the full
anharmonic system with SO (3) symmetry, and in the other limit
restores the HO trace formula with U (3) symmetry.
''

This might be useful to us in leading us to physically interesting systems
with non-abelian symmetries. However, have no time today to continue...
%
%Bengtsson I, Br�annlund J and � Zyczkowski K 2002 Int. J. Mod. Phys. A 17 4675
%give example of

\item[2011-05-16 Predrag]
Mark Srednicki has several interesting recent papers on Riemann zeros.
In {\em Nonclasssical Degrees of Freedom in the Riemann Hamiltonian},
\arXiv{1105.2342},
he says: ``
The Hilbert-Polya conjecture states that the imaginary parts of the zeros of
the Riemann zeta function are eigenvalues of a quantum hamiltonian. If so,
conjectures by Katz and Sarnak put this hamiltonian in Altland and Zirnbauer's
universality class C. This implies that the system must have a nonclassical
two-valued degree of freedom. In such a system, the dominant primitive periodic
orbits contribute to the density of states with a phase factor of -1, which
partially resolves a previously mysterious sign problem for oscillatory
contributions to the density of the Riemann zeros.
''

\item[2011-05-16 Predrag] I believe that Ruelle's linear response
theory is fundamentally wrong, as deterministic diffusion transport
coefficients are nowhere differentiable.
Lucarini \etal\ do like it. In
{\em Relevance of sampling schemes in light of Ruelle's linear response
 theory}, \arXiv{1105.2527}, they say: ``
We reconsider the theory of the linear response of non-equilibrium steady
states to perturbations. We first show that by using a general functional
decomposition for space-time dependent forcings, we can define elementary
susceptibilities that allow to construct the response of the system to
general perturbations. Starting from the definition of SRB measure, we
then study the consequence of taking different sampling schemes for
analyzing the response of the system. We show that only a specific choice
of the time horizon for evaluating the response of the system to a
general time-dependent perturbation allows to obtain the formula first
presented by Ruelle. We also discuss the special case of periodic
perturbations, showing that when they are taken into consideration the
sampling can be fine-tuned to make the definition of the correct time
horizon immaterial. Finally, we discuss the implications of our results
in terms of strategies for analyzing the outputs of numerical experiments
by providing a critical review of a formula proposed by Reick.
''

\item[2011-07-04 Predrag]
Vrahatis\rf{Vrah95} claims to know how to compute periodic orbits. He says
``
The accurate computation of periodic orbits of nonlinear mappings and the
precise knowledge of their properties are very important for studying the
behavior of many dynamical systems of physical interest. In this paper,
we present an efficient numerical method for locating and computing to
any desired accuracy periodic orbits (stable, unstable, and complex) of
any period. The method described here is based on the topological degree
of the mapping and is particularly useful, since the only computable
information required is the algebraic signs of the components of the
mapping. This method always converges rapidly to a periodic orbit
independently of the initial guess and is particularly useful when the
mapping has many periodic orbits, stable and unstable, close to each
other, all of which are desired for the application. We illustrate this
method first on a two-dimensional quadratic mapping, used in the study of
beam dynamics in particle accelerators, to compute rapidly and accurately
its periodic orbits of periods p = 1, 5, 16, 144, 1296, 10368 and then
obtain periodic orbits of its four-dimensional complex version for
periods which also reach up to the thousands.
''

There are many articles on their periodic orbit searches
\HREF{http://ChaosBook.org/library/PaVr03.pdf}{this one} on {\em
Computing Periodic Orbits Of Nondifferentiable/Discontinuous Mappings
Through Particle Swarm Optimization} downloadable from his collaborator
Parsopoulos
\HREF{http://www.math.upatras.gr/~kostasp/public.html}{homepage}.

\item[2011-07-22 Predrag]
Gao, Xie and Lan\rf{GaXiLa11}
accelerate convergence of cycle expansions by dynamical conjugacies.

The key idea of this paper, of replacing the stability of an unstable
fixed-point or periodic orbit that a critical point is preperiodic to, by
the root corresponding to the order of the critical point was developed
in detail in a careful study of convergence by Artuso \etal\rf{AACII},
where it is shown that the change in convergence is due to a single fixed
point whose preimage is the critical point. It is shown how to modify the
cycle expansion to fix the convergence. The original $1/\zeta$ is kept,
but the pole induced by the critical point singularity is explicitly
factored out. The method is essentially a quadratic conjugacy restricted
to the critical point (Ulam map to tent map being the trivial example).

The innovation of this paper that goes beyond \refref{AACII} is the
explicit study of natural measures of such maps and use of conjugacies to
excise the singularities in the Ulam map (and its Misiurowicz family
generalization) settings. While \refref{AACII} motivates excision of the
singularity by a detailed study of many families of periodic orbits, the
authors accomplish this more elegantly, by a simple, well designed
conjugacy.

They are looking at the series of generalized Ulam maps, or what Ruelle
calls ``Misiurewicz maps'', where the critical point is preperiodic to,
ie  mapped onto an unstable cycle and thus rendered non-contracting.
Period 2 below is the next in sequence. The interesting new one is the
``Golden mean'' map, see for example exercise 11.6 in ChaosBook.org.
There the critical point is a part of the 3-cycle, so you know (at least
numerically) where the 3 measure singularities are. The answer is
somewhere in the literature. Perhaps in L.~Billings and E.~M.~Bollt,
``Invariant densities for skew tent maps,'' Chaos Solitons and Fractals
12, 365 (2001).

The step in $\rho(x)$ at $x_f$ is suspicious - perhaps the problem is that
the prefactor of the $1/\sqrt{|x-x_f|}$ singularity is different on the two
sides of $x_f$. How can a half-singularity at $x=0$ map into both sides of
$x_f$ neighborhood? Presumably the conjugacy (which they do not explain)
should be piecewise analytic, not smooth as in their figures.

Their $g(x)$ is a bad news - they seem to have introduced infinite slope
at an arbitrary point of the map. It's probably an artifact of the
unexplained method for constructing conjugacies - nothing interesting
happens here in the original dynamics. If they can argue that any periodic
orbit that includes critical point has this problem, that is interesting.
Again, the good conjugacy is probably piecewise analytic - the natural
measure they get has worrisome steps.

Here (and in all finite grammar cases they study) working out the
symbolic dynamics and Markov graphs of this map would help - they have to
understand which cycles form the fundamental set and which (families of)
shadowed cycles are causing wild oscillations in their figures, before
embarking on constructing a conjugacy. Original map has no cycles of
infinite instability, so their subsequent troubles presumably come from a
badly chosen conjugacy.

There can no be analytic conjugacy in higher dimensions - measure
singularities always sit on fractal sets, just observe pictures of
natural measure on the H\'enon attractor. Still, if you one find a
conjugacy that excises the neighborhood of the nearly-attracting
13-cycle, that would deal with the main impediment to zeta function
convergence in this case. The method cannot be generalized to higher
dimensions. For private amusement, just try constructing a 2-d conjugacy
for something like a H\'enon map $\to$ Lozi map. Good luck.

There is immense literature on measures of 1-d maps (A. Boyarski? A.
Lasota and M.C. Mackey\rf{LM94}? G. Froyland? E. Bollt? J. M.
Aguirregabiria, Robust chaos with prescribed natural invariant measure
and Lyapunov exponent; \arXiv{0907.3790}? D.J. Driebe, Fully Chaotic Map
and Broken Time Symmetry (Kluwer, 1999)? ... and authors would profit
from using some of that work to illustrate their ideas. They should do a
literature search on measures of 1-d maps. This is reminiscent of work on
natural measures published since 1980's by Hungarian school (Szapfalusy,
Tel, ...), Bollt, and many others.

1-d maps with a single critical point are very special, and unfortunately
little of this is useful in higher dimensions - already for the H\'enon
attractor there is a fractal set of critical points (ie, stable-unstable
manifold tangencies) and their images. No conjugacy or a finite set of
conjugacies can help there...


\item[2011-07-26 Predrag]
Katsanikas, Patsis and Pinotsis% \etal\rf{XXX},
{\em Chains of rotational tori and filamentary structures close to
  high multiplicity periodic orbits in a 3D galactic potential},
  \arXiv{1103.3981}, might be of interest - it struggles with visualization
of tori in 4\dmn.

\item[2011-08-24 Predrag]                                       \toCB
Liao\rf{Liao09,WaLiLi11,Liao11,Liao11a} is able to integrate Lorenz
equations with the 800-digit precision, using Mathematica with the
400th-order Taylor expansion for continuous functions. He gets ``clean''
numerical simulation of chaotic solution of Lorenz equation in a long
interval $0\leq t \leq 1000$ LTU (Lorenz time unit) with negligible
truncation and round-off error.   He found that,  to gain a solution in
$0\leq t \leq T_c$,  the initial conditions must be at least in the
accuracy of $10^{-2T_c/5}$.   Thus,  when $T_c = 1000$ LTU,  the initial
condition must be  in the accuracy of 400-digit precision at least. The
averaged velocity fluctuation in  a cube meter of fluid is $3.722 \times
10^{-30}$ m/s, so he starts simulations with initial standard deviation
$\sigma=10^{-30}$ (not clear why Lorenz origins in fluid dynamics justify
precisely $\sigma=10^{-30}$, but never mind) and integrate
deterministically (that is wrong too - integration should be
Fokker-Planck, not deterministic). The effects kicks in at 80 LTU, and by
120 LTU initial conditions are forgotten. ``This strongly suggests that
chaos builds a bridge from the micro-level uncertainty to   macroscopic
randomness, and  thus  is an origin of macroscopic randomness and  time's
arrow.''

\item[2011-09-15 Predrag]
Gutkin and Osipov, %\rf{XXX},
\arXiv{1109.3329}, say: ``
By considering symbolic dynamics of the system one can introduce a
natural ultrametric distance between periodic orbits and organize them
into clusters. Each cluster consists of orbits approaching closely each
other in the phase space. We study the distribution of cluster sizes for
the baker's map in the asymptotic limit of long trajectories. This
problem is equivalent to the one of counting degeneracies in the length
spectrum of the {\it de Bruijn} graphs. Based on this fact, we derive the
probability $\P_k$ that $k$ randomly chosen periodic orbits belong to the
same cluster, find asymptotic behaviour of the largest
cluster size $|Cl_{\max}|$ and derive the probability $P(t)$ that a
random periodic orbit belongs to a cluster of the size smaller than
$t|Cl_{\max}|$, $t\in[0,1]$.
''

\item[2011-10-09 Predrag]
Klebanoff and E. Bollt\rf{KlBo011} \emph{Convergence analysis of
{Davidchack and Lai's} algorithm for finding periodic orbits} might be
worth a read. Here are also two Guckenheimer\rf{ChoGuck99,GM00aut} on
computing periodic orbits.

This might be of interest to Chandrites: Olvera and Vargas\rf{OlVa94},
\emph{A continuation method to study periodic orbits of the Froeschl\'e
map} write: `` The dynamics of many Hamiltonian systems with three
degrees of freedom is represented by the Froeschl\'e map which is
symplectic and four-dimensional. In this paper we study sequences of
periodic orbits approaching the invariant tori.''

\item[2011-07-19 Yueheng Lan]
Recently, I am working on the intermittency problem with periodic orbit
theory. We are trying to find a practical way for effectively doing
computations with cycle expansions. Besides the 1-d intermittency map, is
there a 2-d or 3-d system that is both intermittent and very closely
related to physics? We want our scheme to be tested in a real physical
situation.

I met a professor at a systems biology meeting who mentioned a way for
searching for periodic orbits with synchronization. It seems that no
Jacobian is needed at all in their computation. Please take a look at Lin
\etal\rf{LiMaFeCh10}, \emph{Locating unstable periodic orbits: {When}
adaptation integrates into delayed feedback control}.  This may provides
another approach for cycle detection. I do not know how efficient this
method is.

(Predrag put a copy into the ChaosBook.org/library,
\HREF{http://ChaosBook.org/library/LiMaFeCh10.pdf}{click here}.
Superficial impression - does not look serious.).

(Predrag: while I am at it,
\HREF{http://ChaosBook.org/library/goldstein01.djvu}{click here} for a
pirated copy of Goldstein\rf{goldstein01} \emph{Classical Mechanics} 3.
edition,
and
\HREF{http://ChaosBook.org/library/Jammer66.djvu}{click here}  for a
pirated copy of Max Jammer\rf{Jammer66}
     \emph{The conceptual developement of quantum mechanics}
)

\item[2012-02-08 Predrag] Daniel Borrero ran into this: Kourosh
Zarringhalam\rf{Zarrin06} 2006
\HREF{http://gradworks.umi.com/32/31/3231362.html}{PhD thesis}
\emph{CUPOLETS: Chaotic unstable periodic orbits theory and
applications}. You can download the copy from there - just in case, I put
a copy in \HREF{http://ChaosBook.org/library/Zarrin06.pdf}{here}. Never
published, so probably safely ignored. The emphasis is on wavelets, and
leaves me cold. He writes:

``[...] abundance
of unstable periodic orbits can be utilized in a wide variety of
theoretical and practical applications [19]. In particular, chaotic
communication techniques and methods of controlling chaos depend on this
property of chaotic attractors [12, 13]. In the first part of this
thesis, a control scheme for stabilizing the unstable periodic orbits of
chaotic systems is presented and the properties of these orbits are
investigated. The technique allows for creation of thousands of periodic
orbits. These approximated chaotic unstable periodic orbits are called
cupolets (Chaotic Unstable Periodic
Orbit-lets). We show that these orbits can be passed through
a phase transformation to a compact cupolet state that possesses a
wavelet-like structure and can be used to construct adaptive bases. The
cupolet transformation can be regarded as an alternative to Fourier and
wavelet transformations. In fact, this new framework provides a continuum
between Fourier and wavelet transformations and can be used in variety of
applications such as data and music compression, as well as image and
video processing. The key point in this method is that all of these
different dynamical behaviors are easily accessible via small controls.
This technique is implemented in order to produce cupolets which are
essentially approximate periodic orbits of the chaotic system. The orbits
are produced with small perturbations which in turn suggests that these
orbits might not be very far away from true periodic orbits. The controls
can be considered as external numerical errors that happen at some points
along the computer generated orbits. This raises the question of
shadowability of these orbits. It is very interesting to know if there
exists a true orbit of the system with a slightly different initial
condition that stays close to the computer generated orbit. This true
orbit, if it exists, is called a shadow and the computer generated orbit
is then said to be shadowable by a true orbit. We will present two
general purpose shadowing theorems for periodic and nonperiodic orbits of
ordinary differential equations. The theorems provide a way to establish
the existence of true periodic and non-periodic orbits near the
approximated ones. Both theorems are suitable for computations and the
shadowing distances, i.e., the distance between the true orbits and
approximated orbits are given by quantities computable form the vector
field of the differential equation.
''

\item[2012-04-30 Predrag]                           \toCB
Might want to read {beim Graben} and Potthast\rf{GrPo12}. They cite
\refref{pre88top,BuKe03}, and say: ``
Moore\rf{Moore90,Moore91} has proven
that a Turing machine can be mapped onto a generalized shift as a
generalization of symbolic dynamics\rf{LindMar95}, which in turn becomes
represented by a piecewise affine-linear map at the unit square using
G\"odel encoding and symbologram reconstruction \refref{pre88top,BuKe03}. These
nonlinear dynamical automata have been studied and further developed
by us.
''

\item[2012-06-27 Predrag]                                       \toCB
Might want to suggest a look at ChaosBook.org to
Guralnik,  Pehlevan, and Guralnik, %\rf{XXX},
{\em On the asymptotics of the Hopf characteristic function},
\arXiv{1201.2793}, though we have never looked at $\beta$ large limit.

They write:

\renewcommand{\ssp}{x}
The invariant measure $\mu$ determines the frequency with which the a
chaotic trajectory may be found in a region of \statesp. The \emph{Hopf
characteristic function} is the Fourier Stieltjes transform of this
measure. For  a  dynamical system with \statesp\ variables
$\ssp_1,\cdots,\ssp_d$, the characteristic function $Z(\beta)$ is
defined\rf{hopf52,frisch} as the time average of \nobreak
$\exp\left(i\beta\cdot \ssp(t)\right)$
\begin{align}
Z(\beta)\equiv\left< \exp\left(i{\beta}\cdot{\ssp(t)}\right) \right>
 =\int \dMsr(\ssp)\exp\left(i{\beta}\cdot\ssp\right)\,,
\end{align}
where $\dMsr(\ssp)$ is the formal representation of the invariant
measure. The small $J$ behavior of the Hopf characteristic function
determines the equal time correlation functions of a chaotic attractor,
via the Taylor--Maclaurin series expansion in $\beta$. (Hopf
article\rf{hopf52} can be read
\HREF{http://www.iumj.indiana.edu/IUMJ/dfulltext.php?year=1952&volume=1&artid=51004}
{here}.)
The large $J$ behavior is related to the geometry of the attractor,
specifically its dimensionality. [...] the impossibility of defining  a
finite distribution function on the attractor or fractal.

When the invariant measure can be written in terms of a probability
density function, $\dMsr(\ssp)=\msr(\ssp)d^d X$,  where $\msr(\ssp)$
is everywhere finite, $Z(\beta)$ and $\msr(\ssp)$ are Fourier
transforms of each other. However in many systems of interest, the
dynamics is dissipative and a probability density function which is
everywhere finite does not exist, although it may exist as a
distribution.

If an everywhere finite $\msr$ did exist, it would satisfy
$\frac{d\msr}{dt} = -\msr \vec\nabla\cdot\vec v >0$ along any trajectory,
where $\vec v(\ssp)$ is the velocity $\frac{d\ssp}{dt}$, which is
inconsistent with Poincar\'e recurrence.

A sufficient condition for the existence of a Fourier transform is that
$Z(\beta)$ be ${\cal L}^2$, such that the integral $\int d^d J\,
Z(\beta)^*Z(\beta)$ converges.  Thus it would seem natural to define a
dimensionality corresponding to the maximum value of $s\le n$ such that
the integral
\begin{align}\label{defcon}
I_s=\int_{|J|>\epsilon} d^d J\, |J|^{s-n} |Z(\beta)|^2
\end{align}
converges.


The absence of a distribution function means that the measure $\dMsr$ can
not be written as $\msr(\ssp)d\ssp$ for any function $\msr(\ssp)$. In
this case the Fourier transform of the characteristic function, $\int
d\beta Z(\beta)\exp(i\beta\cdot \ssp)$, can not converge. [...] The
relationship between fractal measures and the asymptotics of their
Fourier-Stieltjes transform has been discussed in

R.~Ketzmerick, G.~Petschel and T.~Geisel, {\it Slow decay
of temporal correlations in quantum systems with Cantor spectra},
Phys.Rev.Lett {\bf 69} (1992) 695--698.

where it was shown that
\begin{align}
\frac{1}{J}\int_0^J dJ'\, |Z(J')|^2 \sim J^{-{\cal D}_2}
\end{align}
in the limit of large $J$, where ${\cal D}_2$ is the correlation
dimension of the fractal measure.

\renewcommand{\ssp}{a}

For incompressible flows is
Hopf\rf{hopf52} notes that
\beq
Z(\beta) = Z(\beta + \partial\phi)
\ee{HopfChFctIncmpr}
for any single-valued scalar function $\phi(x)$ which vanishes on the
boundary of integration volume. He notes many other things, refer to
``his pupil,'' \etc, - worth a look for what a real scholarly paper used
to look like.


\item[2012-07-18 Predrag]                                       %\toCB
De Paula \etal\rf{DePSWP12},
{\em Bifurcation control of a parametric pendulum}
(or
\HREF{http://www.worldscientific.com/doi/pdfplus/10.1142/S0218127412501118}
{click here}) give a good overview of chaos control techniques. They too
cite only Auerbach \etal\rf{pchaot} as the paper to cite on identifying
`UPOs'.

\item[2012-07-24 Predrag] Added Allahem and Bartsch\rf{AllBar12} {\em
Chaotic dynamics in multidimensional transition states} to
\HREF{http://ChaosBook.org/library/AllBar12.pdf}{ChaosBook.org/library}.


\item[2012-08-27 Predrag]
Takeuchi, Kazumasa and Sano, Masaki\rf{TaSa07} write: ``
The thermodynamic formalism for dynamical systems with many degrees of
freedom is extended to deal with time averages and fluctuations of some
macroscopic quantity along typical orbits, and applied to coupled map
lattices exhibiting phase transitions. Thereby, it turns out that a seed
of phase transition is embedded as an anomalous distribution of unstable
periodic orbits, which appears as a so-called q-phase transition in the
spatiotemporal configuration space. This intimate relation between phase
transitions and q-phase transitions leads to one natural way of defining
transitions and their order in extended chaotic systems. Furthermore, a
basis is obtained on which we can treat locally introduced control
parameters as macroscopic `temperature' in some cases involved with
phase transitions.
''

This might be related to Predrag's phase transitions\rf{ACK89}; recheck.
Amusingly,
Benza and V. Callegaro\rf{BeCa80} ripped off the title of my lectures:
\emph{Phase transitions on strange sets: the {Ising} quasicrystal};
``A quantum Ising spin chain with nearest-neighbour couplings
  arranged in a quasiperiodic sequence is considered. The Cantor set
  structure of the energy spectrum is analysed in terms of the
  thermodynamic description of multifractals. Evidence is given that the
  spectrum of scales develops a singular behaviour: this is associated
  with a first-order phase transition of a new type. It is argued that
  this effect involves, not only quantum spins, but the whole class of
  phonon-like propagation problems on quasiperiodic chains.
  ''
and I never noticed. Should read this one too.


\item[2012-08-27 Predrag]
Koh, Yang Wei and Takatsuka, Kazuo\rf{KohTak07} are people for whom $d=9$
is a high dimension. They write:
``Methods to search for periodic orbits are usually implemented with the
Newton-Raphson type algorithms that extract the orbits as fixed points.
When used to find periodic orbits in flows, however, many such approaches
have focused on using mappings defined on the Poincar\'e surfaces of
section, neglecting components perpendicular to the surface of section.
We propose a Newton-Raphson based method for Hamiltonian flows that
incorporates these perpendicular components by using the full
{\monodromyM}. We investigated and found that inclusion of these
components is crucial to yield an efficient process for converging upon
periodic orbits in high dimensional flows. Numerical examples with as
many as nine degrees of freedom are provided to demonstrate the
effectiveness of our method.''

The only care about Hamiltonian systems. Should
\HREF{http://ChaosBook.org/library/KohTak07.pdf} {check it out}.

\item[2012-08-27, 2012-09-26 Predrag] Read
Saiki\rf{Saiki07} on ``detection of unstable periodic
orbits in continuous-time dynamical systems;''
\HREF{http://ChaosBook.org/library/Saiki07.pdf}{check it out}.
Only thing we should cite it for is finding 1000+ Lorenz \po s.
He writes: ``
we focus our attention to the Newton-Raphson-Mees
method\rf{Mees81,ParChu89} (maybe we should buy Parker and Chua) with a
damping coefficient, where we do not need to have trouble of choosing the
appropriate Poincar\'e section.
[...]
A disadvantage is that the sorts of UPOs which can be detected by this
method are limited by the instability and the period of UPOs and the
numerical accuracy.
''                                                  \toCB
The Newton-Raphson-Mees method is described in detail in this paper. The
constraint that he credits Mees\rf{Mees81} for is that instead of using a
Poincar\'e section one constrains Newton step guess initial point
increments $\Delta \ssp^{(j)}$ to be orthogonalized to the orbit,
\beq
\braket{\vel{\xInit}}{\Delta \ssp^{(j)}}= 0
\,.
\ee{MeesNewton}
We do that in ChaosBook.org, should remember to credit Mees.
He tests it on Lorenz system, determine more than 1000 \po s. No attempt
is made to use averaging formulas over \po s. This might be worth a
second look: he proposes that the damping parameter be selected according
to the stability exponent $\Lyap$ (Floquet exponent) and the period
\period{} of a periodic orbit of the dynamical system, his Eq.~(15). For
Floquet exponents he cites  Chicone\rf{Chicone2006}.

He credits Rempel and Chian\rf{ReCh07} for finding chaotic saddles of the
complex system described by a PDE, which generates spatio-temporal
chaotic behavior. Need to read it.

We also need to look at Saiki and Yamada\rf{YamSai97,SaiYam98}

\item[2012-08-27 Predrag]
And then there is Mainieri\rf{Mainieri04}.



\item[2012-09-14 Predrag]
Read Charney and DeVore\rf{ChaDeVore79}. Hugely cited, huge amount of
blah-blah, amazingly primitive from dynamical systems perspective - I
wonder how Lorenz felt about it. Let's not ever cite it. It is a variant
of Lorenz 2-layer model\rf{Lorenz63a}. They make a point that a
barotropic channel model can have several equilibria, some unstable. Big
surprise. They butcher the model and keep 6 (six!) Fourier modes; \eqv\
conditions then yield a cubic equation. The {\stabmat} \Mvar\ is
[$6\!\times\!6$] but block diagonalized by symmetry into two
[$3\!\times\!3$], thus the eigenvalues are generically a real $\eigExp$
and a complex conjugate pair $\eigRe \pm i\,\eigIm$. They really really
do not understand that linearized stability eigenvectors of \eqva\ point
in whatever direction in the \statesp. They compute a particular
situation where there are 3, one that we would call `laminar' or lower
state, almost linear in Fourier modes, the one unstable (meteorologists
call it metastable `blocking' state) in-between that we would think of as
edge state, and one we would call `upper branch' that is unstable. A
window into the mind of the older, pre-\statesp\ generation.

\item[2012-09-14 Predrag]
Charney and DeVore\rf{ChaDeVore79} cite
H. Scarf\rf{Scarf67}
for a method to locate at least one \eqv\ of a continuous mapping. He says
``Brouwer fixed point theorem\rf{Brouwer1} states that
a continuous mapping of a simplex into itself has at least one fixed point.
''

\item[2012-09-26 Predrag]
Read Giona and Cerbelli\rf{GioAdr98,GioCer05} {\em Connecting the spatial
structure of periodic orbits and invariant manifolds in hyperbolic
area-preserving systems}.

\item[2012-08-12 PC]
Georgiou, Dettmann and Altmann\rf{GeDeAl12}
\emph{Faster than expected escape for a class of fully chaotic maps},
\arXiv{1207.7000}, have a potentially interesting trace formula. Am I
minced meat? Remind them to cite ChaosBook.org. Funny thing is, Dettmann is a
contributor to it. Is he ashamed of his youthful digressions?

\item[2012-10-10 Predrag]
For a theorist like me, Avlund et al.\rf{AEOGS10} experimental / theoretical
paper
\emph{Observation of periodic orbits on curved two-dimensional geometries}
is a remarkable achievement. One author is my former PhD student, three
others are my collaborators. The first sentence of the paper says
``semiclassics, in particular, periodic orbit theory, provides a powerful
connection to the dynamics of the analogous classical system [1-4].'' Of
the four textbooks cited, one is on periodic orbit theory. What is this
fear of ChaosBook.org? Am I minced meat? Do I have advanced halitosis?
Can somebody explain to me where ChaosBook.org falls short?

\item[2012-11-12 Predrag] I am saving various Francesco photos of white
boards, \po\ solutions etc. in  directory \texttt{pipes/Fedele/}, not all
of if checked into svn repository \texttt{pipes}, so ask me if you want
to have a look. These emails are not working for me - there are about 30
emails between Francesco and Roman in my inbox, and there is no way to
read them in a coherent way or extract information from them. It's
hopeless, it's almost like doing science by Feysbookistan posts.

\item[2012-07-19 Predrag]
Petalas \etal\rf{PePaVr08}
and other such papers find periodic orbits by minimizing error functions,
the same idea as Lan's method\rf{CvitLanCrete02,lanVar1}.

\item[2012-09-26 Predrag]
Read Gao \etal\rf{GGLTL09} \emph{... unstable periodic orbits of
nonlinear mappings by a novel quantum-behaved particle swarm
optimization}: they seem to have yet another variational method for
finding \po s. Of course, no single reference to Copenhagen or to
Lan\rf{CvitLanCrete02,lanVar1}.

\item[2012-09-13 Predrag]
Kazantsev also obsesses about `sensitivity of attractors' to
perturbations\rf{Kazantsev01,Kazantsev01a}; that might be worth including into
ChaosBook.org and evaluating using correct cycle expansion formulas
(read the paper
\HREF{http://www.nonlin-processes-geophys.net/8/281/2001/} {here}).
He writes: ``
A description of a deterministic chaotic system in
terms of unstable \po s is used to develop
a method of an a priori estimate of the sensitivity of statistical
averages of the solution to small external influences.
This method allows us to determine the forcing perturbation
which maximizes the norm of the perturbation of a statistical
moment of the solution on the attractor. The method was
applied to the barotropic ocean model in order to determine
the perturbation of the wind field which provides the greatest
perturbation of the model's climate. The estimates of perturbations
of the model's time mean solution and its mean
variance were compared with directly calculated values. The
comparison shows that some 20 UPOs is sufficient to realize
this approach and to obtain a good accuracy.

[...] To be able to determine the ``most dangerous'' external influence
to the climatic model, one has to solve an inverse problem and use an a
priori technique in the prediction of climate changes. One way of
developing an a priori technique is based on fluctuation-dissipation
relation, obtained by Kraichan (1959) for Hamiltonian systems and used in
Branstator and Haupt (1998), Gritsoun and Dymnikov (1999), Gritsoun
(2001) for climate models with attractors. This approach is rather simple
and can easily been realized, but its precision provides only a
qualitative understanding of the problem.

Another way to develop an a priori technique and to obtain
a better precision is based on the periodic orbit theory.
[...]
In practice, we cannot take into account all periodic orbits.
However, only a limited number of low-period orbits
may be sufficient for some purposes. This point of view is
argued in Hunt and Ott (1996b), Hunt and Ott (1996a). Numerous
recent studies confirm this hypothesis.
[...]
The accuracy obtained
in the approximation is rather good even when the number
of UPOs used is not large. Thus, the attractor of the Lorenz
model, its dimension, Lyapunov exponents
and Lyapunov vectors were characterized by the UPO set in
Franceschini et al. (1993), Eckhard and Ott (1994) and Trevisan
and Pancotti (1998). Attractor dimension and statistical
averages of the barotropic ocean model have been approximated
by UPOs in Kazantsev\rf{Kazantsev98}.

[...]
Sparrow (1982) proposed to use the
Newton method to locate unstable periodic orbits on the attractor
of the Lorenz model.
This method requires $O(d^3)$ operations
per iteration due to the requirement to calculate the
matrix of the Newton process and to solve the system of
equations with this matrix. Therefore, the method is limited
for use only with low-dimensional systems. To find UPOs
of high-dimensional systems, one can use the method proposed
in Kazantsev (1998). This method requires as many
operations per iteration as the model does (usually  $O(d^{3/2})$
or even $O(d \ln d)$).

In this paper, we describe an algorithm of evaluation of the
variations in statistical averages of the model solution caused
by perturbations in the right-hand side of the model. These
perturbations are suppose to be small and the linear response
of the model is considered.
''
This might be the same as the open project on ChaosBook.org/project page.

Probably we have to read Chian \etal\rf{CSRBHK07} as well (fetch it
\HREF{http://www.nonlin-processes-geophys.net/14/17/2007/} {here}).
These guys have written a paper with Freddy\rf{CRMRF02}, so maybe
they know what they are doing.

\item[2012-08-13]
Study Kazantsev (1998)\rf{Kazantsev98}
\emph{Unstable periodic orbits and attractor of the barotropic ocean
model}, Kazantsev (2001)\rf{Kazantsev01} \emph{Sensitivity of the
attractor of the barotropic ocean model to external influences: approach
by unstable periodic orbits}. Compare his `minimisation procedure' to
ours. Also reference dasbuch/WWW/extras/BB-SkeletonChaos.htm and
Fazendeiro, Boghosian\rf{BFLTC11,BBLTFC11}. Fazendeiro I met in Trieste
2009, see lectures/Trieste09 `grades'. Cite all papers of this paragraph
in ChaosBook.org.

\item[2012-08-28 Francesco] Check Boghosian \etal\rf{BFLTC11}, \emph{New
variational principles for locating periodic orbits of differential
equations} (\HREF{http://ChaosBook.org/library/BFLTC11.pdf}{click here}),
for  another variational variant for periodic orbits. They write: ``
We present new methods for the determination of periodic
orbits of general dynamical systems. Iterative algorithms for finding
solutions by these methods, for both the exact continuum case, and for
approximate discrete representations suitable for numerical
implementation, are discussed. Finally, we describe our approach to the
computation of unstable periodic orbits of the driven Navier-Stokes
equations, simulated using the lattice Boltzmann equation.
''

They relax the enforcing of the dynamics trying to find the orbit that is
minimized the mismatch $F=||xdot-v(x))||^2$ .. a true orbit is the
absolute minimum $F=0$ ... so near the minimum one gets orbits with a
small random forcing ..... note that here the periodic orbit is imposed
exactly, $x(0)=x(T)$.

\item[2012-11-12 Predrag]
\HREF{http://gradworks.umi.com/3541851.pdf} {This thesis} by
\HREF{http://hilbert.math.tufts.edu/~bruceb/AcademicLineage/index.html}
{Boghosian}'s
student Spencer Smith\rf{SmiBog11,SmithThesis12} is interesting (Google scholar
alerts me to such citations! Cool.) He writes:

``
Another method that we
tried was based on a paper by Cvitanovi\'c\rf{lanVar1}. The main idea of
this paper is to use an artificial periodic orbit, a loop in phase space
that does not necessarily follow the local velocity vector JdH, as a
starting point for reduction. At each point on this loop, there might be
a discrepancy between the dynamical velocity vector and the vector
tangent to the loop itself. One can then use the length of the difference
between these two vectors (suitably scaled), integrated over the loop, as
a cost functional over all possible loops. Then a discretization of the
loop will allow the use of Newton's method to reduce this cost function
and hopefully end up with a true periodic orbit. This method worked
sporadically, and we eventually abandoned it. The method that ended up
working very well is a modification of a method outlined in
\refrefs{BFLTC11,BBLTFC11}. It has much in common with \refref{lanVar1},
but treats the period as a main variable to be varied, among other
differences. We will go over this method and the modifications to it in
detail, but first we will describe how to generate close returns to feed
to this method.
''

The he seems to do what I wanted Lan to do but Lan would not:

``For four vortices and a typical loop discretization of two hundred
points, this amounts to a Hessian matrix with over 2.5 million elements.
It is easy to see that the size of the matrix and therefore the
computational complexity of solving the linear equation, can quickly
become prohibitive. Fortunately, this matrix is very sparse as well as
symmetric'' [Predrag: see his Eq.~(5.65)]. ``Thus, the number of non-zero
entries is $4N_v (3N_v + 1)N_d + 1$ which is only linear in Nd, and
therefore much less than the total number of possible entries $(2N_vN_d +
1)^2$. For four vortices and two hundred loop points, only 1.6\%\ of the
entries are non-zero, and the matrix is certainly sparse.''

Smith then accounts for all of the marginal directions and does all the
right stuff. His is a Hamiltonian system, so it is more complicated than
what we need for our PDEs.  All in all, a very impressive thesis. The
problem arises at the end, and is the usual one - he gets lots of \po s
and has no good way of visualizing them together. I went there, gave them
talk about \statesp\ visualizations, but -again!- it did not sink in.

And
\HREF{http://gallery.bridgesmathart.org/exhibitions/2011-bridges-conference/spencersmith}
{here is} Spencer's art. More such is
\HREF{http://gallery.bridgesmathart.org/exhibitions} {here}.

\item[2012-12-09 Predrag]
Berry writes in \refref{Berry12}
(\HREF{http://www.sps.ch/uploads/media/Mitteilungen.37.pdf} {click
here}): `` A related idea was the invention of cycle expansions by
Predrag Cvitanovic and Bruno Eckhardt; in these, essential use is
made of symbolic dynamics to speed the convergence of the sum over
orbits. ''

I've been one of the most active fans and promoters of Gutzwiller,
and Sir Michael is a friend - so how did Eckhardt become a
co-inventor of cycle expansions? I invented them in 1986, published
in 1988 and Eckhardt then joined me in writing three good papers
about them (one on
\HREF{http://www.cns.gatech.edu/~predrag/papers/preprints.html\#CE89}
{semi-classics}, the other on
\HREF{http://www.cns.gatech.edu/~predrag/papers/preprints.html\#smoothCycl}
{deterministic flows}, the third on
\HREF{http://www.cns.gatech.edu/~predrag/papers/preprints.html\#EckhardtSymm}
{symmetries}). Usually the Maryland group makes a sport of not citing
me, but Sir Michael?

The problem with Gutzwiller and Berry is that they never understood
that the same theory works for the deterministic and quantum chaos.
The problem with Ruelle is that he never understood that the same
theory works for the deterministic and quantum chaos. They will die
convinced that they made totally unrelated discoveries. Only
\HREF{http://chaosbook.org/FieldTheory/quefithe.html} {the snail}
understands.

It is hopeless. Someday someone will actually read ChaosBook.org and
theoretical physics will be saved.

\item[2012-12-13 Predrag]               \toCB
Reading Berry\rf{Berry12}:
``Martin Gutzwiller has published about 40 papers, most of them
alone.'' He published a series of four papers [1-4] on periodic
orbits: \refrefs{gutzwiller67,gutzwiller69,gutzwiller70,gutzwiller71}.
``The most influential was the
last one\rf{gutzwiller71}, containing the celebrated `Gutzwiller trace formula'.
That was a tricky calculation, based on the Van Vleck
formula for the semiclassical propagator, giving the density
of quantum states (actually the trace of the resolvent operator)
as a sum over classical periodic orbits. In particular,
Martin calculated the contribution from an individual unstable
periodic orbit.''
``In 1974, Jacques
Chazarain\rf{Cha74} showed that the trace formula could be operated
'in reverse', so that a sum over energy levels generated a
function whose singularities were the actions of periodic
orbits. This was exact, not semiclassical, and led (often
unacknowledged) to what later came to be called 'inverse
quantum chaology' and 'quantum recurrence spectroscopy'.''
``
In 1975 Michael Tabor and I generalized some of the
results in the first of Martin's semiclassical papers\rf{gutzwiller67} to
get the general trace formula for integrable systems, where
the periodic orbits are not isolated but fill tori\rf{BerTab76}. In nuclear
physics, similar formulas had been obtained by Strutinsky\rf{Strut68a,Strut68b}
in the context of the shell model. Tabor and I used our result
to show that the level statistics in integrable systems are
Poissonian - more about that later. William Miller and Andr�
Voros resolved a puzzle about the application of the trace
formula for a stable orbit: by properly quantizing transverse
to the orbit, they restored the missing quantum numbers;
then Martin's single-orbit quantization rule makes sense,
as the 'thin-torus' limit of Bohr-Sommerfeld quantization.
,,

some other web thingie: ``
The Duistermaat-Guillemin trace formula\rf{DuiGui75} (see also
Colin de Verdi\`{e}re\rf{CdV73} and Chazarain\rf{Cha74})
shows that, for manifolds of strictly negative sectional curvature
the spectrum controls the length of the
shortest closed geodesic. This follows because the wave trace has
singularities at the lengths of closed geodesics, and the curvature assumption
guarantees that the coefficient of the leading singularity is nonzero.
''

\item[2013-01-16 Predrag] More citation gripes: Today Carl pointed
out \HREF{http:\\homepages.warwick.ac.uk/~masdbl/periodic-escape6.pdf}
{this paper} that seems to redo our 1988-95 papers without bothering
to cite even one. Couldn't they at least cite ChaosBook? Go figure...

\item[2008-08-28 Predrag]
Altmann and Tel\rf{AltTel08} give a detailed study
of escape rates, with citations to more recent literature.
 They state only the not-so-good trace formula for
 escape rates and go at great length not to refer to Predrag's
 work on periodic orbits computation of escape rates\rf{inv,AACI,DasBuch}.
 Tel has opinions, and one of them is that there is no point in understanding
 the periodic orbit theory as it can all be run on a computer anyway.

\item[2013-01-26 Predrag]                           \toCB
Dan Lathrop says that 1989 Lathrop and Kostelich\rf{LaKo89}
``Characterization of an experimental strange attractor by periodic 	
orbits,'' his first, and still most cited article (Google Scholar
gives it 319 citations) was inspired by my Dynamics Days talk. They
extract 8 \po s and their single expanding Floquet multiplier from a
chaotic Belousov-Zhabotinsky experimental time series of 65,000
bromide ion concentrations, 3\dmn\ time delay representation.

I reference the paper in chapter \emph{Fixed points, and how to get
them}, \\
\wwwcb{paper.shtml\#cycles}, \\
but do not explain why - have to fix that.

I gave a talk at Dynamics Days, Houston, Texas, 6 Jan 1988, and then
U. Texas Nonlinear Physics Seminar, Austin, Texas, 12 Jan 1988. They
do
% mention me in acknowledgments (as one of the people they talked to) and
cite 1987 Auerbach \etal\rf{pchaot} (473 Google Scholar
citations) and 1988 Cvitanovi\'c\rf{inv} (336 Google Scholar
citations), as well as the ubiquitous later 1988 paper by Grebogi,
Ott and Yorke\rf{yorke2}  (298 Google Scholar citations).

I should check whether this the talk in which Ruelle followed as the
next speaker after me, and when asked: ``Can dynamical zeta functions
be used for anything?'' had to think for a while and then answer:
``Perhaps for counting knots?''.

But I am almost sure it was  Dynamics Days, Austin, 8 Jan 1992;
because I will never forget the dinner in which Ruelle explained to
Swinney why people marry people of different race.

\item[2013-01-27 Predrag]               \toCB
Incorporated remarks from Holmes\rf{HHKR50yChaos} into ChaosBook history
appendix.

%\item[2012-XX-XX Predrag]
%XXX \etal\rf{XXX},
%{\em XXX}, says: ``
%XXX
%''
% (see CNS Zotero collection)


\end{description}

\renewcommand{\ssp}{a}             % PDE state space point
