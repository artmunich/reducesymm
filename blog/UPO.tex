% siminos/blog/UPO.tex
% $Author$ $Date$

\chapter{Periodic orbit theory}
\label{chap:UPO}

This part of the blog collects references to periodic orbit theory.
Some of the articles are in the CNS Zotero collection.

As material is written up, parts of it will migrate from its
current placement into coherent sections, suitable for
inclusion into ChaosBook.org, or, God Forbid, actual {\em
publications}.


\section{Symmetries imply possible existence of \rpo s}
\label{sec:SymRPO}
% Ruslan edited                                 2007-03-11
% Predrag  eliminated siminos/blog/symRLD.tex   2009-10-09

In a dynamical system $\dot{a} = v(a)$ with a strange
invariant set, there exist infinitely many periodic orbits
\[ f^\period{}(a) = a \]
characterized by period $\period{}$, which are dense within
the invariant set. Here $f^t$ is the flow map of the flow
$v$, \ie, $a(t) = f^t(a)$ is the solution of the flow $v$
with initial condition $a(0) = a$.

Let the dynamical system have symmetries represented by the operators
$\Sigma_{k,s}$, where $k \in {\cal K} \subset \mathbb{Z}^p$ are
parameters of discrete symmetries and $s \in {\cal S} \subset \mathbb{R}^q$
are parameters of continuous symmetries.  In other words,
\[ f^t(\Sigma_{k,s} a) = \Sigma_{k,s} f^t(a)\,. \]
In this case it is likely that, in addition to \po s, the dynamical system also
has \rpo s, characterized by the condition
\[ \Sigma_{k,s}f^\period{}(a) = a\,, \]
where, in addition to the period $\period{}$, the \rpo\ is also characterized by
the symmetry parameters $k$ and/or $s$.

In fact, if the symmetry is continuous, then it is much more likely
to find \rpo s, than it is to find exact \po s, since $s = 0$,
corresponding to the \po , is only one specific value in the
continuum of possible values of parameter $s$.

In the case of KS equation, which has continuous symmetry
$\Shift_{\shift/L}$ and discrete symmetry $\Refl$, it is possible to
find \rpo s that satisfy one of the following conditions
\[
  \Shift_{\shift/L}f^\period{}(a) = a\,,
\quad\mbox{or}\quad
  \Refl\Shift_{\shift/L}f^\period{}(a) = a\,.
\]
The first condition is satisfied by \rpo s with shift $\shift$,
while \rpo s that satisfy the second condition are exactly periodic
($\shift = 0$) with period $2\period{}$.

\section{Complex Ginzburg-Landau}
\label{sec:CGL}
% Predrag  collected notes on CGLe here  2012-01-27

\begin{description}

\item[{\Rpo s} of the complex Ginzburg-Landau]
V. Lopez, As far as the discussion about ``drift" is concerned:
In presence of continuous symmetries (for PC, streamwise and
spanwise translations) one should be searching for
{\reqva} and
{\rpo s} - they are most likely more important than
the \eqva\ and spatially periodic orbits.

I like {\rpo s} paper\rf{lop05rel} by
Vanessa L{\'o}pez %\rf{lopezLink},
%        http://www.cse.uiuc.edu/$\sim$vlopez:
``Relative Periodic Solutions of the Complex Ginzburg-Landau Equation"

% [From siads\@siam.org  Dec 22 2004]:
% willing to review this manuscript for Dwight Barkley.

A method of finding {\rpo s} for differential equations with
continuous symmetries is described and its utility demonstrated by computing
{\rpo s} for the one-dimensional complex Ginzburg-Landau
equation (CGLE) with periodic boundary conditions.  A {\rpo} is
a solution that is periodic in time, up to a transformation by an element of the
equation's symmetry group.  With the method used, {\rpo s} are
represented by a space-time Fourier series modified to include the symmetry
group element and are sought as solutions to a system of nonlinear algebraic
equations for the Fourier coefficients, group element, and time period.
{\Rpo s} found for the CGLE exhibit a wide variety of temporal
dynamics,
with the % sum of their positive Lyapunov exponents varying from 5.19 to
% 60.35 and their
unstable dimensions from 3 to 8.

15 Sep 2004, Predrag Cvitanovic:
All of your {\Rpo s} have many unstable
eigendirections. Do you know that there exist no other solutions with a
lesser number of unstable directions (let's say only one)?

--vanessa:
I do not know of any proof that shows that there are no solutions with a
lesser number of unstable directions.  I did not find any solutions with
just one or two unstable directions, but at this point I cannot say
whether that is an indication that there are none or simply that
the procedure I used only identified solutions with 3+ unstable
directions.

15 Sep 2004, Predrag Cvitanovic:
Periodic solutions are useful if they are embedded into a chaotic
attractor. Do you have any measure of whether the typical solutions of
CGL, your parameter values, come close to any of the solutions that you
have found? If the periodic orbit is embedded into a chaotic attractor,
typical solution visits it infinitely often, infinitesimally close.

--vanessa:
This is something that I have not examined in detail.  I have observed (using
the ``naked eye") that the time evolution of typical trajectories (when viewed
by plotting the real versus imaginary part of A(x,t) at different times, as in
Figures 3,5,7 from the paper) sometimes resembles that of the first solution
found (i.e., the patterns displayed in Figure 3). (The first solution
found also happened to be the solution to which the solver I used
converged to most frequently.)  But I do not have a
quantitative measure of this ``resemblance".

After talking to Lopez at SIAM DS05 meeting: I do not think
her \rpo s are the dynamically important ones, so the problem remains wide open.

I read through
Vanessa L{\'o}pez's paper (clearly a summary of a PhD thesis), and while I
a like {\rpo s}, I am very worried about the general drift of the paper.

As I cannot tell how exhaustive is her set of numerical solutions, I
do not know what to make of them - like ZG, she gets all of them with
a large number unstable dimensions. Presumably none of them are close
to the asymptotic inertial manifold. More worrisome still, she uses
ZG to ``average" over periodic orbits. That is regrettable - I hoped
I would never see that stuff again. Dettmann and I tried to get Zoldi
to derive this for us, and Mainieri actually hired him at Los Alamos
to learn how this works. We are all convinced that it makes no sense
whatsoever. Regrettably Lopez used the nonsensical formulas of Zoldi,
so that will just keep confusing future readers. She is in computer
science, so cannot blame a graduate student for trying a formula
published in Phys Rev Lett.

Here are some snippets from \refref{lop05rel}:

We work with the CGLE with cubic nonlinearity in one spatial dimension,
%\end{eqnarray}
\begin{equation}   \label{eq:cgle_pde}
    \frac {\partial A}{\partial t}  =
    R A + (1 + \mathrm{i} \nu) \frac{\partial^2 A}{\partial x^2} - (1 + \mathrm{i} \mu) A|A|^2,
\end{equation}
with periodic boundary conditions
\begin{equation}   \label{eq:cgle_bcs}
    A(x,t)  =  A(x+2\pi,t),
\end{equation}
and, as a convention, when we refer to \textit{the CGLE} we mean
equation~\refeq{eq:cgle_pde} with the boundary conditions
\refeq{eq:cgle_bcs}, unless otherwise noted. Equation~\refeq{eq:cgle_pde}
describes the evolution of a complex-valued field $A(x,t)$.

The CGLE has a three-parameter group ${\mathbb T}^2 \times \mathbb{R}$ of
continuous symmetries generated by space-time translations and a rotation
of the complex field $A$.  That is, if $A(x, t)$ is a solution of the
CGLE, then so is $\mathrm{e}^{\mathrm{i}\eta_1} A(x+\eta_2,  t+\tau)$ for any
element $(\eta_1, \eta_2)$ of the two-torus ${\mathbb T}^2$ and $\tau \in
\reals$.  Thus, we focus our study on time-periodic solutions of the CGLE
relative to the ${\mathbb T}^2$-symmetry, namely, solutions of the CGLE that
satisfy
\begin{equation}  \label{eq:Atpsol}
    A(x,t)  =  \mathrm{e}^{\mathrm{i}\varphi}A(x+S, t+T)
\end{equation}
for some $(\varphi, S) \in {\mathbb T}^2$ and $T > 0$.   Note that after a
period of time $T$, such a solution returns not to itself, but rather to
an element in its ${\mathbb T}^2$-orbit.  Such \emph{relative periodic
solutions} represent invariant three-tori in the CGLE flow.  Previous
studies of time-periodic solutions of the CGLE~\refeq{eq:cgle_pde} have
centered on two types of solutions.  The first are single-frequency
solutions of the form $A(x,t) = B(x) \mathrm{e}^{\mathrm{i}\omega t}$ (see, for
example, \refref{DaVuDel00,doelman89,KaMaPaa96}); these are referred to as
\emph{stationary solutions}.
\toCB

The second type are generalized traveling waves, also called
\emph{coherent structures}, for which $A(x,t) = \rho(x-vt)
\mathrm{e}^{\mathrm{i}\phi(x-vt)} \mathrm{e}^{\mathrm{i}\omega t}$, where $\rho$ and
$\phi$ are real-valued functions and $\omega$ is some frequency (see, for
example, \refref{AKcgl02,BTHmaw01, Mielke02,Mcoh03}).  These coherent
structures reduce to single-frequency solutions by a change to a moving
frame, that is, by a change of variables $x \rightarrow x+vt$, $t
\rightarrow t$.

The relative periodic solutions of interest in this study are different
in that they exhibit more complicated temporal behavior than the
single-frequency solutions just described. Studies of bifurcations to
(stable) invariant two and three-tori for the CGLE are described in
\refref{takac98} and references therein.

The \rpo s we compute, which represent invariant three-tori, are obtained
by working with fixed parameter values for the CGLE, not via a
bifurcation study.

\item[2010-06-02 Predrag] Just heard a talk by
\HREF{http://win.ua.ac.be/~nschloe/}{N. Schl\"omer},
\HREF{http://win.ua.ac.be/~wvroose/}{W. Vanroose}
 and
\HREF{http://personal.maths.surrey.ac.uk/st/D.Avitabile/index.html}{D. Avitabile}
 on
\HREF{http://www.dspdes2010.org/admin/files/fileabstract/A452.pdf}
{Numerical continuation and symmetries} in mesoscopic superconductors.
and
N. Schl\"omer, D. Avitabile and W. Vanroose\rf{SchlAvVa11},
\emph{Numerical bifurcation study of superconducting patterns on a square.}
They fix the overall $U(1)=\SOn{2}$ phase in a BEC problem, do not
reduce the remaining $D_4$ discrete symmetry. Only stable solutions,
we might be able to do this better.
													\toCB
More precisely: numerical methods break the symmetry of any given
problem, and the symmetry-breaking errors can be unpleasantly large. If
symmetry is reduced {\em prior} to numerical work, this does not happen.
Yet another argument why one should always reduce the symmetries prior to
the analysis of any given problem.

\item[2010-06-04 Predrag] to nico.schloemer@ua.ac.be,
Wim.Vanroose@ua.ac.be,
D.Avitabile@bristol.ac.uk, on
\emph{Slice \& dice} $\SOn{2} \times D_4$

Nico et al, I enjoyed your talk. Feel free to ignore the rest of this
email, but if you know literature or have some ideas that might help me
write up methods of symmetry reduction better, let me know.

You probably have not heard Ashley Willis and various spiraling patterns
talks, but in fluid and cardiac dynamics one has to deal with various
(rather simple) continuous and discrete symmetries. All I have found
useful on the subject so far is described "pedagogically" in Chapter 10
Relativity for cyclists (another version is given in Continuous symmetry
reduction and return maps for high-dimensional flows).

I also always quotient discrete symmetries, such as $D_2$ or $D_4$, as
described in Chapter 9 World in a mirror. That kind of thing you
appreciate only after you do it yourself.

As long as you look at a single stable equilibrium or travelling wave,
getting rid of $\SOn{2}$ seems not a problem. It becomes a problem once when
you start looking at sets of unstable solutions.

Visualization of all relative periodic orbits as periodic orbits we
attain only by global symmetry reductions. We do it by "slicing", ie
finding hypersurfaces that cut group orbits just like Poincare sections
cut time-evolution orbits. The problem is that the natural choice; a
hyperplane normal to a group orbit tangent vector (Lia algebra generator)
is only good locally, globally it runs into singularities.

Calling overall complex phase ``gauge invariance'' is a waste of an ugly
but otherwise quite precise word. In quantum field theory one means by
that the local (a very deep concept) not the relatively trivial global
overall complex phase. But AMO physicists do indeed do that.

\item[2010-06-05 Daniele Avitabile]
Our problem has, as you point out, $\SOn{2} \times D_4$ symmetry. What we would
like to achieve is exploring the symmetry-breaking bifurcation scenario
pertaining to $D_4$, in the case of stationary solutions. In other words,
we do not want to simulate the time-dependent solutions of the
Ginzburg-Landau system. We concentrate instead on stationary solutions,
and we want to see how, by breaking $D_4$, we can get new, hopefully
observable, steady states.

\item[2010-06-05 Predrag]
I think that you get equilibrium solutions only within subspaces fixed by
$D_4$ symmetries. The should be also $\SOn{2}$ travelling wave solutions,
perhaps also important. They would come in two flavors - with irrational
velocities, and pre-periodic, belonging to a discrete $C_m$ subgroup of
$\SOn{2}$. And you can bifurcate into Hopf cycles, traveling waves, periodic
orbits, relative periodic orbits etc., unless for some reason the physics
does not like anything time-dependent.

\item[2010-06-05 Daniele Avitabile]
In order to do that we need to factor out the continuous
group symmetry $\SOn{2}$. As far as I understood from your comments, we
should be careful in doing that, as the the phase condition that we chose
now  (seeking a solution in a hyperplane normal to a group-orbit tangent
vector) is valid only locally. Am I correct in interpreting your
statement? If so, I should say that we explore our stationary solutions
only by means of numerical continuation. With this method, the best we
can do *is to search locally*. What we need to make sure is that, once we
have a solution and perturb it, we can still invert our problem and find
a new steady solution. In other words, seeking for a solution locally is
the best we can do, as far as I can tell. Do you think we could be
missing some patterns by using our phase condition? I think your
observation would be absolutely crucial if our continuous symmetry group
was acting on the time-variable as well, and we wanted to simulate them.
A typical case would be continuing time-dependent patterns like spiral
waves or scroll waves. A method that comes to my mind in this context is
the ``Freezing method'' by Tuemmler and Beyn, which you will probably know. I
am pretty sure that this differs from the slicing method that you told us
about. A recent article by Hermann and Gottwald recaps the method and
contains an extensive set of references about it.

Now, let us assume that we manage to factor out $\SOn{2}$, either locally
with our phase condition, or globally via the slicing method. You mention
that you always quotient discrete symmetries. Would you recommend this in
our case? What is the danger in not factoring them out? My gut feeling is
that, by factoring $D_4$, for instance, we will make sure that we will find
just one solution on the group orbit. I don't think this is bad in
general, and I would be pleased to learn what are the problems associated
with it.

\item[2010-06-05 Predrag]
I have read Tuemmler and Beyn and believe ``freezing'' = ``slicing''.
Siminos and I have Hermann and Gottwald on our reading list, but have not
studied it yet. Tuemmler and Beyn impose a slice by adding a new
dimension (the phase parameter) and a condition (a Lagrange multiplier),
while we go one dimension down by restricting the dynamics into the
slice. Both approaches are standard in imposing Poincar\' sections; Rytis
Paskauskas has written that up for Chapter 3 - Discrete time dynamics
(not yet on the public version), and Chapter 13 - Fixed points, and how
to get them, Section 13.4 Flows. My feeling is that Tuemmler and Beyn is
better, but we have not implemented it. If you do use it, let me know how
it works.

[I moved the discussion to Channelflow.org blog, and that killed it -
never heard from them again]

\item[2011-08-04 Predrag]                                       \toCB
Qihuai Liu and Dingbian Qian% \etal\rf{XXX},
{\em Modulated amplitude waves with nonzero phases in
     Bose-Einstein condensates},
\arXiv{1103.5277}, is unintelligible to me, but I would like to
understand what is the relation of MAWs to \reqva, \etc, so this is a
start. They say:

  ``
We consider uniformly propagating coherent structures of form
ansatz
\beq\label{LiQi11:2}
\psi(t,x)=R(x)\exp(i[\Theta(x)-\mu t]),
\eeq
where $R(x)\in\mathbb{R}$ gives the amplitude dynamics of the condensate
wave function, $\theta(x)$ determines the phase dynamics, and the
``chemical potential'' $\mu$, defined as the energy which takes to add
one more particle to the system, is proportional to the number of atoms
trapped in the condensate. When the (temporally periodic) coherent
structure is also spatially periodic, it is called a modulated amplitude
wave (MAW).
% \cite{brusch2000modulated,brusch2001modulated}
Similarly, a solution of the equation with the (temporally periodic)
coherent structure is called a quasi-periodic modulated amplitude wave
(QMAW) if it is also spatially quasi-periodic. If the phase of the
condensate wave function is trivial the MAW is a standing wave. Even the
amplitude $R(x)$ is $L$-periodic, the corresponding condensate wave
function $\psi(x,t)$ may be not periodic with respect to the spatial
variable $x$. If $2\pi/\nu$ and $L$ are rationally related, then
$\psi(x, t)$ is a MAW; otherwise $2\pi/\nu$ and $L$ are incommensurate,
and $\psi(x, t)$ is not periodic but quasi-periodic QMAW with the
frequency $\omega = \langle 2\pi/\nu,L\rangle$.
  ''

They copy their definitions from L. Brusch, M. G. Zimmermann, M. Van
Hecke, M. B\"ar and A. Torcini, ``Modulated amplitude waves and the
transition from phase to defect chaos'', {\em Phys. Rev. Lett. \bf 85},
86--89 (2000); L. Brusch, A. Torcini, M. Van Hecke, M. G. Zimmermann and
M. B\"ar, ``Modulated amplitude waves and defect formation in the
one-dimensional complex Ginzburg-Landau equation'', Physica D 160,
127--148 (2001).

\item[2012-01-15 Predrag] Read Scheuer and Malomed\rf{SMcgl02},
\emph{``Stable and chaotic solutions of the complex {G}inzburg-{L}andau
equation with periodic boundary conditions''}: ``The behavior of the
solutions of the CGLe without diffusion is studied with a periodic
boundary condition.''

\item[2012-01-15 Predrag] Read Doering \etal\rf{doercgl88,doercgl87},
``Low-dimensional behaviour in the complex {G}inzburg-{L}andau
equation'': ``The stability of plane waves of the CGLe is analyzed in a
periodic domain. The mass and energy integrals are estimated to give an
upper bound of the magnitude of the field variable. After that, cone
condition is established for the CGLe which indicates the existence of
the inertial manifold and proves the existence of finite dimensionality
of the attractor in this infinite-dimensional system. The Lyapunov
dimension is estimated.''

\item[2012-01-15 Predrag] Read Torcini\rf{torc96},
\emph{``Order Parameter for the Transition from Phase to
                 Amplitude Turbulence''}:
``The maximal consersed phase gradient is introduced as
                 an order parameter to characterize the transition from
                 phase to defect turbulence in the CGLe. It has a finite
                 value in the PT regime and decreases to zero when the
                 transition to defect turbulence is approached. A
                 modified KSe is able to reproduce the main feature of
                 the stable waves and to explain their origin.''

\item[2012-01-15 Predrag] Read van Hecke and Howard\rf{MvHhole01},
\emph{``Ordered and Self-Disordered Dynamics of Holes and Defects in the
One-Dimensional Complex {G}inzburg-{L}andau Equation.'' }

\item[2012-01-15 Predrag] Read Torcini \etal{TFGphase97}, \emph{``Studies
of Phase Turbulence in the One Dimensional Complex {G}inzburg-{L}andau
Equation''}, and Aranson and L. Kramer\rf{AKcgl02}, and there are a ton
more CGL references in the bib file, inherited from
Lan\rf{LanThesis,lanmaw03}.

\end{description}


\section{\Po s literature}
\label{sec:UPOlit}

\begin{description}

\item[2011-05-15 Predrag]
\HREF{PubMed.gov}{PubMed.gov} is scary. Find a paper you
want, click on "Related citations" on the right, and you get more stuff than you
can ever digest... Zotero helps you generate (imperfect) BibTeX entries.

\item[2011-05-15 Predrag]
Creagh\rf{Creagh94},
{\em Quantum zeta function for perturbed cat maps}, says: ``
The behavior of semiclassical approximations to the spectra of perturbed
quantum cat maps is examined as the perturbation parameter brings the
corresponding classical system into the nonhyperbolic regime. The
approximations are initially accurate but large errors are found to
appear in the traces and in the coefficients of the characteristic
polynomial after nonhyperbolic structures appear. Nevertheless, the
eigenvalues obtained from them remain accurate up to large perturbations.
''
(see CNS Zotero collection)

\item[Recursive Projection Method] (RPM) by
Shroff and Keller\rf{shroff1099} might be of interest to us in numerical
work. RPM stabilizes fixed-point iterative procedures by computing a
projection onto the unstable subspace. On this subspace a Newton or
special Newton iteration is performed, and the fixed-point iteration is
used on the complement. The method is effective when the dimension of the
unstable subspace is small compared to the dimension of the system.
Examples are presented for computing unstable steady states. The RPM can
also be used to accelerate iterative procedures when slow convergence is
due to a few slowly decaying modes. PC also has 2 hard-copy, rather
readable old proceedings reprints by Keller\rf{Keller77,Keller79}. There
are some readable words about homotopy methods in \refref{AllgGeorg88}.


        Here we discuss these continuous symmetries as
a small-step limit of discrete symmetries:

\begin{enumerate}
\item
        symmetries of 3-disk
\item
        $C_n$ symmetry of $n$-slice pizza without reflection symmetry
\item
        Fourier analysis of periodic lattices
\item
        running modes in periodic lattice deterministic
           diffusion
\item
    celestial {\rpo s}
\end{enumerate}

The reading material for 1)-3) is on
http://www.cns.gatech.edu/courses/chaosSpring06

In my \KS\ work with Christiansen, Putkaradze and Lan we
had excluded them ``by hand", by concentrating only on the space of
antisymmetric solutions. That is not good physics, as perturbations off
them mix into the full space of asymmetric solutions.



\item[{\Rpo s} in point vortex systems]

Testing bibtex - these should exist:
\refrefs{lop05rel,McCordMontaldi} and \refref{Vanderb}

Laurent-Polz\rf{Laurent-Polz04} say: We give a method to
determine {\rpo s} in point vortex systems.

perform a symplectic reduction on a fixed point submanifold in
order to obtain a two-dimensional reduced \statesp. The method
is applied to point vortices systems on a sphere and on the
plane, but works for other surfaces with isotropy (cylinder,
ellipsoid, ...). The method permits also to determine some
{\reqva} and heteroclinic cycles connecting these {\reqva}.

{\Reqva} are orbits of the symmetry group action which are in-
variant under the flow, this corresponds here to motions of
point vortices which are stationary in a steadily rotating
frame. The existence and nonlinear stability of {\reqva} formed
of three vortices have been studied respectively in [KN98] and
[PM98].

Periodic orbits on the sphere were determined in [ST,To01]
thanks to the following method: they reduced the system to
two-dimensional systems by a symmetric reduction (using some
finite subgroups of $SO(3)$); the computation of the dynamics
on the reduced spaces permits then to determine periodic
orbits. Our paper is devoted to transpose that method to
determine {\rpo s}. To this end, we combine a symmetric
reduction together with a symplectic reduction.

[Third Referee's Report Dr Newton]:
The manuscript considers {\rpo s} in point vortex systems that
arise from splitting {\reqv} configurations. It more
comprehensively treats problems of the type considered in
Souliere and Tokieda (2002) and Tokieda (2001). It is a very
nice paper.

Angel Duran, Numerical Integration of Hamiltonian Relative
Periodic Solutions. A First Approach
http://www.math.human.nagoya-u.ac.jp/scicade05/
% angel@mac.uva.es

    [1] B. Cano and A. Duran, A technique to improve the error propagation when inte-
grating relative equilibria, BIT 44(2004) 215-235.
    [2] A. Duran and J.M. Sanz-Serna, The numerical integration of relative equilibrium
solutions. Geometric theory, Nonlinearity 11(1998) 1547-1567.
    [3] E. Hairer, Ch. Lubich and G. Wanner, Geometric Numerical Integration. Structure-
Preserving Algorithms for Ordinary Differential Equations, Springer Series in Comput.
Mathematics, Vol. 31, Springer-Verlag 2002.
	\\
\PC{probably cite:
\rf{Creagh91}
    % author = {Creagh, Stephen C. and Littlejohn, Robert G.},
    % title = {Semiclassical trace formulas
    % in the presence of continuous symmetries},
\rf{Creagh92}
    % author = {Creagh, Stephen C. and Littlejohn, Robert G.},
    % title = {Semiclassical trace formulas
    % for systems with non-Abelian symmetry},
\rf{robb}
    % author = {Robbins, Jonathan M.},
    % title = {Discrete symmetries in periodic-orbit theory},
\rf{mta}
    % author = "Abraham, Ralph and Marsden, Jerrold E. and Ratiu, Tudor",
    % title =  "Manifolds, Tensor Analysis, and Applications",
\rf{Visw07b}
    % author = {Viswanath, Divakar},
    % title = {Recurrent motions within plane Couette turbulence},
    }


\item[A. Hramov and A. Koronovskii],
``Detecting unstable periodic spatio-temporal states of spatial
extended chaotic systems,''
\arXiv.org{0708.4349}.

\item[2011-05-15 Predrag]
Wisniacki \etal\rf{WiVeBeBo04},
{\em Classical invariants and the quantization of chaotic systems}, says: ``
Due to their exponential proliferation, long periodic orbits constitute a
serious drawback in Gutzwiller's theory of chaotic systems. Therefore, it
would be desirable that other classical invariants, not suffering from
the same problem, could be used in alternative semiclassical quantization
schemes. In this Rapid Communication, we demonstrate how a suitable
dynamical analysis of chaotic quantum spectra unveils the role played, in
this respect, by classical invariant areas related to the stable and
unstable manifolds of short periodic orbits.
''

\item[2011-05-15 Predrag]
Heusler \etal\rf{HeMuAlBrHa07},
{\em Periodic-orbit theory of level correlations}, says: ``
We present a semiclassical explanation of the so-called
Bohigas-Giannoni-Schmit conjecture which asserts universality of spectral
fluctuations in chaotic dynamics. We work with a generating function
whose semiclassical limit is determined by quadruplets of sets of
periodic orbits. The asymptotic expansions of both the nonoscillatory and
the oscillatory part of the universal spectral correlator are obtained.
Borel summation of the series reproduces the exact correlator of
random-matrix theory.
''

\item[2011-05-15 Predrag]
Ando \etal\rf{AnBoAi07},
{\em Automatic control and tracking of periodic orbits in chaotic systems}, says: ``
Based on an automatic feedback adjustment of an additional parameter of a
dynamical system, we propose a strategy for controlling periodic orbits
of desired periods in chaotic dynamics and tracking them toward the set
of unstable periodic orbits embedded within the original chaotic
attractor. The method does not require information on the system to be
controlled, nor on any reference states for the targets, and it overcomes
some of the difficulties encountered by other techniques. Assessments of
the method's effectiveness and robustness are given by means of the
application of the technique to the stabilization of unstable periodic
orbits in both discrete- and continuous-time systems.
''

\item[2011-05-15 Predrag]
Brack \etal\rf{BrOgYuRe05},
{\em Uniform semiclassical trace formula for
U(3) $\to$ SO(3) symmetry breaking}, says: ``
We develop a uniform semiclassical trace formula for the density of
states of a three-dimensional isotropic harmonic oscillator (HO),
perturbed by a term which term breaks the U (3) symmetry of the HO,
resulting in a spherical system with SO (3) symmetry.
We obtain an analytical uniform trace formula
which in the limit of strong perturbations (or high energy)
asymptotically goes over into the correct trace formula of the full
anharmonic system with SO (3) symmetry, and in the other limit
restores the HO trace formula with U (3) symmetry.
''

This might be useful to us in leading us to physically interesting systems
with non-abelian symmetries. However, have no time today to continue...
%
%Bengtsson I, Br�annlund J and � Zyczkowski K 2002 Int. J. Mod. Phys. A 17 4675
%give example of

\item[2011-05-16 Predrag]
Mark Srednicki has several interesting recent papers on Riemann zeros.
In {\em Nonclasssical Degrees of Freedom in the Riemann Hamiltonian},
\arXiv{1105.2342},
he says: ``
The Hilbert-Polya conjecture states that the imaginary parts of the zeros of
the Riemann zeta function are eigenvalues of a quantum hamiltonian. If so,
conjectures by Katz and Sarnak put this hamiltonian in Altland and Zirnbauer's
universality class C. This implies that the system must have a nonclassical
two-valued degree of freedom. In such a system, the dominant primitive periodic
orbits contribute to the density of states with a phase factor of -1, which
partially resolves a previously mysterious sign problem for oscillatory
contributions to the density of the Riemann zeros.
''

\item[2011-05-16 Predrag] I believe that Ruelle's linear response
theory is fundamentally wrong, as deterministic diffusion transport
coefficients are nowhere differentiable.
Lucarini \etal\ do like it. In
{\em Relevance of sampling schemes in light of Ruelle's linear response
 theory}, \arXiv{1105.2527}, they say: ``
We reconsider the theory of the linear response of non-equilibrium steady
states to perturbations. We first show that by using a general functional
decomposition for space-time dependent forcings, we can define elementary
susceptibilities that allow to construct the response of the system to general
perturbations. Starting from the definition of SRB measure, we then study the
consequence of taking different sampling schemes for analyzing the response of
the system. We show that only a specific choice of the time horizon for
evaluating the response of the system to a general time-dependent perturbation
allows to obtain the formula first presented by Ruelle. We also discuss the
special case of periodic perturbations, showing that when they are taken into
consideration the sampling can be fine-tuned to make the definition of the
correct time horizon immaterial. Finally, we discuss the implications of our
results in terms of strategies for analyzing the outputs of numerical
experiments by providing a critical review of a formula proposed by Reick.
''

\item[2011-07-22 Predrag]
Gao, Xie and Lan\rf{GaXiLa11}
accelerate convergence of cycle expansions by dynamical conjugacies.

The key idea of this paper, of replacing the stability of an unstable
fixed-point or periodic orbit that a critical point is preperiodic to, by
the root corresponding to the order of the critical point was developed
in detail in a careful study of convergence by Artuso \etal\rf{AACII},
where it is shown that the change in convergence is due to a single fixed
point whose preimage is the critical point. It is shown how to modify the
cycle expansion to fix the convergence. The original $1/\zeta$ is kept,
but the pole induced by the critical point singularity is explicitly
factored out. The method is essentially a quadratic conjugacy restricted
to the critical point (Ulam map to tent map being the trivial example).

The innovation of this paper that goes beyond \refref{AACII} is the
explicit study of natural measures of such maps and use of conjugacies to
excise the singularities in the Ulam map (and its Misiurowicz family
generalization) settings. While \refref{AACII} motivates excision of the
singularity by a detailed study of many families of periodic orbits, the
authors accomplish this more elegantly, by a simple, well designed
conjugacy.

They are looking at the series of generalized Ulam maps, or what Ruelle
calls ``Misiurewicz maps'', where the critical point is preperiodic to,
ie  mapped onto an unstable cycle and thus rendered non-contracting.
Period 2 below is the next in sequence. The interesting new one is the
``Golden mean'' map, see for example exercise 11.6 in ChaosBook.org.
There the critical point is a part of the 3-cycle, so you know (at least
numerically) where the 3 measure singularities are. The answer is
somewhere in the literature. Perhaps in L.~Billings and E.~M.~Bollt,
``Invariant densities for skew tent maps,'' Chaos Solitons and Fractals
12, 365 (2001).

The step in $\rho(x)$ at $x_f$ is suspicious - perhaps the problem is that
the prefactor of the $1/\sqrt{|x-x_f|}$ singularity is different on the two
sides of $x_f$. How can a half-singularity at $x=0$ map into both sides of
$x_f$ neighborhood? Presumably the conjugacy (which they do not explain)
should be piecewise analytic, not smooth as in their figures.

Their $g(x)$ is a bad news - they seem to have introduced infinite slope
at an arbitrary point of the map. It's probably an artifact of the
unexplained method for constructing conjugacies - nothing interesting
happens here in the original dynamics. If they can argue that any periodic
orbit that includes critical point has this problem, that is interesting.
Again, the good conjugacy is probably piecewise analytic - the natural
measure they get has worrisome steps.

Here (and in all finite grammar cases they study) working out the
symbolic dynamics and Markov graphs of this map would help - they have to
understand which cycles form the fundamental set and which (families of)
shadowed cycles are causing wild oscillations in their figures, before
embarking on constructing a conjugacy. Original map has no cycles of
infinite instability, so their subsequent troubles presumably come from a
badly chosen conjugacy.

There can no be analytic conjugacy in higher dimensions - measure
singularities always sit on fractal sets, just observe pictures of
natural measure on the H\'enon attractor. Still, if you one find a
conjugacy that excises the neighborhood of the nearly-attracting
13-cycle, that would deal with the main impediment to zeta function
convergence in this case. The method cannot be generalized to higher
dimensions. For private amusement, just try constructing a 2-d conjugacy
for something like a H\'enon map $\to$ Lozi map. Good luck.

There is immense literature on measures of 1-d maps (A. Boyarski? A.
Lasota and M.C. Mackey\rf{LM94}? G. Froyland? E. Bollt? J. M.
Aguirregabiria, Robust chaos with prescribed natural invariant measure
and Lyapunov exponent; \arXiv{0907.3790}? D.J. Driebe, Fully Chaotic Map
and Broken Time Symmetry (Kluwer, 1999)? ... and authors would profit
from using some of that work to illustrate their ideas. They should do a
literature search on measures of 1-d maps. This is reminiscent of work on
natural measures published since 1980's by Hungarian school (Szapfalusy,
Tel, ...), Bollt, and many others.

1-d maps with a single critical point are very special, and unfortunately
little of this is useful in higher dimensions - already for the H\'enon
attractor there is a fractal set of critical points (ie, stable-unstable
manifold tangencies) and their images. No conjugacy or a finite set of
conjugacies can help there...


\item[2011-07-26 Predrag]
Katsanikas, Patsis and Pinotsis% \etal\rf{XXX},
{\em Chains of rotational tori and filamentary structures close to
  high multiplicity periodic orbits in a 3D galactic potential},
  \arXiv{1103.3981}, might be of interest - it struggles with visualization
of tori in 4\dmn.

\item[2011-08-24 Predrag]                                       \toCB
Liao\rf{Liao09,WaLiLi11,Liao11,Liao11a} is able to integrate Lorenz
equations with the 800-digit precision, using Mathematica with the
400th-order Taylor expansion for continuous functions. He gets ``clean''
numerical simulation of chaotic solution of Lorenz equation in a long
interval $0\leq t \leq 1000$ LTU (Lorenz time unit) with negligible
truncation and round-off error.   He found that,  to gain a solution in
$0\leq t \leq T_c$,  the initial conditions must be at least in the
accuracy of $10^{-2T_c/5}$.   Thus,  when $T_c = 1000$ LTU,  the initial
condition must be  in the accuracy of 400-digit precision at least. The
averaged velocity fluctuation in  a cube meter of fluid is $3.722 \times
10^{-30}$ m/s, so he starts simulations with initial standard deviation
$\sigma=10^{-30}$ (not clear why Lorenz origins in fluid dynamics justify
precisely $\sigma=10^{-30}$, but never mid) and integrate
deterministically (that is wrong too - integration should be
Fokker-Planck, not deterministic). The effects kicks in at 80 LTU, and by
120 LTU initial conditions are forgotten. ``This strongly suggests that
chaos builds a bridge from the micro-level uncertainty to   macroscopic
randomness, and  thus  is an origin of macroscopic randomness and  time's
arrow.''

\item[2011-09-15 Predrag]
Gutkin and Osipov, %\rf{XXX},
\arXiv{1109.3329}, say: ``
By considering symbolic dynamics of the system one can introduce a
natural ultrametric distance between periodic orbits and organize them
into clusters. Each cluster consists of orbits approaching closely each
other in the phase space. We study the distribution of cluster sizes for
the baker's map in the asymptotic limit of long trajectories. This
problem is equivalent to the one of counting degeneracies in the length
spectrum of the {\it de Bruijn} graphs. Based on this fact, we derive the
probability $\P_k$ that $k$ randomly chosen periodic orbits belong to the
same cluster, find asymptotic behaviour of the largest
cluster size $|Cl_{\max}|$ and derive the probability $P(t)$ that a
random periodic orbit belongs to a cluster of the size smaller than
$t|Cl_{\max}|$, $t\in[0,1]$.
''

\item[2011-10-09 Predrag]
Klebanoff and E. Bollt\rf{KlBo011} \emph{Convergence analysis of
{Davidchack and Lai's} algorithm for finding periodic orbits} might be
worth a read. Here are also two Guckenheimer\rf{ChoGuck99,GM00aut} on
computing periodic orbits.

This might be of interest to Chandrites: Olvera and Vargas\rf{OlVa94},
\emph{A continuation method to study periodic orbits of the Froeschl\'e
map} write: `` The dynamics of many Hamiltonian systems with three
degrees of freedom is represented by the Froeschl\'e map which is
symplectic and four-dimensional. In this paper we study sequences of
periodic orbits approaching the invariant tori.''

\item[2011-07-19 Yueheng Lan]
Recently, I am working on the intermittency problem with periodic orbit
theory. We are trying to find a practical way for effectively doing
computations with cycle expansions. Besides the 1-d intermittency map, is
there a 2-d or 3-d system that is both intermittent and very closely
related to physics? We want our scheme to be tested in a real physical
situation.

I met a professor at a systems biology meeting who mentioned a way for
searching for periodic orbits with synchronization. It seems that no
Jacobian is needed at all in their computation. Please take a look at Lin
\etal\rf{LiMaFeCh10}, \emph{Locating unstable periodic orbits: {When}
adaptation integrates into delayed feedback control}.  This may provides
another approach for cycle detection. I do not know how efficient this
method is.

(Predrag put a copy into the ChaosBook.org/library,
\HREF{http://ChaosBook.org/library/LiMaFeCh10.pdf}{click here}.
Superficial impression - does not look serious.).

(Predrag: while I am at it,
\HREF{http://ChaosBook.org/library/goldstein01.djvu}{click here} for a
pirated copy of Goldstein\rf{goldstein01} \emph{Classical Mechanics} 3.
edition)


%\item[2011-XX-XX Predrag]
%XXX \etal\rf{XXX},
%{\em XXX}, says: ``
%XXX
%''
% (see CNS Zotero collection)


\end{description}
