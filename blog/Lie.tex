% siminos/blog/Lie.tex
% $Author$ $Date$

\chapter{Lie police}
\label{sect:LiePolice}

\renewcommand{\LieEl}{\ensuremath{g}}  % Predrag Lie group element
\renewcommand{\ssp}{x}

\begin{description}
\item[2011-03-15
\HREF{http://amath.colorado.edu/faculty/jdm}
{Master of Simplecticity}]
%{James Meiss}]
Never did I think I'd read the sentence ``It's a mess'' in a
\emph{Physica D} paper! But cool, nevertheless.

\item[2011-03-25 PC] keep reading:
``looking for periodic orbits in systems with only continuous symmetries
is a fool's errand.'' If you read to the end of Sect. 8 it gets worse:
``a serious nuisance.'' Another first for \emph{Physica D}.

\item[2011-03-17 JM]
In Eq.~(4) of \refref{SiCvi10} you say that every element of a Lie group
that is connected can be written as an exponential. This is false, I
believe. Some Lie groups are connected and some are not. $\SOn{n}$ is
connected, as is $Sp(n)$. But, just like for manifolds (indeed a Lie
group is just a manifold with a group structure), some groups can't be
covered by a single coordinate patch, or in this case by the exponential.
This is also related to the fact that the series for the  Log is not
globally convergent\rf{Hall03}.

\item[2011-03-17 PC]
Of course not {\em every} Lie group element needs to be continuously
connected to identity - that why the finely tuned statement in our humble
paper.
I need a lawyer.

\item[2011-03-17 JM] or a mathematician!

\item[2011-03-17 PC]
no, no, in your case I need a lawyer. We carefully
write above Eq.~(4):

``An element of a compact Lie group
continuously connected to identity can be written as''

We \emph{do not say}
``\emph{\color{red} every} element of a compact Lie group
\emph{\color{red} is} continuously connected to identity...''

% How can it be wrong? Goes back to Peter-Weyl, etc.
We write here
``continuously connected to identity'' so we do not need to point out
that $\On{n}$ has also elements of form (discrete operation $\times$
exponential rep of $\SOn{n}$ group element, where discrete group $Z_2$ is
$\{e,$inversion $\}$ (inversion$)^2=1$, and avoid talking about double
covers, \etc. The paper (as does the discussion of
\HREF{http://chaosbook.org/chapters/continuous.pdf}
{Chapter 10 - Relativity for cyclists}) tries its utmost to minimize the
\HREF{http://chaosbook.org/chapters/appendHist.pdf}{Gruppenpest jargon}
damage, which is a total turnoff to our intended audience of working
plumbers and electricians. If I may be so bold as to cite my foreman
plumber Fabian Waleffe, faculty member of a reputable mathematics
department in a union-busting state, upon my attempt to get him to read
\HREF{http://chaosbook.org/chapters/discrete.pdf}
{Chapter 9 - World in a mirror}, chicken feed in comparison to the
continuous symmetry reduction nightmare:
	\PC{remember to include Barkley's \On{n} as an example of
		semi-direct product.}

``How many Tylenols should I take with this?... (never took group theory,
still need to be convinced that there is any use to this beyond
mind-numbing formalizations.)''

But it is obviously too subtle, will rewrite in the ChaosBook.org.

Now, more seriously - we do not need the explicit exponential form, we
only need to say that for a continuous group (compact or not) we can
linearize the flow in a neighborhood of any group element, and that
the tangent space is spanned by the Lie algebra. How to say it without
too much jargon?

\item[2011-03-17 JM]
Yes, every Lie group has a tangent space (since it is a manifold), and
the tangent space always has a Lie algebra structure. Technically, I
believe that one says that the Lie algebra is the tangent space "at the
identity". So $\SOn{n}$ and $\On{n}$ have the same Lie algebra, the
antisymmetric matrices. The Lie algebra of $Sp(n)$ is the set of
``Hamiltonian matrices.'' More fancily, people talk about the Lie algebra
as the set of left-invariant vector fields... but that always confuses
me.

\item[2011-03-17 PC]
``set of left-invariant vector fields?'' Probably not needed
for compact Lie groups (subgroups of the unitary group). There left
and right vectors are the same thing...

\item[2011-03-17 JM]
For example the symplectic group has elements that are not exponentials.
An example is the Jordan block $B=[[-I, I],[0,I]]$, which I learned from
Alex Dragt. A great reference on Lie groups is Hall\rf{Hall03}.

\item[2011-03-17 PC]
Thanks for introducing me to Dragt; I have now read and like his
\refref{Dragt05} (for CNSers - uploaded to zotero). I prefer the much
deeper \refref{PCgr} to Hall\rf{Hall03}, which I always found rather
pedestrian. \HREF{http://birdtracks.eu}{The Much Deeper} you can have for
a click (by contrast to The Much Cooler and very wonderful
\refref{Meisso7} for which I doled out cool cash). But maybe I should
have a second look... For PDE applications of symmetries
Hoyle\rf{hoyll06} is pretty gentle.


\item[2011-03-17 JM]
$Sp(n) = {\LieEl :  \LieEl^T {\bf \omega} \LieEl = {\bf \omega}}$
where $\LieEl$ is $[2n\!\times\!2n]$.

\item[2011-03-25 PC]
See ChaosBook.org remark {rem:symplectic}  concerning notation for $Sp(n)$.

\item[2011-03-17 JM]
So matrix $B \in Sp(n)$, but it is not the exponential of a
``Hamiltonian matrix'' (a matrix of the form ${\bf \omega}S$, where $S$
is symmetric).

\item[2011-03-25 PC]
This is exercise 3.7.12 in Dragt\rf{Dragt11}
It has not worked for me as yet:
By explicit calculation, $B^T {\bf \omega} B \neq {\bf \omega}$.
I am probably missing something, but it is a worthwhile exercise
to do... (Evangelos? Chao?)

\item[2011-03-25 PC]
Thanks for pointing out that this is called `Hamiltonian matrices.'
Curious: who named this `Hamiltonian matrices'? So far I've tracked it to 1971
(see ChaosBook.org remark {rem:symplectic}).
As a physicist, I hate
this misuse of a term well established in Quantum Mechanics since the
time of Heisenberg (See ChaosBook.org remark {rem:symplectic}). The recovered Irish
alcoholic did quaternions, but I believe that it was the Germans who
formalized the classification of semi-simple Lie algebras towards the end
of 19th century, with Cartan driving in the final stake.

Can you perhaps have a glance at
\HREF{http://chaosbook.org/chapters/continuous.pdf} {Chapter 7 -
Hamiltonian dynamics}, see whether your hair raises? Spurred by you I'm
starting to incorporate more symplectic material, the first draft is
ChaosBook.org section~{sect:toCB}.

I care about splitting hairs because I'm trying to collect all possible
names for every single thing in ChaosBook.org commentary, together with
attributions.

\item[2011-03-17 JM]
But it seems you have the group $\SOn{n}$ or perhaps
$\Un{n}$ in mind. Because in the next paragraph
you say that the Lie algebra of your group is the set
of anti-hermitian matrices, and then you say it can be
brought into the anti-symmetric form. I thought this was
true only for the orthogonal group.

\item[2011-03-25 PC]
You are correct, from then on we use only  $\SOn{n}$, as Evangelos
explains here:

\item[2011-03-17 ES]
As we only have applications in mind where $(x_1,x_2,...,x_d)$ are real
coordinates and since any compact Lie group acting on $\reals^n$ can be
identified with a subgroup of $\On{n}$, we only consider $\On{n}$. Then, as
explained above, we only need elements connected to the identity and
therefore restrict attention to $\SOn{n}$.

\item[2011-03-17 JM]
I wasn't aware of this (subgroup) business. I'll have to read about that.

I take it you submitted your most recent paper to the Morrison
festschrift volume? I wrote a paper for that too.

\item[2011-03-17 PC]
Would you mind having a look at it,
\HREF{http://www.cns.gatech.edu/~predrag/papers/preprints.html\#FrCv11}
{\emph{Reduction of continuous}} \emph{symmetries of chaotic flows by the
method of slices}? It is on the same theme, and I'm afraid that until we
reduce KS, pipe flow and plane Couette flow symmetries there will be more
papers like this...

\item[2011-03-17 JM]
Sure, that was how I started, but then I decided I better read
your \emph{Physica D} paper first!

\item[2011-04-23 PC]
Michael Loss says we should look at Fefferman\rf{Feffer83} ``Uncertainty
principle,'' which he thinks attempts to define symplectic distances.
Fefferman's conjectures were proved by Hofer. I have looked at the paper,
and see no connection to what we do; it's about relations between
phase-space volumes and spectra of the Schr\"odinger operator.

\item[2011-07-08 PC]
In \refref{FrCv11} we say: The minimal distance is a solution of the
extremum conditions
\[ %beq
\frac{\partial ~~}{\partial \gSpace_a} |\ssp - \LieEl(\gSpace)\,\slicep|^2
   =
2\, \braket{\sspRed - \slicep}{\sliceTan{a}}
   = 0
    \,,\qquad
	  \sliceTan{a} = \Lg_a \slicep
\,.
\] % ee{PCsectQ0}
I think the result is right, but our argument is flawed; $[\Lg_a,\LieEl]$
do not commute in general, but we use this only in the limit $\LieEl \to
1$, and there it is OK. Rethink, rederive...

\item[2011-07-08 PC]
In \refref{FrCv11} we say:
``
What about the fixed-point subspace $\pS_\Group$ (see  refeq~{dscr:InvPoints})?
Because of it, the action of \Group\ is globally neither free nor proper,
\etc. All intersections of slices, ridges and {\sset s} contain the
fixed-point subspace $\pS_\Group$. Should we worry? Not really.
''

Need to rewrite this. There are singularities that are artifacts of a
linear slice, described by the associated {\sset}. And then there are
genuine singularities, such as the embedding of an invariant subspace in
the full \statesp. A {\sset} includes the invariant subspace and cannot
`cure' those. Actually we have no example of a {\sset} induced artifact
singularity, so it might be that {\sset} is useless...

\item[2011-06-28 SF]
When I choose the {\sset} so that a trajectory passes near it at a point
far away from the z-axis, it will produce a singularity for that slice.
When I use random slices on the same trajectory, no corresponding
singularity appears. So the {\sset} induced singularities can occur.

The {\sset} is 3D and the strange attractor is 3-4 dimensional (is this
correct?) so shouldn't there always be some points away from the z-axis
where the trajectory passes near the {\sset}? It is just less common than
passing the {\sset} when close to the invariant subspace.

\item[2011-07-8 PC]
another notational nightmare in \refref{FrCv11}; each local slice
$\pSRed_{\slicep} $ should be labeled by its \template\ \slicep, and
distinct \template s should get it's own label, as in
 refsect~{sec:chart}:
``
Each slice $\pSRed{}^{(j)}$, tangential to one of a finite number of
{\template s}  $\slicep{}^{(j)}$ $[\cdots]$.
''
We postponed these labels to later...

\item[2011-07-27 PC]
Here is an  attempt to define symplectic distances.

In \HREF{http://arxiv.org/abs/1107.5180}{arXiv:1107.5180}
Ide and Wiggins say:
``
We develop a method for the estimation of {\bf T}ransport {\bf
I}nduced by the {\bf M}ean-{\bf E}ddy interaction (TIME) in two-dimensional
unsteady flows.
[...]
One of the TIME functions is identical to the Melnikov function that is
used to measure the distance, at leading order in a small parameter,
between the two invariant manifolds that define the Lagrangian lobes.
''

\item[2011-07-27 PC]
What follows is casting eye far ahead - to the role of gauge invariance
in Quantum Field Theories, but just to have it recorded somewhere.
Following articles seem of interest as follow-ups on
Cvitanovi\'c\rf{PCar}, {\em Group theory for {Feynman} diagrams in
non-{Abelian} gauge theories}:

Should add this article to Birdtracks.eu/refs: Astorino\rf{Astor10}
writes ``Jones polynomial arises as special cases: Sp(2), SO(-2), and
SL(2,R). These results are confirmed and extended up to the second order,
by means of perturbation theory, which moreover let us establish a
duality relation between $SO(\pm N)$ and $Sp(\mp N)$ invariants. A
correspondence between the first orders in perturbation theory of SO(-2),
Sp(2) or SU(2) Chern-Simons quantum holonomy's traces and the partition
function of the Q=4 Potts model is built.''

Khellat\rf{Khel10} strikes me as dubious...

Martens\rf{Mart11} writes: ``We calculate the two-loop matching corrections for the
   gauge couplings at the Grand Unification scale in a general framework
   that aims at making as few assumptions on the underlying Grand Unified
   Theory (GUT) as possible. In this paper we present an intermediate
   result that is general enough to be applied to the Georgi-Glashow
   SU(5) as a ``toy model''. The numerical effects in this theory are
   found to be larger than the current experimental uncertainty on $\alpha$s .
   Furthermore, we give many technical details regarding renormalization
   procedure, tadpole terms, gauge fixing and the treatment of group
   theory factors, which is useful preparative work for the extension of
   the calculation to supersymmetric GUTs.
   ''

Tye and Zhang\rf{TyZh10} write: ``
Bern, Carrasco and Johansson have conjectured dual
   identities inside the gluon tree scattering amplitudes.
   We use the properties of the heterotic string and open string tree
   scattering amplitudes to refine and derive these dual identities.
   These identities can be carried over to loop amplitudes using the
   unitarity method. Furthermore, given the $M$-gluon (as well as
   gluon-gluino) tree amplitudes, $M$-graviton (as well as
   graviton-gravitino) tree scattering amplitudes can be written down
   immediately, avoiding the derivation of Feynman rules and the
   evaluation of Feynman diagrams for graviton scattering amplitudes

Eto \etal\rf{EFGKNOV08} write:
   ``We construct the general vortex solution in the color-flavor-locked
   vacuum of a non-Abelian gauge theory, where the gauge group is taken
   to be the product of an arbitrary simple group and U(1). Use of the
   holomorphic invariants allows us to extend the moduli-matrix method
   and to determine the vortex moduli space in all cases. Our approach
   provides a new framework for studying solitons of non-Abelian
   varieties with various possible applications in physics.''

and there is much much more...; will continue some other time.

\item[2011-11-03 PC] Today is that time. I'm sitting in
\HREF{http://intractability.princeton.edu/blog/2011/05/workshop-counting-inference-and-optimization-on-graphs/}
     {Intractability Workshop:}
     \emph{Counting, Inference and Optimization on Graphs}
with a bunch of high-level computer nerds, and I almost afraid to say
what I'll say next (plumbers avoid physicists that say such things): In
constructing our atlas of inertial manifold of turbulent pipe flow, we
fix the $SO(2) \times O(2)$ phase separately on each local chart. The
freedom of doing that is called ``local gauge invariance'' (blame Hermann
Weyl for the ugly word) and in the limit of $\infty$ period cycles, cycle
points are dense and their local charts are infinitesimal, so this is
really local gauge invariance. In the world of computer science they use
this freedom profitably, to reduce the number of terms they use in their
computations. That suggests that there might be a (variational?)
principle that selects an optimal choice of (relative) template phases

Nerds call this 'reparametrization' - it supposedly speeds up calculations.
Have not really seen that in quantum field theory, with exception of light
cone gauges and their relatively recent applications by the Witten cult.

Literature: \refref{CheChe08,YeChe11} and stuff on
\HREF{http://www.hpl.hp.com/personal/Pascal_Vontobel/ciog2011/reading_list_web.html}{this
site} (if you can understand any of it).

Feel free to ignore this remark. It's future research.

\item[2011-11-03 PC] Nothing from an invariant subspace can exit it,
nothing can enter it - symmetry ensures that. How is this separation
ensured in the symmetry-\reducedsp?

\item[2011-11-17 ES] I think this is the reason we should pay attention to the
fact that orbit space is not generally a manifold, but rather a union of manifolds
of different dimension. I think this is directly related to the existence of
fixed point subspaces (which in turn are dynamically invariant).


\item[2011-11-17 ES] H. R. Dullin, H. E. Lomeli, J. D. Meiss, in 
\emph{Symmetry Reduction by Lifting for Maps}, 
\HREF{http://arxiv.org/abs/1111.3887}{arXiv:1111.3887}
say:\\ 
``We study diffeomorphisms that have one-parameter families of continuous
symmetries. For general maps, in contrast to the symplectic case, existence of
a symmetry no longer implies existence of an invariant. Conversely, a map with
an invariant need not have a symmetry. We show that when a symmetry flow has a
global Poincar\'{e} section there are coordinates in which the map takes a
reduced, skew-product form, and hence allows for reduction of dimensionality.
We show that the reduction of a volume-preserving map again is volume
preserving. Finally we sharpen the Noether theorem for symplectic maps. A
number of illustrative examples are discussed and the method is compared with
traditional reduction techniques.''

\item[2011-11-17 ES] We have to read it carefully (my first attempt resulted in the
usual headache you get from Chossat or Golubitsky). Their last example in
Section 3 is relevant to both KS and CLE. It seems to lead to rewriting the original
map in high-order polynomial variables. 
% However it is not obvious to me that 
% the new map is invariant under the group action as I would expect, 
% so I might be missing something. 
Their method looks a lot like the moving frame method of Fels and Olver \rf{FelsOlver98}
(\ie\ they define a slice which leads to a certain coordinate change). They impose 
the requirement of a global ``Poincar\'e section for group orbits'' (\ie\ a global slice?). 
In the process they have to exclude problematic regions in state space. 

\end{description}


\section{ChaosBook.org chapter `Hamiltonian dynamics'}
\label{sect:toCB}

\begin{description}
\item[2011-10-07 Predrag]
Moved this material back to ChaosBook.org newton.tex,
chapter {\em Hamiltonian dynamics}

\end{description}

\renewcommand{\LieEl}{\ensuremath{\gamma}}  % also a Siminos Lie group element
\renewcommand{\ssp}{a}
