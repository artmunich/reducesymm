% siminos/blog/Vlasov.tex
% $Author$ $Date$

\chapter[Vlasov stability]{Stability of Vlasov equilibria}
\label{c-Vlasov}


\begin{bartlett}{
Your are doing a stupid calculation.
}
\bauthor{
P. Cvitanovi\'c
    }
\end{bartlett}


\begin{description}
\item[ES 2010-11-06]
	\PC{you put 'stupid' and 'Cvitanovic' in the Google search, and you sure will
	    get lots of hits. But do you have the exact reference for the above quote?}
	\ES{I am not as organized as
	  \href{http://people.maths.ox.ac.uk/porterm/wisdom/predrag.txt}{Mason Porter},
	    so the reference would be something like this: 'P.~Cvitanovi\'c, personal communication,
	    2004-2009, at least annually, whenever I did something stupid.' Last time was
	    when I mentioned this paper\rf{eckhardt94} on periodic orbit theory
	    for Lorenz flow, so it had to do with others' calculations.}
Most of my last year's work is summarized in the
\href{http://www.cns.gatech.edu/~siminos/papers/siminos11.pdf}{preprint}\rf{siminos11}.
The Vlasov-Poisson system describes collisionless electronic plasmas
coupled with a self consistent electric field.
      \begin{subequations}\label{eq:Vlasov-Poisson-1d}
      \beq\label{eq:Vlasov-1d}
	      \frac{\partial f}{\partial t}
		+v\frac{\partial f}{\partial x}+E\frac{\partial f}{\partial v}=0\,,
      \eeq
      \beq\label{eq:Poisson-1d}
	      \frac{\partial E}{\partial x}=
		    \left(\int_{-\infty}^{+\infty}f\, dv\,-1\right)\,,
      \eeq
      \end{subequations}
where $f(x,v,t)$ the probability distribution function for electrons and $E(x,t)$
the electric field.
You might recognize \refeq{eq:Vlasov-1d} as collisionless Bolzmann equation,
but plasma physicists call everything their own way.
\Reqva\ for this system are known to exist since a
famous 1957 paper, however calculating their stability, analytically or
numerically, remains an open problem. The Vlasov-Poisson system is
Gallilean invariant, so it is always possible to reduce the problem to the
study of \eqva.

I found it strange that no reliable numerical method existed for the study of
stability of \refeq{eq:Vlasov-1d} and thought it was a good opportunity to use
an expansion in orthogonal basis functions to discretize the stability matrix
(matrix of variations)
as done for KSe and Navier-Stokes. It turns out\ES{Unfortunately or
fortunately I found out after I did most of the work},
that people have already tried such an approach with most recent example our
very own R. Pa\v{s}kauskas\rf{paskauskas09}.  For periodic boundary
conditions in $x$ a natural basis choice is Fourier-Hermite functions, for $x$ and
$v$ discretization, respectively. Hermite functions are a natural choice for
velocity discretization because they are defined on an infinite domain
and decay as the Maxwellian
for large $v$, which is the physically relevant asymptotic behavior for
distribution functions.

The bad news is that such a scheme does not really work, even though all papers
on the subject claim success. Expansions of equilibrium distribution functions
in terms of the Fourier-Hermite basis converge exponentially fast, but the
eigenvalue calculation converges only as power-law as we add more Hermite
terms, and quite often convergence is too slow for the method to be of use.
It was like listening to Predrag
saying to me once again ``You are doing a stupid calculation.''
We keep more and more terms in the expansion of equilibrium $f_0$ which
quickly hits round-off; then all we add is numerical noise and still the
error in approximating eigenvalues falls off as a power-law.

Most previous authors\rf{camporeale06,paskauskas09}
focus on rescaling or shifting the Hermite basis to `optimize' convergence.
There is no indication in these papers on whether convergence of eigenvalues is power-law or
exponential, but the relative error reported is usually around $5-10\%$, which
defies the purpose of using spectral methods in the first place. Trying to
rescale the basis as suggested in \refref{paskauskas09} did not help
in the cases I studied.


\item[ES 2010-11-06 What's wrong?]

The problem seems to be related to the linear operator we approximate, which
I will call $\mathcal{A}$. The linear term in \refeq{eq:Vlasov-1d}
(advection, free-streaming or ballistic term, depending on your background)
drives what is called velocity space \emph{fillamentation},
the transfer of energy in very fine velocity scales. When this term is in some
sense significant in \refeq{eq:Vlasov-1d}, one should expect that eigenfunctions
will involve fine velocity scales and our expansion in terms of Hermite functions
will receive significant contributions from high-order terms.

Mathematically, $\mathcal{A}$ is a non self-adjoint operator with
continuous (mathematicians call it essential) spectrum.
The Vlasov-Poisson system
is Hamiltonian and the eigenvalues of equilibria look like
in \reffig{f:VlasovStab}, they come in either quartets of complex eigenvalues,
or pure imaginary eigenvalues. The continuous, neutral spectrum is associated to
the advection term  in \refeq{eq:Vlasov-1d}.
Although we are interested in unstable modes, the non-well
behaved, continuous part of the spectrum is still ``present'' in $\mathcal{A}$
and can interfere with the determination of unstable modes, especially
weakly unstable ones. I feel that this might be related to non-normality
of the eigenvectors (non self-adjointness of $\mathcal{A}$), but as the
latest computational progress took place very quickly I did not have time to
think about it seriously -- I had to write the paper.
%
\SFIG{VlasovEils}{}
{
  Spectrum of an unstable Vlasov equilibrium: $\pm\lambdaV_1$ and
  $\pm\lambdaV_1^*$ are discrete eigenvalues, while $\sigma_{\mathrm{c}}$
  is the continuous spectrum.}
{f:VlasovStab}
%
\SFIG{VlasovEilsSD}{}
{
  Through spectral deformation,
  $\sigma_{\mathrm{c}}$
  is shifted to the left half-plane, while Landau damped modes correspond
  to $\lambdaV_2,\lambdaV_3,\lambdaV_3^*$. As the Hamiltonian structure
  is destroyed by
  spectral deformation, $-\lambdaV_1,-\lambdaV_1^*$ disappear as
  $\sigma_{\mathrm{c}}$ moves to their left. After \refref{crawford89}.}
{f:VlasovSD}
%

\item[ES 2010-11-06 Spectral Deformation]

Having identified the source of the difficulty, I was lucky enough to read a
beautiful paper on the stability of Vlasov-Poisson equilibria
by Crawford and Hislop\rf{crawford89}. The punch line: get the continuous
spectrum out of the way. The technique used is \emph{spectral deformation},
introduced for Schr\"odinger equation by very serious mathematicians that I
don't dare to cite. Crawford and Hislop\rf{crawford89, hislop89} extended
the method to non-selfadjoint problems and in particular the Vlasov-Poisson
system. The trick is to introduce a family of operators
\beq
  \VlasovLrzSD=\SDop{\theta}\VlasovLrz\SDopI{\theta}
\eeq
where $\SDop{\theta}$ is a linear, unitary transformation for $\theta$ real.
When $\theta$ is analytically continued in the
complex plane, $\SDop{\theta}$ becomes non-unitary and the continuous spectrum
is deformed, see \reffig{f:VlasovSD}. As long as discrete eigenvalues ``stay
out of the way'' of the continuous spectrum, they remain unchanged.

Crawford and Hislop choose a relatively simple but clever transformation that
``damps'' the continuous spectrum. They are interested in problems
that yield to analytical study and thus do not consider nonlinear
waves. Their transformation however is exactly what we needed. For large enough
$|\Im\theta|$ the continuous spectrum is well separated from eigenvalues with
positive real part and the convergence of eigenvalues with the number of
Hermite terms retained in the expansion of $\VlasovLrzSD$ converges
exponentially fast. My feeling is that the asymptotic power law
rate of convergence does not change, but the ``transient'' exponential
convergence lasts long enough for the method to be usable (to benchmark
the method I computed eigenvalues for a simple example to machine precision;
in plasma physics such accuracy is considered useless).

The Hermite coefficients of eigenvectors of $\VlasovLrz$,
obtained from $\VlasovLrzSD$ by the inverse
transformation, still fall off as a power law.
This was to be expected. The eigenfunction we approximate involves fine
velocity scales and there is nothing to do about it. However, in (plasma)
physics we are interested in moments of the distribution function, and in
particular low order moments, for instance for the electric field eigenmode.
These are represented accurately.

\item[ES 2010-11-06 The future.] I hope by now all this sounds familiar,
because my plan for the future is to visit an old friend:
\FPoper. Non self-adjoint operator, essential spectrum getting in the way,
singular natural measure ($\simeq$ fillamentation),
and at the end: we get the eigenvalues correctly
by periodic orbit theory.

I know that brainless discretization of state space is strongly discouraged, but could
it be an option if we first applied spectral deformation and ``shrunk'' the essential
spectrum? It worked for Vlasov: by changing the spectral properties of the operator
we did not have to worry as much about the functional space over which we defined
it and the basis used to discretized it.
{\FPoper} is not that much different
(actually given its Hamiltonian structure Vlasov is closer to Liouville).
Could it help in cases where periodic orbit theory does not do well, for instance
with non-hyperbolicity? As an example in Lorenz flow periodic orbit theory
converges very slowly, presumably due to almost non-hyperbolic dynamics.
Could spectral deformation help periodic theory converge faster?

The shortest path to permanent position seems to be to stop thinking about this
and work on applications of the method in plasma physics.
There are many problems to apply it to and
it can be generalized to higher dimensions, Vlasov-Maxwell \etc.

However, {\FP} is more interesting.
I want to try to apply spectral deformation to {\FP} or
Koopman operator and discretize it in a mundane basis, perhaps in
Hermite functions (since it worked for Lippolis thesis).
If it works, I think it would be great news as we could then
compute averages without thinking about state-space geometry.

What do you think? Does it sound insane? Has it been done? It seems that
people who know {\FP} or Koopman, also know spectral deformation,
but usually these are people who do not compute.


\item[ES 2010-12-02 \FPoper.]
													\toCB
In literature I've encountered the following names for \FPoper:
 Transfer operator (sounds more descriptive),
Ruelle operator or {Ruelle-\FP} operator (sounds appropriate,
since satisfying {\FP} theorem is not enough to define an operator, right?)
Also the three names in {Ruelle-\FP} can be permuted at will (but Ruelle can
never be in the middle).

\item[PC 2010-12-03 \FPoper.]
													\toCB
That's funny, I guess you have not read Das Buch - there it is called
``evolution oper\-ator'' instead of Ruelle operator for a reason -
it's a more general thing than what Sinai, Bowen, Ruelle, and other ergodicists
had in mind generalizing transfer matrix probabilities
to the non-negative operator evolution of
probability densities  over function (Banach?) spaces.
It is certainly not the Perron-Frobenius oper\-ator
(I follow Pesin's ordering of the two names),
which I interpret to refer to moving  around deterministically \statesp\ densities
of trajectories; in Das Buch the
evolution oper\-ators are in general weighted, and derivatives of their leading
eigenvalues yield the cycle expansions for the desired averages.
In all fairness, Perron and Frobenius have to do only with finite-dimensional
fully filled matrices with strictly positive matrix elements, which is the
setting for the Perron-Frobenius theorem: the first eigenvalue is one,
the second is real and strictly less than one. Evolution oper\-ators act on
function spaces, with entries that can be real or complex (semi-classical quantization)
of any sign.

I thing I discuss this in some remark in ChaosBook. Ruelle is (when you
talk to him personally) \emph{very annoyed} that people refer to \emph{his}
operator as a ``\FPoper,'' and says so somewhere in his writings (should try to
track that reference down): nothing
of the sort can be found in the publications of Perron and Frobenius,
he says.

\end{description}
