% siminos/blog/ChaosBook.tex
% $Author$ $Date$

\chapter{ChaosBook.org blog}
\label{chap:ChaosBook}

\begin{description}

\item[2011-10-24 PC]
I have added this chapter to collect your ChaosBook.org edits /
comments. If you svn co dasbuch, you have the dasbuch/book/chapter/*.tex
source code and can clip and paste formulas to here. I never refer to a
chapter by it's current number, as chapter numbers change from edition to
edition - latter on (years hence) trying to figure out what ``Chapter
17'' is can be quite confusing. Internally, each chapter is kept track
off by its file name, for example, in this blog ``continuous'' refers to
\refchap{c-continuous} {\em Relativity for cyclists}.

\item[2011-12-08 PC] Of possible interest from
\HREF{http://chaos.aip.org/?track=CHAOSDEC11}{chaos.aip.org Dec. 11 2011}:

\HREF{http://www.mathjax.org/}{MathJax} is an open-source JavaScript
display engine that produces high-quality math in all modern browsers. In
addition to higher-quality, cross-platform, browser-agnostic equation
rendering, readers of AIP online journals can now copy equations from
journal articles and paste them directly into text editors like Word,
LaTeX, MathType, and research wikis, as well as into calculation software
like Maple, Mathematica, and others.

To see MathJax in action, visit
\HREF{http://chaos.aip.org/resource/1/chaoeh/v21/i4/p043126_s1}{your
favorite Chaos article} and select the Read Online option. Once in the
HTML view, go to the navigation bar and turn on MathJax. From there you
can copy and paste any equation into your favorite MathML-enabled editor.
\HREF{http://www.mathjax.org/demos/copy-and-paste/}{Watch the video}.

And here is a Springer search engine for LaTeX snippets:
\HREF{http://www.latexsearch.com/}{www.latexsearch.com}.

\end{description}

\section{ChaosBook.org broken links}
\label{c-brokenLinks}

Predrag should fix these (mark when fixed):
\begin{itemize}
  \item[[~]] 2011-09-27
\HREF{http://www.cns.gatech.edu/~predrag/papers/preprints.html\#KS}
{Vachtang's term paper} link to
\\
www.cns.gatech.edu/$\sim$predrag/papers/vachtang.ps.gz
  \item[[~]] 2011-09-27
\HREF{http://www.cns.gatech.edu/~predrag/papers/preprints.html\#KS}
{Vachtang Putkaradze PhD thesis} link to
\\
www.nbi.dk/$\sim$putkarad
  \item[[~]] 2011-12-10
mark here when done with transferring things worth saving from
\texttt{siminos/chao/blog.tex}
  \item[[~]]
\end{itemize}


%%-----   Flows
\section{Chapter: Go with the flow}
\label{c-flows}\noindent dasbuch/book/chapter/flows.tex

\begin{description}

\item[2012-01-02 PC]
Haddad\rf{HaCh08} says:                     \toCB

A dynamical system consists of three elements -- namely, a setting (called
the state space) in which the dynamical behavior takes place, such as a
torus, topological space, manifold, or locally compact metric space; a
mathematical rule or dynamic which specifies the evolution of the system
over time; and an initial condition or state from which the system starts
at some initial time.

\end{description}


%%-----   Maps
%\section{Chapter: Discrete time dynamics}
%\label{c-maps}\noindent dasbuch/book/chapter/maps.tex
%\begin{description}
%
%\item[2012-01-02 CS to PC]
%
%\end{description}

%%-----   Stability
%\section{Chapter: Local stability}
%\label{c-stability}\noindent dasbuch/book/chapter/stability.tex
%\begin{description}
%
%\item[2012-01-02 CS to PC]
%
%\end{description}

%%-----   Cycle stability
\section{Chapter: Cycle stability}
\label{c-invariants}\noindent dasbuch/book/chapter/invariants.tex
\begin{description}

\item[2011-04-05 CS] A question about the last sentence in the first
paragraph of Section 5.4  discussed in the group study last week: Why
does the neighborhood size scale as $1/|\Lambda_{p}|$? Wouldn't it scale
as $|\Lambda_{p}|$?

\item[2011-04-06 PC] Mhm, clearly not written clearly enough, but perhaps
the most important property of an unstable flow that one has to
understand. The product of expanding multipliers $|\Lambda_{p}|$ blows up
exponentially with time, but the \emph{neighborhood shrinks}
exponentially with time, Detroit-like. Does looking at Figure 5.1 help?
Does reading Sect. 1.5.1 help? If you understand it, can you rewrite

                                                    \toCB
``
Nearby points aligned along the stable (contracting) directions  remain
in the neighborhood of the trajectory $\ssp(t)= \flow{t}{\xInit}$; the
ones to keep an eye on are the points which leave the neighborhood along
the unstable directions because all nonlinear phenomena comes from these
directions. The sub-volume $ |\pS_{\xInit}| = \prod_i^e\Delta \ssp_i$ of
the set of points which get no further away from $\flow{t}{\xInit}$ than
$L$, the typical size of the system, is fixed by the condition that
$\Delta \ssp_i \ExpaEig_i = O(L)$ in each expanding direction $i$. Hence
the neighborhood size scales as $|\pS_{\xInit}| \propto
O(L^{d_e})/|\ExpaEig_p| \propto 1/|\ExpaEig_p| $ where $\ExpaEig_p$ is
the product of expanding Floquet multipliers
(5.7) %\refeq{expVol}
only; contracting ones play a secondary role.
''

so it makes sense to you. If you and Adam do not understand it, then
bring it up for discussion in the study group.

\item[2011-04-11 CS]
rewrote Paragraph 1 of Section 5.4 as follows:

\CSedit{
Nearby points aligned along the stable (contracting) directions  remain
in the neighborhood of the trajectory $\ssp(t)= \flow{t}{\xInit}$; the
ones to keep an eye on are the points which leave the neighborhood along
the unstable directions because almost all nonlinear and chaotic
phenomena comes from these directions. The sub-volume $ |\pS_{\xInit}| =
\prod_i^e\Delta \ssp_i$ of the set of points which get no further away
from $\flow{t}{\xInit}$ than $L$, the typical size of the system, is
fixed by the condition that $\Delta \ssp_i \ExpaEig_i = O(L)$ in each
expanding direction $i$. Hence the neighborhood size scales as
$|\pS_{\xInit}| \propto O(L^{d_e})/|\ExpaEig_p| \propto 1/|\ExpaEig_p| $
where $\ExpaEig_p$ is the product of expanding Floquet multipliers
(5.7) %\refeq{expVol}
only(see section 1.5.1 and figure 1.9 for example);
contracting ones play a secondary role}

\item[2011-04-12 PC] Thanks, I have now rewritten the introduction to the
Chapter as well as the section 5.4 is the spirit you suggest, emphasizing
the key role the concept of 'neighborhood' will play.

\end{description}

%%-----   Smooth conjugacies
\section{Chapter: Go straight}
\label{c-conjug}\noindent dasbuch/book/chapter/conjug.tex
\begin{description}

\item[2011-04-11 CS]
Finished this Chapter. Generally able to understand it but feel like
there's whole lot more content underneath that as in the KS
transformation for example. There should be a lot tricks and methods to
construct such regularization. And I wonder what kind of singularity
could be regularized. But I guess that this is not an easy question and
should not be the emphasis to my project.


\item[2011-04-11 CS]
A trivial error: eq.~(6.13) should be ``$\sqrt{x}dx = 2dt$'',
rather than ``$\sqrt{x}dx = \sqrt{2}dt$''.

\item[2011-04-12 PC]
No error is 'trivial.' Thanks.

\end{description}


%%-----   Newton
\section{Chapter: Hamiltonian dynamics}
\label{c-newton}\noindent dasbuch/book/chapter/newton.tex
\begin{description}

\item[2012-01-02 CS to PC]
I am wondering how to choose phase space
coordinates? Do phase/state space coordinates have any requirement and
whether conservation of space volume is such a requirement? What's the
meaning of conservation of phase space requirement? In my understanding
in Hamiltonian flows, conservation of phase space volume means
conservation of energy, am I right?

\item[[2012-01-02 PC]
Now, it is more subtle than that; time dependent flow can be symplectic,
but energy is not conserved; I think Percival and D.
Richards\rf{PerRich82N} (I have it in the
\HREF{http://www.cns.gatech.edu/CNS-only/LibraryCat2.htm} {CNS library})
discuss that well. Symplectic invariance is \emph{much stronger}
requirement than either either energy conservation or phase-space volume
conservation, see \HREF{http://chaosbook.org/chapters/newton.pdf}
{Section 7.4 Poincar\'e invariants} and
\HREF{http://chaosbook.org/chapters/appendStability.pdf} {Appendix D.4
Stability of Hamiltonian flows}: symplectic transformations preserve area
for each $(q,p)$ dual coordinate  pair.

\end{description}


%%-----   Billiards
%\section{Chapter: Billiards}
%\label{c-billiards}\noindent dasbuch/book/chapter/billiards.tex
%\begin{description}
%
%\item[[2012-01-02 PC]
%
%\end{description}

%%-----   Discrete symmetries
\section{Chapter: World in a mirror}
\label{c-discrete}\noindent dasbuch/book/chapter/discrete.tex
\begin{description}

\item[2011-03-30  CS to PC]
I have a question from the paragraph
following the definition of free action: The splitting of a group
\emph{G} into a stabilizer \emph{$G_{p}$} and \emph{m-1} coset
\emph{$cG_{p}$} relates to an orbit \emph{$M_{p}$} to \emph{m-1} other
distinct orbits \emph{$cM_{p}$}. All of them have equivalent stabilizers,
or more precisely, the points on the same group orbit have
\emph{conjugate stabilizers}: \emph{$G_{cp} = cG_{p}c^{-1}$}. For the
last sentence, does it mean that if \emph{$G_{p}$}  is a stabilizer of
\emph{$M_{p}$}, then \emph{$cG_{p}c^{-1}$} is a stabilizer of
\emph{$cM_{p}$}?

\item[2011-03-31 PC] Yes, you are right. I have now incorporated ``if
\emph{$G_{p}$}  is a stabilizer of \emph{$M_{p}$}, then
\emph{$cG_{p}c^{-1}$} is a stabilizer of \emph{$cM_{p}$}'' into
discrete.tex, thanks.

I intend to excise the dreaded word `stabilizer' from the text, just have
forgotten to do it
\HREF{http://www.flickr.com/photos/birdtracks/4259634492/in/set-72157606259014811/}
{[click]} here. Suggestion - print out the chapter, replace by hand word
`stabilizer' everywhere by 'symmetry' and let's sit together and see
whether the chapter is  easier to read.

\item[2011-04-19 CS]
BTW, any recommendation for a hands on book for Lie group and Lie
algebra? I think I need a deeper knowledge about this.

\item[2011-04-20 PC] I suggested in the siminos/blog/
that you check out a book from the library that Meiss recommends. Do it, see
whether it eases the pain. You can also get
Gilmore and Letellier\rf{GL-Gil07b}
{\em The Symmetry of Chaos} out of the CNS library. It only does the
invariant polynomial reduction (Siminos and I believe that is useless in
higher dimensions), but it is pretty good on discrete symmetries.

\end{description}


%%-----   Continuous symmetries
\section{Chapter: Relativity for cyclists}
\label{c-continuous}
\noindent dasbuch/book/chapter/continuous.tex

\begin{description}

\item[2009-10-03 PC]
moved to here flotsam from ChaosBook.org chapter continuous.tex,
{\em Relativity for cyclists}.

The moving frames method allows the determination of (non-polynomial)
invariants of the group action by a simple and efficient
algorithm that works well in high-dim\-ens\-ion\-al \statesp s.
    \PC{Vaggelis, add references here? {\bf ES}: mmm... SiminosThesis?}
    \PC{Vaggelis, why ``(non-polynomial)'' invariants?
        length$^2$ is polynomial {\bf ES}: It is the only one though.
        {\bf PC}: is this an answer?}

In nomenclature of \refpage{hOdes}, the {\jacobianM}
maps the initial, {Lagrangian coordinate frame} into
the current, {Eulerian coordinate frame}.
\index{Lagrangian!coordinates}
\index{Eulerian!coordinates}


[1] Group Theory with Applications in Chemical Physics, Patrick W. M.:
``A sum of the elements belonging to a class is called Dirac character.''


[2] Group Representation Theory for Physicists
 By Jin-Quan Chen, Jialun Ping (looks nice, get it):

``A sum of the elements belonging to a class is called class operator.
Class operators need not be unitary.They commute with every element of
$\Group$, hence they commute among themselves.
They are closed under multiplication. They form a class algebra.''

Symmetry and Structure: Readable Group Theory for Chemists, 3rd Edition
John Wiley and Sons Ltd, Sep 2007

Applied Group Theory, G. G. Hall;
review by John McKay
The Mathematical Gazette, Vol. 53 (1969), p. 457 on JSTOR


\item[2009-01-07 Predrag] Dropped this: \\
``
 for some  $\period{p}$ is sometimes
 misleadingly called a \emph{spatio-temporal symmetry} of the
 solution $\ssp_p$, where `spatio' refers to segments of an
 orbit mapped into other segments within the \statesp\ of a
 dynamical system, not to a spatially extended system.
 ''

``
For this, and the preceding chapter only, we also relax the nomenclature
a bit:
   `trajectory' is here understood in the dynamical sense,
   both in the full and in the \reducedsp,
   while `orbit' sometimes refers to `group orbit.'
   `Equilibrium' refers to fixed point of a map as well,
   with `fixed point' reserved to the group actions only.
''
``{\bf Domain for fundamentalists}''

as misleading.

\item[2009-01-07 Predrag]
clipped from Fassbender, Mackey, Mackey and Xu 1997; boring, so omit...
A real $[\!2n\times\!2n]$ matrix of the form
\beq
H = \MatrixII{E}{F}{G}{-E^T}
\ee{HamMatr1}
is said to be Hamiltonian if $E, F, G \in \reals^{n \times n}$,
 with $F^T = F$ and $G^T = G$. Equivalently, one
may characterize the $[\!2n\times\!2n]$  Hamiltonian matrices $H$
in $\in \reals^{2n \times 2n}$ by
\beq
({\bf \omega}H)^T = {\bf \omega}H
\,.
\ee{HamMatr2a}
Or, equivalently:

Let $S$ be a real symmetric matrix. Matrices of the form ${\bf \omega}S$ are
called \emph{Hamiltonian matrices}. It is easily checked that
the exponential of
${\bf \omega}S$,
%(10)
\index{Hamiltonian!matrix}
\beq
\LieEl = e^{{\bf \omega}S}
\ee{flotSymplExcp}
is a symplectic matrix.

\item[2009-01-07 Predrag]
Vectors in the dual space $\overline{q}$
transform as
\[
q'^a = \LieEl^a{}_b q^b \, .
\]


For continuous symmetries the isotropy subgroup $\Group_{{x}}$
can be any continuous or discrete subgroup of $\Group$,
    \ESedit{
though not always, for reasons illustrated by the $D_3$
example.
    }

\item[2009-01-07 Predrag]

Wulff\rf{Wulff03} refers to ``a relative periodic orbit which
is non-degenerate modulo isotropy.''

See
\HREF{http://personal.maths.surrey.ac.uk/st/C.Wulff/publicationsframes.html}{Wulff
papers}.


\item[2009-01-07 Predrag]
Our labeling convention is usually
lexical, \ie, we label a cycle by the periodic point whose label has the
lowest value, and we label a class of degenerate cycles
by the one with the lowest label $\hat{p}$.
In what follows we shall drop the hat in $\hat{p}$ when it is clear
from the context that we are dealing with symmetry distinct classes
of cycles.
%    \PC{\refexer{exer:protoLorenz}:
%    % figs/lorenz\_attractor.eps is 0.23MB - use one's own of 30KB?
%    figs/miranda\_stone\_lorenz.eps 120KB
%        - reduce to about 30KB?}



\item[2009-01-07 Predrag] track down these: \\
Sartori Nouvo Cimento 1991 \\
Chossat Nonlinearity 1993  \\
K{\oe}ning proc cambridge phil 1996 \\
Leis Documenta mathematica 1997

\item[2011-04-19 CS]
I am stuck at trying to understand eq.~(10.16):
\[
\oint\frac{d\theta}{2\pi}(\textbf{T}u(\theta))^{\emph{T}}\textbf{T}u(2\pi-\theta)
 = \sum\limits_{m=0}^{\infty}m^{2}(u_{1}^{(m)2}+u_{2}^{(m)2})
\]
First, why would there be a integral on left hand side? Second, why the
integrand is $(\textbf{T}u(\theta))^{\emph{T}}\textbf{T}u(2\pi-\theta)$?
Shouldn't it be $(\textbf{T}u(\theta))^{\emph{T}}\textbf{T}u(\theta)$
according to eq. (10.11)?

\item[2011-04-20 PC] $u(\theta)$ is a function on the interval $[0,2\pi]$,
hence integral on the left side (LHS). Compact support, hence the infinite sum
on the RHS. If I remember right (check notes or the textbook from
our Math Methods course) for Fourier transforms, a convolution on the
left hand side gives me a product on the right hand side. If I'm wrong,
let me know so I fix this; as you say, does not look like $L^2$ norm...

\end{description}

%%-----   Qualitative dynamics, pedestrian
%\section{Chapter: Charting the state space}
%\label{c-knead}\noindent dasbuch/book/chapter/knead.tex
%\begin{description}\item[[2012-01-?? PC]
%
%\end{description}

%%-----   Qualitative dynamics, for cylists
\section{Chapter: Stretch, fold, prune}
\label{c-smale}\noindent dasbuch/book/chapter/smale.tex
\begin{description}

\item[2011-07-19 CS]
About Example 12.3 H\'enon repeller complete horseshoe:
First, in figure 12.4, why after one step's evolution, point B occupies
point D's original spot and also point C and point D are in the stable
manifold? I do not think this is just a coincidence but still haven't
figure out the reason.

\item[2011-07-22 CS] I read the Smale Horseshoe part in Kai's thesis. Now
I understand why the binary sequence has to be assigned this way. Only in
this way can the symbol represent the actual map. It's hard to describe
it but this picture found from wiki helped me a lot.

%%%%%%%%%%%%%%%%%%%%%%%%%%%%%%%%%%%%%%%%%%%%%%%%%%%%%%%%%%%%%%%%%%
\SFIG{SmaleHorseshoe} %{HMV}
{}{
This is a Smale horseshoe that I snitched from an unnamed wiki
without attribution. Cute, no?
    }{Fig:SmaleHorseshoe}
%%%%%%%%%%%%%%%%%%%%%%%%%%%%%%%%%%%%%%%%%%%%%%%%%%%%%%%%%%%%%%%%%

In Kai's thesis, he mentioned ``since the horseshoe is a diffeomorphism
we need to know both the future and the past.'' With the help of this
picture I kind of understand this.

Also I found something interesting that animate the procedure of H\'enon
map horseshoe:
\HREF{http://www.ibiblio.org/e-notes/Chaos/henon.htm}
{www.ibiblio.org/e-notes/Chaos/henon.htm}

\item[2011-07-23 PC]                                        \inCB
I had noticed Demidov's website before - you are right, these simulations
are very instructive, I have now added a remark about them to ChaosBook.
He uses a different definition for parameters $a$ and $b$ from H\'enon,
but unfortunately uses the same letters. His definition is natural if one
is interested in Julia sets, but unfortunately not the one H\'enon used,
and I always try to follow the foundational papers, rather than confusing
everybody with sly parameter redefinitions.


\item[2011-07-20 PC]                                        \toCB
Concerning the H\'enon attractor \underline{not} being symmetric across
the diagonal in general: check my
\HREF{http://chaosbook.org/version13/Maribor11.shtml}{Maribor lectures}.
In \HREF{http://chaosbook.org/overheads/dimension/dimension.pdf}{piece
\#5}: ``Dynamics in infinitely many dimensions'' slide 10 shows the
stable / unstable manifolds for the canonical H\'enon attractor - clearly
very asymmetric.

%%%%%%%%%%%%%%%%%%%%%%%%%%%%%%%%%%%%%%%%%%%%%%%%%%%%%%%%%%%%%%%%%%
\SFIG{Demidov_a-6_b-1}
{}{
PC: The Smale backward-forward horseshoe generated by the
Demidov\rf{DemChaos} java applets for the H\'enon parameter values
$(a,b) = (6,-1)$.
    }{Fig:Demidov}
%%%%%%%%%%%%%%%%%%%%%%%%%%%%%%%%%%%%%%%%%%%%%%%%%%%%%%%%%%%%%%%%%

\item[2011-07-24 PC]                                        \toCB
H\'enon's parametrization\rf{henon}:
\index{Henon@H\'enon map}
\index{map!H\'enon}
\bea
    x_{n+1}&=&1-ax^2_n+b y_n
        \continue
    y_{n+1}&=& x_n
\,.
\label{eq2.1a}
\eea
Demidov's parametrization\rf{DemChaos} of the H\'enon map is:
\bea
    x_{n+1}' &=& a'+ {x'}{}^2_n + b' y_n'
        \continue
    y_{n+1}' &=& x_n'
\,.
\label{DemidHen}
\eea
Dividing through by $a'$ we get
\(
\frac{x_{n+1}'}{a'} = 1 + a'\left(\frac{x_n'}{a'}\right)^2 + b'\frac{y_n'}{a'}
\,,
\)
so the two parametrizations are related by:
    \CS{my guess was
\[ %beq
x={x'}/{a'}
\,,\qquad
a=-{1}/{a'}
\,,\qquad b= {b'}/{a'}
\] %ee{DemidHenChao}
    }
\beq
x={x'}/{a'}
\,,\quad
y={y'}/{a'}
\,;\qquad
a=-{a'}
\,,\quad b= {b'}
\,.
\ee{DemidHenPar}
You need the transformation between two definitions, if you are
going to use Demidov's simulations to test your ideas, and it helps greatly
if the transformation formula is the correct one. If
I am right, if you chose
\(
a'=-6
\,,\quad
b'= -1
\)
Demidov's java applets reproduce the figures in ChaosBook?

\end{description}


%%-----   Finding fixed points
%\section{Chapter: Fixed points, and how to get them}
%\label{c-cycles}\noindent dasbuch/book/chapter/cycles.tex
%\begin{description}\item[[2012-01-?? PC]
%
%\end{description}


%%-----   Walk about: Markov graphs
%\section{Chapter: Walkabout: Transition graphs}
%\label{c-Markov}\noindent dasbuch/book/chapter/Markov.tex
%\begin{description}\item[[2012-01-?? PC]
%
%\end{description}
%
%%%-----   Counting
%\section{Chapter: Counting}
%\label{c-count}\noindent dasbuch/book/chapter/count.tex
%\begin{description}\item[[2012-01-?? PC]
%
%\end{description}
%
%%%-----   Transporting densities
%\section{Chapter: Transporting densities}
%\label{c-measure}\noindent dasbuch/book/chapter/measure.tex
%\begin{description}\item[[2012-01-?? PC]
%
%\end{description}
%
%%%-----   Averaging
%\section{Chapter: Averaging}
%\label{c-average}\noindent dasbuch/book/chapter/average.tex
%\begin{description}\item[[2012-01-?? PC]
%
%\end{description}
%
%%%-----   Trace formulas
%\section{Chapter: Trace formulas}
%\label{c-trace}\noindent dasbuch/book/chapter/trace.tex
%\begin{description}\item[[2012-01-?? PC]
%
%\end{description}


%%-----   Spectral determinants
%\section{Chapter: Spectral determinants}
%\label{c-det}\noindent dasbuch/book/chapter/det.tex
%\begin{description}\item[[2012-01-?? PC]
%
%\end{description}
%
%%%-----   Cycle expansions
%\section{Chapter: Cycle expansions}
%\label{c-recycle}\noindent dasbuch/book/chapter/recycle.tex
%\begin{description}\item[[2012-01-?? PC]
%
%\end{description}
%
%%%-----   Discrete symmetries
%\section{Chapter: Discrete factorization}
%\label{c-symm}\noindent dasbuch/book/chapter/symm.tex
%\begin{description}\item[[2012-01-?? PC]
%
%\end{description}

%%%-----   Why cycle?
%\section{Chapter: }\label{c-flows}\noindent dasbuch/book/chapter/getused.tex
%
%
%%%-----   Why does it work?
%\section{Chapter: }\label{c-flows}\noindent dasbuch/book/chapter/converg.tex
%
%
%%%-----   Intermittency
%\section{Chapter: }\label{c-flows}\noindent dasbuch/book/chapter/inter.tex
%
%
%%%-----   Relativity for cyclists
%\section{Chapter: }\label{c-flows}\noindent dasbuch/book/chapter/rpo.tex
%
%
%%%-----   Diffusion confusion
\section{Chapter: Deterministic diffusion}
\label{c-flows}\noindent dasbuch/book/chapter/diffusion.tex
\begin{description}
\item[2011-12-19 PC] An interesting paper, worth study:
Georgie Knight, Orestis Georgiou, Carl Dettmann, and Rainer Klages say in
\emph{Where to place a hole to achieve a maximal diffusion coefficient},
 \arXiv{1112.3922}: ``
A particle moving deterministically in a chaotic spatially extended
environment can exhibit normal diffusion, with its mean square
displacement growing proportional to the time. Here we consider the
dependence of the diffusion coefficient on the size and the position of
dynamical channels (`holes') linking spatial regions. The system
properties can be obtained analytically via a Taylor-Green-Kubo formula
in terms of a functional recursion relation, leading to a diffusion
coefficient varying with the hole positions and non-monotonically on
their size. We derive analytic formulas for small holes in terms of
periodic orbits covered by the holes. The asymptotic regimes that we
observe show deviations from a simple random walk approximation, a
phenomenon that should be ubiquitous in dynamical systems and might be
observed experimentally. The escape rate of the corresponding open system
is also calculated. The resulting parameter dependencies are compared
with the ones for the diffusion coefficient and explained in terms of
periodic orbits.
''

\end{description}

%%-----   PDEs
\section{Chapter: Turbulence?}
\label{c-PDEs}\noindent dasbuch/book/chapter/PDEs.tex 30aug2011
\begin{description}


\item[2011-08-24 PC to Chao] ChaosBook.org chapter
\HREF{http://ChaosBook.org/paper.shtml\#PDEs}{Turbulence?}
explains the relation between $L$ and the hyperviscosity
$\nu$, and it should be the fastest introduction to \KSe.

\item[2011-08-28 Chao]

In the second paragraph of section 26.1.1, it is said ``KS equation is
Galilean invariant: if $u(x,t)$ is a solution, then $v+u(x+2vt,t)$ with
$v$ an arbitrary constant velocity is also a solution.'' So I substitute
$v+u(x+2vt,t)$ into KS equation given by 26.2, but it seems that the new
equation does not keep the original form with additional term $3vu_x$.
Would you please show me explicitly how this works?

\item[2011-08-30 Predrag] thanks for catching this typo, it should be
$v+u(x-vt,t)$. Corrected now.

\item[2011-08-28 Chao]
I do not understand the dimensional analysis in the following
paragraph. Why does time has the dimension of square of length? Why does
viscosity has the same dimension with time? Plus, where does
``viscosity'' show up in the original KS equation 26.2?

\item[2011-08-30 Predrag] thanks for catching this typo (I had not
entered the hyper-viscosity parameter $\nu$ in the defining equation).

\item[2011-08-28 Chao]
In the second paragraph of page 516, we obtain two equilibrium points and
each of them has totally three dimensional manifolds and three Floquet
multipliers. Since KS equation is one-dimensional, where are the two
other dimensions from? I saw that you put the two equilibrium points in
three-dimensional space and extend the coordinate to be $c_+ =
(\sqrt{c},0,0),c_- = (-\sqrt{c},0,0)$, but this does not bring in the
dependence on the other two dimensions, right?

\noindent
[ ] {\bf 2011-08-30 Predrag} mark this box once you have entered the
answers into the blog.

\item[2011-09-10 PC]
\KSe\ is a PDE in one or two spatial dimensions, so it corresponds to an
\emph{infinity} of ODEs. \Eqv\ condition for 1 spatial dimension KS sets
the left-hand of the equation to zero, so the right hand side (in the 1
spatial dimension case \emph{only}) becomes a 4th order ODE in $d/dx$.
One integral you can do, so the system becomes a 3rd order ODE +
integration constant, or 3 first order ODEs. The best papers on this are

\noindent
[ ] Michelson\rf{Mks86}

\noindent
[ ] Lan and Cvitanovi{\'c}\rf{lanCvit07}, discussed here in
\refsect{s:lanCvit07}

\noindent
[ ] Lan\rf{LanThesis} thesis

\noindent
Chao, mark above boxes once you have read the above sources and entered
material learned from them either here, or in separate sections, one for
each source.

\item[2011-08-28 Chao]
Up to this point I am confused with another point that in Chapter 4, when
defining Floquet multipliers and stable/unstable manifold, we
introduced Jacobian matrix of continuous time-dependent map $f(x,t)$. We
expand the map f to first order and get Jacobian matrix as the
coefficient matrix. But here in KS system, we do not know the explicit
form of $f(x,t)$, which should be the solution. Then how do we deduce the
Jacobian matrix without knowing the map explicitly?

\item[2011-09-10 PC] You almost never have an explicit formula for the
\JacobianM, it is always a numerically computed linearized map. What you
do have is an explicit formula for the {\stabmat} ${\Mvar}$, in the case
of KS you have written it out in \refeq{KSstabMat2}.

\end{description}


%-----   Dimension of turbulence
\section{Chapter: Dimension of turbulence}
\label{c-dimension}\noindent dasbuch/book/chapter/dimension.tex
\begin{description}

\item[2011-12-01 PC to Jim Yorke]
I agree with you - why attach any names to equations?  I like the thing
referred to as ``Kaplan-Yorke dimension''. Do you have a more descriptive
name for it?

\item[2011-12-01 Jim Yorke to PC] The Kaplan Yorke Dimension is a formula
involving Lyapunov exponents. So we called the KY Dimension. No -- just
joking. We called it the Lyapunov Dimension. Logical?

\item[2011-12-01 PC to Jim Yorke] \emph{Who is the 2000th person who
invoked Lyapunov's name in vain?} Poor Lyapunov. He is a surname, not
concept, but his name gets attached to zillion things he had nothing to
do with.

There is something called ``Lyapunov equation'' which actually is in his
thesis, at least for the strictly contracting fixed point case.

Then there things like ``Lyapunov Covariant Vectors'' that he has nothing
to do with in any imaginable way which yield a ``physical dimension'' of
attractors of PDEs such as Kuramoto-Sivashinsky and Landau-Ginzburg. That
DOES NOT count Lyapunov exponents, it counts non-hyperbolically connected
``Covariant Vectors'', and seems to give a number roughly twice the KY
dimension.

So I'll stick to calling the KY dimension  ``Kaplan Yorke dimension'', and
leave poor Belorussian aristocrat out of this.

\end{description}

%-----   Feigenbaum for cyclists
\section{Chapter: Universality in transitions to chaos}
\label{c-UFO}\noindent dasbuch/book/chapter/UFO.tex
\begin{description}
\item[2011-12-01 PC to Jim Yorke]
\emph{Who is the 2nd person who invented General Relativity?}
(Answer: Who remembers?)

Dear Jim, I know you do not find renormalization in period doubling
important, but if you do mention it:


Feigenbaum discovered and fully formulated period-doubling universality
in 1975, you can click on the link to Feigenbaum's first report from 1976
\HREF{http://www.cns.gatech.edu/~predrag/papers/preprints.html\#Trans2chaos}{here}.
In 1981 Lanford satisfied himself that the iterative method we used and
knew was contracting was indeed contracting. He refers only to the
Feigenbaum paper.
\HREF{http://hal.archives-ouvertes.fr/docs/00/21/74/80/PDF/ajp-jphyscol197839C513.pdf}
{Coullet and Tresser} refer to the Feigenbaum paper. It's there something
essential that is missing in the 1976 formulation?

I'm very fond of all three of them - all creative and crazy as bats. I
just do not get their names on the equation that was widely known and
publicized well before 1978?

As far as I can tell, Lyubich renamed the equations. I asked him why? He
wrote:

``In 90s, I talked to both Feigenbaum and Tresser, and my conclusion was
that Coullet-Tresser discovered the phenomenon independently, though
slightly later. Also, they seemed to recognize better importance of the
dynamical universality (while Feigenbaum focused more on  the parameter
phenomenon.) I felt that Coullet-Tresser did  not receive a proper credit
for their insights, so I attached all three names to the phenomenon.''

OK, that's nice. Feigenbaum and I do not recognize ``importance of the
dynamical universality'', whatever that might mean. While at it, why not
give credit to the guy who wrote the fixed point equation first? What am
I, dog meat? Never mind...

People reinvent stuff all the time. Myrheim and I generalized period
doubling to
\HREF{http://www.cns.gatech.edu/~predrag/papers/preprints.html\#ComplexRenorm}
{infinity of renomalizations} in the complex plane, but once I was told
that Golberg, Sinai and Khanin did it first (for period tripling), I
refer to them even though we discovered it without knowing that they did
it, both in 1983.

I agree with you - why attach any names to equations? Anyway, it's just
the way things are. Pretty soon the attribution problems will sort
themselves out by themselves - heart attacks and homicidal Atlanta
drivers will take care of that. Cheers \& red socks forever

\item[2011-12-01 Jim Yorke to PC] Thank you very much for the detailed
and enlightening discussion.

\end{description}

%%-----   Complex universality
%\section{Chapter: Complex universality}
%\label{c-complex}\noindent dasbuch/book/chapter/complex.tex
%\begin{description}\item[[2012-01-?? PC]
%
%\end{description}

%%-----  "Semiclassics" for noise
%\section{Chapter: Noise}
%\label{c-noise}\noindent dasbuch/book/chapter/noise.tex
%\begin{description}\item[[2012-01-?? PC]
%
%\end{description}

%%-----   Finding cycles variationally
%\section{Chapter: Relaxation for cyclists}
%\label{c-relax}\noindent dasbuch/book/chapter/relax.tex
%\begin{description}\item[[2012-01-?? PC]
%
%\end{description}


%%%-----   Appendices
%%\appendix
%
%
%%%-----   A brief history of chaos
%\section{Chapter: }\label{c-flows}\noindent dasbuch/book/chapter/appendHist.tex
%
%
%%%-----   Maps and billiards
%\section{Chapter: }\label{c-flows}\noindent dasbuch/book/chapter/appendB.tex
%
%
%%%-----   Linear algebra, Hamiltonian Jacobians
%\section{Chapter: }\label{c-flows}\noindent dasbuch/book/chapter/appendStability.tex
%
%
%%%-----   Cycles
%\section{Chapter: }\label{c-flows}\noindent dasbuch/book/chapter/appendCycle.tex
%
%%%-----   Symbolic dynamics techniques
%\section{Chapter: }\label{c-flows}\noindent dasbuch/book/chapter/appendSymb.tex
%
%
%%%-----   Counting
%\section{Chapter: }\label{c-flows}\noindent dasbuch/book/chapter/appendCount.tex
%
%
%%%-----   Implementing evolution
%\section{Chapter: }\label{c-flows}\noindent dasbuch/book/chapter/appendMeasure.tex
%
%%%-----   Applications
%\section{Chapter: }\label{c-flows}\noindent dasbuch/book/chapter/appendApplic.tex
%
%
%%%-----   Discrete symmetries
%\section{Chapter: }\label{c-flows}\noindent dasbuch/book/chapter/appendSymm.tex
%
%
%%%-----   Coveregence of spectral determinants
%\section{Chapter: }\label{c-flows}\noindent dasbuch/book/chapter/appendConverg.tex
%
%%%-----   Stat mech
%\section{Chapter: }\label{c-flows}\noindent dasbuch/book/chapter/appendStatM.tex
%
%
%%%-----   Infinite dimensional operators
%\section{Chapter: }\label{c-flows}\noindent dasbuch/book/chapter/appendWirzba.tex
%
%
%%%-----   Statistical Mechanics
%\section{Chapter: }\label{c-flows}\noindent dasbuch/book/chapter/statmech.tex
