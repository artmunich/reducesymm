% siminos/blog/ChaosBook.tex
% $Author$ $Date$

\chapter{ChaosBook.org blog}
\label{chap:ChaosBook}

\begin{description}

\item[2011-10-24 PC]
I have added this chapter to collect your ChaosBook.org edits /
comments. If you svn co dasbuch, you have the dasbuch/book/chapter/*.tex
source code and can clip and paste formulas to here. I never refer to a
chapter by it's current number, as chapter numbers change from edition to
edition - latter on (years hence) trying to figure out what ``Chapter
17'' is can be quite confusing. Internally, each chapter is kept track
off by its file name, for example, in this blog ``continuous'' refers to
\refchap{c-continuous} {\em Relativity for cyclists}.

\item[2011-12-08 PC] Of possible interest from
\HREF{http://chaos.aip.org/?track=CHAOSDEC11}{chaos.aip.org Dec. 11 2011}:

\HREF{http://www.mathjax.org/}{MathJax} is an open-source JavaScript
display engine that produces high-quality math in all modern browsers. In
addition to higher-quality, cross-platform, browser-agnostic equation
rendering, readers of AIP online journals can now copy equations from
journal articles and paste them directly into text editors like Word,
LaTeX, MathType, and research wikis, as well as into calculation software
like Maple, Mathematica, and others.

To see MathJax in action, visit
\HREF{http://chaos.aip.org/resource/1/chaoeh/v21/i4/p043126_s1}{your
favorite Chaos article} and select the Read Online option. Once in the
HTML view, go to the navigation bar and turn on MathJax. From there you
can copy and paste any equation into your favorite MathML-enabled editor.
\HREF{http://www.mathjax.org/demos/copy-and-paste/}{Watch the video}.

\end{description}

\section{ChaosBook.org broken links}
\label{c-brokenLinks}

Predrag should fix these (mark when fixed):
\begin{itemize}
  \item[[~]] 2011-09-27
\HREF{http://www.cns.gatech.edu/~predrag/papers/preprints.html\#KS}
{Vachtang's term paper} link to
\\
www.cns.gatech.edu/$\sim$predrag/papers/vachtang.ps.gz
  \item[[~]] 2011-09-27
\HREF{http://www.cns.gatech.edu/~predrag/papers/preprints.html\#KS}
{Vachtang Putkaradze PhD thesis} link to
\\
www.nbi.dk/$\sim$putkarad
  \item[[~]]
  \item[[~]]
\end{itemize}


%%-----   Flows
%\section{Chapter: Go with the flow}
%\label{c-flows}\noindent dasbuch/book/chapter/flows.tex
%
%\begin{description}
%
%\item[2011-09-11 PC]
%
%\end{description}


%%-----   Maps
%\section{Chapter: Discrete time dynamics}
%\label{c-maps}\noindent dasbuch/book/chapter/maps.tex
%\begin{description}
%
%\item[2011-03-19 CS to PC]
%
%\end{description}

%%-----   Stability
%\section{Chapter: Local stability}
%\label{c-stability}\noindent dasbuch/book/chapter/stability.tex
%\begin{description}
%
%\item[2011-03-19 CS to PC]
%
%\end{description}

%%-----   Cycle stability
%\section{Chapter: Cycle stability}
%\label{c-invariants}\noindent dasbuch/book/chapter/invariants.tex
%\begin{description}
%
%\item[2011-04-06 PC]
%
%\end{description}


%%-----   Smooth conjugacies
%\section{Chapter: Go straight}
%\label{c-conjug}\noindent dasbuch/book/chapter/conjug.tex
%\begin{description}
%
%\item[2011-04-11 CS]
%
%\end{description}


%%-----   Newton
\section{Chapter: Hamiltonian dynamics}
\label{c-newton}\noindent dasbuch/book/chapter/newton.tex
\begin{description}

\item[2011-03-19 CS to PC]
I am wondering how to choose phase space
coordinates? Do phase/state space coordinates have any requirement and
whether conservation of space volume is such a requirement? What's the
meaning of conservation of phase space requirement? In my understanding
in Hamiltonian flows, conservation of phase space volume means
conservation of energy, am I right?

\item[2011-03-25 PC]
Now, it is more subtle than that; time dependent flow can be symplectic,
but energy is not conserved; I think Percival and D.
Richards\rf{PerRich82N} (I have it in the
\HREF{http://www.cns.gatech.edu/CNS-only/LibraryCat2.htm} {CNS library})
discuss that well. Symplectic invariance is \emph{much stronger}
requirement than either either energy conservation or phase-space volume
conservation, see \HREF{http://chaosbook.org/chapters/newton.pdf}
{Section 7.4 Poincar\'e invariants} and
\HREF{http://chaosbook.org/chapters/appendStability.pdf} {Appendix D.4
Stability of Hamiltonian flows}: symplectic transformations preserve area
for each $(q,p)$ dual coordinate  pair.

\end{description}


%%-----   Billiards
%\section{Chapter: Billiards}
%\label{c-billiards}\noindent dasbuch/book/chapter/billiards.tex
%\begin{description}
%
%\item[2011-03-25 PC]
%
%\end{description}


%%-----   Discrete symmetries
%\section{Chapter: World in a mirror}
%\label{c-discrete}\noindent dasbuch/book/chapter/discrete.tex
%\begin{description}
%
%\item[2011-03-31 PC]
%
%\end{description}


%%-----   Continuous symmetries
\section{Chapter: Relativity for cyclists}
\label{c-continuous}
\noindent dasbuch/book/chapter/continuous.tex

\begin{description}

\item[2009-10-03 PC]
moved to here flotsam from ChaosBook.org chapter continuous.tex,
{\em Relativity for cyclists}.

The moving frames method allows the determination of (non-polynomial)
invariants of the group action by a simple and efficient
algorithm that works well in high-dim\-ens\-ion\-al \statesp s.
    \PC{Vaggelis, add references here? {\bf ES}: mmm... SiminosThesis?}
    \PC{Vaggelis, why ``(non-polynomial)'' invariants?
        length$^2$ is polynomial {\bf ES}: It is the only one though.
        {\bf PC}: is this an answer?}

In nomenclature of \refpage{hOdes}, the {\jacobianM}
maps the initial, {Lagrangian coordinate frame} into
the current, {Eulerian coordinate frame}.
\index{Lagrangian!coordinates}
\index{Eulerian!coordinates}


[1] Group Theory with Applications in Chemical Physics, Patrick W. M.:
``A sum of the elements belonging to a class is called Dirac character.''


[2] Group Representation Theory for Physicists
 By Jin-Quan Chen, Jialun Ping (looks nice, get it):

``A sum of the elements belonging to a class is called class operator.
Class operators need not be unitary.They commute with every element of
$\Group$, hence they commute among themselves.
They are closed under multiplication. They form a class algebra.''

Symmetry and Structure: Readable Group Theory for Chemists, 3rd Edition
John Wiley and Sons Ltd, Sep 2007

Applied Group Theory, G. G. Hall;
review by John McKay
The Mathematical Gazette, Vol. 53 (1969), p. 457 on JSTOR


\item[2009-01-07 Predrag] Dropped this: \\
``
 for some  $\period{p}$ is sometimes
 misleadingly called a \emph{spatio-temporal symmetry} of the
 solution $\ssp_p$, where `spatio' refers to segments of an
 orbit mapped into other segments within the \statesp\ of a
 dynamical system, not to a spatially extended system.
 ''

``
For this, and the preceding chapter only, we also relax the nomenclature
a bit:
   `trajectory' is here understood in the dynamical sense,
   both in the full and in the \reducedsp,
   while `orbit' sometimes refers to `group orbit.'
   `Equilibrium' refers to fixed point of a map as well,
   with `fixed point' reserved to the group actions only.
''
``{\bf Domain for fundamentalists}''

as misleading.

\item[2009-01-07 Predrag]
clipped from Fassbender, Mackey, Mackey and Xu 1997; boring, so omit...
A real $[\!2n\times\!2n]$ matrix of the form
\beq
H = \MatrixII{E}{F}{G}{-E^T}
\ee{HamMatr1}
is said to be Hamiltonian if $E, F, G \in \reals^{n \times n}$,
 with $F^T = F$ and $G^T = G$. Equivalently, one
may characterize the $[\!2n\times\!2n]$  Hamiltonian matrices $H$
in $\in \reals^{2n \times 2n}$ by
\beq
({\bf \omega}H)^T = {\bf \omega}H
\,.
\ee{HamMatr2a}
Or, equivalently:

Let $S$ be a real symmetric matrix. Matrices of the form ${\bf \omega}S$ are
called \emph{Hamiltonian matrices}. It is easily checked that
the exponential of
${\bf \omega}S$,
%(10)
\index{Hamiltonian!matrix}
\beq
\LieEl = e^{{\bf \omega}S}
\ee{flotSymplExcp}
is a symplectic matrix.

\item[2009-01-07 Predrag]
Vectors in the dual space $\overline{q}$
transform as
\[
q'^a = \LieEl^a{}_b q^b \, .
\]


For continuous symmetries the isotropy subgroup $\Group_{{x}}$
can be any continuous or discrete subgroup of $\Group$,
    \ESedit{
though not always, for reasons illustrated by the $D_3$
example.
    }

\item[2009-01-07 Predrag]

Wulff\rf{Wulff03} refers to ``a relative periodic orbit which
is non-degenerate modulo isotropy.''

See
\HREF{http://personal.maths.surrey.ac.uk/st/C.Wulff/publicationsframes.html}{Wulff
papers}.


\item[2009-01-07 Predrag]
Our labeling convention is usually
lexical, \ie, we label a cycle by the periodic point whose label has the
lowest value, and we label a class of degenerate cycles
by the one with the lowest label $\hat{p}$.
In what follows we shall drop the hat in $\hat{p}$ when it is clear
from the context that we are dealing with symmetry distinct classes
of cycles.
%    \PC{\refexer{exer:protoLorenz}:
%    % figs/lorenz\_attractor.eps is 0.23MB - use one's own of 30KB?
%    figs/miranda\_stone\_lorenz.eps 120KB
%        - reduce to about 30KB?}



\item[2009-01-07 Predrag] track down these: \\
Sartori Nouvo Cimento 1991 \\
Chossat Nonlinearity 1993  \\
K{\oe}ning proc cambridge phil 1996 \\
Leis Documenta mathematica 1997

\end{description}


%%-----   Qualitative dynamics, pedestrian
%\section{Chapter: Charting the state space}
%\label{c-knead}\noindent dasbuch/book/chapter/knead.tex
%\begin{description}\item[2011-09-?? CS]
%
%\end{description}


%%-----   Qualitative dynamics, for cylists
\section{Chapter: Stretch, fold, prune}
\label{c-smale}\noindent dasbuch/book/chapter/smale.tex
\begin{description}

\item[2011-07-20 PC]                                        \toCB
Concerning the H\'enon attractor \underline{not} being symmetric across
the diagonal in general: check my
\HREF{http://chaosbook.org/version13/Maribor11.shtml}{Maribor lectures}.
In \HREF{http://chaosbook.org/overheads/dimension/dimension.pdf}{piece
\#5}: ``Dynamics in infinitely many dimensions'' slide 10 shows the
stable / unstable manifolds for the canonical H\'enon attractor - clearly
very asymmetric.

\item[2011-07-24 PC]                                        \toCB
H\'enon's parametrization\rf{henon}:
\index{Henon@H\'enon map}
\index{map!H\'enon}
\bea
    x_{n+1}&=&1-ax^2_n+b y_n
        \continue
    y_{n+1}&=& x_n
\,.
\label{eq2.1a}
\eea
Demidov's parametrization\rf{DemChaos} of the H\'enon map is:
\bea
    x_{n+1}' &=& a'+ {x'}{}^2_n + b' y_n'
        \continue
    y_{n+1}' &=& x_n'
\,.
\label{DemidHen}
\eea
Dividing through by $a'$ we get
\(
\frac{x_{n+1}'}{a'} = 1 + a'\left(\frac{x_n'}{a'}\right)^2 + b'\frac{y_n'}{a'}
\,,
\)
so the two parametrizations are related by:
    \CS{my guess was
\[ %beq
x={x'}/{a'}
\,,\qquad
a=-{1}/{a'}
\,,\qquad b= {b'}/{a'}
\] %ee{DemidHenChao}
    }
\beq
x={x'}/{a'}
\,,\quad
y={y'}/{a'}
\,;\qquad
a=-{a'}
\,,\quad b= {b'}
\,.
\ee{DemidHenPar}
You need the transformation between two definitions, if you are
going to use Demidov's simulations to test your ideas, and it helps greatly
if the transformation formula is the correct one. If
I am right, if you chose
\(
a'=-6
\,,\quad
b'= -1
\)
Demidov's java applets reproduce the figures in ChaosBook?

\end{description}


%%-----   Finding fixed points
%\section{Chapter: Fixed points, and how to get them}
%\label{c-cycles}\noindent dasbuch/book/chapter/cycles.tex
%\begin{description}\item[2011-09-?? CS]
%
%\end{description}


%%-----   Walk about: Markov graphs
%\section{Chapter: Walkabout: Transition graphs}
%\label{c-Markov}\noindent dasbuch/book/chapter/Markov.tex
%\begin{description}\item[2011-09-?? CS]
%
%\end{description}
%
%%%-----   Counting
%\section{Chapter: Counting}
%\label{c-count}\noindent dasbuch/book/chapter/count.tex
%\begin{description}\item[2011-09-?? CS]
%
%\end{description}
%
%%%-----   Transporting densities
%\section{Chapter: Transporting densities}
%\label{c-measure}\noindent dasbuch/book/chapter/measure.tex
%\begin{description}\item[2011-09-?? CS]
%
%\end{description}
%
%%%-----   Averaging
%\section{Chapter: Averaging}
%\label{c-average}\noindent dasbuch/book/chapter/average.tex
%\begin{description}\item[2011-09-?? CS]
%
%\end{description}
%
%%%-----   Trace formulas
%\section{Chapter: Trace formulas}
%\label{c-trace}\noindent dasbuch/book/chapter/trace.tex
%\begin{description}\item[2011-09-?? CS]
%
%\end{description}


%%-----   Spectral determinants
%\section{Chapter: Spectral determinants}
%\label{c-det}\noindent dasbuch/book/chapter/det.tex
%\begin{description}\item[2011-09-?? CS]
%
%\end{description}
%
%%%-----   Cycle expansions
%\section{Chapter: Cycle expansions}
%\label{c-recycle}\noindent dasbuch/book/chapter/recycle.tex
%\begin{description}\item[2011-09-?? CS]
%
%\end{description}
%
%%%-----   Discrete symmetries
%\section{Chapter: Discrete factorization}
%\label{c-symm}\noindent dasbuch/book/chapter/symm.tex
%\begin{description}\item[2011-09-?? CS]
%
%\end{description}

%%%-----   Why cycle?
%\section{Chapter: }\label{c-flows}\noindent dasbuch/book/chapter/getused.tex
%
%
%%%-----   Why does it work?
%\section{Chapter: }\label{c-flows}\noindent dasbuch/book/chapter/converg.tex
%
%
%%%-----   Intermittency
%\section{Chapter: }\label{c-flows}\noindent dasbuch/book/chapter/inter.tex
%
%
%%%-----   Relativity for cyclists
%\section{Chapter: }\label{c-flows}\noindent dasbuch/book/chapter/rpo.tex
%
%
%%%-----   Diffusion confusion
%\section{Chapter: }\label{c-flows}\noindent dasbuch/book/chapter/diffusion.tex


%%-----   PDEs
\section{Chapter: Turbulence?}
\label{c-PDEs}\noindent dasbuch/book/chapter/PDEs.tex 30aug2011
\begin{description}


\item[2011-08-24 PC ]

\item[2011-08-28 Chao]
I do not understand the dimensional analysis in the following
paragraph. Why does time has the dimension of square of length? Why does
viscosity has the same dimension with time? Plus, where does
``viscosity'' show up in the original KS equation 26.2?

\item[2011-08-30 Predrag] thanks for catching this typo (I had not
entered the hyper-viscosity parameter $\nu$ in the defining equation).
As to the rest, we have talked about it and understood it verbally:

\end{description}

%%-----  "Semiclassics" for noise
%\section{Chapter: Noise}
%\label{c-noise}\noindent dasbuch/book/chapter/noise.tex
%\begin{description}\item[2011-09-?? CS]
%
%\end{description}

%%-----   Finding cycles variationally
%\section{Chapter: Relaxation for cyclists}
%\label{c-relax}\noindent dasbuch/book/chapter/relax.tex
%\begin{description}\item[2011-09-?? CS]
%
%\end{description}


%%%-----   Appendices
%%\appendix
%
%
%%%-----   A brief history of chaos
%\section{Chapter: }\label{c-flows}\noindent dasbuch/book/chapter/appendHist.tex
%
%
%%%-----   Maps and billiards
%\section{Chapter: }\label{c-flows}\noindent dasbuch/book/chapter/appendB.tex
%
%
%%%-----   Linear algebra, Hamiltonian Jacobians
%\section{Chapter: }\label{c-flows}\noindent dasbuch/book/chapter/appendStability.tex
%
%
%%%-----   Cycles
%\section{Chapter: }\label{c-flows}\noindent dasbuch/book/chapter/appendCycle.tex
%
%%%-----   Symbolic dynamics techniques
%\section{Chapter: }\label{c-flows}\noindent dasbuch/book/chapter/appendSymb.tex
%
%
%%%-----   Counting
%\section{Chapter: }\label{c-flows}\noindent dasbuch/book/chapter/appendCount.tex
%
%
%%%-----   Implementing evolution
%\section{Chapter: }\label{c-flows}\noindent dasbuch/book/chapter/appendMeasure.tex
%
%%%-----   Applications
%\section{Chapter: }\label{c-flows}\noindent dasbuch/book/chapter/appendApplic.tex
%
%
%%%-----   Discrete symmetries
%\section{Chapter: }\label{c-flows}\noindent dasbuch/book/chapter/appendSymm.tex
%
%
%%%-----   Coveregence of spectral determinants
%\section{Chapter: }\label{c-flows}\noindent dasbuch/book/chapter/appendConverg.tex
%
%%%-----   Stat mech
%\section{Chapter: }\label{c-flows}\noindent dasbuch/book/chapter/appendStatM.tex
%
%
%%%-----   Infinite dimensional operators
%\section{Chapter: }\label{c-flows}\noindent dasbuch/book/chapter/appendWirzba.tex
%
%
%%%-----   Statistical Mechanics
%\section{Chapter: }\label{c-flows}\noindent dasbuch/book/chapter/statmech.tex
