% siminos/blog/lasers.tex
% $Author$ $Date$

\chapter{Laser physics: The lingo}
\label{chap:lasers}

\renewcommand{\LieElrep}{\ensuremath{\mathbf{G}}} % Siminos Lie group element
\renewcommand{\LieEl}{\ensuremath{\gamma}}  % also a Siminos Lie group element
\renewcommand{\gSpace}{\ensuremath{{\bf \theta}}}   % group rotation parameters


\begin{description}

\item[2011-05-27 Predrag, laser report from Snowbird DS11]
												\toCB
By fortuitous coincidence, Evangelos choice of \cLe\ is iconic
{\em both}  for
the weathermen (a model of baroclinic instability) and for the laser
community (they always have \SOn{2} or $S^1$ invariance).

\HREF{http://empslocal.ex.ac.uk/people/staff/he219/}{Hartmut Erzgr\"aber},
\HREF{http://www.enm.bris.ac.uk/staff/berndk/}{Bernd  Krauskopf} (his PhD adviser), and
\HREF{http://empslocal.ex.ac.uk/people/staff/smw206/}{Sebastian M. Wieczorek}
have tried all things we have tried in reducing \SOn{2} symmetry, and
have settled for invariant polynomial basis going from 6 dimensions to 40,
if I remember right. They study dynamics of
lasers coupled by a passive resonator (the papers are on Wieczorek
homepage, I have not found the right one yet).

												\toCB
Laser people talk about $S^1$ rather than \SOn{2} symmetry;
I do not like this notation, I think $S^1$ refers to a geometrical
object, a circle, while \SOn{2} is the symmetry group.


\item[2011-08-30 Predrag]
{\em Frequency locking by external forcing in systems with rotational
symmetry } by Lutz Recke, Anatoly Samoilenko, Viktor Tkachenko, and
Serhiy Yanchuk\rf{ReSaTkYa11}: `` study locking of the modulation
frequency of a relative periodic orbit (modulated wave) in a
$S^{1}$-equivariant system of ordinary differential equations under an
external forcing of modulated wave type. Applications are external
electrical and/or optical forcing of selfpulsating states of lasers.
''

I find this paper very clear and instructive, so I'm reproducing parts of
it here. My comments are to their text, clipped from the arXiv source
file:

\begin{equation}
\frac{dx}{dt}=\vel(x)+\gamma g(x,\beta t,\alpha t).
\label{01}
\end{equation}
Here $\vel:\mathbb{R}^{n}\to\mathbb{R}^{n}$ and
$g:\mathbb{R}^{n}\times\mathbb{R}^{2}\to\mathbb{R}^{n}$ are
$C^{l}$-smooth with $l>3$, and $\alpha>0,$ $\beta>0,$ $\gamma\ge0$ are
the control parameters. The vector field $\vel$ is $S^{1}$-equivariant
(Predrag:
$S^{1}$-equivariant is what they call it in laser physics - we call it
$\SOn{2}$ symmetry. In our notation, $A$ is the Lie algebra element $\Lg$.)
\begin{equation}
\vel(e^{A\xi}x)=e^{A\xi}\vel(x)
\label{eq:symf}
\end{equation}
for all $x\in\mathbb{R}^{n}$ and $\xi\in\mathbb{R}$, where $A$
is a non-zero real $n\times n$-matrix such that $A^{T}=-A$ and $e^{2\pi A}=I$.
The forcing term $g$ is $2\pi$-periodic with respect
to the second and third arguments
\[
g(x,\psi+2\pi,\varphi)=g(x,\psi,\varphi)=g(x,\psi,\varphi+2\pi),
\]
 and has the symmetry
\begin{equation}
g(e^{A\xi}x,\psi,\varphi+\xi)=e^{A\xi}g(x,\psi,\varphi)
\label{eq:symg}
\end{equation}
 for all $x\in\mathbb{R}^{n}$ and $\varphi,\psi,\xi\in\mathbb{R}.$
Assume that the unperturbed system
\begin{equation}
\frac{dx}{dt}=\vel(x)\label{02}
\end{equation}
 has an exponentially orbitally stable modulated wave solution
\begin{equation}
x(t)=e^{A\alpha_{0}t}x_{0}(\beta_{0}t).
\label{qp}
\end{equation}
                                                                \toCB
Here $x_{0}:\mathbb{R}\to\mathbb{R}^{n}$ is $2\pi$-periodic, and
$\alpha_{0}>0$, $\beta_{0}>0$ are the \emph{wave and modulation
frequencies} of the solution (\ref{qp}). (Predrag: this `modulated wave
solution' is in our notation \rpo\ with period $\period{}=\beta_{0}t$ and
phase $\gSpace =\alpha_{0}t$).
Assume that the following nondegeneracy
condition holds:
\begin{equation}
\mbox{The vectors}\ Ax_{0}(\psi)\ \mbox{and}\
     \frac{dx_{0}}{d\psi}(\psi)\ \mbox{are linearly independent.}
\label{eq:nondeg}
\end{equation}
(Predrag:
the group tangent $\groupTan(x) =  Ax_{0}(\psi)$ and $\vel(x_{0}(\psi))$
are not aligned, \ie, exclude \reqva.)
Then (\ref{eq:nondeg}) is true for all $\psi\in\mathbb{R}$
if it holds for one $\psi$. Condition (\ref{eq:nondeg}) implies
that the set
\begin{equation}
\mathcal{T}_{2}:=\{(e^{A\varphi}x_{0}(\psi))\in\mathbb{R}^{n}:\quad\varphi,\psi\in\mathbb{R}\},
\label{eq:torus}
\end{equation}
which is invariant with respect to the flow corresponding to (\ref{02}),
is a two-dimensional torus.
(Predrag:
For $\SOn{2}$ the \rpo\ invariant set is then a
2-torus.)

They describe open sets in the three-dimensional space
of the control parameters $\alpha$, $\beta$ and $\gamma$ with $|\alpha-\alpha_{0}|\gg1$
and $\beta\approx\beta_{0}$, where stable locking of the modulation
frequencies $\beta$ of the forcing and $\beta_{0}$ of the modulated
wave solution (\ref{qp}) occurs, i.e. where the following holds:
For almost any solution $x(t)$ to (\ref{01}), which is at a certain
moment close to $\mathcal{T}_{2}$, there exists $\sigma\in\mathbb{R}$
such that
\begin{equation}
\inf_{\psi}\|x(t)-e^{A\psi}x_{0}(\beta t+\sigma)\|\approx0\mbox{ for large }t.
\label{eq:cond}
\end{equation}
(Predrag:
The \rpo\ is attractive, so solutions approach the \rpo\ 2-torus and
modelock. Apparently the shift $\psi = \beta t$, though I am not sure
about that. The next paragraph is uncommented from the arXiv source file, not
in the published version:)

System (\ref{02}) is equivariant under the $S^{1}$-action $x\mapsto e^{A\xi}x$
in the phase space $\mathbb{R}^{n}$. The solution (\ref{qp}) is
a so-called modulated wave solution or relative periodic orbit to
(\ref{02}) (see, e.g. \cite{Rand82}). %,Renardy1982}).
\PC{find, read reference `Renardy1982'}
Thus, our results describe
the behavior of exponentially orbitally stable modulated wave solutions
to $S^{1}$-equivariant systems under external forces $\gamma g(x,\beta t,\alpha t)$.
The parameters $\alpha$ and $\beta$ are called wave and modulation
frequencies of those forces. The assumed property (\ref{eq:symg})
of the forcing is quite natural in many situations.
For instance, it appears generically when the autonomous system
(\ref{02}) is forced by a modulated
wave {}``additively'', e.g., as
\[
\frac{dx}{dt}=f(x+\gamma_{1}
e^{A\alpha t}\bar{g}_{1}(\beta t))+\gamma_{2}e^{A\alpha t}\bar{g}_{2}(\beta t).
\]

The present paper extends previous work on this topic for particular
types of the vector field $f$ and the forces $g$ \cite{Schneider2005,Samoilenko2005,Recke2010}.
In particular, the considered special cases in \cite{Schneider2005,Samoilenko2005,Recke2010}
are doubly degenerate in the sense that not only the averaged forcing
%(\ref{av})
vanishes but also the second averaging term turns to zero.
Remark that in \cite{Recke1998} related results are described for
the case that both differences of modulation and wave frequencies
are small, and \cite{Recke1998a} concerns the case when the internal
state as well as the external forcing are not modulated. For an even
more abstract setting of these results see \cite{Chillingworth2000}.

Systems of the type (\ref{01}) appear as models for the dynamical
behavior of self-pulsating lasers under the influence of external
periodically modulated optical and/or electrical signals, see, e.g.
\cite{Radziunas2006,Bandelow1998,Lichtner2007,Nizette2001,Peterhof1999,Sieber2002,Wieczorek2005},
and for related experimental results see \cite{Feiste1994,Sartorius1998}.
In (\ref{01}) the state variable $x$ describes the electron density
and the optical field of the laser. In particular, the Euclidian norm
$\|x\|$ describes the sum of the electron density and the intensity
of the optical field. The $S^{1}$-equivariance of (\ref{02}) is
the result of the invariance of autonomous optical models with respect
to shifts of optical phases. The solution (\ref{qp}) describes a
so-called self-pulsating state of the laser in the case that the laser
is driven by electric currents which are constant in time. In those
states the electron density and the intensity of the optical field
are time periodic with the same frequency. Self-pulsating states usually
appear as a result of Hopf bifurcations from so-called continuous
wave states, where the electron density and the intensity of the optical
field are constant in time.

External forces of the type $\gamma e^{i\alpha t}\bar{g}(\beta t)$
\[
g(x,\beta t,\alpha t)=\gamma e^{i\alpha t}\bar{g}(\beta t)
\]
appear for describing an external optical injection with optical frequency
$\alpha$ and modulation frequency $\beta$. In spatially extended
laser models those forces appear as inhomogeneities in the boundary
conditions. After homogenization of those boundary conditions and
finite dimensional mode approximations (or Galerkin schemes) one ends
up with systems of type (\ref{01}) with general forces of the type
$\gamma g(x,\beta t,\alpha t)$ with (\ref{eq:symg}). External forces
of the type \[
g(x,\beta t,\alpha t)=g_{1}(x,\beta t)\mbox{ with }g_{1}(e^{A\xi}x,\psi)=e^{A\xi}g_{1}(x,\psi)\]
 appear for describing an external electrical injection with modulation
frequency $\beta$.

(Predrag:
As far as I can tell, they do not slice: the \rpo\ is attractive, so they use its
co-moving frame, or `co-rotating coordinates'.)

                                                                \toCB
In the co-rotating coordinates $x(t)=e^{A\alpha_{0}t}y(\beta_{0}t)$
the unperturbed equation (\ref{02}) reads as
\begin{equation}
\beta_{0}\frac{dy}{d\psi}=\vel(y)-\alpha_{0}Ay.
\label{newcoord}\end{equation}
The quasiperiodic solution (\ref{qp}) to (\ref{02}) now appears
as $2\pi$-periodic solution $y(\psi)=x_{0}(\psi)$ to (\ref{newcoord}).
The corresponding variational equation
(Predrag:
linearization, $\vel'(x)$ is the {\stabmat} $\Mvar(x)$ in our notation.)
around this solution is
\begin{equation}
\beta_{0}\frac{dy}{d\psi}=\left(\vel'(x_{0}(\psi))-\alpha_{0}A\right)y.
\label{0per1}\end{equation}
It is easy to verify
% (see Section \ref{sec:Unperturbed-system}),
that (\ref{0per1}) has two linear independent (because of assumption
(\ref{eq:nondeg})) periodic solutions
                                                                \toCB
\begin{equation}
q_{1}(\psi):=Ax_{0}(\psi),\; q_{2}(\psi):=\frac{dx_{0}}{d\psi}(\psi),
\label{q}\end{equation}
 which correspond to the two trivial Floquet multipliers 1 of the
periodic solution $x_{0}$ to (\ref{newcoord}). One of these Floquet
multipliers appears because of the $S^{1}$-equivariance of (\ref{newcoord}),
and the other one because (\ref{newcoord}) is autonomous. From the
exponential orbital stability of (\ref{qp}) it follows that the trivial
Floquet multiplier 1 of the periodic solution $x_{0}$ to (\ref{newcoord})
has multiplicity two, and the absolute values of all other multipliers
are less than one. Therefore, there exist two solutions $p_{1}(\psi)$
and $p_{2}(\psi)$ to the adjoint variational equation
\begin{equation}
\beta_{0}\frac{dz}{d\psi}=-\left(f'(x_{0}(\psi))^{T}+\alpha_{0}A\right)z
\label{0per2}\end{equation}
 with \[
p_{j}^{T}(\psi)q_{k}(\psi)=\delta_{jk}\]
 for all $j,k=1,2$ and $\psi\in\mathbb{R}$, where $\delta_{jk}=1$
for $j=k$ and $\delta_{jk}=0$ otherwise.
(Predrag:
this is no doubt very important remark, but I have not digested it yet:
the left marginal vectors form an orthogonal subspace?)

(Predrag: They state a few mode-locking theorems that I am not interested
in at the moment)

One of the main statements of these theorem is that
there exist lower-dimensional {}``resonant'' manifolds $\mathfrak{N}_{j}$
corresponding to the frequency locking. These manifolds attract all
solutions from the neighborhood of $\mathcal{T}_{2}$. Hence, the
asymptotic behavior of solutions is described by (\ref{xm}) and has
the modulation frequency $\beta$.
The perturbed dynamics is, in the leading order, a motion along
$\mathcal{T}_{2}$ with the new modulation frequency.

(Predrag: In Sect. 4 they consider the linearized unperturbed system and introduce
an appropriate basis (matrix $\Phi_{0}$, constructed from \refeq{0per2}),
which locally splits the coordinates along the invariant torus
(\ref{eq:torus}) and transverse to it. Further, this basis is then used
for the introduction of appropriate local coordinate system. Perhaps slicing
 / moving frame symmetry reduction happens here: they say things like this:)

[...]
Therefore, if $n>3$, the $(n-2)$-dimensional
bundle $\mathfrak{Z}_{2}$ is trivial and there exists a smooth map
$\Phi_{0}:\mathbb{T}_{1}\to\mathcal{L}(\mathbb{R}^{n-2},\mathbb{R}^{n})$,
which is isomorphism between $\mathfrak{Z}_{2}$ and $\mathbb{T}_{1}\times\mathbb{R}^{n-2}.$
[...]

(Predrag: I give up here. but it might be worth understanding this construction)




\end{description}

\renewcommand{\ssp}{a}
