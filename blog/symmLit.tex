%from ChaosBook.org \Chapter{flows}{27aug2008}{Go with the flow}
% $Author: predrag $ $Date: 2009-08-23 11:03:24 -0400 (Sun, 23 Aug 2009) $

% \section{Literature survey}



\paragraph{Examples of systems with discrete symmetries:}
One has
a $\Ztwo$ symmetry in the Lorenz system (\refrem{rem:Lorenz}), % \rf{lorenz,GO},
the Ising model,
and in the 3\dmn\ anisotropic Kepler
potential\rf{gut82,TW,CC92},
a $D_3=\Zn{3v}$ symmetry in H\'enon-Heiles type potentials\rf{HH,JS,rich,laur},
a $D_4=\Zn{4v}$ symmetry in quartic oscillators\rf{EHP,MWR},
in the pure $x^2 y^2$ potential\rf{Mat,CP} and
in hydrogen in a magnetic field\rf{EW1},
and a $D_2=\Zn{2v} = V_4 =\Ztwo\times \Ztwo$ symmetry
in the stadium billiard\rf{robbDisc}.
A very nice application of desymmetrization is
carried out in \refref{BVdisc}.
\index{stadium billiard}\index{billiard!stadium}
\index{Ising model}

\paragraph{Nomenclature:}

I like a lot Ian Melbourne's work on
 \href{http://personal.maths.surrey.ac.uk/st/I.Melbourne/symmetric_attractors.html}
 {symmetric attractors}.
I find the distinction between instantaneous symmetry and symmetry
on average very meaningful and potentially useful for us (it will appear
in the thesis soon.)

{\bf PC} Nov 2008:
Do Fels and Olver\rf{FelsOlver98,FelsOlver99} really say
``moving coframes'' rather than ``comoving frames''?

{\bf ES}  Nov 2008: Yes, they really do. They refer to
Cartan's method as method of ``moving frames'' and they claim
it to be a special (and less rigorous) case of the moving
coframe method. I do not know Cartan's method and the two
papers of Fels and Olver\rf{FelsOlver98,FelsOlver99} are
lengthy and technical. Olver's book is readable but it
doesn't describe Cartan's method. I think they say ``moving''
rather than ``comoving'' frames because one only comoves in
the direction of group action.
Method of ``moving frames'' purpose is to generate
functionally independent invariants. ``Fundamental'' here means that they can be
used to generate all other invariants.
The name ``moving coframes'' arises through the use of
Maurer-Cartan form which is a coframe on the Lie group
$\Group$, in the sense that they form a pointwise basis for
the cotangent space.

\HREF{http://www.ams.org/notices/200503/fea-bloch.pdf}
     {Bloch \etal}\rf{BlMaZe05} say:
``A classical reference for the rolling disk is [German
sociologist]
\HREF{http://alo.uibk.ac.at/webinterface/library/ALO-BOOK_V01?objid=12421&zoom=6}
{Vierkandt}\rf{Vierkandt1892}, who showed
something very interesting: On an appropriate
symmetry-reduced space, namely, the constrained velocity
phase space modulo the action of the group of Euclidean
motions of the plane, all orbits of the system are
periodic.''

M. Rumberger and J. Scheurle,
\HREF{http://dynamics.mi.fu-berlin.de/danse/bookpapers/RumSch.ps.gz}
     {The orbit space method: theory and application}\rf{RumSch01}
has a clear and simple discussion of orbits spaces,
Hilbert bases etc. This paper studies linearizations of orbit space
trajectories obtained by mapping dynamics into Hilbert bases.
In Sect. 7 they apply this to Kuramoto-Sivashinsky.

\paragraph{A brief history of relativity:}\label{cont:rpoCond}

Lan has some relative equilibria (traveling waves) for KS in his
thesis\rf{Lan:Thesis}, %http://chaosbook.org/projects/theses.html
 and for complex LG in a paper on ``MAWs.''
Viswanath\rf{Visw07b} % arXiv.org/physics/0604062
found them in the plane Couette problem.
Siminos and Davidchack have for box size $L=22$ some \eqva.
Striking application of \rpo s has been the discovery
of ``choreographies" of $N$-body problems%
\rf{CheMon00,CGMS02,McCordMontaldi}.
