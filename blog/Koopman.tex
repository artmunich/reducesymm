% siminos/blog/Koopman.tex
% $Author$ $Date$

\chapter{Koopman operator}
\label{c-Koopman}
% merge into ChaosBook \section{Appendix: Implementing evolution}
% \label{c-appendMeasure}\noindent
% dasbuch/book/chapter/appendMeasure.tex

\renewcommand{\ssp}{x}
\renewcommand{\vel}{\ensuremath{v}}   % state space velocity
\newcommand{\Ubf}{{\bf\mathbf{U}}}
\newcommand{\Mbf}{{\bf\mathbf{M}}}
\newcommand{\Ibf}{{\bf\mathbf{I}}}
\newcommand{\fb}{{\bf f}}
\newcommand{\jb}{{\boldsymbol j}}
\newcommand{\df}[2]{\frac{\partial #1}{\partial #2}}


\begin{description}

\item[2011-10-24 PC] I have created this in order to prepare a
possible appendix or a new chapter for  ChaosBook.org.

\end{description}

\section{Koopman eigenvalues for a limit cycle}


    \PC{clipped from \refref{Bagheri13}}
An observable is a function that associates a scalar to each state $\ssp$ .
In this paper, we consider bounded and smooth (i.e. physical) observable
functions. The vector-valued observable $\obser(\ssp) : \pS \to \reals^m$ in
equation (2.1b) is defined as $\obser(\ssp) = [g_1(\ssp), g_2(\ssp), . . . , g_m(\ssp)]$,
where each component $g_j(\ssp)$ is a scalar function over $\pS$. The time
evolution of $\obser(\ssp)$ is governed by a convection equation (Lasota and
Mackey\rf{LM94})
    \PC{looks different from  my generator for {\FPoper}, chapter
    measure.tex}
\beq
\frac{\partial g}{\partial t}
= (\vel(\ssp) \cdot \nabla)g = \Aop g
\ee{Bagh(2.2)}
The above system is linear, time-invariant (since $\vel(\ssp)$ is
autonomous) and infinite dimensional. Moreover, it is a hyperbolic
system, with the physical interpretation of an observable as a `passive
tracer' transported by the vector-field $\vel(\ssp)$. We formally write the
solution to \refeq{Bagh(2.2)} as
\[
\obser(\ssp(\zeit)) = e^{\Aop \zeit}\obser(\xInit) = \Uop^\zeit \obser(\xInit),
\]
where the evolution operator $\Uop^t : S(\pS) \to S(\pS)$ is defined as the
Koopman operator\rf{K31,LM94}. In the theory of semigroups (Pazy 1983),
the linear operator $L$ is the infinitesimal generator of the operator
$\Uop^t$.

The Koopman operator $\Uop^t$, which is
adjoint to the Perron-Frobenious operator, is the appropriate evolution
operator to represent the dynamics of attractors and their stable
manifolds

The \monodromyM\  $\Mbf_{ij}=\partial_j \PoincM^{\cl{p}}_i(\sspRed_a)$
of dimension $(d-1)\times (d-1)$ governs the dynamics of the small
perturbation $\delta\sspRed$.

    \PC{clipped from Chapter {invariants.tex}}
Even though the \jacobianM\ $\Mbf(\ssp)$
depends upon $\ssp$ (the `starting' point of the periodic
orbit),
its eigenvalues do not, so we may write for its
eigen\-vectors $\jEigvec[j]$
(sometimes referred to as `covariant Lyapunov vectors,'
or, for \po s, as `Floquet vectors')
\index{covariant Lyapunov vector}\index{Lyapunov!covariant vector}
\index{Floquet!multiplier}
\beq
\Mbf(\ssp)\, \jEigvec[j](\ssp)
   = \ExpaEig_{j} \,\jEigvec[j] (\ssp)
\,,\qquad
\ExpaEig_{j}
= % \sign{p}^{(j)}
e^{\eigExp[j] \period{p} }
\,.
\ee{cplxExpaEig}
where Floquet exponents $\eigExp[j] = \eigRe[j] \pm i\eigIm[j]$
% and $\sign{p}^{(j)}$
are independent of $\ssp$.

If we order the eigenvalues of $\Mbf$
(Floquet multipliers) as
    \PC{I number complex pairs consecutively, so not $<$ but $\leq$}
%
\begin{eqnarray}
%&&|\ExpaEig_j| <1 \qquad {\rm for}\qquad j=1,\dots,d-1
&& |\ExpaEig_{1}| \geq |\ExpaEig_2|\geq \dots \geq|\ExpaEig_{d-1}|,
\label{eq:general:exponent}
\end{eqnarray}
%
then the limit cycle is stable if  $|\ExpaEig_{1}|<1$.


The
two most important characteristics of the limit cycle are  thus the
fundamental frequency and the \PCedit{leading} Lyapunov exponent,
defined  by
    \PC{ChaosBook reserves $\period{p}$ for `little trace' (over
        prime cycle $p$), $\period{p}$ for the period. Also changed
        $\sigma_j$ to $\ExpaEig_j$, $\sigma$ (that looks awkward in
        your paper) to \Lyap, $s$ to $\PoincM$, $L$ to $\Aop$, $\Ubf$
        to $\sspRed$. I'm a
        bit confused about $\Aop$ at this moment - that I use for the
        \evOper\ {\Lop} - maybe it should be $\transp{\Aop}$,
        whatever that means?. All of these are macros, and can be
        reverted back to your notation.}
\begin{equation}
\omega = \frac{\period{p}}{2\pi}, \qquad
\Lyap=\frac{1}{\period{p}}\ln|\ExpaEig_1|,
\label{eq:lyap:fund}
\end{equation}
%
respectively.


%\noindent{\bf Derivation of equation \refeq{eq:traceFormula}}\\
Here we follow the derivations  presented in
\refrefs{CBtrace,PG97}, % \cite{cvitanovic_ch18, gaspardBook},
    \PC{\refRef{PG97} follows \refref{pexp}, not sure it is
    any better than the original. Should recheck...}
except that the analysis is restricted to the simpler case of a
stable limit cycle. Using a Dirac-delta function, the Koopman
operator for any bounded observable function $\obser(\ub)$ can be written
as
%
\begin{equation}\nonumber
\Uop^t \obser(\ub) =
  \int_{\pS}\delta\left(\wb-\flow{t}{\ub}\right)\obser(\wb)d\wb
\,,
\end{equation}
%
where $\ub$ belongs to the manifold $\pS\subset\mathbb{R}^n$.
This form of the Koopman operator is similar to the form of integral
operators, for which one may define the trace as,
%
\[
\tr \Uop =\int_{\pS}\Uop(\ub,\ub)d\ub.
\]
%
where $\Uop$ is the kernel.
Inspired\footnote{
Integral operators are in $L^2$ space as well as compact, that is
they can in many aspects be treated as finite rank matrices. The
kernel of $\Uop^t$ is singular and the operator is certainly not in
$L^2$. However, if the dynamical system $\fb$ is real-analytic, it
has an analytic continuation to a complex extension of the state
space, where the singularity can be ``removed'' and the problem
reduced to the standard theory of integral operators\rf{CBconverg}.}
% \citep{cvitanovic_ch23}.}
by this definition, we define the trace of Koopman operator as
%
\begin{equation}
 \tr \Uop^t = \int_{\pS}\delta(\ub-\flow{t}{\ub})d\ub.
\label{eq:tr2}
\end{equation}
%
%
From \refeq{eq:tr2}, one observes that the trace of $\Uop^t$ counts
the number of times that trajectories return to their starting point
after $t$ time units.
    \PC{This does not `count'. See \wwwcb{/chapter/count.pdf}
    ``Topological zeta function for flows.''
    }
The trace of $\Uop^t$ in only non-zero if $\ub=\flow{t}{\ub}$.
These trajectories are precisely those on the limit cycle or  the
equilibrium.

    \PC{yet another name for the time-forward map: `propagator'}
To proceed, we decompose the propagator $f_t$ into two parts, the
Poincar\'e map $\PoincM$ and a scalar function $\tau$.  The Poincar\'e map
captures only part of the periodic dynamics, since the flow component
tangent to the limit cycle, which is not in the span of the
Poincar\'e surface, has not \PCedit{been} taken into account.
%The longitudinal component does not affect the stability, since exactly after one period $\period{p}$ the trajectory returns to its initial position, e.g.\  there is no decay or growth of perturbations along the tangential direction,  which would yield a Jacobian matrix that equals one.
%Although, this  longitudinal component does not affect the stability it is necessary for the description of the full dynamics, which means that it is significant for determining the Koopman eigenvalues.
Assuming the longitudinal state component has a certain mean velocity
$v$ as it traverses the limit cycle, one may transform this component
to a time coordinate system using the relation $v dt$.  Thus the full
dynamics is described by the Poincar\'e map $\PoincM$ and by some function
$\tau(\sspRed)$  that provides the (non-constant) time interval between
successive points $\sspRed$ on Poincar\'e surface, e.g. $t_{k+1}  = t_k
+ \tau(\sspRed_k).$
    \PC{`some function' = `first return function'. See \wwwcb{/chapter/maps.pdf}
    ``Poincar\'e sections.''
    }
%
%
Applying $\tau$ recursively, we may write ($k+1)th$ time as a
function first point and initial time,
%
\begin{eqnarray}
 t_{k+1}  = t _1+ \sum_{j=0}^{k-1}\tau(\PoincM^j\sspRed_1).
 \label{eq:poirec2}
\end{eqnarray}
%
%The fixed point  $\sspRed_a$ in \refeq{eq:fixedPoint} correspond to a
%point on the limit cycle with the period,
%%
%\begin{eqnarray}
%  \period{p}  &=& \sum_{j=1}^{k_p}\tau(\PoincM^j\sspRed_1).
%\end{eqnarray}
%

%
%In section \ref{sec:stability:periodic}, we determined the stability of a limit cycle in terms of the  eigenvalues of the linearized Poincar\'e map.  One can make use of that analysis to write the trace as a function of the Jacobian matrix $\Mbf$. However, this is only part of the dynamics, since the flow component tangent to the limit cycle, which is not in the span of the Poincar\'e surface, has not be taken into account.
%The longitudinal component does not affect the stability, since exactly after one period $\period{p}$ the trajectory returns to its initial position, e.g.\  there is no decay or growth of perturbations along the tangential direction,  which would yield a Jacobian matrix that equals one.  Nevertheless longitudinal direction is necessary for the description of the full dynamics, and hence it contributes to the trace of $\Uop^t$.
 %
Now, factor the kernel of $\Uop^t$ \refeq{eq:tr2} into two parts
%
\begin{equation}
\tr \Uop^t = \int_{\mathcal{P(\ub)}
=0}d\sspRed \int_0^{\tau(\sspRed)}\!\!\!\!dt\,
   \delta\!\left (\sspRed-\PoincM^{k}\sspRed\right )
   \delta\!\left (t-\sum_{j=0}^{k-1}\tau(\PoincM^j\sspRed)\right ),
\label{eq:app:kerfac}
\end{equation}
%
where $\PoincM^k$ and $\tau$ are defined above and in
\refeq{eq:poirec2}, respectively.  We treat the two Dirac delta
functions separately, starting with $\PoincM^k$. First recall that the
Dirac delta function applied to a scalar-valued function $g(x)$,
is
%
\[
\int \delta (g(x)) dx = \int \delta (x) |g'(0)^{-1}| dx =  \sum_j\frac{1}{|g'(x_j)|},
\]
%
where $x_j$ are the roots of $g(x)$. This property may be generalized
to $d\!-\!1$ dimensions and applied to the first Dirac-delta in
\refeq{eq:app:kerfac},
%
\begin{equation}
\int_{\mathcal{P}(\ub)=0}\!\!d\sspRed\,\delta(\sspRed-\PoincM^{k}(\sspRed))
 = \frac{1}{|\det (\Ibf- \Mbf^r)|},
\label{eq:app:det0}
\end{equation}
%
where $\Ibf$ denotes the identity matrix.   The second part of the
trace can be written as\rf{CBtrace} % \citep{cvitanovic_ch18}
%
\begin{equation}
\int_{0}^{\tau(\sspRed)}\delta (t-\sum_{j=0}^{k-1}\tau(\PoincM^j\sspRed))d t
   =\period{p} \sum_{r=1}^\infty\delta (t-r\period{p}).
\label{eq:app:fac2lim}
\end{equation}
%
%
Inserting the  identities \refeq{eq:app:det0} and
\refeq{eq:app:fac2lim} in \refeq{eq:app:kerfac}, we get the trace
formula for a single limit cycle,
    \PC{Shervin, here you used $T_p$, not $t_p$}
%
\begin{equation}
\tr \Uop^t = \period{p}\sum_{r=1}^\infty \frac{\delta (t-r\period{p})}{|\det (\Ibf- \Mbf^r)|},
\label{eq:traceFormula}
\end{equation}
%
which was first derived in \refref{pexp}, %{1991CvitanovicEckhardt},
here given in the special case of  a single limit cycle.
The trace formula is a sum whose terms are nonzero only for integers
of the cycle period.  The $r$th nonzero term describes how much
after the $r$th return to the Poincar\'e section a small neighborhood
volume (\ie\  a tube) of the stable limit cycle has retracted. This
relation thus connects the trace of $\Uop^t$ to the dynamics in the
local stable manifold of the limit cycle.
    \PC{to Shervin - changed to Poincar\'e throughout}


\textbf{Multi-index notation}.
Define multi-index as an array of $d$ integers:
    \PC{might want to move this much earlier in the ChaosBook}
\[
\jb = [j_{1}, j_{2}, \dots, j_{d}] \in \naturals^d
\,,
\]
\ie, elements are non-negative integers $j_k=0,1,2,\dots$. Then
\[
\jb! = j_{1}! j_{2}! \cdots j_{d}!
\,,
\qquad
|\jb| = j_{1}+j_{2}+\cdots+j_{d}
\,,
\]
and
\[
\partial^\jb = \df{^{j_{1}}}{x^{j_{1}}}\df{^{j_{2}}}{x^{j_{2}}}
            \dots\df{^{j_{d-1}}}{x^{j_{d-1}}}
\,.
\]


Consider the product of $d-1$ Floquet multipliers
\[
\ExpaEig
=  \ExpaEig_{1} \ExpaEig_{2}  \cdots \ExpaEig_{d-1}
= e^{ \period{p} (\eigRe[1]+\eigRe[2]+\cdots+\eigRe[d-1])}
\,,
\]
(the imaginary parts cancel in the product), and define
    \PC{ponder- complex pairs contribute here with multiplicity 2}
\[
{\boldsymbol \mu} = [\eigRe[1],\eigRe[2],\cdots,\eigRe[d-1]] \in \reals^d
\,.
\]
$\ExpaEig$ can now be raised to $\jb$th power as
\beq
\ExpaEig^\jb = e^{ \period{p} {\boldsymbol \mu}\cdot\jb}
             = \prod_{k=1}^{d-1} \ExpaEig_{k}^{j_{k}}
             =  \ExpaEig_{1}^{j_{1}} \ExpaEig_{2}^{j_{2}}
                \cdots  \ExpaEig_{d-1}^{j_{d-1}}
\,.
\ee{multIndLambda}

The Koopman eigenvalues are the poles of the
Laplace transform of trace of $\Uop^t$
%
\[
\int_0^\infty e^{-\eigenvL\zeit} \tr \Uop^t dt =\tr \frac{1}{s-\Aop}
\,,
\]
%
\ie, the poles of the
resolvent of $\Aop$. By inserting \refeq{eq:traceFormula} in the
left-hand side of above equation one obtains,
    \PC{I call it \Fd\ when it includes the weight of an observable. I guess
    this might be a `Zeta function', but who called it that first?}
%
\[
\tr \frac{1}{\eigenvL-\Aop} = \df{}{\eigenvL} \ln (Z(\eigenvL)),
\]
%
where $Z(\eigenvL)$ is the Zeta function,
%
\[
Z(\eigenvL) = \exp \left [-\sum_{r=1}^\infty\frac {1}{r} \frac{e^{-\eigenvL\period{p} r}}{|\det(\Ibf-\Mbf^r)|}
         \right ].
\]
%
Now, since the determinant does not depend on the basis which $\Mbf$
is described in, we may write it in terms of the eigenvalues of
$\Mbf$,
%
\begin{equation}
\frac{1}{|\det (\Ibf- \Mbf^r)|} = \prod_{k=1}^{d-1} \frac{1}{1-\ExpaEig_k^r},
\label{eq:prod2sum}
\end{equation}
%
where we have assumed that $|\ExpaEig_k|<1$ for all $k$. Note that
the Taylor series of the function $(1-x)^{-1}(1-y)^{-1} $ is
%
\[
(1-x)^{-1}(1-y)^{-1} = 1+x+y+ x^2 + xy + y^2+\dots,
\]
%
when $|x|<1,|y|<1$. Each term in the product \refeq{eq:prod2sum} may
thus be written as an infinite sum.
Using multi-index notation \refeq{multIndLambda} we may write
\refeq{eq:prod2sum} as
    \PC{Use the multi-index notation in ChaosBook}
    \PC{Shervin - Note that I have edited multi-index notation in
    half of your formulas - please recheck.}
%
\[
\frac{1}{|\det(\Ibf-\Mbf^r)|}
= \sum_{\jb} \ExpaEig^{r\jb}
\,,
\]
%
and consequently the Zeta function  as
%
\[
Z(\eigenvL) = \exp \left [-\sum_{r=1}^\infty\frac {1}{r}
  (e^{-\eigenvL\period{p}}  \sum_{\jb} \ExpaEig^\jb)^{r}
             \,\right].
\]
%
Finally,  applying the identity $\sum x^r/r= -\ln(1-x)$,  we obtain
the final form of the Zeta function for a stable limit cycle
%
%
\begin{equation}
Z(\eigenvL) 	= \prod_{\jb}^\infty
   \left [1-e^{-\eigenvL\period{p}} \ExpaEig^\jb\right]
\,.
\label{eq:app:zeta2}
\end{equation}
The zeros $Z(\eigenvL)=0$ are given by the zeros of individual terms in the
product, \ie
%
\[
e^{-\period{p}(\eigenvL- {\boldsymbol \mu}\cdot\jb )} = 1
\,.
\]
Taking the logarithm of both sides, we obtain
    \PC{this works for complex multipliers as well, as
    \(\ExpaEig_k \ExpaEig_{k+1} = \ExpaEig_k \ExpaEig_k^* = |\ExpaEig_k|^2.\)}
%
\begin{equation}
\eigenvL_{\jb,m}={\boldsymbol \mu}\cdot\jb + {2\pi i m}/{\period{p}}
\label{eq:app:tr:limit:final}
\end{equation}
%
with $m=0,\pm 1, \pm2, \dots $. For our
particular choice of analytic observables the spectrum of $\Uop^t$ is
reduced to its minimal components, namely any integer multiple of
the stability eigenvalues.


\section{Koopmanista blog}

\begin{description}

\item[2012-05-11 Predrag]                       \toCB
Koopman operators are so cool, that it is no wonder that
\HREF{http://www.engr.ucsb.edu/~mgroup/joomla/}{Igor Mezi\'c} is
so enamored with Koopmania\rf{LevnMezi08,RoMeBaSchHe09}:

{\em Analysis of dynamical systems using the
\HREF{http://www.scribd.com/mezicgroup/d/33974016-M-Budisic-Analysis-of-Dynamical-Systems-Using-the-Koopman-Operator-Formalism}
{Koopman operator formalism}}

{\em Spectral theory of nonlinear fluid flows based on the
\HREF{http://www.scribd.com/mezicgroup/d/75593768-I-Mezic-Spectral-Theory-of-Nonlinear-Fluid-Flows-Based-on-the-Koopman-Operator}
{Koopman operator}}

\item[2012-06-15 Predrag]
In {\em Applied Koopmanism} Budi{\v s}i{\'c}\etal\rf{BuMoMe12} say: ``
Methods from dynamical systems analysis, rely on Poincar\'e's geometric
picture that focuses on "dynamics of states". [...] This overview
presents an alternative framework for dynamical systems, based on the
"dynamics of observables" picture. The central object is the Koopman
operator: an infinite-dimensional, linear operator that is nonetheless
capable of capturing the full nonlinear dynamics. [...]  how methods that
appeared in different papers and contexts all relate to each other
through spectral properties of the Koopman operator.  [...] three main
concepts: Koopman mode analysis, Koopman eigenfunctions and Fourier
quotients. We also discuss continuous indicators of ergodicity and
mixing.
''

\item[2012-08-03 Igor]
\emph{Applied Koopmanism} by Marko Budi\v{s}i\'c, Ryan M. Mohr and Igor
Mezi\'c, \arXiv{1206.3164} is our latest that attempts to review what we
have done with Koopmania in the past and reviews and gives references for
negative Sobolev.

\item[2012-09-01 PC]							\toCB
Bagheri paper merits a closer read - a copy of the paper is
\HREF{http://ChaosBook.org/library/Bagheri12.pdf}{here}
(2013-02-25: superseded by \refref{Bagheri13}, find it
\HREF{http://ChaosBook.org/library/Bagheri13.pdf}{here}). He writes:
``
[...] we characterize the behavior of Koopman modes and
eigenvalues for flows developing and undergoing oscillations. In
particular, we focus on the saturation dynamics near the critical
threshold for oscillation, where the amplitude is governed by the
Stuart-Landau equation. The Koopman-mode theory is also valid far from
bifurcation, although it is not explicitly treated in this work. We will
assess a numerical algorithm referred to as Dynamic Mode Decomposition
(DMD), in terms of its ability to approximate Koopman modes for transient
and asymptotic dynamics. The algorithm is scalable to very
high-dimensional systems and relies only on post-processing collected
data.
''

Discusses only systems where one
unstable focus $f (u_s) = 0$ co-exists with a stable limit cycle.

``Prior to Mezi\'c and Banaszuk (2004) and Mezi\'c (2005), essentially
all applied literature on evolution operators have focused on the
Perron-Frobenious operator. The Koopman operator $\Uop^t$
governs the evolution of scalar-valued
observables, such as the kinetic energy and probe signals, in some
functional space. Our essential ingredients for describing the flow
dynamics are the complex eigenfunctions and eigenvalues of $\Uop^t$. The
concept is best introduced by an example.''

He then gives explicit
eigenvalues and eigenfunctions for
\beq
\dot{x}(t) = \mu x - x^3 \,,\qquad \mu  < 0
\,,
\ee{Bagh(1.1)}
as well as a 3\dmn\ flow of Noack \etal.
    \PC{This would be good problem set for ChaosBook?}

\item[2012-02-20  Shervin Bagheri]
shervin.bagheri@mech.kth.se writes: `` I recently wrote a paper on
the spectral decomposition of the Koopman operator for the case of
single stable limit cycle (applied to the von Karman vortex street
behind a circular cylinder). Part of my work have been strongly
influenced by the papers you wrote on the trace formula for flows.  I
have attached this paper, hoping you find an interesting hydrodynamic
application of the theory you have developed in the past.''

\item[2013-01-20 Predrag] Dear Shervin, thanks for sending me this paper
- I already have your \refref{RoMeBaSchHe09} on my 'to read' list,
but I have not studied it yet. Currently I have relegated Koopmanism
to an appendix, but it surely should be brought back into the main
text, so I'm very much looking forward to learning about all things
Koopman and more from you.

\wwwcb{} is supposed to cover all things chaotic and nonlinear that
an advanced grad student should know or at least have heard about, so
let's take a proactive stance: if you think there are formulas,
derivations, etc. in your papers I should read and include, please
tell me, and I'll do my schoolwork.

BTW, as you are not referring to anything specific in Version 12, can
you change the (Cvitanovi\'c et al. 2008) citation to year 2013, no
version? And similarly for the referenced chapters, to Version 14?
From edition to edition ChaosBook changes substantively, but your
references to chapter names have not changed between these versions.

Recently there are more and more papers which reproduce our work from
twenty years ago without a single citation, or (for example,
wikipedia entries) that misassign credit for almost every entry. I
did not care about it before (hey, I actually get paid for doing what
I love!), but I'm starting to care, mostly because I want people to
cite ChaosBook.org - it is meant to be more useful than the original
papers. So I turned into a kvetch, and wrote a very myopic history of
\wwwcb{/chapters/appendHist.pdf} {``Periodic orbit theory''} plus a
sarcastic \wwwcb{/chapters/recycle.pdf} {``Alternative Periodic Orbit
Theories''} remark. If you read them, and find something wrong in
them, please reign me in and let me know what to fix.

\item[2013-01-25 Predrag] Shervin, I have pondered using a stable
fixed point, and then the stable limit cycle as the first step in
motivating \Fd s. I do that for noisy flows in
\refrefs{LipCvi08,CviLip12} (not migrated to ChaosBook yet). It is
very intuitive and instructive for the Ornstein-Uhlenbeck system,
with full set of Hermite-polynomial eigenfunctions, but the problem
here is (is it a problem?) that for stable fixed points the measure
is a Dirac delta on the fixed point / limit cycle. Set of (right?)
eigenfunctions is given in \refref{CBconverg} example ``The simplest
eigenspectrum - a single fixed point,'' but the left eigenfunctions
are buried later in the text as derivatives of delta functions, so
this is not very helpful, pedagogically. What to do?

So here, experimentally, I've edited (in ChaosBook.org notation) and
commented clips from your manuscript\rf{Bagheri13}. If it makes
sense, we might experimentally migrate this to Koopmania of
\wwwcb{/chapters/appendMeasure.pdf}. For long time I've felt guilty
to have exiled Koopman to an appendix, so maybe the whole thing might
return to the text proper. But that makes the book yet longer...
Uurgh.



\noindent{\bf Koopman eigenvalues for the cylinder flow}\\
%%
%\begin{figure}
%\begin{center}
%\includegraphics*[width=.46\textwidth]{\figdir floquet_sec2}\hfill
%\includegraphics*[width=.46\textwidth]{\figdir spec_cyl_sec2}
%\end{center}
%\begin{picture}(0,0)
%\put(-10,94){$(a)$}
%\put(194,94){$(b)$}
%\put(288,2){Im$(\lambda)$}
%\put(-25,55){Re$(\lambda)$}
%\put(88,2){Im$(\lambda)$}
%\end{picture}
%\caption{Left figure shows Floquet multipliers (left) of the cylinder flow at $Re=50$. Right figure shows the Koopman eigenvalues associated with Floquet multiplier marked with red symbol. }
%\label{fig:floquet}
%\end{figure}
%
Consider the cylinder flow at  $Re_{c,3D}>Re>Re_c$, where
$Re_{c,3D}=182$ is the critical Reynolds number for which the limit
cycle becomes unstable. Suppose that the frequency and the Lyapunov
exponent of limit cycle are $\omega$ and $\Lyap$ respectively.
%
Then, the Koopman eigenvalues $\eigenvL_{j_1,0,\dots,0,m}$ obtained
from  \refeq{eq:app:tr:limit:final} corresponding to this stability
eigenvalue are
    \PC{fix this}
%
\begin{equation}
\eigenvL_{j,m}  =  j \Lyap + im\omega, %= -j0.03 + i m 0.79,
\label{eq:tracePeriod}
\end{equation}
%
with $j=0,1,2\dots$ and $m=0,\pm1,\pm 2,\dots$.
%
Thus, for any stable limit cycle, the Koopman eigenvalues form a
lattice on the lower half of the complex plane. The marginal
eigenvalues on the horizontal imaginary axis corresponding to $j=0$
correspond to the non-decaying time-averaged mean ($m=0)$ and
periodic dynamics ($m\neq 0)$ on the limit cycle.
    \PC{I like `periodic dynamics' - it is Fourier decomposition of
    the initial density, which can rotate but cannot change, along
    the cycle in components $m$. Do I understand you right?}
The remaining
eigenvalues $j\neq 0$ are decaying and describe the transient
behavior of flow in the local stable manifold of the limit cycle.

\item[2013-01-27 Igor Mezi\'c]
    Attached are some new contributions for your enjoyment!
Chaos\_22\_047510.pdf
8422K ; LanMezicLinearization.pdf
480K; MauroyMezicIsochron.pdf
2034K; annurev-fluid-011212-140652.pdf 838K.

How are things?

\item[2013-01-27 Predrag]
Mein Gott!  a terabyte of Koopmanistas at work! I'll chew on this
too... Things were manageable until  you sent me four more papers to
read. I seem to be only person left who actually reads articles. So
19th century...

we plod on

\end{description}

\renewcommand{\vel}{F}      % state space velocity vector
