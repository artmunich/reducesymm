% siminos/blog/excite.tex
% $Author$ $Date$

\chapter{Excitable media}
\label{c-excite}

\renewcommand{\LieEl}{\ensuremath{g}}  % Predrag Lie group element
\renewcommand{\gSpace}{\ensuremath{\theta}}   % group rotation parameters
\renewcommand{\ssp}{x}

\begin{description}

\item[2012-01-04 PC] moved siminos/blog/excite.tex to Luis blog,
DOGS/saldana/excite.tex, as it contains Predrag's notes relevant to Luis' project

\item[2013-04-20 PC] moved back from Luis blog (Saldana never got that far).

\item[2009-12-06 Predrag]
    \PC{2011-10-05 experimental: $\hat{\ssp}$ to indicate a \reducedsp\ coordinate}
[2010-01-07 incorporated most of this history into ChaosBook]

\item[2009-12-10 Barkley spirals reduction]
V. N. Biktashev, A. V. Holden, and E. V. Nikolaev\rf{BiHoNi96},
``Spiral wave meander and symmetry of the plane.''
They say
``
We present a group-theoretic approach based on the
{\em well-known space reduction method}, used to separate the motions in the
system into superposition of those `along' orbits of the
Euclidean symmetry group, and `across' the group orbits. It can
be interpreted as passing to a reference frame attached to the
spiral wave's tip. The system of ODEs governing the tip
movement describes the movements along the group orbits. The
motions across the group orbits are described by a PDE which
\emph{lacks the Euclidean symmetry}. Consequences of the Euclidean
symmetry on the spiral wave dynamics are discussed. We derive
the model system for bifurcation from rigid to biperiodic
rotation [Barkley 1994] from a priori symmetry considerations.
''


Barkley, Kness, and Tuckerman\rf{BaKnTu90},
``Spiral wave dynamics in a simple model of excitable
		 media: Transition from simple to compound rotation.''

From Barkley\rf{Barkley94}:

The noncompact Euclidean symmetry group $E_2$ (or $E(2)$) is the group of
distance-preserving symmetries in the plane: translations, rotations and
reflections.
(See als Beyn\rf{BeTh04} eq.~(2.12).)
                                    \toCB

\emph{Rotating waves} are steady
in a frame rotating with frequency $\omega$. In fluid flows with
axial symmetry they are non-axisymmetric flow patterns that
rotate as a rigid body with a uniform angular velocity\rf{Rand82}.
                                    \toCB

\emph{Modulated rotating waves} are quasiperiodic states characterized
by two frequencies $(\omega,\omega')$ which are
periodic in a frame rotating with frequency $\omega$. They are
have\rf{Rand82}
a period $\period{}$ such that the flow patterns
at times $t$ and $t+\period{}$ differ by a rotation by $2\pi n/m$,
$0 \leq n < m/s$,
where $m$ is the number of wave peaks, $s$ the order of
axisymmetry of the wave pattern, $\omega$ the average angular
velocity of the wave, and
                                    \toCB
\beq
\Omega_M= 2\pi /\period{}
	\,,\qquad
\Omega_W= s(\omega+n \Omega_M)/m
\ee{MRW}
are fundamental frequencies, \ie, the temporal power spectrum
is of form $a\Omega_M+b \Omega_W$ where $a$ and $b$ are integers.

\emph{Modulated traveling wave}s are periodic states characterized
by frequency $\omega$ in a uniformly translating frame.

Barkley constructed a 5-dimensional model of meandering spiral dynamics
(later derived from symmetry-reduced bifurcation theory in
\refref{BiHoNi96}) that describes codimension-two bifurcation, with Hopf
eigenmodes interact with the Euclidean eigenmodes.

\item[2010-06-12 PC]                                            \toCB
One has to be in the co-rotating frame for eigenvalue to be marginal, and
the velocity be the corresponding right eigenvector. That is even clearer
when you start computing stability of \rpo s. See whether there is
anything smart to be said for the corresponding left eigenvectors - in
the theory of rotating spirals they call them ``Response Functions'' (and
capitalize them), while the right group tangent vectors they call
``Goldstone modes'' and they make a big deal out of them (I hope to blog
my reading report into the Siminos blog soon). I still need to write this
up in the ChaosBook, sect. 10.3 ``Stability,'' would like to see how you
think about it.


\item[2009-12-23 Evangelos]
Biktashev, Holden, and Nikolaev\rf{BiHoNi96} assume that group
orbits foliate state space which I think is equivalent to a
free and regular group action, that is they have the same
restrictions as we do.  I found the Anosov and Arnol'd\rf{AnAr88} reference
they cite for their symmetry reduction method. It mentions that
one can locally, in the vicinity of a point which is not fixed by
the group can reduce the order of a system of differential equations
by the dimension of the group. I do not have Arnold's \emph{Geometrical Methods
in Classical Mechanics} book here, but I think it has a more extended
discussion of symmetries.

\item[2009-12-11 Predrag] I have added Evangelos PDF of Biktashev,
Holden, and Nikolaev\rf{BiHoNi96} to \wwwcb{/library}: student Lautrup.

\item[2009-12-12 Predrag] Need to check Lahme and Miranda\rf{LaMi99}
``Karhunen-Loeve decomposition in the presence of symmetry''. They say
``
The Karhunen-Loeve (KL) decomposition is
  widely used for data which often exhibit some symmetry.
  For a finite group, we derive an
  algorithm using group representation theory to reduce the
  cost of determining the KL basis. We demonstrate the
  technique on a Lorenz-type ODE system. For a compact group
  such as tori or SO(3,R) the method also applies: we
  consider the circle group $S^1$.
''
                                    \toCB


\item[2009-12-16 Erik Martens] % <erik.martens@ds.mpg.de>
Have a look at
Ed Ott and Tom Antonsen's reduction of $\infty$-dim Kuramoto phase-oscillator
problem to a $n$-dimensional problem:

1) proof that a low-dimensional manifold for the order parameter exists:
\\
\HREF{link.aip.org/link/?CHAOEH/18/037113/1}
{http://link.aip.org/link/?CHAOEH/18/037113/1}

2) this manifold is attractor for the order parameter:
\\
\HREF{http://link.aip.org/link/?CHAOEH/19/023117/1}
{http://link.aip.org/link/?CHAOEH/19/023117/1}

\item[2009-12-17 Predrag]
The story on equivariance seems to start with M. Field\rf{Field70}, and
on symmetries in presence of bifurcations with Ruelle\rf{ruell73}. Ruelle
is a bit hard to read. I believe that he proves that the
\stabmat/\jacobianM\ evaluated at an \eqv/fixed point $\ssp \in \pS_G$
decomposes into linear irreducible represenations of \Group, and that
stable/unstable manifold continuations of its eigenvectors inherit their
symmetry properties. He finds it remarkable that corresponding
bifurcations can go from an \eqv\ to a rotationally invariant periodic
orbit (\ie, \reqv), and not to other points.
	                                                           \toCB

Spiral wave formation in nonlinear excitable reaction-diffusion media was
first observed in 1970 by Zaikin and Zhabotinsky\rf{ZaZha70}.
Winfree\rf{Winfree73,Winfree1980} noted that spiral tips execute
meandering motions.
% others tend to cite Winfree 1980 monograph\rf{Winfree1980}.
Barkley and collaborators\rf{BaKnTu90,Barkley94} showed that the
noncompact Euclidean symmetry of this class of systems precludes
nonlinear entrainment of translational and rotational drifts and leads to
observed quasiperiodic and meandering behaviors.

Skyped Dwight.
Dwight says:

Zaikin, Winifree were indeed the first spiralers.

Dwight 1994 knew about symmetry reduction but did not
include it. He uses it in principle always, but by ``post-processing''
- simulates as usual, but then post-processes data into the slice.
Just as we recommend.

Biktashev, Holden, and Nikolaev\rf{BiHoNi96}
skirts around the issue - they wrote it
 but they did implement it.
%, so Dwight did not want his name on the paper.

\HREF{http://dynamics.mi.fu-berlin.de/persons/fiedler.php}
{Fiedler} would justly claim he found it first (influential talk
at Newton Inst).

Fiedler \etal\rf{FiSaScWu96} however did nothing numerical.

Beyn is the first to do nontrivial freezing in numerical
application.

Kevrekidis added scaling.

Biktashev is currently implementing Beyn, see {\bf 2010-03-30}
blog entry.

\item[2009-12-19 Predrag]
Reading Fiedler \etal\rf{FiSaScWu96}. They say:

$\Group \cdot \ \ssp_0$ is dipheomorphic in $\Group/H$,
where
\beq
H := \{ h \in \Group | h \,\ssp_0 = \ssp_0 \}
\,.
\ee{FieldIsotr}
A \Group-equivariant flow in a tubular neighborhood of a
\reqv\ $\Group \cdot \,\ssp_0, \ssp_0 \in \pS$,
with compact isotropy $H$ of $\ssp_0$, can be represented
by a {\em skew product} flow
\beq
\dot{\LieEl} = \LieEl \, \bf{a}(v)
	\,,\qquad
\dot{v} = \phi(v)
\,.
\ee{FiedlSkPr}
$\bf{a}$ is in the Lie algebra of \Group, and $v$ is in a linear
slice \pSRed, transverse to the group action. The slice  \pSRed\
is called a {\em Palais slice}\rf{Pal61}, and $(\LieEl,v)$ are Palais
coordinates near \reqv. As the choice of the Palais slice is
arbitrary, these coordinates are not unique. The skew product flow
\refeq{FiedlSkPr}  was first
written down by Mielke\rf{Mielke91}, in the context of buckling
in the elasticity theory.
        \index{Palais slice}\index{slice!Palais}

Reading Fiedler and Turaev\rf{FiTu98}.
Fetch it at
\\
\HREF{http://dynamics.mi.fu-berlin.de/preprints/BF_Normal.pdf}
{Normal forms, resonances, and meandering tip motions}
     near relative equilibria of {Euclidean} group actions.
They have a nice discussion of the semidirect product structure of the
Euclidean group $E(2)$.


\item[2010-03-30 Predrag]
We need to read
A. J. Foulkes and V. N. Biktashev\rf{FouBik10},
``Riding a Spiral Wave: Numerical Simulation of Spiral Waves
in a Co-Moving Frame of Reference.''
They describe an approach to numerical simulation of spiral
waves dynamics of large spatial extent, using small
computational grids.
``
An idea of dynamics in the space of symmetry group
orbits [31], when applied to a reaction-diffusion system
of equations and the Euclidean symmetry group, leads to
a description which is formally equivalent to considering
the solution in a moving frame of reference (FoR) such
that the spiral wave maintains a certain position and orientation
in this frame [29]. We call it comoving FoR.
We calculate
the dynamics of the spiral wave in a comoving FoR;
as a result, the core of the spiral never approaches the
boundaries of the computation box, which allows computations
of drift and meandering of large spatial extent using
small numerical grids. EZRide, a simple software implementation
of this approach, which is based on the popular spiral
wave simulator EZ-SPIRAL is provided on the authors' website.
Our approach can be compared to the approach proposed
by Beyn and V. Th\"ummler\rf{BeTh04} and further developed
by Hermann and Gottwald\rf{HerGot10}. Their approach also exploits
symmetry group orbits, but is different in some
essential details. We shall discuss the similarities and
differences when we will have introduced our method.
''

\item[2010-03-30 Predrag]
We need to read
Foulkes, Barkley, Biktashev and Biktasheva\rf{FoBaBiBi10},
``Alternative stable scroll waves and conversion of autowave turbulence.''
For a copy, \HREF{http://chaosbook.org/library/FoBaBiBi10.pdf}
{click here}. You will need to type \texttt{student} and then \texttt{Lautrup} .


\item[2010-03-31 Predrag]
We need to read
\HREF{http://www.maths.usyd.edu.au/u/gottwald/papers.html}
    {Hermann and Gottwald}\rf{HerGot10}. They say:
``
We modify the freezing method introduced by Beyn and
Thuemmler, 2004, for analyzing rigidly rotating spiral waves
in excitable media. The proposed method is designed to stably
determine the rotation frequency and the core radius of
rotating spirals, as well as the approximate shape of spiral
waves in unbounded domains. In particular, we introduce
spiral wave boundary conditions based on geometric
approximations of spiral wave solutions by Archimedean
spirals and by involutes of circles. We further propose a
simple implementation of boundary conditions for the case
when the inhibitor is non-diffusive, a case which had
previously caused spurious oscillations.
We then utilize the method to numerically analyze the large
core limit. The proposed method allows us to investigate the
case close to criticality where spiral waves acquire infinite
core radius and zero rotation frequency, before they begin to
develop into retracting fingers. We confirm the linear
scaling regime of a drift bifurcation for the rotation
frequency and the core radius of spiral wave solutions close
to criticality. This regime is unattainable with conventional
numerical methods.
''


\item[2010-06-05, 2011-04-20 Predrag ] Lord and Vera
Th\"ummler\rf{LoTh10}
minimize the $L^2$ norm against a
`fixed profile' or `reference function:'
``results include our new idea of comparing the wave to a
reference function.'' Perhaps new in context of stochastic
PDEs...
They refer to
(deterministic) traveling wave as `Goldstein mode.'
												\toCB
They say in \HREF{http://arxiv.org/abs/1006.0428},
{\em Freezing stochastic traveling waves}:
``
 We consider traveling wave solutions to stochastic
PDEs and corresponding wave speed. As a particular example we
consider the Nagumo equation with multiplicative noise which we mainly consider
in the Stratonovich sense. A standard approach to determine the position and
hence speed of a wave is to compute the evolution of a level set. We compare
this approach against an alternative where the wave position is found by
minimizing the $L^2$ norm against a fixed profile. This approach can also be
used to stop (or freeze) the wave and obtain a stochastic partial differential
algebraic equation that we then discretize and solve. Although attractive as it
leads to a smaller domain size it can be numerically unstable due to large
convection terms. We compare numerically the different approaches for
estimating the wave speed. Minimization against a fixed profile works well
provided the support of the reference function is not too narrow. We then use
these techniques to investigate the effect of both Ito and Stratonovich noise
on the Nagumo equation as correlation length and noise intensity increases.
''

\item[2012-02-25 PC]
Ashwin and Melbourne\rf{AshMe97,AshMeNi99} write:``
In the context of equivariant dynamical systems with a compact Lie group,
Field and Krupa have given sharp upper bounds on the
drifts associated with relative equilibria and relative periodic orbits.
Generically, these are maximal tori.
In this paper, we generalize the results to noncompact
Lie groups. The drifts now correspond to tori or lines. Which of these
drifts is preferred, compact or unbounded, depends: there are
examples where compact drift is preferred (Euclidean group in the plane),
where unbounded drift is preferred (Euclidean group in three-dimensional
space) and where neither is preferred (Lorentz group).

Our results partially explain the quasiperiodic (Winfree) and linear
(Barkley) meandering of spirals in the plane, as well as the drifting
behaviour of spiral bound pairs (Ermakova et al). In addition, we obtain
predictions for the drifting of the scroll solutions (scroll waves and
scroll rings, twisted and linked) considered by Winfree and Strogatz.
''

\item[2012-02-25 PC]
Chossat\rf{Choss02},
{\em The reduction of equivariant dynamics to the
        orbit space for compact group actions}
might be worth a read:

``
Symmetry introduces degeneracies in dynamical systems, as well as in
bifurcation problems. An `obvious' idea is to project the dynamics onto
the quotient space obtained by identifying points in phase space which
lie in the same group orbits (the so-called orbit space). Several
difficulties arise. First, the orbit space is not, in general a manifold.
Second, how does one explicitly realize the orbit space, and how does one
compute and analyze the projected dynamics? In this paper I will describe
the methods which have been developed in order to answer these questions,
and I will show on three examples how they apply. We shall see that,
although not always suitable to treat equivariant dynamics, these methods
sometimes lead to insightful reductions.
''

\end{description}



\renewcommand{\ssp}{a}
