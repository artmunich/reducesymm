% siminos/blog/excite.tex
% $Author$ $Date$

\chapter{Excitable media}
\label{c-excite}

\renewcommand{\LieEl}{\ensuremath{g}}  % Predrag Lie group element
\renewcommand{\gSpace}{\ensuremath{\theta}}   % group rotation parameters
\renewcommand{\ssp}{x}

\begin{description}

\item[2009-12-06 Predrag]
    \PC{2011-10-05 experimental: $\hat{\ssp}$}
[2010-01-07 incorporated most of this history into ChaosBook]
Went to Bielefeld, talked to
\HREF{http://www.math.uni-bielefeld.de/~beyn}{Wolf-J\"urgen Beyn}.
Beyn and his former PhD student Vera
Th\"ummler\rf{BeTh04,Thum05} `freeze' traveling waves. Their
papers are available under `Preprints' link on Beyn website.
They are mathematical, proving existence of spectra etc., but
also applied to numerical examples. Scalar Nagumo equation is a
good testing ground, as one of the traveling wave solutions is
known analytically.

Beyn did his `freezing' at the same time as Rowley and
Marsden\rf{rowley_reduction_2003}, but they published first (a
year earlier?), so \refref{BeTh04} cites their `reconstruction
equation.' Beyn does not understand the `method of
connections,' either.

Beyn group\rf{BeTh04}  research focuses on the numerical
computation and stability of traveling wave solutions (and more
generally relative equilibria) of reaction-diffusion systems on
unbounded domains, \ie, of parabolic partial differential
equations (PDEs). On the real line they are of the form
\beq
u_t = A \, u_{xx} + f(u,u_x)
	\,,\quad
u: \reals \times [0,\infty) \to \reals
	\,,
A \in \reals^{d,d}
\,.
\ee{parabolicPDE}
A traveling wave is a special type of
a relative equilibrium of equivariant evolution equations,
where the action is given by translation,
   % \PC{copied this to ChaosBook}
\beq
\LieEl(\velRel) \, u(x)
  = \hat{u}(x - \velRel t)\,,\quad  \velRel \in \reals^d
\,,
\ee{BeThTW}
$\hat{u}(x)$ is the waveform and $\velRel$ the velocity. In the
comoving frame equations \refeq{parabolicPDE} take form
\beq
\hat{u}_t
 = A\, \hat{u}_{xx} + \velRel \, \hat{u}_x + f(\hat{u},\hat{u}_x)
	\,,\quad
\hat{u}_x \in \reals
	\,,\;
t \ge 0
\,,
\ee{comovingPDE}
with $\hat{u}(x)$ a stationary solution (the `freezing of the wave')
\beq
0= A \, \hat{u}{''} + \velRel \, \hat{u}{'} + f(\hat{u},\hat{u}{'})
\ee{BeThREQV}
The pair $(\hat{u},\velRel$) is an equilibrium solution of the
partial differential algebraic equation (PDAE)
\refeq{BeThREQV} which is constructed by inserting the co-moving
frame ansatz into PDE and adding an {\em additional phase
condition}. They say: ``By transforming the PDE into a
corresponding PDAE (partial differential algebraic equation)
via a freezing ansatz\rf{BeTh04} the relative equilibrium can
be analyzed as a stationary solution of the PDAE.''

Beyn finds Sandstede \etal\
articles\rf{FiSaScWu96,SaScWu97,SaScWu99a,FiTu98}
`remarkable.'
Sandstede \etal\rf{SaScWu97} use center manifold reduction
theory to study stability of \reqva. The main idea is to
transform the flow into a `skew product' form. One part is
orthogonal to the group orbit of the initial field $u(0)$, and
the other part acts within the group orbit and depends upon the
position in the orthogonal direction.

{\bf ES 2009-12-23} This decomposition is important in Krupa\rf{Krupa90} study
of bifurcations of relative equilibria, using a local slice. I think
he also introduces the tubular neighborhood. The book by Chossat
and Lauterbach\rf{ChossLaut00} gives a readable presentation
and I need it to compare their stability analysis for \reqva\ with
our Appendix, but I can't find it here.

%%%%%%%%%%%%%%%%%%%%%%%%%%%%%%%%%%%%%%%%%%%%%%%%%%
\SFIG{BeThEquiTraj}
{}{
Two trajectories
$\ssp(t)$, $\sspRed(t)$ are equivalent up to a group rotation
$\LieEl(t)$ as long as they belong to the same group orbit
$\pS_{\ssp(t)}$.
}
{fig:BeThEquiTraj}
%%%%%%%%%%%%%%%%%%%%%%%%%%%%%%%%%%%%%%%%%%%%%%%%%%
%
Beyn~\etal\rf{BeTh04} find it more convenient to make
use of the equivariance by extending the system rather than
reducing it as in bifurcation analysis, by adding an additional
parameter and a phase condition. They too write the solution as
${u}(t) = \LieEl(t)\,\hat{u}(t)$, see
\reffig{fig:BeThEquiTraj}, a composition of the action of a
time dependent group element $\LieEl(t)$ with a `frozen
solution' $\hat{u}(t)$ in a given Banach space, with
$\LieEl(t)$, $\hat{u}(t)$ to be determined. They keep the
`frozen solution' as constant as
possible%
\PC{`As constant as
possible' makes sense only for \reqva, not \rpo s which they do
not discuss.} by introducing a set of algebraic constraints
(phase conditions), $\psi(\hat{u}, \gSpace) = 0$, which fix the
extra degrees of freedom. Their number is given by the
dimension of the Lie group.


The freezing approach\rf{BeTh04} applies to traveling
waves and, more generally, to \reqva\rf{ChossLaut00,SaScWu99},
solutions which are
equilibria in an appropriately co-moving frame. They occur
in systems with underlying symmetry, such as
rotating waves on the real line and spiral waves in two space
dimensions. An example of a system with `rotating waves on the
real line' is the complex Ginzburg Landau equation, while
spirals are typical of the reaction-diffusion systems.

Consider an evolution equation in a Banach space $\pS$  of the form
\beq
u_t = \vel(u)
\ee{BeThEvEq}
with an equivariant right hand side $\vel$, i.e.
$\LieEl\,\vel(u) = \vel(\LieEl\,u)$ where $\LieEl : \Group \to
GL(\pS), \gSpace \to \LieEl(\gSpace)$ denotes the action of a
Lie group $\Group$ on  $\pS$. Here --in the parlance of applied
mathematicians-- $\vel$ is defined on a dense subspace of some
Banach space (generally infinite dimensional) and is
equivariant with respect to the action of a finite dimensional
(not necessarily compact) Lie group. In other words, the
$\infty$-dimensional functional PDE \statesp\ is spanned by
nicely shaped waveforms $u(x)$, nothing too kinky.

The equation \refeq{BeThEvEq}
can be transformed via the ansatz
$u(t) = \LieEl(t)\hat{u}(t)$ into the equivalent system
\beq
\hat{u}_t = \vel(\hat{u})
  - \LieEl(\gSpace)^{-1} \LieEl_\gSpace(\gSpace) \, \hat{u} \lambda
	\,,\qquad
\lambda = \gSpace_t
\,,
\ee{BeThReconstr}
where subscripted quantities imply partial derivatives with
respect to the subscript. Beyn~\etal\rf{BeTh04} then `freeze'
the traveling wave by fixing a \slice, and using a dot product
w.r.t. the group tangent at the slice point. The evolution of
$\gSpace(t)$ describes the motion on the group manifold. They
denote by $T_\gSpace \Group$ the tangent space of $\Group$ at
$\gSpace$. Introducing Lagrange parameters $\lambda(t) =
\LieEl_t(t) \in T_\gSpace \Group$ they impose `freezing ansatz'
on the extended system of equations \refeq{BeThReconstr}
\beq
\hat{u}_t = \vel(\hat{u})
  - \LieEl(\gSpace)^{-1} \LieEl_\gSpace(\gSpace) \, \hat{u} \lambda
	\,,\qquad
0 = \pi(\hat{u},\lambda)
\,,
\ee{BeThSlice}
with a ``phase condition
$\pi : \pSRed \times T_\gSpace \Group \to \reals^N$,
$N= \mbox{dim }\Group$,
which has to satisfy some regularity conditions.''
They differentiate
$
h \circ \LieEl \, \hat{u}
$
with respect to $\LieEl$, define group tangent at unity
$\groupTan \in \times T_1 \Group$, $\lambda = d\gSpace_l(1)
\groupTan$ ($l$ for left multiplication), and with some algebra
I'm too bored to type arrive at the extra equation(s)
\[
 0= \psi(\hat{u},\groupTan) = \pi(\hat{u},d\gSpace_l(1) \groupTan)
\,,
\]
which - if I've decoded it right - is the ChaosBook slice condition.

Their `freezing ansatz'\rf{BeTh04} appears to be identical
to the Rowley and Marsden\rf{rowley_reduction_2003} and our
slicing, except that `freezing' is formulated as an
additional constraint, just as when we compute periodic
orbits of ODEs we add Poincar\'e section as an additional
constraint, \ie, increase the dimensionality of the problem
by 1 for every continuous symmetry. They prefer it this
way, as they are taking derivatives. They know that the
slice is local and things can diverge.
		\PC{copy to continuous.tex}

They illustrate `freezing' by numerical computations for the
quintic complex Ginzburg Landau equation (QCGL), which is
equivariant w.r.t. the action of the group $(\LieEl_r,\LieEl_t)
\in \Group = S^1 \times \reals$ on $u(x) \in \reals^2$. The
action is given by translation in the domain and rotation in
the image, i.e.
	\PC{copy to continuous.tex - an example?}
\bea
\LieEl \, u(x) &=& R_{\LieEl_r^{-1}} u(x - \LieEl_t)
	\,,\qquad
R_{\LieEl_r^{-1}} =
   \left(\barr{cc}
   \cos\theta  &  \sin\theta   \\
  -\sin\theta  &  \cos\theta
   \earr\right)
 \,,
 \label{QCGLrotation}
\eea
where subscripts are now just subscripts,
$r$ implying rotation and $t$ implying translation.

They refer to Chap.~8 of Govaerts\rf{Govaerts00} for a review
of numerical methods that employ equivariance with respect to
compact, and mostly discrete groups.
	\PC{copy to discrete.tex}

\item[2009-12-10 Barkley spirals reduction]
V. N. Biktashev, A. V. Holden, and E. V. Nikolaev\rf{BiHoNi96},
``Spiral wave meander and symmetry of the plane.''
They say
``
We present a group-theoretic approach based on the
{\em well-known space reduction method}, used to separate the motions in the
system into superposition of those `along' orbits of the
Euclidean symmetry group, and `across' the group orbits. It can
be interpreted as passing to a reference frame attached to the
spiral wave's tip. The system of ODEs governing the tip
movement describes the movements along the group orbits. The
motions across the group orbits are described by a PDE which
\emph{lacks the Euclidean symmetry}. Consequences of the Euclidean
symmetry on the spiral wave dynamics are discussed. We derive
the model system for bifurcation from rigid to biperiodic
rotation [Barkley 1994] from a priori symmetry considerations.
''

Spiral wave formation in nonlinear excitable reaction-diffusion
media was first observed
in 1970 by Zaikin and Zhabotinsky\rf{ZaZha70}.
Winfree\rf{Winfree73,Winfree1980} noted that spiral tips
execute meandering motions.
% others tend to cite Winfree 1980 monograph\rf{Winfree1980}.
Barkley and collaborators\rf{BaKnTu90,Barkley94} showed that
the noncompact Euclidean symmetry of this class of systems
precludes nonlinear entrainment of translational and rotational
drifts and leads to observed quasiperiodic and meandering
behaviors.

Barkley, Kness, and Tuckerman\rf{BaKnTu90},
``Spiral wave dynamics in a simple model of excitable
		 media: {T}ransition from simple to compound rotation.''

From Barkley\rf{Barkley94}:

The noncompact Euclidean symmetry group $E_2$ or $E(2)$ is the group
of distance-preserving symmetries in the plane: translations, rotations
and reflections.
(See Beyn\rf{BeTh04} eq.~(2.12).)
	\PC{copy to continuous.tex}

Rotating waves are steady
in a frame rotating with frequency $\omega$. In fluid flows with
axial symmetry they are non-axisymmetric flow patterns that
rotate as a rigid body with a uniform angular velocity\rf{Rand82}.
	\PC{copy to continuous.tex}

Modulated rotating waves are quasiperiodic states characterized
by two frequencies $(\omega,\omega')$ which are
periodic in a frame rotating with frequency $\omega$. They are
have\rf{Rand82}
a period $\period{}$ such that the flow patterns
at times $t$ and $t+\period{}$ differ by a rotation by $2\pi n/m$,
$0 \leq n < m/s$,
where $m$ is the number of wave peaks, $s$ the order of
axisymmetry of the wave pattern, $\omega$ the average angular
velocity of the wave, and
	\PC{copy to continuous.tex}
\beq
\Omega_M= 2\pi /\period{}
	\,,\qquad
\Omega_W= s(\omega+n \Omega_M)/m
\ee{MRW}
are fundamental frequencies, \ie, the temporal power spectrum
is of form $a\Omega_M+b \Omega_W$ where $a$ and $b$ are integers.

Modulated traveling waves are periodic states characterized
by frequency $\omega$ in a uniformly translating frame.

Barkley constructed a 5-dimensional model of meandering
spiral dynamics (later derived from symmetry-reduced
bifurcation theory in \refref{BiHoNi96} that describes
codimension-two bifurcation, with Hopf eigenmodes interact with the
Euclidean eigenmodes.

The story on equivariance seems to start with M. Field\rf{Field70},
and on symmetries in presence of bifurcations with
Ruelle\rf{ruell73}. Ruelle is a bit hard to read. I believe that
he proves that the \stabmat/\jacobianM\ evaluated
at an \eqv/fixed point $\ssp \in \pS_G$ decomposes
into linear irreducible represenations of \Group,
and that stable/unstable manifold continuations of its
eigenvectors inherit their symmetry properties.
He finds it remarkable that corresponding bifurcations can
go from an \eqv\ to a rotationally invariant periodic
orbit (\ie, \reqv), and not to other points.
	\PC{copy to continuous.tex}

\item[2009-12-23 Evangelos]
Biktashev, Holden, and Nikolaev\rf{BiHoNi96} assume that group
orbits foliate state space which I think is equivalent to a
free and regular group action, that is they have the same
restrictions as we do.  I found the Anosov and Arnol'd\rf{AnAr88} reference
they cite for their symmetry reduction method. It mentions that
one can locally, in the vicinity of a point which is not fixed by
the group can reduce the order of a system of differential equations
by the dimension of the group. I do not have Arnold's \emph{Geometrical Methods
in Classical Mechanics} book here, but I think it has a more extended
discussion of symmetries.

\item[2009-12-11 Predrag] I have added Evangelos PDF of
Biktashev, Holden, and Nikolaev\rf{BiHoNi96} to
\wwwcb{/library}.

\item[2009-12-12 Predrag] Need to check Lahme and Miranda\rf{LaMi99}
``Karhunen-Loeve decomposition in
    the presence of symmetry''. They say
``
The Karhunen-Loeve (KL) decomposition is
  widely used for data which often exhibit some symmetry.
  For a finite group, we derive an
  algorithm using group representation theory to reduce the
  cost of determining the KL basis. We demonstrate the
  technique on a Lorenz-type ODE system. For a compact group
  such as tori or SO(3,R) the method also applies: we
  consider the circle group $S^1$.
''
                                    \toCB


\item[2009-12-16 Erik Martens] % <erik.martens@ds.mpg.de>
Have a look at
Ed Ott and Tom Antonsen's reduction of $\infty$-dim Kuramoto phase-oscillator
problem to a $n$-dimensional problem:

1) proof that a low-dimensional manifold for the order parameter exists:
\\
\HREF{link.aip.org/link/?CHAOEH/18/037113/1}
{http://link.aip.org/link/?CHAOEH/18/037113/1}

2) this manifold is attractor for the order parameter:
\\
\HREF{http://link.aip.org/link/?CHAOEH/19/023117/1}
{http://link.aip.org/link/?CHAOEH/19/023117/1}

\item[2009-12-17 Predrag] Skyped Dwight.
Dwight says:

Zaikin, Winifree were indeed the first spiralers.

Dwight 1994 knew about symmetry reduction but did not
include it. He uses it in principle always, but by ``post-processing''
- simulates as usual, but then post-processes data into the slice.
Just as we recommend.

Biktashev, Holden, and Nikolaev\rf{BiHoNi96}
skirts around the issue - they wrote it
 but they did implement it.
%, so Dwight did not want his name on the paper.

\HREF{http://dynamics.mi.fu-berlin.de/persons/fiedler.php}
{Fiedler} would justly claim he found it first (influential talk
at Newton Inst).

Fiedler \etal\rf{FiSaScWu96} however did nothing numerical.

Beyn is the first to do nontrivial freezing in numerical
application

Kevrekidis added scaling.

Biktashev is currently implementing Beyn, see 2010-03-30
blog entry.

\item[2009-12-19 Predrag]
Reading Fiedler \etal\rf{FiSaScWu96}. They say:

$\Group \cdot \ \ssp_0$ is dipheomorphic in $\Group/H$,
where
\beq
H := \{ h \in \Group | h \,\ssp_0 = \ssp_0 \}
\,.
\ee{FieldIsotr}
A \Group-equivariant flow in a tubular neighborhood of a
\reqv\ $\Group \cdot \,\ssp_0, \ssp_0 \in \pS$,
with compact isotropy $H$ of $\ssp_0$, can be represented
by a {\em skew product} flow
\beq
\dot{\LieEl} = \LieEl \, \bf{a}(v)
	\,,\qquad
\dot{v} = \phi(v)
\,.
\ee{FiedlSkPr}
$\bf{a}$ is in the Lie algebra of \Group, and $v$ is in a linear
slice \pSRed, transverse to the group action. The slice  \pSRed\
is called a {\em Palais slice}, and $(\LieEl,v)$ are Palais
coordinates near \reqv. As the choice of the Palais slice is
arbitrary, these coordinates are not unique. The skew product flow
\refeq{FiedlSkPr}  was first
written down by Mielke\rf{Mielke91}, in the context of buckling
in the elasticity theory.


Reading Fiedler and Turaev\rf{FiTu98}.
Fetch it at
\\
\HREF{http://dynamics.mi.fu-berlin.de/preprints/BF_Normal.pdf}
{Normal forms, resonances, and meandering tip motions}
     near relative equilibria of {Euclidean} group actions.
They have a nice discussion of the semidirect product structure of the
Euclidean group $E(2)$.


\item[2010-03-30 Predrag]
We need to read
A. J. Foulkes and V. N. Biktashev\rf{FouBik10},
``Riding a Spiral Wave: Numerical Simulation of Spiral Waves
in a Co-Moving Frame of Reference.''
They describe an approach to numerical simulation of spiral
waves dynamics of large spatial extent, using small
computational grids.
``
An idea of dynamics in the space of symmetry group
orbits [31], when applied to a reaction-diffusion system
of equations and the Euclidean symmetry group, leads to
a description which is formally equivalent to considering
the solution in a moving frame of reference (FoR) such
that the spiral wave maintains a certain position and orientation
in this frame [29]. We call it comoving FoR.
We calculate
the dynamics of the spiral wave in a comoving FoR;
as a result, the core of the spiral never approaches the
boundaries of the computation box, which allows computations
of drift and meandering of large spatial extent using
small numerical grids. EZRide, a simple software implementation
of this approach, which is based on the popular spiral
wave simulator EZ-SPIRAL is provided on the authors' website.
Our approach can be compared to the approach proposed
by Beyn and V. Th\"ummler\rf{BeTh04} and further developed
by Hermann and Gottwald\rf{HerGot10}. Their approach also exploits
symmetry group orbits, but is different in some
essential details. We shall discuss the similarities and
differences when we will have introduced our method.
''

\item[2010-03-30 Predrag]
We need to read
\HREF{http://www.maths.usyd.edu.au/u/gottwald/papers.html}
    {Hermann and Gottwald}\rf{HerGot10}. They say:
``
We modify the freezing method introduced by Beyn and
Thuemmler, 2004, for analyzing rigidly rotating spiral waves
in excitable media. The proposed method is designed to stably
determine the rotation frequency and the core radius of
rotating spirals, as well as the approximate shape of spiral
waves in unbounded domains. In particular, we introduce
spiral wave boundary conditions based on geometric
approximations of spiral wave solutions by Archimedean
spirals and by involutes of circles. We further propose a
simple implementation of boundary conditions for the case
when the inhibitor is non-diffusive, a case which had
previously caused spurious oscillations.
We then utilize the method to numerically analyze the large
core limit. The proposed method allows us to investigate the
case close to criticality where spiral waves acquire infinite
core radius and zero rotation frequency, before they begin to
develop into retracting fingers. We confirm the linear
scaling regime of a drift bifurcation for the rotation
frequency and the core radius of spiral wave solutions close
to criticality. This regime is unattainable with conventional
numerical methods.
''


\item[2010-06-05, 2011-04-20 Predrag ] Lord and Vera
Th\"ummler\rf{LoTh10}
minimize the $L^2$ norm against a
`fixed profile' or `reference function:'
``results include our new idea of comparing the wave to a
reference function.'' Perhaps new in context of stochastic
PDEs...
They refer to
(deterministic) traveling wave as `Goldstein mode.'
												\toCB
They say in \HREF{http://arxiv.org/abs/1006.0428},
{\em Freezing stochastic traveling waves}:
``
 We consider traveling wave solutions to stochastic
PDEs and corresponding wave speed. As a particular example we
consider the Nagumo equation with multiplicative noise which we mainly consider
in the Stratonovich sense. A standard approach to determine the position and
hence speed of a wave is to compute the evolution of a level set. We compare
this approach against an alternative where the wave position is found by
minimizing the $L^2$ norm against a fixed profile. This approach can also be
used to stop (or freeze) the wave and obtain a stochastic partial differential
algebraic equation that we then discretize and solve. Although attractive as it
leads to a smaller domain size it can be numerically unstable due to large
convection terms. We compare numerically the different approaches for
estimating the wave speed. Minimization against a fixed profile works well
provided the support of the reference function is not too narrow. We then use
these techniques to investigate the effect of both Ito and Stratonovich noise
on the Nagumo equation as correlation length and noise intensity increases.
''


\end{description}



\renewcommand{\ssp}{a}
