% siminos/blog/baroclinic.tex
% $Author$ $Date$

\chapter{Baroclinic}
\label{chap:baroclinic}

\begin{description}

\item[2011-05-27 Predrag, weather report from Snowbird DS11]
	\toCB
Learned from Pierrehumbert that the baroclinic models are to
weathermen what ``harmonic oscillator'' is to quantum mechanics. It has a
continuous East-West translational symmetry, \ie, in a periodic box it
needs to be sliced (the \SOn{2} of periodic box quotiented out).

Tried to proselytize Christian Wolfe\rf{WoSa07}, Scripts,  make him slice
the baroclinic instability, and, perchance, if I get him there, recycle
it, former Gibson style. Pierrehumbert says that this would be persuasive
to weather people, convince them to go looking for exact unstable
invariant solutions. After one hour Wolfe said he was converted.

Tried ditto with Pierrehumbert and Silber postdoc Yi-Ping Ma (Knobloch
trained, has worked with Spiegel at Woods Hole GFD). He does not know any
geophysical fluid dynamics yet, so I'm sceptical that he will do
anything.

\item[2011-05-27 Annalisa Bracco]
if you ever need a code that produce baroclinic instability, I have plenty.
Worked with the Baroclinic Instability Man as a postdoc.
I also have an easy atmospheric model (global, on a sphere etc)
that reproduces very well baroclinic instability as per observations
and can be run as aqua-planet to simplify things.

\item[2011-05-27 Predrag]
Joe Pedlosky? We plumbers could make Joe happy?
Let's do it before August, in time for Woods Hole. The first step is to slice
your simulations for a minimal periodic cell of interest (narrow but turbulent).
The second step might be either to determine ``physical dimension'' using
Wolfe-Samelson\rf{WoSa07} Lyapunov vectors (that is just simulation) or find
some traveling waves (that is Krylov-Arnoldi nontrivial work,
but for low-dimensional discretizations might be doable by Newton).

\end{description}
