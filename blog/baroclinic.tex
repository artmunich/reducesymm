% siminos/blog/baroclinic.tex
% $Author$ $Date$

\chapter{Baroclinic}
\label{chap:baroclinic}

\begin{description}

\item[2011-05-27 Predrag, weather report from Snowbird DS11]
	\toCB
Learned from Pierrehumbert that the baroclinic models are to
weathermen what ``harmonic oscillator'' is to quantum mechanics. It has a
continuous East-West translational symmetry, \ie, in a periodic box it
needs to be sliced (the \SOn{2} of periodic box quotiented out).

Tried to proselytize Christian Wolfe\rf{WoSa07}, Scripts,  make him slice
the baroclinic instability, and, perchance, if I get him there, recycle
it, former Gibson style. Pierrehumbert says that this would be persuasive
to weather people, convince them to go looking for exact unstable
invariant solutions. After one hour Wolfe said he was converted.

Tried ditto with Pierrehumbert and Silber postdoc Yi-Ping Ma (Knobloch
trained, has worked with Spiegel at Woods Hole GFD). He does not know any
geophysical fluid dynamics yet, so I'm sceptical that he will do anything.

\item[2011-05-27 Annalisa Bracco]
If you ever need a code that produce baroclinic instability, I have plenty.
Worked with the Baroclinic Instability Man as a postdoc.
I also have an easy atmospheric model (global, on a sphere etc)
that reproduces very well baroclinic instability as per observations
and can be run as aqua-planet to simplify things.

\item[2011-05-27 Predrag]
Joe Pedlosky? We plumbers could make Joe happy?
Let's do it before August, in time for Woods Hole. The first step is to slice
your simulations for a minimal periodic cell of interest (narrow but turbulent).
The second step might be either to determine ``physical dimension'' using
Wolfe-Samelson\rf{WoSa07} Lyapunov vectors (that is just simulation) or find
some traveling waves (that is Krylov-Arnoldi nontrivial work,
but for low-dimensional discretizations might be doable by Newton).

\item[2011-10-15 Annalisa]
'Baroclynic' means that the instability is driven by density
difference; 'barotropic' means not.
She has shown me some simulations.
Baroclinic Instability is modeled by a 2-layer incompressible
viscous fluid in a channel with
no-slip side walls, periodic in streamwise direction, top layer
driven by
`atmospheric stream', \ie\ a constant total streamwise volume flow per unit
time. She simply imposes uniform streamwise velocity of unit size,
ignoring the boundary condition (could use a parabolic profile).
With free slip the layers are still unstable in the same way,
just the boundary behavior is different. Oceanographers prefer no-slip,
perhaps because of coastal stream.
The bottom layer is not forced, no slip, no Eckman layer
(Eckman layer models friction at the bottom; that would make sense when she works
with 50 layers, but not two).
The two layers are coupled by the difference $\Phi_2 - \Phi_1$.
Laplacian of stream function is vorticity.
Each layer is computed in terms of vorticity equations as a 2-dimensional
fluid. The lower layer has higher fluid density, and they are coupled
across their interface by difference of vorticity. This is about factor
two; it is related to the
R\"osby deformation radius $L_R$. The spanwise $y$ width is $L_R/2\pi = 1/2$.
Unless the width is larger than $L_R$, no instability. The streamwise
aspect ration is about 8.

They tend to the barotropic solution (vortices rotating the same way on
top and bottom).

\begin{enumerate}
  \item [Simulation 1)]
is linear: dropped the nonlinear term \ie\ the Jacobian of the vorticity
and the stream function. Instability is seeded by a small random field of
prescribed power spectrum, but the resulting instability is a localized
wave (about 3 rolls) with streamwise/spanwise ratio of about 1/2 set
$L_R$. That sets the scale of the instability off the laminar solution.
Initial noise does not matter, as the instability grows very fast. The
two layer vorticities are opposite.
  \item [Simulation 2)] is fully nonlinear, all other parameters the same.

  \item [Simulation 3)]
\end{enumerate}

Run 2-layers, spanways

\end{description}
