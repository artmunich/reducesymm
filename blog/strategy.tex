% siminos/blog/strategy.tex
% $Author$ $Date$

\chapter{Strategy, to write up}

\section{How to read me}

Throughout:  {\textdollar} on the margin
{\steady}
indicates that the text has been transferred to
articles and thesis siminos/*/,  or to ChaosBook.org desymmetrization
chapters, such as
\HREF{http:\\ChaosBook.org/continuous.pdf}
{continuous.pdf}.
%
For those whose Freud needs brushing up:
`Desymmetrization and its discontents' is a pun
on \HREF{http://en.wikipedia.org/wiki/Civilization_and_Its_Discontents}
{Civilization and its discontents}.

How to read this blog: go first to the latest blog post, end
of \refchap{c-DailyBlog}. For specialized topics, consult the
Contents.


\section{Email list}

create email list of our cohort:  Bristol, etc. participants

\subsection{Advertise arXiv rpo paper}

crosslink the paper with nonlin dynamics

email individually arXiv paper link to colleagues who might comment
    on the paper



\section{Papers to write}

\subsection{\emph{Physica D} ``Continuous symmetry...''}

\begin{description}

\item[2010-05-25 Vaggelis]
Will we follow editor's suggestion and resubmit with minor modifications only?
Deadline for receipt of the {\bf final} manuscripts is July 1st 2010.
Will we go for an arXiv version?

\item[2010-06-07 Predrag]
OK, now we have resubmitted, with minor edits. I think one should always submit
any article that is worth publishing also to arXiv;
it is open to anyone, rich or poor, and it is
more likely to reach the intended audience than only a publication through
any single journal.

\end{description}

\subsection{Reduced trace formulas?}

\begin{description}
 \item[2010-06-17 Vaggelis]
Since I have all rpo's up to level 7 for CLE I think I should try
to apply ``Continuous symmetry reduced trace formulas'' so that I get an incentive
to understand the paper. After all this group theory, it should be easier now.
 \item[2010-06-18 Predrag]
Would be nice if you did - both to understand the group theory better, and
also because I am not sure I have not missed some important detail about
invariant subspaces when I wrote the paper. Would be great to recycle KS 
next, if CLE works. 
\end{description}


\subsection{\emph{SIAM J. Appl. Dyn. Syst.}}

\begin{description}

\item[2010-06-07 Predrag] Next, the
``Continuous symmetry reduction of Kuramoto-Sivashinsky ...'' paper:
40,000 \rpo s and noplace to go?
Can include movies and more graphics

\end{description}

\subsection{PRL on recycling energy}

\section{Write next NSF proposal }

\section{Spruce up personal websites}

ES homepage with publication list, pdf files of talks, movies

Real ES homepage with links to the publication(s)

\section{Zoteromania}

\begin{description}

\item[2008-07-18 Predrag] about webtools for generating BibTeX:
www.zotero.org
        will pick up most books from Amazon, etc; but
        better to find a book first on
\HREF{http://www.worldcat.org}{www.worldcat.org}
          or
\HREF{http://scholar.google.com}{scholar.google.com}, then zotero it
          in a collection, and export in BibTeX format

\item[2009-12-22 Evangelos]
What do you think about setting up a cns group at zotero:
\HREF{http://www.zotero.org/blog/synchronize-pdfs-and-collaborate-with-zotero/}
{www.zotero.org/blog/synchronize-pdfs-and-collaborate-with-zotero}
  to keep track of important and hard of find papers?

\item[2010-01-07 Predrag] Sounds great, would not mind paying 20 bucks for the
storage. The only thing is - have not found it yet in their FAQs
\begin{enumerate}
   \item access should
be controlled by a group administrator, as materials are copyrighted.
  {\bf ES} My understanding is this is true.
   \item should maintain siminos.bib in the format we have now, the
crap zotero sticks into ziminoz.bib drives me nuts.
  {\bf ES} There is some script that controls export of bib files
      and it is user-tunable. The problem is it might require lots
      of work -- I have no time for hacking.
\end{enumerate}
Why don't you play with a potential CNS zotero group site for a while,
see how it works? {\bf ES} Will do and blog about it.
\item{2010-02-05 ES}
I've experimentally set up a cns zotero group.

\begin{tabular}{rl}
 User name 	& cns\\
 pass 		& you don't need it for access\\
 email 		& vaggelis.siminos@gmail.com (temporary)\\
Group name 	& cns\\
group site 	& \HREF{http://www.zotero.org/groups/cns}
                   {http://www.zotero.org/groups/cns} \\
 info   	& \HREF{http://www.zotero.org/documentation/quick_start_guide}
              {Zotero - Quick Start Guide}\\
\end{tabular}\\

To subscribe to it you will need a free zotero account, register
\HREF{https://www.zotero.org/user/register/}{here}. Then navigate
to the group \HREF{http://www.zotero.org/groups/cns}{site} and
ask to become member of the group. To sync with the library you
will need to enable syncing at your zotero firefox plugin preferences.

There are only some test entries in the library right now, we will need
to think a bit about the structure before creating collections. So it's
better not to invite more people to use it right now.

Group access options are:
\begin{description}
 \item[Public, Open Membership] Anyone can view your group online and join the group instantly.
 \item[Public, Closed Membership] Anyone can view your group online, but members must apply or be invited.
 \item[Private Membership] Only members can view your group online and must be invited to join.
\end{description}

ES has chosen Public, Closed Membership. I've also fine tuned the settings so that
% anyone in the web can see the database, but
only group members can change the database and see attachments, that
is pdf files. We can also restrict the privilege to change entries and files to group admins. This would
prevent accidental loss of data due to someone not knowing what he is doing, but then it won't be
possible for members to easily contribute entries.

\item[2010-01-07 ES] I'am leaning towards using tags rather than collections to organize the content
of my libraries but I am not sure about CNS library. I've been using collections for a while now
and the main problem is that many papers can fit to several collections.
So I either have to copy papers to all possible collections, or I end up having to browse
through multiple collections to find a paper.

The problem is that the web-interface does not display tags and does not support search as far as I can
tell. So one would have to use his zotero client to access the papers in a sensible way. Furthermore
since there is no obvious organization it would be hard to browse the library. So I think it would be
better to use both tags and an organization to collections.

\item[2010-05-13 Evangelos]
I think this is a good opportunity to test zotero cns library, so I've been uploading these
papers. Please check whether you can sync with the library.

\item[2010-05-12 PC] I seem to be able to access the papers on
\HREF{http://www.zotero.org/groups/cns}
{http://www.zotero.org/groups/cns} as user predrag@nbi.dk. Little creaky
- shows things like "APS Full Text PDF" which turns out to be a GaTech
library link to a Phys Rev Lett. They seem all correctly listed in
\HREF{http://www.zotero.org/groups/cns/items}
{http://www.zotero.org/groups/cns/items}, though.

I seem to be syncing, in the sense that my firefox zotero add-on
does show the cns library. Have not tried uploading a
anything yet.

\item[2010-05-12 ES] They seem to have two levels for the attachments.
Links look like:

APS Full Text PDF  (pdf, 618 KB)

They provide details and the link by which the files were downloaded if you press
on APS Full Text PDF. They provide the stored copy of the file if you press
on (pdf, 618 KB). I find it easier to use the firefox add-on to get the attachments though.

The best  way to add something to the library
is to drag it from your personal library and drop it to cns library within the zotero plugin.

\item[2010-05-14 ES]

I now can see Predrag's additions to zotero library.




\end{description}
