% siminos/blog/lit.tex
% $Author$ $Date$

\chapter{Literature}
\label{c:lit}

\begin{description}

\item[2012-01-10 Predrag] Collecting papers to read, and our notes on them in
this chapter.

\end{description}

\section{Symmetry literature}
\label{sec:symmLit}

\begin{description}

\item[2012-02-18 Predrag]
% Predrag 2013-04-20 moved to here from DOGS/saldana/blog.tex
% \section{Freezing solutions of equivariant evolution equations}
% \label{s:BeTh04}
Worth a read: Beyn and Th\"ummler\rf{BeTh04}. Their papers are
available under `Preprints' link on
\HREF{http://www.math.uni-bielefeld.de/~beyn/AG_Numerik/html/en/preprints/index.html}
{Beyn website}. Read also my notes in \refchap{c-excite}, and the
Chaosook   refrem {rem:SSLieGr} ``A brief history of relativity.''

\item[2010-06-05 Predrag]
I have read Beyn and Th\"ummler\rf{BeTh04} and believe ``freezing'' =
``slicing''. Siminos and I have Hermann and Gottwald\rf{HerGot10} on our
reading list, but have not studied it yet. Beyn and Th\"ummler impose a
slice by adding a new dimension (the phase parameter) and a condition (a
Lagrange multiplier), while we go one dimension down by restricting the
dynamics into the slice. Both approaches are standard in imposing
Poincar\' sections; Rytis Paskauskas has written that up for Chapter 3 -
Discrete time dynamics (not yet on the public version), and Chapter 13 -
Fixed points, and how to get them, Section 13.4 Flows. My feeling is that
Beyn and Th\"ummler is better, but we have not implemented it.

\item[2009-12-23 Predrag]
Went to Bielefeld, talked to
\HREF{http://www.math.uni-bielefeld.de/~beyn/AG_Numerik/html/en/people/}
{Wolf-J\"urgen Beyn}. Beyn
and his former PhD student Vera Th\"ummler\rf{BeTh04,Thum05} `freeze'
traveling waves. Their papers are mathematical, proving existence of spectra etc.,
but also applied to numerical examples. Scalar Nagumo equation is a good
testing ground, as one of the traveling wave solutions is known
analytically.

Beyn did his `freezing' at the same time as Rowley and
Marsden\rf{rowley_reduction_2003}, but they published first (a
year earlier?), so \refref{BeTh04} cites their `reconstruction
equation.' Beyn does not understand the `method of
connections,' either.

To understand why they call it `freezing' the figures are useful,
and especially the animations, like
\HREF{http://www.mathematik.uni-bielefeld.de/fgweb/Preprints/movies/fg03022_02.mpg}
{this one}. Of course, a turbulent state will not be 'frozen,'
but much of the hysterical fast drifting will be gone.

Beyn group\rf{BeTh04}  research focuses on the numerical
computation and stability of traveling wave solutions (and more
generally relative equilibria) of reaction-diffusion systems on
unbounded domains, \ie, of parabolic partial differential
equations (PDEs). On the real line they are of the form
\beq
u_t = A \, u_{xx} + f(u,u_x)
	\,,\quad
u: \reals \times [0,\infty) \to \reals
	\,,
A \in \reals^{d,d}
\,.
\ee{parabolicPDE}
A traveling wave is a special type of
a relative equilibrium of equivariant evolution equations,
where the action is given by translation,
   % \PC{copied this to ChaosBook}
\beq
\LieEl(\velRel) \, u(x)
  = \hat{u}(x - \velRel t)\,,\quad  \velRel \in \reals^d
\,,
\ee{BeThTW}
$\hat{u}(x)$ is the waveform and $\velRel$ the velocity. In the
comoving frame equations \refeq{parabolicPDE} take form
\beq
\hat{u}_t
 = A\, \hat{u}_{xx} + \velRel \, \hat{u}_x + f(\hat{u},\hat{u}_x)
	\,,\quad
\hat{u}_x \in \reals
	\,,\;
t \ge 0
\,,
\ee{comovingPDE}
with $\hat{u}(x)$ a stationary solution (the `freezing of the wave')
\beq
0= A \, \hat{u}{''} + \velRel \, \hat{u}{'} + f(\hat{u},\hat{u}{'})
\ee{BeThREQV}
The pair $(\hat{u},\velRel$) is an equilibrium solution of the
partial differential algebraic equation (PDAE)
\refeq{BeThREQV} which is constructed by inserting the co-moving
frame ansatz into PDE and adding an {\em additional phase
condition}. They say: ``By transforming the PDE into a
corresponding PDAE (partial differential algebraic equation)
via a freezing ansatz\rf{BeTh04} the relative equilibrium can
be analyzed as a stationary solution of the PDAE.''

Beyn finds Sandstede \etal\
articles\rf{FiSaScWu96,SaScWu97,SaScWu99a,FiTu98}
`remarkable.'
Sandstede \etal\rf{SaScWu97} use center manifold reduction
theory to study stability of \reqva. The main idea is to
transform the flow into a `skew product' form. One part is
orthogonal to the group orbit of the initial field $u(0)$, and
the other part acts within the group orbit and depends upon the
position in the orthogonal direction.

{\bf ES 2009-12-23} This decomposition is important in Krupa\rf{Krupa90} study
of bifurcations of relative equilibria, using a local slice. I think
he also introduces the tubular neighborhood. The book by Chossat
and Lauterbach\rf{ChossLaut00} gives a readable presentation
and I need it to compare their stability analysis for \reqva\ with
our Appendix, but I can't find it here.

%%%%%%%%%%%%%%%%%%%%%%%%%%%%%%%%%%%%%%%%%%%%%%%%%%
\SFIG{BeThEquiTraj}
{}{
Two trajectories
$\ssp(t)$, $\sspRed(t)$ are equivalent up to a group rotation
$\LieEl(t)$ as long as they belong to the same group orbit
$\pS_{\ssp(t)}$.
}
{fig:BeThEquiTraj}
%%%%%%%%%%%%%%%%%%%%%%%%%%%%%%%%%%%%%%%%%%%%%%%%%%
%
Beyn~\etal\rf{BeTh04} find it more convenient to make
use of the equivariance by extending the system rather than
reducing it as in bifurcation analysis, by adding an additional
parameter and a phase condition. They too write the solution as
${u}(t) = \LieEl(t)\,\hat{u}(t)$, see
\reffig{fig:BeThEquiTraj}, a composition of the action of a
time dependent group element $\LieEl(t)$ with a `frozen
solution' $\hat{u}(t)$ in a given Banach space, with
$\LieEl(t)$, $\hat{u}(t)$ to be determined. They keep the
`frozen solution' as constant as
possible%
\PC{`As constant as
possible' makes sense only for \reqva, not \rpo s which they do
not discuss.} by introducing a set of algebraic constraints
(phase conditions), $\psi(\hat{u}, \gSpace) = 0$, which fix the
extra degrees of freedom. Their number is given by the
dimension of the Lie group.


The freezing approach\rf{BeTh04} applies to traveling
waves and, more generally, to \reqva\rf{ChossLaut00,SaScWu99},
solutions which are
equilibria in an appropriately co-moving frame. They occur
in systems with underlying symmetry, such as
rotating waves on the real line and spiral waves in two space
dimensions. An example of a system with `rotating waves on the
real line' is the complex Ginzburg Landau equation, while
spirals are typical of the reaction-diffusion systems.

Consider an evolution equation in a Banach space $\pS$  of the form
\beq
u_t = \vel(u)
\ee{BeThEvEq}
with an equivariant right hand side $\vel$, i.e.
$\LieEl\,\vel(u) = \vel(\LieEl\,u)$ where $\LieEl : \Group \to
GL(\pS), \gSpace \to \LieEl(\gSpace)$ denotes the action of a
Lie group $\Group$ on  $\pS$. Here --in the parlance of applied
mathematicians-- $\vel$ is defined on a dense subspace of some
Banach space (generally infinite dimensional) and is
equivariant with respect to the action of a finite dimensional
(not necessarily compact) Lie group. In other words, the
$\infty$-dimensional functional PDE \statesp\ is spanned by
nicely shaped waveforms $u(x)$, nothing too kinky.

The equation \refeq{BeThEvEq}
can be transformed via the ansatz
$u(t) = \LieEl(t)\hat{u}(t)$ into the equivalent system
\beq
\hat{u}_t = \vel(\hat{u})
  - \LieEl(\gSpace)^{-1} \LieEl_\gSpace(\gSpace) \, \hat{u} \lambda
	\,,\qquad
\lambda = \gSpace_t
\,,
\ee{BeThReconstr}
where subscripted quantities imply partial derivatives with
respect to the subscript. Beyn~\etal\rf{BeTh04} then `freeze'
the traveling wave by fixing a \slice, and using a dot product
w.r.t. the group tangent at the slice point. The evolution of
$\gSpace(t)$ describes the motion on the group manifold. They
denote by $T_\gSpace \Group$ the tangent space of $\Group$ at
$\gSpace$. Introducing Lagrange parameters $\lambda(t) =
\LieEl_t(t) \in T_\gSpace \Group$ they impose `freezing ansatz'
on the extended system of equations \refeq{BeThReconstr}
\beq
\hat{u}_t = \vel(\hat{u})
  - \LieEl(\gSpace)^{-1} \LieEl_\gSpace(\gSpace) \, \hat{u} \lambda
	\,,\qquad
0 = \pi(\hat{u},\lambda)
\,,
\ee{BeThSlice}
with a ``phase condition
$\pi : \pSRed \times T_\gSpace \Group \to \reals^N$,
$N= \mbox{dim }\Group$,
which has to satisfy some regularity conditions.''
They differentiate
$
h \circ \LieEl \, \hat{u}
$
with respect to $\LieEl$, define group tangent at unity
$\groupTan \in \times T_1 \Group$, $\lambda = d\gSpace_l(1)
\groupTan$ ($l$ for left multiplication), and with some algebra
I'm too bored to type arrive at the extra equation(s)
\[
 0= \psi(\hat{u},\groupTan) = \pi(\hat{u},d\gSpace_l(1) \groupTan)
\,,
\]
which - if I've decoded it right - is the ChaosBook slice condition.

Their `freezing ansatz'\rf{BeTh04} appears to be identical
to the Rowley and Marsden\rf{rowley_reduction_2003} and our
slicing, except that `freezing' is formulated as an
additional constraint, just as when we compute periodic
orbits of ODEs we add Poincar\'e section as an additional
constraint, \ie, increase the dimensionality of the problem
by 1 for every continuous symmetry. They prefer it this
way, as they are taking derivatives. They know that the
slice is local and things can diverge.
                                    \toCB

They illustrate `freezing' by numerical computations for the
quintic complex Ginzburg Landau equation (QCGL), which is
equivariant w.r.t. the action of the group $(\LieEl_r,\LieEl_t)
\in \Group = S^1 \times \reals$ on $u(x) \in \reals^2$. The
action is given by translation in the domain and rotation in
the image, i.e.
	\PC{copy to continuous.tex - an example?}
\bea
\LieEl \, u(x) &=& R_{\LieEl_r^{-1}} u(x - \LieEl_t)
	\,,\qquad
R_{\LieEl_r^{-1}} =
   \left(\barr{cc}
   \cos\theta  &  \sin\theta   \\
  -\sin\theta  &  \cos\theta
   \earr\right)
 \,,
 \label{QCGLrotation}
\eea
where subscripts are now just subscripts,
$r$ implying rotation and $t$ implying translation.

They refer to Chap.~8 of Govaerts\rf{Govaerts00} for a review
of numerical methods that employ equivariance with respect to
compact, and mostly discrete groups.
	\PC{copy to discrete.tex}                          \toCB
% Predrag 2013-04-20: end of text moved to here from DOGS/saldana/blog.tex


\item[2012-02-19 Dan Goldman]
Please read the two Shapere-Wilczek papers\rf{ShWi87,ShWi06} on the
geometric approach to low Re swimming, and the engineering paper from my
CMU colleagues Hatton and Choset\rf{HaCho10} (click
\HREF{http://www.cs.cmu.edu/afs/cs.cmu.edu/Web/People/biorobotics/papers/DSCC2010_Hatton_Choset.pdf}
{here}). We have a paper in submission to PNAS (we are answering some
reviews) which applies these methods to a granular 3 link swimmer,
despite the fact that we don't have an equivalent of NS equations for
granular media. Using our empirical granular drag laws, the CMU guys
predict the optimal forward and turning movements for a 3-link swimmer to
10\% even at high joint angular excursions. I'm presently refining the
paper and making it readable (a reviewer complaint) and I'll send it
along once I do so.

I could really use a physicist to help interpret and help us (me + CMU
guys) take the next step.


\item[2012-02-19 Predrag] It's pretty impressive, the first
announcement\rf{ShWi87,ShWi89} in 1987, the long version\rf{ShWi06} in
2006. {\em J. Fluid. Mech.} decided it might be good enough once Wilczek
got Nobel Prize?
Perhaps Koiller, Ehlers, and Montgomery\rf{KoEhlMo96} review is helpful?
Or \refref{ShWi89a,DeArKo04}?

\item[2012-02-19 Greg Huber]
If you are asking if I can solve their problem, the answer is ``Yes'', I
am sort of an expert on this sort of thing! I have a paper\rf{HuKoYa11}
{\em Micro-swimmers with hydrodynamic interactions} on precisely this
topic. Here is the abstract and a
\HREF{http://www.itp.ucsb.edu/sites/default/files/Huber-Koehler-Yang.pdf}
{link} (or
\HREF{http://ChaosBook.org/library/HuKoYa11.pdf}{click here}):

Abstract:  The low-Reynolds-number motions of Purcell's three-link
swimmer, and of a closely related two-paddle swimmer, are investigated
and compared using slender-body theory and resistive-force theory. The
results are compared (in the case of the three-link swimmer) with the
resistive-force calculations of Becker, Koehler and Stone\rf{BeKoSt03}
(BKS). In particular, we examine the effect of hydrodynamic interaction
and slenderness on the displacement and efficiency of the swimmers. The
BKS analysis is, for the most part, confirmed and extended. However,
deviations of up to 43\% are found in cases where the swimmer propels
itself with large stroke angles. Finally, we discuss recent experimental
data in light of our numerical results.

Also, there is our PRL on Spiroplasma\rf{YaWoHu09} which has some
similarities to a three-link swimmer (click
\HREF{www.itp.ucsb.edu/sites/default/files/Huber-Koehler-Yang.pdf}{here},
or
\HREF{http://ChaosBook.org/library/YaWoHu09.pdf}{here}).

\item[2012-02-25 Predrag]
\HREF{http://www.massey.ac.nz/~rmclachl/}
{McLachlan} \etal\rf{McLPerlQui03} write in
{\em Lie group foliations: dynamical systems and integrators}
(I added it to Zotero): ``
Foliate systems are those which preserve some (possibly
singular) foliation of phase space, such as systems with integrals,
systems with continuous symmetries, and skew product systems. We study
numerical integrators which also preserve the foliation.
\toCB

\begin{quote}
The case in
which the foliation is given by the orbits of an action of a Lie group
has a particularly nice structure, which we study in detail, giving
conditions under which all foliate vector fields can be written as the
sum of a vector field tangent to the orbits and a vector field invariant
under the group action.
\end{quote}

This allows the application of many techniques of
geometric integration, including splitting methods and Lie group
integrators.
''

\item[2012-03-26 Predrag]
In \emph{Heteroclinic orbits in a spherically invariant system}
Armbruster and Chossat\rf{ArCho91} look at some bifurcations with
\On{3}-symmetry, might be a model to try slicing on in a non-abelian
setting.

\item[2012-05-16  Parameswaran Nair]  vpnair@optonline.net writes
on saddle solutions of Yang-Mills:

I attributed the conjecture to Hitchin; it was actually due to Atiyah and
Jones. "The only finite action solutions of the YM equations are
instantons, either self-dual or antiself-dual." This was the conjecture
for which the refs provide counter examples.

\HREF{http://ChaosBook.org/library/Schiff91.pdf}{Here is} the paper by my
student Schiff\rf{Schiff91}, who writes:
``
Following a proposal of Burzlaff (Phys.Rev.D 24 (1981) 546), we find
solutions of the classical equations of motion of an abelian Higgs model
on hyperbolic space, and thereby obtain a series of non-self-dual
classical solutions of four-dimensional SU(3) gauge theory. The lowest
value of the action for these solutions is roughly 3.3 times the standard
instanton action.
''

``
In physics, despite the fact that the non-self-dual solutions correspond
to saddle points, and not minima, of the Yang-Mills functional, to do a
correct semiclassical approximation by a saddle-point evaluation of the
path integral, it is certainly necessary to include a contribution due to
nonself-dual solutions, and if it should be the case that there is a
non-self-dual solution with action lower than the instanton action (this
question is currently open, and of substantial importance), then such a
contribution would even dominate. Unfortunately, it is questionable
whether the semiclassical approximation can give a reliable picture of
quantized gauge theories; it has been argued that in four-dimensional
gauge theory small quantum fluctuations
around classical solutions cannot be responsible for
confinement, unlike in certain lower-dimensional
theories. But it may still be possible to extract some
physics from the semiclassical approach. A first step in
such a direction would be to obtain a good understanding
of the full set of non-self-dual solutions and their properties.
[...] We pursue an old idea, due to Burzlaft
[10], for obtaining a non-self-dual, "cylindrically symmetric"
solution for gauge group SU(3). If we write
$\reals^4 = \reals \times \reals^3$, and identify some SU(2) [or SO(3)] subgroup
of SU(3), with generators that we will denote T', then we
can look at the set of SU(3) gauge potentials which are invariant
under the action of the group generated by the
sum of the T"s and the generators of rotations on the IR
factor of E (we choose the T"s and the IR rotation generators
to satisfy the same commutation relations). We
call such potentials "cylindrically symmetric" (in analogy
to the standard notion of cylindrical symmetry in IR,
which involves writing R =RXIR and requiring rotational
symmetry on the E factor). Such potentials will
be specified by a number of functions of two variables:
the coordinate on the IR factor of E (which we will
denote x), and the radial coordinate of the E factor
(which we will denote y). Clearly the equations of motion
for such cylindrically symmetric potentials (if they are
consistent) will reduce to equations on the space
I (x,y ):y ~ 0].
,,

The earlier work is by Sibner, Sibner, Uhlenbeck\rf{SiSiUhl89}. They
write ``
The Yang-Mills functional for connections on principle SU(2) bundles over
S 4 is studied. Critical points of the functional satisfy a system of
second-order partial differential equations, the Yang-Mills equations. If,
in particular, the critical point is a minimum, it satisfies a
first-order system, the self-dual or anti-self-dual equations. Here, we
exhibit an infinite number of finite-action non-minimal unstable critical
points. They are obtained by constructing a topologically nontrivial loop
of connections to which min-max theory is applied. The construction
exploits the fundamental relationship between certain invariant
instantons on S 4 and magnetic monopoles on H 3. This result settles a
question in gauge field theory that has been open for many years.
''

% 2012-05-16 fund this:
%@article{
%author = {Gil Bor and Richard Montgomery},
%title = {{SO(3)} invariant {Yang-Mills} fields which are not self-dual},
%}

Bor\rf{Bor92} writes
``
We prove the existence of a new family of non-self-dual finite-energy
solutions to the Yang-Mills equations on Euclidean four-space, with SU(2)
as a gauge group. The approach is that of ``equivariant geometry:''
attention is restricted to a special class of fields, those that satisfy
a certain kind of rotational symmetry, for which it is proved that (1) a
solution to the Yang-Mills equations exists among them; and (2) no
solution to the self-duality equations exists among them. The first
assertion is proved by an application of the direct method of the
calculus of variations (existence and regularity of minimizers), and the
second assertion by studying the symmetry properties of the linearized
self-duality equations. The same technique yields a new family of
non-self-dual solutions on the complex projective plane.
''

\item[2012-06-13 Predrag]
Fran�ois Gay-Balmaz and Darryl D. Holm
{\em Parameterizing interaction of disparate scales: Selective decay by
Casimir dissipation in fluids}, \arXiv{1206.2607} say: ``
The problem of parameterizing the interactions of disparate scales in
fluid flows is addressed by considering a property of two-dimensional
incompressible turbulence. The property we consider is selective decay,
in which a Casimir of the ideal formulation (enstrophy in 2D flows)
decays in time, while the energy stays essentially constant. This paper
introduces a mechanism that produces selective decay by enforcing Casimir
dissipation in fluid dynamics. [...] a general theory of selective decay
is developed that uses the Lie-Poisson structure of the ideal theory.
[...] may be useful in turbulent geophysical flows where it is
computationally prohibitive to rely on the slower, indirect effects of a
realistic viscosity, such as in large-scale, coherent, oceanic flows
interacting with much smaller eddies.

[...] From the viewpoint of Noether's theorem, energy is conserved in
ideal fluid dynamics because of the time-translation symmetry of the
Lagrangian in Hamilton's principle for ideal fluid motion. A second type
of fluid conservation law arises via Noether's theorem because of
relabelling symmetry of the Lagrangian. Relabelling symmetries smoothly
transform the labels of the fluid parcels without changing the Eulerian
quantities on which Hamilton's principle for ideal fluids depends. The
conservation laws associated with relabelling symmetries are called
Casimirs, because in the Hamiltonian formulation of ideal fluid dynamics
in the Eulerian representation their Lie-Poisson brackets with any other
functionals vanish identically. Thus, the Casimirs arise from a geometric
symmetry of the Eulerian representation of ideal fluid dynamics. This
relabelling symmetry is also responsible for Kelvin's circulation theorem
in ideal fluid dynamics, which immediately leads to the conservation of
the Casimirs for 2D ideal incompressible flow.

The Kelvin circulation theorem and the associated Casimir conservation
laws are kinematic, because they hold for any choice of Hamiltonian in
the Eulerian representation. Energy conservation is dynamic.

[...] The two types of ideal fluid constants of motion, energy and
Casimirs, typically have quite different dependencies on spatial
gradients of the solutions. Consequently, the interplay between them can
be interpreted as an interaction between larger and smaller scales (or
coherence lengths, or spectral wavenumbers).
''

It is all 2D Eulerian, and mathematics is heavy.

\item[2012-05-20 Jeff Greensite] has written a book\rf{Greensite11} of
possible interest, \emph{An introduction to the confinement problem}.

\HREF{http://en.wikipedia.org/wiki/Gribov_ambiguity}{Gribov ambiguity wiki}
(edits by Predrag):

Gauge fixing means choosing a representative from each gauge orbit. The
space of representatives is a submanifold and represents the gauge fixing
condition. Ideally, every gauge orbit will intersect this submanifold
once and only once. This is generally impossible globally, especially for
non-abelian gauge theories, because of topological obstructions and the
best that can be done is make this condition true locally. A gauge fixing
submanifold may not intersect a gauge orbit at all or it may intersect it
more than once. This is called a Gribov\rf{Gribov77} ambiguity.

Gribov ambiguities lead to a nonperturbative failure of the BRST
symmetry, among other things.

A way to resolve the problem of Gribov ambiguity is to restrict the
relevant functional integrals to a single \emph{Gribov region} or {\em
fundamental modular region} whose boundary is called a \emph{Gribov
horizon}.

{\em Gribov copies} play a crucial role in the infrared (IR) regime while
it can be neglected in the perturbative ultraviolet (UV)
regime\rf{Gribov77,Zwanz89,Zwanz93}. The restriction to the Gribov region
(defined in such a way that the Faddeev-Popov operator is strictly
positive) can be achieved by adding a nonlocal term, commonly known as
`horizon term', to the YM action\rf{Zwanz89,Zwanz93,Zwanz92}. This is a
nonlocal term in the 4-dimensional Euclidean space, written as an
integral over the `horizon function.'

Greensite: ``In non-Abelian theories, there are many gauge copies -
Gribov copies - that satisfy the Coulomb gauge condition. The Gribov
region is the space of all Gribov copies with positive Faddeev-Popov
eigenvalues. Configurations of the Gribov horizon have at least one FP
eigenvalue $\lambda =0$. What counts for confinement is the density of
eigenvalues $\rho(\lambda)$ near $\lambda =0$, and the `smoothness' of
these near-zero eigenvalues.

The Gribov horizon is a convex manifold in the space of gauge fields,
both in the continuum and on the lattice. The Gribov region, bounded by
that manifold, is compact.
''

Amusingly, they can find the first 200 eigenstates of the lattice
Faddeev-Popov operator on each time-slice of each lattice configuration
by the Arnoldi algorithm.

See also Heinzl\rf{Heinzl96,HeRuSch08}, as well as
\refrefs{RuSchVo02,MaScha94,vanBaal91,DellAnZwan91,Cutkosky84,Singer78}.
Review of \refref{VaZw12} promises ``to give a pedagogic review of the
ideas of Gribov and the subsequent construction of the GZ action,
including many other topics related to the Gribov region.''

Nele Vandersickel
\HREF{http://physik.uni-graz.at/~dk-user/talks/Vandersickel20100225.pdf}
{talk} gives a compact overview, might be useful for writing this up.
Chapter 3 of her thesis, \arXiv{1104.1315}, gives a pedagogic overview of
the Gribov-Zwanziger framework, not available yet in the literature.

\item[2012-06-14 Predrag]
\HREF{http://marcofrasca.wordpress.com/about/}{Frasca} is either insane
or just yet another ignorant field theorist: ``I have worked on almost
all fields of physics'' (???). I checked the publication list, and it is
no  L.D. Landau. But the blog is informative:

{\bf [2012-06-05 Frasca]} (edited by Predrag):

``The answer to the question of the mass gap in Yang-Mills theory has
made enormous progress mostly by the use of lattice computations and,
quite recently, with the support of theoretical analysis. Contrarily to
common wisdom, the most fruitful attack to this problem is using Green
functions. The reason why this was not a greatly appreciated approach
relies on the fact that Green functions are gauge dependent.
Nevertheless, they contain physical information that is gauge independent
and this is exactly what we are looking for: The mass gap.''

(Predrag: this I interpret in the spirit of Gutzwiller - the Green function
is coordinate dependent, but it's trace - which yield the spectrum - is
coordinate invariant.)

``\HREF{http://marcofrasca.wordpress.com/2011/01/28/the-saga-of-landau-gauge-progators-a-short-history/}
{The Saga of Landau-Gauge Propagators: A Short History}'' is a good read:

``We cannot forever ignore the low energy behavior of QCD as its complete
understanding could have impact at unexpected large scales.''

\item[2012-06-15 Predrag]
Laufer and Orland\rf{LauOrl12} say in
{\em The geometry of {Yang-Mills} orbit space on the lattice}: ``
We find coordinates, the metric tensor, the inverse metric tensor and the
Laplace-Beltrami operator for the orbit space of Hamiltonian SU(2) gauge
theory on a finite, rectangular lattice. This is done using a complete
axial gauge fixing. The Gribov problem can be completely solved, with no
remaining gauge ambiguities.
''
\item[2013-01-20  Predrag]
This really belongs to planar field theory, but for time being I note
it here: Lucini and Panero\rf{LucPan13} (in Chaosbook.org/library)
might be of interest. All I get is one sentence and a reference only
to \refref{PlanFieldThe}.

\item[2013-03-27  Predrag] Do not understand this article:
Jim\'enez-Lara and J. Llibre\rf{JimLli11},
{\em Periodic orbits and nonintegrability of generalized
classical {Yang--Mills Hamiltonian} systems}.

                                                \toCB
D. Biswas \etal\rf{BALL92} {\bf Existence of stable periodic orbits
in the $x^2y^2$ potential: a semiclassical approach}, re-derives the
Dahlqvist and G. Russberg\rf{DR_prl} result. Also read Nip
\etal\rf{NTOD92} {\em Search for regular orbits in the $x^2y^2$
potential problem}

Hu\rf{HJLW01}
{\em General initial value form of the semiclassical propagator},
write: ``
We show a general initial value form of the semiclassical propagator.
Similar to cellular dynamics, this formulation involves only the
nearby orbits approximation: the evolution of nearby orbits is
approximated by linearized dynamics. This phase space smearing
formulation keeps the accuracy of the original Van Vleck-Gutzwiller
propagator. As an illustration, we present a simple initial value
form of the semiclassical propagator. It is nonsingular everywhere
and is efficient for numeric implementation.
''

\item[2013-04-16  Predrag] There seems to be whole literature on
classical Yang-Mills (CYM). In
{\em Entropy production in classical Yang-Mills theory from Glasma
initial conditions} Hideaki Iida,  Teiji Kunihiro,  Berndt M\"uller,
Akira Ohnishi,  Andreas Sch\"afer,  and Toru T. Takahashi,
\arXiv{1304.1807}, % \rf{IKMOST13}
write:

Pure Yang-Mills theory in temporal gauge with the Hamiltonian in the
noncompact (A, E) scheme on a cubic spatial lattice. The initial
condition satisfies Gauss' law; check its validity as well as Energy
conservation carefully at every time step. Define distance (6), (7) that
is gauge invariant under residual (time independent) gauge transformations.

                                                    \inCB
They call the stability matrix `Hessian', and its eigenvalues at time
slice the `local Lyapunov exponents (LLEs)'\rf{KMOSTY10}: LLE plays the
role of a ``temporally local'' Lyapunov exponent, which specifies the
departure rate of two trajectories in a short time period. Then they say
this (?): ``For a system where stable and unstable modes couple with each
other as in the present case, an LLE does not generally agree with the
Lyapunov exponent in a long time period.'' ``\refRef{KMOSTY10} introduced
another kind of Lyapunov exponent called the intermediate Lyapunov
exponent (ILE), which is an ``averaged Lyapunov exponent'' for an
intermediate time period; i.e., a time period which is sufficiently small
compared to the thermalization time but large enough to sample a
significant fraction of phase space. By its definition (13) it
is the set of stability exponents for a finite time \jacobianM.

``Two comments are in order, here: A Lyapunov exponent [PC: not the
Lyapunov exponent, they mean the stability exponent] can be (real)
positive, negative, zero or purely imaginary. Liouville's theorem tells
us that the determinant of the time evolution matrix U is unity, implying
that the sum of all positive and negative ILEs is zero. The KS entropy is
given as a sum of positive Lyapunov exponents. The second comment
concerns gauge invariance of the Lyapunov exponents. In the Appendix we
show that LLE and ILE are indeed gauge invariant under time-independent
gauge transformations in the temporal gauge.''



\end{description}
