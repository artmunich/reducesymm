% siminos/blog/lit.tex
% $Author$ $Date$

\chapter{Literature}
\label{c:lit}

\begin{description}

\item[2012-01-10 Predrag] Collecting papers to read, and our notes on them in
this chapter.

\end{description}

\section{Symmetry literature}
\label{sec:symmLit}

\begin{description}

\item[2012-02-19 Dan Goldman]
Please read the two Shapere-Wilczek papers\rf{ShWi87,ShWi06} on the
geometric approach to low Re swimming, and the engineering paper from my
CMU colleagues Hatton and Choset\rf{HaCho10} (click
\HREF{http://www.cs.cmu.edu/afs/cs.cmu.edu/Web/People/biorobotics/papers/DSCC2010_Hatton_Choset.pdf}
{here}). We have a paper in submission to PNAS (we are answering some
reviews) which applies these methods to a granular 3 link swimmer,
despite the fact that we don't have an equivalent of NS equations for
granular media. Using our empirical granular drag laws, the CMU guys
predict the optimal forward and turning movements for a 3-link swimmer to
10\% even at high joint angular excursions. I'm presently refining the
paper and making it readable (a reviewer complaint) and I'll send it
along once I do so.

I could really use a physicist to help interpret and help us (me + CMU
guys) take the next step.


\item[2012-02-19 Predrag] It's pretty impressive, the first
announcement\rf{ShWi87,ShWi89} in 1987, the long version\rf{ShWi06} in
2006. {\em J. Fluid. Mech.} decided it might be good enough once Wilczek
got Nobel Prize?
Perhaps Koiller, Ehlers, and Montgomery\rf{KoEhlMo96} review is helpful?
Or \refref{ShWi89a,DeArKo04}?

\item[2012-02-19 Greg Huber]
If you are asking if I can solve their problem, the answer is ``Yes'', I
am sort of an expert on this sort of thing! I have a paper\rf{HuKoYa11}
{\em Micro-swimmers with hydrodynamic interactions} on precisely this
topic. Here is the abstract and a
\HREF{http://www.itp.ucsb.edu/sites/default/files/Huber-Koehler-Yang.pdf}
{link} (or
\HREF{http://ChaosBook.org/library/HuKoYa11.pdf}{click here}):

Abstract:  The low-Reynolds-number motions of Purcell's three-link
swimmer, and of a closely related two-paddle swimmer, are investigated
and compared using slender-body theory and resistive-force theory. The
results are compared (in the case of the three-link swimmer) with the
resistive-force calculations of Becker, Koehler and Stone\rf{BeKoSt03}
(BKS). In particular, we examine the effect of hydrodynamic interaction
and slenderness on the displacement and efficiency of the swimmers. The
BKS analysis is, for the most part, confirmed and extended. However,
deviations of up to 43\% are found in cases where the swimmer propels
itself with large stroke angles. Finally, we discuss recent experimental
data in light of our numerical results.

Also, there is our PRL on Spiroplasma\rf{YaWoHu09} which has some
similarities to a three-link swimmer (click
\HREF{www.itp.ucsb.edu/sites/default/files/Huber-Koehler-Yang.pdf}{here},
or
\HREF{http://ChaosBook.org/library/YaWoHu09.pdf}{here}).

\item[2012-02-25 Predrag]
\HREF{http://www.massey.ac.nz/~rmclachl/}
{McLachlan} \etal\rf{McLPerlQui03} write in
{\em Lie group foliations: dynamical systems and integrators}
(I added it to Zotero): ``
Foliate systems are those which preserve some (possibly
singular) foliation of phase space, such as systems with integrals,
systems with continuous symmetries, and skew product systems. We study
numerical integrators which also preserve the foliation. The case in
which the foliation is given by the orbits of an action of a Lie group
has a particularly nice structure, which we study in detail, giving
conditions under which all foliate vector fields can be written as the
sum of a vector field tangent to the orbits and a vector field invariant
under the group action. This allows the application of many techniques of
geometric integration, including splitting methods and Lie group
integrators.
''

\end{description}
