%Scalefactor = 2.5
\color{black}
\indent
Computationally, the slicing problem is finding the parameters of group action
as a function of time that maps the points $x(t)$ in the full state 
space to their representatives $\hat{x}(t) = g(-\theta(t)) x(t)$ on the slice.
One can also obtain the dynamics within the slice hyperplane by directly integrating
\begin{equation}
	\hat{v}(\hat{x}) = v(\hat{x}) - \dot{\theta} (\hat{x}) t(\hat{x}) 
	\, , \qquad
	\dot{\theta} (\hat{x}) = \langle v(\hat{x}) | t' \rangle / \langle t(\hat{x})  | t' \rangle
	\nonumber
\end{equation}  
where $v(\hat{x})$ and $\hat{v}(\hat{x})$ are velocities in symmetry equivariant
and symmetry reduced state spaces; and $t(x)$ is the group tangent evaluated
at the point $x$. For the derivation of these equations and the detailed
discussion of the method of slices, see [2].
Reduced trajectory of the Porter-Knobloch system is the red curve in the main
figure which is not a projection since within the slice, $y_2$ component of
the points are zero.
