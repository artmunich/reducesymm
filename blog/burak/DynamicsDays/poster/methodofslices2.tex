%Scalefactor = 2.5
\color{black}
\begin{equation}
	\mathcal{M}_x = \{g x | g \in G\} \nonumber
\end{equation}
Points on a group orbit are dynamically equivalent. The general symmetry
reduction problem is finding a representation of the system,  such that every 
group orbit is represented by a single representative point.

In the method of slices, one looks for a submanifold $\hat{\mathcal{M}}$ in 
the full state space in such a way that every group orbit intersects $\hat{\mathcal{M}}$ 
only once. Points $\hat{x}$ at which the group orbits pierce the slice are taken
as the representatives of the group orbit, thus, on the slice, equivalent set
of points $\mathcal{M}_{\hat{x}}$ are represented by a single point $\hat{x}$.

While in general, finding a good slice is a non-trivial problem, for the case
at hand, a single hyperplane that includes $\hat{x}' = (1,0,0,0)$ and is perpendicular
to the group tangent evaluated at this point, $\textbf{T} \hat{x}' = t' = (0,1,0,0)$ where
$\bf{T}$ is the $SO(2)$ Lie algebra element, can represent every group orbit 
that has a non-zero first mode component. This slice is illustrated in a 3D 
projection in Figure 1.
