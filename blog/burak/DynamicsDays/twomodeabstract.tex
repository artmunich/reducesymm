%Abstract Template


\documentclass[11pt]{article}
\usepackage{calc}
\usepackage{color}
\usepackage{amsfonts}
\usepackage{latexsym}
\usepackage{placeins}
\ifx\pdftexversion\undefined
  \usepackage[dvips]{graphicx}
\else
  \usepackage[pdftex]{graphicx}
\fi
\usepackage{amssymb}
\usepackage{authblk}
\usepackage{amsmath}
\usepackage[cp1250]{inputenc}
\usepackage[OT4]{fontenc}

\addtolength{\voffset}{-3.5cm} \addtolength{\textheight}{4cm}

\renewcommand\Authfont{\scshape\small}
\renewcommand\Affilfont{\itshape\small}
\setlength{\affilsep}{1em}

\newcommand{\smalllineskip}{\baselineskip=15pt}
\newcommand{\keywords}[1]{{\footnotesize\hspace{0.68cm}{\textit{Keywords}: }#1\par
  \vskip.7\baselineskip}}
\renewenvironment{abstract}[0]{\small\rm
        \begin{center}ABSTRACT
        \\ \vspace{8pt}
        \begin{minipage}{5.2in}\smalllineskip
        \hspace{1pc}}{\end{minipage}\end{center}\vspace{-1pt}}
\newcommand{\emailaddress}[1]{\newline{\sf#1}}

\let\LaTeXtitle\title
\renewcommand{\title}[1]{\LaTeXtitle{\large\textsf{\textbf{#1}}}}

%%%TITLE
\title{Reduction of continuous SO(2) symmetry of a 2-mode system using method of slices}
\date{}

%%AFFILIATIONS
\author[1]{Nazmi Burak Budanur} 
\author[1]{P. Cvitanovi\'{c}}
\affil[1]{School of Physics and Center for Nonlinear Dynamics,
		  Georgia Inst. of Technology,
		  Atlanta, GA  30332, USA \emailaddress{budanur3@gatech.edu}}


%%DOCUMENT
\begin{document}
\maketitle

%%PLEASE PUT YOUR ABSTRACT HERE
\begin{abstract}

Bifurcations of solutions to 2-mode SO(2) equivariant ODE normal forms to 
the third order in mode amplitudes are studied extensively in \cite{Dang86} 
and \cite{PoKno05}. We take a four dimensional SO(2)-equivariant system of 
this kind as the simplest example which can exhibit chaos while having a continuous 
symmetry, and discuss its representations in equivariant state space, polar
coordinates, and invariant polynomials. We show that invariant polynomial 
representation enables us to find all the relative equilibria of such system. 
To reduce the continuous symmetry, we apply the method of slices and show that 
the singularities related to this technique can be avoided by choice of a 
special slice template for this system. We construct Poincar\'e return maps 
on the slice hyperplane to determine the relative periodic orbits of the 
system. We visualize each step of this process without projecting solutions
onto a submanifold since the slice hyperplane for this system is three dimensional.
We show that our method can be straightforwardly generalized and used to reduce 
the continuous SO(2)-symmetry when N-modes are present, as long as the amplitude 
of the first Fourier mode is non-zero. 

\end{abstract}
%%THE END OF ABSTRACT

\begin{thebibliography}{99}
\small

\bibitem{Dang86} G. Danglmayr,  {\it Dyn. Sys.} {\bf 1}, pp. 159--185 (1986)

\bibitem{PoKno05} J. Porter, E. Knobloch, {\it Physica D} {\bf 201}, pp. 318--334 (2005)

%\bibitem{atlas12} P. Cvitanovi\'{c}, D. Borrero-Echeverry, K. Carroll, B. Robbins,
%E. Siminos, {\it Chaos} {\bf 22}, 047506 {2012}

\end{thebibliography}
\end{document}
