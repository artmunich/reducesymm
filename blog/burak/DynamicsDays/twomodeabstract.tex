%Abstract Template


\documentclass[11pt]{article}
\usepackage{calc}
\usepackage{color}
\usepackage{amsfonts}
\usepackage{latexsym}
\usepackage{placeins}
\ifx\pdftexversion\undefined
  \usepackage[dvips]{graphicx}
\else
  \usepackage[pdftex]{graphicx}
\fi
\usepackage{amssymb}
\usepackage{authblk}
\usepackage{amsmath}
\usepackage[cp1250]{inputenc}
\usepackage[OT4]{fontenc}

\addtolength{\voffset}{-3.5cm} \addtolength{\textheight}{4cm}

\renewcommand\Authfont{\scshape\small}
\renewcommand\Affilfont{\itshape\small}
\setlength{\affilsep}{1em}

\newcommand{\smalllineskip}{\baselineskip=15pt}
\newcommand{\keywords}[1]{{\footnotesize\hspace{0.68cm}{\textit{Keywords}: }#1\par
  \vskip.7\baselineskip}}
\renewenvironment{abstract}[0]{\small\rm
        \begin{center}ABSTRACT
        \\ \vspace{8pt}
        \begin{minipage}{5.2in}\smalllineskip
        \hspace{1pc}}{\end{minipage}\end{center}\vspace{-1pt}}
\newcommand{\emailaddress}[1]{\newline{\sf#1}}

\let\LaTeXtitle\title
\renewcommand{\title}[1]{\LaTeXtitle{\large\textsf{\textbf{#1}}}}

%%%TITLE
\title{Reduction of continuous SO(2) symmetry of a 2-mode system using method of slices}
\date{}

%%AFFILIATIONS
\author[1]{Nazmi Burak Budanur}
\author[1]{P. Cvitanovi\'{c}}
\affil[1]{School of Physics and Center for Nonlinear Dynamics,
		  Georgia Inst. of Technology,
		  Atlanta, GA  30332, USA \emailaddress{budanur3@gatech.edu}}


%%DOCUMENT
\begin{document}
\maketitle

%%PLEASE PUT YOUR ABSTRACT HERE
\begin{abstract}

Danglmayr~\cite{Dang86} and Porter~\&\ Knobloch~\cite{PoKno05} have introduced a family of 2-Fourier mode SO(2)-equivariant ODEs  in order to study bifurcations of solutions of dynamical systems in presence of symmetries. A 4-dimensional system of this kind is perhaps the simplest example of a system with a continuous symmetry that can exhibit chaos, so we use it to illustrate the role symmetries play in chaotic dynamics. We show that a continuous symmetry induces drifts in the 4-dimensional state space dynamics, drifts which obscure the chaotic dynamics. Change of equations of motions to a locally symmetry-invariant `comoving' frame does not eliminate these drifts: that is only attained by a \emph{symmetry reduction} - reformulation of dynamics in a 3-dimensional symmetry-reduced state space, where every group orbit (set of all points reached by actions of the group of all symmetries of the equations of motion) is replaced by a point. Porter~\&\ Knobloch system is a particularly nice illustration of how this works, as in 3 dimensions we are able to visualize everything.

We compare three symmetry reduction methods: polar coordinates, invariant polynomial bases, and the `method of slices'. An invariant polynomial basis is convenient for determination of all relative equilibria of such system. Our conclusion, however, is that the most insight is offered by the method of slices. While in general a number of local slices are needed to cover a strange attractor~\cite{atlas12}, for the Porter~\&\ Knobloch system there we define a unique slice hyperplane that captures \emph{all} symmetry-reduced dynamics. A Poincar\'e return map within the slice hyperplane enables us to reduce the dynamics further, essentially to a unimodal map, and determine, in principle, all relative periodic orbits of the system. We can visualize each step of this process without having to project solutions onto a submanifold since the slice hyperplane for this system is three dimensional. We argue that our method can reduce the SO(2)-symmetry also for $N$-Fourier modes truncations of PDEs such as the Kuramoto-Sivashinsky, pipe flows, etc., as long as the amplitude of the first Fourier mode is non-zero.

\end{abstract}
%%THE END OF ABSTRACT

\begin{thebibliography}{99}
\small

\bibitem{Dang86} 
G. Danglmayr,  {\it Dyn. Sys.} {\bf 1}, pp. 159--185 (1986).

\bibitem{PoKno05} 
J. Porter and E. Knobloch, {\it Physica D} {\bf 201}, pp. 318--334 (2005).

\bibitem{atlas12} 
P. Cvitanovi\'{c}, D. Borrero-Echeverry, K. Carroll, B. Robbins and
E. Siminos, {\it Chaos} {\bf 22}, 047506 (2012).

\end{thebibliography}
\end{document}
