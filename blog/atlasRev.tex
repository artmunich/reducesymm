% siminos/blog/atlasRev.tex
% $Author$ $Date$

\section{Atlas paper revisions}
\label{sect:atlasRev}

{\color{red} enter here the internal discussion of referee comments}

\begin{description}

\item[2012-07-01 Predrag] Mein Gott. Will proud ignorance of fellow
plumbers ever end?

\end{description}

\subsection{Response 1.1}
\label{sect:Response1.1}

\begin{description}

\item[2012-07-01 Predrag] Not sure what to do with this report -
there are different paths to wisdom, and we cannot
write a different paper. Our point is that symmetry reduction is
necessary, so downplaying it is not what we want to do...

\bigskip

Referee: ``This paper is mainly a review of some techniques for symmetry reduction
that are useful for high-dimensional systems. In particular, the authors
present a method of constraining to a subspace called a "slice," or a set
of such subspaces. The method itself is not new, having been used in the
authors' references 3-6, but this paper presents a nice overview and
several interesting examples.''

\item[2012-07-16 Evangelos]
	Regarding novelty, I think that the idea of bringing together
	charts into an atlas has been proposed many times (including our
	own papers) but has not been implemented elsewhere. Should we
	emphasize this in the text?

{\bf [2012-07-16 Predrag]} to Evangelos: I have answered this question
several times in the blog above, but again, we presumably know the
literature, so
(1) can you give me one or more external references in which the
algorithm of constructing an atlas for the symmetry-reduced \statesp\ is
described?
(2) can you give me one or more external references in which the
$(d\!-\!2)$\dmn\ \emph{\poincBord} ${\cal S} \subset \PoincS$ is defined,
discussed, computed, explained?
(3) can you give me one or more references authored by us (or anyone
else) in which the \emph{ridge} is defined and constructed? I.e., a pair
of \KCedit{$(d\!-\!N)$\dmn\ } local \slice\ hyperplanes intersects in a
{ridge}, a \KCedit{$(d\!-\!N-\!1)$\dmn\ } hyperplane {\PoincS} of points
$\sspRed^*$ shared by a pair of charts and thus satisfying the \slice\
condition \refeq{PCsectQ0}.
{\bf [2012-07-17 Evangelos]} No, I can't and
that's exactly my point. This is all new but we do not emphasize it
properly, so it gives the impression of it all being review material.
I've slightly rephrased a sentence in the introduction to reflect novelty,
but I think we need to do more than this - emphasize it in the abstract,
conlcusions and maybe introductory paragraph.

Also, I would remove, if possible, credit to Chaosbook.org for figures -
people think that if this is already in a book, then it has to be review
material.
 {\bf [2012-07-17 Predrag]} Done.


\item[(0)  x ] 2012-07-01 Predrag:

\item[(3) |?|] 2012-07-01 Predrag:

\item[(4) |?|] 2012-07-01 Predrag:

\item[(4) |x|] 2012-07-01 Predrag:

\item[minor (3) |?|] 2012-07-01 Predrag:

\item[minor 6. p9 |x|] 2012-07-01 Predrag:
 referee does not understand the difference between the `method of
 co-moving frames' and the `method of connections',
 \reffig{fig:BeThMconnect}. Shouldn't we weave Evangelos clarification
 into the text proper? {\bf [2012-07-17 Evangelos]} Done.

\item[minor (9) |?|] 2012-07-01 Predrag: can you weave this into the text?: ``
\edit{
Blah, blah}
''

\item[minor (11) |?|] 2012-07-01 Predrag:

\end{description}

\subsection{Response 1.2}
\label{sect:Response1.2}

\begin{description}

\item[2012-07-14 Predrag] In response to referee's plaint:
``I described the manuscript as self-focussed because about one third of
the references are to the authors work.'' I removed a bunch of
Cvitanovi\'ciania, including Froehlich and Cvitanovi\'c (2011) paper, with
a heavy heart - Stefan did this work as undergraduate...

{\bf [2012-07-16 Evangelos]} I think you should not remove this. This is were
		slicing is formulated as an extremum condition. Also,
		I think the reviewer is pissed off about not citing (his)
		fluid dynamics papers - he does not care about this one.

{\bf [2012-07-16 Predrag]} thanks - Froelich is back.


\end{description}
