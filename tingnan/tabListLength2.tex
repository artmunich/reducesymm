%% GitHub cvitanov/reducesymm/tingnan/tabListLength2.tex

%\item[2014-05-28 Predrag] Corrected, updated  \reftab{tab:ListLength2}
%    using Tingnan's \reftab{tab:length2withfloquet}.

\begin{table}
\centering
\caption[]{
All elementary cell periodic orbits of topological length 2. Listed is
the itinerary (even direction = short jump, odd direction = long jump),
the expanding Floquet multiplier $\ExpaEig$,
the Floquet exponent $\Lyap$,
the period $\period{}$,
and
the \rpo\ shift, an integer combination of lattice vectors ? and ?. The
multiplicites can be 1 (not possible for a billiard), 2, 3, 4, 6 and 12,
depend on the symmetry of the orbit (a subgroup of the point group \Dn{6}).
Symmetry reduction to the elementary cell for the running mode \cycle{05}
is illustrated in \reffig{diskDirectionsElCell}. This solution has no
symmetry, hence its multiplicity is 12.
    }
    \begin{tabular}{l|r|l|l|l}
    \multicolumn{1}{c}{cycle}
    & \multicolumn{1}{|c|}{$\ExpaEig_p$}
      & \multicolumn{1}{c|}{$\Lyap_p$}
       & \multicolumn{1}{c|}{$\period{p}$}
          & \multicolumn{1}{c}{shift} \\
    \hline
06  & 4.539722 & 2.52144304
                 & 0.6      & (0,0)  \\
28  &          & &          & (0,0)  \\
4,10  &        & &          & (0,0) \\
    \hline
49   & 23.53082 & 1.178969
                 & 2.678876 & (0,-1) \\
18   &         & &          & (-1,1)   \\
5\underline{10} &         & &          & (1,0)  \\
4,11 &         & &          & (-1,0)  \\
3,10 &         & &          & (1,0)  \\
16 &           & &          & (0,1) \\
29 &           & &          & (0,1) \\
07 &           & &          & (1,-1) \\
6,11 &         & &          & (0,-1)  \\
05 &           & &          & (1,-1) \\
38 &           & &          & (-1,1) \\
27 &           & &          & (-1,0)  \\
    \hline
17 & 33.580486 & 0.88569725
                 & 3.967434 & (0,0) \\
39 &           & &          & (0,0) \\
5,11 &         & &          & (0,0) \\
    \hline
59   & 105.5775 & 0.945605
                  &  4.927474 & (-1,-1) \\
37   &          & &         & (-2,1) \\
3,11 &          & &         & (1,1) \\
7,11 &          & &         & (1,-2) \\
15  &           & &         & (-1,2) \\
19   &          & &         & (2,-1) \\
	\hline
    \end{tabular}
\label{tab:ListLength2}
\end{table}
