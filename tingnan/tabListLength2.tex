%% GitHub cvitanov/reducesymm/tingnan/tabListLength2.tex
%% extracted from predrag/articles/q_fredholm/qfred6.tex
%%  Predrag - final submission version			16 sep 93
%% Cvitanovic,  Rosenqvist, Rugh, Vattay submission to CHAOS J.  3/5-93

%\begin{table}
%\caption[]{ All periodic orbits of topological length 2
%(this table from \refref{FredDet} to be used as a template).
%Listed are the topological length of the cycle,
%its expanding eigenvalue  $\ExpaEig$'s, its period $\period{}$
%and its symbolic dynamics itinerary (see \reftab{TSYM}).
%  }
%\begin{tabular}{|c|r|r|r|}
%\hline
%$n_p$ & $\Lambda_p$~~~~~~~~~~~~
%                            &    $\period{p}$~~~~~~~~~
%                                                 & itinerary\\ \hline
%2 & 0.71516752438$\times 10^1$ &  0.6076252185107 & 0 \\
%2 &-0.29528463259$\times 10^1$ &  ALL FAKE NUMBERS & 1 \\
%   \hline
%2 &-0.98989794855$\times 10^1$ &  0.3333333333333 & 10 \\
%   \hline
%2 &-0.13190727397$\times 10^3$ & -0.2060113295833 & 100 \\
%2 & 0.55896964996$\times 10^2$ &  0.5393446629166 & 110 \\
%   \hline
%2 &-0.10443010730$\times 10^4$ & -0.8164965809277 & 1000 \\
%2 & 0.57799826989$\times 10^4$ &  0.0000000000000 & 1100 \\
%2 &-0.10368832509$\times 10^3$ &  0.8164965809277 & 1110 \\
%   \hline
%2 &-0.76065343718$\times 10^4$ & -1.4260322065792 & 10000 \\
%2 & 0.44455240007$\times 10^4$ & -0.6066540777738 & 11000 \\
%2 & 0.77020248597$\times 10^3$ &  0.1513755016402 & 10100 \\
%2 &-0.71068835616$\times 10^3$ &  0.2484632276044 & 11100 \\
%2 &-0.58949885284$\times 10^3$ &  0.8706954728949 & 11010 \\
%2 & 0.39099424812$\times 10^3$ &  1.0954854155465 & 11110 \\
%   \hline
%2 &-0.54574527060$\times 10^5$ & -2.0341342556665 & 100000 \\
%2 & 0.32222060985$\times 10^5$ & -1.2152504370215 & 110000 \\
%2 & 0.51376165109$\times 10^4$ & -0.4506624359329 & 101000 \\
%2 &-0.47846146631$\times 10^4$ & -0.3660254037844 & 111000 \\
%2 &-0.63939998436$\times 10^4$ &  0.3333333333333 & 110100 \\
%2 &-0.63939998436$\times 10^4$ &  0.3333333333333 & 101100 \\
%2 & 0.39019387269$\times 10^4$ &  0.5485837703548 & 111100 \\
%2 & 0.10949094597$\times 10^4$ &  1.1514633582661 & 111010 \\
%2 &-0.10433841694$\times 10^4$ &  1.3660254037844 & 111110 \\
%   \hline
%\end{tabular}
%\label{tabListLength2}
%\end{table}
\begin{table}
\centering
\caption[]{
All elementary cell periodic orbits of topological length 2. Listed is
the itinerary (even direction = short jump, odd direction = long
jump), the expanding Floquet multiplier $\ExpaEig$, the period $\period{}$, and
the \rpo\ shift, an integer combination of lattice vectors ? and ?. The multiplicites can be 1 (not possible for a billiard),
2, 3, 4, 6 and 12, depend on the symmetry of the orbit (a subgroup of
the point group \Dn{6}). }
    \begin{tabular}{l|r|l|l}
    \multicolumn{1}{c}{cycle} &
    \multicolumn{1}{|c|}{$\ExpaEig_p$} & \multicolumn{1}{c|}{$\period{p}$}
     & \multicolumn{1}{|c}{shift} \\
    \hline
06  & 4.539722 & 0.6      & (0,0)  \\
28  &          &          & (0,0)  \\
4,10  &        &          & (0,0) \\
    \hline
49   & 22.178242 & 2.678876 & (0,-1) \\
18   &         &          & (-1,1)   \\
5,10 &         &          & (1,0)  \\
4,11 &         &          & (-1,0)  \\
3,10 &         &          & (1,0)  \\
16 &           &          & (0,1) \\
29 &           &          & (0,1) \\
07 &           &          & (1,-1) \\
6,11 &         &          & (0,-1)  \\
05 &           &          & (1,-1) \\
38 &           &          & (-1,1) \\
27 &           &          & (-1,0)  \\
    \hline
17 & 33.580486 & 3.967434 & (0,0) \\
39 &           &          & (0,0) \\
5,11 &         &          & (0,0) \\
    \hline
59   & 105.577493 & 4.927474 & (-1,-1) \\
37   & 105.577496 &          & (-2,1) \\
3,11 & 105.577497 &          & (1,1) \\
7,11 & 105.577497 &          & (1,-2) \\
15  &  105.577497 &          & (-1,2) \\
19   & 105.577501 &          & (2,-1) \\
	\hline
    \end{tabular}
\label{tab:ListLength2}
\end{table}

\begin{table}[htbp]
  \centering
  \caption{Comparison between Tingnan's numerical program and
  the ChaosBook analytic formula
  (\HREF{http://www.streamsound.dk/book1/chaos/chaos.html\#300/z}
  {click here}), for cycle type $\bar{0}$. $\Delta\Lambda_e$ and $\Delta t_p$ show the differences of expanding eigenvalue $\ExpaEig$'s and  the period $\period{}$'s}
    \begin{tabular}{|r|r|rr|r|r|}
	\hline
    $\Lambda_e$ & $t_p$    & \multicolumn{2}{c|}{disklabel} &\multicolumn{1}{c|}{ $\Delta\Lambda_e$} & \multicolumn{1}{c|}{$\Delta t_p$} \\\hline
    4.539722204 & 0.600000 & 0     & 6     &  0     & 0 \\
    4.539722204 & 0.600000 & 2     & 8     &  0     & 0 \\
    4.539722204 & 0.600000 & 4     & 10    &  0     & 0 \\
    33.580485941 & 3.967434 & 1     & 7     &  7.11E-15 & 5.33E-15 \\
    33.580485941 & 3.967434 & 3     & 9     &  7.11E-15 & 5.33E-15 \\
    33.580485941 & 3.967434 & 5     & 11    &  7.11E-15 & 5.33E-15 \\
    \hline
    \end{tabular}%
  \label{tab:comparison0}%
\end{table}%
\begin{table}[htbp]
  \centering
  \caption{Comparison between my numerical program and chaosbook formula  (\HREF{http://www.streamsound.dk/book1/chaos/chaos.html\#300/z}
    {click here}), for cycle type $\bar{1}$. $\Delta\Lambda_e$ and $\Delta t_p$ show the differences of expanding eigenvalue $\ExpaEig$'s and  the period $\period{}$'s}
    \begin{tabular}{|r|r|rrr|r|r|}
    \hline
        $\Lambda_e$ & $t_p$    & \multicolumn{3}{c|}{disklabel} &\multicolumn{1}{c|}{ $\Delta\Lambda_e$} & \multicolumn{1}{c|}{$\Delta t_p$} \\\hline
    26.34522 & 1.703848 & 0     & 4     & 8     & 4.30E-08 & 2.00E-15 \\
    26.34522 & 1.703848 & 2     & 6     & 10    & 5.40E-08 & 2.00E-15 \\
	\hline
    \end{tabular}%
  \label{tab:comparison1}%
\end{table}%
