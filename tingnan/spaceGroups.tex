% GitHub cvitanov/reducesymm/tingnan/spaceGroups.tex

% Predrag                                       2014-07-14
	
\chapter{Notes on space groups}
\label{c-spaceGroups}




\begin{description}

\item[2014-07-14 Predrag]
Separated out here our notes on condensed matter theorists'
standard version of space groups (the one we are too lazy to learn)

\item[2014-04-18 Predrag]
Dresselhaus \etal\ textbook\rf{Dresselhaus07}
(\HREF{http://chaosbook.org/library/Dresselhaus07.pdf}{click here})
is good on discrete
and space (but not continuous) groups.
The MIT~course~6.734
\HREF{http://stuff.mit.edu/afs/athena/course/6/6.734j/www/group-full02.pdf}
{online version} contains much of the same material.

Chapter {\em 9. Space Groups in Real Space} is quite clear on matrix
representation of space groups. The translation group $T$ is a normal
subgroup of \Group\ and defines the Bravais lattice. The cosets by
translation $T$ (set all all group elements obtained by all translations)
form a factor group $\Group/T$, isomorphic to the point group $g$
(rotations). All irreducible representations of \Group\ can be compounded
from irreducible representations of $g$ and $T$.

Section {\em 9.3 Two-Dimensional Space Groups}: In the international
crystallographic notation, our hexagonal lattice \#17 is called $p6mm$,
with point group $6mm$.
\beq
\Group = \{
E, C_6^+, C_6^-, C_3^+, C_3^-, C_2,
\sigma_{d1}, \sigma_{d2}, \sigma_{d3},
\sigma_{v1},\sigma_{v2}, \sigma_{v3}
\}
\ee{D12generators}
Prefix $p$ indicates that the unit cell is primitive (not centered). This
is a simple or {\em symmorphic} group, which makes calculations easier.
The Bravais lattice is two equilateral triangles, not sure how to relate
it to our hexagonal `elementary cell'? A Brilloun zone? Bravais `unit cell'
is illustrated in Fig.~E.2. ChaosBook `Fundamental
domain' makes an appearance in Fig.~10.2.

The main trick in quantum-mechanical calculations is to go to the
\emph{reciprocal} space (see Fig.~E.2), in our case with the full
$\Gamma$ point, $k=0$, wave vector symmetry (see Table~10.1), and `Large
Representations'. This is something we have not tried in deriving the
trace formula for deterministic diffusion.

Sect. {\em 10.5 Characters for the Equivalence Representation} look
like those for the point group, sort of. We should probably work
out problems 10.1 and 10.2.

\item[2014-04-26 Predrag] We should read Chapter 7 of Gaspard\rf{PG97}.
Kimberly and I both have a hard copy. Gaspard discusses irreps of
the translation group in some detail.

\item[2014-04-26 Predrag]
I'm quite convinced that this problem is a solved problem, we just have
to understand how characters are used to project irreps of space groups.
One has to go to the reciprocal lattice, and utilize utilize the concept
of the `star'. All physical chemists and crystallographers know how to do
this - we just need to be good students and read the stuff. Our case is
the most symmetric, $p6mm$ lattice - it is surely worked out in some
paper in a way that we can understand. They call our `fundamental domain'
the `motif' or the `asymmetric unit'.

I found projects in \HREF{http://www-f1.ijs.si/~ziherl/SF.html} {this
course} easy to read, especially Toma\v{z} \v{C}endak, who reviews the
space groups theory in a pretty simple way, and Zavadlav has very pretty
wallpaper groups illustrations. Ziherl recommends Elliott and  Dawber,
{\em Symmetry in Physics}\rf{ELLIOTT}.

Joseph Sidighi likes Cotton\rf{Cotton08} {\em Chemical applications of
group theory}, which has no characters for space groups, but a very
pretty discussion of their geometry in Chapt.~11. Cotton was ``the most
influential inorganic chemist to ever have lived.''

If we succeed in factorization, this would merit a publication.
It is OK if you do not succeed in factorization - I have failed myself, so
who am I to cast the first stone:)

\item[2014-05-05 Tingnan] My numbers of elementary cell prime cycles,
Lyapunov exponents and diffusion coefficients are listed and compared
to the Schreiber calculation in \reftab{TCELL1}. I still have the problem
for the convergence of diffusion coefficients. Did we miss a factor of 2
somewhere?

\item[2014-05-28 Tingnan] I got it!

I see the trouble more clearly as indicated in the previous blogs: The
averaging over the space group (a point group action followed by a
translation). I think in Pavel's approach we still needs to worry
because his three group actions do not commute with each other?

Let us speculate a bit about the group actions in the lattice group. We
will use the same notation as in Dresselhaus \etal\
textbook\rf{Dresselhaus07},
\[
	\mathbf{a} = \{R_g\vert\tau\}
\,
\]
where $g$ is an element of the point group and $\tau$ is a translation.

%A potentially useful commutation relation is posted here (9.15):
%\[	\{R_g\vert\tau\}\{\epsilon\vert\tau\}\{R_g\vert\tau\}^{-1} = \{\epsilon\vert R_gt\}
%\]

Since our group is $p6mm$ and is symmorphic, the irreducible representation
of the group (of the wave vector) is simple (10.35),
\[
D_{k}^{\Gamma_{i}}(\{R_{g}\vert\mathbf{R}_{n}\})=e^{i\mathbf{k\cdot R}_{n}}D^{\Gamma_{i}}(R_{g})
\,
\]
where $D^{\Gamma_{i}}(R_{g})$ is the irreps of the point group $C_{6v}$,
which can be readily found (\HREF{http://www.cryst.ehu.es}{click here}).

\item[2014-06-03] Finally I got Gaspard's book from the GIL service. Here
are updates on reciprocal lattice.

Suppose we have $M$ dimensional phase space flow $f^{t}(\mathbf{X})$
with $L<M$ position coordinates. The equation of motion is invariant
under a space group $G$. Let us first examine the translation subgroup
$T$ of $G$.

With a proper choice of {\Poincare} section $\mathcal{P}$, the flow
$f^{t}(\mathbf{X})$ can be expressed as a suspended flow $F^{t}(\mathbf{x},\tau,\mathbf{l})$
(Gaspard 7.8), where $\mathbf{x}\in\mathcal{P}$ is on the section,
$\tau\in[0,T(\mathbf{x}))$ the time interval before first return,
and $\mathbf{l}$ a spatial vector denotes the center of the element
cell the current point belongs to. The flow is controlled by a set
of mappings:

\[
\begin{cases}
\mathbf{x}_{n+1}=\phi(\mathbf{x}_{n}) & \mathbf{x\in\mathcal{P}}\\
t_{n+1}=t_{n}+T(\mathbf{x}_{n})\\
\mathbf{l}_{n+1}=\mathbf{l}_{n}+a(\mathbf{x}_{n}) & a(\mathbf{x}_{n})\in\mathcal{L}
\end{cases},
\]
where we denote the Bravais lattice $\mathcal{L}$. We can relate
the mappings with the flow:
\begin{align*}
F^{t}(\mathbf{x},\tau,\mathbf{l}) & =\left\{ \phi^{n}\mathbf{x},\tau+t-\sum_{j=0}^{n-1}T(\phi^{j}\mathbf{x}),\mathbf{l}+\sum_{j=0}^{n-1}a(\phi^{j}\mathbf{x})\right\} ,\\
 & \qquad\mathrm{for}\qquad0\leq\tau+t-\sum_{j=0}^{n-1}T(\phi^{j}\mathbf{x})<T(\phi^{n}\mathbf{x}).
\end{align*}


We can define invariant measures based on the special coordinates
$(\mathbf{x},\tau,\mathbf{l})$ (Gaspard 7.20-7.21):
\begin{align*}
\left\langle A(\mathbf{X})\right\rangle  & =\sum_{\mathbf{l}\in\mathcal{L}}\int\mu(d\mathbf{x}d\tau)A(\mathbf{x},\tau,\mathbf{l})\\
 & =\sum_{\mathbf{l}\in\mathcal{L}}\frac{1}{\vert\mathcal{P}\vert}\int_{\mathcal{P}}d\mathbf{x}\int_{0}^{T(\mathbf{x})}\frac{d\tau}{\left\langle T\right\rangle _{\nu}}A(\mathbf{x},\tau,\mathbf{l})\\
 & =\sum_{\mathbf{l}\in\mathcal{L}}\frac{1}{\left\langle T\right\rangle _{\nu}}\left\langle \int_{0}^{T(\mathbf{x})}d\tau A(\mathbf{x},\tau,\mathbf{l})\right\rangle _{\nu}
\end{align*}
where the subscript $\nu$ denotes integration over the volume of
the {\Poincare} section.

Now with some quantities defined let us proceed to the fourier transforms
of the quantities we are interested. Define the projection operators by
\begin{align*}
\hat{E}_{\mathbf{k}} & =\sum_{\mathbf{R}_{n}}e^{-i\mathbf{k}\cdot\mathbf{R}_{n}}\hat{P}_{\{\varepsilon\vert\mathbf{R}_{n}\}},
\end{align*}
in terms of spatial translation operators
\[
\hat{P}_{\{\varepsilon\vert\mathbf{R}_{n}\}}\phi(\mathbf{r})=\phi(\mathbf{r}+\mathbf{R}_{n}),
\]
with $\mathbf{R}_{n}$ is the Bravais lattice vector, and $\mathbf{k}\in\mathcal{B}$,
the first Brillouin zone. For an infinite lattice $\mathbf{k}$ is
continuous. We first needs to check and orthogonality and completeness
of the projection operators (Gaspard 7.30-7.32) :
\begin{align*}
\hat{E}_{\mathbf{k}}\hat{E}_{\mathbf{k^{\prime}}} & =\sum_{\mathbf{R}_{n}}e^{-i\mathbf{k}\cdot\mathbf{R}_{n}}\hat{P}_{\{\varepsilon\vert\mathbf{R}_{n}\}}\sum_{\mathbf{R}_{m}}e^{-i\mathbf{k^{\prime}}\cdot\mathbf{R}_{m}}\hat{P}_{\{\varepsilon\vert\mathbf{R}_{m}\}}\\
 & =\sum_{\mathbf{R}_{n}\mathbf{R}_{m}}e^{-i(\mathbf{k}\cdot\mathbf{R}_{n}+\mathbf{k}^{\prime}\cdot\mathbf{R}_{m})}\hat{P}_{\{\varepsilon\vert\mathbf{R}_{n}+\mathbf{R}_{m}\}}\\
 & =\sum_{\mathbf{R}_{m}}\sum_{\mathbf{R}_{l}}e^{-i(\mathbf{k}\cdot(\mathbf{R}_{l}-\mathbf{R}_{m})+\mathbf{k}^{\prime}\cdot\mathbf{R}_{m})}\hat{P}_{\{\varepsilon\vert\mathbf{R}_{l}\}}\\
 & =\sum_{\mathbf{R}_{m}}e^{-i(\mathbf{k}^{\prime}-\mathbf{k})\cdot\mathbf{R}_{m}}\sum_{\mathbf{R}_{l}}e^{-i\mathbf{k}\cdot\mathbf{R}_{l}}\hat{P}_{\{\varepsilon\vert\mathbf{R}_{l}\}}\\
 & =\hat{E}_{\mathbf{k}}\sum_{\mathbf{R}_{m}}e^{-i(\mathbf{k}^{\prime}-\mathbf{k})\cdot\mathbf{R}_{m}}\\
 & =\hat{E}_{\mathbf{k}}\vert\mathcal{B}\vert\delta(\mathbf{k}^{\prime}-\mathbf{k})
\end{align*}

\begin{align*}
\frac{1}{\vert\mathcal{B}\vert}\int_{\mathcal{B}}\hat{E}_{\mathbf{k}}d\mathbf{k} & =\frac{1}{\vert\mathcal{B}\vert}\sum_{\mathbf{R}_{n}}\hat{P}_{\{\varepsilon\vert\mathbf{R}_{n}\}}\int_{\mathcal{B}}e^{-i\mathbf{k}\cdot\mathbf{R}_{n}}d\mathbf{k}\\
 & =\frac{1}{\vert\mathcal{B}\vert}\sum_{\mathbf{R}_{n}}\hat{P}_{\{\varepsilon\vert\mathbf{R}_{n}\}}\vert\mathcal{B}\vert\delta_{\mathbf{R}_{n},0}\\
 & =\hat{P}_{\{\varepsilon\vert0\}}\\
 & =\hat{I}.
\end{align*}

to be continued.

\end{description}
