% reducesymm/QFT/dailyBlogKS.tex
% Predrag  created              Sep 2 2013
% continues siminos/blog/dailyBlog.tex as of that date

\chapter{Kamal's Gribov term paper}
\label{c-dailyBlogKS}

\begin{description}
\item[2013-11-25  Predrag to Kamal] Created this for you to blog
your QFT final paper.

\item[2013-10-12  Kamal] I will be away during the month of January for my wedding.
I will be working from home and updating through GitHub.
Please let me know if you have anything you want me to take care of.
Further, it will be great for me if you could send me the material
for the QFT final project. As you might have experienced, speed is
an issue with my working on new stuff so I want to start as early
as possible to get something useful out of it given the time constraints.
Also the TA duties are killing me. I am working on it and want to be productive fast.

\item[2013-10-17 Predrag to Kamal] I was thinking that it might be useful
to learn about the \emph{`Gribov ambiguity'}. My notes are in
\refchap{c-Gribov}. Have a look... The posts there summarize all my notes
about the \emph{`Gribov ambiguity'}. I fear it is too difficult, starting
with your current background in QFT, but it is one of directions in which
our current symmetry-reduction work might develop in the future...

So, instead of the final exam in the Quantum Field Theory Course, please
some of the references I have listed (or, better still, find more accessible
references by web searches) and write up here, as you learn, about 10-15
pages of what you have learned. The delivery deadline is

\begin{center}
{\large December 10, 2013 14:20pm}
\end{center}
\clearpage

\begin{center}
{\large Introduction to Gribov Ambiguity}
\end{center}
1.\textbf{GribovAmbiguity}
All modern theories trying to explain fundamental physics are field theories which describe the configuration of a field and its dynamics as the interaction between fields and their evolution in space and time. These fields may be scalar, vector or tensor fields and they transform under a gauge transformation accordingly to give different configurations of the field. However, for some of these configurations the physical observables, which are in a direct way the reality that we perceive/observe, do not change under gauge transformations. For example, in electromagnetism, the following transformations,
\begin{subequations}\begin{align}
 \bold{A} &\rightarrow \bold{A} + \nabla \psi  \\
 \phi &\rightarrow \phi - \frac{\partial \psi}{\partial t}\end{align} \end{subequations}
keep the observables $\bold{E}$ and $\bold{B}$ (electric and magnetic fields respectively) unchanged. This is an example of a gauge transformation.

\textbf{What is gauge freedom and gauge fixing?}
The invariance of physically observables quantities with respect to a gauge transformation implies that the system has redundant degrees of freedom in field variables. All field configurations that transform into one another through gauge transformations are physically equivalent and, therefore, for correct predictions, should be counted as one. Gauge fixing is the mathematical procedure of selecting an equivalence class for each set of physically identical field configurations. A coherent and consistent prescription of selecting the representative configurations (also known as gauge fixing) out of all possible detailed configurations is required to make 




 \textbf{What is a gauge theory?}
Gauge theory is a field theory in which the lagrangian is equivariant under a lie group of continuous transformations. Such a group of transformations is called a symmetry group of the theory. A gauge field is a vector field associated with the generators of the symmetry group and the particles that arise due to quantization of the gauge field are called gauge bosons.








\textbf{What is gauge fixing?}



\textbf{What is Gribov ambiguity?}

2. Details
Faddeev-Popov operator
Has all eigenvalues positive in the region called the Gribov region while it has at least on vanishing eigenvalue at the gribov horizon.
Gribov region or fundamental modular region
Gribov horizon

3.Confinement ?


\end{description}
\renewcommand{\ssp}{a}
