% reducesymm/QFT/dailyBlogKS.tex
% Predrag  created              Sep 2 2013
% continues siminos/blog/dailyBlog.tex as of that date

\chapter{Gribov ambiguity}
\label{c-dailyBlogKS}

\noindent
Kamal Sharma\\final term paper for Fall 2013 QFT course.
\\\\


\section{Gauge theories}
All modern theories trying to explain fundamental physics are field theories which describe the configuration of a field and its dynamics as the interaction between fields and their evolution in space and time. These fields may be scalar, vector or tensor fields and they transform under a gauge transformation accordingly to give different configurations of the field. However, for some of these configurations the physical observables, which are in a direct way the reality that we perceive/observe, do not change under gauge transformations. For example, in electromagnetism, the following transformations,
\begin{subequations}\begin{align}
 \bold{A} &\rightarrow \bold{A} + \nabla \psi  \\
 \phi &\rightarrow \phi - \frac{\partial \psi}{\partial t}\end{align} \end{subequations}
keep the observables $\bold{E}$ and $\bold{B}$ (electric and magnetic
fields respectively) unchanged. This is an example of a gauge
transformation.

\paragraph{What is gauge freedom and gauge fixing?}
The invariance of physically observables quantities with respect to a
gauge transformation implies that the system has redundant degrees of
freedom in field variables. All field configurations that transform into
one another through gauge transformations are physically equivalent and,
therefore, for correct predictions, should be counted as one. Gauge
fixing is the mathematical procedure of selecting an equivalence class
for each set of physically identical field configurations. A coherent and
consistent prescription of selecting the representative configurations
(also known as gauge fixing) out of all possible detailed configurations
is required to make
    \PC{make what?}


 \paragraph{What is a gauge theory?}
According to \PCedit{(symmgauge freedom)} there are two types of theories
that can be called \lq gauge theories\rq, the Yang-Mills theories and
constrained Hamiltonian theories. The Hamiltonian theories subsume the
Yang-Mills theories. Such theories have a common striking feature known
as gauge freedom. Gauge freedom is a fancy way of saying that the
theory has two kinds of variables -- physical and unphysical-- due to
which the initial value problem in such theories is ill-defined. This
means that a set of initial conditions does not uniquely determine the
evolution of all the dynamical variables of the theory. The set of
`physical' variables will evolve the same way but the `unphysical'
variables can evolve in infinite number of arbitrary ways thus allowing
infinite number of solutions for the same initial conditions. It is
possible to interpret classical gauge theories as deterministic only if
we consider the physical dynamical variables as the complete description
of physical reality. If two states differ only in their unphysical
variables, they represent the same physical configuration of the field.

Speaking in mathematical dialect, a gauge theory is a field theory in
which the Lagrangian is invariant under a Lie group of continuous
transformations. Such a group of transformations is called a symmetry
group of the theory. A gauge field is a vector field associated with the
generators of the symmetry group and the particles that arise due to
quantization of the gauge field are called gauge bosons.

\paragraph{Consequences of gauge symmetry}
The presence of gauge symmetry has significant consequences on the
results of the theory. Let us consider a free non-relativistic particle
in one dimensional space. The Hamiltonian of such a system is given just
by the kinetic energy of the particle $ H= -\frac{1}{2}
\frac{d^{2}}{dx^{2}} $. The energy eigenvalue spectrum of this problem is
continuous in the absence of symmetry. In the case of a periodic boundary
condition when the points $x$ and $ x+pL, p \epsilon I $ are physically
identified with each other, the wavefunction has a gauge symmetry
$\Psi(x)=\Psi(x+pL)$. The spectrum now becomes discrete and this is how
gauge symmetry affects the physical configuration space. This is a simple
example which illustrates that a gauge theory must obey some constraints
that identify the physically identical configurations and that it makes
the spectrum more restrictive.

\paragraph{Abelian and non-Abelian gauge theories}
The continuous symmetry operations on the Lagrangian which keep the action invariant constitute a Lie group. The generators of infinitesimal transformations of such a Lie group define the algebra of such a symmetry group. If the generators of the gauge symmetry group commute then the theory is said to be Abelian, otherwise non-Abelian.

\paragraph{Gribov copies}
In 1978, Gribov\rf{Gribov77} showed that for a non-Abelian theory, for
example SU(2) and SU(3), the local gauge group imposes more stringent
constraints than it does on an Abelian gauge theory. The
\emph{transversality condition} $ \partial \cdot A=0$ fixes the gauge
uniquely for Abelian gauge theories but in the case of non-Abelian gauge
theories, there exist distinct phase space configurations $A$ and $A'$
related by a finite gauge transformations $A' = U A$ such that
$\partial\cdot A=0$ and $\partial\cdot A'=0$ where $A \neq A'$. These
distinct configurations are called Gribov copies and in a non-Abelian
gauge theory they have an additional constraint of being physically
identical. The subspace of the full state space that contains only
physically distinct configurations of the field is called a fundamental
modular region and is free of any Gribov copies. In 1978,
Singer\rf{Singer78} showed that in non-Abelian theories Gribov copies are
unavoidable and that the physical configuration space is topologically
non-trivial. On the other hand, for an Abelian theory the physical
configuration space is a linear vector space.

\paragraph{Quantum ChromoDynamics and confinement}
Quantum ChromoDynamics is a theory which describes the strong
interactions at sub-atomic level. Strong interaction is the force between
quarks and gluons. At very high energies these sub-atomic particles are
asymptotically free which means that they behave like free particles. But
these particles have not been observed. This is because in low energy
conditions they interact with each other and form bound states called
hadrons, such as proton and neutron. This phenomenon is called
\emph{confinement}.

In spite of QCD being qualitatively similar to QED, the problem of
confinement is not very well understood. In QED, the method of
perturbative expansion using Feynman diagrams has proved very successful
because the small value of the QED coupling constant makes the
contribution of higher terms more and more negligible. In QCD the value
of the coupling constant is a function of energy and becomes larger as
the energy is lowered. This phenomena called `infrared slavery" is
responsible for the failure of perturbation theory for low energy
phenomena. For understanding the low energy behaviour of the theory,
various non-perturbative methods have been developed to describe
confinement. Different methods work well in different conditions and so
QCD is a patchwork of different methods that work in different
conditions.

\paragraph{Dynamical implications of non-Abelian nature of gauge symmetry
group}
The existence of Gribov copies and suppression of infrared modes due to
their closeness to Gribov horizon leads to an interesting feature: The
calculation of the gluon propagator under a non-Abelian theory leads to
expulsion of the gluon from the physical spectrum of the solution. This
is seen as non-existence of a physical pole in the gluon propagator.

\paragraph{Gauge orbit and physical configuration}
Each physical configuration of the field $ A_{phys} $ is associated with
a corresponding gauge orbit which is collection of all physically
identical configurations. The physicalconfiguration space is the space of
all gauge orbits modulo the group of local gauge transformations
$\mathcal{G}={U}$, $$P=\mathcal{A}/\mathcal{G}.$$


\section{Yang-Mills theory}
As QCD is a specific case of a general Yang-Mills theory, it is a good
example of the general theory. Consider the compact group $SU(N)$ of
[$N\!\times\!N$]
unitary matrices $U$ of determinant one. These matrices can be expressed
as
\[
U = \exp(-ig\theta_{a}X_{a})
\,,
\]
where $X^{a}$ are the generators of
SU(N) group and satisfy commutation relations
\[
[X^{a},X^{b}]=i f_{abc}X^{c}
\,.
\]
These generators are defined to be hermitian and normalizable
as follows: $$X^{\dagger}=X,$$ $$Tr[X_{a}X_{b}]=\frac{\delta_{ab}}{2}.$$
Now, the generators $X_{a}$ belong to the adjoint representation of the
group $SU(N)$, i.e. $$U X_{a}U^{\dagger}=X_{b} (D^{A})_{ba},$$ with
$(D^{A}(X_{a}))_{bc}=-if_{abc}$. Here $f^{abc}$ are the structure
constants of $SU(N)$ and have the following property, $$f^{abc}f^{dbc}=N
\delta^{ad}.$$

We can construct a Lagrangian which by design would be invariant under the above defined $SU(N)$ group. The Yang-Mills action for this lagrangaian would be $$S_{YM}=\int d^{4}x \frac{1}{2} Tr F_{\mu\nu}F_{\mu\nu},$$ whereby $F_{\mu\nu}$ is the field strength $$ F_{\mu\nu}=\partial_{\mu}A_{\nu}-\partial_{\nu}A_{\mu}-ig[A_{\mu},A_{\nu}],$$ and $A_{\mu}$ are the gluon fields that belong to adjoint representation of $SU(N)$ symmetry, i.e. $$A_{\mu}=A^{a}_{\mu}X^{a}.$$ The field strength is given by $$F_{\mu\nu}=\partial_{\mu}A^{a}_{\mu}+gf_{akl}A^{k}_{\mu}A^{l}_{\nu}.$$ \ $A_{\mu}$ under the $SU(N)$ symmetry transforms as $$ A'_{\mu}=UA_{\mu}U^{\dagger}-\frac{i}{g}(\partial_{\mu}U)U^{\dagger}.$$ We find that $$F'_{\mu\nu}=U F_{\mu\nu}U^{\dagger},$$ and can now see that Yang-Mills action is invariant under $SU(N)$ symmetry. The infinitesimal transformations can thus be written as $$\delta A^{a}_{\mu}=-D^{ab}_{\mu}\theta^{b},$$ with $D^{ab}_{\mu}$ the covariant derivative in the adjoint representation $$ D^{ab}_{\mu}=\partial_{\mu}\delta^{ab}-g f^{abc}A^{c}_{\mu}.$$ There is also a matter part of the action but we will work with pure Yang-Mills action.

\paragraph{Faddeev-Popov Ghosts}
Faddeev-Popov ghosts or ghost fields are additional fields which are introduced into gauge field theories to maintiain the consistency of path integral formulation. In order for the quantum field theories to deliver unambiguous and sensible results we need to avoid overcounting of feynman diagrams that correspond to physically equivalent processes. In a gauge field theory, each physical configuration has infinite number of full state space configurations all of which lie on a gauge orbit. Selecting a representative configuration from this equivalence class is required in order for path integral method to work. But usually there is no such prescription of selecting such a representative configurations. However, it is possible to modify the action by adding extra terms called \emph{ghost-fields} that break gauge symmetry. In general, ghost fields can add or break gauge symmetry in a field theory. This method is called Faddeev-Popov procedure. Ghost fields are mathematical tools and represent virtual particles in feynman diagrams. They are also essential for unitarity.

\paragraph{Faddeev-Popov Quantization}
To understand the Gribov problem we first need to have a look at
Faddeev-Popov quantization \PCedit{(reference 53)}. The Yang-Mills action
 is given by
\[
S_{YM}= \frac{1}{4}\int d^{d}x
F^{a}_{\mu\nu}F^{a}_{\mu\nu}
\,.
\]
Our naive assumption that the
generating functional $Z(J)$ would be given by $$Z(J)=\int
[dA]exp[-S_{YM}+\int dx J^{a}_{\mu}A^{a}_{\mu}].$$ But this functional is
not well defined. We can look at the quadratic part of the action,
$$Z(J)_{quadr}=\int[dA] exp[-\frac{1}{4}\int dx
(\partial_{\mu}A_{\nu}(x)-\partial_{\nu}A_{\mu}(x))^{2}+\int dx
J^{a}_{\mu}(x)A^{a}_{\mu}(x)$$ $$ =\int[dA] exp[-\frac{1}{2}\int dx dy
A^{a}_{\nu}(x)[\delta^{ab} \delta (x-y)(\partial^{2}
\delta_{\mu\nu}-\partial_{\mu}\partial_{\nu})A^{b}_{\mu}(y))+\int dx
J^{a}_{\mu}(x)A^{a}_{\mu}(x) $$ which after a gaussian integration gives
$$ Z(J)_{quadr}=(detA)^{-\frac{1}{2}} \int[dA] e^{-\frac{1}{2}\int dx dy
J^{a}_{\nu}(x) A_{\mu\nu}(x,y)^{-1} J^{a}_{\mu}(y)},$$ with
$A_{\mu\nu}(x,y)=
\delta(x-y)(\partial^{2}\delta{\mu\nu}-\partial_{\mu}\partial_{\nu})$,
is ill defined because $A_{\mu\nu}(x,y)$ is not invertible. There is
something wrong with the generating functional.

Following the derivation of gauge fixed action given in
\PCedit{(ref 16)}:
\[
S=S_{YM}+\int dx (\bar{c}\partial_{\mu}D^{ab}_{\mu}c^{b}
 -\frac{1}{2 \alpha}(\partial_{\mu}A^{a}_{\mu})^{2})
\,.
\]
If we take the limit $\alpha \rightarrow 0$, we have the Landau Gauge which stays a fixed point under normalization. If $\alpha=1$, it is called the Feynman gauge in which the gluon propagator is the simplest.

\paragraph{What is BRST symmetry?}
Fixing the gauge cause the local gauge symmetry to break. However, after fixing the gauge, a new symmetry called the BRST symmetry appears which is basically the symmetry of the g{ab}host fields.For example, inserting a \emph{b-field} $$ S=S_{YM} + \int d^{d}x (b^{a} \partial_{\mu} A_{\mu} +\alpha \frac{b^{a})^2}{2} + \bar{c}^{a}\partial_{\mu}D^{ab}_{\mu} c^{b}),$$ where Z(J) is now given as $$Z(J)=\int[dA][dc][d\bar{c}][db] e^{[-S+\int dx J^{a}_{\mu{a}=0}A^{a}_{\mu}]}.$$ Here \emph{b} is the bosonic field. The action for this theory has a new symmetry called the BRST symmetry, $$sS=0,$$ with $$s A^{a}_{\mu}=-(D_{\mu}c)^{a}, sc^{a}=\frac{1}{2}gf^{abc}c^{b}c^{c},$$\\ $$s\bar{c}^{a}=b^{a}, sb^{a}=0 $$ \\
$$s\bar{\psi}_{\alpha}=-i g c^{a} (X^{a})^{ij} \psi^{j}_{\alpha}, s\bar{\psi}^{i}_{\alpha}=-i g \psi^{j}_{\alpha} c^{a} (X^{a})^{ji}.$$
This BRST symmetry property is the proof that Yang-Mills theory is unitary in perturbation theory. Introduction of BRST symmetry introduces extra particles called ghost particles $c$ and $\bar{c}$. Like other ghost particles, these particles too violate spin-statistics thorem.

\section{Gribov ambiguity}

For any kind of gauge orbit the gauge fixing condition might have one, more or no solutions, i.e. the slice might intersect the gauge orbit once, more than once or never. Consider two Gribov copies $A_{\mu}$ and $A'_{\mu}$ related by a gauge transformation $$ A'_{\mu}=U A_{\mu} U^{\dagger}-\frac{i}{g}(\partial_{\mu}U)U^{\dagger},$$ which obviously satisfy the transversality condition $$\partial_{\mu}A_{\mu}=0 \ \partial_{\mu}A'_{\mu}=0. $$ These equations when combined and expanded to first order gives $$ -\partial_{\mu}(\partial_{\mu}\alpha+i g [\alpha,\partial_{\mu}])=0 $$ which is equivalent to $$-\partial_{\mu}D_{\mu} \alpha  =0.$$ This means that the relevant Gribov copies are in a space orthogonal to the trivial null space. The transversality condition, $\partial_{\mu}A_{\mu}$, also implies that Faddeev-Popov operator is Hermitian, $$-\partial_{\mu}D_{\mu}=\partial_{\mu}D_{\mu}.$$ Thus existence of Gribov copies are connected to zero eigenvalues of the Faddeev-Popov operator.

\paragraph{Important observation}
For small $A_{\mu}$, the equation reduces to the eigenvalue equation $$-\partial^{2}_{\mu}\psi=\epsilon \psi,$$ has positive eigenvalues but this cannot be gauranteed for large $A_{\mu}$. This means that for large $A_{\mu}$ the eigenvalues of Fadeev-Popov operator are zero.


\paragraph{Possible Solutions?}
\textbf{Gribov Region and Gribov Horizon}\\
We need to improve the gauge fixing for non-Abelian theories. This can
be done by finding the Gribov region $\Omega$ which is defined as
subspaces with positive eigenvalues of Faddeev-Popov operator, $$\Omega
= {A^{a}_{\mu},\partial_{\mu}A^{a}_{\mu}=0, \mathcal{M}^{ab}>0},$$
where $\mathcal{M}$ is the Faddeev-Popov operator,
$$\mathcal{M}^{ab}(x,y)=-\partial_{\mu}D^{ab}_{\mu}\delta(x-y).$$ This is
the region which obeys Landau gauge and where the FP operator is positive
definite. The border of the Gribov region is the manifold where the first
eigenvalue of the FP operator becomes zero. This is known as Gribov
horizon. The eigenvalues become negative on the other side of the Gribov
horizon.

Another way of choosing a Gribov region is to select those points on the gauge orbit which have minimum $A^{2}$. This definition also agrees with our previous definiton on Gribov region.

\paragraph{Properties of the Gribov region}
\begin{enumerate}
  \item
It is important that each group orbit passes through the Gribov region
because we want to take into account all possible physical
configurations. Gribov showed that for every configuration
infinitesimally close to the Gribov horizon, there exists a Gribov copy
on the other side of the horizon infinitesimally close the horizon. It
has been rigorously proved that every gauge orbit passes through the
Gribov region.
  \item
Gribov region is a convex manifold.\PCedit{(reference 12)}
  \item
Gribov region is bounded in every direction.
\end{enumerate}
Unfortunately, despite of all the nice properties, it has been shown
\PCedit{(reference 86)} that Gribov region still contains Gribov copies.

\paragraph{Another possible solution: Fundamental Modular Region}
Now when even the Gribov region has Gribov copies, let us define a
fundamental modular region as a more restrictive subspace of all the
configurations which have all absolute minima of the functional. We
shall select only the configurations closest to the region. This is
called the fundamental modular region or the minimal Landau gauge.

\paragraph{Properties of FMR}
\begin{enumerate}
  \item
All gauge orbits intersect with FMR
  \item
$ A_{\mu}=0 $ belongs to FMR as 0 is the smallest norm.
  \item
FMR is convex and bounded in every direction.
  \item
The boundary of $\Lambda$, $\delta \Lambda$ has some points common with
the Gribov horizon.
  \item Gribov copies exist on the boundary.
\end{enumerate}

\paragraph{Other attempts at gauge fixing}
Singer showed that suitable regularity conditions at infinity does not
leave any continuous gauge choices. This means that there is no unique
representative of the gauge orbit that is continuous in the space of
gauge orbits. A gauge free of Gribov copies is a singular gauge and very
difficult to handle in computations and also violates Lorentz
invariance. In 2005, Ghiotti, Kalloniatis, and Williams tried to
improve the Fadeev-Popov gauge fixing by including the determinant into
the action but in such a method the number of Gribov copies is not
accounted for and no further calculations have been done along these
lines. Slavnov, in 2008 and 2010, pointed out that if we do not take
the absolute value of the determinant of the Faddeev-Popov operator and
integrate over all Gribov copies, their effects will cancel out. The
disadvantage is that if we make approximations then the errors can get
very large. Stochastic quantization with stochastic gauge fixing
introduces a gauge-fixing force which is tangent to the gauge orbit. More
work needs to be done in this method.

\paragraph{Summary}
In order to get correct predictions from non-Abelian field theories,
which are susceptible to large number of gauge copies, we need to choose
a representative of each gauge orbit. Some new methods and mathematical
tricks have been explored but none have given a consistent recipe for
selection of the representatives of these equivalence classes. However,
if approximations are made in a clever manner, some of the methods can
give us practically usable results. There are also other methods, such as
the semi-classical approach by Gribov \PCedit{(reference ?)} which I have
not discussed here.



\renewcommand{\ssp}{a}
