% reducesymm/QFT/Gribov.tex
% Predrag  created              Sep 2 2013
% continues siminos/blog/dailyBlog.tex as of that date

\chapter{Gribov ambiguity}
\label{c-Gribov}


% Clipped from \refref{atlas12}:
% \\
Symmetry reduction in dynamics (including classical field theories such
as the \NSe) closely parallels the reduction of gauge symmetry in
quantum field theories. There, the freedom of choosing moving frames
 is called `gauge freedom' and a
particular prescription for choosing a representative from each gauge
orbit is called `gauge fixing'. Just like the slice hyperplanes
may intersect a group orbit many times, a gauge
fixing submanifold may not intersect a gauge orbit, or it may intersect
it more than once (`Gribov ambiguity')\rf{Gribov77,VaZw12} In this
context a chart is called a `Gribov' or `fundamental modular' region and
its border is called a `Gribov horizon' (a convex manifold in the
space of gauge fields). The Gribov region is compact and bounded by the
Gribov horizon. Within a Gribov region the `Faddeev-Popov operator'
(analogue of the group orbit tangent vector) is strictly positive, while on
the Gribov horizon it has at least one vanishing eigenvalue.

{\bf [2011-07-08 Predrag]} from \refref{FrCv11}:                     \toCB
\\
% At this point
It is worth noting that imposing the global and fixed slice
%\refeq{PCsectQ}
is not the only way to separate equivariant dynamics into `group
dynamics' and `shape' dynamics\rf{BeTh04}. In modern mechanics and even
field theory (where elimination of group-directions is called
`gauge-fixing') it is natural to separate the flow {\em locally} into
group dynamics and a transverse, `horizontal'
flow\rf{Smale70I,AbrMars78}, by the `method of
connections'\rf{rowley_reduction_2003}. From our point of view, such
approaches are not useful, as they do not reduce the dynamics to a
lower-dimensional \reducedsp\ $\pS/\Group$.



\begin{description}
\item[2013-11-23  Predrag] Moved all matters Gribov from \texttt{siminos/blog/lit.tex} to here.

\item[2012-05-20 Jeff Greensite] has written a book\rf{Greensite11} of
possible interest, \emph{An introduction to the confinement problem}.

\HREF{http://en.wikipedia.org/wiki/Gribov_ambiguity}{Gribov ambiguity wiki}
(edits by Predrag):

Gauge fixing means choosing a representative from each gauge orbit. The
space of representatives is a submanifold and represents the gauge fixing
condition. Ideally, every gauge orbit will intersect this submanifold
once and only once. This is generally impossible globally, especially for
non-abelian gauge theories, because of topological obstructions and the
best that can be done is make this condition true locally. A gauge fixing
submanifold may not intersect a gauge orbit at all or it may intersect it
more than once. This is called a Gribov\rf{Gribov77} ambiguity.

Gribov ambiguities lead to a nonperturbative failure of the BRST
symmetry, among other things.

A way to resolve the problem of Gribov ambiguity is to restrict the
relevant functional integrals to a single \emph{Gribov region} or {\em
fundamental modular region} whose boundary is called a \emph{Gribov
horizon}.

{\em Gribov copies} play a crucial role in the infrared (IR) regime while
it can be neglected in the perturbative ultraviolet (UV)
regime\rf{Gribov77,Zwanz89,Zwanz93}. The restriction to the Gribov region
(defined in such a way that the Faddeev-Popov operator is strictly
positive) can be achieved by adding a nonlocal term, commonly known as
`horizon term', to the YM action\rf{Zwanz89,Zwanz93,Zwanz92}. This is a
nonlocal term in the 4-dimensional Euclidean space, written as an
integral over the `horizon function.'

Greensite: ``In non-Abelian theories, there are many gauge copies -
Gribov copies - that satisfy the Coulomb gauge condition. The Gribov
region is the space of all Gribov copies with positive Faddeev-Popov
eigenvalues. Configurations of the Gribov horizon have at least one FP
eigenvalue $\lambda =0$. What counts for confinement is the density of
eigenvalues $\rho(\lambda)$ near $\lambda =0$, and the `smoothness' of
these near-zero eigenvalues.

The Gribov horizon is a convex manifold in the space of gauge fields,
both in the continuum and on the lattice. The Gribov region, bounded by
that manifold, is compact.
''

Amusingly, they can find the first 200 eigenstates of the lattice
Faddeev-Popov operator on each time-slice of each lattice configuration
by the Arnoldi algorithm.

See also Heinzl\rf{Heinzl96,HeRuSch08}, as well as
\refrefs{RuSchVo02,MaScha94,vanBaal91,DellAnZwan91,Cutkosky84,Singer78}.
Review of \refref{VaZw12} promises ``to give a pedagogic review of the
ideas of Gribov and the subsequent construction of the GZ action,
including many other topics related to the Gribov region.''

Nele Vandersickel
\HREF{http://physik.uni-graz.at/~dk-user/talks/Vandersickel20100225.pdf}
{talk} gives a compact overview, might be useful for writing this up.
Chapter 3 of her thesis, \arXiv{1104.1315}, gives a pedagogic overview of
the Gribov-Zwanziger framework, not available yet in the literature.


\item[2012-06-15 Daniel]
Wow! Kinda lost me here... How does this apply exactly? Is the
point to make an analogy between our slices and these Gribov regions?
All this QFT stuff kind of comes out of left field. Discussing
it with Predrag convinced me that it may be interesting to have this here
because there appear to be some strong parallels with what we're doing. I
may need a little massaging to make it more palatable, though.

\item[2012-06-15 Predrag]
Laufer and Orland\rf{LauOrl12} say in
{\em The geometry of {Yang-Mills} orbit space on the lattice}: ``
We find coordinates, the metric tensor, the inverse metric tensor and the
Laplace-Beltrami operator for the orbit space of Hamiltonian SU(2) gauge
theory on a finite, rectangular lattice. This is done using a complete
axial gauge fixing. The Gribov problem can be completely solved, with no
remaining gauge ambiguities.
''

\item[2012-06-15 Evangelos]
This seems very interesting and I have to read it. It will most probably
confuse people, if we are really addressing fluid-dynamicists.

{\bf [2012-06-15 Predrag]} I am trying to reach out to quantum filed
theorists, that's why these facts are made explicit here - otherwise they
think it is just plumbing, nothing to do with Fundamental Physics..

Maybe shorten it? For instance Faddeev-Popov operator is not introduced
here, so we might avoid reference to it altogether. In the last sentence,
is the Gribov region or the Gribov horizon a convex manifold?

\item[2010-09-28 ES: Faddeev-Popov ghosts]                    \toCB
(moved to here from froehlich/blog)
\\
From my random readings, supposedly making up for my inability to attend
colloquia in French: Faddeev in a
\HREF{http://www.scholarpedia.org/article/Faddeev-Popov_ghosts}{scholarpedia
article} discusses the difficulties in Yang-Mills quantization that led
him and Popov to introduce fictitious fields, now known as
\emph{Faddeev-Popov ghosts}. The problem was that of gauge fixing,
essentially of working on a slice. Faddeev says:

\begin{ttfamily}
It was clear that the equivalence principle had to be taken into account.
In the functional integral framework, the equivalence principle implies
that one has to integrate over classes of gauge equivalent fields instead
of integrating over all fields $A_\mu^a$.

The choice of the representatives in the classes of equivalent fields is
realized by means of a gauge condition (gauge fixing), for instance,
\[
    \partial_{\mu} A_{\mu}^{a} = 0 .
\]
This condition defines a plane in the set of all fields, which is
intersected by the gauge orbits defined by
\[
    A_{\mu} = A_{\mu}^{a}t_{a} \to A_{\mu}^{\Omega}
            = \Omega A_{\mu} \Omega^{-1} + \partial_{\mu} \Omega \Omega^{-1} .
\]
In this context, the difference among abelian and non-abelian cases
becomes clear. In the abelian case, we take $\Omega(x) =
\exp{i\Lambda(x)}$ and a gauge orbit is defined by
\[
    A_{\mu} \to A_{\mu} + \partial_{\mu} \Lambda ,
\]
which is just a linear shift. Thus all the abelian orbits intersect the
gauge surface at the same angle.

In the non-abelian case, the gauge orbit equations are non-linear and the
intersection angle depends on the field parameterizing the orbit. It is
clear that this must be taken into account in the functional integral.
\end{ttfamily}

See the wikipedia link to put this in the proper context. As an abelian
case Faddeev lists quantum electrodynamics where the group is $U(1)$ (the
same as in \cLe\ if we think in terms of complex variables). As a
non-abelian example he lists the Standard Model: $U(1)\times SU(2) \times
SU(3)$.

Faddeev relies on the group being connected to write group action in
exponential form, as Stefan does. $\On{2}$ in Kuramoto-Sivashinsky is
non-abelian so I am worried that there might be more work required. Are
all non-connected compact Lie groups non-abelian and vice versa?

The weight factor Faddeev and Popov introduced might be helpful in
trace-formulas for non-abelian groups.

\item[2012-06-15 Evangelos]
Something that still confuses me: Is `Faddeev-Popov operator' really the
gauge orbit tangent vector, or is it the group generator? {\bf
[2012-06-15 Predrag]} I think `group generator' would not have enough
information (why should it be position dependent?), so it should be
something like our group orbit tangent vector, except positivity of
eigenvalues sounds like a projection on a given direction across the
slice.




\end{description}
\renewcommand{\ssp}{a}
