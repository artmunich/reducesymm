% reducesymm/QFT/dailyBlog.tex
% $Author$ $Date$
% Predrag  switched to github.com               jul  8 2013
% former siminos/blog/dailyBlog.tex

\chapter{Daily QFT blog}
\label{c-DailyBlog}


\section{Is QED finite?}
\label{sect:finiteQED}

\begin{description}

\item[2013-10-22 Nio Makiko to Warren]  nio@riken.jp  wrote:\\
  Thank you for useful information on knots theory and muon g-2.
 We will carefully examine them.

 As for numerical values of Set V, we are reluctant to open these values
 before publication.  Once we finish writing up the detailed paper and
 posting it on arXiv,  I will let you know.

\item[2013-10-23  Warren D Smith] warren.wds@gmail.com\\
well, I don't have any choice :).  However I point out perhaps you could send
me the numbers for $\alpha^4$ diagrams even if you want to wait on the
$\alpha^5$  diagrams?

Meanwhile I found out more interesting info which the textbooks do not know!

Cvitanovi\'c 1977 conjectured that gauge-invariant quenched diagram sets
always have small sum of diagram values, indeed small enough that he
thought quenched QED series
would always converge\rf{Cvit77b}
    \PC{`quenched' = no internal electron loops, \ie, $m_e \to \infty$
    approximation}.
He apparently originally thought this even
for unquenched but Lautrup\rf{Lautrup77} disproved it and Dyson had long had a
highly convincing argument\rf{Dyson52} (Reprinted p.255-6 in Selected papers
of Freeman Dyson with
commentary, AMS 1996)
-- and I now have even more convincing
arguments -- that generic QED series diverge for any nonzero $\alpha$.)
This was due to Cvitanovi\'c's empirical observation of amazingly huge
cancellations within gauge-invariant diagram sets, especially in
quenched QED.
    \PC{Suslov\rf{Suslov99} argues that t'Hooft and Lautrup
     renormalons are not significant? I have not studied it.}

However, what Cvitanovi\'c did not know, was that
\HREF{http://www.itp.ac.ru/en/persons/bogomolny-evgeny-borisovich/}
{Bogomolny} and Kubyshin 1981-1982
found estimates of the growth rate of generic QED series, and also for the
quenched QED subseries,
and indeed for QED for diagrams with k electron loops only (k fixed)\rf{BogKub81}.
I only found this out the other day, but I had long known about
estimates due to Dyson and others predicting divergence for QED
series.  It is just that this B+K work by permitting arbitrary fixed
k, directly addresses Cvitanovi\'c's quenched-convergence conjecture and
massively conflicts with it -- for quenched QED the prediction is that
at Nth order we get a quenched diagram sum growing factorially with N:

Evgeny B Bogomolny and Yu A Kubyshin:
Asymptotic estimates for graphs with a fixed number of fermion loops
in quantum electrodynamics.
\\
1. The choice of the form of the
steepest-descent solutions,
Soviet J. Nucl. Phys. 34,6 (1981) 853-858.
Asymptotic estimates for diagrams with a fixed number of fermion loops
in quantum electrodynamics.
\\
2. The extremal configurations with the
symmetry group $O(2)\times O(3)$,
Soviet J. Nuclear Phys. 35 ,1 (1982) 114-119.
    \PC{there is a hard copy at GT library, 4th Floor East
Call Number: 	QC173.I252X (or microfilm?)}

I believe in the sort of arguments Bo+Ku are making, albeit the
details are questionable.
(E.g. saddlepoint asymptotic estimates are not rigorous unless you prove stuff
about the saddlepoints and about tail estimates, which they never proved, and
probably nobody can prove.)

Note that this believed N! growth for fixed-k (including quenched) QED diagrams
(Lautrup's diagrams also feature N! growth) is far faster than the
believed growth -- more like (N/2)! -- for the \emph{full} QED series!!
This fact that subseries diverge far more rapidly than full series
can only be explained by presuming that
the values at different k cancel each other amazingly well when we sum
over all k.
This indicates Cvitanovi\'c was extremely wrong asymptotically, and that
cancellations
quite different than his observations ultimately occur which seem even
more dramatic.

However, Cvitanovi\'c was correct that amazingly large cancellations
(also) occur, empirically, within the quenched diagrams alone, at
least for the small N that have been reached by computer.   And I
presume that Kreiman's knot
ideas\rf{Kreimer00} and my ``fractal distribution'' empirical observations are a
partial explanation of why.

So a consistent picture is now developing about how QED perturbative series (and
various interesting sub-series, such as quenched) allegedly behave
asymptotically
(Although I never saw anything saying all that I just said in one place...)

\item[2013-10-23  Warren to Predrag]  do you have a graph-theoretic
characterization of `gauge invariant diagram set' or know how many such
(minimal) sets there are at order N?   E.g. does their count grow
exponentially, super-exponentially, polynomially or what?  You gave a
formula for the count of quenched GI-sets which grows only polynomially
but I suspect exponential or faster growth for unquenched QED.  Even an
incomplete graph-theory understanding might be adequate to get good growth
bounds.

\item[2013-10-23 Warren]
Although I do not fully understand ``gauge invariant classes of Feynman diagram''
I have figured out enough now to prove that their count grows
ultimately superexponentially in unquenched QED.  Specifically I now
claim to have a proof that

  Number of GI classes of Feynman diagrams in QED at $\alpha^N$ order
\[
  \to 0.01 *  96^{-N/4} * N! / [ (N/2)! * (N/4)! ]
\]
for an infinite set of integers $N>0$.
And this clearly grows superexponentially.
(My bound is very unlikely to be optimal.)

\item[2013-10-23 Predrag to Warren]
of course, I'm very interested in this discussion, but can you do me a
favor and actually read my paper, and edit your initial email
accordingly, before we wrangle with further details?

Many of your statements are addressed in my paper, and I can answer thyem
more efficiently if you go through them first. I love Dyson dearly, but
his statement is an elementary statement about asymptotic series for
factorials. Mine is about mass-shell gauge invariant quantities (perhaps
planar?), a topic that is gaining some traction now, unfortunately not
applicable to QED. Ain't they weird? They are perfectly happy developing
methods for gauge theories that do not work on the 'trivial' U(1) case?
Go figure...

If you can get gauge sets added up, that would be useful, but they
probably do not have them: I assume they  use the second method of
computing magnetic moment,
\HREF{http://www.cns.gatech.edu/\%7Epredrag/papers/preprints.html\#g-2}
{eq (6.22)}  [that I believe I post-invented
after Schwinger? but probably I'm deluding myself..] I think that mixes up
gauge sets.

\item[2013-10-23 Warren]
Actually, I already had read it
\HREF{http://www.cns.gatech.edu/~predrag/papers/NPB77.pdf}
{here}.
The problem is not me not reading it; it is me being an idiot.  And/or, you.
Actually, it is quite likely that I am more of an idiot now than you
were in 1977, but I suspect we both are capable of considerable
idiocy...

[there is much more still on 2013-10-23, but I better go to bed...]

\item[2013-11-23  Predrag]
\HREF{WarrenSmithQED131123.html} {Warren Smith} draft.


\end{description}


\section{Quantum Field Theory}
\label{sect:QFT}

\begin{description}
\item[2013-11-23  Predrag] Moved all matters QFT from
\texttt{siminos/blog/Lie.txt} to here.


\item[2011-07-27 PC]
What follows is casting eye far ahead - to the role of gauge invariance
in Quantum Field Theories, but just to have it recorded somewhere.
Following articles seem of interest as follow-ups on
Cvitanovi\'c\rf{PCar}, {\em Group theory for {Feynman} diagrams in
non-{Abelian} gauge theories}:

Should add this article to Birdtracks.eu/refs: Astorino\rf{Astor10}
writes ``Jones polynomial arises as special cases: Sp(2), SO(-2), and
SL(2,R). These results are confirmed and extended up to the second order,
by means of perturbation theory, which moreover let us establish a
duality relation between $SO(\pm N)$ and $Sp(\mp N)$ invariants. A
correspondence between the first orders in perturbation theory of SO(-2),
Sp(2) or SU(2) Chern-Simons quantum holonomy's traces and the partition
function of the Q=4 Potts model is built.''

Khellat\rf{Khel10} strikes me as dubious...

Martens\rf{Mart11} writes: ``We calculate the two-loop matching corrections for the
   gauge couplings at the Grand Unification scale in a general framework
   that aims at making as few assumptions on the underlying Grand Unified
   Theory (GUT) as possible. In this paper we present an intermediate
   result that is general enough to be applied to the Georgi-Glashow
   SU(5) as a ``toy model''. The numerical effects in this theory are
   found to be larger than the current experimental uncertainty on $\alpha$s .
   Furthermore, we give many technical details regarding renormalization
   procedure, tadpole terms, gauge fixing and the treatment of group
   theory factors, which is useful preparative work for the extension of
   the calculation to supersymmetric GUTs.
   ''

Tye and Zhang\rf{TyZh10} write: ``
Bern, Carrasco and Johansson have conjectured dual
   identities inside the gluon tree scattering amplitudes.
   We use the properties of the heterotic string and open string tree
   scattering amplitudes to refine and derive these dual identities.
   These identities can be carried over to loop amplitudes using the
   unitarity method. Furthermore, given the $M$-gluon (as well as
   gluon-gluino) tree amplitudes, $M$-graviton (as well as
   graviton-gravitino) tree scattering amplitudes can be written down
   immediately, avoiding the derivation of Feynman rules and the
   evaluation of Feynman diagrams for graviton scattering amplitudes

Eto \etal~rf~{EFGKNOV08} write:
   ``We construct the general vortex solution in the color-flavor-locked
   vacuum of a non-Abelian gauge theory, where the gauge group is taken
   to be the product of an arbitrary simple group and U(1). Use of the
   holomorphic invariants allows us to extend the moduli-matrix method
   and to determine the vortex moduli space in all cases. Our approach
   provides a new framework for studying solitons of non-Abelian
   varieties with various possible applications in physics.''

and there is much much more...; will continue some other time.

\item[2011-11-03 PC] Today is that time. I'm sitting in
\HREF{http://intractability.princeton.edu/blog/2011/05/workshop-counting-inference-and-optimization-on-graphs/}
     {Intractability Workshop:}
     \emph{Counting, Inference and Optimization on Graphs}
with a bunch of high-level computer nerds, and I almost afraid to say
what I'll say next (plumbers avoid physicists that say such things): In
constructing our atlas of inertial manifold of turbulent pipe flow, we
fix the $SO(2) \times O(2)$ phase separately on each local chart. The
freedom of doing that is called ``local gauge invariance'' (blame Hermann
Weyl for the ugly word) and in the limit of $\infty$ period cycles, cycle
points are dense and their local charts are infinitesimal, so this is
really local gauge invariance. In the world of computer science they use
this freedom profitably, to reduce the number of terms they use in their
computations. That suggests that there might be a (variational?)
principle that selects an optimal choice of (relative) template phases

Nerds call this 'reparametrization' - it supposedly speeds up calculations.
Have not really seen that in quantum field theory, with exception of light
cone gauges and their relatively recent applications by the Witten cult.

Literature: \refref{CheChe08,YeChe11} and stuff on
\HREF{http://www.hpl.hp.com/personal/Pascal_Vontobel/ciog2011/reading_list_web.html}{this
site} (if you can understand any of it).

Feel free to ignore this remark. It's future research.

\item[2012-06-14 Predrag]
\HREF{http://marcofrasca.wordpress.com/about/}{Frasca} is either insane
or just yet another ignorant field theorist: ``I have worked on almost
all fields of physics'' (???). I checked the publication list, and it is
no  L.D. Landau. But the blog is informative:

{\bf [2012-06-05 Frasca]} (edited by Predrag):

``The answer to the question of the mass gap in Yang-Mills theory has
made enormous progress mostly by the use of lattice computations and,
quite recently, with the support of theoretical analysis. Contrarily to
common wisdom, the most fruitful attack to this problem is using Green
functions. The reason why this was not a greatly appreciated approach
relies on the fact that Green functions are gauge dependent.
Nevertheless, they contain physical information that is gauge independent
and this is exactly what we are looking for: The mass gap.''

(Predrag: this I interpret in the spirit of Gutzwiller - the Green function
is coordinate dependent, but it's trace - which yield the spectrum - is
coordinate invariant.)

``\HREF{http://marcofrasca.wordpress.com/2011/01/28/the-saga-of-landau-gauge-progators-a-short-history/}
{The Saga of Landau-Gauge Propagators: A Short History}'' is a good read:

``We cannot forever ignore the low energy behavior of QCD as its complete
understanding could have impact at unexpected large scales.''


\item[2013-02-22 PC] Fomin and Pylyavskyy\rf{FomPyl12}
{\em Tensor diagrams and cluster algebras}, {\arXiv{1210.1888}},
is a major orgy in birdtracking. Should study it some day.

\item[2012-05-16  Parameswaran Nair]  vpnair@optonline.net writes
on saddle solutions of Yang-Mills:

I attributed the conjecture to Hitchin; it was actually due to Atiyah and
Jones. "The only finite action solutions of the YM equations are
instantons, either self-dual or antiself-dual." This was the conjecture
for which the refs provide counter examples.

\HREF{http://ChaosBook.org/library/Schiff91.pdf}{Here is} the paper by my
student Schiff\rf{Schiff91}, who writes:
``
Following a proposal of Burzlaff (Phys.Rev.D 24 (1981) 546), we find
solutions of the classical equations of motion of an abelian Higgs model
on hyperbolic space, and thereby obtain a series of non-self-dual
classical solutions of four-dimensional SU(3) gauge theory. The lowest
value of the action for these solutions is roughly 3.3 times the standard
instanton action.
''

``
In physics, despite the fact that the non-self-dual solutions correspond
to saddle points, and not minima, of the Yang-Mills functional, to do a
correct semiclassical approximation by a saddle-point evaluation of the
path integral, it is certainly necessary to include a contribution due to
nonself-dual solutions, and if it should be the case that there is a
non-self-dual solution with action lower than the instanton action (this
question is currently open, and of substantial importance), then such a
contribution would even dominate. Unfortunately, it is questionable
whether the semiclassical approximation can give a reliable picture of
quantized gauge theories; it has been argued that in four-dimensional
gauge theory small quantum fluctuations
around classical solutions cannot be responsible for
confinement, unlike in certain lower-dimensional
theories. But it may still be possible to extract some
physics from the semiclassical approach. A first step in
such a direction would be to obtain a good understanding
of the full set of non-self-dual solutions and their properties.
[...] We pursue an old idea, due to Burzlaft
[10], for obtaining a non-self-dual, "cylindrically symmetric"
solution for gauge group SU(3). If we write
$\reals^4 = \reals \times \reals^3$, and identify some SU(2) [or SO(3)] subgroup
of SU(3), with generators that we will denote T', then we
can look at the set of SU(3) gauge potentials which are invariant
under the action of the group generated by the
sum of the T"s and the generators of rotations on the IR
factor of E (we choose the T"s and the IR rotation generators
to satisfy the same commutation relations). We
call such potentials "cylindrically symmetric" (in analogy
to the standard notion of cylindrical symmetry in IR,
which involves writing R =RXIR and requiring rotational
symmetry on the E factor). Such potentials will
be specified by a number of functions of two variables:
the coordinate on the IR factor of E (which we will
denote x), and the radial coordinate of the E factor
(which we will denote y). Clearly the equations of motion
for such cylindrically symmetric potentials (if they are
consistent) will reduce to equations on the space
I (x,y ):y ~ 0].
,,

The earlier work is by Sibner, Sibner, Uhlenbeck\rf{SiSiUhl89}. They
write ``
The Yang-Mills functional for connections on principle SU(2) bundles over
S 4 is studied. Critical points of the functional satisfy a system of
second-order partial differential equations, the Yang-Mills equations. If,
in particular, the critical point is a minimum, it satisfies a
first-order system, the self-dual or anti-self-dual equations. Here, we
exhibit an infinite number of finite-action non-minimal unstable critical
points. They are obtained by constructing a topologically nontrivial loop
of connections to which min-max theory is applied. The construction
exploits the fundamental relationship between certain invariant
instantons on S 4 and magnetic monopoles on H 3. This result settles a
question in gauge field theory that has been open for many years.
''

% 2012-05-16 fund this:
%@article{
%author = {Gil Bor and Richard Montgomery},
%title = {{SO(3)} invariant {Yang-Mills} fields which are not self-dual},
%}

Bor\rf{Bor92} writes
``
We prove the existence of a new family of non-self-dual finite-energy
solutions to the Yang-Mills equations on Euclidean four-space, with SU(2)
as a gauge group. The approach is that of ``equivariant geometry:''
attention is restricted to a special class of fields, those that satisfy
a certain kind of rotational symmetry, for which it is proved that (1) a
solution to the Yang-Mills equations exists among them; and (2) no
solution to the self-duality equations exists among them. The first
assertion is proved by an application of the direct method of the
calculus of variations (existence and regularity of minimizers), and the
second assertion by studying the symmetry properties of the linearized
self-duality equations. The same technique yields a new family of
non-self-dual solutions on the complex projective plane.
''

\item[2012-04-19 Daniel] ``graveyard of obvious ideas'' rings a little
    aggressive, no? ``if you are a master of quantum-mechanics or QFT symmetries
and their linear irreducible representations,\rf{PCgr} you may leave your
baggage at the door'' rings a little aggressive, too.
\\{\bf Predrag:} Get's edgy. In ``master of their linear irreducible
    representations'' I make fun of myself. Let the referee object to
    that?


{\bf [2012-06-14 Predrag]} Grin and bear it. Pulling my Senior discount
card here.

\item[2013-07-15 Predrag] I've collected a bunch of QFT e-books:

5449Grigorenko06.pdf

Abarbanel 2013.pdf

CoKaWa04.pdf

DasFerbel03.pdf

Zee03.pdf

hao89.pdf

Nichkawde13.pdf

Das06.pdf

Milton01.pdf

SeoSan12.pdf

NagashimaI10.pdf


\item[2013-01-20  Predrag]
This really belongs to planar field theory, but for time being I note
it here: Lucini and Panero\rf{LucPan13} (in Chaosbook.org/library)
might be of interest. All I get is one sentence and a reference only
to \refref{PlanFieldThe}.

\item[2013-03-27  Predrag] Do not understand this article:
Jim\'enez-Lara and J. Llibre\rf{JimLli11},
{\em Periodic orbits and nonintegrability of generalized
classical {Yang--Mills Hamiltonian} systems}.

                                                \toCB
D. Biswas \etal\rf{BALL92} {\bf Existence of stable periodic orbits
in the $x^2y^2$ potential: a semiclassical approach}, re-derives the
Dahlqvist and G. Russberg\rf{DR_prl} result. Also read Nip
\etal\rf{NTOD92} {\em Search for regular orbits in the $x^2y^2$
potential problem}

Hu\rf{HJLW01}
{\em General initial value form of the semiclassical propagator},
write: ``
We show a general initial value form of the semiclassical propagator.
Similar to cellular dynamics, this formulation involves only the
nearby orbits approximation: the evolution of nearby orbits is
approximated by linearized dynamics. This phase space smearing
formulation keeps the accuracy of the original Van Vleck-Gutzwiller
propagator. As an illustration, we present a simple initial value
form of the semiclassical propagator. It is nonsingular everywhere
and is efficient for numeric implementation.
''

\item[2013-04-16  Predrag] There seems to be whole literature on
classical Yang-Mills (CYM). In
{\em Entropy production in classical Yang-Mills theory from Glasma
initial conditions} Hideaki Iida,  Teiji Kunihiro,  Berndt M\"uller,
Akira Ohnishi,  Andreas Sch\"afer,  and Toru T. Takahashi,
\arXiv{1304.1807}, % \rf{IKMOST13}
write:

Pure Yang-Mills theory in temporal gauge with the Hamiltonian in the
noncompact (A, E) scheme on a cubic spatial lattice. The initial
condition satisfies Gauss' law; check its validity as well as Energy
conservation carefully at every time step. Define distance (6), (7) that
is gauge invariant under residual (time independent) gauge transformations.

                                                    \inCB
They call the stability matrix `Hessian', and its eigenvalues at time
slice the `local Lyapunov exponents (LLEs)'\rf{KMOSTY10}: LLE plays the
role of a ``temporally local'' Lyapunov exponent, which specifies the
departure rate of two trajectories in a short time period. Then they say
this (?): ``For a system where stable and unstable modes couple with each
other as in the present case, an LLE does not generally agree with the
Lyapunov exponent in a long time period.'' ``\refRef{KMOSTY10} introduced
another kind of Lyapunov exponent called the intermediate Lyapunov
exponent (ILE), which is an ``averaged Lyapunov exponent'' for an
intermediate time period; i.e., a time period which is sufficiently small
compared to the thermalization time but large enough to sample a
significant fraction of phase space. By its definition (13) it
is the set of stability exponents for a finite time \jacobianM.

``Two comments are in order, here: A Lyapunov exponent [PC: not the
Lyapunov exponent, they mean the stability exponent] can be (real)
positive, negative, zero or purely imaginary. Liouville's theorem tells
us that the determinant of the time evolution matrix U is unity, implying
that the sum of all positive and negative ILEs is zero. The KS entropy is
given as a sum of positive Lyapunov exponents. The second comment
concerns gauge invariance of the Lyapunov exponents. In the Appendix we
show that LLE and ILE are indeed gauge invariant under time-independent
gauge transformations in the temporal gauge.''

\end{description}
