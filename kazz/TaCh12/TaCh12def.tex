% siminos/lyapunov/TaCh12/TaCh12def.tex
% $Author$ $Date$

% from pipes/KreEck12/KreEck12def.txt

  \newcommand{\PCedit}[1]{{\color{blue}#1}}
    \ifdraft
  \newcommand{\PC}[2]%{$\footnotemark\footnotetext{[#1 Predrag] #2}$}
                        {\begin{quote}\PCedit{[#1 Predrag] #2}\end{quote}}
  \newcommand{\toCB}{\marginpar{\footnotesize 2CB}}  % to compare with ChaosBook
  \newcommand{\inCB}{\marginpar{\footnotesize in CB}} % entered in ChaosBook
    \else
  \newcommand{\PC}[2]{\begin{quote}\PCedit{[Referee comment] #2}\end{quote}}
  \newcommand{\toCB}{}
  \newcommand{\inCB}{}
    \fi %end of internal draft switch

  \newcommand{\TKedit}[1]{{\color{red}#1}}
  \newcommand{\TK}[2]%{$\footnotemark\footnotetext{[#1 Tobias] #2}$}
                     {\begin{quote}\TKedit{[#1 Tobias] #2}\end{quote}}
  \newcommand{\BE}[2]{$\footnotemark\footnotetext{[#1 Bruno] #2}$}
  \newcommand{\BEedit}[1]{{\color{green}#1}}

  \newcommand{\wwwcb}[1]{       % keep homepage flexible:
                  {\tt \href{http://ChaosBook.org#1}
              {ChaosBook.org#1}}}
  \newcommand{\weblink}[1]{{\tt \href{http://#1}{#1}}}
  \newcommand{\HREF}[2]{
              {\href{#1}{#2}}}
  \newcommand{\arXiv}[1]{
              {\tt \href{http://arXiv.org/abs/#1}{\goodbreak arXiv:#1}}}

% \newcommand{\etc}{{etc.}}       % APS
% \newcommand{\etal}{{\em et al.}}    % etal in italics, APS too
\newcommand{\ie}{{i.e.}}        % APS
\newcommand{\turn}{turning point}    % {turnback} ??
\newcommand{\statesp}{phase space}
\newcommand{\Statesp}{Phase space}
\newcommand{\stateDsp}{phase-space}
\newcommand{\StateDsp}{Phase-space}
\newcommand{\dmn}{-dimensional}  %  experimental 220ct2009
\newcommand{\po}{periodic orbit}
\newcommand{\Po}{Periodic orbit}
\newcommand{\rpo}{rela\-ti\-ve periodic orbit}
%   \newcommand{\rpo}{equi\-vari\-ant periodic orbit}
\newcommand{\Rpo}{Rela\-ti\-ve periodic orbit}
%   \newcommand{\Rpo}{Equi\-vari\-ant periodic orbit}
\newcommand{\eqv}{equi\-lib\-rium}
\newcommand{\Eqv}{Equi\-lib\-rium}
\newcommand{\eqva}{equi\-lib\-ria}
\newcommand{\Eqva}{Equi\-lib\-ria}
\newcommand{\eqb}{equilibrium}
\newcommand{\Eqb}{equilibrium}
\newcommand{\eqba}{equilibria}
\newcommand{\Eqba}{Equilibria}
\newcommand{\reqv}{rela\-ti\-ve equi\-lib\-rium}
%   \newcommand{\reqv}{equi\-vari\-ant equilibrium}
%   \newcommand{\reqv}{travelling wave}
\newcommand{\Reqv}{Rela\-ti\-ve equi\-lib\-rium}
%   \newcommand{\Reqv}{Equi\-variant equi\-librium}
%   \newcommand{\Reqv}{travelling wave}
\newcommand{\reqva}{rela\-ti\-ve equi\-lib\-ria}
%   \newcommand{\reqva}{equivariant equilibria}
\newcommand{\Reqva}{Rela\-ti\-ve equi\-lib\-ria}
%   \newcommand{\Reqva}{Equivariant equilibria}
\newcommand{\phaseVel}{phase velocity}      % pipe slicing
\newcommand{\phaseVels}{phase velocities}   % pipe slicing
\newcommand{\PhaseVel}{Phase velocity}      % pipe slicing
\newcommand{\PhaseVels}{Phase velocities}   % pipe slicing
\newcommand{\equilibrium}{equi\-lib\-rium}
\newcommand{\equilibria}{equi\-lib\-ria}
\newcommand{\Equilibria}{Equi\-lib\-ria}
\newcommand{\Reynolds}{\textit{Re}}  % Reynolds number
\newcommand{\pCf}{plane Couette flow}
\newcommand{\PCf}{Plane Couette flow}
\newcommand{\KS}{Kuramoto-Siva\-shin\-sky}
\newcommand{\KSe}{Kuramoto-Siva\-shin\-sky equation}
\newcommand{\NS}{Navier-Stokes}
\newcommand{\NSE}{Navier-Stokes Equations}
\newcommand{\NSe}{Navier-Stokes equations}
\newcommand{\Fd}{spec\-tral det\-er\-min\-ant}
\newcommand{\stagp}{stagnation point}
\newcommand{\Stagp}{Stagnation point}

\newcommand{\expctE}{\ensuremath{E}}    % E space averaged
\newcommand{\tildeL}{\ensuremath{\tilde{L}}}
\newcommand\period[1]{{\ensuremath{T_{#1}}}}         %continuous cycle period
\newcommand{\shift}{\ensuremath{d}}
\newcommand{\Fix}[1]{\ensuremath{\mathrm{Fix}\left(#1\right)}}
\newcommand{\PoincS}{\ensuremath{{\cal P}}}  % symbol for Poincare section
\newcommand{\PoincM}{\ensuremath{P}}       % symbol for Poincare map
\newcommand{\PoincC}{\ensuremath{U}}       % symbol for Poincare constraint function
\newcommand{\obser}{\ensuremath{a}}     % an observable from phase space to R^n
\newcommand{\Obser}{\ensuremath{A}}     % time integral of an observable
\newcommand{\pS}{\ensuremath{{\cal M}}}          % symbol for state space
\newcommand{\ssp}{\ensuremath{x}}                % state space point
%\newcommand{\zeit}{\ensuremath{t}}  %time variable Ashley
\newcommand{\zeit}{\ensuremath{\tau}}  %time variable ChaosBook
\newcommand{\pSpace}{x}       % Hamiltonian phase space x=(q,p) coordinate
\newcommand{\vel}{\ensuremath{v}}   % state space velocity
\newcommand{\Norm}[1]{\|{#1}\|}
\newcommand{\braket}[2]
		   {\langle{#1}\vphantom{#2}|\vphantom{#1}{#2}\rangle}
\newcommand{\bra}[1]{\langle{#1}\vphantom{ }|}
\newcommand{\ket}[1]{|\vphantom{}{#1}\rangle}
% \newcommand{\expct}    [1]{\left\langle {#1} \right\rangle}
\newcommand{\spaceAver}[1]{\left\langle {#1} \right\rangle}
\newcommand{\timeAver} [1]{\overline{#1}}
\newcommand{\Lint}[1]{\frac{1}{L}\!\oint d#1\,}
\newcommand\stagn{q}      %equilibrium/stagnation point suffix
\newcommand{\Un}[1]{\ensuremath{\textrm{U}(#1)}}         % in DasBuch
\newcommand{\On}[1]{\ensuremath{\textrm{O}(#1)}}
%\newcommand{\SOn}[1]{\ensuremath{\mathbf{SO}(#1)}} % in Siminos thesis
\newcommand{\SOn}[1]{\ensuremath{\textrm{SO}(#1)}}         % in DasBuch
\newcommand{\SUn}[1]{\ensuremath{\textrm{SU}(#1)}}         % in DasBuch
\newcommand{\Spn}[1]{\ensuremath{\textrm{Sp}(#1)}}         % in DasBuch
%\newcommand{\Dn}[1]{\ensuremath{\mathbf{D}_{#1}}    % in Siminos thesis
\newcommand{\Dn}[1]{\ensuremath{\textrm{D}_{#1}}}              % in DasBuch
\newcommand{\Zn}[1]{\ensuremath{\mathbf{Z}_{#1}}}    % in Siminos thesis
%\newcommand{\Zn}[1]{\ensuremath{\textrm{C}_{#1}}}              % in DasBuch
%\newcommand{\Ztwo}{\ensuremath{\mathbf{Z}_2}}      % in Siminos thesis
\newcommand{\Ztwo}{\ensuremath{\textrm{C}_2}}                % in DasBuch
%\newcommand{\Refl}{\ensuremath{\kappa}}            % Siminos uses R for rotations.
\newcommand{\Refl}{\ensuremath{\sigma}}             % in DasBuch
%\newcommand{\Shift}{\ensuremath{\tau}}
\newcommand{\Rot}[1]{\ensuremath{C^{#1}}}           % in DasBuch, e.g. C^{1/3}
%\newcommand{\Rot}[1]{\ensuremath{R(#1)}}           % Siminos uses R for rotations.
\newcommand{\EQV}[1]{\ensuremath{EQ_{#1}}} %experimental
% \newcommand{\EQV}[1]{\ensuremath{q_{#1}}} %ChaosBook
% \newcommand{\EQV}[1]{\ensuremath{E_{#1}}} %Ruslan
% E_0: u = 0 - trivial equilibrium
% E_1,E_2,E_3, for 1,2,3-wave equilibria
\newcommand{\REQV}[2]{\ensuremath{TW_{#1#2}}} % #1 is + or -
% TW_1^{+,-} for 1-wave traveling waves (positive and negative velocity).
\newcommand{\PO}[1]{\ensuremath{PO_{#1}}}
% PO_{period to 2-4 significant digits} - periodic orbits
\newcommand{\RPO}[1]{\ensuremath{RPO_{#1}}}
% RPO_{period to 2-4 significant digits} - relative PO.  We use ^{+,-}
% to distinguish between members of a reflection-symmetric pair.
\newcommand{\tEQ}{\ensuremath{{EQ}}}
% \newcommand{\bu}{\ensuremath{{\bf u}}}
\newcommand{\bv}{\ensuremath{{\bf v}}}
\newcommand{\be}{{\bf e}}
% \newcommand{\bx}{{\bf x}}
\newcommand{\bCell}{\ensuremath{\Omega}}
\newcommand{\bNarrow}{\ensuremath{\Omega_{GHC}}}
\newcommand{\bHKW}{\ensuremath{\Omega_{HKW}}}

\newcommand{\rf}     [1] {~\cite{#1}}
\newcommand{\refeq}  [1] {(\ref{#1})}
\newcommand{\refeqs} [2]{(\ref{#1}--\ref{#2})}
%%%%%%%%%%%%%%%%%%%%%% CHAOS J SPECIFIC %%%%%%%%%%%%%%%%%%%%%%%%%%%%
\newcommand{\reffig} [1] {Fig.~\ref{#1}}
\newcommand{\reffigs} [2] {Figs.~\ref{#1} and~\ref{#2}}
\newcommand{\refFig} [1] {Fig.~\ref{#1}}
\newcommand{\refFigs} [2] {Figs.~\ref{#1} and~\ref{#2}}
\newcommand{\reftab} [1] {Table~\ref{#1}}
\newcommand{\refTab} [1] {Table~\ref{#1}}
\newcommand{\reftabs}[2] {Tables~\ref{#1} and~\ref{#2}}
%\newcommand{\refref} [1] {Ref.~\onlinecite{#1}}
%\newcommand{\refRef} [1] {Ref.~\onlinecite{#1}}
%\newcommand{\refrefs}[1] {Refs.~\onlinecite{#1}}
%\newcommand{\refRefs}[1] {Refs.~\onlinecite{#1}}
\newcommand{\refref} [1] {Ref.~{#1}}
\newcommand{\refRef} [1] {Ref.~{#1}}
\newcommand{\refrefs}[1] {Refs.~{#1}}
\newcommand{\refRefs}[1] {Refs.~{#1}}
\newcommand{\refsect}[1] {sect.~\ref{#1}}
\newcommand{\refsects}[2] {sects.~\ref{#1} and \ref{#2}}
\newcommand{\refSect}[1] {Sect.~\ref{#1}}
\newcommand{\refSects}[2] {Sects.~\ref{#1} and \ref{#2}}
\newcommand{\refappe}[1] {appendix~\ref{#1}}
\newcommand{\refappes}[2] {appendices~\ref{#1} and~\ref{#2}}
\newcommand{\refAppe}[1] {Appendix~\ref{#1}}
\newcommand{\refrem} [1] {remark~\ref{#1}}

\newcommand{\continue}{\nonumber \\ }
\newcommand{\nnu}{\nonumber}
\newcommand{\ee}[1] {\label{#1} \end{equation}}
\newcommand{\ceq}{\nonumber \\ & & }
\newcommand{\barr}{\begin{array}}
\newcommand{\earr}{\end{array}}

\newtheorem{rmark}{{\small\textsf{\textbf{Remark}}}}%[section]
\newcommand{\remark}[2]{
    %\begin{quotation}
    \vspace*{1ex}

      \begin{rmark}
%        {\small\em\noindent {\small\sf \underline{ #1} ~} #2 }
        {\small\em\noindent {\small\sf #1 ~} #2 } % 2010-12-12 experiment
      \end{rmark}
    \vspace*{1ex}
    %\end{quotation}
              }
