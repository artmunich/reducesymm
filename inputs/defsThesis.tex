% siminos/inputs/defsThesis.tex
% $Author$ $Date$

%%%%%%%%%%%% MACROS, Siminos thesis specific %%%%%%%%%%

\newcommand{\vf}{v}	%%% keep notation for vector field flexible. For the time being follow Das Buch.
\newcommand{\Lint}[1]{\frac{1}{L}\!\oint d#1\,}
\newcommand{\ode}{ODE}
\newcommand{\Rls}[1]{\ensuremath{\mathbb{R}^{#1}}}
\newcommand{\Clx}[1]{\ensuremath{\mathbb{C}^{#1}}}
\newcommand{\conj}[1]{\ensuremath{\bar{#1}}}
\newcommand{\trace}{\mbox{\rm trace}\,}
\newcommand{\Manif}{\ensuremath{\mathcal{M}}}
\newcommand{\Order}[1]{\mathrm{O}(#1)}
%%%%%%%%%%%% Fibre bundles

\newcommand{\tSp}{E}
\newcommand{\bSp}{X}
\newcommand{\prj}{\pi}

%%%%%%%% Symmetries
\newcommand{\On}[1]{\ensuremath{O(#1)}}
\newcommand{\Rg}[1]{\Rls{#1}}
\newcommand{\Idg}{\ensuremath{\mathbf{1}}}
\newcommand{\SOn}[1]{\ensuremath{SO(#1)}}
\newcommand{\Dn}[1]{\ensuremath{D_{#1}}}
\newcommand{\Cn}[1]{\ensuremath{C_{#1}}}
\newcommand{\Zn}[1]{\ensuremath{Z_{#1}}}
\newcommand{\En}[1]{\ensuremath{E(#1)}}
%\newcommand{\Zn}[1]{\ensuremath{C_#1}}         % in DasBuch
%\newcommand{\Ztwo}{\ensuremath{\mathbf{Z}_2}}   % in thesis (obsolete)
\newcommand{\Ztwo}{\ensuremath{D_1}}           % in DasBuch & thesis
\newcommand{\Refl}{\ensuremath{\kappa}}         %%%% Changed this, use R for rotations.
\newcommand{\Shift}{\ensuremath{\tau}}
\renewcommand{\shift}{\ensuremath{\ell}}
\newcommand{\Rot}[1]{\ensuremath{R(#1)}}
\newcommand{\Rotn}[1]{\ensuremath{R_{#1}}}
\newcommand{\Drot}{\ensuremath{\zeta}}
\newcommand{\Lg}{\mathfrak{a}}
\newcommand{\stab}[1]{\ensuremath{\Sigma_{#1}}}
\newcommand{\globstab}[1]{\ensuremath{\Sigma_{#1}}} % Change to be the same as stab. Was \Sigma^\ast_{#1}
\newcommand{\Str}[1]{\ensuremath{\mathcal{S}_{#1}}} % Stratum
\newcommand{\Fix}[1]{\ensuremath{\mathrm{Fix}\left(#1\right)}}
\newcommand{\Nlz}[1]{\ensuremath{N(#1)}}
\newcommand{\doubleperiod}[1]{{\ensuremath{\mathcal{T}_{#1}}}}
%%%%%%%%%%%%%%%%%%%%%%

\newcommand{\nameit}{\ensuremath{w-}}
\newcommand{\bbUplus}{\Fix{\Dn{1}}}
\newcommand{\bbUone}{\Shift_{1/4}\Fix{\Dn{1}}}
\newcommand{\bbU}{\mathbb{U}}
\newcommand{\refneq}[1]{(\ref{#1})}
\newcommand{\refFigToFig}[2]{Figures~\ref{#1} to~\ref{#2}}
\newcommand{\reffigpart}[2]{figure~\ref{#1}(#2)}
\newcommand{\refFigpart}[2]{Figure~\ref{#1}(#2)}


%%%%%%%%%%% Theorems %%%%%%%%%%%%%%%%%%%%%%%%%%%%%%%%%%
\newtheorem{definition}{Definition}[chapter]
\newtheorem{theorem}[definition]{Theorem}
\newtheorem{lemma}[definition]{Lemma}
\newtheorem{proposition}[definition]{Proposition}
\newtheorem{example}[definition]{Example}

\newcommand{\refLem}[1]{Lemma~\ref{#1}}
\newcommand{\refThe}[1]{Theorem~\ref{#1}}
\newcommand{\refDef}[1]{Definition~\ref{#1}}
\newcommand{\refPro}[1]{Proposition~\ref{#1}}

%%%%%%%%% Flows

\newcommand{\Le}{Lorenz equations}
\newcommand{\CLe}{Complex Lorenz equations}
\newcommand{\CLf}{Complex Lorenz flow}
\newcommand{\AGHe}{Armbruster-Guckenheimer-Holmes flow}


%%%%%%%%%%% Equations

\newcommand{\cont}{\,, \\ }


%%%%%%%%%%%% Abbreviations, Siminos thesis specific %%%%%%

%\renewcommand{\etc}{{\em etc.}}       % etcetera in italics
%\renewcommand{\ie}{{that is}}     % use Latin or English?  Decide later.
%\renewcommand{\cf}{{\em cf.}}
%\renewcommand{\etal}{{\em et al.}}              % etcetera in italics

\renewcommand{\statesp}{phase space}
\renewcommand{\Statesp}{Phase space}


%%%%%%%%%%%%%% Solution labels %%%%%%%%%%%%%%%%%%%%
\newcommand{\EQB}[1]{\ensuremath{\mathrm{E}_{#1}}} 
\newcommand{\REQB}[1]{\ensuremath{\mathrm{Q}_{#1}}} % For ODE's, use REQV from chaosbook for PDE's

%%%%%%%%%%%%%%% From KS rpo paper %%%%%%%%%%%%%%%%%%
%%%%%%%%%%%%%%% Sundry symbols within math eviron.: %%%%%%%%%%%%
\newcommand{\jEigvecKS}[1]{\ensuremath{{\mathbf e}^{(#1)}}}   % jacobiam eigenvector, redefined here to	avoid conflict with chaosbook notation. Used in ksStSp chapter.


%%%%%%%%%%%%%% Penalizing loops %%%%%%%%%%%%%%%%%

\newcommand\fp[2]{{\frac{\partial #1}{\partial #2}}}
\newcommand\fder[2]{{\frac{d #1}{d #2}}}
\newcommand\fsd[2]{{d #1/d #2}}
\newcommand\fsp[2]{{\partial #1/\partial #2}}
\newcommand\fps[3]{{\frac{\partial^2 #1}{\partial #2 \partial #3}}}
\newcommand\fsps[3]{{\partial^2 #1/\partial #2 \partial #3}}

\newcommand\Js{\mathbf{\tilde{J}}}
\newcommand\JL{\mathbf{\tilde{J}}_L}
\newcommand\Jp{\mathbf{J}_p}

\newcommand\dtds{\frac{v.\tilde{v}}{v^2}}
\newcommand\hdtds{\frac{u.\tilde{u}}{u^2}}


%%%%%%%%%%%%%%%%%%%%%%%%%%%%%%%%%%%%%%%%%%%%%%%%%%%%

