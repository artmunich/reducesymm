% siminos/inputs/defsThesis.tex
% $Author$ $Date$

%%%%%%%%%%%% MACROS, Siminos thesis specific %%%%%%%%%%

%%%%%%%%%%%%%%% REFERENCING EQUATIONS ETC. %%%%%%%%%%%%%%%%%%%%%%%%%%%%%%%
	% \renewcommand{\refref} [1] {Ref.~\cite{#1}}
	% \renewcommand{\refrefs}[1] {Refs.~\cite{#1}}
\renewcommand{\refeq}  [1] {(\ref{#1})}
\renewcommand{\refeqs} [2]{(\ref{#1}--\ref{#2})}
\renewcommand{\reffig} [1] {Figure~\ref{#1}}
\renewcommand{\reffigs} [2] {Figures~\ref{#1} and~\ref{#2}}
\renewcommand{\reftab} [1] {Table~\ref{#1}}
\renewcommand{\reftabs}[2] {Tables~\ref{#1} and~\ref{#2}}
\renewcommand{\refsect}[1] {Sect.~\ref{#1}}
\renewcommand{\refsects}[2] {Sects.~\ref{#1} and \ref{#2}}
\renewcommand{\refchap}[1] {Chapter~\ref{#1}}
\renewcommand{\refchaps}[2] {Chapters~\ref{#1} and \ref{#2}}
\renewcommand{\refchaptochap}[2] {Chapters~\ref{#1} to \ref{#2}}
\renewcommand{\refappe}[1] {Appendix~\ref{#1}}
\renewcommand{\refappes}[2] {Appendices~\ref{#1} and \ref{#2}}
\renewcommand{\refrem} [1] {Remark~\ref{#1}}
\renewcommand{\refexam}[1] {Example~\ref{#1}}
\renewcommand{\refexer}[1] {Exercise~\ref{#1}}
\renewcommand{\refsolu}[1] {Solution~\ref{#1}}




\newcommand{\vf}{v}	%%% keep notation for vector field flexible. For the time being follow Das Buch.
\newcommand{\Lint}[1]{\frac{1}{L}\!\oint d#1\,}
\newcommand{\ode}{ODE}
% \newcommand{\Rls}[1]{\ensuremath{\mathbb{R}^{#1}}}
\newcommand{\Clx}[1]{\ensuremath{\mathbb{C}^{#1}}}
\newcommand{\conj}[1]{\ensuremath{\bar{#1}}}
\newcommand{\trace}{\mbox{\rm trace}\,}
\newcommand{\Manif}{\ensuremath{\mathcal{M}}}
\newcommand{\Order}[1]{\mathrm{O}(#1)}
%%%%%%%%%%%% Fibre bundles

\newcommand{\tSp}{E}
\newcommand{\bSp}{X}
\newcommand{\prj}{\pi}

%%%%%%%% Symmetries
% \newcommand{\On}[1]{\ensuremath{\mathrm{O}(#1)}}
\newcommand{\Rg}[1]{\Rls{#1}}
\newcommand{\Idg}{\ensuremath{\mathbf{1}}}
% \newcommand{\SOn}[1]{\ensuremath{\mathrm{SO}(#1)}}
% \newcommand{\Dn}[1]{\ensuremath{\mathrm{D}_{#1}}}
\newcommand{\Cn}[1]{\ensuremath{\mathrm{C}_{#1}}}
% \newcommand{\Zn}[1]{\ensuremath{\mathrm{Z}_{#1}}}
\newcommand{\En}[1]{\ensuremath{\mathrm{E}(#1)}}
%\newcommand{\Zn}[1]{\ensuremath{C_#1}}         % in DasBuch
%\newcommand{\Ztwo}{\ensuremath{\mathbf{Z}_2}}   % in thesis (obsolete)
% \newcommand{\Ztwo}{\ensuremath{\mathrm{D}_1}}           % in DasBuch & thesis
% \newcommand{\Refl}{\ensuremath{\kappa}}         %%%% Changed this, use R for rotations.
\newcommand{\Shift}{\ensuremath{\tau}}
\renewcommand{\shift}{\ensuremath{\ell}}
\newcommand{\velRel}{\ensuremath{c}}    % relative state velocity
% \newcommand{\Rot}[1]{\ensuremath{R(#1)}}
\newcommand{\Rotn}[1]{\ensuremath{R_{#1}}}
\newcommand{\Drot}{\ensuremath{\zeta}}
% \newcommand{\stab}[1]{\ensuremath{\Sigma_{#1}}}
\newcommand{\globstab}[1]{\ensuremath{\Sigma_{#1}}} % Change to be the same as stab. Was \Sigma^\ast_{#1}
\newcommand{\Str}[1]{\ensuremath{\mathcal{S}_{#1}}} % Stratum
\newcommand{\Fix}[1]{\ensuremath{\mathrm{Fix}\left(#1\right)}}
\newcommand{\Nlz}[1]{\ensuremath{N(#1)}}
\newcommand{\doubleperiod}[1]{{\ensuremath{\mathcal{T}_{#1}}}}
%%%%%%%%%%%%%%%%%%%%%%

\newcommand{\nameit}{\ensuremath{w-}}
\newcommand{\bbUplus}{\Fix{\Dn{1}}}
\newcommand{\bbUone}{\Shift_{1/4}\Fix{\Dn{1}}}
\newcommand{\bbU}{\mathbb{U}}
% \newcommand{\refneq}[1]{(\ref{#1})}
\newcommand{\refFigToFig}[2]{Figures~\ref{#1} to~\ref{#2}}
\newcommand{\reffigTofig}[2]{Figures~\ref{#1} to~\ref{#2}}
\newcommand{\reffigpart}[2]{Figure~\ref{#1}(#2)}
\newcommand{\refFigpart}[2]{Figure~\ref{#1}(#2)}


%%%%%%%%%%% Theorems %%%%%%%%%%%%%%%%%%%%%%%%%%%%%%%%%%
\newtheorem{definition}{Definition}[chapter]
\newtheorem{theorem}[definition]{Theorem}
\newtheorem{lemma}[definition]{Lemma}
\newtheorem{proposition}[definition]{Proposition}
\newtheorem{example}[definition]{Example}

\newcommand{\refLem}[1]{Lemma~\ref{#1}}
\newcommand{\refThe}[1]{Theorem~\ref{#1}}
\newcommand{\refDef}[1]{Definition~\ref{#1}}
\newcommand{\refPro}[1]{Proposition~\ref{#1}}
\newcommand{\refExa}[1]{Example~\ref{#1}}


%%%%%%%%% Flows

\newcommand{\Le}{Lorenz equations}
\newcommand{\rLor}{\rho}    % parameter r in Lorenz paper
\renewcommand{\CLe}{Complex Lorenz equations}
\renewcommand{\CLf}{Complex Lorenz flow}
\newcommand{\RerCLor}{\rho_1}    % real      part of parameter r, CLe
\newcommand{\ImrCLor}{\rho_2}    % imaginary part of parameter r, CLe
\newcommand{\AGHe}{Armbruster-Guckenheimer-Holmes flow}

%%%%%%%%%%% Equations

\newcommand{\cont}{\,, \\ }

%%%%%%%%%%%% Abbreviations, Siminos thesis specific %%%%%%

%\renewcommand{\etc}{{\em etc.}}       % etcetera in italics
%\renewcommand{\ie}{{that is}}     % use Latin or English?  Decide later.
%\renewcommand{\cf}{{\em cf.}}
%\renewcommand{\etal}{{\em et al.}}              % etcetera in italics

% PC Jul 3 2009: removed Siminos redefinions:
%\renewcommand{\statesp}{phase space}
%\renewcommand{\Statesp}{Phase space}
% PC Aug 20 2009: temporarily disabled ChaosBook definions:
\renewcommand{\reducedsp}{reduced state space}
\renewcommand{\Reducedsp}{Reduced state  space}
\newcommand{\slice}{slice}
\newcommand{\Slice}{Slice}

\newcommand{\sspRed}{\ensuremath{\bar{x}}}    % reduced state space point
% \newcommand{\velRed}{\ensuremath{\bar{v}}}    % PC reduced state space velocity
\newcommand{\velRed}{\ensuremath{u}}    % ES reduced state space velocity

\renewcommand{\slicep}{\ensuremath{\ssp^*}}   % slice-fixing point
\renewcommand{\sliceTan}{\ensuremath{t^*}}    % group orbit tangent at slice-fixing
\renewcommand{\csection}{cross-section}
\renewcommand{\groupTan}{\ensuremath{t}}    % group orbit tangent
\renewcommand{\Group}{\ensuremath{\Gamma}}    % Siminos Lie group
%\newcommand{\Group}{\ensuremath{G}}         % Predrag Lie or discrete group
%\newcommand{\Lg}{\mathfrak{a}}             % Siminos Lie algebra generator
\renewcommand{\Lg}{\ensuremath{\mathbf{T}}}   % Predrag Lie algebra generator
%\newcommand{\LieEl}{\ensuremath{\mathbb{G}}}  % Wiczek project Lie group element
\renewcommand{\LieEl}{\ensuremath{g}}  % Predrag Lie group element

%%%%%%%%%%%%%% Solution labels %%%%%%%%%%%%%%%%%%%%
\newcommand{\EQB}[1]{\ensuremath{\mathrm{E}_{#1}}}
\newcommand{\REQB}[1]{\ensuremath{\mathrm{Q}_{#1}}} % For ODE's, use REQV from chaosbook for PDE's

% Redefine using mathrm, it is a label not a math symbol
\renewcommand{\EQV}[1]{\ensuremath{\mathrm{E}_{#1}}}
% E_0: u = 0 - trivial equilibrium
% E_1,E_2,E_3, for 1,2,3-wave equilibria
\renewcommand{\REQV}[2]{\ensuremath{\mathrm{TW}_{#1#2}}} % #1 is + or -
% TW_1^{+,-} for 1-wave traveling waves (positive and negative velocity).
\renewcommand{\PO}[1]{\ensuremath{\mathrm{PO}_{#1}}}
% PO_{period to 2-4 significant digits} - periodic orbits
\renewcommand{\RPO}[1]{\ensuremath{\mathrm{RPO}_{#1}}}
% RPO_{period to 2-4 significant digits} - relative PO.  We use ^{+,-}
% to distinguish between members of a reflection-symmetric pair.
% Gibson likes:
\renewcommand{\tEQ}{\ensuremath{\mathrm{EQ}}}


%%%%%%%%%%%%%%% From KS rpo paper %%%%%%%%%%%%%%%%%%
\newcommand{\jEigvecKS}[1]{\ensuremath{{\mathbf e}^{(#1)}}}   % jacobiam eigenvector, redefined here to	avoid conflict with chaosbook notation. Used in ksStSp chapter.


%%%%%%%%%%%%% Operators %%%%%%%%%%%%%%%%%%%%%%%
\newcommand{\PperpOp}{\mathbf{P}^{\perp}}
\newcommand{\Pperp}{P^{\perp}}

%%%%%%%%%%%%%% Penalizing loops %%%%%%%%%%%%%%%%%

\newcommand\fp[2]{{\frac{\partial #1}{\partial #2}}}
\newcommand\fder[2]{{\frac{d #1}{d #2}}}
\newcommand\fsd[2]{{d #1/d #2}}
\newcommand\fsp[2]{{\partial #1/\partial #2}}
\newcommand\fps[3]{{\frac{\partial^2 #1}{\partial #2 \partial #3}}}
\newcommand\fsps[3]{{\partial^2 #1/\partial #2 \partial #3}}

\newcommand\Js{\mathbf{\tilde{J}}}
\newcommand\JL{\mathbf{\tilde{J}}_L}
\newcommand\Jp{\mathbf{J}_p}

\newcommand\dtds{\frac{v.\tilde{v}}{v^2}}
\newcommand\hdtds{\frac{u.\tilde{u}}{u^2}}


%%%%%%%%%%%%%%%%%%%%%%%%%%%%%%%%%%%%%%%%%%%%%%%%%%%%
