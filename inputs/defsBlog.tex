% defsBlog.tex
% $Author$ $Date$

% includes experimental definitions, to be transfered to ChaosBook.org
% lines prefaced by  %DB: are already included in ChaosBook.org macros


%%%%%%%%%%%%%%% VAGGELIS MACROS %%%%%%%%%%%%%%%%%%%%%%
\newcommand{\refFigToFig}[2]{Figures~\ref{#1} to~\ref{#2}}

%%%%%%%%%%%%%%% DasBuch MACROS %%%%%%%%%%%%%%%%%%%%%%

%  \SFIG{#1}    % f_name.png (or .pdf)
%       {#2}    % short caption text
%       {#3}    % full caption text
%       {#4}    % f-figure-label
\newcommand{\SFIG}[4]{\begin{figure}[h]
              %\hspace*{-0.10\textwidth}
              \hspace*{0.10\textwidth}
              \begin{minipage}[b]{0.55\textwidth}
                      \caption[#2]{#3}
                      \label{#4}
              \end{minipage}~~~~~%
              \begin{minipage}[b]{0.40\textwidth}
                      \includegraphics[width=1.00\textwidth]{#1}
              \end{minipage}
              %\hfill
              \end{figure} }

%%%%%%%%%%%%%%% WALLY's FAVORITE MACROS %%%%%%%%%%%%%%%%%%%%%%
\def\xh{\mathbf{\hat{x}}}  \def\yh{\mathbf{\hat{y}}}  \def\zh{\mathbf{\hat{z}}}

\def\Pv{{\mathcal{P}_v}}  \def\Pe{{\mathcal{P}_\eta}}

\newcommand{\bvec}[1]{\boldsymbol{#1}}

\def\vphi{\bvec{\phi}}
\def\vpsi{\bvec{\psi}}
\def\grad{\bvec{\nabla}}
\def\vv{\bvec{v}}      % \def\vv{{\mbox {\boldmath $v$}}}
\def\vx{\bvec{x}}      % \def\vx{{\mbox {\boldmath $x$}}}
\def\vk{\bvec{k}}      % \def\vk{{\mbox {\boldmath $k$}}}
\def\uh{\hat{u}} \def\vh{\hat{v}} \def\wh{\hat{w}} \def\eh{\hat{\eta}}
\def\ub{\overline{u}} \def\wb{\overline{w}}
\def\eg{{\it e.g.\ }} \def\ie{{\it i.e.\ }}
\def\Real{{\mathbb R}}


%%%%%%%%%%%%%%% GIBSON FAVORITE MACROS %%%%%%%%%%%%%%%%%%%%%%

\newcommand{\bu}{\ensuremath{{\bf u}}}
\newcommand{\bv}{\ensuremath{{\bf v}}}
\newcommand{\bff}{\ensuremath{{\bf f}}}
\newcommand{\dbu}{\delta {\bf u}}
\newcommand{\dbv}{\delta {\bf v}}
\newcommand{\hbu}{\hat{{\bf u}}}
\newcommand{\hbv}{\hat{{\bf v}}}
\newcommand{\hu}{\hat{u}}
\newcommand{\hv}{\hat{v}}
\newcommand{\hw}{\hat{w}}
%\newcommand{\bnabla}{{\boldmath \nabla}} % what's wrong with this?
\newcommand{\be}{{\bf e}}
\newcommand{\bx}{{\bf x}}
\newcommand{\ex}{{\hat{\bf x}}} % unit vectors
\newcommand{\ey}{{\hat{\bf y}}}
\newcommand{\ez}{{\hat{\bf z}}}
\newcommand{\Om}{\Omega}    % JFG mantra

\newcommand{\bPhi}{{\bf \Phi}}
\newcommand{\bphi}{{\bf \phi}}
\newcommand{\bhphi}{{\bf \hat{\phi}}}
\newcommand{\bU}{{\bf U}}
\newcommand{\bW}{{\bf W}}
\newcommand{\lapl}{\nabla^2}
% \newcommand{\tEQ}{\ensuremath{{\text{EQ}}}}
\newcommand{\tNB}{\ensuremath{{\text{NB}}}}
\newcommand{\tLB}{\ensuremath{{\text{LB}}}}
\newcommand{\tUB}{\ensuremath{{\text{UB}}}}
\newcommand{\tLM}{\ensuremath{{\text{LM}}}}
\newcommand{\tNS}{\ensuremath{{\text{NS}}}}
\newcommand{\tCFD}{\ensuremath{{\text{CFD}}}}
\newcommand{\uEQ}{\ensuremath{\bu_{\text{\tiny EQ}}}}
\newcommand{\vEQ}{\ensuremath{\bv_{\text{\tiny EQ}}}}
\newcommand{\uLM}{\ensuremath{\bu_{\text{\tiny LM}}}}
\newcommand{\uLB}{\ensuremath{\bu_{\text{\tiny LB}}}}
\newcommand{\uNB}{\ensuremath{\bu_{\text{\tiny NB}}}}
\newcommand{\uNBtwo}{\ensuremath{\bu_{\text{\tiny NB2}}}}
\newcommand{\uUB}{\ensuremath{\bu_{\text{\tiny UB}}}}
\newcommand{\vLM}{\ensuremath{\bv_{\text{\tiny LM}}}}
\newcommand{\vLB}{\ensuremath{\bv_{\text{\tiny LB}}}}
\newcommand{\vNB}{\ensuremath{\bv_{\text{\tiny NB}}}}
\newcommand{\vUB}{\ensuremath{\bv_{\text{\tiny UB}}}}
\newcommand{\bbR}{\mathbb{R}}
\newcommand{\bbU}{\mathbb{U}}
\newcommand{\bbUsymm}{\bbU_{S}}
%\newcommand{\half}{\frac{1}{2}}
\newcommand{\pd}[2]{\frac{\partial #1}{\partial #2}}
\newcommand{\Norm}[1]{\|{#1}\|}
%\newcommand{\grad}{\boldsymbol{\nabla}}

%\newcommand{\Refl}{\ensuremath{R}}
\newcommand{\Shift}{\ensuremath{\tau}}
% \newcommand{\Shift}{\ensuremath{\mathbf{S}}}
\renewcommand{\shift}{\ensuremath{\ell}}
\newcommand{\nameit}{\ensuremath{w-}}

%%%%%%%%%%%% Abbreviations, Siminos thesis specific %%%%%%

\newcommand{\cont}{\,, \\ }
% PC Jul 3 2009: removed Siminos redefinions:
%\renewcommand{\statesp}{phase space}
%\renewcommand{\Statesp}{Phase space}
% PC Aug 20 2009: temporarily disabled ChaosBook definions:
\renewcommand{\reducedsp}{reduced state space}
\renewcommand{\Reducedsp}{Reduced state  space}
\newcommand{\Manif}{\ensuremath{\mathcal{M}}}
\newcommand{\bbUplus}{\Fix{\Dn{1}}}

\newcommand{\BCs}{boundary conditions}
\newcommand{\eps}{\epsilon}
%\newcommand{\vh}{\hat{v}}
\newcommand{\fh}{\hat{f}}
\newcommand{\Reynolds}{\textit{Re}}  % Reynolds number
\newcommand{\Rey}{\text{Re}}
\newcommand{\dt}{\Delta t}
% \renewcommand{\etal}{{\bf et. al}} %PC: this is wierdly punctuated
\newcommand{\Dx}{\xi}
\newcommand{\ignore}[1]{}
\newcommand{\comment}[1]{{\bf [*}\footnote{{\tt [** Comment:} #1 {\tt **]}}{\bf *]}}
%\newcommand{\bu}{{\bf u}}
%\newcommand{\bv}{{\bf v}}
%\newcommand{\dbu}{{\bf \delta u}}
\newcommand{\bnabla}{{\boldmath \nabla}}
%\newcommand{\grad}{{\bf \nabla}} % defined in FW's

\newcommand{\bg}{{\bf g}}
%\newcommand{\pd}[2]{\frac{\partial #1}{\partial #2}}
\newcommand{\bt}{{$\bullet$}}

%%%%%%%%%%%%%%% TEXT MACROS %%%%%%%%%%%%%%%%%%%%%%

\newcommand{\ew}{eigen\-value}
\newcommand{\Ew}{eigen\-value}
\newcommand{\ev}{eigen\-vector}
\newcommand{\Ev}{eigen\-value}
\newcommand{\ef}{eigen\-function}
\newcommand{\Ef}{eigen\-value}
\newcommand{\steady}{\textdollar~}

\newcommand{\ubranch}{upper branch}
\newcommand{\Ubranch}{Upper branch}
\newcommand{\lbranch}{lower branch}
\newcommand{\Lbranch}{Lower branch}
\newcommand{\newfp}{NU {\eqb}}
%DB: % \newcommand{\NS}{Navier-Stokes}
\newcommand{\NSe}{Navier-Stokes equation}

%%%%%%%%%%%%%%% EXPERIMENTAL EXERCISES IN NOTATION %%%%%%%%%%%%%%%%%%%%

%%%%%%%%%% FLOWS: %%%%%%%%%%%%%%%%%%%%%%%%%%%%

 %DB: \newcommand\flow[2]{{f^{#1}(#2)}}
\renewcommand\velField[1]{{F(#1)}}  % ODE velocity field
 %DB: \newcommand\invFlow{F}
 %DB: \newcommand\hflow[2]{{\hat{f}^{#1}(#2)}}
 %DB: \newcommand\timeflow{{f^t}}
 %DB: \newcommand\tflow[2]{{\tilde{f}^{#1}(#2)}}
 %DB: %\newcommand\tflow{\tilde{f}^\tau}          %RECHECK USE OF THIS!
 %DB: \newcommand\xInit{{x_0}}      %initial x
 %DB: %\newcommand\xInit{\xi}       %initial x, Spiegel notation
 %DB: \newcommand{\DOF}{D}         % Hamiltonian deegree of freedom
 %DB: \newcommand\pSpace{x}     % phase space x=(q,p) coordinate
 %DB: \newcommand\coord{q}      % configuration space p coordinate
\renewcommand{\vel}{F}      % state space velocity vector
 %DB: \newcommand{\pVeloc}{v}           % state-space velocity
 %DB: \newcommand{\para}{\parallel}
 %DB: \newcommand\multiX{x}     %multi point n-dim vector
 %DB: \newcommand\multiF{f}     %multi point n-dim vector mapping
 %DB: \newcommand{\unitVec}{\hat{n}}        %unit vector

 %DB: \newcommand{\xzero}[1]{{x_{#1}^{*}}}  %equilibrium point?
 %DB: \newcommand\stagn{q}      %equilibrium/stagnation point suffix

 %DB: \renewcommand\Im{{\rm Im\,}}
 %DB: \renewcommand\Re{{\rm Re\,}}
 %DB: \newcommand{\sign}[1]{\sigma_{#1}}
 %DB: \newcommand{\pS}{{\cal M}}       % symbol for state space
\renewcommand{\ssp}{a}             % state space point
 %DB: \newcommand{\PoincS}{{\cal P}}       % symbol for Poincare section
 %DB: \newcommand{\PoincM}{{P}}    % symbol for Poincare map
 %DB: \newcommand{\PoincC}{{U}}    % symbol for Poincare constraint function
 %DB: \newcommand{\matId}{{\bf 1}}     % matrix identity

%%%%%%%%%% LINEARIZED FLOWS: %%%%%%%%%%%%%%%%%%%%%%%%%%%%

 %DB: \newcommand{\timeStep}{{\delta \tau}} %integration step
\renewcommand{\deltaX}{{\delta a}}  %trajectory displacement

 %DB: \newcommand{\Mvar}{{A}}       % stability matrix/matrix of variations
\renewcommand{\derF}[1]{{DF |_{#1}}}    % Gibson stability matrix
 %DB: \newcommand{\jMps}{{\bf J}}      % jacobiam matrix, full phase space
\renewcommand{\derf}[2]{{Df^{#1}|_{#2}}}    % Gibson jacobiam  matrix
 %DB: \newcommand{\jMConfig}{{\bf j}}      % jacobiam matrix, configuration space
 %DB: \newcommand{\jConfig}{j}     % jacobian, configuration space
 %DB: \newcommand{\jMP}{{\bf \hat{J}}}   % jacobian matrix, Poincare return
 %DB: \newcommand{\monodromy}{{\bf M}}   % monodromy matrix, full Poincare cut

 %DB: \newcommand{\ExpaEig}{\Lambda}
 %DB: \newcommand\Lyap{\lambda}     %Lyapunov exponent
%
 %DB: \newcommand{\eigExp}{\lambda} % complex eigenexponent
 %DB: % Guckenheimer&Holmes:    lambda = alpha + i beta
 %DB: % Hirsch-Smale:       lambda = a     + i b
 %DB: % Boyce-di Prima: lambda = mu    + i nu
 %DB: \newcommand{\eigRe}{\mu}  % Re eigenexponent
 %DB: \newcommand{\eigIm}{\nu}  % Im eigenexponent

 %DB: \newcommand{\gSpace}{{\bf \theta}}   % group rotation parameters

 %DB:      %%%%%%%%%% periods: %%%%%%%%%%%%%%%%%%%%%%%%%%%%
 %DB: \newcommand\period[1]{{T_{#1}}}           %continuous cycle period
 %DB: %\newcommand\period[1]{{\tau_{#1}}}
 %DB: \newcommand{\cl}[1]{{n_{#1}}} % discrete length of a cycle, Predrag
 %DB: %\newcommand{\cl}[1]{|#1|}    % the length of a periodic orbit, Ronnie

 %DB: \newcommand{\rpprime}{{\tilde{p}}}    % relative periodic prime orbit

 %DB: %%%%%%%%%%%%%%% symmetric, asymmetric orbits: %%%%%%%%%%%%%%%%%%%%%%%%%%%%
 %DB: \newcommand{\sym}{{s}}
 %DB: \newcommand{\nsym}{{n_s}}
 %DB: \newcommand{\asym}{{a}}
 %DB: \newcommand{\nasym}{{n_a}}
 %DB: % fundamental domain:
 %DB: \newcommand{\pf}{{\tilde p}}
 %DB: \newcommand{\nf}{n_{\tilde p}}
 %DB: \newcommand{\symf}{{\tilde s}}
 %DB: \newcommand{\nsymf}{n_{\tilde s}}
