% siminos/inputs/defsThesis.tex
% $Author$ $Date$

%%%%%%%%%%%% MACROS, Froehlich specific %%%%%%%%%%
\newcommand{\MvarRed}{\ensuremath{\overline{A}}}
           % stability mat, reduced statesp
\newcommand{\toCB}{\marginpar{\footnotesize 2CB}}  % to compare with ChaosBook
\newcommand{\dotProd}[2]{\left\langle {#1}|{#2} \right\rangle}
\newcommand{\dual}[1]{{#1}^\dag}

%%%%%%%%%%% Equations, Siminos specific %%%%%%%%%%
\newcommand{\cont}{\,, \\ }

%%%%%%%%%%%%%%% REFERENCING EQUATIONS ETC. %%%%%%%%%%%%%%%%%%%%%%%%%%%%%%%
\newcommand{\refDef}[1]{Definition~\ref{#1}}
\newcommand{\refdef}[1]{definition~\ref{#1}}
\newcommand{\refFigToFig}[2]{Figures~\ref{#1} to~\ref{#2}}
\newcommand{\reffigTofig}[2]{Figures~\ref{#1} to~\ref{#2}}
\newcommand{\reffigpart}[2]{Figure~\ref{#1}(#2)}
\newcommand{\refFigpart}[2]{Figure~\ref{#1}(#2)}

%%%%%%%%%%%% MACROS, Siminos specific %%%%%%%%%%
\newcommand{\vf}{v}	%%% keep notation for vector field flexible.
                    % For the time being follow Das Buch.
\newcommand{\conj}[1]{\ensuremath{\bar{#1}}}
\newcommand{\Manif}{\ensuremath{\mathcal{M}}}
\newcommand{\Order}[1]{\mathrm{O}(#1)}

%%%%%%%% Symmetries, Siminos specific %%%%%%%%%%
\renewcommand{\shift}{\ensuremath{\ell}}
\newcommand{\Shift}{\ensuremath{\tau}}
\newcommand{\Idg}{\ensuremath{\mathbf{1}}}
\newcommand{\Cn}[1]{\ensuremath{\mathrm{C}_{#1}}}
\newcommand{\En}[1]{\ensuremath{\mathrm{E}(#1)}}
\newcommand{\Rotn}[1]{\ensuremath{R_{#1}}}
\newcommand{\Drot}{\ensuremath{\zeta}}
\newcommand{\globstab}[1]{\ensuremath{\Sigma_{#1}}} % Change to be the same as stab. Was \Sigma^\ast_{#1}
\newcommand{\Str}[1]{\ensuremath{\mathcal{S}_{#1}}} % Stratum
\newcommand{\Nlz}[1]{\ensuremath{N(#1)}}
\newcommand{\doubleperiod}[1]{{\ensuremath{\mathcal{T}_{#1}}}}

%%%%%%%%%%% Theorems, Siminos specific %%%%%%%%%%
    \ifarticle
\newtheorem{definition}{Definition}[section]
    \else
\newtheorem{definition}{Definition}[chapter]
    \fi %end of article switch
\newtheorem{theorem}[definition]{Theorem}
\newtheorem{lemma}[definition]{Lemma}
\newtheorem{proposition}[definition]{Proposition}
% \newtheorem{example}[definition]{Example}

\newcommand{\refLem}[1]{Lemma~\ref{#1}}
\newcommand{\refThe}[1]{Theorem~\ref{#1}}
\newcommand{\refPro}[1]{Proposition~\ref{#1}}
% \newcommand{\refExa}[1]{Example~\ref{#1}}


%%%%%%%%%%%%%% Solution labels %%%%%%%%%%%%%%%%%%%%
\newcommand{\EQB}[1]{\ensuremath{\mathrm{E}_{#1}}}
\newcommand{\REQB}[1]{\ensuremath{\mathrm{Q}_{#1}}} % For ODE's, use REQV from chaosbook for PDE's

% Redefine using mathrm, it is a label not a math symbol
\renewcommand{\EQV}[1]{\ensuremath{\mathrm{E}_{#1}}}
% E_0: u = 0 - trivial equilibrium
% E_1,E_2,E_3, for 1,2,3-wave equilibria
\renewcommand{\REQV}[2]{\ensuremath{\mathrm{TW}_{#1#2}}} % #1 is + or -
% TW_1^{+,-} for 1-wave traveling waves (positive and negative velocity).
\renewcommand{\PO}[1]{\ensuremath{\mathrm{PO}_{#1}}}
% PO_{period to 2-4 significant digits} - periodic orbits
\renewcommand{\RPO}[1]{\ensuremath{\mathrm{RPO}_{#1}}}
% RPO_{period to 2-4 significant digits} - relative PO.  We use ^{+,-}
% to distinguish between members of a reflection-symmetric pair.
% Gibson likes:
\renewcommand{\tEQ}{\ensuremath{\mathrm{EQ}}}

%%%%%%%%%%%%% Operators %%%%%%%%%%%%%%%%%%%%%%%
\newcommand{\PperpOp}{\mathbf{P}^{\perp}}
\newcommand{\Pperp}{P^{\perp}}

%%%%%%%%%%%%%% Penalizing loops %%%%%%%%%%%%%%%%%

\newcommand\fp[2]{{\frac{\partial #1}{\partial #2}}}
\newcommand\fder[2]{{\frac{d #1}{d #2}}}
\newcommand\fsd[2]{{d #1/d #2}}
\newcommand\fsp[2]{{\partial #1/\partial #2}}
\newcommand\fps[3]{{\frac{\partial^2 #1}{\partial #2 \partial #3}}}
\newcommand\fsps[3]{{\partial^2 #1/\partial #2 \partial #3}}

\newcommand\Js{\mathbf{\tilde{J}}}
\newcommand\JL{\mathbf{\tilde{J}}_L}
\newcommand\Jp{\mathbf{J}_p}

\newcommand\dtds{\frac{v.\tilde{v}}{v^2}}
\newcommand\hdtds{\frac{u.\tilde{u}}{u^2}}


%%%%%%%%%%%%%%%%%%%%%%%%%%%%%%%%%%%%%%%%%%%%%%%%%%%%
