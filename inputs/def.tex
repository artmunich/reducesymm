% def.tex
% $Author$ $Date$

%%%%%%%%%%%%%%%%%%%%%%%%%%%%%%%%%%%%%%%%%%%%%%%%%%%%%%%%%%%%%%%%%%%%%%%%%
%% defines macros used throughout ChaosBook and related
%%%%%%%%%%%%%%%%%%%%%%%%%%%%%%%%%%%%%%%%%%%%%%%%%%%%%%%%%%%%%%%%%%%%%%%%%

%               Predrag          9oct2009
%               Predrag         12jun2008
%               Predrag         15dec2008
%               Predrag         29oct2005
%               Predrag         13jul2005
%               Predrag         24apr2005
%               Predrag         14feb2005
%               Predrag         22jan2005
%               Predrag         16nov2004
%               Predrag         13jun2004
%               Predrag          3may2004
%               Predrag         10apr2004
%               Predrag         21feb2004
%               Predrag          4oct2003
%               Predrag         30aug2003
%               Predrag         20jun2003
%               Predrag         17jan2003
%               Predrag          6dec2002
%               Predrag          7jul2002
%               Predrag         19nov2000
%               Ronnie          23sep2000
% Predrag disabled \basedirectory machine identifier    25aug2000
% Predrag created               30oct1994

\ifpaper % prepare for B&W paper printing:
%        \newcommand{\href}[2]{{#2}}  % no hyperref
       \newcommand{\HREF}[2]{{#2}}
       \renewcommand{\color}[1]{}       % B&W
       \newcommand{\wwwcb}[1]{{\tt ChaosBook.org#1}}
       \newcommand{\wwwgt}{{\tt birtracks.eu}}
       \newcommand{\wwwQFT}[1]{{\tt ChaosBook.org/\-Field\-Theory#1}}
       \newcommand{\wwwcnsQFT}[1]{{\tt ChaosBook.org/\-Field\-Theory#1}}
       \newcommand{\weblink}[1]{{\tt #1}}
       \newcommand{\arXiv}[1]{ {\tt arXiv:#1}}
       \newcommand{\mpArc}[1]{{\tt \goodbreak mp\_arc~#1}}
\else % prepare hyperlinked pdf
        \newcommand{\wwwcb}[1]{       % keep homepage flexible:
                  {\tt \href{http://ChaosBook.org#1}
              {ChaosBook.org#1}}}
       \newcommand{\wwwgt}{{\tt \href{http://birtracks.eu}
              {birtracks.eu}}}
       \newcommand{\wwwQFT}[1]{
                  {\tt \href{http://ChaosBook.org/FieldTheory#1}
              {ChaosBook.org/\-Field\-Theory#1}}}
       \newcommand{\wwwcnsQFT}[1]{
                  {\tt \href{http://ChaosBook.org/FieldTheory#1}
              {ChaosBook.org/\-Field\-Theory#1}}}
       \newcommand{\weblink}[1]{{\tt \href{http://#1}{#1}}}
       \newcommand{\HREF}[2]{
              {\href{#1}{#2}}}
       \newcommand{\mpArc}[1]{
              {\tt \href{http://www.ma.utexas.edu/mp_arc-bin/mpa?yn=#1}
                   {\goodbreak mp\_arc~#1}}}
       \newcommand{\arXiv}[1]{
              {\tt \href{http://arXiv.org/abs/#1}{\goodbreak arXiv:#1}}}
\fi

%%%%%%%%%%%%%%%%%%%%%% QUOTATIONS %%%%%%%%%%%%%%%%%%%%%%%%%%%%%%%%%%%%%%
%
%  the learned/witty quotes at the chapter and section headings
%
\newsavebox{\bartName}
\newcommand{\bauthor}[1]{\sbox{\bartName}{\parbox{\textwidth}{\vspace*{0.8ex}
       %\hspace*{\fill}
       \hspace{2em}---\small\noindent #1}}}
\newenvironment{bartlett}{\hfill\begin{minipage}[t]{0.65\textwidth}\small}%
{\hspace*{\fill}\nolinebreak[1]\usebox{\bartName}\vspace*{1ex}\end{minipage}}
%
%  a quotation inserted into the text
%
\newenvironment{txtquote}{\begin{quotation} \small}{\end{quotation}}

\newcommand{\student}{Henri Roux}
%\newcommand{\student}{Jens J. Jensen}

%%%%%%%%%%%%%%%%%%%%%% INDEXING %%%%%%%%%%%%%%%%%%%%%%%%%%%%%%%%%%%%%%%%%
\newcommand{\indx}[1] {#1\index{#1}}    % do not need to repeat the word

\newcommand{\file}[1]{$\footnotemark\footnotetext{{\bf file} #1}$}
% PC 9sep2008 commented out (is it used?):
%\newcommand{\lecture}[2]{ \addtocontents{toc}
%           {{\scriptsize #1}{\sf\small lecture: \scriptsize #2}} }

%%%%%%%%%%%%%%% EQUATIONS %%%%%%%%%%%%%%%%%%%%%%%%%%%%%%%
\newcommand{\beq}{\begin{equation}}
\newcommand{\continue}{\nonumber \\ }
\newcommand{\nnu}{\nonumber}
\newcommand{\eeq}{\end{equation}}
\newcommand{\ee}[1] {\label{#1} \end{equation}}
\newcommand{\bea}{\begin{eqnarray}}
\newcommand{\ceq}{\nonumber \\ & & }
\newcommand{\eea}{\end{eqnarray}}
\newcommand{\barr}{\begin{array}}
\newcommand{\earr}{\end{array}}

%%%%%%%%%%%%%%% REFERENCING EQUATIONS ETC. %%%%%%%%%%%%%%%%%%%%%%%%%%%%%%%
\newcommand{\rf}     [1] {~\cite{#1}}
\newcommand{\refref} [1] {ref.~\cite{#1}}
\newcommand{\refRef} [1] {Ref.~\cite{#1}}
\newcommand{\refrefs}[1] {refs.~\cite{#1}}
\newcommand{\refRefs}[1] {Refs.~\cite{#1}}
\newcommand{\refeq}  [1] {(\ref{#1})}
\newcommand{\refeqs} [2]{(\ref{#1}--\ref{#2})}
\newcommand{\refpage}[1] {page~\pageref{#1}}
\newcommand{\reffig} [1] {figure~\ref{#1}}
\newcommand{\reffigs} [2] {figures~\ref{#1} and~\ref{#2}}
\newcommand{\refFig} [1] {Figure~\ref{#1}}
\newcommand{\refFigs} [2] {Figures~\ref{#1} and~\ref{#2}}
\newcommand{\reftab} [1] {table~\ref{#1}}
\newcommand{\refTab} [1] {Table~\ref{#1}}
\newcommand{\reftabs}[2] {tables~\ref{#1} and~\ref{#2}}
\newcommand{\refsect}[1] {sect.~\ref{#1}}
\newcommand{\refsects}[2] {sects.~\ref{#1} and \ref{#2}}
\newcommand{\refSect}[1] {Sect.~\ref{#1}}
\newcommand{\refSects}[2] {Sects.~\ref{#1} and \ref{#2}}
\newcommand{\refchap}[1] {chapter~\ref{#1}}
\newcommand{\refChap}[1] {Chapter~\ref{#1}}
\newcommand{\refchaps}[2] {chapters~\ref{#1} and~\ref{#2}}
\newcommand{\refchaptochap}[2] {chapters~\ref{#1} to~\ref{#2}}
\newcommand{\refappe}[1] {appendix~\ref{#1}}
\newcommand{\refappes}[2] {appendices~\ref{#1} and~\ref{#2}}
\newcommand{\refAppe}[1] {Appendix~\ref{#1}}
\newcommand{\refrem} [1] {remark~\ref{#1}}
\newcommand{\refexam}[1] {example~\ref{#1}}
\newcommand{\refExam}[1] {Example~\ref{#1}}
\newcommand{\refexer}[1] {exercise~\ref{#1}}
\newcommand{\refExer}[1] {Exercise~\ref{#1}}
\newcommand{\refsolu}[1] {solution~\ref{#1}}

%%%%%%%%%%%%%%  Abbreviations %%%%%%%%%%%%%%%%%%%%%%%%%%%%%%%%%%%%%%%%
%%% APS (American Physiology Society, it seems) style:
%%%     Latin or foreign words or phrases should be roman, not italic.
%%%     Insert a `hard' space after full points
%%%                                         that do not end sentences.

\newcommand{\etc}{{etc.}}       % APS
\newcommand{\etal}{{\em et al.}}    % etal in italics, APS too
\newcommand{\ie}{{i.e.}}        % APS
\newcommand{\cf}{{\em cf.\ }}     % APS
\newcommand{\eg}{{e.g.\ }}        % APS, OUP, hard space '\eg\ NextWord'
% \newcommand{\etc}{{\em etc.}}     % etcetera in italics
% \newcommand{\ie}{{that is}}       % use Latin or English?  Decide later.
% \newcommand{\cf}{{cf.}}
% \newcommand{\eg}{{\it e.g.,\ }}   % Wirzba 2sep2001

%%%%%%%%%%%%%%% ChaosBook Abbreviations %%%%%%%%%%%%%%%%%%%%%%%%

\newcommand{\evOper}{evolution oper\-ator}
\newcommand{\EvOper}{Evolution oper\-ator}
 %% \newcommand{\evOp}{Ruelle operator} %could be ``evolution'' instead?
%\newcommand{\FPoper}{Frobenius-Perron oper\-ator}
\newcommand{\FPoper}{Perron-Frobenius oper\-ator} % Pesin's ordering
\newcommand{\FP}{Perron-Frobenius}
\newcommand{\statesp}{state space}
\newcommand{\Statesp}{State space}
\newcommand{\fixedpnt}{fixed point}
\newcommand{\Fixedpnt}{fixed point}
\newcommand{\maslov}{topological}
\newcommand{\Maslov}{Topological}
%\newcommand{\Maslov}{Keller-Maslov}
\newcommand{\jacobian}{Jacobian}        % determinant
% \newcommand{\jacobianM}{fundamental matrix} % no known standard name?
% \newcommand{\jacobianMs}{fundamental matrices}  %
% \newcommand{\JacobianM}{Fundamental matrix} %
% \newcommand{\JacobianMs}{Fundamental matrices}  %
\newcommand{\jacobianM}{Jacobian matrix}  % back to Predrag's name 20oct2009
\newcommand{\jacobianMs}{Jacobian matrices}   % matrices
\newcommand{\JacobianM}{Jacobian matrix} %
\newcommand{\JacobianMs}{Jacobian matrices}  %
\newcommand{\FloquetM}{Floquet matrix} % specialized to periodic orb
\newcommand{\FloquetMs}{Floquet matrices}  %
% \newcommand{\stabmat}{matrix of variations}   % Arnold, says Vattay
\newcommand{\stabmat}{stability matrix}     % stability matrix, velocity gradients
\newcommand{\Stabmat}{Stability matrix}     % Stability matrix
\newcommand{\stabmats}{stability matrices}
\newcommand{\monodromyM}{monodromy matrix} % monodromy matrix, Poincare cut
\newcommand{\MonodromyM}{Monodromy matrix} % monodromy matrix, Poincare cut
\newcommand{\dzeta}{dyn\-am\-ic\-al zeta func\-tion}
\newcommand{\Dzeta}{Dyn\-am\-ic\-al zeta func\-tion}
\newcommand{\tzeta}{top\-o\-lo\-gi\-cal zeta func\-tion}
\newcommand{\Tzeta}{Top\-o\-lo\-gi\-cal zeta func\-tion}
\newcommand{\BERzeta}{BER zeta func\-tion}
%\newcommand{\tzeta}{Artin-Mazur zeta func\-tion} %alternative to topological
\newcommand{\qS}{semi\-classical zeta func\-tion}
%\newcommand{\qS}{Gutz\-willer-Voros zeta func\-tion}
\newcommand{\Gt}{Gutz\-willer trace formula}
\newcommand{\Fd}{spec\-tral det\-er\-min\-ant}
%\newcommand{\fd}{spec\-tral det\-er\-min\-ant} %in many articles
\newcommand{\FD}{Spec\-tral det\-er\-min\-ant}
\newcommand{\cFd}{semiclass\-ic\-al spec\-tral det\-er\-mi\-nant}
\newcommand{\cFD}{Semiclass\-ic\-al spec\-tral det\-er\-mi\-nant}
% \newcommand{\cFd}{semiclass\-ic\-al Fred\-holm det\-er\-mi\-nant}
\newcommand{\Vd}{Vattay det\-er\-mi\-nant}
\newcommand{\cycForm}{cycle averaging formula}
\newcommand{\CycForm}{Cycle averaging formula}
\newcommand{\freeFlight}{mean free flight time}
\newcommand{\FreeFlight}{Mean free flight time}
\newcommand{\pdes}{partial differential equations}
\newcommand{\Pdes}{Partial differential equations}
\newcommand{\dof}{dof}         % Hamiltonian deegree of freedom
% \newcommand{\dof}{deegree of freedom}
\newcommand\Poincare{Poincar\'e }

%%%%%%%%%%%%%%% VECTORS, MATRICES %%%%%%%%%%%%%%%%%%%%%%%%%%%%%%%%%%%%%%%%%
% Commented out AMS-style pmatrix, which is incompatible with TeX/LaTeX pmatrix
% used throughout dasbuch. Fri Oct 12 15:51:03 EDT 2007

\newcommand{\MatrixII}[4]{\left(
\begin{array}{cc}
{#1}  &  {#2} \\
{#3}  &  {#4} \end{array} \right)}
% a problem with \pmatrix 12oct 2007
%\newcommand{\MatrixII}[4]{
%   \begin{pmatrix}{#1}  &  {#2} \\
%                  {#3}  &  {#4} \end{pmatrix}}

\newcommand{\MatrixIII}[9]{
  \pmatrix{ {#1}  &  {#2} &  {#3} \cr
            {#4}  &  {#5} &  {#6} \cr
            {#7}  &  {#8} &  {#9}
          }               }
% \newcommand{\MatrixIII}[9]{
%    \begin{pmatrix} {#1}  &  {#2} &  {#3} \\
%                    {#4}  &  {#5} &  {#6} \\
%                    {#7}  &  {#8} &  {#9}  \end{pmatrix}}


% \newcommand{\transpVectorII}[2]{
%    \begin{pmatrix}{#1}  &  {#2}  \end{pmatrix}}

\newcommand{\VectorII}[2]{\left(
\begin{array}{cc}
{#1}  &  {#2} \end{array} \right)}
%
%\newcommand{\VectorII}[2]{
%  \pmatrix{ {#1} \cr {#2}}
%}

% \newcommand{\VectorII}[2]{
%    \begin{pmatrix} {#1} \\
%                    {#2}  \end{pmatrix}}

% \newcommand{\VectorIII}[3]{
%    \begin{pmatrix} {#1} \\
%                    {#2} \\
%                    {#3} \end{pmatrix}}

\newcommand{\transpVectorII}[2]{
  \pmatrix{ {#1}  &  {#2}}
}

\newcommand{\VectorIII}[3]{
  \pmatrix{ {#1} \cr
            {#2} \cr
            {#3}
          }
}

\newcommand{\combinatorial}[2]{ {#1 \choose #2}}

%%%%%%%%%%%%%%% Sundry symbols within math eviron.: %%%%%%%%%%%%

\newcommand{\obser}{\ensuremath{a}}     % an observable from phase space to R^n
\newcommand{\Obser}{\ensuremath{A}}     % time integral of an observable
\newcommand{\onefun}{\iota} % the function that returns one no matter what
\newcommand{\defeq}{=}      % the different equal for a definition
\newcommand {\deff}{\stackrel{\rm def}{=}}
\newcommand{\reals}{\mathbb{R}}
\newcommand{\complex}{\mathbb{C}}
\newcommand{\integers}{\mathbb{Z}}
\newcommand{\rationals}{\mathbb{Q}}
\newcommand{\naturals}{\mathbb{N}}
\newcommand{\LieD}{{{\cal L}\!\!\llap{-}\,\,}}  % {{\pound}} % Lie Derivative
\newcommand{\half}{{\scriptstyle{1\over2}}}
\newcommand{\pde}{\partial}
\newcommand{\pdfrac}[2]{{\partial #1 \over \partial #2}}
\renewcommand\Im{\ensuremath{{\rm Im}\,}}
\renewcommand\Re{\ensuremath{{\rm Re}\,}}
\renewcommand{\det}{\mbox{\rm det}\,}
\newcommand{\Det}{\mbox{\rm Det}\,}
\newcommand{\tr}{\mbox{\rm tr}\,}
\newcommand{\Tr}{\mbox{\rm tr}\,}
%\newcommand{\Tr}{\mbox{Tr}\,}
\newcommand{\sign}[1]{\sigma_{#1}}
%\newcommand{\sign}[1]{{\rm sign}(#1)}
\newcommand{\mInv}{{I}}                 % material invariant
\newcommand{\msr}{\ensuremath{\rho}}                % measure
\newcommand{\Msr}{{\mu}}                % coarse measure
\newcommand{\dMsr}{{d\mu}}              % measure infinitesimal
\newcommand{\SRB}{{\rho_0}}             % natural measure
\newcommand{\vol}{{V}}                  % volume of i-th tile
\newcommand{\prpgtr}[1]{\delta\negthinspace\left( {#1} \right)}
%\newcommand{\Zqm}{\ensuremath{Z_{qm}}}         % Gutz-Voros zeta function
\newcommand{\Zqm}{\ensuremath{\det(\hat{H} - E)_{sc} }} % semicls spectr. det:
\newcommand{\Fqm}{\ensuremath{F_{qm}}}
\newcommand{\zfct}[1]{\zeta ^{-1}_{#1}}
\newcommand{\zetaInv}{\ensuremath{1/\zeta}}
% \newcommand{\zetaInv}{{\zeta^{-1}}}
\newcommand{\zetatop}{\ensuremath{1/\zeta_{\mbox{\footnotesize top}} }}
\newcommand{\zetaInvBER}[1]{1/\zeta_{\mbox{\footnotesize BER}}(#1)}
\newcommand{\BER}[1]{{\mbox{\footnotesize BER}}} % Baladi-Ruelle-Eckmann
\newcommand{\eigCond}{\ensuremath{F}}           % eigenvalue cond. function
\newcommand{\expct}    [1]{\left\langle {#1} \right\rangle}
\newcommand{\spaceAver}[1]{\left\langle {#1} \right\rangle}
\newcommand{\timeAver} [1]{\overline{#1}}
\newcommand{\norm}[1]{\left\Arrowvert \, #1 \, \right\Arrowvert}
\newcommand{\pS}{\ensuremath{{\cal M}}}          % symbol for state space
\newcommand{\ssp}{\ensuremath{x}}                % state space point
\newcommand{\tissp}{\tilde{\Delta\ssp}} % Rytis \CostFct
\newcommand{\pSpace}{x}       % Hamiltonian phase space x=(q,p) coordinate
\newcommand{\coord}{q}        % configuration space p coordinate
\newcommand{\DOF}{\ensuremath{D}}          % Hamiltonian deegree of freedom
\newcommand{\NWS}{\ensuremath{\Omega}}     % symbol for the non--wandering set
\newcommand{\AdmItnr}{\Sigma}      % set of admissible itineraries
\newcommand{\intM}[1]{{\int_\pS{\!d #1}\:}} %phase space integral
\newcommand{\Cint}[1]{\oint\frac{d#1}{2 \pi i}\;} %Cauchy contour integral
\newcommand{\PoincS}{{\cal P}}     % symbol for Poincare section
\newcommand{\PoincM}{\ensuremath{P}}       % symbol for Poincare map
\newcommand{\PoincC}{\ensuremath{U}}       % symbol for Poincare constraint function
\newcommand{\arc}{\ensuremath{s}}          % symbol for billiard wall arc
\newcommand{\mompar}{\ensuremath{p}}       % billiard wall parall. momentum
\newcommand{\restCoeff}{\ensuremath{\gamma}}  % billiard wall restitution coeff
\newcommand{\timeIn}[1]{{t^{-}_{#1}}} % billiard wall time of arrival
\newcommand{\timeOut}[1]{{t^{+}_{#1}}}   % billiard wall time of departure
%\newcommand{\PoincS}{\partial{\cal M}}          % billiard Poincare section
\newcommand{\Lop}{\ensuremath{{\cal L}}}       % evolution operator
\newcommand{\Uop}{\ensuremath{{\cal K}}}       % Koopman operator, Driebe notation
\newcommand{\Aop}{\ensuremath{{\cal A}}}       % evolution generator
\newcommand{\TrOp}{\ensuremath{{\cal T}}}       % transfer operator, like in statmech
\newcommand{\matId}{\ensuremath{{\bf 1}}}      % matrix identity
\newcommand{\eigenvL}{\ensuremath{s}}      % evolution operator eigenvalue
\newcommand{\eigenvG}{\ensuremath{m}}      % compact group eigenvalues
\newcommand{\inFix}[1]{{\in \mbox{\footnotesize Fix}f^{#1}}}
\newcommand{\inZero}[1]{{\in \mbox{\footnotesize Zero} \, f^{#1} }}
\newcommand{\xzero}[1]{{x_{#1}^{*}}}
\newcommand{\fractal}{{\cal F}}
\newcommand{\contract}{F}
% \newcommand{\presentation}{P} % PC commented out 7sep2008
\newcommand{\orderof}[1]{o(#1)} % Rytis 22mar2005

     %%%%%%%%%% flows: %%%%%%%%%%%%%%%%%%%%%%%%%%%%
\newcommand\flow[2]{{f^{#1}(#2)}}
\newcommand{\vel}{\ensuremath{v}}   % state space velocity
\newcommand\velField[1]{{v(#1)}}    % ODE velocity field
\newcommand\invFlow{F}
\newcommand\hflow[2]{{\hat{f}^{#1}(#2)}}
\newcommand\timeflow{{f^t}}
\newcommand\tflow[2]{{\tilde{f}^{#1}(#2)}}
%\newcommand\tflow{\tilde{f}^\tau}        %RECHECK USE OF THIS!
\newcommand\xInit{{x_0}}        %initial x
%\newcommand\xInit{\xi}     %initial x, Spiegel notation
\newcommand{\para}{\parallel}
\newcommand\multiX{x}       %multi point n-dim vector
\newcommand\multiF{f}       %multi point n-dim vector mapping

   %%%%%%% 3D physical flow
\newcommand{\stagp}{stagnation point}
\newcommand{\Stagp}{Stagnation point}
\newcommand{\relstagp}{traveling stagnation point}
\newcommand{\Relstagp}{Traveling stagnation point}
\newcommand{\velgradmat}{velocity gradients matrix}

   %%%%%%%% Siminos thesis %%%%%%%%%%%%%%%%%%%%%%%%%%%%
\newcommand{\Le}{Lorenz equations}
\newcommand{\rLor}{\rho}    % parameter r in Lorenz paper
\newcommand{\cLe}{complex Lorenz equations}
\newcommand{\cLf}{complex Lorenz flow}
\newcommand{\CLe}{Complex Lorenz equations}
\newcommand{\CLf}{Complex Lorenz flow}
\newcommand{\RerCLor}{\rho_1}    % real      part of parameter r, CLe
\newcommand{\ImrCLor}{\rho_2}    % imaginary part of parameter r, CLe
% \newcommand{\AGHe}{Armbruster-Guckenheimer-Holmes flow}

     %%%%%%%%%% periods: %%%%%%%%%%%%%%%%%%%%%%%%%%%%
\newcommand\period[1]{{\ensuremath{T_{#1}}}}         %continuous cycle period
%\newcommand\period[1]{{\tau_{#1}}}
\newcommand{\cl}[1]{{\ensuremath{n_{#1}}}}   % discrete length of a cycle, Predrag
%\newcommand{\cl}[1]{|#1|}  % the length of a periodic orbit, Ronnie
\newcommand{\nCutoff}{N}    % maximal cycle length
                % maximal stability cutoff:
\newcommand{\stabCutoff}{\ExpaEig_{\mbox{\footnotesize max}}}
\newcommand{\timeSegm}[1]{{\tau_{#1}}}      %billiard segment time period
\newcommand{\timeStep}{\ensuremath{{\delta \tau}}}  %integration step
\newcommand{\deltaX}{\ensuremath{{\delta x}}}       %trajectory displacement
\newcommand{\unitVec}{\ensuremath{\hat{n}}}     %unit vector

\newcommand{\Mvar}{\ensuremath{A}}  % stability matrix
\newcommand{\derF}[1]{\ensuremath{A(#1)}}   % Predrag stability matrix
 %\newcommand{\derF}[1]{{DF |_{#1}}}        % Gibson stability matrix
\newcommand{\jMps}{\ensuremath{J}}   % jacobian matrix, phase space/state space
% \newcommand{\jMps}{\ensuremath{{\bf J}}}  % bold fundamental matrix phase space
\newcommand{\derf}[2]{\ensuremath{{J}^{#1}(#2)}}    % Predrag fundamental matrix
% \newcommand{\derf}[2]{\ensuremath{{\bf J}^{#1}(#2)}}  % Predrag bold fundamental matrix
 % \newcommand{\derf}[2]{{Df^{#1}|_{#2}}}   % Gibson fundamental matrix
\newcommand{\jMConfig}{\ensuremath{{\bf j}}}    % fundamental matrix, configuration space
\newcommand{\jConfig}{\ensuremath{j}}      % jacobian, configuration space
\newcommand{\jMP}{\ensuremath{\hat{J}}}   % jacobian matrix, Poincare return
% \newcommand{\jMP}{\ensuremath{{\bf \hat{J}}}}   % bold jacobian matrix, Poincare return
\newcommand{\monodromy}{\ensuremath{M}}   % monodromy matrix, full Poincare cut
% \newcommand{\monodromy}{\ensuremath{{\bf M}}}   % bold monodromy matrix, full Poincare cut
                   % Fredholm det jacobian weight:
%\newcommand{\jEigvec}[1]{\ensuremath{{\bf e}^{(#1)}}}   % jacobiam eigenvector
%\newcommand{\jEigvecT}[1]{\ensuremath{{\bf e}_{(#1)}}}   % jacobiam eigenvector transposed
\newcommand{\jEigvec}[1][]{\ensuremath{{\bf e}^{(#1)}}} % jacobiam eigenvector
\newcommand{\jEigvecT}[1][]{\ensuremath{{\bf e}_{(#1)}}}   % jacobiam eigenvector transposed
\newcommand{\oneMinJ}[1]
           {\left|\det\!\left(\matId-\monodromy_p^{#1}\right)\right|}
\newcommand{\maslovInd}{\ensuremath{m}}        % Maslov index
\newcommand{\ExpaEig}{\ensuremath{\Lambda}}
\newcommand{\Lyap}{\ensuremath{\lambda}}            %Lyapunov exponent

%%   optional parameter comes in [\ldots], for example
%%   \newcommand\eigRe[1][ ]{\ensuremath{\mu_{#1}}}
%%   no subscript: \eigRe\
%%   with subscript j: \eigRe[j]
%%
% \newcommand{\eigExp}[1][ ]{\ensuremath{\lambda_{#1}}}   % complex eigenexponent
%%  Guckenheimer&Holmes:  lambda = alpha + i beta
%%  Hirsch-Smale:         lambda = a     + i b
%%  Boyce-di Prima:       lambda = mu    + i nu
% \newcommand{\eigRe}[1][ ]{\ensuremath{\mu_{#1}}}    % Re eigenexponent
% \newcommand{\eigIm}[1][ ]{\ensuremath{\nu_{#1}}}    % Im eigenexponent

\newcommand{\eigExp}[1][]{
     \ifthenelse{\equal{#1}{}}{\ensuremath{\lambda}}{\ensuremath{\lambda^{(#1)}}}}
\newcommand{\eigRe}[1][]{
     \ifthenelse{\equal{#1}{}}{\ensuremath{\mu}}{\ensuremath{\mu^{(#1)}}}}
\newcommand{\eigIm}[1][]{
     \ifthenelse{\equal{#1}{}}{\ensuremath{\omega}}{\ensuremath{\omega^{(#1)}}}}

\newcommand\LyapTime{T_{\mbox{\footnotesize Lyap}}} %Lyapunov time
\newcommand{\hatx}{{\hat{x}}}
% \newcommand{\hatx}{{\hat{x}_t}}               %RECHECK USE OF THIS!
\newcommand{\hatp}{{\hat{x}_p}}
\newcommand{\phat}{{\hat{p}}}
\newcommand{\curvR}{\rho}           %billiard curvature
\newcommand{\dz}{{\delta z}}
\newcommand{\dth}{{\delta \theta}}
\newcommand{\delh}{{\delta h}}
\newcommand{\NN}{{\cal N}}

%%%%%%%%%%%%%%% symbolic dynamics %%%%%%%%%%%%%%%%%%%%%%%%%%%%%%%%%%
\newcommand{\MarkGraph}{Transition graph} % following Yorke
\newcommand{\markGraph}{transition graph} % following Yorke
% \newcommand{\MarkGraph}{Markov graph}
\newcommand{\admissible}{admissible}
\newcommand{\Admissible}{Admissible}
\newcommand{\inadmissible}{inadmissible}
\newcommand{\cycle}[1]{\ensuremath{\overline{#1}}}
\newcommand{\cycpt}{_{p,m}}
\newcommand\sumprime{\mathop{{\sum}'}}
\newcommand{\pseudos}{\pi}
% \newcommand{\pseudos}{{p_1+p_2+\dots+p_k}}
% \newcommand{\pseudos}{{\{p_1 p_2 \dots p_k\}}}
\newcommand{\block}[1]{\ensuremath{#1}} % PC 07sep2008: conflict with beamer
\newcommand{\prune}[1]{\ensuremath{\_{#1}\_}}        % fits into math env.
%\newcommand{\prune}[1]{\ldots{#1}\ldots}
%\newcommand{\strng}[1]{$\_#1\_$}    % fits into text without $'s
% replaced \strng by \prune throughout
\newcommand{\biinf}[2]{\ensuremath{\cdots#1.#2\cdots}}
\newcommand{\rctngl}[2]{\ensuremath{[#1.#2]}}
\newcommand{\BKsym}[1]{\ensuremath{S_{#1}}}
\newcommand{\Ksym}[1]{{\ensuremath{\sigma_{#1}}}}
\newcommand{\Ssym}[1]{{\ensuremath{s_{#1}}}}
\newcommand{\gmax}{\ensuremath{\hat{\gamma}}}
\newcommand{\Spast}{\ensuremath{S^\textrm{-}}}       % past itenerary
\newcommand{\Sfuture}{\ensuremath{S^\textrm{\scriptsize +}}} % future itenerary
\newcommand{\Sbiinf}{\ensuremath{S}}             % biinf. itenerary
\newcommand{\str}{\ensuremath{\epsilon_{1},\epsilon_{2}, \ldots} } % Ronnie's problems

%%%%%%%%%%%%%%% Ronnie's problems %%%%%%%%%%%%%%%%%%%%%%%%%%%%%%%%
\newcommand{\estr}[1] {\epsilon_{1},\epsilon_{2}, \ldots, \epsilon_{#1}}
\newcommand{\eestr}[2] {#1,\epsilon_{1},\epsilon_{2}, \ldots,\epsilon_{#2}}

%%%%%%%%%%%% loopDef.tex, defCrete.tex specific %%%%%%%%%%%%%
% Predrag   defCrete.tex             4mar2003
% Predrag   loopDefs.tex            10jul2003
\newcommand{\descent}{Newton descent}
\newcommand{\Descent}{Newton Descent}
\newcommand{\CostFct}{Cost function}    % functional to minimize
\newcommand{\costFct}{cost function}    % functional to minimize
\newcommand{\costF}{F^2}        % cost function,
\newcommand{\Loop}{L}
\newcommand{\pVeloc}{v}         % phase-space velocity
\newcommand{\lSpace}{\tilde{x}}     % a point on a loop
\newcommand{\lVeloc}{\tilde{v}}     % loop tangent
\newcommand{\damp}{\Delta\tau}      % descrete fictitous time step
% \newcommand{\pSpaceDer}[1]{x^{(#1)}}
% \newcommand{\lSpaceDer}[1]{\tilde{x}^{(#1)}}

%%%%%%%%%%%%%% ks.tex specific %%%%%%%%%%%%%%%%%%%%%%%%%%%%
\newcommand{\KS}{Kuramoto-Sivashinsky}
\newcommand{\KSe}{Kuramoto-Sivashinsky equation}
\newcommand{\pCf}{plane Couette flow}
\newcommand{\PCf}{Plane Couette flow}
\newcommand{\dmn}{-dimensional}  %  experimental 220ct2009
%\newcommand{\dmn}{\ensuremath{d}}  %  n-dimensional
%\newcommand{\dmn}{\ensuremath{\!-\!d}}  %  n-dimensional
\newcommand{\expctE}{\ensuremath{E}}    % E space averaged
\newcommand{\tildeL}{\ensuremath{\tilde{L}}}
\newcommand{\EQV}[1]{\ensuremath{EQ_{#1}}} %experimental
% \newcommand{\EQV}[1]{\ensuremath{q_{#1}}} %ChaosBook
% \newcommand{\EQV}[1]{\ensuremath{E_{#1}}} %Ruslan
% E_0: u = 0 - trivial equilibrium
% E_1,E_2,E_3, for 1,2,3-wave equilibria
\newcommand{\REQV}[2]{\ensuremath{TW_{#1#2}}} % #1 is + or -
% TW_1^{+,-} for 1-wave traveling waves (positive and negative velocity).
\newcommand{\PO}[1]{\ensuremath{PO_{#1}}}
% PO_{period to 2-4 significant digits} - periodic orbits
\newcommand{\RPO}[1]{\ensuremath{RPO_{#1}}}
% RPO_{period to 2-4 significant digits} - relative PO.  We use ^{+,-}
% to distinguish between members of a reflection-symmetric pair.
% Gibson likes:
\newcommand{\tEQ}{\ensuremath{{EQ}}}

%%%%%%%%%%%%%%% Lorentz gas section %%%%%%%%%%%%%%%%%%%%%%%%%%%%%%%%
\def\hn{\hat n}
\newcommand\hM{\hat \pS}
%\def\hM{\widehat M}
\newcommand\hx{\hat x}
\def\tx{\tilde x}
\def\tpk{_{\tilde p,k}}
\def\tpk{}              %why redefined?
\def\ttime{\sigma_{\tilde{p}}}

%%%%%%%%%%%%%%% Henon map specific %%%%%%%%%%%%%%%%%%%%%%%%%%%%%%%%%%
\newcommand{\fullTent}{full tent map}
\newcommand{\FullTent}{Full tent map}
% \newcommand{\fullTent}{Ulam tent map}
% \newcommand{\FullTent}{Ulam tent map}
\newcommand{\logisticm}{quadratic map}
\newcommand{\Logisticm}{Quadratic map}
\newcommand{\stretchf}{`stretch \&\ fold'}
\newcommand{\Stretchf}{`Stretch \&\ fold'}
\newcommand{\ofm}{once-folding map}
\newcommand{\Ofm}{Once-folding map}
\newcommand{\mHt}{map of the H\'enon type}
\newcommand{\mHts}{maps of the H\'enon type}
\newcommand{\MHts}{Maps of the H\'enon type}
\newcommand{\opres}{orientation preserving}
\newcommand{\Opres}{Orientation preserving}
\newcommand{\orev}{orientation reversing}
\newcommand{\Orev}{Orientation reversing}
\newcommand{\nws}{non--wandering set}
\newcommand{\stranges}{non--wandering set}
%\newcommand{\stranges}{strange set}
\newcommand{\ki}{kneading value}
\newcommand{\Ki}{Kneading value}
\newcommand{\ks}{kneading sequence}
\newcommand{\turn}{turning point}    % {turnback} ??
\newcommand{\Turn}{Turning point}    % {Turnback} ??
\newcommand{\pturn}{primary turning point}    % {turnback} ??
\newcommand{\Pturn}{Primary turning point}    % {Primary turnback} ??
\newcommand{\topc}{topological coordinate}
\newcommand{\Topc}{Topological coordinate}
\newcommand{\critVal}{f(x_c)}
\newcommand{\topcv}{maximal value}
\newcommand{\Topcv}{Maximal value}
\newcommand{\toppar}{topological parameter}
\newcommand{\toppp}{topological parameter plane}
\newcommand{\topp}{symbol square}
\newcommand{\Topp}{Symbol square}
\newcommand{\bimappr}{bimodal approximation}
\newcommand{\Bimappr}{Bimodal approximation}
\newcommand{\henappr}{bimodal approximation}
\newcommand{\fourfa}{four-folds approximation}
\newcommand{\Fourfa}{Four-folds approximation}
\newcommand{\snbif}{saddle-node bifurcation}
\newcommand{\Snbif}{Saddle-node bifurcation}

%%%%%%%% Noisy stuff %%%%%%%%%%%%%%%%%%%%%%%%%%%%
\newcommand{\Fokker}{Fokker-Planck}
% \newcommand{\Fokker}{Fokker-Planck  oper\-ator}
\newcommand{\DiffC}{\ensuremath{D}}         % diffusion constant
% \newcommand{\Lnoise}[1]{{\cal L}^{#1}}    % noisy evolution operator, Lippolis
\newcommand{\Lnoise}[1]{{\cal L}_D^{#1}}    % noisy evolution operator, ChaosBook
\newcommand{\Lmat}[1]{{{\bf L}_{#1}}}      % evolution matrix
\newcommand{\orbitDist}{{z}}     % Langevin distance from orbit point

%%%%%%%% Siminos macros %%%%%%%%%%%%%%%%%%%%%%%%%%%%%%
\newcommand{\Rls}[1]{\ensuremath{\mathbb{R}^{#1}}}
%\newcommand{\Idg}{\ensuremath{\mathbf{1}}}
%\newcommand{\Clx}[1]{\ensuremath{\mathbb{C}^{#1}}}
%\newcommand{\conj}[1]{\ensuremath{\bar{#1}}}
%\newcommand{\trace}{\mbox{\rm trace}\,}
%\newcommand{\On}[1]{\ensuremath{\mathbf{O}(#1)}}
\newcommand{\Un}[1]{\ensuremath{\textrm{U}(#1)}}         % in DasBuch
\newcommand{\On}[1]{\ensuremath{\textrm{O}(#1)}}
%\newcommand{\SOn}[1]{\ensuremath{\mathbf{SO}(#1)}} % in Siminos thesis
\newcommand{\SOn}[1]{\ensuremath{\textrm{SO}(#1)}}         % in DasBuch
%\newcommand{\Dn}[1]{\ensuremath{\mathbf{D}_{#1}}    % in Siminos thesis
\newcommand{\Dn}[1]{\ensuremath{\textrm{D}_{#1}}}              % in DasBuch
%\newcommand{\Zn}[1]{\ensuremath{\mathbf{Z}_{#1}}}    % in Siminos thesis
\newcommand{\Zn}[1]{\ensuremath{\textrm{C}_{#1}}}              % in DasBuch
%\newcommand{\Ztwo}{\ensuremath{\mathbf{Z}_2}}      % in Siminos thesis
\newcommand{\Ztwo}{\ensuremath{\textrm{C}_2}}                % in DasBuch
%\newcommand{\Refl}{\ensuremath{\kappa}}            % Siminos uses R for rotations.
\newcommand{\Refl}{\ensuremath{\sigma}}             % in DasBuch
%\newcommand{\Shift}{\ensuremath{\tau}}
\newcommand{\Rot}[1]{\ensuremath{C^{#1}}}           % in DasBuch, e.g. C^{1/3}
%\newcommand{\Rot}[1]{\ensuremath{R(#1)}}           % Siminos uses R for rotations.
%\newcommand{\Drot}{\ensuremath{\zeta}}
%\newcommand{\Lg}{\mathcal{G}}
%\newcommand{\stab}[1]{\ensuremath{\Sigma_{#1}}}
\newcommand{\stab}[1]{\ensuremath{G_{#1}}}
\newcommand{\shift}{\ensuremath{d}}
\newcommand{\gSpace}{\ensuremath{{\bf \theta}}}   % group rotation parameters
\newcommand{\velRel}{\ensuremath{c}}    % relative state velocity
\newcommand{\Fix}[1]{\ensuremath{\mathrm{Fix}\left(#1\right)}}

\newcommand{\pSRed}{\ensuremath{\bar{\cal M}}} % reduced state space
\newcommand{\sspRed}{\ensuremath{y}}    % reduced state space point, experiment
% \newcommand{\sspRed}{\ensuremath{\bar{x}}}    % reduced state space point
% \newcommand{\velRed}{\ensuremath{\bar{v}}}    % PC reduced state space velocity
\newcommand{\velRed}{\ensuremath{u}}    % ES reduced state space velocity

\newcommand{\slicep}{{\ensuremath{y'}}}   % slice-fixing point, experimental
% \newcommand{\slicep}{\ensuremath{\ssp'}}   % slice-fixing point
%\newcommand{\sliceTan}[1]{\ensuremath{t_{#1}(y')}}    % tangent at slice-fixing, experimental
\newcommand{\sliceTan}[1]{\ensuremath{t'_{#1}}}    % group orbit tangent at slice-fixing
\newcommand{\groupTan}{\ensuremath{t}}    % group orbit tangent
%\newcommand{\Group}{\ensuremath{\Gamma}}    % Siminos Lie group
\newcommand{\Group}{\ensuremath{G}}         % Predrag Lie or discrete group
%\newcommand{\Lg}{\mathfrak{a}}             % Siminos Lie algebra generator
\newcommand{\Lg}{\ensuremath{\mathbf{T}}}   % Predrag Lie algebra generator
%\newcommand{\LieEl}{\ensuremath{\mathbb{G}}}  % Wiczek project Lie group element
\newcommand{\LieEl}{\ensuremath{g}}  % Predrag Lie group element

%%%%%%%%%%%%%%% symmetric, asymmetric orbits: %%%%%%%%%%%%%%%%%%%%%%%%%%%%
\newcommand{\sym}{{s}}
\newcommand{\nsym}{{n_s}}
\newcommand{\asym}{{a}}
\newcommand{\nasym}{{n_a}}
% fundamental domain:
\newcommand{\pf}{{\tilde p}}
\newcommand{\nf}{n_{\tilde p}}
\newcommand{\symf}{{\tilde s}}
\newcommand{\nsymf}{n_{\tilde s}}
%\newcommand\stagn{*}        %equilibrium/stagnation point suffix
\newcommand\stagn{q}      %equilibrium/stagnation point suffix
\newcommand{\rpprime}{{\tilde{p}}}  % relative periodic prime orbit

%%%%%%%%%%%%%%% relative periodic orbits: %%%%%%%%%%%%%%%%%%%%%%%%%%%%
\newcommand{\po}{periodic orbit}
\newcommand{\Po}{Periodic orbit}
\newcommand{\rpo}{relative periodic orbit}
%   \newcommand{\rpo}{equivariant periodic orbit}
\newcommand{\Rpo}{Relative periodic orbit}
%   \newcommand{\Rpo}{Equivariant periodic orbit}
\newcommand{\eqv}{equilibrium}
\newcommand{\Eqv}{Equilibrium}
\newcommand{\eqva}{equilibria}
\newcommand{\Eqva}{Equilibria}
\newcommand{\reqv}{relative equilibrium}
%   \newcommand{\reqv}{equivariant equilibrium}
%   \newcommand{\reqv}{travelling wave}
\newcommand{\Reqv}{Relative equilibrium}
%   \newcommand{\Reqv}{Equivariant equilibrium}
%   \newcommand{\Reqv}{travelling wave}
\newcommand{\reqva}{relative equilibria}
%   \newcommand{\reqva}{equivariant equilibria}
\newcommand{\Reqva}{Relative equilibria}
%   \newcommand{\Reqva}{Equivariant equilibria}
\newcommand{\equilibrium}{equilibrium}
\newcommand{\equilibria}{equilibria}
\newcommand{\Equilibria}{Equilibria}
% \newcommand{\equilibrium}{steady state}
% \newcommand{\equilibria}{steady states}
% \newcommand{\Equilibria}{Steady states}
% \newcommand{\reducedsp}{orbit space}
% \newcommand{\Reducedsp}{Orbit space}
\newcommand{\reducedsp}{reduced state space}
\newcommand{\Reducedsp}{Reduced state space}
\newcommand{\fixedsp}{fixed-point subspace}
\newcommand{\Fixedsp}{Fixed-point subspace}
\newcommand{\csection}{cross-section}
\newcommand{\Csection}{Cross-section}
\newcommand{\slice}{slice}
\newcommand{\Slice}{Slice}
\newcommand{\mslices}{method of slices}
\newcommand{\Mslices}{Method of slices}
\newcommand{\mframes}{method of moving frames}
\newcommand{\Mframes}{Method of moving frames}
\newcommand{\Hec}{Heteroclinic connection}
\newcommand{\hec}{heteroclinic connection}
\newcommand{\HeC}{Heteroclinic Connection}

%%%%%%%%%%%%%% Quantum mechanical stuff %%%%%%%%%%%%%%%%%%%%%%%%%%%%
\newcommand{\HamPrincFct}[4]{R_{#4}({#1},{#2},{#3})}
               % \HamPrincFct{q}{q'}{t}{j}

%%%%%%%%%%%%%% SPECIFIC TO lattFT.tex NOTES %%%%%%%%%%%%%%%%%%%%%%%%%%%%
\newcommand{\unit}{{\bf 1}}
\newcommand{\hopMat}{{\bf h}}
\newcommand{\hop}{h}
\newcommand{\fix}{\marginpar{$\diamond$}}
\newcommand{\source}{{J}}
\newcommand{\sourceFT}{{\tilde{J}}}
\newcommand{\derSource}{{d~\over d\source}}
\newcommand{\derSourceFT}{{d~\over d\sourceFT}}
\newcommand{\field}{{\phi}}     % used in lattFT.tex
%\newcommand{\field}{{x}}       % not a good notation
\newcommand{\fieldFT}{{\tilde{\phi}}}
\newcommand{\derField}{{d~\over d\field}}
\newcommand{\saddleField}{{\field^c}}
\newcommand{\saddleCoord}{{\coord^c}}
\newcommand{\Laplacian}{\Delta}
% \newcommand{\Prpgtr}{{G_0}}       % modified in lattFT.tex
\newcommand{\Prpgtr}{{M}}
\newcommand{\PrpgtrFT}{{\tilde{G}_0}}
% \newcommand{\InvPrpgtr}{{G_0^{-1}}}   % modified in lattFT.tex
\newcommand{\InvPrpgtr}{{M^{-1}}}
\newcommand{\GreenF}{{G}}
\newcommand{\Df}[1]{f^{'}_{#1}}
\newcommand{\nosum}{\not\!\!{\scriptstyle\sum}}
\newcommand{\doublespace}{\baselineskip = \normalbaselineskip \multiply\baselineskip by 2}
%%%%%%%%%%%%%% end of SPECIFIC TO lattFT.tex NOTES %%%%%%%%%%%%%%%%%%%%%%%%%%%%

%%%%  gli commandi di Commandottore Roberto   %%%%%%%%%%%%
\newcommand {\tidue}{{\mbox{\bf T}}^{2}}
\newcommand {\bom}[1]{\mbox{\boldmath $#1$}}
\newcommand {\polit}{{\cal P}_{T}}
\newcommand {\id}{{\ \hbox{{\rm 1}\kern-.6em\hbox{\rm 1}}}}
\newcommand {\ep}{\epsilon}

%%%%%%% Wirzba scattering.tex  %%%%%%%%%%%%%%%%%%%%%%%%%%%%
\newcommand{\gesim}{\mbox{\raisebox{-.6ex}{$\,{\stackrel{>}{\sim}}\,$}}}
\newcommand{\lesim}{\mbox{\raisebox{-.6ex}{$\,{\stackrel{<}{\sim}}\,$}}}
\newcommand{\Ageom}[1]{{\bf A}^{#1}}
\newcommand{\Aghost}[1]{{\underline{\bf A}}^{#1}}
\newcommand{\Acreep}[1]{{\mathbb{A}}^{#1}}
%\newcommand{\Acreep}[1]{{\hat{\bf A}}^{#1}}

%%%%%%%%%%%%%%%%%%%%%% birdtracks SPECIFIC %%%%%%%%%%%%%%%%%%%%%%%%%%%%%%%
%% from def_group.tex
\newcommand{\PP}{{\mathbf P}}                   % projection operator
\newcommand{\RR}{{\mathbf R}}    % real part, projection operator
\newcommand{\QQ}{{\mathbf Q}}    % imaginary part, projection operator

%% Young diagrams (multiplication-stuff due to C. Holm -- cheers!)
%% command \btrackYt[size of one box (optional)]{filename}{number of boxes}
\newdimen\onebox
\newdimen\boxsize
\newcount\boxnum
\gdef\mult#1#2#3{% #1 = #2 * #3
    \ifx#1\relax\else%
      \ifx#2\relax\else%
        #1=#2%
        \ifx#3\relax\else%
          \multiply#1#3%
        \fi%
      \fi%
    \fi}

\newcommand{\btrack}[1]{\raisebox{-2.0ex}[3.5ex][2.5ex]
    {\includegraphics[height=5ex]{Fig/f_#1.eps}\negthinspace} }
    %{\epsfig{file=Fig/f_#1.eps,height=5ex}\negthinspace} }
%% A is 7/5-ths taller
\newcommand{\btrackA}[1]{\raisebox{-3.0ex}[4.5ex][3.5ex]
         { \epsfig{file=Fig/f_#1.eps,height=7ex}\negthinspace} }
%% B is 9/5-ths taller
\newcommand{\btrackB}[1]{\raisebox{-4.0ex}[5.5ex][4.5ex]
          { \epsfig{file=Fig/f_#1.eps,height=9ex}\negthinspace} }
%% BB is 11/5 larger
\newcommand{\btrackBB}[1]{\raisebox{-5.0ex}[6.5ex][5.5ex]
          { \epsfig{file=Fig/f_#1.eps,height=11ex}\negthinspace} }
%% C is 1/2 smaller
\newcommand{\btrackC}[1]{\raisebox{-0.4ex}[1.75ex][1.25ex]
          { \epsfig{file=Fig/f_#1.eps,height=2.5ex}\negthinspace} }
%% birtrack to be drawn:
\newcommand{\zzzz}{{\tt birdTrack}}
%%   Birdtracks with vertical alignment info
%%%% copied from Anders Johansen inputs/anders_def.tex  15 May 2002
\newlength{\verti}
\newcommand{\btrackAl}[3]{%
    \setlength{\verti}{-#3pt*5+2.5pt}% -(5pt*m)+2.5pt  m=#3
    \setlength{\boxsize}{#2pt*5}%
    \raisebox{\verti}{\includegraphics[width=\boxsize]{Fig/f_#1.eps}}}
%% Birdtracks with sizes in terms of #Young diagram boxes
\newcommand{\btrackYq}[3][5pt]{%
    \boxnum=#3%
    \onebox=#1%
    \mult{\boxsize}{\onebox}{\boxnum}%
    \parbox{\boxsize}{\includegraphics[width=\boxsize]{Fig/f_#2.eps}}}

%%%%%%%%%%%%%%%%%% FEYNMANN DIAGRAMS %%%%%%%%%%%%%%%%%%%%%%%%%%%%%%%%
\thicklines
\newlength{\Fsize}   % allow for easy resizing of diagrams
\newlength{\Fdotsize}
\setlength{\Fsize}{20pt}
\setlength{\Fdotsize}{5pt}
\setlength{\unitlength}{\Fsize}
\newcommand{\Fdot}{  % vertex
        \begin{picture}(0,0)
        \setlength{\unitlength}{\Fdotsize}
    \put(0,0){\circle*{1}}
        \end{picture}}
%Propagator naming conventions: \F(d|D)(h|[c](u|d|l|r)(u|d|l|r))
%d=dotted, D=solid, h=horizontal, c=curved, u=up, d=down, l=left, r=right
%The straight propagators are specified u|d then l|r
%The curved propagators end up in the location specified by the last letter
\newcommand{\Fdh}{   % horizontal dotted propagator
    \begin{picture}(0,0)
    \setlength{\unitlength}{\Fsize}
    \qbezier[10](0,0)(0.5,0)(1,0)
    \end{picture}}
\newcommand{\FDh}{   % horizontal solid propagator
        \begin{picture}(0,0)
        \setlength{\unitlength}{\Fsize}
    \put(0,0){\line(1,0){1}}
        \end{picture}}
\newcommand{\FDur}{  % diagonal solid propagators
        \begin{picture}(0,0)
        \setlength{\unitlength}{\Fsize}
    \put(0,0){\line(1,1){0.7}}
        \end{picture}}
\newcommand{\FDdr}{
        \begin{picture}(0,0)
        \setlength{\unitlength}{\Fsize}
        \put(0,0){\line(1,-1){0.7}}
        \end{picture}}
\newcommand{\FDul}{
        \begin{picture}(0,0)
        \setlength{\unitlength}{\Fsize}
        \put(0,0){\line(-1,1){0.7}}
        \end{picture}}
\newcommand{\FDdl}{
        \begin{picture}(0,0)
        \setlength{\unitlength}{\Fsize}
        \put(0,0){\line(-1,-1){0.7}}
        \end{picture}}
\newcommand{\Fdcul}{  % curved propagators
        \begin{picture}(0,0)
        \setlength{\unitlength}{\Fsize}
    \qbezier[15](0,0)(-1,1)(-1,0)
        \end{picture}}
\newcommand{\FDcdl}{
        \begin{picture}(0,0)
        \setlength{\unitlength}{\Fsize}
        \qbezier(0,0)(-1,-1)(-1,0)
        \end{picture}}
\newcommand{\Fdcur}{
        \begin{picture}(0,0)
        \setlength{\unitlength}{\Fsize}
        \qbezier[15](0,0)(1,1)(1,0)
        \end{picture}}
\newcommand{\FDcur}{
        \begin{picture}(0,0)
        \setlength{\unitlength}{\Fsize}
        \qbezier(0,0)(1,1)(1,0)
        \end{picture}}
\newcommand{\FDcdr}{
        \begin{picture}(0,0)
        \setlength{\unitlength}{\Fsize}
        \qbezier(0,0)(1,-1)(1,0)
        \end{picture}}
\newcommand{\Fdclu}{
        \begin{picture}(0,0)
        \setlength{\unitlength}{\Fsize}
        \qbezier[20](0,0)(-1,1)(1,1)
        \end{picture}}
\newcommand{\FDcld}{
        \begin{picture}(0,0)
        \setlength{\unitlength}{\Fsize}
        \qbezier(0,0)(-1,-1)(1,-1)
        \end{picture}}
\newcommand{\FDcru}{
        \begin{picture}(0,0)
        \setlength{\unitlength}{\Fsize}
        \qbezier(0,0)(1,1)(-1,1)
        \end{picture}}
\newcommand{\FDcrd}{
        \begin{picture}(0,0)
        \setlength{\unitlength}{\Fsize}
        \qbezier(0,0)(1,-1)(-1,-1)
        \end{picture}}
\setlength{\unitlength}{1pt}

%%%%%%%%%%%%      stuff below this line will probably be dropped%%%%%%%%%%%

\renewcommand{\b}{\beta}
\newcommand{\w}{\omega}
\newcommand{\p}{2\pi}
\newcommand{\J}{\mbox{  \rule[.03ex]{.03em}{1.5ex} \hspace*{-0.9em} \rm J}}
\newcommand{\f}{\varphi}

%%%%%%%%%%%%%%  Bibliography abbreviations %%%%%%%%%%%%%%%%%%%%%%%%%%%%%%%%
%\newcommand{\AP}[1]{{\em Ann.\ Phys.}\/ {\bf #1}}
%\newcommand{\CHAOS}[1]{{\em CHAOS}\/ {\bf #1}}
%\newcommand{\CM}[1]{{\em Cont.\ Math.}\/ {\bf #1}}
%\newcommand{\CMP}[1]{{\em Commun.\ Math.\ Phys.}\/ {\bf #1}}
%\newcommand{\INCB}[1]{{\em Il Nuov.\ Cim.\ B}\/ {\bf #1}}
%\newcommand{\JCP}[1]{{\em J.\ Chem.\ Phys.}\/ {\bf #1}}
%\newcommand{\JETP}[1]{{\em Sov.\ Phys.\ JETP}\/ {\bf #1}}
%\newcommand{\JETPL}[1]{{\em JETP Lett.}\/ {\bf #1}}
%\newcommand{\JMP}[1]{{\em J.\ Math.\ Phys.}\/ {\bf #1}}
%\newcommand{\JMPA}[1]{{\em J.\ Math.\ Pure Appl.}\/ {\bf #1}}
%\newcommand{\JPA}[1]{{\em J.\ Phys.}\/ {\bf A  #1}}
%\newcommand{\JPB}[1]{{\em J.\ Phys.}\/ {\bf B  #1}}
%\newcommand{\JPC}[1]{{\em J.\ Phys.\ Chem.}\/ {\bf #1}}
%\newcommand{\JchemP}[1]{{\em J.\ Chem.\ Phys.}\/ {\bf #1}}
%\newcommand{\NPA}[1]{{\em Nucl.\ Phys.}\/ {\bf A #1}}
%\newcommand{\NPB}[1]{{\em Nucl.\ Phys.}\/ {\bf B #1}}
%\newcommand{\NONLIN}[1]{{\em Nonlinearity}\/ {\bf #1}}
%\newcommand{\PLA}[1]{{\em Phys.\ Lett.}\/ {\bf A #1}}
%\newcommand{\PLB}[1]{{\em Phys.\ Lett.}\/ {\bf B #1}}
%\newcommand{\PRA}[1]{{\em Phys.\ Rev.}\/ {\bf A #1}}
%\newcommand{\PRD}[1]{{\em Phys.\ Rev.}\/ {\bf D #1}}
%\newcommand{\PRL}[1]{{\em Phys.\ Rev.\ Lett.}\/ {\bf #1}}
%\newcommand{\PST}[1]{{\em Phys.\ Scripta}\/ {\bf T #1}}
%\newcommand{\RMS}[1]{{\em Russ.\ Math.\ Surv.}\/ {\bf #1}}
%\newcommand{\USSR}[1]{{\em Math.\ USSR.\ Sb.}\/ {\bf #1}}
%\newcommand{\ZNat}[1]{{\em Z. Naturforschung}\/ {\bf #1}}

% \newcommand{\R}{\mbox{  \rule[.07ex]{.03em}{1.5ex} \hspace*{-0.75em} \rm R}}
% \newcommand{\nextchapter}{\newpage{\pagestyle{empty}\cleardoublepage}}

%\newcommand {\vett}[2]{\bepar{c} #1 \\ #2 \epar}
%\newcommand {\vettc}[2]{\bepar{c,c} #1 & #2 \epar}
%\newcommand {\vettq}[2]{\left[ \begin{array}{c,c} #1 & #2 \end{array} \right]}
%\newcommand {\ron}{\rho_{n}}
%\newcommand {\ronu}{\rho_{n+1}}
%\newcommand {\ggat}{\Gamma_{T}}
%\newcommand {\gga}{\Gamma}
%\newcommand {\ggan}{\Gamma_{n}}
%\newcommand {\figurino}[3]{\begin{figure}[b]  \vspace{#1}
%            \caption{\protect\small #2}  \label{#3} \end{figure}}
% \newcommand {\zitadef}{\prod_{p}(1-t_{p})}
%\def\t{\tilde}
%\def\h{\hat}
%\def\tf{\tilde f}
%\def\hf{\hat f}
% \def\time{\sigma_p}
%       %% try to understand what does this do?:
% \def\pmb#1{\setbox0=\hbox{$#1$}\kern-.025em\copy0\kern-\wd0
%        \kern-0.05em\copy0\kern-\wd0\kern-.025em\raise.0233em\box0}

    % Roberto's versions:
% \newcommand{\reals}{\mbox{\bf R}} % used in Ronnie's problems
% \newcommand {\natur}{{\hbox{{\rm I}\kern-.2em\hbox{\rm N}}}}
%\newcommand {\relativi}{{\ \hbox{{\rm Z}\kern-.4em\hbox{\rm Z}}}}
% \newcommand {\reali}{{\hbox{{\rm I}\kern-.2em\hbox{\rm R}}}}
% \newcommand{\RR}{{I\negthinspace\!R}}     % Reals: we honor Tresser

%%%% all \Fig to be eliminated in favor of \FIG eventually %%%%%%%%%
%%%% eventually eliminate epsfig by \includegraphics everywhere %%%%
%   \Fig{#1}    % \epsfig{file=Fig/f_name.ps,width=?cm} ... here
%   {#2}    % short caption text
%   {#3}    % full caption text
%   {#4}    % f-figure-label
%       defined here:
% \newcommand{\Fig}[4]{\begin{figure}
%           \centering{#1}
%                       \caption[#2]{#3}
%                       \label{#4} \end{figure} }

%%%%%%%%%%%%%%%%%%%%%%%%%%%%%%%%%%%%%%%%%%%%%%%%%%%%
% changes bars package collides with everything, abandoned Apr 2000
% \setlength{\changebarwidth}{0.5cm}    % margin changes bars width
% %?\setcounter{\changebargray}{85}     % margin changes bars blackness
%%%%%%%%%%%%%%%%%%%%%%%%%%%%%%%%%%%%%%%%%%%%%%%%%%%%
