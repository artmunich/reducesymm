% inputs/layout.tex
% $Author$ $Date$

% from ChaosBook                        Predrag          5jun2008

\newcommand{\authorSFPC}[1] % Date 6 Jun 2009, for example
     {\hfill (S. Froehlich and P. Cvitanovi\'c, #1)}
\newcommand{\authorSF}[1]
     {\hfill (S. Froehlich, #1)}
\newcommand{\authorRW}[1]
     {\hfill (R. Wilczek, #1)}
\newcommand{\authorES}[1]
     {\hfill (E. Siminos, #1)}
\newcommand{\authorPC}[1]
     {\hfill (P. Cvitanovi\'c, #1)}


%%%%%%%% for figure layouts:
%% fine tuning floatingfigure \usepackage[nooneline] captions %%%%%
%        \renewcommand{\captionfont}{\footnotesize} % \sffamily}
%        \renewcommand{\captionlabelfont}{\footnotesize \bfseries \rmfamily}
%        % \renewcommand{\captionlabeldelim}{.\quad}
%        % \setlength{\captionmargin}{10pt}

% \FIG{#1}    % \includegraphics[width=0.40\textwidth]{Fig/f_name.ps}  ...
%     {#2}    % short caption text omitted here
%     {#3}    % full caption text
%     {#4}    % f-figure-label
\newcommand{\FIG}[4]{\begin{widefigure}[t]{#4}
    \widefigparts{\begin{minipage}[b]{0.98\textwidth}
                    #1\end{minipage}}{#3}\end{widefigure}}

\newcommand{\FFIG}[5]{\begin{floatingfigure}[v]{0.40\textwidth}
                      \noindent
                      \includegraphics[#1]{#2}
                      \caption[#3]{#4}
                      \label{#5}
              \end{floatingfigure} }

%   \BFIG{#1}   % width=#1\textwidth
%        {#2}   % f_name.ps
%    {#3}   % short caption text
%    {#4}   % full caption text
%    {#5}   % f-figure-label
%       defined here:
\newcommand{\BFIG}[5]{\begin{figure}
              \hspace*{0.10\textwidth}%
              \begin{minipage}[b]{1.00\textwidth}
              \centering{
                      \includegraphics[width=#1\textwidth]{Fig/#2}
                     }
                      \caption[#3]{#4}
                      \label{#5}
              \end{minipage}
              \end{figure} }

%  \SFIG{#1}    % f_name.png (or .pdf)
%       {#2}    % short caption text
%       {#3}    % full caption text
%       {#4}    % f-figure-label
\newcommand{\SFIG}[4]{\begin{figure}[h]
              %\hspace*{-0.10\textwidth}
              \hspace*{0.10\textwidth}
              \begin{minipage}[b]{0.55\textwidth}
                      \caption[#2]{#3}
                      \label{#4}
              \end{minipage}~~~~~%
              \begin{minipage}[b]{0.40\textwidth}
                      \includegraphics[width=1.00\textwidth]{#1}
              \end{minipage}
              %\hfill
              \end{figure} }

\newcommand{\MultiFIG}[6]{\begin{figure}[h]
              %\hspace*{-0.10\textwidth}
              \hspace*{0.10\textwidth}
              \begin{minipage}[b]{0.35\textwidth}
                      \caption[#4]{#5}
                      \label{#6}
              \end{minipage}~~~~~%
              \begin{minipage}[b]{0.60\textwidth}
                      \includegraphics[width=1.00\textwidth]{#1}
                      \includegraphics[width=1.00\textwidth]{#2}
                      \includegraphics[width=1.00\textwidth]{#3}
              \end{minipage}
              %\hfill
              \end{figure} }


%%%%%%%%%%%%%%% FORMATING: PAGINATION FOR bin/mkbook %%%%%%%%%%%%%%%%%%%%%%%

\newcommand{\Resume}{\section*{\textsf{\textbf{R\'esum\'e}}}
    %   \addcontentsline{toc}{section}{{~~~~Resum\'e}}
                \addtocontents{toc}{{\small r\'esum\'e \thepage ~}} }
\newcommand{\ResumeEnd}{}

\newcommand{\Remarks}{\section*{\textsf{\textbf{Commentary}}}
        \addtocontents{toc}{{\small commentary \thepage ~ }} }
\newcommand{\RemarksEnd}{\ifOUP   %\end{enumerate}
        \end{reading} \else  \fi}

\newcommand{\Problems}[2]{
                \renewcommand{\inputfile}{#1 -  #2}
    \ifboyscout \begin{exercisesBS} \else \begin{exercises} \fi
    \begin{enumerate}}
\newcommand{\ProblemsEnd}{\end{enumerate}
    \ifboyscout \end{exercisesBS} \else \end{exercises} \fi}

\newcommand{\Solution}[4]{
        \section*{Chapter~\ref{c-#1}. #4}
                \renewcommand{\inputfile}{#2 -  #3}
        % PC:  modified 28 may 2000
                %         \renewcommand{\chapName}{#1}
                %        \setcounter{startSect}{\thepage}
                        }
\newcommand{\Solutions}{
        \section{\textsf{\textbf{Solutions}}}
        %\addcontentsline{toc}{section}{{~~~~Solutions}}
                        }


%%%%%%%%%%%%%%% NUMBERED ENVIRONMENTS %%%%%%%%%%%%%%%%%%%%%%%%%%%%%%%
% REPLACE {13.5cm} by \textwidth counter in these environments

    \ifarticle
\newcommand{\exerbox}[1]{}
\newcommand{\exercise}[2]{}{}
\newcommand{\solution}[3]{}{}{}
    \else
\newtheorem{exerc}{\textsf{\textbf{Exercise}}}[chapter]
% \newtheorem{exerc}{}[chapter]
% \newtheorem{exerc}{{$\bullet$}}[chapter]
 \newcommand{\exercise}[2]{
        %\vskip -13mm
         \noindent
         \begin{exerc}{
\renewcommand{\theenumi}{\alph{enumi}}
\renewcommand{\labelenumi}{\textbf{(\alph{enumi})\ }}
    {\noindent\small
         ~~\textsf{\textbf{#1}} ~
           \slshape\sffamily{#2}  } % \textsl would not work...
    }
         \vskip -1mm
% removed the line: % \noindent\rule[.1mm]{\linewidth}{.5mm}
         \end{exerc}
                          }

\newcommand{\Exercise}[2]{      %environment for obligatory problems
        %\vskip -13mm
        \noindent
        \begin{exerc}{
\renewcommand{\theenumi}{\alph{enumi}}
\renewcommand{\labelenumi}
    {\textsf{\textbf{ (\alph{enumi})\ }}}
        {\noindent
         ~~\textsf{\textbf{\underline{#1}}} ~
           \slshape\sffamily{#2}  } % \textsl would not work...
        }
         \vskip -1mm
        \end{exerc}
                          }

\newcommand{\solution}[3]{
        {\noindent\small
         \textsf{\textbf{Solution \ref{#1}~-~#2}} %LABEL - TITLE
         \slshape\sffamily{#3}                    %TEXT
         }
         \vskip  1ex  %4mm
% removed the line: % \noindent\rule[.1mm]{\linewidth}{.5mm}
                        }
\newcommand{\exerbox}[1]
{\marginpar[{\color{red}\sf\footnotesize\hfill exercise \ref{#1} }]
               {\color{red}\sf\footnotesize exercise \ref{#1} \hfill}}
    \fi %end of article switch


    \ifarticle
\newtheorem{exmple}{\noindent\small\textsf{\textbf{Example}}}[section]
\newcommand{\example}[2]{
    \vskip -13mm
        %\begin{offset}
        \begin{exmple}
           \noindent\small
           \textsf{\textbf{#1}} ~
       \slshape\sffamily{#2}
       % \textsl would not work...
    \end{exmple}
    %\end{offset}
        \vskip -1mm
             }
    \else
\newtheorem{rmark}{{\small\textsf{\textbf{Remark}}}}[chapter]
\newcommand{\remark}[2]{
        % \begin{quotation}
        \begin{rmark}
        {\small\em\noindent {\small\sf \underline{ #1} ~} #2 }
    \end{rmark}
    % \end{quotation}
              }
\newtheorem{exmple}{\noindent\small\textsf{\textbf{Example}}}[chapter]
\newcommand{\example}[2]{
    \vskip -13mm
        %\begin{offset}
        \begin{exmple}
           \noindent\small
           \textsf{\textbf{#1}} ~
       \slshape\sffamily{#2}
       % \textsl would not work...
    \end{exmple}
    %\end{offset}
        \vskip -1mm
             }
    \fi %end of article switch
