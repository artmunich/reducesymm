% Latex macros

                            \newif\ifdraft \newif\ifpaper
%   \drafttrue\paperfalse      % draft version, commented

%    \draftfalse\papertrue      % final version, no hyperlinks, for printing

 \draftfalse\paperfalse     % final version, hyperlinks

\newcommand{\edit}[1]{{\color{blue} #1}} % for referees
%\newcommand{\edit}[1]{#1}               % for the journal


\ifpaper % prepare for B&W paper printing:
       \newcommand{\HREF}[2]{{#2}}
       \newcommand{\wwwcb}[1]{{\tt ChaosBook.org#1}}
       \newcommand{\arXiv}[1]{ {\tt arXiv:#1}}
\else % prepare hyperlinked pdf
        \newcommand{\wwwcb}[1]{       % keep homepage flexible:
                  {\tt \href{http://ChaosBook.org#1}
              {ChaosBook.org#1}}}
       \newcommand{\HREF}[2]{
              {\href{#1}{#2}}}
       \newcommand{\arXiv}[1]{
              {\tt \href{http://arXiv.org/abs/#1}{\goodbreak arXiv:#1}}}
\fi

%%%%%%%%%%%% MACROS, ChaosBook.org def.tex %%%%%%%%%%

\newcommand{\beq}{\begin{equation}}
\newcommand{\continue}{\nonumber \\ }
\newcommand{\nnu}{\nonumber}
\newcommand{\eeq}{\end{equation}}
\newcommand{\ee}[1] {\label{#1} \end{equation}}
\newcommand{\bea}{\begin{eqnarray}}
\newcommand{\ceq}{\nonumber \\ & & }
\newcommand{\eea}{\end{eqnarray}}
\newcommand{\barr}{\begin{array}}
\newcommand{\earr}{\end{array}}

\newcommand{\rf}     [1] {~\cite{#1}}
\newcommand{\refref} [1] {ref.~\cite{#1}}
\newcommand{\refRef} [1] {Ref.~\cite{#1}}
\newcommand{\refrefs}[1] {refs.~\cite{#1}}
\newcommand{\refRefs}[1] {Refs.~\cite{#1}}
\newcommand{\refeq}  [1] {(\ref{#1})}
\newcommand{\refeqs} [2]{(\ref{#1}--\ref{#2})}
\newcommand{\refpage}[1] {page~\pageref{#1}}
\newcommand{\reffig} [1] {figure~\ref{#1}}
\newcommand{\reffigs} [2] {figures~\ref{#1} and~\ref{#2}}
\newcommand{\refFig} [1] {Figure~\ref{#1}}
\newcommand{\refFigs} [2] {Figures~\ref{#1} and~\ref{#2}}
\newcommand{\reftab} [1] {table~\ref{#1}}
\newcommand{\refTab} [1] {Table~\ref{#1}}
\newcommand{\reftabs}[2] {tables~\ref{#1} and~\ref{#2}}
\newcommand{\refsect}[1] {sect.~\ref{#1}}
\newcommand{\refsects}[2] {sects.~\ref{#1} and \ref{#2}}
\newcommand{\refSect}[1] {Sect.~\ref{#1}}
\newcommand{\refSects}[2] {Sects.~\ref{#1} and \ref{#2}}

\newcommand\Poincare{Poincar\'e }
\newcommand{\statesp}{state space}
\newcommand{\Statesp}{State space}
\newcommand{\stabmat}{stability matrix}     % stability matrix, velocity gradients
\newcommand{\Stabmat}{Stability matrix}     % Stability matrix
\newcommand{\jacobianM}{Jacobian matrix}  % back to Predrag's name 20oct2009
\newcommand{\jacobianMs}{Jacobian matrices}   % matrices
\newcommand{\JacobianM}{Jacobian matrix} %
\newcommand{\JacobianMs}{Jacobian matrices}  %
\newcommand{\stretchf}{`stretch \&\ fold'}
\newcommand{\Stretchf}{`Stretch \&\ fold'}
\newcommand{\turn}{turning point}    % {turnback} ??
\newcommand{\Turn}{Turning point}    % {Turnback} ??
\newcommand{\nws}{non--wandering set}
\newcommand{\po}{periodic orbit}
\newcommand{\Po}{Periodic orbit}
\newcommand{\rpo}{relative periodic orbit}
\newcommand{\Rpo}{Relative periodic orbit}
\newcommand{\eqv}{equilibrium}
\newcommand{\Eqv}{Equilibrium}
\newcommand{\eqva}{equilibria}
\newcommand{\Eqva}{Equilibria}
\newcommand{\reqv}{relative equilibrium}
\newcommand{\Reqv}{Relative equilibrium}
\newcommand{\reqva}{relative equilibria}
\newcommand{\Reqva}{Relative equilibria}
\newcommand{\reducedsp}{reduced state space}
\newcommand{\Reducedsp}{Reduced state space}
\newcommand{\fixedsp}{fixed-point subspace}
\newcommand{\Fixedsp}{Fixed-point subspace}
\newcommand{\csection}{cross-section}
\newcommand{\Csection}{Cross-section}
\newcommand{\slice}{slice}
\newcommand{\Slice}{Slice}
\newcommand{\mslices}{method of slices}
\newcommand{\Mslices}{Method of slices}
\newcommand{\mframes}{{method of moving frames}}
\newcommand{\Mframes}{{Method of moving frames}}

\newcommand{\cLe}{complex Lorenz equations}
\newcommand{\cLf}{complex Lorenz flow}
\newcommand{\CLe}{Complex Lorenz equations}
\newcommand{\CLf}{Complex Lorenz flow}
\newcommand{\KS}{Kuramoto-Sivashinsky}
\newcommand{\KSe}{Kuramoto-Sivashinsky equation}
\newcommand{\pCf}{plane Couette flow}
\newcommand{\PCf}{Plane Couette flow}

\newcommand{\etc}{{etc.}}       % APS
\newcommand{\etal}{{\em et al.}}    % etal in italics, APS too
\newcommand{\ie}{{i.e.}}        % APS
\newcommand{\cf}{{\em cf.\ }}     % APS
\newcommand{\eg}{{e.g.\ }}        % APS, OUP, hard space '\eg\ NextWord'

\newcommand{\reals}{\mathbb{R}}
\newcommand{\Rls}[1]{\ensuremath{\mathbb{R}^{#1}}}
\newcommand{\pde}{\partial}
\newcommand {\id}{{\ \hbox{{\rm 1}\kern-.6em\hbox{\rm 1}}}}
\newcommand{\pS}{\ensuremath{{\cal M}}}          % symbol for state space
\newcommand{\ssp}{\ensuremath{x}}                % state space point
\newcommand\xInit{{x_0}}        %initial x
\newcommand\flow[2]{{f^{#1}(#2)}}
\newcommand{\vel}{\ensuremath{v}}   % state space velocity
\newcommand{\Mvar}{\ensuremath{A}}  % stability matrix
\newcommand{\Lyap}{\ensuremath{\lambda}}            %Lyapunov exponent
\newcommand{\eigExp}[1][]{
     \ifthenelse{\equal{#1}{}}{\ensuremath{\lambda}}{\ensuremath{\lambda^{(#1)}}}}
\newcommand{\eigRe}[1][]{
     \ifthenelse{\equal{#1}{}}{\ensuremath{\mu}}{\ensuremath{\mu^{(#1)}}}}
\newcommand{\eigIm}[1][]{
     \ifthenelse{\equal{#1}{}}{\ensuremath{\omega}}{\ensuremath{\omega^{(#1)}}}}
\newcommand{\PoincS}{\ensuremath{\cal P}}     % symbol for Poincare section
\newcommand{\PoincM}{\ensuremath{P}}       % symbol for Poincare map
\newcommand{\tr}{\mbox{\rm tr}\,}


\newcommand\stagn{q}      %equilibrium/stagnation point suffix
\newcommand{\cycle}[1]{\ensuremath{\overline{#1}}}
\newcommand\period[1]{{\ensuremath{T_{#1}}}}         %continuous cycle period
\newcommand{\EQV}[1]{\ensuremath{EQ_{#1}}} %experimental
% E_0: u = 0 - trivial equilibrium
% E_1,E_2,E_3, for 1,2,3-wave equilibria
\newcommand{\REQV}[2]{\ensuremath{TW_{#1#2}}} % #1 is + or -
% TW_1^{+,-} for 1-wave traveling waves (positive and negative velocity).

\newcommand{\rLor}{\rho}    % parameter r in Lorenz paper
\newcommand{\RerCLor}{\rho_1}    % real      part of parameter r, CLe
\newcommand{\ImrCLor}{\rho_2}    % imaginary part of parameter r, CLe

\newcommand{\gSpace}{\ensuremath{{\bf \theta}}}   % group rotation parameters
\newcommand{\stab}[1]{\ensuremath{G_{#1}}}
\newcommand{\velRel}{\ensuremath{c}}    % relative state velocity
\newcommand{\pVeloc}{v}         % phase-space velocity
\newcommand{\Fix}[1]{\ensuremath{\mathrm{Fix}\left(#1\right)}}
\newcommand{\pSRed}{\ensuremath{\overline{\cal M}}} % reduced state space
\newcommand{\sspRed}{\ensuremath{y}}    % reduced state space point, experiment
\newcommand{\velRed}{\ensuremath{u}}    % ES reduced state space velocity
\newcommand{\slicep}{{\ensuremath{y'}}}   % slice-fixing point, experimental
\newcommand{\sliceTan}[1]{\ensuremath{t'_{#1}}}    % group orbit tangent at slice-fixing
\newcommand{\groupTan}{\ensuremath{t}}    % group orbit tangent
\newcommand{\Group}{\ensuremath{G}}         % Lie or discrete group
\newcommand{\Lg}{\ensuremath{\mathbf{T}}}   % Lie algebra generator
\newcommand{\LieEl}{\ensuremath{g}}  % Lie group element
\newcommand{\Un}[1]{\ensuremath{\textrm{U}(#1)}}
\newcommand{\On}[1]{\ensuremath{\textrm{O}(#1)}}
\newcommand{\SOn}[1]{\ensuremath{\textrm{SO}(#1)}}

%%%%%%%%%%%% MACROS, CLe paper specific %%%%%%%%%%

\newcommand{\vf}{v}	%%% keep notation for vector field flexible.
\newcommand{\Lint}[1]{\frac{1}{L}\!\oint d#1\,}
\newcommand{\ode}{ODE}
\newcommand{\Clx}[1]{\ensuremath{\mathbb{C}^{#1}}}
\newcommand{\conj}[1]{\ensuremath{\bar{#1}}}
\newcommand{\trace}{\mbox{\rm trace}\,}
\newcommand{\Manif}{\ensuremath{\mathcal{M}}}
\newcommand{\Order}[1]{\mathrm{O}(#1)}

\newcommand{\REQB}[1]{\ensuremath{\mathrm{Q}_{#1}}} 
\renewcommand{\REQV}[2]{\ensuremath{\mathrm{Q}_{#1#2}}} % #1 is + or -
\renewcommand{\EQV}[1]{\ensuremath{\mathrm{E}_{#1}}} %experimental

\renewcommand\Poincare{Poincar\'e}  

%%%%%%%%%%%% Fibre bundles

\newcommand{\tSp}{E}
\newcommand{\bSp}{X}
\newcommand{\prj}{\pi}

%%%%%%%% Symmetries
%
\renewcommand{\sspRed}{\ensuremath{\overline{x}}}    % reduced state space point, experiment
\renewcommand{\slicep}{{\ensuremath{\overline{x}'}}}   % slice-fixing point, experimental
\renewcommand{\sliceTan}[1]{\ensuremath{t'_{#1}}}    % group orbit tangent at slice-fixing
\renewcommand{\groupTan}{\ensuremath{t}}    % group orbit tangent

\newcommand{\Rg}[1]{\Rls{#1}}
\newcommand{\Idg}{\ensuremath{\mathbf{1}}}
\newcommand{\Cn}[1]{\ensuremath{\mathrm{C}_{#1}}}
\newcommand{\En}[1]{\ensuremath{\mathrm{E}(#1)}}
\newcommand{\Shift}{\ensuremath{\tau}}
\newcommand{\Rotn}[1]{\ensuremath{R_{#1}}}
\newcommand{\Drot}{\ensuremath{\zeta}}
\newcommand{\Str}[1]{\ensuremath{\mathcal{S}_{#1}}} % Stratum
\newcommand{\Nlz}[1]{\ensuremath{N(#1)}}
\newcommand{\doubleperiod}[1]{{\ensuremath{\mathcal{T}_{#1}}}}
\newcommand{\slicepComp}[2]{{\ensuremath{\overline{#1}'_{#2}}}}   % slice-fixing point component
\newcommand{\Subgroup}{H}
\newcommand{\LieElrep}{\ensuremath{\mathbf{G}}}
\newcommand{\sset}{singular set}

%%%%%%%%%%%%%%%%%%%%%%

\newcommand{\nameit}{\ensuremath{w-}}
\newcommand{\bbUplus}{\Fix{\Dn{1}}}
\newcommand{\bbUone}{\Shift_{1/4}\Fix{\Dn{1}}}
\newcommand{\bbU}{\mathbb{U}}
% \newcommand{\refneq}[1]{(\ref{#1})}
\newcommand{\refFigToFig}[2]{Figures~\ref{#1} to~\ref{#2}}
\newcommand{\reffigTofig}[2]{Figures~\ref{#1} to~\ref{#2}}
\newcommand{\reffigpart}[2]{Figure~\ref{#1}(#2)}
\newcommand{\refFigpart}[2]{Figure~\ref{#1}(#2)}


%%%%%%%%%%% Theorems %%%%%%%%%%%%%%%%%%%%%%%%%%%%%%%%%%
\newtheorem{definition}{Definition}[section]
\newtheorem{theorem}[definition]{Theorem}
\newtheorem{lemma}[definition]{Lemma}
\newtheorem{proposition}[definition]{Proposition}
\newtheorem{example}[definition]{Example}

\newcommand{\refLem}[1]{Lemma~\ref{#1}}
\newcommand{\refThe}[1]{Theorem~\ref{#1}}
\newcommand{\refDef}[1]{Definition~\ref{#1}}
\newcommand{\refPro}[1]{Proposition~\ref{#1}}
\newcommand{\refExa}[1]{Example~\ref{#1}}


%%%%%%%%% Flows

% \newcommand{\AGHe}{Armbruster-Guckenheimer-Holmes flow}

%%%%%%%%%%% Equations

\newcommand{\cont}{\,, \\ }

%%%%%%%%%%%% Abbreviations, Siminos thesis specific %%%%%%

\newcommand{\Gelement}{\ensuremath{g}}         % group element in \Group
\newcommand{\LGelement}[1]{\ensuremath{g(#1)}} % Lie group element in \Group

%%%%%%%%%%%%%%% From KS rpo paper %%%%%%%%%%%%%%%%%%
\newcommand{\jEigvecKS}[1]{\ensuremath{{\mathbf e}^{(#1)}}}   


%%%%%%%%%%%%% Operators %%%%%%%%%%%%%%%%%%%%%%%
\newcommand{\PperpOp}{\mathbf{P}^{\perp}}
\newcommand{\Pperp}{P^{\perp}}

%%%%%%%%%%%%%% Appendix %%%%%%%%%%%%%%%%%

\newcommand\fp[2]{{\frac{\partial #1}{\partial #2}}}
\newcommand\fder[2]{{\frac{d #1}{d #2}}}
\newcommand\fsd[2]{{d #1/d #2}}
\newcommand\fsp[2]{{\partial #1/\partial #2}}
\newcommand\fps[3]{{\frac{\partial^2 #1}{\partial #2 \partial #3}}}
\newcommand\fsps[3]{{\partial^2 #1/\partial #2 \partial #3}}

\newcommand\Js{\mathbf{\tilde{J}}}
\newcommand\JL{\mathbf{\tilde{J}}_L}
\newcommand\Jp{\mathbf{J}_p}

\newcommand\dtds{\frac{v.\tilde{v}}{v^2}}
\newcommand\hdtds{\frac{u.\tilde{u}}{u^2}}


%%%%%%%%%%%%%%%%%%%%%%%%%%%%%%%%%%%%%%%%%%%%%%%%%%%%
