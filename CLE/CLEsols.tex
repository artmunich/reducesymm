% siminos/CLE/CLEsols.tex
% $Author$ $Date$

\subsection{\label{s:CLEsols} An example: Solutions of \cLe}

In the case of {\cLe}  the origin \EQV{0} is an \eqv\ of
\refeq{eq:CLe} for any value of the parameters. It is stable
for $0<\RerCLor<\rho_{1c}$ and unstable for
$\rho_{1c}<\RerCLor$, where\rf{FowlerCLE82}
\[
	\rho_{1c} = 1 + {(e+\ImrCLor)(e-\sigma \ImrCLor)}/{(\sigma+1)^2}
\,.
\]
At the bifurcation\rf{ruell73} a pair of eigenvalues crosses
the imaginary axis with imaginary part
\beq
	\omega_c = {\sigma (e + \ImrCLor)}/{(\sigma+1)}
\,,
\ee{eq:omegaCLE}
and a \emph{relative equilibrium} \REQV{}{1} with constant
angular velocity $\omega_c$ is born. For $\omega_c =0$ the
\reqv\ degenerates to an \SOn{2}-orbit of \eqva. As the
existence of a \reqv\ in a system with \SOn{2} symmetry is
the generic situation, we follow \refref{BakasovAbraham93}
and set $\ImrCLor=0$ and $e \neq 0$.

To find the location of the \reqv\ it is convenient to work
\JFGedit{in} polar coordinates
\beq
(x_1,x_2,y_1,y_2,z) =
    (r_1 \cos\theta_1,r_1\sin\theta_1,
     r_2\cos\theta_2,r_2\sin\theta_2,z)
\,,
\label{eq:CartToPol}
\eeq
where $r_1 \geq 0 \,,r_2 \geq 0$.
\JFGedit{The} \CLe\ \refeq{eq:CLe} take \JFGedit{the} form
\[ %\beq
\left(
\begin{array}{c}
\dot{r}_1\\
\dot{\theta}_1\\
\dot{r}_2\\
\dot{\theta}_2\\
\dot{z}
\end{array}
\right)
=
\left(
\begin{array}{c}
 -\sigma\left(r_1 - r_2\cos\theta\right) \\
 -\sigma\frac{r_2}{r_1}\sin \theta  \\
 -r_2 + r_1\left((\rho_1-z)\cos \theta - \rho_2 \sin\theta\right)\\
  e  + \frac{r_1}{r_2}\left((\rho_1-z)\sin\theta +\rho_2 \cos\theta\right)\\
 -b z + r_1 r_2\cos\theta
\end{array}
\right)
,
\] %\ee{eq:PolarCLe}
For
rotationally invariant flows the dynamics depends only
on the relative angle $\theta = \theta_1-\theta_2$
(\JFGedit{which} is why one speaks of `relative' equilibria).
This observation enables us to recast the \cLe\
in the  4-dimensional \reducedsp:
\beq
\left(
\begin{array}{c}
\dot{r}_1\\
\dot{r}_2\\
\dot{\theta}\\
\dot{z}
\end{array}
\right)
=
\left(
\begin{array}{c}
 -\sigma\left(r_1 - r_2\cos\theta\right) \\
 -r_2 + (\rho_1-z)r_1\cos \theta\\
  -e -\left(\sigma\frac{r_2}{r_1}
 +(\rho_1-z)\frac{r_1}{r_2}\right)\sin\theta\\
 -b z + r_1 r_2\cos\theta
\end{array}
\right)
\label{eq:PolarCLeTheta}
\eeq
where we have set $\rho_2=0$. The full 5-dimensional evolution can be
regained by integrating the two driven `reconstruction' equations:
\beq
\left(
\begin{array}{c}
\dot{\theta}_1\\
\dot{\theta}_2
\end{array}
\right)
=
\left(
\begin{array}{c}
-\sigma\frac{r_2}{r_1}\sin\theta  \\
 e + (\rho_1-z)\frac{r_1}{r_2}\sin\theta
\end{array}
\right)
\,.
\label{eq:PolarCLeAngles}
\eeq
In general $\theta_1$ and
$\theta_2$ change in time, but for the \reqva\ the
difference between them is constant.
The condition for a \reqv\ is that all
time derivatives in \refeq{eq:PolarCLeTheta} vanish, while
$\dot{\theta}_1=\dot{\theta}_2\neq 0$ (if
$\dot{\theta}_1=\dot{\theta}_2=0$ we have a group orbit
of \eqva\ instead).
The \reqv\
$\REQV{}{1}$ is given by
\bea
(r_1,r_2,\theta,z) &=&
\left(\sqrt{b \,(\rho_1-d)},  \sqrt{b d \,({\rho_1}-d}),
     \cos^{-1}({1}/{\sqrt{d}}),  \rho_1-d
\right)
\,,
\label{eq:E1-PC}
\eea
where $d=1 + {e^2}/{(\sigma +1)^2}$, and
its angular velocity is
\beq
\dot{\theta}_{i}
= {\sigma e}/{(\sigma + 1)}
\,,
\label{eq:REQV1veloc}
\eeq
with period
$\period{{\REQV{}1}}= 2\pi (\sigma + 1)/\sigma e$.
For the parameter values \refeq{eq:CLeR}, the \reqv\ is at
\beq
\ssp_{\REQV{}1} = (r_1,r_2,\theta,z) =
     (8.48527,
      8.48562,
      0.00909,
      26.9999)
\,,
\label{eq:Q1}
\eeq
rotating with the period $\period{{\REQV{}1}}=69.115$.\ES{Period feels too large, 
I'll double check in my notebooks.}

As $\RerCLor$ is increased,  a secondary bifurcation from
\REQV{}{1} results in a \emph{\rpo} \refeq{RPOrelper1}, or,
more precisely, in the quasiperiodic 2-frequency
\emph{modulated traveling wave}\rf{Krupa90}.
Calculation of \JFGedit{the} \REQV{}{1} stability eigenvalues
\PublicPrivate{}{
(see \ref{s:StabReq} for a calculation of stability of
\reqva\ in equivariant variables)
    } %end \PublicPrivate{}{
yields a weakly unstable spiral-out
\eqv\
\beq
(\eigExp_{1,2},\eigExp_3,\eigExp_4)
= (0.0938 \pm 10.1945 i,-11.0009,-13.8534)
\,.
\ee{eq:CLeREQBstab}
With further increase in $\RerCLor$ the dynamics turns
chaotic, with \JFGedit{an} infinity of unstable {\rpo s}. Large numbers of
these can be computed by methods described
elsewhere\rf{SCD07,SiminosThesis}.

The role of \JFGedit{the} above exact invariant solutions is illustrated by the
portrait of \cLf\ \statesp\ in  \reffig{fig:CLE}, with the
\reqv\ \REQV{}{1} and three repetitions of \JFGedit{the} \cycle{01} \rpo\
superimposed over a generic chaotic orbit. Repeats of
\cycle{01} \JFGedit{trace out a torus ergodically} , so in a system with
a $1$-dimensional continuous symmetry the organizational
blocks of a strange attractor are circles (\reqva) instead of
points (\eqva), and partially hyperbolic tori (\rpo s)
instead of closed loops (\po s). It is difficult to
understand the geometry of the flow by looking at such tori.

The large imaginary part of $\eigExp_{1}$ in
\refeq{eq:CLeREQBstab} implies that the simulation has to be
run up to time of order of at least 70 for the strange
attractor in \reffig{fig:CLE} to start filling in. Dynamics
is organized by the interplay of the stable and unstable
manifolds of \eqv\ \EQV{0} and \reqv\ \REQV{}{1}, but the
symmetry-induced drift along the direction of rotation blurs
the picture and the notion of recurrence becomes relative. In
what follows, it is this confusing situation (as well as the
theoretical fact\rf{Cvi07} that dynamical zeta functions have
their support on \rpo s) that motivates the search for
effective methods to project the dynamics onto a \reducedsp.
