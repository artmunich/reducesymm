% siminos/CLE/inputs/defsCLE.tex
% $Author$ $Date$
%
% Predrag                                  2009-10-09
%%%%% based on ../inputs/defsThesis.tex

%\newcommand{\edit}[1]{{\color{blue} #1}} % for referees
\newcommand{\edit}[1]{#1}               % for the journal

%%%%%%%%%%%%%%%%%%%%%% COMMENTS %%%%%%%%%%%%%%%%%%%
    \ifdraft    % display comments in text
    \newcommand{\PublicPrivate}[2]
{\marginpar{\color{blue}$\Downarrow$\footnotesize PRIVATE}%
{\color{blue}#2}%
\marginpar{\color{blue}$\Uparrow$\footnotesize PRIVATE}
           }
\newcommand{\PC}[1]{$\footnotemark\footnotetext{PC: #1}$}
\newcommand{\PCedit}[1]{{\color{red}#1}}
\newcommand{\ES}[1]{$\footnotemark\footnotetext{ES: #1}$}
\newcommand{\ESedit}[1]{{\color{green}#1}}
\newcommand{\JFG}[1]{$\footnotemark\footnotetext{JFG: #1}$}
\newcommand{\RLD}[1]{$\footnotemark\footnotetext{RLD: #1}$}
\newcommand{\RLDedit}[1]{{\color{magenta}#1}}
\renewcommand{\file}[1]{$\footnotemark\footnotetext{{\bf file} #1}$}
\def\mycomment #1#2 {\noindent \textbf{\underline{#1}}: \emph{#2}}
    \else   % drop comments
\newcommand{\PublicPrivate}[2]{#1}
\newcommand{\PC}[1]{}
\newcommand{\PCedit}[1]{#1}
\newcommand{\ES}[1]{}
\newcommand{\ESedit}[1]{#1}
\newcommand{\JFG}[1]{}
\newcommand{\RLD}[1]{}
\newcommand{\RLDedit}[1]{#1}
\newcommand{\file}[1]{}
\def\mycomment #1#2 {}
    \fi
%%%%%%%%%%%%%%%%%%%%%% COMMENTS END %%%%%%%%%%%%%%%

%%%%%%%%%%%% MACROS, CLe paper specific %%%%%%%%%%

\newcommand{\vf}{v}	%%% keep notation for vector field flexible. For the time being follow Das Buch.
\newcommand{\Lint}[1]{\frac{1}{L}\!\oint d#1\,}
\newcommand{\ode}{ODE}
\newcommand{\Clx}[1]{\ensuremath{\mathbb{C}^{#1}}}
\newcommand{\conj}[1]{\ensuremath{\bar{#1}}}
\newcommand{\trace}{\mbox{\rm trace}\,}
\newcommand{\Manif}{\ensuremath{\mathcal{M}}}
\newcommand{\Order}[1]{\mathrm{O}(#1)}

\renewcommand{\csection}{slice}

\newcommand{\REQB}[1]{\ensuremath{\mathrm{Q}_{#1}}} % For ODE's, use REQV from chaosbook for PDE's

%%%%%%%%%%%% Fibre bundles

\newcommand{\tSp}{E}
\newcommand{\bSp}{X}
\newcommand{\prj}{\pi}

%%%%%%%% Symmetries
%
\newcommand{\Rg}[1]{\Rls{#1}}
\newcommand{\Idg}{\ensuremath{\mathbf{1}}}
\newcommand{\Cn}[1]{\ensuremath{\mathrm{C}_{#1}}}
\newcommand{\En}[1]{\ensuremath{\mathrm{E}(#1)}}
\newcommand{\Shift}{\ensuremath{\tau}}
\newcommand{\Rotn}[1]{\ensuremath{R_{#1}}}
\newcommand{\Drot}{\ensuremath{\zeta}}
\newcommand{\globstab}[1]{\ensuremath{\Sigma_{#1}}} % Change to be the same as stab. Was \Sigma^\ast_{#1}
\newcommand{\Str}[1]{\ensuremath{\mathcal{S}_{#1}}} % Stratum
\newcommand{\Nlz}[1]{\ensuremath{N(#1)}}
\newcommand{\doubleperiod}[1]{{\ensuremath{\mathcal{T}_{#1}}}}
\renewcommand{\sliceTan}{\ensuremath{t^*}}

%%%%%%%%%%%%%%%%%%%%%%

\newcommand{\nameit}{\ensuremath{w-}}
\newcommand{\bbUplus}{\Fix{\Dn{1}}}
\newcommand{\bbUone}{\Shift_{1/4}\Fix{\Dn{1}}}
\newcommand{\bbU}{\mathbb{U}}
% \newcommand{\refneq}[1]{(\ref{#1})}
\newcommand{\refFigToFig}[2]{Figures~\ref{#1} to~\ref{#2}}
\newcommand{\reffigTofig}[2]{Figures~\ref{#1} to~\ref{#2}}
\newcommand{\reffigpart}[2]{Figure~\ref{#1}(#2)}
\newcommand{\refFigpart}[2]{Figure~\ref{#1}(#2)}


%%%%%%%%%%% Theorems %%%%%%%%%%%%%%%%%%%%%%%%%%%%%%%%%%
\newtheorem{definition}{Definition}[section]
\newtheorem{theorem}[definition]{Theorem}
\newtheorem{lemma}[definition]{Lemma}
\newtheorem{proposition}[definition]{Proposition}
\newtheorem{example}[definition]{Example}

\newcommand{\refLem}[1]{Lemma~\ref{#1}}
\newcommand{\refThe}[1]{Theorem~\ref{#1}}
\newcommand{\refDef}[1]{Definition~\ref{#1}}
\newcommand{\refPro}[1]{Proposition~\ref{#1}}
\newcommand{\refExa}[1]{Example~\ref{#1}}


%%%%%%%%% Flows

\newcommand{\AGHe}{Armbruster-Guckenheimer-Holmes flow}

%%%%%%%%%%% Equations

\newcommand{\cont}{\,, \\ }

%%%%%%%%%%%% Abbreviations, Siminos thesis specific %%%%%%

\newcommand{\Gelement}{\ensuremath{g}}         % group element in \Group
\newcommand{\LGelement}[1]{\ensuremath{g(#1)}} % Lie group element in \Group

%%%%%%%%%%%%%%% From KS rpo paper %%%%%%%%%%%%%%%%%%
\newcommand{\jEigvecKS}[1]{\ensuremath{{\mathbf e}^{(#1)}}}   % jacobiam eigenvector, redefined here to	avoid conflict with chaosbook notation. Used in ksStSp chapter.


%%%%%%%%%%%%% Operators %%%%%%%%%%%%%%%%%%%%%%%
\newcommand{\PperpOp}{\mathbf{P}^{\perp}}
\newcommand{\Pperp}{P^{\perp}}

%%%%%%%%%%%%%% Penalizing loops %%%%%%%%%%%%%%%%%

\newcommand\fp[2]{{\frac{\partial #1}{\partial #2}}}
\newcommand\fder[2]{{\frac{d #1}{d #2}}}
\newcommand\fsd[2]{{d #1/d #2}}
\newcommand\fsp[2]{{\partial #1/\partial #2}}
\newcommand\fps[3]{{\frac{\partial^2 #1}{\partial #2 \partial #3}}}
\newcommand\fsps[3]{{\partial^2 #1/\partial #2 \partial #3}}

\newcommand\Js{\mathbf{\tilde{J}}}
\newcommand\JL{\mathbf{\tilde{J}}_L}
\newcommand\Jp{\mathbf{J}_p}

\newcommand\dtds{\frac{v.\tilde{v}}{v^2}}
\newcommand\hdtds{\frac{u.\tilde{u}}{u^2}}


%%%%%%%%%%%%%%%%%%%%%%%%%%%%%%%%%%%%%%%%%%%%%%%%%%%%
