% siminos/CLE/intro.tex
% $Author$ $Date$

The basic question that arises during any attempt to understand
nonlinear field theories\rf{chfield}, either classical or
quantum, is whether any order and logical organization can be
identified within the bewildering wealth of solutions. Dynamics
within the chaotic attractor of a low-dimensional, continuous
time, (state-space-)volume contracting flow can be in many
cases\rf{gilmore2003}\ES{perhaps also cite chaosbook.org? But
Gilmore book is like a library of systems so it serves better
here.} understood by reduction to a discrete time map within a
Poincar\'e section. For sufficiently strong volume contraction
such a mapping provides a complete topological characterization
of the attractor\rf{gilmore2003}. Moreover the set of compact
invariant solutions, equilibria and periodic orbits, organize
the dynamics around them and most importantly they can be used
to quantitatively approximate the natural measure and calculate
``observable'' quantities, such as Lyapunov exponents and
escape rates, within the framework of cycle
expansions\rf{DasBuch}.

In higher dimensional flows, such as those that arize from truncations of, formally
infinite dimensional, partial differential equations (PDEs) such a simple picture
need not hold. Nevertheless, many of the early examples of chaotic attractors where
observed in very drastic truncations of PDEs, such as the Lorenz flow\rf{lorenz}.
Christiansen, Cvitanovi\'{c} and Putkaradze\rf{Christiansen97} study of 
\KSe\ (henceforth KS) has shown that the dynamics of this PDE in one spatial dimension, close to the 
onset of ``turbulence'', can be understood in terms of an approximatelly one-dimensional map.
Further studies of more disordered \KS\ systems by Y.~Lan and Cvitanovi\'{c}\rf{LanThesis,lanCvit07}
suggest that a collection of return maps are required to ``cover'' the asymptotic dynamics.

In both Christiansen \etal\ and Lan and Cvitanovi\'{c} studies, a restriction to the 
subspace of odd solutions has been imposed. Such a restriction eliminates the translational 
symmetry of the KS system as well as many physically important phenomena, such as
traveling waves. Traveling solutions, however, are ubiquitous in physics and present 
in most of the recent fluid simulations\rf{KawKida01,FE03,WK04,Visw07b,GHCW07} 
and experiments\rf{science04}\ES{update references} 
that aim to uncover the solutions around which fluid flows are organized.
The question now arises, for a given (truncation of a) PDE, with
given boundary conditions and system parameters, such that
traveling wave solutions are supported, how is one to construct suitable return maps
for the dynamics, provided such a compact description can be afforded. \ES{We have to refer to KS paper
and my thesis as they show the need for symmetry reduction in KS.}

In the context of low dimensional, dissipative dynamical systems it has become clear, as illustrated
in the recent monograph of Gilmore and Letellier\rf{GL-Gil07b}, that before a return
map can be constructed one has to ``factor out'' the symmetry to obtain an \emph{image}
dynamical system in which symmetry related solutions have been identified. In \refsect{s:symDyn} we
review some basic notions of symmetry in dynamical systems and motivate the need for symmetry 
reduction. In \refsect{s:Hilbert} we briefly review the tool by which this is achieved: projection 
to a Hilbert basis. The required polynomial bases can be algorithmicaly determined
and the method has been routinelly used in Hamiltonian systems, see, for example, 
Cushman and Bates\rf{cushman_global_1997} or Marsden and Ratiu\rf{marsden_introduction_1999}. 
Unfortunately the cost of the algorithms that compute Hilbert bases is presently prohibitive 
for state space dimensions larger than ten. A ten-dimensional dynamical system is already high-dimensional
but what we are interested in is systems that result from fully resolved simulations of PDEs.
Therefore typical state space dimensions are of the order of few tenths to few hundreds for \KSe, and
well into the tenths of thousands for \PCf. In order to keep the conceptual framework as simple as possible
and clearly illustrate the difficulties associated with symmetry reduction 
we will nevertheless confine ourselfs to a five-dimensional, rotationaly symmetric, 
system of ODEs, the \CLe\ reviewed in \refsect{s:introCLE}. Higher-dimensional flows will
be examined elsewhere\rf{SiminosThesis,SCD09b}.

In \refsect{sec:mf} we review the \emph{moving frame method}, a very direct and efficient method to compute 
symmetry invariant bases that goes back at least to Cartan. In our view the method consists in mapping all solutions
to a submanifold or \emph{slice} of state space that plays a role similar to a Poincar\'e section. In contrast to the Hilbert basis approach this 
is a local method for the symmetry groups we are interested in as the invariant bases are not defined in subsets 
of state-space and one is forced to use more than one slices. We discuss some remedies to the situation 
in \refsect{sec:mf}\ES{Create subsection if that part stays in the paper.} and show that local moving frames 
suffice for the purpose of reducing the flow to return maps in \refsect{s:laserMFnum}.

In \refsect{sec:MovFrameODE} we show how moving frame method is connected with a method for symmetry reduction studied in
the context of PDEs by Rowley and Marsden\rf{rowley_reconstruction_2000}, see also \rf{rowley_reduction_2003}, 
and Beyn and Th\"ummler\rf{BeTh04,Thum05}. Essentially one restricts integration on a slice transverse to 
the group action at a given state-space point. We show that this can be thought of as a differential formulation 
of the moving frame method and thus suffers from the same restrictions, being essentially local 
and bound to encounter singularities.  

In \refsect{s:StabReq} we derive an expression for the stability of relative equilibria (traveling waves) in the reduced space
in a form that, to our knowledge, is not presented elsewhere and is rather straightforward to apply without explicit calculation of
the reduced system. 






