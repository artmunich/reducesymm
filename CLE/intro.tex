% siminos/CLE/intro.tex
% $Author$ $Date$

The basic question that arises during any attempt to understand nonlinear field theories\rf{chfield}, 
either classical or quantum, is whether any order and logical organization can be identified
within the bewildering wealth of solutions. Dynamics within the chaotic attractor of
a low-dimensional, continuous time, (state-space-)volume contracting flow can be in many
cases\rf{gilmore2003}\ES{perharps also cite chaosbook.org? But Gilmore book is like
a library of systems so it serves better here.} understood by reduction to a discrete time map 
within a Poincar\'e section. For sufficiently strong volume contraction such a mapping provides
a complete topological characterization of the attractor\rf{gilmore2003}. Moreover the set
of compact invariant solutions, equilibria and periodic orbits, organize the dynamics 
around them and most importantly they can be used to quantitatively approximate the natural
measure and calculate ``observable'' quantities, such as
Lyapunov exponents and escape rates, within the framework of
cycle expansions\rf{DasBuch}.

In higher dimensional flows, such as those that arize from truncations of, formally
infinite dimensional, partial differential equations (PDEs) such a simple picture 
need not hold. Nevertheless, many of the early examples of chaotic attractors where
observed in very drastic truncations of PDEs, such as the Lorenz flow\rf{lorenz}.
Christiansen, Cvitanovi\'{c} and Putkaradze\rf{Christiansen97} study of 
\KSe\ (henceforth KS) has shown that the dynamics of an one dimensional PDE close to the 
onset of ``turbulence'' can be understood in terms of an approximatelly one-dimensional map.
Further studies of more disordered \KS\ systems by Y.~Lan and Cvitanovi\'{c}\rf{LanThesis,lanCvit07} 
suggest that a collection of return maps are required to ``cover'' the asymptotic dynamics.

In both Christiansen \etal\ and Lan and Cvitanovi\'{c} studies a restriction to the 
subspace of odd solutions has been posed. Such a restriction eliminates the translational 
symmetry of the KS system as well as many physically important phenomena, such as
traveling waves. Traveling solutions, however, are present in most of the fluid
simulations and experiments mentioned in \refsect{s:hopf}\ES{no such section: collect citations and replace.} 
and ubiquitous in physics. The question now arises, for a given (truncation of a) PDE, with
given boundary conditions and system parameters, such that
traveling wave solutions are supported, how is one to construct suitable return maps
for the dynamics, provided such a compact description can be afforded?

In terms of low dimensional dynamical systems the answer is: through symmetry reduction. 



