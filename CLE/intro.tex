% siminos/CLE/intro.tex
% $Author$ $Date$
%
In his seminal paper\rf{lorenz} E. Lorenz reduced both
the continuous time and the discrete symmetries of the
Lorenz equations in order to construct a 1-dimensional 
return map that gives deep insights into the nature of solutions.
For strongly contracting, low-dimensional, continuous time
flows with discrete spatial symmetry
Gilmore, Lefranc and Letellier\rf{gilmore2003,GL-Gil07b}
have systematized and generalized 
Lorenz's approach to construction of discrete time return maps, through 
use of topological templates, Poincar\'e sections (thus
reducing the continuous time and the discrete symmetry)
and invariant polynomial bases (in order to reduce the
discrete spatial symmetries).
The approach yields symbolic dynamics for the flow,
enumeration and labelling of all periodic orbits up to a given 
topological period which then makes it possible to 
periodic orbit theory\rf{DasBuch} and calculate accurately
the observable long-time dynamical averages, such as Lyapunov exponents 
and escape rates.

Christiansen \etal\rf{Christiansen97} have constructed such 
low-dimensional return maps for the
high-dimensional (formally
infinite dimensional)  flows that arise from finite discretizations
of the  partial differential equations (PDEs) such as the
\KSe\ (KS) close to the 
onset of ``turbulence /spatio-temporal chaos''.
For more ``turbulent'' KS systems Lan and Cvitanovi\'{c}\rf{LanThesis,lanCvit07}
have shown that a collection of return maps might
be required to cover all of the asymptotic dynamics.

These studies were facilitated by a restriction to the flow-invariant 
subspace of odd solutions, but at a price: this restriction eliminates the translational 
symmetry of the KS system and with it physically important phenomena, such as
traveling waves. Traveling (or ``relative'') unstable
coherent solutions are ubiquitous in physics and 
play a key role in topological organization of turbulent hydrodynamic flows,
as recently revealed both by simulations\rf{KawKida01,FE03,WK04,Visw07b,GHCW07} 
and experimentation\rf{science04}\ES{update references}.
The question that we address here is how is one to construct suitable return maps
for flows with continuous symmetries.
\ES{We have to refer to KS paper
and my thesis as they show the need for symmetry reduction in KS.}

In the context of low-dimensional, dissipative dynamical flow 
Gilmore and Letellier\rf{GL-Gil07b} have shown the before return
map can be constructed one has to ``quotient'' the symmetry and
replace the dynamics by a physicaly equivalent 
reduced, desymmetrized flow, a flow in which each family of symmetry related states
is replaced by a single representative. 
In \refsect{s:symDyn} we
review the basic notions of symmetry in dynamics 
and motivate the need for symmetry 
reduction.
\refSect{s:introCLE} introduces the \SOn{2}\ equivariant
 \cLe\ (CLE),
the 5-dimensional set of ODEs that we use throughout the paper,
a conceptually  simple framework which clearly
illustrate the difficulties associated with  different
methods of symmetry reduction.
Application to high-dimensional flows will
be discussed elsewhere\rf{SiminosThesis,SCD09b}.

In \refsect{s:Hilbert} we describe one of the standard tools
by which the continuous symmetry reduction can be achieved,  projection 
to a Hilbert basis. These invariant polynomial bases can be algorithmicaly determined
by methods routinely used for both Hamiltonian \ESedit{and dissipative} systems. 
Unfortunately, the computational cost of such algorithms is at present prohibitive 
for state space dimensions larger than ten,
but we study fully resolved simulations of PDEs, with state space dimensions 
of the order of few tenths to few hundreds (for \KSe), and
well into the tenths of thousands (for pipe and \pCf s). 

In \refsect{sec:mf} we review the \emph{\mframes}, a direct and efficient method 
to compute symmetry invariant bases that goes back to Cartan. 
In our view the method consists in mapping all solutions to a submanifold 
or \emph{slice} of state space that plays a role similar to a Poincar\'e section. 
In contrast to the Hilbert basis approach this is a local method for the symmetry groups 
we are interested in as the invariant bases are not defined in subsets 
of state-space and one is forced to use more than one slices. 
We discuss some remedies to the situation 
in \refsect{sec:mf}\ES{Create subsection if that part stays in the paper.} 
and show that local moving frames 
suffice for the purpose of reducing the flow to return maps in \refsect{s:laserMFnum}.

In \refsect{sec:MovFrameODE} we show how {\mframes}
is connected with a method for symmetry reduction studied in
the context of PDEs by Rowley and Marsden\rf{rowley_reconstruction_2000}, see also \rf{rowley_reduction_2003}, 
and Beyn and Th\"ummler\rf{BeTh04,Thum05}. 
	\PC{the credits have to be totally rewritten - I am entering relevant literature
	    into ChaosBook `real soon'}
Essentially one restricts integration on a slice transverse to 
the group action at a given state-space point. We show that this can be thought of as a differential formulation 
of the moving frame method and thus suffers from the same restrictions, being essentially local 
and bound to encounter singularities.  

In \ref{s:StabReq} we derive an expression for the stability of \reqva\ (traveling waves) in the reduced space
in a form that is rather straightforward to apply without explicit calculation of
the reduced system.






