% siminos/CLE/intro.tex
% $Author$ $Date$

In his seminal paper, E. Lorenz\rf{lorenz} reduced the
continuous time and discrete spatial symmetries of the
3-dimensional Lorenz equations, resulting in a 1-dimensional
return map that yields deep insights\rf{tucker1-2} into the
nature of chaos in this flow. For strongly contracting,
low-dimensional flows, Gilmore, Lefranc and
Letellier\rf{gilmore2003,GL-Gil07b} systematized construction
of such discrete time return maps, through use of topological
templates, Poincar\'e sections (to reduce the continuous time
invariance) and invariant polynomial bases (to reduce the
spatial symmetries). They showed that in presence of spatial
symmetries one has to  `quotient' the symmetry and replace
the dynamics by a physically equivalent reduced,
desymmetrized flow, in which each family of symmetry-related
states is replaced by a single representative. This approach
leads to symbolic dynamics and labeling of all \po s up to a
given topological period. Periodic orbit theory can then
yield accurate estimates of long-time dynamical averages,
such as Lyapunov exponents and escape rates\rf{DasBuch}.

In a series of papers Cvitanovi\'{c}, Putkaradze,
Christiansen and Lan%
\rf{Christiansen97,chfield,LanThesis,CvitLanCrete02,lanVar1,lanCvit07}
showed that effectively low-dimensional return maps can be
constructed for high-dimensional (formally infinite
dimensional)  flows described by dissipative partial
differential equations (PDEs) such as the \KSe\ (KS). Such
flows have state-space topology vastly more complicated than
the Lorenz flow, and collections of local Poincar\'e sections
together with maps from a section to a section are required
to capture all of the important asymptotic dynamics. These KS
studies were facilitated by a restriction to the
flow-invariant subspace of odd solutions, but at a price:
elimination of the translational symmetry of the KS system
and with it physically important phenomena, such as traveling
waves. Traveling (or `relative') unstable coherent solutions
are ubiquitous and play a key role in organization of
turbulent hydrodynamic flows, as revealed both by
simulations\rf{KawKida01,FE03,WK04,Visw07b,GHCW07} and
experimentation\rf{science04}.
    \ES{update references}
For KS\rf{SCD07,SiminosThesis}, and even for a relatively
low-dimensional flow such as the
\cLe\rf{GibMcCLE82,FowlerCLE82} used as an example here, with
the simplest possible continuous (rotational) spatial
symmetry, the symmetry-induced drifts obscure the underlying
hyperbolic dynamics.

The question that we address here is how one can construct
suitable return maps for arbitrarily high-dimensional but
strongly dissipative flows in presence of continuous
symmetries. Our exposition is based in part on
\refrefs{SiminosThesis,DasBuch,Wilczak09}. The reader is
referred to the monographs of Golubitsky and
Stewart\rf{golubitsky2002sp}, Hoyle\rf{hoyll06},
Olver\rf{OlverInv}, Bredon\rf{Bredon72}, and
Krupa\rf{Krupa90} for more depth and rigor than would be wise
to wade into here.

In \refsect{s:symDyn} we review the basic notions of symmetry
in dynamics. \refSect{s:introCLE} introduces the \SOn{2}\
equivariant \cLe\ (CLE), a 5-dimensional set of ODEs that we
use throughout the paper to illustrate the strengths and
drawbacks of different approaches of symmetry reduction. In
\refsect{s:symSol} we describe important classes of solutions
and their symmetries: \eqva, \reqva, \po\ and \rpo s, and use
them motivate the need for symmetry reduction.

In \refsect{s:Hilbert} we briefly review one of the standard tools
by which the spatial symmetry reduction can be achieved:
projection to a Hilbert basis.
In \refsect{sec:mf} we review the {\mframes}, a direct and
efficient method to compute symmetry invariant bases which
goes back to Cartan, and in \refsect{sec:CLeMovFr} we
illustrate this method in application to the \cLe. The method
maps all solutions to a `slice,' a submanifold  of state
space that plays a role for group orbits akin to the role
Poincar\'e sections play in reducing continuous time
invariance. In contrast to the Hilbert basis approach, slices
are local, and one might need to use more than one slice to
capture the flow globally. In \refsect{s:laserMFnum} we show
that a single local slice can suffice for the purpose of
reducing the \cLe\ flow to a return map.
    \ES{Create subsection if that part stays in the paper.
    {\bf PC} dropped:
    We discuss some remedies to the
    situation in \refsect{sec:mf}.}
In \refsect{sec:MovFrameODE} we recast the {\mframes}
into the equivalent, differential \mslices, with
time integration
restricted to a slice fixed by a given \statesp\ point.
As the slice is local, both methods suffer from the same
restrictions, with generic trajectories within a slice bound to encounter
singularities.

In \ref{s:StabReq} we derive an expression for the stability
of \reqva\ (traveling waves) in \reducedsp\
that can be applied without an explicit
calculation of the reduced system.
    \PC{unless you clean up \ref{s:StabReq} now, it will have to
        be omitted.}
	\PC{perhaps add ChaosBook.org historical notes here, or a bit earlier in
	the introduction, sis out of the way.}
