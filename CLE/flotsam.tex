% $Author$ $Date$
%
% Predrag created file siminos/CLE/flotsam.tex      Dec 20 2009

\chapter{Flotsam}

This chapter contains material which has not been included in
publication 
{\tt siminos/CLE/CLE.tex}.

{\bf ES 2009-12-19 replaced:}
 On the other hand when one faces
nonlinear field theories\rf{chfield}, either classical or
quantum, the identification existense of order and organization
identified within the bewildering wealth of solutions. Dynamics
within the chaotic attractor of a low-dimensional, continuous
time, (state-space-)volume contracting flow can be in many
cases\rf{gilmore2003} understood by reduction to a discrete time map within a
Poincar\'e section. For sufficiently strong volume contraction
such a mapping provides a complete topological characterization
of the attractor\rf{gilmore2003}. Moreover the set of compact
invariant solutions, equilibria and periodic orbits, organize
the dynamics around them.

{\bf ES 2009-12-19 dropped: }
Nevertheless, many of the early examples of chaotic attractors where
observed in very drastic truncations of PDEs, such as the Lorenz flow\rf{lorenz}.

, see, for example, 
Cushman and Bates\rf{cushman_global_1997} or Marsden and Ratiu\rf{marsden_introduction_1999}


{\bf ES 2009-12-20 dropped}, not sure it is true and does not offer to the discussion: 

When one takes syzygies into account in rewriting the
dynamical system, singularities are introduced. For instance
if we solve \refeq{eq:syzLaser} for $u_2$ and substitute into \refeq{eq:CLEip}
the latter reads
\beq
\begin{split}
  \dot{u}_1 &=2\,\sigma\,(u_4-u_1)\,,\\
  \dot{u}_2 &=-2\left(\,\frac{u_3^2+u_4^2}{u_1} - \rho_2\, u_3 -\,(\rho_1-u_5)\,u_4\right)\,,\\
  \dot{u}_3 &=-(\sigma\, +1)\,u_3+\rho_2\, u_1+e\, u_4\,,\\
  \dot{u}_4 &=-(\sigma\, +1)\,u_4+\,(\rho_1-u_5)\,u_1+\sigma\, \frac{u_3^2+u_4^2}{u_1}-e\,u_3\,,\\
  \dot{u}_5 &=u_4-b\, u_5\,
\end{split}
\label{eq:CLEipSyz}
\eeq
clearly singular as $u_2\rightarrow 0$. 


Moreover when
one \emph{lifts} the dynamics from the quotient space
$\Manif/G$ to the original space $\Manif$ the transformations
have singularities at the \fixedsp s of
the isotropy subgroups in $\Manif$, in the optimal case, \cf
\refref{GL-Gil07b}. Those singularities do not seem to
restrict our ability to use invariant polynomials to obtain
symmetry reduced projections of the dynamics.

{\bf ES 2009-12-21 dropped:}
The next (or a complimentary) level of organization of (real)
{\Le} attractor is provided by the dense set of \po s
embedded in it\rf{DV03,DasBuch}. 

The lowest level of organization of the
familiar (real) Lorenz system that does not posses a
continuous symmetry can be
understood\rf{DasBuch,SiminosThesis} in terms of the unstable
manifolds of the equilibrium at the origin and the two
(discrete-symmetry-related) equilibria $\EQV{1,2}$. 


, an equilibrium
in a frame rotating with constant angular velocity .
Alternatively we might say that a relative equilibrium is a
periodic orbit that is invariant (as a set) under the action
of $\LGelement{l_p}\in\Group$ for any $l_p$.
    This implies constant angular velocity by
    equivariance.
As we will see in \refsect{s:StabReq}, \REQV{}{1} of {\cLe}
for the parameter set we study here is unstable with one
complex expanding eigenvalue. Yet, being a periodic orbit,
its unstable manifold is three-dimensional, with one
eigendirection corresponding to the direction of $\vf$ which
also coincides with the direction of rotations of the system.
In \reffig{fig:CLE} we plot one trajectory on the unstable
manifold of \REQV{}{1}. While it spirals away from
$\REQV{}{1}$ it also ``drifts'' along the direction of
rotations of the system. This drifting motion obscures
understanding of the stretching mechanism along the expanding
eigendirection and subsequently the folding of the unstable
manifold back to itself.


{\bf ES 2009-12-21 redundant/replaced:}
In {\cLe} a secondary
bifurcation from \REQV{}{1} is expected, according to Krupa's
theorem\rf{Krupa90}, to result in \emph{relative periodic
orbits} that satisfy
\beq
	\LGelement{l_p}x(t+T_p)=x(t)\,,
\eeq
{\ie} they ``return'' after a time period $T_p$ to a point
that maps to the initial one under a group transformation
$\LGelement{l_p}$ with group parameter period $l_p$. A {\rpo}

{\bf ES 2009-12-21 dropped, nobody cares, not important for this paper:}

This assumption is questionable unless it
is forced by the physics of the problem, which I cannot
follow very well. It leads to non-generic bifurcation
behavior, while one would like a model of a physical system
to be robust under perturbations (of the model). Furthermore,
the fact that the Hopf cycle in the general case is an
$\SOn{2}$-orbit has gone unnoticed. The \reqv\ can be
interpreted as an \eqv\ in a rotating frame and the measured
electric field of the laser would be the same in both cases.

{\bf ES 2009-12-21 rephrased:}

Moreover, we wish to emphasize the local nature
of the slices constructed here. As the existense of a slice
depends on the regularity of the group action\rf{FelsOlver99}
and thus for most interesting group actions a slice is bound to
exist only locally.

{\bf ES 2009-12-21 dropped:}

\begin{example}
Consider the standard action of $\SOn{2}$ on \Rls{2}:
\beq
	(x,y) \mapsto (x\cos\theta -y \sin\theta,\,x\sin\theta +y \cos\theta )
\eeq
which is regular on $\Rls{2}\backslash\{0\}$. Thus we can define
a {\csection} by, for instance, the
positive $y$ axis: $x=0,\,y>0$.
We can now construct a moving frame as follows. We write out
explicitly the group transformations:
\begin{subequations}
\begin{align}
 	\overline{x} &= x \cos\theta - y \sin\theta\label{eq:explSO2stnd1}\cont
	\overline{y} &= x \sin\theta + y \cos\theta\label{eq:explSO2stnd2}\,.
\end{align}
\end{subequations}
Then set $\overline{x}=0$ and solve \refeq{eq:explSO2stnd1} for the group
parameter to obtain the moving frame
\beq
	\theta=\tan^{-1}\frac{x}{y}
	\label{eq:SO2stndMF}
\eeq
which brings any point  back to the {\csection}.
\footnote{Implementation note: Here it is important that
$\tan^{-1}$ distinguishes quadrants on the $(x,y)$ plane so
that we get the correct geometric operation.}
Substituting \refeq{eq:SO2stndMF} in the remaining equation,
we get the $\SOn{2}$-invariant expression
\beq
	\overline{y} = \sqrt{x^2+y^2}\,.
\eeq
\end{example}

