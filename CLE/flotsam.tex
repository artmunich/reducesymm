% $Author$ $Date$
%
% Predrag created file siminos/CLE/flotsam.tex      Dec 20 2009

\chapter{Flotsam}

This chapter contains material which has not been included in
publication
{\tt siminos/CLE/CLE.tex}.

{\bf ES 2009-12-22:}

I find the following notion of global vs local reasonable but it contradicts
the usage of local in the text, especially in comparison of invariant polynomials
to \mframes\ and \slice s. Here by \emph{global} method we mean
one which permits reducing dynamics to return maps associated with a finite set
of \Poincare\ sections, while a \emph{local}
method is one that is specific to a solution, even if the latter is a global object such as a periodic
orbit.




{\bf ES 2009-12-19 replaced:}
 On the other hand when one faces
nonlinear field theories\rf{chfield}, either classical or
quantum, the identification existense of order and organization
identified within the bewildering wealth of solutions. Dynamics
within the chaotic attractor of a low-dimensional, continuous
time, (state-space-)volume contracting flow can be in many
cases\rf{gilmore2003} understood by reduction to a discrete time map within a
Poincar\'e section. For sufficiently strong volume contraction
such a mapping provides a complete topological characterization
of the attractor\rf{gilmore2003}. Moreover the set of compact
invariant solutions, equilibria and periodic orbits, organize
the dynamics around them.

{\bf ES 2009-12-19 dropped: }
Nevertheless, many of the early examples of chaotic attractors where
observed in very drastic truncations of PDEs, such as the Lorenz flow\rf{lorenz}.

, see, for example,
Cushman and Bates\rf{CushBat97} or Marsden and Ratiu\rf{MarsdRat94}


{\bf ES 2009-12-20 dropped}, not sure it is true and does not offer to the discussion:

When one takes syzygies into account in rewriting the
dynamical system, singularities are introduced. For instance
if we solve \refeq{eq:syzLaser} for $u_2$ and substitute into \refeq{eq:CLEip}
the latter reads
\beq
\begin{split}
  \dot{u}_1 &=2\,\sigma\,(u_4-u_1)\,,\\
  \dot{u}_2 &=-2\left(\,\frac{u_3^2+u_4^2}{u_1} - \rho_2\, u_3 -\,(\rho_1-u_5)\,u_4\right)\,,\\
  \dot{u}_3 &=-(\sigma\, +1)\,u_3+\rho_2\, u_1+e\, u_4\,,\\
  \dot{u}_4 &=-(\sigma\, +1)\,u_4+\,(\rho_1-u_5)\,u_1+\sigma\, \frac{u_3^2+u_4^2}{u_1}-e\,u_3\,,\\
  \dot{u}_5 &=u_4-b\, u_5\,
\end{split}
\label{eq:CLEipSyz}
\eeq
clearly singular as $u_2\rightarrow 0$.


Moreover when
one \emph{lifts} the dynamics from the quotient space
$\Manif/G$ to the original space $\Manif$ the transformations
have singularities at the \fixedsp s of
the isotropy subgroups in $\Manif$, in the optimal case, \cf
\refref{GL-Gil07b}. Those singularities do not seem to
restrict our ability to use invariant polynomials to obtain
symmetry reduced projections of the dynamics.

{\bf ES 2009-12-21 dropped:}

, an equilibrium
in a frame rotating with constant angular velocity .
Alternatively we might say that a relative equilibrium is a
periodic orbit that is invariant (as a set) under the action
of $\LGelement{l_p}\in\Group$ for any $l_p$.
    This implies constant angular velocity by
    equivariance.

As we will see in \refsect{s:StabReq}, \REQV{}{1} of {\cLe}
for the parameter set we study here is unstable with one
complex expanding eigenvalue. Yet, being a periodic orbit,
its unstable manifold is three-dimensional, with one
eigendirection corresponding to the direction of $\vf$ which
also coincides with the direction of rotations of the system.
In \reffig{fig:CLE} we plot one trajectory on the unstable
manifold of \REQV{}{1}. While it spirals away from
$\REQV{}{1}$ it also ``drifts'' along the direction of
rotations of the system. This drifting motion obscures
understanding of the stretching mechanism along the expanding
eigendirection and subsequently the folding of the unstable
manifold back to itself.


{\bf ES 2009-12-21 redundant/replaced:}
In {\cLe} a secondary
bifurcation from \REQV{}{1} is expected, according to Krupa's
theorem\rf{Krupa90}, to result in \emph{relative periodic
orbits} that satisfy
\beq
	\LGelement{l_p}x(t+T_p)=x(t)\,,
\eeq
{\ie} they ``return'' after a time period $T_p$ to a point
that maps to the initial one under a group transformation
$\LGelement{l_p}$ with group parameter period $l_p$. A {\rpo}

{\bf ES 2009-12-21 dropped:}

rephrased: To cite Fels and Olver\rf{FelsOlver98}:
moving frames where ``first introduced by Gaston Darboux
and brought to maturity by \'Elie Cartan.''


{\bf ES 2009-12-21 dropped:}

We have noted that \SOn{2} acts regularly and freely on
$X^*=\Rls{5}\backslash\{x_1=x_2=y_1=y_2=0\}$ and thus we are
guaranteed to find the fundamental invariants by the method
of moving frames if we restrict attention to $X^*$.


Here we can use the fact that
$- \ssp \cdot \Lg\cdot\Lg \cdot \slicep
 = (\ssp \cdot \slicep )_4 =
    x_1 x_1^{*}
   +x_2 x_2^{*}
   +y_1 y_1^{*}
   +y_2 y_2^{*}
$
is the dot-product restricted to the 4-dimensional
representation of $\SOn{2}$.

A generic  $ \slicep $ can be brought to form $ \slicep  =
(0,1,y_1^{*},y_2^{*})$ by a rotation and rescaling. Then $\Lg
\cdot \slicep   = (1,0,y_2^{*},-y_1^{*})$, and
\beq
\frac{(\vel \cdot \Lg \cdot \slicep )}{(\ssp \cdot\slicep )_4} =
\frac{\vel_1 + \vel_3 y^{*}_2 -\vel_4 y^{*}_1}
     {x_2 + y_1 y^{*}_1 + y_2 y^{*}_2}
%\frac{\vel_1 x^{*}_2 -\vel_2 x^{*}_1 + \vel_3 y^{*}_2 -\vel_4 y^{*}_1}
%     {\vel_1 x^{*}_1 + \vel_2 x^{*}_2 + \vel_3 y^{*}_1 + \vel_4 y^{*}_2}
\,.
\label{PCsectSin}
\eeq


 {\bf ES 2009-12-21 dropped (and PC agrees):}

We say that $\Group$ acts locally freely on \pS\ if for any
$\ssp\in\pS$ the isotropy subgroup $\stab{\ssp}$ is a
discrete subgroups of $\Group$. An r-dimensional compact Lie
group $\Group$ acts \emph{locally freely} on $\pS$ if and
only if it has $r-dimensional$ orbits\rf{FelsOlver99}. A
group $\Group$ acts freely on $\pS$ if all isotropy subgroups
are trivial: \stab{\ssp}=\{e\} for all $\ssp\in \Manif$. If
in addition for each point $\ssp\in \Manif$ there exists an
arbitrarily small neighborhood $U$ such that each orbit of
$\Group$ intersects $U$ in a pathwise connected subset, then
the group acts regularly.

{\bf ES 2009-12-21 dropped:}

A group $\Group$ acts semi-regularly on $\pS$ if all its
orbits have the same dimension. Therefore the group orbits of
a group that acts semi-regularly foliate $\pS$. A sufficient
condition for a semi-regular action is that for any
$\ssp\in\pS$ the isotropy subgroup $\stab{\ssp}$ is a
discrete subgroup of $\Group$. If a Lie group $\Group$ acts
semi-regularly on a manifold $\Manif$,

Note that locally free action implies semi-regular action.

{\bf ES 2010-01-22 dropped:}

In this section we present the {\mframes}, introduced by G. Darboux and
systematized by \'E. Cartan\rf{CartanMF}.
Fels and Olver\rf{FelsOlver98,FelsOlver99} reformulated the method so that a
moving frame is simply an equivariant mapping from the space on which a
group acts to the group parameter.  In \refsect{s:mfReqb} we will
exploit the geometric interpretation of moving frames for a more direct and
efficient approach to symmetry reduction. \refrefs{FelsOlver98,FelsOlver99}

A moving frame is a smooth
$\Group$-equivariant mapping $\rho$ from $\Manif$ to the
$\Group$, that is to the group parameters. \ES{perharps not
needed: One distinguishes between left moving frames for which
the equivariance condition is $\rho(\LieEl x)=\LieEl\rho(x)\,,\
x\in\Manif\,,\ \LieEl\in\Group$ and right moving frames for
which the equivariance condition is $\rho(\LieEl
x)=\rho(x)\LieEl^{-1}\,,\ x\in\Manif\,,\ \LieEl\in\Group$.  As shown in \refref{FelsOlver99} a moving frame exists
in a neighborhood of a point $x\in\Manif$ if and only if
$\Group$ acts freely and regularly near x.} For the practical
construction of a moving frame for an $N-$dimensional Lie group
$\Group$ we will need to define a slice.
Therefore the existense of a moving frame depends on
group orbits having the same dimension.

re-insert?: To construct a moving frame, let $K\subset\Manif$ be a {\csection}. For $x\in \Manif$, let
$\LieEl=\rho(x)$ be the unique group element that maps $x$
to the {\csection}: $g x = \rho(x) x\, \in K$. Then
$\rho:\Manif\rightarrow \Group$ is a right moving frame\rf{FelsOlver98}.

A {\csection} $K$ can be defined by means of level sets of
functions $K_i(x)=c_i$, where $x\in V$ and $i=1,\ldots,r$. If
the $K_i(x)$ coincide with the local coordinates $x_i$ on the
manifold $V$, \ie~$K_i(x)=x_i$, 

{\bf ES:Moved example from movingFrames.tex here, some bits should still be rescued:}

In this section we illustrate symmetry reduction through
the use of invariants computed
with the moving frame method in the example of \cLe.
The $z$-axis is the \fixedsp\ of \SOn{2} acting by
\refeq{eq:SO2act} on \Rls{5}. Therefore we can define
a coordinate {\csection} on $\Rls{5}\backslash\{x_1=x_2=y_1=y_2=0\}$
by, for instance,
\beq%\label{eq:CLEsliceSO2}
x_1=0,\,x_2>0\,.
\eeq
We can now construct a moving frame for the action
\refeq{eq:SO2act} of $\SOn{2}$ as follows. We write out
explicitly the group transformations:
\begin{subequations}%\label{eq:CLEnorm}
\begin{align}
 	\overline{x}_1 &= x_1 \cos\theta - x_2 \sin\theta\cont
	\overline{x}_2 &= x_1 \sin\theta + x_2 \cos\theta\cont
	\overline{y}_1 &= y_1 \cos\theta - y_2 \sin\theta\cont
	\overline{y}_2 &= y_1 \sin\theta + y_2 \cos\theta\cont	
	\overline{z} &= z\,.
\end{align}
\end{subequations}
We set $\overline{x}_1=0$ and solve
\refeq{eq:CLEexplSO2a} for the group parameter to obtain the moving frame
\beq
	\theta=\tan^{-1}\frac{x_1}{x_2}
% 	\label{eq:CLEmf}
\eeq
which brings any point  back to the {\csection}.
Here it is important that
$\tan^{-1}$ distinguishes quadrants in the $(x_1,x_2)$ plane to ensure that the
transformation results in the correct geometric
interpretation, \ie\ to ensure $x_2>0$.
Substituting \refeq{eq:CLEmf} in the remaining equations \refeq{eq:CLEnorm} we
get the invariants
\beq
\begin{split}
	\overline{x}_2 &=  r_1 = \sqrt{x_1^2+x_2^2} \cont
	\overline{y}_1 &= {(x_2 y_1-x_1 y_2)}/{r_1}\cont
	\overline{y}_2 &= {(x_1 y_1+x_2 y_2)}/{r_1}\cont	
	\overline{z} &= z\,.
% 	\label{eq:invLaser}
\end{split}
\eeq
    \ES{The solution $\theta = 2
    \tan^{-1}\frac{-x_2+\sqrt{x_1^2+x_2^2}}{x_1}$ was
    returned by Mathematica. If we use $\theta =
    \tan^{-1}\frac{x_2}{x_1}$ without taking care of the
    quadrant our results are multiplied by $sgn(x_2)$.}


{\bf ES: I will need to rewrite the following}

In terms of projecting dynamics
on variables \refeq{eq:invLaser} (or applying the equivalent
procedure of rotating points back to the \slice) this means that
we need to take into account the direction along which
we approach zero and use the `angle
of descent' as the angle with which we rotate points back to the \slice, if such
points have exactly $x=0$.

Note that the invariants are not defined on
the subspace $U_S$ defined by $x_1=x_2=0$ even though the
group action is non-regular only in a subset of $U_S$, the
$z$-axis $x_1=x_2=y_1=y_2=0$ which is the \fixedsp\ of \SOn{2}.
In the spirit of \refref{GL-Gil07b} the transformations \refeq{eq:invLaser}
can therefore be characterized as non-optimal, in the sense
that we have singularity in a proper superset of $\Fix{\SOn{2}}$.
    \PC{``proper superset''? Il cano no parla questa lingua}
The reason the transformations fail on $U_S$ and not only on the $z$-axis
can be traced back to the way we construct the moving frame. The action
of the group can be thought of as a direct sum of irreducible
actions and the corresponding invariant (linearly irreducible)
subspaces are the $(x_1,\,x_2)$ and $(y_1,\,y_2)$
planes.
%PC OK \ES{Not sure if planes is acceptable term here.}.
Since irreducible subspaces are by definition group-invariant
implies that we could define a moving frame in any one of them
independently. The singular subspace would then be determined
by the fixed points of group action in this subspace alone.
For instance, by choosing an angle in the $(x_1,\,x_2)$ irreducible subspace
as the moving frame map, the singular set is the point
$x_1=x_2=0$ in this irreducible subspace. Going back to the full
$5$-dimensional space the singular set of the transformations
is still given by $x_1=x_2=0$.
%}

While a singularity in $\Fix{\SOn{2}}$ cannot be reached by generic orbits
as {\fixedsp s} are flow invariant, the same is not true for {\fixedsp s}
of irreducible representations of the group action\ES{make sure!!!}.
The fact that trajectories of \cLf\ in \reffig{fig:CLEmf} stay away from $r_1=0$ is therefore
fortuitous.
