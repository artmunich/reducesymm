% siminos/CLE/symRedGeneral.tex
% $Author$ $Date$

%Discussion of symmetry reduction methods
% extracted from siminos/thesis/chapters/symODEs.tex

The purpose of symmetry reduction in differential equations
is to project the dynamics to a space in which the symmetry
group $\Group$ acts trivially. Such a space is called \emph{orbit
 space} because each group orbit of a point in original space
is mapped to a single point in orbit space, or \emph{quotient
space} because the symmetry has been ``divided out'' or
simply \emph{reduced space}. If the original space is a
manifold $\Manif$ it is then customary to write the quotient
space as $\Manif/\Group$. The resulting dynamical system is
called \emph{image} of the original.

\ES{merge this:  Here we
first motivate the method by considering its
finite-rotations version, and then derive its differential
formulation, following \refref{rowley_reduction_2003}.
%
We shall refer to $\pS/\Group$ as {\em `\reducedsp'}.
In the literature this is alternatively called
`desymmetrized {\statesp},'
% `reduced \statesp,'
`orbit space,'
`quotient space,'
or
`image space,'
obtained by mapping equivariant dynamics to invariant dynamics
by methods such as
`moving frames,'
`\csection s,'
`\slice s,'
`Hilbert bases,'
\etc.
}
    \PC{Cite literature that uses each of the above}

\PublicPrivate{}{
The stratification
of $\Manif$ induced by the group action is carried over to
the quotient space with each disconnected set in a stratum
mapped to the same manifold in quotient space. Yet, a
fundamental problem with symmetry reduction is that the orbit
space is in general not a manifold. Unless the action of the
group is free, group orbits do not have the same dimension
and different strata are mapped to manifolds of different
dimension. We will see this property of quotient space
manifest itself in different ways depending on the reduction
method but always introducing some singularity even though
there is nothing singular about $\Manif$ or the flow of the
dynamical system on it.
        \ES{Either define stratum, or simplify the discussion
		here.}
}