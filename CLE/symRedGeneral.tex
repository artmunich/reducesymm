%Discussion of symmetry reduction methods
% extracted from siminos/thesis/chapters/symODEs.tex

The purpose of symmetry reduction in differential equations is to project the dynamics to a space
in which the symmetry group $G$ acts trivially. Such a space is called \emph{orbit space} because each
group orbit of a point in original space is mapped to a single point in orbit space, or \emph{quotient
space} because the symmetry has been ``divided out'' or simply \emph{reduced space}. If the original
space is a manifold $\Manif$ it is then customary to write the quotient space as $\Manif/\Group$.
The resulting dynamical system is called \emph{image} of the original.

The stratification of $\Manif$ induced by the group action is carried over to the quotient space with each disconnected set in a stratum mapped to the same manifold in quotient space.
Yet, a fundamental problem with symmetry reduction is that the orbit space is in general not a manifold.
Unless the action of the group is free, group orbits do not have the same dimension and different
strata are mapped to manifolds of different dimension. We will see this property of quotient space
manifest itself in different ways depending on the reduction method but always introducing some
singularity even though there is nothing singular about $\Manif$ or the flow of the dynamical system
on it.


\subsection{Hilbert basis approach}


The most common approach to symmetry reduction is through the use of a Hilbert basis of invariant
polynomials. One computes a (non-unique) basis of linearly independent polynomials, invariant under the action
of the symmetry group (\cf \refref{gatermannHab} for a discussion of methods) and either rewrites
the dynamics in this basis or maps the solutions to the polynomials.
We will describe how this approach works for the example of \CLe\ in \refsect{sec:CLe}.
The reader is referred to the book of Gilmore and
Lettelier\rf{GL-Gil07b} for a very detailed discussion of
symmetry reduction. For the action \refeq{eq:SO2act} of
$\SOn{2}$ on \Rls{5} a Hilbert basis\rf{GL-Gil07b}  is
\beq
\begin{split}
	u_1 &= x_1^2+x_2^2 \cont
	u_2 &= y_1^2+y_2^2 \cont
	u_3 &= x_1 y_2-x_2 y_1\cont
	u_4 &= x_1 y_1+x_2 y_2\cont
	u_5 &= z\,.
	\label{eq:ipLaser}
\end{split}
\eeq
The polynomials in a Hilbert basis are linearly independent, but functionally dependent through
relations called \emph{syzygies}. For polynomials \refeq{eq:ipLaser} the syzygy is
\beq
 	u_1u_2 -u_3^2-u_4^2 =0\,.
	\label{eq:syzLaser}
\eeq

When one takes syzygies into account in rewriting the dynamical
system, singularities are introduced. Moreover when one
\emph{lifts} the dynamics from the quotient space $\Manif/G$ to
the original space $\Manif$ the transformations have
singularities at the \fixedsp s of the isotropy subgroups in
$\Manif$, in the optimal case, \cf \refref{GL-Gil07b}. Those
singularities do not seem to restrict our ability to use
invariant polynomials to obtain symmetry reduced projections of
the dynamics as we will see in \refchap{chap:lasers}.

What restricts the utility of Hilbert basis methods is that the
determination of a Hilbert basis becomes computationally
prohibitive as the dimension of the system or of the group
increases\rf{gatermannHab,ChossLaut00} and typically
computations are constrained to dimension smaller than ten. As
our goal is to quotient continuous symmetries of
high-dimensional flows, specifically those arising from
truncations of the \KS\ and Navier-Stokes flows
and thus we need an efficient framework.


\subsection{Moving frames}
\label{sec:mf}

In this section we present the method of moving frames of
Cartan\rf{CartanMF} in the formulation of Fels and
Olver\rf{FelsOlver98,FelsOlver99}, also \cf~\refref{OlverInv}
for a pedagogical exposition and the proofs of theorems
listed here. Its purpose is to generate functionally
independent invariants for the action of a group $\Group$ on
a manifold $\Manif$ under certain assumptions, and is not
restricted to reduction problems.

In the following let $\Group$ be $r$-dimensional and act on a $n$-dimensional manifold $\Manif$.
\begin{definition}
 A moving frame is a smooth $\Group$-equivariant mapping $\rho:\,\Manif \rightarrow \Group$.
\end{definition}
One distinguishes between left moving frames for which the equivariance condition is $\rho(\gamma x)=\gamma\rho(x)\,,\ x\in\Manif\,,\ \gamma\in\Group$ and right moving frames for which the equivariance condition is $\rho(\gamma x)=\rho(x)\gamma^{-1}\,,\ x\in\Manif\,,\ \gamma\in\Group$.

The following existence theorem for moving frames will be very important.

\begin{theorem}
 A moving frame exists in a neighborhood of a point $x\in \Manif$ if and
 only if $\Group$ acts freely and regularly near x.
\end{theorem}

For groups acting regularly we can define a {\csection} for the group orbits.
    \PublicPrivate{}{
We follow \refrefs{Mostow57,Pal61,GuiSte90,DuiKol00,rowley_reconstruction_2000}
in referring to a local group-orbit section as a `slice.'
\refRefs{Bredon72,GuiSte90} and others refer to global
group-orbit sections as `cross-sections,' a term that
already has a different and well established meaning in
physics.
    \PC{I am probably wrong in giving up on `cross-sections.'
    Duistermaat\rf{DuiKol00} refers to `slices' on p. 103,
    but it goes back at least to Guillemin\rf{GuiSte90} in
    1984 and Palais\rf{Pal61} in 1961. Tends to be discussed
    in much more difficult context than ours - symplectic
    groups, sections in absence of global charts, etc..
    Palais\rf{Pal61} seems to imply problems arise for
    non-compact Lie groups - ours are compact, should be easier.
    He refers to \rf{Mostow57}. Bredon\rf{Bredon72} discusses
    both cross-sections and slices.
    Guillemin\rf{GuiSte90} is in the CNS library. They define
    `cross-section' say that finding it is very rare. On p.
    200 they say ``existence of a global section is a very
    stringent condition on a group action. The notion of
    `slice' is weaker but has a much broader range of
    existence.'' A submanifold $S$ containing $x$ is called a
    {\em slice} through $x$ if it is invariant under isotropy
    $G_x(S)=S$ plus some other conditions. If $x$ is a fixed
    point of $G$, than slice is invariant under the whole
    group.
    The slice theorem is explained in
    \HREF{http://eom.springer.de/S/s120150.htm}
    {Encyclopaedia of Mathematics}.
    I cannot find anyplace in the thesis your sources
    for term `cross-section'; please complete references
    here if you have better ones.}
    }
\index{slice}\index{cross-section}


\begin{proposition}%[\rf{OlverInv}]
 \label{pro:crossExists}
 Let $\Group$ act regularly on a $n$-dimensional manifold
 $\Manif$ with $r$-dimensional orbits. Define a (local)
 \emph{{\csection}} to be an $(n-r)$-dimensional submanifold $K$
 of $\Manif$ such that $K$ intersects each orbit
 transversally and at most once. If a Lie group $\Group$ acts
 regularly on a manifold $\Manif$, then one can construct a
 local {\csection} passing through any point $x\in \Manif$.
\end{proposition}

\ES{Is this {\csection} related to a cross-section in a $\Group$-bundle?
In other words, can we interpret
the latter as a submanifold in total space \tSp\ of a $\Group$-bundle $(\tSp,\pi,\tSp/\Group)$? }


\begin{theorem}
 Assume the conditions of \refPro{pro:crossExists} hold and
 let $K\subset\Manif$ be a {\csection}. For $x\in \Manif$, let
 $\gamma=\rho(x)$ be the unique group element that maps $x$
 to the {\csection}: $g x = \rho(x) x\, \in K$. Then
 $\rho:\Manif\rightarrow \Group$ is a right moving frame.
\end{theorem}

A {\csection} $K$ can be defined by means of level sets of
functions $K_i(x)=c_i$, where $x\in V$ and $i=1,\ldots,r$. If
the $K_i(x)$ coincide with the local coordinates $x_i$ on the
manifold $V$, \ie~$K_i(x)=x_i$, then we call $K$ a
\emph{coordinate \csection}.

\begin{example}
Consider the standard action of $\SOn{2}$ on \Rls{2}:
\beq
	(x,y) \mapsto (x\cos\theta -y \sin\theta,\,x\sin\theta +y \cos\theta )
\eeq
which is regular on $\Rls{2}\backslash\{0\}$. Thus we can define
a {\csection} by, for instance, the
positive $y$ axis: $x=0,\,y>0$.
We can now construct a moving frame as follows. We write out
explicitly the group transformations:
\begin{subequations}
\begin{align}
 	\overline{x} &= x \cos\theta - y \sin\theta\label{eq:explSO2stnd1}\cont
	\overline{y} &= x \sin\theta + y \cos\theta\label{eq:explSO2stnd2}\,.
\end{align}
\end{subequations}
Then set $\overline{x}=0$ and solve \refeq{eq:explSO2stnd1} for the group
parameter to obtain the moving frame
\beq
	\theta=\tan^{-1}\frac{x}{y}
	\label{eq:SO2stndMF}
\eeq
which brings any point  back to the {\csection}.\footnote{Implementation note: Here it is important that $\tan^{-1}$
distinguishes quadrants on the $(x,y)$ plane so that we get the correct geometric operation.} Substituting \refeq{eq:SO2stndMF} in the remaining equation, we get
the $\SOn{2}$-invariant expression
\beq
	\overline{y} = \sqrt{x^2+y^2}\,.
\eeq
\end{example}

The above \emph{normalization} procedure for the computation of
invariants applied in the example of $\SOn{2}$ can be applied
in much more general situations as follows. Assume $\Group$
acts (locally) freely
    \ES{The condition of free action can be
    relaxed\rf{OlverInv}.}
on \Manif\  and thus $\Group$-orbits have the same dimension,
say $r$, as $\Group$.  Choose a coordinate {\csection}
$K=\{x_1=c_1,\ldots,x_r=c_r\}$ defined by the first $r$
coordinates (relabel coordinates as necessary). Introduce
local coordinates $g=(g_1,\ldots,g_r)$ on $\Group$ in the
neighborhood of the identity. The group transformations are
\beq
	\overline{x}= g \cdot x = w(g,x)\,.
	\label{eq:transNorm}
\eeq
Equating the first $r$ components of the function $w$ to the constants in the definition
of the {\csection} $K_i(x)=c_i$ yields the \emph{normalization equations} for $K$:
\beq
	\overline{x}_1=w_1(g,x)=c_1,\ldots,\overline{x}_r=w_r(g,x)=c_r\,.
	\label{eq:normalization}
\eeq
From the definition of {\csection} and the Implicit Function Theorem the normalization equations
\refeq{eq:normalization} can always be solved for the group parameters in terms of $x$,
yielding the moving frame associated with $K$: $g=\gamma(x)$. Substitution
of the moving frame equation back in \refeq{eq:transNorm} will yield the $n-r$
\emph{fundamental invariants}, in the sense that any other invariant can be expressed
as a function of $\overline{x}_{r+1}\ldots\overline{x}_n$ and they are functionally independent.
Thus they serve to distinguish orbits in the neighborhood of the {\csection}, \ie~two points lie on the same group
orbit if and only if all the fundamental invariants agree. For proof \cf~\refrefs{FelsOlver98,FelsOlver99}.
