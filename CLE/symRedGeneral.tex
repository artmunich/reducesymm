% siminos/CLE/symRedGeneral.tex
% $Author$ $Date$

Action of a symmetry group \Group\ on \pS\ stratifies
the \statesp\ into a set of group orbits,
 each group orbit an
equivalence class. The goal of {\em symmetry reduction} is the
identification of a unique point as the
representative of a group orbit, and the replacement
of the original \statesp\ by
the space of such points, the {\em \reducedsp}.
In the literature this space is alternatively called
\emph{desymmetrized \statesp},
\emph{symmetry-reduced space},
\emph{orbit space}, or \emph{quotient space}
$\pS/\Group$ because symmetry has been `divided out.'
Symmetry group \Group\ of equivariant dynamics acts trivially
in reduced space, and the resulting dynamical system, called
by Gilmore and Lettelier\rf{GL-Gil07b} the \emph{image}, is
symmetry {\em invariant}, in the sense that its symmetry group is
the identity.
The mapping equivariant dynamics to invariant dynamics is
implemented
by methods such as
{\em \mframes},
{\em \mslices} or {\em \csection s},
{\em Hilbert bases},
\etc.
%
\ES{merge this:  Here we
first motivate the method by considering its
finite-rotations version, and then derive its differential
formulation, following \refref{rowley_reduction_2003}.
{\bf PC} irrelevant remark: I did not `follow'
\refref{rowley_reduction_2003}. I first derived the equations,
then identified them in earlier papers, as is what usually happens
}
    \PC{Cite literature that uses each of the above}

In atomic physics, geophysics and other
low-dimensional physical problems with spatial symmetries,
the symmetry reduction is customarily implemented just as we
did it in \refeq{eq:CartToPol}, by going to the `natural'
coordinate system (polar, cylindrical, \etc). That works well
for linear systems, but not so well for nonlinear flows;
note, for example, that these coordinate transformations
introduce non-physical singularities
in \refeq{eq:PolarCLeTheta}at $r_1=0$ and $r_2=0$.

What are we really doing when redefining dynamics in terms of
such invariant coordinates? We are recasting equivariant
dynamics of $(\ssp_1,\ssp_2,\cdots)$ coordinates in terms of
rotationally invariant lengths
$(r_1=(\ssp_1^2+\ssp_2^2)^2,\cdots)$, volumes and other
invariant quantities. Indeed, the problem of symmetry
reduction had been elegantly solved nearly a century ago.
Physical laws should have the same form in
symmetry-equivalent coordinate frames, so they are often
formulated in terms of functions (Hamiltonians, Lagrangians,
$\cdots$) invariant under a given set of symmetries. Given a
symmetry, what is the most general functional form of such
law? According to the Hilbert-Weyl theorem, for a compact
group $\Group$ there exists a finite $\Group${-invariant}
Hilbert polynomial basis $\{u_1,u_2, \dots,u_m\}$, $ m \geq
d$, such that any $\Group${-invariant} polynomial can be
written as a multinomial
\beq
h(\ssp) = p(u_1(\ssp),u_2(\ssp), \dots,u_m(\ssp))
    \,,\qquad \ssp \in \pS
\,.
\ee{HilbWeyl}
The dynamical equations then follow from the chain rule
\beq
 \dot{ u}_i=\frac{\partial u_i}{\partial x_j} \, \dot{x}_j
 \,,
\ee{HilbChainRl}
upon substitution $\{x_1,x_2,\cdots,x_d\}$ $\to$
$\{u_1,u_2,\cdots,u_m\}$. One can either rewrite the dynamics
in this basis, or one can simply plot the `image' of
solutions computed in the original, equivariant basis in
terms of these invariant polynomials.

Unfortunately, while the idea is elegant, an explicit
construction of $\Group${-invariant} basis can in practice be
a laborious undertaking. One trades in the equivariant
\statesp\ coordinates $\{x_1,x_2,\cdots,x_d\}$ for a
non-unique set  of $m \geq d$ invariant polynomials
$\{u_1,u_2,\cdots,u_m\}$. These polynomials are linearly
independent, but functionally dependent through $m - d + N$
nonlinear relations called \emph{syzygies}. Their
determination becomes quickly computationally prohibitive as
the dimension of the system and/or group
increases\rf{gatermannHab,ChossLaut00}, and in practice
typical computations are confined to dimensions less than
ten. As our goal is to quotient continuous symmetries of
high-dimensional flows, high meaning $10^2$-$10^6$ coupled
ODEs, such as those arising from truncations of the \KS\ and
Navier-Stokes flows, the Hilbert basis are at present not a
feasible option.

Nevertheless, as the symmetry reduction of moderate-dimension
flows by the method invariant polynomials offers a clean
benchmark for other approaches symmetry-reduction, we start
by showing how it works for \cLf.
