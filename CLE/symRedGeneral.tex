% siminos/CLE/symRedGeneral.tex
% $Author$ $Date$

%Discussion of symmetry reduction methods
% extracted from siminos/thesis/chapters/symODEs.tex

The purpose of symmetry reduction in differential equations
is to project the dynamics to a space in which the symmetry
group $\Group$ acts trivially. Such a space is called \emph{orbit
 space} because each group orbit of a point in original space
is mapped to a single point in orbit space, or \emph{quotient
space} because the symmetry has been ``divided out'' or
simply \emph{reduced space}. If the original space is a
manifold $\Manif$ it is then customary to write the quotient
space as $\Manif/\Group$. The resulting dynamical system is
called \emph{image} of the original.

\ES{merge this:  Here we
first motivate the method by considering its
finite-rotations version, and then derive its differential
formulation, following \refref{rowley_reduction_2003}.
%
We shall refer to $\pS/\Group$ as {\em `\reducedsp'}.
In the literature this is alternatively called
`desymmetrized {\statesp},'
% `reduced \statesp,'
`orbit space,'
`quotient space,'
or
`image space,'
obtained by mapping equivariant dynamics to invariant dynamics
by methods such as
`moving frames,'
`\csection s,'
`\slice s,'
`Hilbert bases,'
\etc.
}
    \PC{Cite literature that uses each of the above}

In the following it will be useful to introduce the
notion of a \emph{\slice}, an $(n-r)$-dimensional submanifold $K$
of $\Manif$ such that $K$ intersects all group orbits in an
open neighborhood of $\ssp \in K$
transversally and at most once.
In other words, \slice\ is a Poincar\'e section for group
orbits. As is the case for the dynamical Poincar\'e sections,
in general a single \slice\ does not suffice to intersect all
group orbits of points in \pS. Fels and Olver\rf{FelsOlver99}
call a manifold $K$ that intersects \emph{all} group orbits a
\emph{regular cross-section}, and refer to a \slice\ as a local
cross-section. Here we prefer the term slice
to cross-section as the latter has a well established usage in the physics
literature.

Loosely speaking, one can construct a local {\csection} passing
through any point $x\in \Manif$ if the group orbits of \Group\
have the same dimension, therefore not in the {\fixedsp}
of a continuous subgroup of \Group. The interested reader is referred to
\refref{FelsOlver99} for rigorous treatment.

As the systems that we study here are not associated with a
variational principle, Noether's theorem does not
apply, and in general no conserved
quantities are associated with continuous symmetries of
such systems.
    \PC{include discussion, references from the blog here.}
    \PC{not sure about this: ``
Such a conserved quantity would restrict dynamical
trajectories to an invariant manifold locally transverse to
the direction of group action. In the contrary here the
system also evolves along the direction of group action.
    ''
Being on a constant energy surface does not mean we do
not evolve in time, for example.
{\bf ES:} You are right, this statement is not correct in
general. What I had in mind was cases such as
conservation of angular momentum in central force 		
problems fixes the plane of motion.
}


\PublicPrivate{}{
The stratification
of $\Manif$ induced by the group action is carried over to
the quotient space with each disconnected set in a stratum
mapped to the same manifold in quotient space. Yet, a
fundamental problem with symmetry reduction is that the orbit
space is in general not a manifold. Unless the action of the
group is free, group orbits do not have the same dimension
and different strata are mapped to manifolds of different
dimension. We will see this property of quotient space
manifest itself in different ways depending on the reduction
method but always introducing some singularity even though
there is nothing singular about $\Manif$ or the flow of the
dynamical system on it.
        \ES{Either define stratum, or simplify the discussion
		here.}
}
