% siminos/CLE/05fixMe.tex
% $Author$ $Date$

% CLe reduced
% From siminos/thesis/chapters/lasersSym.tex

%\subsection{Stability of \reqva}
%\label{s:StabReq}

Using the {\mframes} to map dynamics on a slice $\pSRed$
as in \refsect{sec:mf} or restricting
integration on the slice as in \refsect{sec:MovFrameODE},
the reduced \statesp\ is
identified (at least locally) with the {\slice}
${\pSRed}$. This provides a means of calculating stability
of \reqva\ in reduced \statesp. Proof that such a notion of stability
of \reqva\ is meaningful is given, for instance, in \refref{Krupa90}.
Our treatment is similar in spirit 
to that of Chossat and Lauterbach\rf{ChossLaut00} but our final 
prescription allows to calculate stability in reduced space
from the equivariant vector field in a more straightforward manner.
\ES{To Predrag: I couldn't find Chossat and Lauterbach book here.
If you have it would it be easy to have a look at their expression 
for stability of \reqva? Is this claim correct? I remember that my 
motivation to derive it was that their expression is too implicit
to use in practice, while Krupa's is totally abstract. But maybe
I was missing something.}
To simplify notation we only treat the one-dimensional group case 
and drop all group parameter related indices, writing $\Lg_{ij}$ instead of
$(\Lg_1)_{ij}$ \etc

In similar manner to our decomposition of vector field in \refsect{sec:MovFrameODE}
we decompose $\vel(\ssp)$ for a point $\ssp$ on a \reqv\ into
a part along the group action tangent and a part $\velRed$ transverse to it,
% \beq
% 	\vel(\ssp)=\vel_\shortparallel(\ssp)+\velRed(\ssp)\,,
% \ee{flowSplit}
using the projection operator
\beq
 	\PperpOp_{ij}(\ssp)=\delta_{ij}-
    \frac{\groupTan(\ssp)_i \groupTan(\ssp)_j}{\groupTan(\ssp)^2}
\ee{transvProj}
that projects a $d$-dimensional flow $v(\ssp)$ onto
flow
\beq
	\dot{\sspRed} = \velRed(\ssp) = \vel(\ssp)
    - \groupTan(\ssp) \frac{\groupTan(\ssp)^T \vel(\ssp)}{\groupTan(x)^2}
\ee{transvFlowSlice}
in a $(d\!-\!1)$-dimensional {\slice} transverse to the
direction fixed by the point $\ssp$.
    \PC{the problem is that this formula is based on
    the \emph{method of connections} of Rowley
    \etal\rf{rowley_reduction_2003} which we cannot use
    because its additional \emph{geometric phase}. They prove
    a theorem
    - as well as probably Beyn - that it is OK for \reqva.
    {\bf ES:} Yes, this is why we only use it for stability of \reqva. 
	      I will have to try rpo stability using projection we use
	      for integration on the slice, but it is too late for this paper.
	      I had the well known problem back then, I did not understand
	      derivation of Rowley and Marsden.
    }
To compute stability eigenvalues of \reqv\
we only need to consider the linearization of $\velRel$
which is identified with the restriction of $\vf$ in reduced space.
The {\stabmat} $\bar{\Mvar}_{ij}$ is then given by
    \PC{Here the projection operator \refeq{transvProj} is OK,
    as the action of the group on $\ssp_{\REQV{}{1}}$ is trivial?
    Not sure...
    }
 \beq
	\bar{\Mvar}_{ij} = \frac{\partial}{\partial x_j}(\PperpOp  \vf)_i
		= \Pperp_{ik}\frac{\partial  \vf_k}{\partial x_j}
          +\frac{\partial \Pperp_{ik}}{\partial x_j} \vf_k
		\label{eq:stabreqvdef}
 \eeq
Now
% ES: Useless, what was I thinking?
% \beq
% 	\Pperp_{ik}	= \delta_{ik}-\frac{\groupTan(x)_i\groupTan(x)_k}{\groupTan(x)^2}
% 			= \delta_{ik}-\frac{\Lg_{im} x_m \Lg_{k\ell} x_\ell}{\groupTan(x)^2}
% \eeq
% and
\bea
	\frac{\partial \Pperp_{in}}{\partial x_j}  &=&  -\frac{\partial}{\partial x_j}\left(\frac{\groupTan(x)_i \groupTan(x)_n}{\groupTan(x)^2}\right)\continue
% 			&=& -\left(\frac{\Lg_{iq}\delta_{jq} \groupTan(x)_n}{\groupTan(x)^2}+\frac{\groupTan(x)_i \Lg_{n\ell}\delta_{j\ell}}{\groupTan(x)^2}-\frac{\groupTan(x)_i \groupTan(x)_n}{\groupTan(x)^4}\frac{\partial}{\partial x_j}\groupTan(x)^2 \right)\continue
			&=& -\left(\frac{\Lg_{ij} \groupTan(x)_n}{\groupTan(x)^2}+\frac{\groupTan(x)_i \Lg_{nj}}{\groupTan(x)^2}-2\frac{\groupTan(x)_i \groupTan(x)_n}{\groupTan(x)^4}\Lg_{mj}\groupTan(x)_m \right)\continue
% 			&=& -\frac{1}{\groupTan(x)^2}\left(\Lg_{ij}\groupTan(x)_n+\groupTan(x)_i \Lg_{nj}-2\frac{\groupTan(x)_i \groupTan(x)_n}{\groupTan(x)^2}\Lg_{mj}\groupTan(x)_m \right)\continue
			&=& -\frac{1}{\groupTan(x)^2}\left(\groupTan(x)_n\left(\Lg_{ij}-\frac{\groupTan(x)_i }{\groupTan(x)^2}\Lg_{mj}\groupTan(x)_m\right)+\groupTan(x)_i\left( \Lg_{nj}-\frac{ \groupTan(x)_n}{\groupTan(x)^2}\Lg_{mj}\groupTan(x)_m\right) \right)\continue
% 			&=& -\frac{1}{\groupTan(x)^2}\left(\groupTan(x)_n\left(\delta_{im}-\frac{\groupTan(x)_i }{\groupTan(x)^2}\groupTan(x)_m\right)\Lg_{mj}+\groupTan(x)_i\left( \delta_{nm}-\frac{ \groupTan(x)_n}{\groupTan(x)^2}\groupTan(x)_m\right)\Lg_{mj} \right)\continue
			&=& -\frac{1}{\groupTan(x)^2}\left(\groupTan(x)_n \Pperp_{im} \Lg_{mj}+\groupTan(x)_i \Pperp_{nm} \Lg_{mj} \right)\,.
\eea

Therefore \refeq{eq:stabreqvdef} takes the form
% \beq
% 	\bar{\Mvar}_{ij}=\Pperp_{in}\frac{\partial  \vf_n}{\partial x_j}-\frac{1}{\groupTan(x)^2}\left(\groupTan(x)_n \Pperp_{im} \Lg_{mj}+\groupTan(x)_i \Pperp_{nm} \Lg_{mj} \right)\vf_n
% \eeq
% or in matrix form
\beq
	\mathbf{\bar{\Mvar}}=\PperpOp \mathbf{A}-\frac{1}{\groupTan(x)^2}\left( \left[\vf^T \groupTan(x)\right] \left(\PperpOp \Lg\right) +\groupTan(x) \otimes \left[\vf^T \left( \PperpOp \Lg\right)\right] \right)\,,
	\label{eq:reqvStab}
\eeq
where $A_{ij}=\frac{\partial \vf_i}{\partial x_j}$. This
expression allows to calculate \reducedsp\ stability of \reqva\ working
in the equivariant variables, without explicit knowledge of
the vector field  $\velRed$ in reduced space. Applying
\refeq{eq:reqvStab} for \reqv\ \REQV{}{1} of \cLe\ we obtain
\beq
(\eigExp_{1,2},\eigExp_3,\eigExp_4, \eigExp_5)
= (0.0938179 \pm 10.1945 i,-11.0009,-13.8534)
\eeq
the same eigenvalues \refeq{eq:CLeREQBstab} we computed in
polar coordinates along with an extra zero eigenvalue. This extra
eigenvalue appears since we work here in equivariant variables
in which a marginal direction due to $\SOn{2}$ symmetry exists. Being
able to compute stability eigenvalues of \reqva\ directly in
equivariant variables is not as important in \cLe\ since the
system can be easily written in polar representation \refeq{eq:PolarCLeTheta}
but it can be convenient in higher dimensions where an
explicit change of variables is often intractable.
