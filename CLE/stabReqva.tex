% siminos/CLE/05fixMe.tex
% $Author$ $Date$

% CLe reduced
% From siminos/thesis/chapters/lasersSym.tex

%\subsection{Stability of \reqva}
%\label{s:StabReq}

Using the {\mframes} to map dynamics on a slice $\mathcal{K}$
as in \refsect{sec:mf} or restricting
integration on the slice as in \refsect{sec:MovFrameODE},
the reduced \statesp\ is
identified (at least locally) with the {\slice}
$\mathcal{K}$. This provides a means of calculating stability
of \reqva\ in reduced \statesp. For a
rigorous treatment of stability of \reqva\ see \refref{Krupa90}.
    \ES{Perhaps some paper by Field
    also achieves that and I will have to check Chossat and
    Lauterbach\rf{ChossLaut00} again for the degree of `rigor'}
Our treatment is similar in spirit
to that of Chossat and Lauterbach\rf{ChossLaut00} but we obtain
expressions that allow one to compute stability in reduced space
from the equivariant vector field in a straightforward manner.

For a point $x$ on a \reqv\ that lies on {\slice}
$\mathcal{K}$ we decompose $\vel(x)$
in \refeq{eq:difeq} in a part $\vel_\shortparallel$ parallel
to the group action and a part $\vel_\perp$ transverse to it,
\beq
	\vel(\ssp)=\vel_\shortparallel(\ssp)+\vel_\perp(\ssp)\,,
\ee{flowSplit}
using the projection operator
\beq
 	\PperpOp_{ij}(\ssp)=\delta_{ij}-
    \frac{(\Lg \ssp)_i (\Lg \ssp)_j}{(\Lg \ssp)^2}
\ee{transvProj}
that projects a $d$-dimensional flow $v(\ssp)$ onto
flow
\beq
	\dot{\ssp}_\perp = \vel_\perp(\ssp) = \vel(\ssp)
    - \Lg \ssp \frac{(\Lg\ssp)\cdot \vel(\ssp)}{(\Lg \ssp)^2}
\ee{transvFlowSlice}
in a $(d\!-\!1)$-dimensional {\slice} transverse to the
direction fixed by the point $\ssp$.
To compute stability eigenvalues of \reqv\
we only need to consider the linearization of $\vf_\perp$
which is identified with the restriction of $\vf$ in reduced space.
The {\stabmat} $\bar{\Mvar}_{ij}$ is then given by
    \PC{Here the projection operator \refeq{transvProj} is OK,
    as the action of the group on $\ssp_{\REQV{}{1}}$ is trivial?
    Not sure...
    }
 \beq
	\bar{\Mvar}_{ij} = \frac{\partial}{\partial x_j}(\PperpOp \cdot  \vf)_i
		= \Pperp_{ik}\frac{\partial  \vf_k}{\partial x_j}
          +\frac{\partial \Pperp_{ik}}{\partial x_j} \vf_k
		\label{eq:stabreqvdef}
 \eeq
Now
\beq
	\Pperp_{ik}	= \delta_{ik}-\frac{(\Lg \cdot x)_i(\Lg \cdot x)_k}{(\Lg \cdot x)^2}
			= \delta_{ik}-\frac{\Lg_{im} x_m \Lg_{k\ell} x_\ell}{(\Lg \cdot x)^2}
\eeq
and
\bea
	\frac{\partial \Pperp_{in}}{\partial x_j}  &=&  -\frac{\partial}{\partial x_j}\left(\frac{\Lg_{iq} \cdot x_q \Lg_{n\ell} \cdot x_\ell}{(\Lg \cdot x)^2}\right)\continue
			&=& -\left(\frac{\Lg_{iq}\delta_{jq} \Lg_{n\ell}x_\ell}{(\Lg \cdot x)^2}+\frac{\Lg_{iq}x_q \Lg_{n\ell}\delta_{j\ell}}{(\Lg \cdot x)^2}-\frac{\Lg_{iq}x_q \Lg_{n\ell}x_\ell}{(\Lg \cdot x)^4}\frac{\partial}{\partial x_j}(\Lg \cdot x)^2 \right)\continue
			&=& -\left(\frac{\Lg_{ij} \Lg_{n\ell}x_\ell}{(\Lg \cdot x)^2}+\frac{\Lg_{iq}x_q \Lg_{nj}}{(\Lg \cdot x)^2}-2\frac{\Lg_{iq}x_q \Lg_{n\ell} x_\ell}{(\Lg \cdot x)^4}\Lg_{mj}\Lg_{mp}x_p \right)\continue
% 			&=& -\frac{1}{(\Lg \cdot x)^2}\left(\Lg_{ij}\Lg_{n\ell}x_\ell+\Lg_{iq}x_q \Lg_{nj}-2\frac{\Lg_{iq}x_q \Lg_{n\ell} x_\ell}{(\Lg \cdot x)^2}\Lg_{mj}\Lg_{mp}x_p \right)\continue
			&=& -\frac{1}{(\Lg \cdot x)^2}\left(\Lg_{n\ell}x_\ell\left(\Lg_{ij}-\frac{\Lg_{iq}x_q }{(\Lg \cdot x)^2}\Lg_{mj}\Lg_{mp}x_p\right)+\Lg_{iq}x_q\left( \Lg_{nj}-\frac{ \Lg_{n\ell} x_\ell}{(\Lg \cdot x)^2}\Lg_{mj}\Lg_{mp}x_p\right) \right)\continue
			&=& -\frac{1}{(\Lg \cdot x)^2}\left(\Lg_{n\ell}x_\ell\left(\delta_{im}-\frac{\Lg_{iq}x_q }{(\Lg \cdot x)^2}\Lg_{mp}x_p\right)\Lg_{mj}+\Lg_{iq}x_q\left( \delta_{nm}-\frac{ \Lg_{n\ell} x_\ell}{(\Lg \cdot x)^2}\Lg_{mp}x_p\right)\Lg_{mj} \right)\continue
			&=& -\frac{1}{(\Lg \cdot x)^2}\left(\Lg_{n\ell}x_\ell \Pperp_{im} \Lg_{mj}+\Lg_{iq}x_q \Pperp_{nm} \Lg_{mj} \right)\continue
\eea

Therefore \refeq{eq:stabreqvdef} takes the form
\beq
	\bar{\Mvar}_{ij}=\Pperp_{in}\frac{\partial  \vf_n}{\partial x_j}-\frac{1}{(\Lg \cdot x)^2}\left(\Lg_{n\ell}x_\ell \Pperp_{im} \Lg_{mj}+\Lg_{iq}x_q \Pperp_{nm} \Lg_{mj} \right)\vf_n
\eeq
or in matrix form
\beq
	\mathbf{\bar{\Mvar}}=\PperpOp \mathbf{A}-\frac{1}{(\Lg \cdot  x)^2}\left( \left[\vf \cdot \left(\Lg \cdot x\right)\right] \left(\PperpOp \Lg\right) +\left(\Lg \cdot x\right) \otimes \left[\vf \cdot \left( \PperpOp \Lg\right)\right] \right)\,,
	\label{eq:reqvStab}
\eeq
where $A_{ij}=\frac{\partial \vf_i}{\partial x_j}$. This
expression allows to calculate stability of \reqva\ working
in the equivariant variables, without explicit knowledge of
the vector field  $\vf_\perp$ in reduced space. Applying
\refeq{eq:reqvStab} for \reqv\ \REQV{}{1} of \cLe\ we obtain
\beq
	\eigRe[1]\pm i\eigIm[1]= 0.0938\pm 10.1945i,\,
    \eigExp[3]=-11.0009,\, \eigExp[4]= -13.8534,\, \eigExp[5]=0,
\eeq
the same eigenvalue \refeq{eq:CLeREQBstab} we computed in
polar coordinates along with a zero eigenvalue.  This last
eigenvalues is present here since we work in equivariant variables
in which a marginal direction due to $\SOn{2}$ symmetry exists. Being
able to compute stability eigenvalues of \reqva\ directly in
equivariant variables is not as important in \cLe\ since the
system can be easily written in polar representation \refeq{eq:PolarCLeTheta}
but it can be in higher dimensions where an
explicit change of variables is often intractable.
