% siminos/CLE/symDyn.tex
% $Author$ $Date$

\section{\label{s:symDyn} Symmetries of dynamical systems}

Consider a system of \ode s of the form
\beq
	\dot{\ssp} = \vf(\ssp)
	\label{eq:difeq}
\eeq
with $\vf$ a smooth vector field and $x\in\pS\subset\Rls{d}$.
%     \PC{kind of things already cleaned up in ChaosBook.org; This is wrong.
%     $\vf$ is a tangent field, $f^t$ is a mapping. You must define
% 	tangent bundle if you really want to say this. But why say it?}
Any compact Lie group acting on $\Rls{d}$ can be identified
with a subgroup of $\On{d}$, \cf\ for example \refref{golubII},
so we assume the symmetry group is a subgroup of $\Group\subseteq\On{d}$ in what
follows.
Here we are interested in the role continuous symmetries play in dynamics, so we
assume $\Group$ is a compact Lie group. For much of what we need to
understand it suffices to study the 1-dimensional compact Lie group \SOn{2},
the symmetry of \cLe, our main example in this paper. Generalization to
symmetries of PDEs such as \KS\ and \pCf\ is immediate.

A group element $\LieEl\in\On{d}$ is a symmetry of
\refeq{eq:difeq} if for every solution $x(t)$, $\LieEl x(t)$ is
also a solution. Equivalently, $\LieEl$ is a symmetry of
\refeq{eq:difeq} if
\beq
	\vf(\LieEl x) =\LieEl \vf(x)
	\label{eq:equiv}
\eeq
for all $x\in\Rls{d}$. We say that
$\vf$ \emph{commutes} with $\LieEl$ or that $\vf$ is
$\LieEl$-\emph{equivariant}. When $\vf$ commutes with all
$\LieEl\in\Group$ we say that $\vf$ is $\Group$-equivariant.
The finite time flow $\flow{t}{\LieEl x}$ through $\LieEl
x$ satisfies the equivariance condition
\beq\label{eq:equivFinite}
\flow{t}{\LieEl x}=\LieEl\flow{t}{x}
\eeq
from definition of symmetry and
uniqueness of solutions. In physics literature the term
$invariant$ is most commonly used; in
Hamiltonian systems a symmetry is manifested as invariance of the
Hamiltonian under the symmetry group action.
%\ES{To Predrag: Do you agree with this statement?}.
% PC: OK


An element of a compact Lie group
continuously connected to identity can be written as
\beq
\LieEl(\gSpace)=e^{\gSpace \cdot \Lg }
	\,,\qquad
\gSpace \cdot \Lg  = \sum \gSpace_a \Lg_a,\; a=1,2, \cdots, N
\,,
\ee{FiniteRot}
where
$\gSpace \cdot \Lg$
is a {\em Lie algebra} element,  and $\gSpace_a$ are the parameters
of the transformation.
{Unitary} transformations $ \exp(\gSpace \cdot {\Lg}) $ are
generated by sequences of infinitesimal steps of form
\beq
\LieEl(\delta\gSpace) \simeq 1 + \delta \gSpace \cdot \Lg
% \LieEl{}_i{}^j \simeq \delta_i^j +  \delta \gSpace_a \, (\Lg_a)_i^j
    \,,\quad
\delta\gSpace \in \reals^N
    \,,\quad
|\delta \gSpace| \ll 1
    \, ,
\ee{intsmLieTransf}
where $\Lg_a$, the {\em generators} of infinitesimal
transformations, are a set of linearly independent
$[d\!\times\!d]$ anti-hermitian matrices, $(\Lg_a)^\dagger =
- \Lg_a$, acting linearly on the $d$-dim\-ens\-ion\-al \statesp\
$\pS$. Repeated indices are summed throughout this section.

For continuous groups the {Lie algebra}, \ie,
the set of $N$ generators $\Lg_a$ of infinitesimal
transformations, takes the role that the $|\Group|$ group
elements play in the theory of discrete groups. The flow field
at the \statesp\ point $\ssp$ induced by the action of the group
is given by the set of $N$ tangent fields
\beq
\groupTan_a(\ssp)_{i}= (\Lg_a){}_{i}{}^j \ssp_j
\,.
\ee{GroupTangField}


% \subsection{$\SOn{2}$ equivariance}
% Predrag                           Sep 19 2009
% extracted from siminos/thesis/chapters/symInfm.tex
% Predrag                           Aug 22 2009
%       extracted from wilczak/blog/flow.tex

% A flow $\dot{\ssp}= \vel(\ssp)$ is $\Group$-equivariant
% \refeq{GvCommut} if
%
% \beq
% \vel(\ssp)=\LieEl^{-1} \, \vel(\LieEl \, \ssp)
% \,,\qquad \mbox{for all } \LieEl \in {\Group}
% \,.
% \ee{eq:FiniteRot}
For an infinitesimal transformation \refeq{intsmLieTransf}
the $\Group$-equivariance condition \refeq{eq:equiv}
becomes
\[
\vel(\ssp) =(1-\gSpace \cdot \Lg) \, \vel(\ssp+\gSpace \cdot \Lg \ssp) + \cdots
       =\vel(\ssp)- \gSpace \cdot \Lg \vel(\ssp)
             + \frac{d\vel}{d\ssp} \,\gSpace \cdot \Lg \ssp + \cdots
\,.
\]
The $\vel(x)$ cancel, and $\gSpace_a$ are arbitrary. Denote
the group flow tangent field at \ssp\ by
$\groupTan_a(\ssp)_{i}= (\Lg_a){}_{i}{}^j \ssp_j$. Thus the
infinitesimal, Lie algebra $\Group$-equivariance condition is
\beq
%  \left(
%    \Lg_a  - \groupTan_a(\ssp) \cdot \frac{\partial}{\partial \ssp}
%  \right) \vel(\ssp) =
  \groupTan_a(\vel)  - \Mvar(\ssp) \, \groupTan_a(\ssp) =0
  \,,
\ee{inftmInv}
where $\Mvar = {\pde \vel}/{\pde \ssp}$ is the \stabmat.
\beq
{\cal L}_{\groupTan_a} \vel =
\left.\left(
  \Lg_a - \frac{\partial}{\partial y}(\Lg_a \ssp)
 \right) \vel(y)\right|_{y=\ssp}
\ee{LieDeriv}
is known as the {\em Lie derivative} of the dynamical flow
field $\vel$ along the direction of the infinitesimal
group-rotation induced flow $\groupTan_a(\ssp)= \Lg_a \ssp$.

The equivariance condition \refeq{inftmInv} states that the two
flows, one induced by the dynamical vector field $\vel$, and
the other by the group tangent field $\groupTan$, commute if
the Lie derivatives (or the `Lie brackets ' or `Poisson
brackets') vanish.
In general checking equivariance as a Lie algebra condition
\refeq{inftmInv} is easier than checking it for global,
finite angle rotations \refeq{eq:equiv}.
