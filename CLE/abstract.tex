Two approaches to continuous symmetry reduction suited to
reducing higher- and even very
high-dimensional dissipative flows to local return maps are
presented. In the Hilbert polynomial basis approach the
equivariant dynamics is rewritten in terms of invariant
coordinates. In the method of moving frames the state space
is `sliced' locally in such a way that each group orbit of
symmetry-equivalent points is represented by a single point.
In either approach numerical computations can be performed in
the original, full state-space representation, and then the
solutions can be projected onto the symmetry reduced state
space. The two methods are illustrated by
reduction of the complex Lorenz system, a 5-dimensional
dissipative flow with rotational symmetry.
