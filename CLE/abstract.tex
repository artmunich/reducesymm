We present two continuous symmetry reduction methods for
reducing high-dimensional dissipative flows to local return
maps. In the Hilbert polynomial basis approach the equivariant
dynamics is rewritten in terms of invariant coordinates. In the
`method of moving frames' or `method of slices' the state space
is `sliced' locally in such a way that each group orbit of
symmetry-equivalent points is represented by a single point. In
either approach, numerical computations can be performed in the
original state-space representation, and the solutions are then
projected onto the symmetry-reduced state space. The two methods
are illustrated by reduction of the complex Lorenz system, a
5-dimensional dissipative flow with rotational symmetry. While
the Hilbert polynomial basis approach appears not feasible for
high-dimensional flows, the symmetry reduction by the `method of
moving frames' offers hope.
