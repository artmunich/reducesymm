Two approaches to continuous symmetry reduction suited to
reducing higher- and even very high-dimensional
dissipative flows to local return maps are presented. 
\JFG{suggestion:''We present two continuous symmetry 
reduction methods for reducing high-dimensional dissipative 
flows to local return maps.''}
In the Hilbert polynomial basis approach,
    \ES{I did not try to focus in Hilbert basis approach as
    it is well described elsewhere and I think it is not
    applicable to very high-dimensional flows due to
    computational restrictions in determination of the basis.
    It is included for comparison purposes and completeness.
    We also don't offer any details on how to compute a basis
    systematically as it would take up too much space to
    describe it properly: even in Gilmore and
    Letelier\rf{GL-Gil07b} details are not clear for the
    compact group case, Chossat and
    Lauterbach\rf{ChossLaut00} focus on discrete groups, only
    advanced algebraic geometry books provide the algorithms
    and I think also Gatermann\rf{gatermannHab}. I think in
    the abstract we should mention comparison to Hilbert
    basis approach but do not treat it as something that can
    be used at present for symmetry reduction in high
    dimensions.
    {\bf PC}: I agree - but why did you put it first in the
    thesis and in paper? I put it 2nd on continuous.tex. Too late
    now, I'll try to incorporate what you say here.}
the 
equivariant dynamics is rewritten in terms of invariant
coordinates. In the method of moving frames, the state space
is `sliced' locally in such a way that each group orbit of
symmetry-equivalent points is represented by a single point.
In either approach, numerical computations can be performed in
the original, full\JFG{suggest dropping ``, full''}  state-space representation, and the
solutions \JFGedit{are then} projected onto the symmetry-reduced state
space. The two methods are illustrated by
reduction of the complex Lorenz system, a 5-dimensional
dissipative flow with rotational symmetry.
