% siminos/CLE/conclusions.tex
% $Author: predrag $ $Date: 2010-01-26 11:13:04 -0500 (Tue, 26 Jan 2010) $

    \ES{I did not try to focus in Hilbert basis approach as
    it is well described elsewhere and I think it is not
    applicable to very high-dimensional flows due to
    computational restrictions in determination of the basis.
    It is included for comparison purposes and completeness.
    We also don't offer any details on how to compute a basis
    systematically as it would take up too much space to
    describe it properly: even in Gilmore and
    Letelier\rf{GL-Gil07b} details are not clear for the
    compact group case, Chossat and
    Lauterbach\rf{ChossLaut00} focus on discrete groups, only
    advanced algebraic geometry books provide the algorithms
    and I think also Gatermann\rf{gatermannHab}. I think in
    the abstract we should mention comparison to Hilbert
    basis approach but do not treat it as something that can
    be used at present for symmetry reduction in high
    dimensions.
}

\PublicPrivate{}{

Two approaches to continuous symmetry reduction suited to
reducing higher- and even very
high-dimensional dissipative flows to local return maps are
presented. In the Hilbert polynomial basis approach
the
equivariant dynamics is rewritten in terms of invariant
coordinates. In the method of moving frames the state space
is `sliced' locally in such a way that each group orbit of
symmetry-equivalent points is represented by a single point.
In either approach numerical computations can be performed in
the original, full state-space representation, and then the
solutions can be projected onto the symmetry reduced state
space. The two methods are illustrated by
reduction of the complex Lorenz system, a 5-dimensional
dissipative flow with rotational symmetry.

    \ES{I did not try to focus in Hilbert basis approach as
    it is well described elsewhere and I think it is not
    applicable to very high-dimensional flows due to
    computational restrictions in determination of the basis.
    It is included for comparison purposes and completeness.
    We also don't offer any details on how to compute a basis
    systematically as it would take up too much space to
    describe it properly: even in Gilmore and
    Letelier\rf{GL-Gil07b} details are not clear for the
    compact group case, Chossat and
    Lauterbach\rf{ChossLaut00} focus on discrete groups, only
    advanced algebraic geometry books provide the algorithms
    and I think also Gatermann\rf{gatermannHab}. I think in
    the abstract we should mention comparison to Hilbert
    basis approach but do not treat it as something that can
    be used at present for symmetry reduction in high
    dimensions.
    {\bf PC}: I agree - but why did you put it first in the
    thesis and in paper? I put it 2nd on continuous.tex. Too late
    now, I'll try to incorporate what you say here.}


In \refsect{s:Hilbert} we briefly review one of the standard tools
by which the spatial symmetry reduction can be achieved,
projection to a Hilbert basis.
In \refsect{sec:mf} we review the {\mframes}, a direct and
efficient method to compute symmetry invariant bases
    } %end \PublicPrivate{}{


We have discussed \mframes\ and its continuous time
formulation, \mslices, as candidates for symmetry reduction
in high dimensional flows. In contrast to Hilbert basis
approach that is restricted to low dimensional systems, both
formulations are very efficient in high-dimensions as they
can be performed as a linear operation, either applying a
finite group transformation on trajectories, or the
infinitesimal group generator on the velocity field. In
contrast to using co-moving frames local to each traveling
solution, dynamics mapped or restricted on a slice render all
associated traveling solutions stationary in the same set of
coordinates. Unfortunately, both formulations induce
singularities in subsets of \reducedsp. The \sset\ depends on
the \slice-fixing condition and the group action but not on
the formulation (\mslices\ or \mframes) followed. In general,
the nonlinear flow is allowed to explore the \sset, a fact
especially detrimental to the continuous time formulation as
it can render integration on the slice problematic.
    \PC{incorporate into summary:
These invariant polynomial
bases can be algorithmically determined
for both Hamiltonian and dissipative systems.
\\
Unfortunately, the computational cost of such algorithms is
at present prohibitive for state space dimensions larger than
ten, but we study fully resolved simulations of PDEs, with
state space dimensions of the order of few tenths to few
hundreds (for \KS\ flow), and well into the tenths of
thousands (for pipe and \pCf s).
    }


We have nevertheless shown, in the example of $5$-dimensional
\cLe, that the location of the \sset\ can be manipulated by
judicious choice of slice fixing point so that its effect on
the \reducedsp\ flow is ameliorated. Moreover, even when
parts of the attractor are stretched by large, moving frame
induced jumps, we can still extract geometrical information
on the dynamics by constructing a return map through a
\Poincare\ section that does not intersect the singular set,
either preceding (\refsect{s:cleCoordSlice}) or succeeding
(\refsect{s:laserMFnum}) application of the moving frame
transformations. Therefore, in higher-dimensional flows, with
more involved symmetry group action and larger sets of
stationary solutions, where a single slice and \Poincare\
section would not suffice, we can still expect to cover
\reducedsp\ with multiple slices, obtaining a set of discrete
maps involving multiple \Poincare\ sections. As shown in
\refref{SiminosThesis} the coexistence of four equilibria,
two \reqva\ and a nested \fixedsp\ structure in an
effectively $8$-dimensional \KS\ system\rf{SCD07} complicates
matters considerably and will be the subject of a forthcoming
publication\rf{SCD09b}.
