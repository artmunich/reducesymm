\PublicPrivate{}{
Implementing symmetry reduction in any of the above ways, the
reward is the same: The dynamics are reduced to a return map
to the Poincar\'e section, which due to the very strong
contraction is approximately $1$-dimensional. The dynamics on
the Poincar\'e section are parametrized by the Euclidean
distance of points along the unstable manifold, as we did for
the Lorenz example. The return map is unimodal and allows for
systematic determination of all cycles of a given length.
Here we were able to determine all cycles up to length $7$,
using the return map of \reffig{fig:CLEinvRM} to generate
guesses, and the multiple-shooting algorithm\ES{reference.}
to refine them to machine accuracy.
}% end Private

We have presented symmetry reduction of \CLe\ attractor through the use of
(a) a Hilbert basis,  (b) invariants generated by the moving frame method,
(c) a combination of use of a Poincar\'e section and a slice and (d) through
the infinitesimal application of a moving frame map, termed the method of
slices. We observe that with the exception of the Hilbert basis method all
approaches are local in nature and essentially interelated. The Hilbert basis
method unfortunately does not scale well with state-space and group dimension
and therefore cannot be used in high-dimensional problems.  

In both the finite and infinitesimal time formulation of a moving frame map 
we might understand the singularities through the analogy between a slice
and a Poincar\'e section. In symmetry reduction singularities 
are introduced at points in space that are fixed under the group action. 
Fixed points of time evolution are atypical and do not lie on a Poincar\'e section 
unless the section has been specifically chosen so that it contains them.
Therefore one cannot expect, for a flow with many fixed points, to
get away with using just a single Poincar\'e section. The same is
true for fixed points of group actions and their generalization, \fixedsp.
One cannot expect to use a single \slice\ to cover different \fixedsp s
of continuous (sub)groups.

\CLe\ example is deceptively simple: a single non-trivial \fixedsp\ in combination
with a single equilibrium and a relative equilibrium. Therefore a single Poincar\'e
section and a single slice suffice to reduce the flow to a return map with no
real penalty paid for the exclusion of the, rather boring dynamically, $z$-axis.
This was true even in the suboptimal case of a moving frame generated through a
coordinate \csection\ that was singular on an even greater subspace.
As shown in \refref{SiminosThesis} the
coexistense of four equilibria, two relative equilibria and a nice nested \fixedsp\
structure in an effectivelly $8$-dimensional \KS\ system\rf{SCD07}
\ES{Not sure about this, after all we have only tried the suboptimal case of a
coordinate slice but not a slice transverse to the group action on a relative 
equilibrium:
suffices to create a situation in which more slices are needed. 
}
complicates matters considerably and will be the subject of 
a forthcomming publication\rf{SCD09b}.
