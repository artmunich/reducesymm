% siminos/CLE/conclusions.tex
% $Author: predrag $ $Date: 2010-01-26 11:13:04 -0500 (Tue, 26 Jan 2010) $

\PublicPrivate{
          }{ % switch to private
    \PC{a rational person might actually use the already
    written up summary, so here it is, use it or lose it.}
\noindent
\underline{The message}: If a dynamical systems has a symmetry, use it!
Here we have described how, and offered two approaches to continuous symmetry
reduction. In the {\em \mslices} one fixes a `slice'
$(\sspRed - \slicep )^T \groupTan' =0$, a hyperplane normal
to the group tangent \sliceTan{} that
cuts across group orbits in the neighborhood of the
slice-fixing point $\slicep$. Each  class of
symmetry-equivalent points is represented by a single point,
with the symmetry-reduced dynamics in the \reducedsp\
$\pS/\Group$ given by \refeq{EqMotMFrame}:
    \index{group!orbit, slice}
\[
\velRed = \vel - \dot{\gSpace}  \cdot \groupTan
    \,,\qquad
\dot{\gSpace} = (\vel \cdot \groupTan')/(\groupTan \cdot \groupTan')
\,.
% from \label{EqMotMFrame} \label{MFdtheta}\\
\]
In practice one runs the dynamics in the full \statesp,
and post-processes the trajectory by the \mframes.
In the {\em Hilbert polynomial basis} approach one transforms
the equivariant \statesp\ coordinates into invariant ones, by
a nonlinear coordinate transformation
\[
\{x_1,x_2,\cdots,x_d\} \to \{u_1,u_2,\cdots,u_m\}
\,,
\]
and study the invariant `image' of dynamics
\refeq{HilbChainRl} rewritten in terms of invariant
coordinates.

In practice, continuous symmetry reduction is considerably
more involved than the discrete symmetry reduction to a
fundamental domain. Slices are only
local sections of group orbits, and Hilbert polynomials are
non-unique and difficult to compute for high-dimensional
flows. However, there is no need to actually recast the
dynamics in the new coordinates: either approach can be used
as a visualization tool, with all computations carried out in
the original coordinates, and then, when done, projecting the
solutions onto the symmetry \reducedsp\ by post-processing
the data. The trick is to construct a good set of symmetry
invariant Poincar\'e sections,
and that is always a dark art, with or without a symmetry.

%Predrag from discrete.tex                   2feb2009
We conclude with a few general observations: Higher
dim\-ens\-ion\-al dynamics requires study of compact
invariant sets of higher dimension than 0-dim\-ens\-ion\-al
\eqva\ and 1-dim\-ens\-ion\-al \po s studied so far. We
expect, for example, partially hyperbolic invariant tori to
play important role. In this chapter we have focused on the
simplest example of such compact invariant sets, where
invariant tori are a robust consequence of a global
continuous symmetry of the dynamics. The direct product
structure of a global symmetry that commutes with the flow
enables us to reduce the dynamics to a desymmetrized
$(d\!-\!1\!-\!N\!)$-dim\-ens\-ion\-al {\reducedsp}
$\pS/\Group$.

\Reqva\ and \rpo s are the hallmark of systems with
continuous symmetry. Amusingly, in this extension of
`periodic orbit' theory from unstable 1-dim\-ens\-ion\-al
closed periodic orbits to unstable
$(N\!+\!1)$-dim\-ens\-ion\-al compact manifolds $\pS_p$
invariant under continuous symmetries, there are either no or
proportionally few periodic orbits. In presence of a
continuous symmetry, likelihood of finding a {\po} is {\em
zero}. \Rpo s are almost never eventually periodic, \ie, they
almost never lie on periodic trajectories in the full
{\statesp}, so looking for periodic orbits in systems with
continuous symmetries is a fool's errand.

However, dynamical systems are often equivariant under a
combination of continuous symmetries and discrete coordinate
transformations, for example the
orthogonal group $\On{n}$. In presence of discrete
symmetries \rpo s within discrete symmetry-invariant
subspaces are eventually periodic. Atypical as they are (no
generic chaotic orbit can ever enter these discrete invariant
subspaces) they will be important for periodic orbit theory, as
there the shortest orbits dominate, and they tend to be the
most symmetric solutions.
    } %end \PublicPrivate{
%%%%%%%%%%%%%%%%%%%%%


We have presented two approaches to continuous symmetry
reduction of higher-dimensional flows, and illustrated them
by reduction of the complex Lorenz system, a 5-dimensional
dissipative flow with rotational symmetry.
In either approach numerical computations can be performed in
the original, full state-space representation, and then the
solutions can be projected onto the symmetry reduced state
space.

In the Hilbert invariant polynomial basis approach the
equivariant dynamics is rewritten in terms of invariant
coordinates. These invariant polynomial bases can be
algorithmically determined for both Hamiltonian and
dissipative systems. Our goal is symmetry-reduce fully
resolved simulations of PDEs, with state space dimensions of
the order of few tenths to few hundreds (for \KS\ flow), and
well into the tenths of thousands (for pipe and \pCf s).
Unfortunately, the computational cost of polynomial bases
algorithms is at present prohibitive for state space
dimensions larger than ten, so the invariant polynomial basis
approach is not a feasible option. We included it here solely
for comparison purposes.

In the {\mframes} (or its continuous time, differential
version, the \mslices) the \statesp\ is `sliced' locally in
such a way that each group orbit of symmetry-equivalent
points is represented by a single point. The {\mframes} turns
out to be a direct and efficient method to reduce the flow to
a symmetry-invariant \reducedsp, suited to reduction of
higher- and even very high-dimensional dissipative flows to
local return maps.

In contrast to Hilbert basis approach that is restricted to
low dimensional systems, the {\mframes} is efficient for
high-dimensional symmetry reductions, as they can be
implemented as a linear group operations, either by applying
a finite group transformation on trajectories, or the
infinitesimal group generator on the velocity field. In
contrast to co-moving frames local to each traveling
solution, dynamics restricted to a slice renders all \reqva\
stationary in the same set of coordinates. An inconvenience
inherent in the linear slices formulation is that they are
local, and the reduced flow encounters singularities in
subsets of \reducedsp, with the reduced trajectory exhibiting
large, slice induced jumps.

The \sset\ depends on the \slice-fixing condition. We shown,
on the example of $5$-dimensional \cLe, that the location of
the \sset\ can be manipulated by judicious choice of slice
fixing point, and geometrical information on the dynamics can
be extracted by constructing a return map through a
\Poincare\ section that does not intersect the singular set.
In higher-dimensional flows, with more involved symmetry
group action and larger sets of stationary solutions, where a
single slice and \Poincare\ section would not suffice, we can
still expect to cover \reducedsp\ with multiple slices,
obtaining a set of discrete maps involving multiple
\Poincare\ sections. As to a high-dimensional applications:
it was shown in \refref{SiminosThesis} that the coexistence
of four equilibria, two \reqva\ and a nested \fixedsp\
structure in an effectively $8$-dimensional \KS\
system\rf{SCD07} complicates matters considerably. This
application of symmetry reduction of a PDE is the
subject of a forthcoming publication\rf{SCD09b}.
