% introducing CLe
% from siminos/thesis/lasersSym.tex

\CLe\ were introduced by Gibbon and McGuinness\rf{GibMcCLE82} as a low-dimensional model
of baroclinic instability in the atmosphere.
As the name suggests they turned out to be a complex generalization
of Lorenz equations \refeq{Lorenz}:
\beq
\index{Complex Lorenz equations}
\begin{split}
 \dot{x} &=-\sigma x+ \sigma y \cont
 \dot{y} &=(\rLor-z)x-a y \cont
 \dot{z} &= \frac{1}{2}\left(x y^*+x^*y\right)-b z\,,
 \label{eq:CLe}
\end{split}
\eeq
where now $x,y$ are complex variables, $z$ is real, while the
parameters $\sigma,\,b$ are real and $\rLor=\RerCLor+i
\ImrCLor$, $a=1-i e$ are complex.
In all numerical examples
that follow, the parameters will be set to the Lorenz values
$\RerCLor=28,\, b=8/3,\, \sigma=10,\, a=1$, unless explicitly
stated otherwise.

Ning and Haken\rf{NingHakenCLE90} have shown
that equations isomorphic to \CLe\ also appear as a
truncation of Maxwell-Bloch equations describing a single
mode, detuned, ring laser.
%with $x,y$ and $z$
%proportional to electric field, polarization and population inversion, respectively.
They set $e+\ImrCLor=0$ so that a detuned
\eqv\ exists.
    \ES{This assumption is questionable unless it
is forced by the physics of the problem, which I cannot
follow very well. It leads to non-generic bifurcation
behavior, while one would like a model of a physical system
to be robust under perturbations (of the model). Furthermore,
the fact that the Hopf cycle in the general case is an
$\SOn{2}$-orbit has gone unnoticed. The \reqv\ can be
interpreted as an \eqv\ in a rotating frame and the measured
electric field of the laser would be the same in both cases.
    }
Bakasov and Abraham\rf{BakasovAbraham93} criticize this
choice as being ``degenerate'' and show that one can use
\CLe\ with $\ImrCLor=0$ and $e \neq 0$ to describe detuned lasers.
As we explain in \refsect{sec:Eqv0}, the choice of Ning and
Haken leads to non-generic bifurcations.
