%from ChaosBook.org \Chapter{flows}{27aug2008}{Go with the flow}
% $Author$ $Date$

% \section{Literature survey}



\paragraph{Examples of systems with discrete symmetries:}
One has
a $\Ztwo$ symmetry in the Lorenz system (\refrem{rem:Lorenz}), % \rf{lorenz,GO},
the Ising model,
and in the 3\dmn\ anisotropic Kepler
potential\rf{gut82,TW,CC92},
a $D_3=\Zn{3v}$ symmetry in H\'enon-Heiles type potentials\rf{HH,JS,rich,laur},
a $D_4=\Zn{4v}$ symmetry in quartic oscillators\rf{EHP,MWR},
in the pure $x^2 y^2$ potential\rf{Mat,CP} and
in hydrogen in a magnetic field\rf{EW1},
and a $D_2=\Zn{2v} = V_4 =\Ztwo\times \Ztwo$ symmetry
in the stadium billiard\rf{robbDisc}.
A very nice application of desymmetrization is
carried out in \refref{BVdisc}.
\index{stadium billiard}\index{billiard!stadium}
\index{Ising model}

    \PublicPrivate{
    }{ % switch \PublicPrivate{
\paragraph{Nomenclature:}
Cite Huygens, Poincar\'e for \reqva.
Remember to mention the Lorenz kind of `image' discussed by
Gilmore and Letellier\rf{GL-Gil07b}: if $z$ belongs to the symmetry
invariant $\mbox{Fix}(G)$ subspace, one can replace dynamics in the
full space by $\dot{z}$, $\ddot{z}$, $\cdots$. That is
$G$-invariant by construction.
\PCedit{Replace `desymmetrized {\statesp}' by
    `orbit space?' `image space?'}
Cite literature that uses
`desymmetrized space,'
`image space,'
`orbit space,'
`quotient space.'


% \paragraph{Literature:}
For a summary of discrete groups see \refref{stevenj}.
Chapter 3 of Rebecca Hoyle\rf{hoyll06} is a very
student-friendly overview of all of the group theory
a nonlinear dynamicist might need, with exception of
quotienting, reduction of dynamics to fundamental domains
which is not discussed at all. Curiously, we have not read any of the
group theory books that she recommends as background reading,
which just confirms that there are too many group theory books in all.
For example, one that you will not find useful at all is
\refref{GroupThe}. The reason is presumably that in physics (where
much of the modern group theory was first formulated)
the focus is on the linear representations used in
quantum mechanics, crystallography and quantum field theory.
We shall need these techniques in
\refChap{c-symm} where we reduce the
linear action of evolution operators to irreducible subspaces.
However, here we are looking at nonlinear dynamics, and the
emphasis is on the symmetries of orbits, their isotropy-group
reductions, and the isotypic decomposition of their linear
stability matrices.

Chapter 4 of Rebecca Hoyle\rf{hoyll06} is a
student-friendly introduction
to the treatment of bifurcations in presence of symmetries
worked out in full detail and generality in monographs by
Golubitsky, Stewart and Schaeffer\rf{golubII},
Golubitsky and Stewart\rf{golubitsky2002sp} and
Chossat and Lauterbach\rf{ChossLaut00}.

Amazon.com reviewers rave about \refref{ITO96};
looks nice, get it.
Check out
 \refrefs{ChePiWa02,Jaco05,Kett95,Hall67}.
\PC{Idea: a conserved quantity $\to$ continuous symmetry.
    Can one use Noether theorem?
\ESedit{ES: I think that we unfortunately cannot use it for dissipative
  systems. Noether's theorem applies provided
  we can formulate the system as a variational problem. It has caused
  me a great deal of confusion tending to think of symmetries in
  this noetherian way even for dissipative systems.
  Some recent progress(?) beyond Noether's theorem is summarized in \href{http://arxiv.org/abs/math-ph/0511035}{\url{math-ph/0511035}}.
  }
    }

% A book(but elementary, not useful):
% http://www.sst.ph.ic.ac.uk/people/d.vvedensky/courses.html

I like a lot Ian Melbourne's work on
 \href{http://personal.maths.surrey.ac.uk/st/I.Melbourne/symmetric_attractors.html}{symmetric attractors}.
I find the distinction between instantaneous symmetry and symmetry
on average very meaningful and potentially useful for us (it will appear
in the thesis soon.)
    } % end \PublicPrivate{

\paragraph{A brief history of relativity:}\label{cont:rpoCond}
    %
\PC{this is a repeat of \refsect{s:symIntro}, make into one text}
    %
The relative equilibria and relative periodic solutions
are related to
equilibria and periodic solutions respectively
of the Hamiltonian reduced by the symmetries.
They appear in many physical situations,
such as motion of rigid bodies, gravitational
$N$-body problems, molecules and nonlinear waves.
Chenciner\rf{AlbChe98,Chenc03} says
the first attempt to find (relative) periodic solutions of the
$N$-body problem by
was the 1896 short note by Poincar\'e\rf{Poinc1896},
in the context of the 3-body problem.
\Reqva\ of the $N$-body problem
(known in this context as the Lagrange points, stationary in
the co-rotating frame) are circular motions in the inertial frame,
and {\rpo s} correspond to quasiperiodic motions in the inertial frame.
In the planar $N$-body problem the equations are unchanged by
rotations. \Reqva\ can exist in a rotating frame,
and are called central configurations.
For \rpo s in celestial mechanics see also \refref{Broucke75}.
However, according to
Cushman, Bates\rf{CushBat97} and Yoder\rf{Yode88},
C. Huygens\rf{Huyg1673} understood the \reqva\ of a
spherical pendulum many
years before publishing them in 1673.

Lan has some relative equilibria (traveling waves) for KS in his
thesis\rf{Lan:Thesis}, %http://chaosbook.org/projects/theses.html
 and for complex LG in a paper on ``MAWs.''
Viswanath\rf{Visw07b} % arXiv.org/physics/0604062
found them in the plane Couette problem.
Siminos and Davidchack have for box size $L=22$ some equilibria.
Striking application of \rpo s has been the discovery
of ``choreographies" of $N$-body problems%
\rf{CheMon00,CGMS02,McCordMontaldi}.
