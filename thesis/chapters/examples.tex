\renewcommand{\inputfile}{\version\ - edited 2007-03-11 examples}
% Chapter Introduction, section Example dynamical systems used throughout the thesis.
% $Author$ $Date$
% Label: s:exampleIntro

In this section we briefly introduce some dynamical systems that will be used as simple examples
to demonstrate various concepts in later chapters.

\paragraph{Lorenz, \refchap{chap:Lorenz}:}
%
\ES{Perhaps say Lorenz flow, complex Lorenz flow etc.}
Lorenz introduced his celebrated equations\rf{lorenz} as a
severe truncation of the Navier-Stokes equations describing
Rayleigh-Benard flow. They read
\index{Lorenz equations}
\begin{align}
\dot{x} &= \sigma (y-x) \notag \\
\dot{y} &= \rho x - y - x z \label{eq:lorenz}\\
\dot{z} &= x y - b z \notag
\end{align}
where $\sigma$ the Prandtl number, $\rho$ the Rayleigh number
and $b$ an aspect ratio of the problem.

    \PublicPrivate{}{
\paragraph{Armbruster-Guckenheimer-Holmes:}
%
Armbruster, Guckenheimer and Holmes, \refref{AGHO288},
introduced a simple system with $\On{2}$ symmetry:
\index{Armbruster-Guckenheimer-Holmes flow}
\beq
\begin{split}
  \dot{z}_1 &=\bar{z}_1 z_2
              + z_1\left(\mu_1+ e_{11}|z_1|^2+e_{12}|z_2|^2\right) \\
  \dot{z}_2 &=\pm z_1^2
              + z_2\left(\mu_2+ e_{21}|z_1|^2+e_{22}|z_2|^2\right)
  \label{eq:AGH}
\end{split}
\eeq
where $z_1,z_2\in \mathbf{C}$ and $\mu_j$ and $e_{jk}$ real
parameters. Details on the motivation of those equation will be
given in a later chapter.
%This system corresponds to the first few terms in the center
%manifold reduction of a $O(2)$-symmetric partial differential
%equation near a codimension two bifurcation.
    } %end \PublicPrivate

\PublicPrivate{}{
\paragraph{Complex Lorenz, \refchap{chap:lasers}:}
%
Gibbon and McGuinness\rf{GibMcCLE82} introduced {\CLf}
\beq
\begin{split}
	\dot{x}_1 &= -\sigma x_1 + \sigma y_1\cont
	\dot{x}_2 &= -\sigma x_2 + \sigma y_2\cont
	\dot{y}_1 &= (r_1-z) x_1 - r_2 x_2 -y_1-e y_2 \cont
	\dot{y}_2 &= r_2 x_1 + (r_1-z) x_2 + e y_1- y_2\cont
	\dot{z} &= -b z + x_1 y_1 + x_2 y_2
	\label{eq:introCLeR}
\end{split}
\eeq
as a low-dimensional model of baroclinic instability in the
atmosphere. {\CLf} is equivariant under the action of
$\SOn{2}$.
}%end \PublicPrivate

% \paragraph{\KS, \refchap{chap:KSe}:}
% The \KS\ system
% \beq
%   u_t = F(u) = -{\textstyle\frac{1}{2}}(u^2)_x-u_{xx}-u_{xxxx}
%     \,,\qquad   x \in [-L/2,L/2]
% \ee{intro:ks}
% was introduced by
% Kuramoto and Tsuzuki\rf{ku} as a phase equation for
% reaction-diffusion systems described by Complex Ginzburg-Landau
% equation and independently by Sivashinsky\rf{siv} to describe
% instabilities in laminar flame fronts.
