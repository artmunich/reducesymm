\renewcommand{\inputfile}{\version\ - edited 2007-03-11 examples}
% Chapter Introduction, section Example dynamical systems used throughout the thesis.
% $Author$ $Date$
% Label: s:exampleIntro

In this section we briefly introduce some dynamical systems that will be used as simple examples
to demonstrate various concepts in later chapters.

\subsection{\Le}

\ES{Perharps say Lorenz flow, complex Lorenz flow etc.} Lorenz, \refref{lorenz1}, introduced his celebrated equations as a severe truncation of the Navier-Stokes equations describing Rayleigh-Benard flow. They read
\index{Lorenz equations}
\begin{align}
\dot{x} &= \sigma (y-x) \notag \\
\dot{y} &= r x - y - x z \label{eq:lorenz}\\
\dot{z} &= x y - b z \notag
\end{align}
where $\sigma$ the Prandtl number, $r$ the Rayleigh number and $b$ an aspect ratio of the problem.

\subsection{Complex Lorenz equations}
\label{sect:CLe}

The \CLe\ were introduced by Gibbon and McGuinness\rf{GibMcCLE82} as a low-dimensional model
of baroclinic instability in the atmosphere.
As the name suggest the equations turned out to be a complex generalization
of Lorenz equations:
\beq
\index{Complex Lorenz equations}
\begin{split}
 \dot{x} &=-\sigma x+ \sigma y \,,\\
 \dot{y} &=(r-z)x-a y \,,\\
 \dot{z} &= \frac{1}{2}\left(x \bar{y}+\bar{x}y\right)-b z\,,
 \label{eq:CLe}
\end{split}
\eeq
where now $X,Y$ are complex variables, $Z$ is real, while the parameters $\sigma,\,b$ are real and $r=r_1+i r_2$, $a=1-i e$ are
complex. In the special case $r_2= -e$, complex Lorenz equations also appear as a truncation of Maxwell-Bloch equations describing
detuned lasers\rf{NingHakenCLE90}
   \ES{I have to figure out which laser physicist derived which equation. Until then this
       paper is the one that makes the first explicit connection between CLE and laser equations.}.

\subsection{\AGHe}

Armbruster, Guckenheimer and Holmes, \refref{AGHO288}, introduced a simple system with $\On{2}$ symmetry:
\index{Armbruster-Guckenheimer-Holmes flow}
\beq
\begin{split}
  \dot{z}_1 &=\bar{z}_1 z_2 + z_1\left(\mu_1+ e_{11}|z_1|^2+e_{12}|z_2|^2\right) \\
  \dot{z}_2 &=\pm z_1^2 + z_2\left(\mu_2+ e_{21}|z_1|^2+e_{22}|z_2|^2\right)\,,
  \label{eq:AGH}
\end{split}
\eeq
where $z_1,z_2\in \mathbf{C}$ and $\mu_j$ and $e_{jk}$ real parameters. Details on the motivation
of those equation will be given in a later chapter. %This system corresponds to the
%first few terms in the center manifold reduction of a $O(2)$-symmetric partial differential
%equation near a codimension two bifurcation.
