% Discussion of applications for the work, speculation on the future
% $Author$ - $Date$ - $Id: loma.tex 846 2008-08-13 19:40:49Z predrag $

As shown in Chapters \ref{c:n00bs} and \ref{c:wthec}, the
\statesp\ of \KS\ is rife with dynamical invariants.
Here we speculate about what this says about the
structure of \KS\ \statesp.  In the course of this
discussion, several new avenues of research are suggested.

\section{Turbulence as a walk about exact coherent states}
The turbulent attractor is filled with a myriad of \eqva\ and \reqva.
    \PC{not true. Number of \eqva\ and \reqva\ is finite, and for
        boxes and Reynolds number\ considered here, small}
These states are roughly split into two groups, the lower drag, `lower branch'
states and their higher drag counterparts, the `upper branch' states.
The lower branch states seem to cluster around ?LB, and provide
a barrier to laminarization.

The backbone of this barrier is formed by the
stable manifold of ?LB.  Being codimension 1, it partitions the
neighborhood of ?LB\ into two parts: the laminar side and
the turbulent side.  The two sides are identified by the time it takes
each half the ?LB's unstable manifold to approach the laminar state.
One side takes a rather direct route, monotonically approaching ?LM.
The one side stands in stark contrast to this.  At low values of Reynolds number,
it never laminarizes -- being heteroclinically connected to ?UB.
    \PC{you mean that ?UB\ is an attractor? I doubt it, it
        has unstable directions at the bifurcation, I believe.
        So one relaminarizes, does not get stuck close to the
        bifurcation. (not sure about this)}
As
Reynolds number\ is increased, the system undergoes a global bifurcation,
breaking the heteroclinic connection.  For higher Reynolds number, this orbit spends longer
and longer time exploring the turbulent region of \statesp\ -- the inertial manifold
envisioned by Hopf~\cite{Hopf48}.

As we widen our pinhole view of \statesp\ to include more \eqva, the picture
becomes richer.  There are several \eqva\ which are connected to ?LB.  This
allows us to see how the stable manifold of ?LB\ extends further out into \statesp.  The heteroclinic connections
show exactly where the stable manifold of ?LB\ intersects the unstable manifold
of these \eqva.  For example, ?NB\ has a single complex instability.  This plane is split
by a heteroclinic connection\ to ?LB.
Orbits near the heteroclinic connection\ approach ?LB, then laminarize or become
more turbulent depending on which side of the stable manifold
of ?LB\ that they approach.
\refFig{f:NB_turb_laminar_edge} depicts this splitting.


\begin{figure}
\begin{center}
% \includegraphics[width=0.45\textwidth]{NB_turb_laminar_edge_275}~~
% \includegraphics[width=0.45\textwidth]{NB_turb_laminar_edge_340}
\end{center}
\caption[A visualization of the laminar basin boundary around ?NB.]
{Several trajectories were seeded at a distance of $0.0003$ around ?NB\ in
the plane corresponding to its leading complex instability.  In this
plot the polar angle corresponds to the initial phase, and the radius
is the distance to the laminar state, after 275 time units (left) and after
340 time units (right).
The red point corresponds to the ?NB\ to ?LB\ connection, and the
blue circle corresponds to the distance of ?LB\ from laminar.}
 \label{f:NB_turb_laminar_edge}
\end{figure}

Based on this figure, roughly half of the orbits which visit the neighborhood
of ?NB\ laminarize immediately, while the other half are kicked back towards
turbulence.  The heteroclinic connection\ here straddles one of these tipping points.
The fulcrum of the other is not so clear.  Orbits in that neighborhood appear
to make a relatively close pass to $?au_x ?LB$, but attempts to nail down
an heteroclinic connection\ have been unsuccessful so far.

It is possible that a relatively stable periodic orbit straddles this splitting.
After some time, we see from figure \ref{NBLB_split_series}
that these orbits are nearly periodic,
making repeated visits to the neighborhood of $\tau_x ?LB$ and $\tau_{xz} ?LB$.

\begin{figure}
\centering
% \includegraphics[width=0.75\textwidth]{NBLB_split_series}
\caption[A time series of indicating distance to ?LB.]
{The distance to $\tau_x ?LB$ (blue) and $\tau_{xz} ?LB$ (red), as a function
of time, on an orbit initiated near the laminar-turbulence split
in the unstable manifold of ?NB.}
\label{NBLB_split_series}
\end{figure}

This view suggests that \statesp\ could be partitioned by the stable
and unstable manifolds of the \eqva.  Evolution in time could be replaced by
a walk on a Markov graph, where each node corresponds to an \eqv.  The heteroclinic connections
indicate how to allocate edges on this graph.
How such Markov graph might
look like \ifboyscout to an engineer \fi
is illustrated by \reffig{f:eqvaMarkov}.

%
 \begin{figure}
\centering
% \includegraphics[width=0.45\textwidth]{EqvaMarkov}
 \caption[A Markov graph for \KS.]
    {
 A sketch of what a coarse partitioning Markov graph
 based on unstable-stable manifolds connections might
 look like for \KS.}
\label{f:eqvaMarkov}
\end{figure}

This sketch is meant to imply that turbulence in \KS\ is transient.
\PC{I strongly disagree. Even for small boxes it is probably
    not transient on a fractal set of aspect ratio/Reynolds number
    parameter values, and for large aspect ratios it is probably
    not transient almost anywhere, in any practical sense.}
This view is supported by both experimental
and numerical studies\rf{hof2006flt}.  ?LB\ has been continued to
$Reynolds number = 10000$\rf{WGW07}, and likely could be continued to
even higher values Reynolds number.  Even in this regime, ?LB, still
has a $1d$ unstable manifold, and thus could still possibly be considered
to act as laminar-turbulent fulcrum.

The heteroclinic connections computed here highlight a path
to linearization.  They sit at on the edge of a `hole' in the unstable
manifolds of the source \eqva.  If a turbulent trajectory falls inside
this hole it laminarizes.  Supporting evidence of the transience of turbulence
in \KS\ could be found by tracking these heteroclinic connections to higher Reynolds number, to see
if they are persistent as well as to see if the size of this hole decreases.
If this view is correct, the decrease in the size of this hole
should correlate with the rate of laminarization.
%
% If ?LB\ sets the lower bound of the turbulent attractor, then ?UB\ sets the
% upper bound.  In figure ?,

These new \eqva\ and heteroclinic connections are useful for describing the skeleton of \statesp, but
the major piece missing from this picture is the periodic orbit musculature
to go around it.  Initial guesses for periodic orbits are hard to
come by.  But, construction of a Markov partition of \statesp\ would
give us the joints of \statesp, around which we can thread periodic orbits.
Once we obtain a number of these, the trace formula \cite{DasBuch}
can be used to compute long time averages of the flow, or in the case
that it is a repeller, it gives us a means to compute its escape rate --
the rate of laminarization.

\section{Refinement of Heteroclinic Connections}
One way of improving a Markov partition of \statesp\ would be to find
more heteroclinic connections.  The method used in this thesis to compute heteroclinic connections
is rather naive, making their computations unnecessarily expensive.  There
are several ways in which this situation can be improved.  One major inefficiency of
the algorithm presented here is that it does not make use of linearization
around the target \eqv.  In the algorithm presented
here, this is done for the source \eqv, but it is more complicated to
do for the target.  A heteroclinic connection\ is an infinite time object, so computing
its entire extent as it approaches an unstable \eqv\ is impossible.  However, if we make use
of the linear stability of the target \eqv, we can eliminate the calculation
of all but a finite segment of the heteroclinic connection.

The idea is that rather then running a time-consuming integration through
the neighborhood of the target \eqv, we can predict how close an approximate heteroclinic connection\
will get just by its decomposition into a linear combination of the eigenvectors of the \eqv.
In the case of ?LB, all we would really need to know is the
the approximate connection's projection onto ?LB's $1d$ unstable manifold.
Then, rather then seeking the distance of closest approach, we only need to find out how big
of a cross section it has in the unstable directions, when it enters
the neighborhood of ?LB.

To do this, we need to make use of a projection onto the leading eigenvectors
of the {\stabmat} $\Mvar$, for the target \eqv.  The eigenvectors are generally
not mutually orthogonal.  This makes the expression of a particular point
in phase space in terms of an eigenvector basis somewhat more difficult.
Using more eigenvectors in the expansion will change the magnitude
of the projection onto the other more unstable eigenvectors.
However, our hopes are somewhat buoyed by the fact that all but a few
of the eigenvectors will have no contribution, since they are so stable.

Let ${\bf x}$ be the location of a point in \statesp\ which we want
to express in terms of the eigenvectors ${\bf e}_i$ of an \eqv\ located
at the origin.
We would like to find $c_i$ which minimize the residual,
\begin{equation}
 r = |{{\bf x} - \sum_{i=0}^N c_i {\bf e}_i}|\,,
\end{equation}
or, equivalently, we wish to approximately solve for ${\bf c}$ in
\begin{equation}
 {\bf M} {\bf c} = {\bf x}\,,
\label{e:eigproj}
\end{equation}
where the columns of ${\bf M}$ are formed by the eigenvectors ${\bf e}_i$.

One approach is to compute the Moore-Penrose pseudoinverse,
$\bf M^+$.  Pseudoinverse is a matrix which satisfies $\bf M^+ M = \matId$,
and solves \refeq{e:eigproj} in a least-squares sense.
If compute the singular value decomposition of ${\bf M}$, $\bf M = U \Sigma V^*$, then $\bf M^+ = V \Sigma^+ U^*$,
where the elements of $\bf \Sigma^+$ are the reciprocal elements of the nonzero entries of $\bf \Sigma$
(or zero otherwise).

Hopefully, by implementing this idea, the accuracy of the heteroclinic connections already computed
here can be improved and new heteroclinic connections can be found.  If Schmiegel's thesis \cite{Schmi99} is
any indication, there's plenty where those came from.
% JH: The projection operator approach seems to have some convergence problems...
% The projection operator \cite{birdtracks}
% which projects onto the sub-eigenspace corresponding
% to eigenvalue $\lambda_i$ is given by:
% \begin{equation}
%  {\bf P}_i = \prod_{j \neq i} \frac{{\bf A} - \lambda_j \matId}{\lambda_i - \lambda_j}\,.
% \end{equation}
% Let us assume that the target \eqv\ has a $1d$ unstable manifold (as is the case of ?LB), with
% two marginal directions from translational symmetry.  Label the unstable direction as $\lambda_0$,
% so the projection operator onto that direction becomes:
% \begin{equation}
%  {\bf P}_0 = \frac{A^2}{\lambda_0^2} \prod_{j = 3}^{\infty} \frac{{\bf A} - \lambda_j \matId}{\lambda_0 - \lambda_j}\,.
% \end{equation}
%
% This can be rewritten as a power series in $\bf A$:
% \begin{equation}
% {\bf P}_0 = \frac{A^2}{\lambda_0^2} \left(\matId \prod_{j=3}^{\infty} \frac{-\lambda_j}{\lambda_i - \lambda_j} + \dots\right)
% \end{equation}
%
% Since ${\bf A}$ is infinite we need to truncate this to $N$ terms, corresponding
% to the $N$ most unstable directions.  The higher order terms can be safely ignored since
% they should have a vanishingly small projection due to their high stability.
% In order to compute the projection ${\bf P_i x}$, we need to be able to compute
% ${\bf A}^n {\bf x}$.
% We don't have access to this directly, but we do know the forward
% time map of the system in the neighborhood of the target \eqv\ is
% given by
% \begin{equation}
% \bf{x}(t) = e^({\bf A} t) {\bf x}\,.
% \end{equation}
% Expanding in a Taylor series gives:
% \begin{equation}
% \bf{x}(t) = 1 + t \bf{A x} + 1/2 t^2 \bf{A}^2 \bf{x} + \dots
% \end{equation}
% So, if we know the value of this series for N different values of t (N timesteps
% in the future from x), we get the much more manageable linear problem of order N, which
% is to find $a_i$ so that$Pj x = Sum a_i x_i$ where $x_i$ is the value of x, i timesteps in the future.
%
% Once we know that, we can find the size of a given orbits projection onto the unstable
% manifold of the target equilibrium, which we can seek to minimize with any number
% of routines.  I expect that this will probably shave a nice amount of time from
% the computations.  But I don't think it will be miraculous, since you need to
% get pretty close the target to use this anyway.

\section{For the Experimentalist}


Results of this thesis suggest a few avenues of exploration for experimentalists as well.
Stereoscopic Particle Image Velocimetry (PIV) allows the reconstruction of a slice
of a velocity field.  These experiments fall into two categories: turbulence suppression and
observation of turbulence.

Lower branch type states such as ?NNB\ and ?EQsev\ seem
particularly important in terms of turbulence suppression.  Their relative proximity to laminar
in \statesp\ (see \refsect{s:eqvMeasures})
could make them excellent targets as gateways between turbulence and laminar flow.
By inputting the relative small amount of energy needed to reach ?EQsev, one could induce
turbulence relatively cheaply.  Conversely, it also indicates the type of perturbations that would need
to be suppressed in order to prevent loss of laminar flow.

To test these ideas, it would be useful to set up an experiment in which states analogous
to ?EQsev\ would be induced.  There should be a rather sharp change in turbulent lifetimes
as the magnitude of the perturbation is increased.  Furthermore, the magnitude of the perturbation
needed should prove to be minimal as compared to other types which have been tried in the past.

The idea of turbulence as a walk among coherent states, guided by heteroclinic connections could
also be tested experimentally.  Experimental observation of the heteroclinic connections may be somewhat impractical
due to their extreme sensitivity to small changes in the system.  However, it may be possible
to examine the results of experiment in light of the Markov partition that they suggest.

In particular, one should observe close passes to ?LB-type states, with a
corresponding slowing in the rate of change of the flow field, followed by eruptions
towards ?UB-type states.  If a repertoire of \eqva\ and their heteroclinic connections were to be computed
for the geometry being studied, one could proceed to partition \statesp, and compare
the expected Markov graph, to the experimentally exhibited behavior.
