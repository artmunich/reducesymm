% Discussion of applications for the work, speculation on the future
% $Author$ - $Date$ - $Id: loma.tex 846 2008-08-13 19:40:49Z predrag $


\section{\KS\ return map}

Having identified in \refchap{chap:ksStSp} the unstable manifolds that are most likely to provide
a coarse organization the \KS\ $L=22$ attractor, the obvious next step is to identify
suitable Poincar\'e sections for their study. Contrary to the \CLe\ example of \refchap{chap:laserSym}
where a global section was found and the dynamics was described as a first return map to the section,
in the case of \KS\ equation we will need more than one sections. Each section will be used to
capture the dynamics of the unstable manifold of a (relative) equilibrium until it starts folding back 
to itself. Parameterizing each manifold by Euclidean length on the manifold a forward map from section to section will be constructed and convolution of those maps will result in a return map. This approach
meshes very well with the construction of Markov partition of the dynamics.

Of course the unstable manifolds of objects of interest for \KS\ dynamics are often high-dimensional,
\eg $4$-dimensional for \REQV{\pm}{1} and their visualization and parametrization is a non trivial task.
Nevertheless, we observe that the ratio of real parts of the leading stability eigenvalues for the case of \REQV{\pm}{1} is approximatelly $3.4$ and thus we expect that the continuation of the strongly unstable eigenspace will play the dominant role.

\section{What are the cycles good for?}

Up to this point we have concentrated in the geometry of the phase space and haven't addressed
the second constituent of the dynamicist's view of turbulence, the natural measure. The periodic
and relative periodic orbits found for \KS\ equation form a skeleton of the dynamics in a geometrical sense but also, through trace formulas and spectral determinants\rf{DasBuch}, provide
a means to accuratelly evaluate the spectra of evolution operators and evaluate the asymptotic
values of observables. 

Quoting \refref{DasBuch} ``the strategy is 1) count, 2) weigh, 3) add up.''
The weights are given by the stability of the cycles, we can use trace formulas to add them up (in
our case the continuous symmetry version in \rf{Cvi07}) but we have to start from the beginning
and complete step number one. Counting means that we are able to organize and label all cycles up
to a given length, establishing a hierarchy that will then be exploited in highly convergent trace
formulas or spectral determinants. The need to organize the periodic and relative periodic orbits 
found for \KS\ equation is underlined in \refsect{s-StOrdKS} 
by the failure of the attempt to check the flow conservation 
sum rule through stability ordering of a set of $20000$ {\po s} and {\rpo s} provided by
Ruslan Davidchack.

% To stress out the need to organize the cycles we present our attempt to use an ordering criterion
% that has been succesfuly used in calculations with periodic orbits, namely stability ordering,
% Stability ordering was introduced by Dahlqvist and Russberg\rf{DR91} in a study of chaotic dynamics for the $(x^2y^2)^{1/a}$ potential. 


