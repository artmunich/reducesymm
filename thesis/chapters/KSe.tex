%\section{\KSe}
%\label{sec:KSe}
 
The \KS\ [henceforth KS] system\rf{ku,siv} was derived by Kuramoto and Tsuzuki\rf{ku} as a phase equation for reaction-diffusion
systems described by Complex Ginzburg-Landau equation and independently by Sivashinsky\rf{siv} to describe instabilities in laminar flame fronts.
It also appears appears in a variety of contexts including thin falling films\rf{BenKS66,LinKS74}, interfacial instabilities between concurent viscous fluids, drift waves in plasmas\ES{citations needed.}. 
% For an instructive derivation from Complex Ginzburg-Landau equation \cf~\rf{}.

Our motivation for its study is that it is one of the simplest nonlinear PDEs that
exhibit spatiotemporally chaotic behavior. Moreover the dynamics in the chaotic regime are
interesting in their own right.

 In the formulation
adopted here, the time evolution of the `flame front velocity'
$u=u(x,t)$ %on a periodic domain $u(x,t) = u(x+L,t)$ 
is given by
\beq
  u_t = F(u) = -{\textstyle\frac{1}{2}}(u^2)_x-u_{xx}-u_{xxxx}
    \,,\qquad   x \in [-L/2,L/2]
    \,,
\ee{ks}
with appropriate boundary conditions, as discussed in \refsect{sec:KSbc}.
Here $t \geq 0$ is the time, and $x$ is the spatial coordinate.
The subscripts $x$ and $t$ denote partial derivatives with respect to
$x$ and $t$. 

\subsection{Boundary conditions and system size}
\label{sec:KSbc}

Ideally when one is interested in studying PDEs that exhibit spatiotemporal chaos one
would like to work in a system of infinite spatial extend, \ie\ in the limit $L\rightarrow\infty$. 
Although solutions of KS equations in this limit have appeared, \cf for example \refrefs{hooper_travelling_1988,feng_multiplicity_2004}\ES{More citations needed. Kevrekidis \etal\ 
mention criticism of periodic BC, but they give no reference.}, it is
more convenient, both computationaly and theoretically, to work with periodic boundary conditions
\beq
  u(x,t) = u(x+L,t)\,,
 \label{eq:KSper}
\eeq
and this is the usual choice in the literature and the one followed in this thesis. 
% The justification in terms of the physics
% of the problem is that we focus our attention to a small part of a larger system, 
% far away from the boundaries. Further 
Justification of this choice will be given
in \refsect{sec:KSeSymm}, in conjuction with the discussion of the symmetries of the system.

Another common choice of boundary conditions is 
\beq
  u(0,t) = u(L,t)=0\,,
 \label{eq:KSper}
\eeq
which restricts the system to the subspace of odd functions. This choice will also be discussed
in \refsect{sec:KSeSymm}.

In what follows
we shall state results of all calculations either in units of the system size $L$
or the `dimensionless system size' $\tildeL=L/2\pi$. 
All numerical results presented in this thesis
are for $\tildeL=22/2\pi = 3.5014\ldots$, unless otherwise
noted. The spefic choise leads to a system that is just ``chaotic'' enough 
to have interesting dynamics, \cf \refsect{sec:KSlit}, but is not in the spatio-temporally chaotic
regime, therefore it is still rather tractable and we can use it as a testbed for
continuous symmetry reduction and a dynamical systems approach.


\subsection{Symmetries of \KSe}
\label{sec:KSeSymm}

In an unbounded domain, $x\in(-\infty,\infty)$, KS equation is equivariant
under the action of the non-compact Euclidean group $\En{1}$:
If $u(x,t)$ is a solution, then
$\Shift({\shift})\, u(x,t) = u(x+\shift,t)$
is an equivalent solution for any shift
$\shift\in\Rls{}$, as is the reflection (`parity' or `inversion')
\beq
    \Refl \, u(x) = -u(-x)
\,.
\ee{KSparity}

Imposing periodic boundary conditions we restrict attention to the fixed
point subspace of the compact subgroup \On{2} of \En{1}:
\beq
	\Shift_{\shift/L}\, u(x,t) = u(x+\shift,t)\,,\qquad \shift\in\left[-L/2,L/2\right]	
\eeq
and reflections \refneq{KSparity}. Here we use subscript notation for shifts to differentiate
with the case of \En{1}.
Moreover, one only considers perturbations within \fixedsp{\On{2}} invariant subspace, \ie\ we do not
consider subharmonic perturbations. The system size $L$ affects the representation
of \On{2}, \cf \refsect{sec:fourKS}. \ES{This statement needs to be checked: 
An important fact is that the isotropy subgroups of
\En{1} (of \En{n} in general) are all compact subgroups, if the action is proper\cite[for example]{ChossLaut00}.
Thus the equivariant bifurcation structure as $L$ increases is not affected, at least as long as we
are not in the spatio-temporally chaotic regime, when the spectrum of stability eigenvalues becomes
quasi-continuous.}
% Translations can
% now be regarded as rotations,
% \beq
% 	\Rot \, u(x)=u(x+L)
% \eeq
Reflection generates the dihedral subgroup $D_1 = \{1, \Refl\}$
of $O(2)$. 

The KS equation is also Galilean invariant: if $u(x,t)$ is a solution,
then $u(x -ct,t) -c $, with $c$ an arbitrary constant
speed, is also a solution. As one can verify by integrating \refeq{ks} with 
respect to $x$ over the periodic domain $[-L/2,L/2]$ the quantity
 $\int_{-L/2}^{L/2} u\,dx$ 
is conserved and we can, without loss of generality, set it equal to zero. This corresponds
to the choice $c=0$, therefore eliminating Galilean invariance.
\ES{Use later on: }

% $G$, the group of actions $ g \in G $ on a
% \statesp\ (reflections, translations, \etc) is a symmetry of the KS
% flow \refeq{ks} if $g\,u_t = F(g\,u)$.

% The KS equation is time translationally invariant, and space translationally invariant
% on a periodic domain under
% the 1-parameter group of
% $O(2): \{\Shift_{\shift/L},\Refl \}$.
% If $u(x,t)$ is a solution, then
% $\Shift_{\shift/L}\, u(x,t) = u(x+\shift,t)$
% is an equivalent solution for any shift
% $-L/2 < \shift \leq L/2$,
% as is the
% reflection (`parity' or `inversion')
% \beq
%     \Refl \, u(x) = -u(-x)
% \,.
% \ee{KSparity}



\subsection{Fourier space}
\label{sec:fourKS}

\On{2} equivariance makes it convenient to work in Fourier space,
\beq
  u(x,t)=\sum_{k=-\infty}^{+\infty} a_k (t) e^{ i k x /\tildeL }
\,,
\ee{eq:ksexp}
with the $1$-dimensional PDE \refeq{ks}
replaced by an infinite set of
ODEs for the complex Fourier coefficients $a_k(t)$:
\beq
\dot{a}_k= \pVeloc_k(a)
     = ( q_k^2 - q_k^4 )\, a_k
    - i \frac{q_k}{2} \sum_{m=-\infty}^{+\infty} a_m a_{k-m}
\,,
\ee{expan}
where $q_k = k/\tildeL$.
Since $u(x,t)$ is real, $a_k=a_{-k}^\ast$, and we can replace the
sum by a $k > 0$ sum. Note that $\dot{a}_0=0$ in
 \refeq{expan} as a result of Galilean invariance and $a_0$ is a conserved quantity
 fixed to $a_0=0$ by the condition $\int_{-L/2}^{L/2} u\,dx$=0. 
In the Fourier basis \On{2} acts absolutely irreducibly on each complex plane
$\left(\Re(a_k),\Im(a_k)\right)$ and the linear part of \refeq{expan} is conveniently
diagonalized. Indeed, the translation operator action on the Fourier coefficients \refeq{eq:ksexp},
represented here by a complex valued vector
$a = \{a_k\in\mathbb{C}\,|\,k = 1, 2, \ldots\}$, is given by
\beq
  \Shift_{\shift/L}\, a = \mathbf{g}(\shift) \, a \,,
  \label{eq:shiftF}
\eeq
where $\mathbf{g}(\shift) = \mathrm{diag}( e^{i q_k\, \shift} )$ is a complex
valued diagonal matrix, which amounts to the $k$-th mode complex plane
rotation by an angle $k\, \shift /\tildeL$.  The reflection acts on
the Fourier coefficients by complex conjugation and a change of sign,
\beq
  \Refl \, a = -a^\ast
\,.
\ee{FModInvSymm}
The reader should observe that this action is the action \refneq{eq:O2stndrd} of \On{2} on \Clx{n} studied
in \refsect{sec:strata}, apart from the minus sign in \refeq{FModInvSymm}. The subgroup lattice remains unchanged but \fixedsp s of the dihedral subgroups are affected\ES{Not sure if this is true, the space is 
infinite dimensional.}. We will discuss the isotropy lattice after we
truncate the tower of equations \refneq{expan}.  \ES{Discuss after discussing truncation: The isotropy lattice remains unchanged, as well as the \fixedsp s with the exception of \Fix{\Dn{m}} which is now given by the conditions
\bea
	a_k=0\ \mathrm{unless}\ k = m j\,,\ j=1,\ldots\lfloor n/m \rfloor\,, \\
	\Re(a_k)=0\ \mathrm{for}\ k=1,\ldots,n\,.
	\label{eq:O2ksDqFix}
\eea
It follows that \Fix{\Dn{1}} is the subspace of antisymmetric functions $Re(z_k)=0,\, \forall k$ 
or $u(-x)=-u(x)$.}

We can now justify the choice of boundary conditions and system size. With periodic boundary conditions
we expect to find traveling wave and modulated amplitude traveling wave solutions. Moreover,
we have observed in numerical simulations that $L=22$ is just after the onset of chaos while still
not spatiomporaly chaotics.


\ES{This is from my first year KS special problem, perharps some parts are usuable: 
Some qualitative comments about the KS equation can now be easilly
given in view of \refeq{expan}. The linear behaviour of the system 
depends on the sign of the quantity $k^2- \left(2\pi/L\right)  k^4$. 
For a sufficiently large system, the first few values of $k$ yield $k^2- \left(2\pi/L\right)
k^4>0$ or $k<\sqrt{L/2\pi}$ and the corresponding Fourier components grow exponentially
with time (unstable components). In other words the anti-diffusion
term in \refeq{eq:KS} (resulting in the term $\sim k^2$ in
\refeq{eq:Fcoef}) dominates over the dissipation term (resulting in
$\sim k^4$ in \refeq{eq:Fcoef}).  On the other hand there are infinitely many larger
wavenumber components with $k>\sqrt{L/2\pi}$ for which the solutions
are bounded (stable components). The role of the bilinear term in
\refeq{eq:Fcoef} is then to excite the larger wavenumber components
while dissipating the smaller wavenumber ones. The result of this
competition is that the asymptotic dynamics of the system are
confined on a low dimensional attractor.}

\ES{Move elsewere: Due to the hyperviscous damping $u_{xxxx}$, long time solutions of KS
equation are smooth, $a_k$ drop off fast
with $k$, and truncations of \refeq{expan} to $16 \leq N \leq 128$
terms yield accurate solutions for system sizes considered here.  
Robustness of the long-time dynamics
of KS as a function of the number of Fourier modes kept in truncations
of \refeq{expan} is, however, a subtle issue.  Adding an extra mode to
a truncation of the system introduces a small perturbation in the
space of dynamical systems.  However, due to the lack of structural
stability both as a function of truncation $N$, and the system size
\tildeL, a small variation in a system parameter can (and often will)
throw the dynamics into a different asymptotic state.  For example,
asymptotic attractor which appears to be chaotic in a $N$-dimensional
\statesp\ truncation can collapse into an attractive cycle
for $(N\!+\!1)$-dimensions.}

\ES{Say that the linear part becomes diagonal in Fourier space.}


\subsection{Truncation}

Dynamical \statesp\ representation of a PDE is $\infty$-dimensional,
but the KS flow is strongly contracting and its non-wondering set,
and, within it, the set of invariant solutions investigated here, is
embedded into a finite-dimensional inertial manifold\rf{FNSTks85} in
a non-trivial, nonlinear way.

