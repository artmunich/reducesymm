% siminos/thesis/chapter/flowCons.tex
% $Author$ $Date$

%\section{Flow conservation sum rules}
%\label{s-Cons-m-flow}

% Predrag:              16 Oct 2007
%% extracted from \Chapter{getused}{14jun2006}{Why cycle?}

If the dynamical system is bounded, all trajectories remain confined for
all times, escape rate \refeq{esc-Rat} vanishes $\gamma=-\eigenvL_0=0$, and the
leading eigenvalue
\PC{ref to \refeq{s_nat_m2} is wrong, say someplace escape rate = leading eig
   }
of the \FPoper\ \refeq{TransOp1} is simply
$\exp(-t\gamma)=1$.
Conservation of material flow thus implies that for bound flows
cycle expansions  of \dzeta s and \Fd s satisfy
exact {\em flow conservation} sum rules:
\bea
1/\zeta(0,0) &=& 1+\sumprime_\pseudos
          { (-1)^k \over
        |\Lambda_{p_1}\cdots \Lambda_{p_k}|}
       = 0
        \continue
F(0,0) &=& 1- \sum_{n=1}^{\infty} c_{n}(0,0)
       = 0
\label{prob-cons}
\eea
obtained by setting $s=0$ in \refeq{pseudo1}, \refeq{Fred-cyc-exp1}
cycle weights
$t_p  = e^{- \eigenvL \period{p}} /|\ExpaEig_p|
\to {1 / |\ExpaEig_p|}$ .
These sum rules depend neither
on the cycle periods $\period{p}$
nor on the observable $\obser(x)$ under investigation,
but only on the cycle stabilities $\Lambda_{p,1}$, $\Lambda_{p,2}$,
$\cdots$, $\Lambda_{p,d}$,
and their significance is purely geometric: they are a measure
of how well periodic orbits tessellate the {\statesp}.
Conservation of material flow
provides the first and very useful test of the quality of
finite cycle length truncations, and is something that
you should always check first when constructing a cycle
expansion for a bounded flow.
