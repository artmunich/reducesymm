%from ChaosBook.org \Chapter{flows}{27aug2008}{Go with the flow}
% $Author$ $Date$

% \section{Literature survey}

%\paragraph{Lorenz equation}{ \label{rem:Lorenz}
% Predrag                           20jan2008
% transferred from halcrow/thesis/chapters/symm.tex
The Lorenz equation \refeq{Lorenz} is
the most celebrated early
illustration of ``deterministic chaos''\rf{lorenz63}
(but not the first -
the honor goes to Dame Cartwright\rf{disc:CarLit45}).
Lorenz's paper, which can be found in reprint
collections \refrefs{cvt89b,hao90b}, is a pleasure to read, and
is still one of the best introductions to the physics
motivating such models.
For a geophysics derivation, see Rothman course notes\rf{Rothman06}.
The equations, a set of ODEs in $\reals^3$, exhibit strange
attractors\rf{tucker1,tucker2,MIViana}.
% whose geometry reflects
%the underlying symmetry of the ODEs.
Fr{\o}yland\rf{froyland} has a nice brief discussion
of Lorenz flow.
Fr{\o}yland and Alfsen\rf{froyland_alfsen} plot
many periodic and heteroclinic orbits of the Lorenz flow;
some of the symmetric ones are included in \refref{froyland}.
Guckenheimer-Williams\rf{Guckenheimer79b} and
Afraimovich-Bykov-Shilnikov\rf{Afraimovich87} offer
in-depth discussion of the Lorenz equation.
The most detailed study of the Lorenz equation
was undertaken by Sparrow\rf{sparrow}.
For a physical interpretation of $\rho$ as ``Rayleigh number.''
see Jackson\rf{jackson89} and Seydel\rf{seydel}.
Lorenz truncation to 3 modes is
so drastic that the model bears no relation to  the physical
hydrodynamics problem that motivated it.
For a detailed pictures of Lorenz invariant manifolds consult
 Vol II of Jackson\rf{jackson89}.
Lorenz attractor is a very thin fractal -- as we saw,
stable manifold thickness is of order $10^{-4}$ -- but its
fractal structure has been accurately resolved by
D. Viswanath\rf{DV03,DV04}.
%\hfill (continued in \refrem{rem:LorenzSymm}.)
%    } %end \paragraph{Lorenz equation:}{ \label{rem:Lorenz}

\paragraph{H\'enon, Lozi maps}{
%chaos.nbi.dk:/users/predrag/nordita/tex/henon_long/henon.tex
%      PC                  10/5-94
The H\'enon map%\rf{henon}
\index{Henon@H\'enon map}
% {\em per se}
is of no particular physical
import in and of itself--its significance lies in
the fact that it is a minimal normal form for modeling flows near a
\snbif,
\index{bifurcation!saddle-node}
\index{saddle-node bifurcation}
and that it is a prototype of the stretching and folding dynamics that
leads to deterministic chaos.  It is generic in the sense that it can
exhibit arbitrarily complicated symbolic dynamics and mixtures of
hyperbolic and non--hyperbolic behaviors.  Its construction was
motivated by the best known early example of `deterministic chaos',
the Lorenz equation\rf{lorenz}, see \refref{lorenz}
and \refrem{rem:Lorenz}.

\index{Lorenz, E.N.}
\index{Pomeau, Y.}
 Y.~Pomeau's studies of the Lorenz
attractor on an analog computer, and his insights into its stretching
and folding dynamics motivated H\'enon\rf{henon} to introduce
the H\'enon map in
1976.
\index{Henon@H\'enon, M.}
H\'enon's and Lorenz's original papers can be found in reprint
collections \refrefs{u_in_c,hao}.  They are a pleasure to read, and
are still the best introduction to the physics
%background
motivating such models.
A detailed description of the dynamics of the H\'enon map is
given by Mira and coworkers\rf{mira}, as well as very many other
authors.
\index{Mira, C.}
% Mira and coworkers\rf{fou,mira}, as well as very many
% other\rf{AP,AGIP,BC85,BC89,BC91,biham_wenzel_89,biham_wenzel_90,
% losal,CGP,GK85,GKM,GK85,hansen_henon,henon,milnor,mira,simo}.

\index{Lozi map}
\index{map!Lozi}
The Lozi map\rf{lozi2} is particularly convenient in investigating the
symbolic dynamics of $2\dmn$ mappings.  Both the Lorenz and Lozi
systems are uniformly smooth systems with singularities.
The continuity of measure
for the Lozi map
was proven by M.~Misiurewicz\rf{mis1},
and the existence of the SRB measure was established by L.-S.~Young.
% \index{SRB measure}  is in indexPointers.tex
\index{natural measure}
\index{measure!natural}
\index{Misiurewicz, M.}
\index{Young, L.-S.}
} %end \paragraph{H\'enon map?}{

\paragraph{\protect Grasshoppers {\em vs.} butterflies}{%
\index{sensitivity to initial conditions}
\index{butterfly effect}
The 'sensitivity to initial conditions' was
discussed by Maxwell, 30 years later by Poincar\'e.
In weather prediction, the
Lorentz' `Butterfly Effect' started its journey in
1898, as a `Grasshopper Effect' in a book review by
W. S. Franklin\rf{Frank1898}.  In 1963 Lorenz
ascribed a `seagull effect' to an unnamed meteorologist,
and in 1972 he repackaged it as the `Butterfly Effect'.
           } % \paragraph{

% Predrag  27mar2008: returned from Halcrow blog to discrete.tex \refref{CEsym}.
% Predrag  20jan2008: moved to Halcrow blog
%
\paragraph{Symmetries of the Lorenz equation}\label{rem:LorenzSymm}
%(continued from \refrem{rem:Lorenz}.)

For Lorenz flow
dimensionalities of stable/unstable manifolds
make possible a
robust heteroclinic connection,
with unstable manifolds of an \eqv\ flowing into the
stable manifold of another \eqva.
How such connections are forced upon us is
best grasped by perusing the chapter 13 `Heteroclinic tangles'
of the inimitable
Abraham and Shaw illustrated classic\rf{abraham:shaw}.
Their beautiful hand-drawn sketches elucidate the origin
of heteroclinic connections in the Lorenz flow (and its high-dimensional
Navier-Stokes relatives) better than any computer simulation.
Miranda and Stone\rf{GL-Mir93} were first to
quotient the $\Ztwo$ symmetry and explicitly construct
the desymmetrized, `proto-Lorenz system,'
by a nonlinear coordinate transformation into the Hilbert-Weyl
polynomial basis
invariant under the action of the symmetry group%
\rf{CoLiSh96}.
For in-depth discussion of symmetry-reduced (`images')
and symmetry-extended (`covers')
topology, symbolic dynamics, periodic orbits,
invariant polynomial bases \etc, of
Lorenz, R\"ossler and many other low-dimensional systems
there is
no better reference than the
Gilmore and Letellier monograph\rf{GL-Gil07b,GL-Let01}.
%
They interpret the proto-Lorenz and its `double
cover' Lorenz as `intensities' being
the squares of `amplitudes,' and call quotiented
flows such as (Lorenz)/$\Ztwo$ `images.'
Our `doubled-polar angle' visualization
\reffig{fig:PoincLorenz}
is a proto-Lorenz in disguise, with the difference: we
integrate the flow and construct Poincar\'e sections and
return maps in the Lorenz $[x,y,z]$ coordinates, without
any nonlinear coordinate transformations.
The Poincar\'e
return map \reffig{fig:RetMapLorenz} is reminiscent
in shape both of the one given by Lorenz
in his original paper, and the one plotted in
a radial coordinate by Gilmore and Letellier.
Nevertheless, it is profoundly different:
our return maps are
from unstable manifold $\to$ itself\rf{CCP96},
and thus intrinsic and coordinate independent. This is necessary
in high-dimensional flows
to avoid problems such as double-valuedness of return map projections
on arbitrary 1\dmn\ coordinates. More importantly, as we know the embedding
of the unstable manifold into the full \statesp, a periodic point
of our return map \emph{is} - regardless of the
length of the cycle -  the periodic point in the full  \statesp,
so no additional Newton searches are needed.
\index{Lorenz, E.N.}
\index{Cartwright, M.L.}
\index{Gilmore, R.}
\index{Letellier, C.}
