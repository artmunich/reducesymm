% Siminos thesis summary
% $Author: siminos $ $Date: 2009-03-13 16:26:49 -0400 (Fri, 13 Mar 2009) $

When statistical assumptions do not hold and coherent
structures are present in spatially extended systems
such as fluid flows, flame fronts and field theories
a dynamical description of turbulent phenomena becomes
necessary.
% Recent experimental and theoretical advances\rf{science04}
% support a dynamical vision of turbulence:
% For any finite  spatial resolution, a turbulent flow follows approximately for a finite time
% a pattern belonging to a { finite alphabet} of admissible patterns.
% The long term dynamics is a {walk through the space of these unstable patterns}.
% When statistical assumptions\rf{frisch} break down and large
% scale coherent structures are present, a dynamical description
% as a walk through a space of unstable patterns becomes natural.
% The question is how to characterize and classify such patterns?
% Here we follow the seminal Hopf paper\rf{hopf48}, and  visualize
% turbulence not as  a sequence of spatial snapshots in turbulent evolution,
% but as a trajectory in an  $\infty$-$d$ \statesp\ in which an
% instant in turbulent evolution is a {unique} point. 
In the dynamical systems approach,
theory of turbulence for a given system, with given boundary conditions,
is given by (a) the geometry of the \statesp\ and (b) the associated measure,
\ie, the likelihood that asymptotic dynamics visits a given \statesp\ region.

In this thesis this vision is pursued in the context of \KS\
system, one of the simplest physically interesting spatially
extended nonlinear systems. Relaxing the restriction of
previous studies to discrete
symmetry invariant subspace a new challenge confronts our
attempt to elucidate the \statesp\ geometry. Continuous
symmetry, in the form of \On{2} acting on
\statesp, endows \statesp\ with additional structure.
Symmetry dictates the bifurcation structure and the type of
observed solutions. At the same time the notion of recurrence becomes
relative: Not only do we have to identify points along the
orbit of the nonlinear group of time evolution but also points
along the orbit of the linear symmetry group. The latter
identification, termed symmetry reduction, although
conceptually simple as the group action is linear, is hard to
implement in practice. The high dimensionality of \statesp\
that results from truncation of a PDE is one of the reasons.
The second reason is that, for most interesting group actions,
the process introduces singularities in the structure of the
reduced \statesp.

Here we propose a scheme to efficiently project dynamics of
solutions computed in the original space to a reduced \statesp,
while eliminating or avoiding singularities. The procedure simplifies
the visualization of high-dimensional flows and provides new
insight into the role the unstable manifols of equilibria and
traveling waves play in organizing the flow. This in turn elucidates the mechanism
that creates unstable modulated traveling waves (periodic orbits in reduced space)
that provide a skeleton of the dynamics. Finally the compact description
of dynamics thus achieved sets the stage for
reduction of the dynamics to mappings between a set of Poincar\'e sections.
