%  conclusion.tex
% $Author$ $Date$

This thesis contribution to the dynamical system's approach to
turbulence is setting up a framework for identifying and
organizing recurrent patterns in the presence of continuous
symmetries. This is done with the following
constraints in mind: 1) high-dimensionality of \statesp\ renders
Hilbert basis rewriting the equations of motion impractical, 2) reduction
can be local but the local pieces have to be joined together in
a way that the global geometry of the attractor is elucidated.

Reduction is performed along two axes: visualization and construction of return maps.
For visualization purposes the method of moving frames is efficient in providing
symbolic expressions for invariants up to moderate dimension. Singularities
in those invariants can be removed in an algorithmic manner and one can then map the
solutions to the invariants, rather than rewriting the equations. Visualizing
reduced \statesp\ in this way helps identify the dynamically important solutions and the role they
play in organizing the flow and lead to an informed choice of local Poincar\'e sections
that capture the dynamics of the associated unstable manifolds.

This leads us to the next step, which is reduction using the geometrical interpretation
of moving frames as a group operation that brings points back to a local cross-section of
group orbits. This is a linear operation for any given point and can be implemented efficiently
even for high-dimensional discretizations of PDEs. The crucial step is to avoid transformation
singularities by restricting attention to local, group-invariant Poincar\'e sections that
do not contain any points on which the transformations become singular.

Application of this procedure to \CLe\ leads to significant
simplification of the phase portraits but most importantly it
reduces the 5-dimensional dynamics to an 1-dimensional return
map that enables us to systematically label and determine all
relative periodic orbits up to a given period. Application to
\KS\ equation reveals the importance of \reqva\ in
organizing the flow and helps piece together the different
invariant objects that ``triangulate'' the geometry of the
attractor. Moreover, the ground is set up for future work to
proceed beyond the visualization stage to the construction of
return (or forward) maps that will provide a concrete
description of the dynamics and will allow quantitative
calculations by means of cycle expansions.
    \PC{
perhaps mention as future generalizations: invariant tori -
``larger'' symmetries?
    }

Returning to the description of the dynamicist's view of turbulence we note that if we are to
interpret turbulence as a walk through unstable recurrent patterns then this thesis provides
a framework to recognize a pattern irrespective of its location, in much the same way as our brain
does\ES{Mention applications of moving coframes to vision recognition here?}.
