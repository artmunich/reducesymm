The research undertaken here is a part and parcel of the
common effort of a large community of scientists, engineers and
mathematicians, working toward the grand goal of developing an effective
dynamical theory of turbulent phenomena in spatially extended systems
such as fluid flows, flame fronts and field theories.
The literature is vast and bewildering. Let us start by a
quote from an imaginary (but typical) paper: ``We have
demonstrated our method on the example of \KSe\ but it is completely
general and applicable to any other PDE. Application to
Navier-Stokes is left as an exercise to the reader.''

While in practice the application of the method to more
complicated problems turns out to be far from trivial, the
exercise leads one to critically re-examine several steps
along the way that the above quoted imaginary paper (profound
as it might be) only glossed over. A rich, perhaps beautiful,
and --surprisingly--  not yet fully explored vein turns out
to be the role continuous symmetries play in detailed
dynamical explorations of `turbulent' flows. They are the red
thread through the journey undertaken in this thesis.


\subsection{\KS\ equation}

%    \PC{this is the INTRODUCTION. You are supposed to explain
%    to your mother or any other reader why \KS, and not Shikamoto-Kurashinski}
The \KS\ system in one spatial dimension
\beq
  u_t = F(u) = -{\textstyle\frac{1}{2}}(u^2)_x-u_{xx}-u_{xxxx}
    \,,\qquad   x \in [-L/2,L/2]
\ee{intro:ks}
has been derived in a variety of contexts (see
\refchap{chap:KSe}) including the dynamics of
reaction-diffusion systems and fluttering flame fronts, and
has been studied extensively as one of the simplest systems
that shares common features with the Navier-Stokes
description of incompressible fluids. Like Navier-Stokes
equations, it contains a term that pumps energy into the
system ($u_{xx}$) and a term that dissipates it ($u_{xxxx}$).
It is transitionally invariant under periodic boundary
condition $u(x+L,t)=u(x,t)$ or in an unbounded domain
$x\in(-\infty,\infty)$, a feature also present in many
physical fluid flows. In
\refrefs{Christiansen97,lanCvit07,LanThesis} dynamics was
restricted to the space of antisymmetric functions
$u(-x,t)=-u(x,t)$ by imposing boundary condition
$u(-L/2,t)=u(L/2,t)=0$, \cf\ \refsect{sec:KSe} for more
details, thus eliminating translational symmetry. Even though
working in the antisymmetric subspace is mathematically and
computationally convenient and the dynamics are far from
trivial, many of the physically important phenomena, such as
traveling waves are eliminated by this restriction. Traveling
solutions, however, are present in most of the fluid
simulations and experiments mentioned in \refsect{s:hopf} and
ubiquitous in physics.

As soon as we relax the antisymmetric boundary conditions and
choose to work with periodic boundaries, \KS\ equation
becomes invariant under the 1-$d$ Lie group of $O(2)$
translations: if $u(x,t)$ is a solution, then $u(x+d,t)$ is
an equivalent solution for any $-L/2 < d \leq L/2$. As a
result, KS can have \emph{traveling wave} or \emph{\reqv}
solutions,
\beq
 u(x,t)= u_o(x-c t)
\eeq
where $c$ the constant velocity of the wave and $u_o(x)$ its
profile. Furthermore, it can have \emph{modulated traveling
wave} or \emph{\rpo} solutions,
\beq
u(x+d,\period{}+t) = u(x,t)
\,,
\ee{KSrpos}
where $\period{}$ the period and $d$ a translation.
Thus recurrence becomes
relative: The periodic orbits that organized \statesp\ in
\refrefs{Christiansen97,lanCvit07,LanThesis} are now not the
only generic compact solutions, we are also faced with relative
periodic orbits, solutions that repeat up to a
translation.
They were already noted by Poincar\'e in his study of the
3-body problem\rf{ChencinerLink,rtb}. In PDEs they are also
known as modulated traveling waves and they have been found and
studied, for example in \KS\ equation\rf{BrKevr96}, Complex
Ginzburg-Landau equation\rf{lop05rel}, and plane Couette
flow\rf{Visw07b}. A recent application of \rpo s has
been the discovery of ``choreographies'' of $N$-body problems%
\rf{CheMon00,CGMS02,McCordMontaldi}.

The main purpose of this thesis will be to investigate the
role played by \reqva\ (traveling waves) and relative
periodic orbits in the geometry of spatially extended systems
with continuous symmetry. Following on the earlier work, we
concentrate on \KS\ equation as it provides a simpler system
for the illustration of key ideas than the technically much
more demanding Navier-Stokes equations. Yet, in the spirit of the
above quoted imaginary paper, we emphasize that we develop
methods in a way that can be applicable in other PDEs. The
choice of \KS\ system size is such that the dynamics in the
physical space looks nothing like ``fully developed
turbulence.'' Instead, the dynamics is dominated by
\emph{coherent structures}, that is localized,
persistent structures. This is precisely the kind of
dynamics in which statistical assumptions\rf{frisch} fail and
which requires a dynamical systems description in the spirit of
Hopf.
In \refchap{chap:KSe} we review \KS\ equation in detail and in
\refchap{chap:kseStSp} we study its \statesp, through \eqva,
traveling waves, heteroclinic connections and \rpo s, for a
specific ``box'' size.
In \refappe{chap:POT} --in a way of motivating the enterprize
undertaken in the main body of the thesis-- we describe our
attempt to organize a set of $30,000$ {\po s} and {\rpo s}
computed by Davidchack\rf{Davidchack_priv} in the
context of periodic orbit theory, without an understanding how
these orbits are organized geometrically. Failure of this
attempt leaves no option but to understand the geometry of
\statesp\ first, label the orbits, find missing ones, and then
use cycle expansions. The first step in  achieving this would
be a ``compactification'' of the \statesp\ by quotienting out
the continuous symmetry, or ``symmetry reduction.''


\subsection{Symmetry reduction}

% \PCedit
Taking it into account  a symmetry of a physical system
usually leads to simplification of the problem.
Indeed, in linear theories, such as quantum mechanics, symmetry is often exploited
through separation of variables. In Hamiltonian mechanics it leads to conserved quantities which can often be
directly exploited. For instance in the central force problem conservation of angular momentum fixes the plane
of motion.
In general exploiting symmetry by identifying points in space or solutions
related by a symmetry operation is the objective of
\emph{symmetry reduction}. The subject has a long history in Hamiltonian mechanics and for general systems
and group actions it usually is rather technical and highly non-trivial, see for example
\refrefs{marsden_introduction_1999,marsden_hamiltonian_2007,cushman_global_1997}.

We motivate the need for symmetry reduction by a preview of the results in the case of $5$-dimensional,
\SOn{2}-symmetric Complex Lorenz equations that will be used as our illustrative example in \refchap{chap:lasers}.
 A quick comparison of
\reffig{fig:CLE} and the continuous symmetry reduced \reffig{fig:CLEinv} counterpart should demonstrate the problem.
Under continuous symmetry the ``stretch and fold'' mechanism that determines the topology of the attractor
is hidden by the, dynamically irrelevant, motion in the group direction. More importantly the dynamics can
be described by the one dimensional first return map of \reffig{fig:CLEinvRM}. Eliminating time-translational
invariance by means of a Poincar\'e section is familiar to the reader as a way of obtaining a discrete time map from
a continuous time flow. In the presence of continuous symmetry one also needs to eliminate the less interesting
linear group invariance by some means before one can obtain the return map of \reffig{fig:CLEinvRM}.

Here we will concentrate on dissipative dynamical systems and high-dimensional truncations
of PDEs. The main problems we are facing are: 1) the high dimensionality of \statesp, 2) the structure of
the \statesp\ induced by the symmetry group action (see \refsect{sec:strata})
which usually prevents carrying out reduction globally.
High dimensionality does not allow us to use a very powerful
tool in symmetry reduction, \emph{Hilbert bases}.
The idea is to form from the equivariant variables
(that commute
with the group action, see \refsect{sec:symIntro})
polynomials invariant under the group action, and rewrite the
dynamics in terms of these (see \refsect{sec:symRed}).
Unfortunately, the determination of a Hilbert basis appears
computationally prohibitive for phase-space dimensions
larger than ten\rf{ChossLaut00,gatermannHab}. Moreover, even
if such a basis were available, rewriting the equation of
motion in a basis of invariant polynomials appears
impractical for high-dimensional flows.
Here we shall circumvent these difficulties by solving the
equations in the equivariant variables, but plotting the
solutions in terms of the invariant variables.

A different approach to symmetry reduction of PDEs has been
presented by Rowley and
Marsden\rf{rowley_reconstruction_2000}, formulated in the
context of {Karhunen-Lo\`{e}ve} expansion, but also applicable
to direct numerical simulations\rf{rowley_reduction_2003}.
The method allows one to integrate a PDE defined in the
\emph{reduced space} along with a \emph{reconstruction
equation} that allows the dynamics in the original space to
be recovered. As noted in
\refref{rowley_reconstruction_2000} the reconstruction
equation can fail for reasons related to the structure of
\statesp\ under the group action,
and one would be forced to cover the reduced space with local
coordinate charts. As the latter procedure is not
straightforward to implement and we do not want to sacrifice
the ability to move back and forth between the initial and
the reduced \statesp\, while obtaining, to some extend, a global
picture of the reduced dynamics, we will need to take a different path.

Our approach to symmetry reduction is centered around the
method of moving frames of Cartan\rf{CartanMF} that we
present in the formulation of Fels and Olver\rf{FelsOlver98}
in \refsect{sec:mf}. It allows the determination of
(non-polynomial) invariants of the group action by a simple
and efficient algorithm that works well in \statesp\
dimension of order $100$. Invariants generated for \CLe\ (see
\refsect{sec:CLeMF}) and for \KS\ equation (see
\refsect{sec:KSeMF}) are singular in subsets of \statesp,
again due to the special structure of the \statesp\ under the
group action. We modify these invariants so that there is no
singularity in regions of dynamical interest and visualize
dynamics of invariant objects in reduced \statesp\ by mapping
solutions computed in the original space to this basis.


Visualization in reduced space provides insight in its
geometry and facilitates the choice of local Poincar\'e
sections as the first step in our attempt to describe the
flow by iteration of setion-to-section maps.
Choosing Poincar\'e sections on which the invariants
determined by the moving frame method are not singular allows
for a crucial simplification of the reduction process: it
allows reduction through a linear transformation at
any point on the Poincar\'e section, see \refsect{sec:CLeMF}.
This simplification allows one to perform the reduction
procedure for points of intersection of solutions with the
Poincar\'e section very efficiently, even for very high
dimensional spaces.

For \KS\ equation construction of discrete time maps in
reduced \statesp\ is still a subject of ongoing work, see
\refchap{chap:tobedone}. Yet the reward of applying this
procedure for visualization is that, when continuous symmetry
is quotiented out, relative periodic orbits become periodic,
traveling waves become \eqva, the dimensionality of their
unstable manifolds is reduced by the dimension of the group
and our understanding of the role solutions play is
dramatically enhanced.
For example, after symmetry reduction of
\KS\ flow we are able to visualize the unstable manifold of a
traveling wave in a compact manner and connect it to the
mechanisms associated to recurrence within the
``chaotic attractor'', see \refchap{chap:kseRedStSp}.

\subsection{Updated version of this thesis}

The Georgia Tech Ph.D. thesis is an educational project that
has come to an end, but it is also an ongoing research
project. Errors and omissions are certainly present in
the official Georgia Tech submission. The author will make
every effort to correct them, with the updated,
pdf-hyperlinked, printer friendly version available in {\tt
ChaosBook.org}, under the link to theses. We recommend
that the reader download this updated version.
