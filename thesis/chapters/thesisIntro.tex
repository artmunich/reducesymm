The research undertaken here is a part and parcel of the
common effort of a large community of scientists, engineers and
mathematicians, working toward the grand goal of developing an effective \ES{\ESedit{dropped: and
applicable}} dynamical theory of turbulent phenomena in spatially extended systems
such as fluid flows, flame fronts and field theories.
The literature is vast and bewildering. Let us start by a
quote from an imaginary (but typical) paper: ``We have
demonstrated our method on the example of \KSe\ but it is completely
general and applicable to any other PDE. Application to
Navier-Stokes is left as an exercise to the reader.''

While that particular `reader' might not have been born yet,
the exercise leads one to critically re-examine several steps along the way that
the above quoted paper (profound but somewhat sketchy when it comes to details) only glossed over\ES{
Am I offending other people by saying something like this?}.
A rich, perhaps beautiful, and --surprisingly--  not yet fully explored
vein turns out to be the role continuous symmetries play in detailed dynamical explorations
of `turbulent' flows. They are the red thread through the journey undertaken in this thesis.

\subsection{\KS\ equation}

    \PC{this is the INTRODUCTION. You are supposed to explain
    to your mother or any other reader why \KS, and not Shikamoto-Kurashinski}
The studies of the \KS\ system
\beq
  u_t = F(u) = -{\textstyle\frac{1}{2}}(u^2)_x-u_{xx}-u_{xxxx}
    \,,\qquad   x \in [-L/2,L/2]
\ee{intro:ks}
in \refrefs{Christiansen97,lanCvit07,LanThesis} restricted the dynamics
to the space of antisymmetric functions $u(-x,t)=-u(x,t)$ by imposing
boundary condition $u(-L/2,t)=u(L/2,t)=0$, \cf\ \refsect{sec:KSe} for more
detail. This restriction eliminated the translational symmetry \KS\ equation
has when supplied with more general periodic boundary condition $u(x+L,t)=u(x,t)$ or
in an unbounded domain $x\in(-\infty,\infty)$. Even though working in the antisymmetric
subspace is mathematically and computationally convenient and the dynamics are far
from trivial, many of the physically important phenomena,
such as traveling waves are eliminated by this restriction.
We do not need to try hard to motivate the
study of traveling solutions: they are present in all fluid simulations and experiments
mentioned in \refsect{s:hopf} and ubiquitous in physics.

%PC removed: "Going on with the discussion in the context of \KS\ system we observe that"
%
As soon as
we relax the antisymmetric boundary conditions and choose to work with periodic boundaries,
\KS\ equation becomes invariant under the 1-$d$ Lie group of $O(2)$ rotations: if
$u(x,t)$ is a solution, then $u(x+d,t)$ is an equivalent
solution for any $-L/2 < d \leq L/2$.
As a result,
KS can have \rpo\ solutions with nonzero shift
\beq
u(x+d,\period{}+t) = u(x,t)
\,.
\ee{KSrpos}
where $\period{}$ is the period and recurrence becomes
relative: The periodic orbits that organized \statesp\ in
\refrefs{Christiansen97,lanCvit07,LanThesis} are now not the
only generic solutions, we are also faced with {\em relative
periodic orbits}, solutions that repeat up to a
translation.
They were already noted by Poincar\'e in his study of the
3-body problem\rf{ChencinerLink,rtb}. In PDEs they are also
known as modulated traveling waves and they have been found and
studied, for example in \KS\ equation\rf{BrKevr96}, Complex
Ginzburg-Landau equation\rf{lop05rel}, and plane Couette
flow\rf{Visw07b}. A recent application of \rpo s has
been the discovery of ``choreographies'' of $N$-body problems%
\rf{CheMon00,CGMS02,McCordMontaldi}.

The main purpose of this thesis will be to investigate the role played by \reqva\ (traveling waves)
and relative periodic orbits in the geometry of spatially extended systems with continuous symmetry. Following
on the earlier work, we concentrate on \KS\ equation as it provides a simpler system for the illustration
of key ideas than the technically much more demanding Navier-Stokes equations. Yet,
following the above cited imaginary paper, we emphasize that we develop methods in
a way that can be applicable in other PDEs.
%\ESedit
{The choice of \KS\ system size is such that the
dynamics in the physical space looks nothing like ``fully
developed turbulence.'' Instead, the dynamics is dominated by coherent
structures, precisely the kind of dynamics in which
statistical assumptions\rf{frisch} fail and require a
dynamical systems description in the spirit of Hopf. In the
absence of a better term we call our \KS\ dynamics
``turbulent'' without any claim that what we deal with is
fluid turbulence but with the purpose of emphasizing the
relation to other flows with coherent structures.}
    \PC{if you really believe that --let's say- B. Hof's pipe is an example of ``chaos,'' more
    power to you. But please do include our referee's precise definitions of
    chaos, spatio-temporal chaos and turbulence in the thesis.
	{\bf ES}: No, I don't consider Hof's pipe to be ``chaos'',
		but do we have a better, widely accepted word
		that describes this Kuramoto-Sivashinsky system? I anyway
		find the definition of ``spatio-temporal chaos'' suggested
		by the referee vague. To me it looks more that it is an issue
		of broken symmetry in \KS\ rather than system size,
		at least that's what I get when I read Manneville's book.
		What I was really trying to
		say is that we call it turbulence because it is not
		what people refer to as chaos. I've dropped both chaotic
		and spatiotemporally chaotic.
       }
In \refchap{chap:KSe} we review \KS\ equation in detail and in
\refchap{chap:kseStSp} we study its \statesp, through \eqva,
traveling waves, heteroclinic connections and \rpo s, for a
specific ``box'' size.
In \refappe{chap:POT} --in a way of motivating the enterprize
undertaken in the main body of the thesis-- we describe our
attempt to organize a set of $20,000$ {\po s} and {\rpo s}
computed by Davidchack\rf{Davidchack_priv} in the
context of periodic orbit theory, without an understanding how
these orbits are organized geometrically. Failure of this
attempt leaves no option but to understand the geometry of
\statesp\ first, label the orbits, find missing ones, and then
use cycle expansions. The first step in  achieving this would
be a ``compactification'' of the \statesp\ by quotienting out
the continuous symmetry, or ``symmetry reduction.''


\subsection{Symmetry reduction}

In physics symmetry usually leads to simplification of a problem so it will sound as a contradiction that it actually
complicates matters in our case.
\PC{recheck usage of `oxymoron?' ES: not sure about the correct usage in English, replaced with contradiction.}
Indeed, in linear theories, such as quantum mechanics, symmetry is often exploited
through separation of variables. In Hamiltonian mechanics it leads to conserved quantities which can often be
directly exploited. For instance in the central force problem conservation of angular momentum fixes the plane
of motion. %In field theories Hamiltonians are constructed so that they respect a certain symmetry.
In general exploiting symmetry by identifying points in space or solutions
related by a symmetry operation is the objective of
symmetry reduction. The subject has a long history in Hamiltonian mechanics and for general systems
and group actions it usually is rather technical and highly non-trivial, see for example
\refrefs{marsden_introduction_1999,marsden_hamiltonian_2007,cushman_global_1997}.

Here we will concentrate on dissipative dynamical systems and high-dimensional truncations
of PDEs. The main problems we are facing are: 1) the high dimensionality of \statesp, 2) the structure of
the \statesp\ induced by the symmetry group action (see \refsect{sec:strata})
usually prevents carrying out reduction globally.
High dimensionality does not allow us to use Hilbert bases (see \refsect{sec:symRed}),
a very powerful tool in symmetry reduction,
since their determination seems computationally prohibitive for anything larger
than a ten-dimensional space\rf{ChossLaut00,gatermannHab}. Moreover, even if such a basis were available,
rewriting the equation of motion in a basis of invariant polynomials appears impractical for high-dimensional
flows. The issue of absence of  a global transformation demonstrates itself in different
ways, see \refsect{sec:symRed}, and we will have to overcome it since we are interested in understanding
how the attractor is organized globally.

We motivate the need for symmetry reduction by a preview of the results in the case of $5$-dimensional,
\SOn{2}-symmetric Complex Lorenz equations that will be used as our illustrative example in \refchap{chap:lasers}.
 A quick comparison of
\reffig{fig:CLE} and the continuous symmetry reduced \reffig{fig:CLEinv} counterpart should demonstrate the problem.
Under continuous symmetry the ``stretch and fold'' mechanism that determines the topology of the attractor
is hidden by the, dynamically irrelevant, motion in the group direction. More importantly the dynamics can
be described by the one dimensional first return map of \reffig{fig:CLEinvRM}. Eliminating time-translational
invariance by means of a Poincar\'e section is familiar to the reader as a way of obtaining a discrete time map from
a continuous time flow. In the presence of continuous symmetry one also needs to eliminate the less interesting
linear group invariance by some means before one can obtain the return map of \reffig{fig:CLEinvRM}.

Our approach to symmetry reduction is centered around the method of moving frames of Cartan\rf{CartanMF}
that we present in the formulation of Fels and Olver\rf{FelsOlver98} in \refsect{sec:mf}. For visualization
purposes it allows the determination of invariants of the group action by a simple
and efficient algorithm that works up to moderate dimension.
Invariants generated for \CLe\ (see \refsect{sec:CLeMF}) for \KS\ equation
(see \refsect{sec:CLeMF}) are singular in subsets of \statesp, a manifestation of the non-global
nature of reduction. We modify those invariants so that there is
no singularity in regions of interest and we use them for visualizing our solutions and obtain insight
for the geometry of \statesp.

Visualization in reduced space facilitates the choice of local Poincar\'e sections as the first step in
our attempt to describe the flow by discrete maps. Choosing Poincar\'e sections on which the invariants
determined by the moving frame method are not singular allows for a crucial simplification of the reduction
process: it allows to apply reduction through a linear transformation at any point on the Poincar\'e section,
see \refsect{sec:mf}\ES{This refers to section on moving frames in group theory intro, but I'll
have to write a small section or chapter dedicated to application of moving frames to dynamics.} and
\refsect{sec:CLeMF}.

For \KS\ equation construction of discrete time maps in reduced \statesp\ is still a subject of ongoing
work, see \refchap{chap:tobedone}. Yet the reward of applying this procedure even for visualization
is that, when continuous symmetry is quotiented out, relative periodic
orbits become periodic, traveling waves become \eqva, the dimensionality of their unstable manifolds
is reduced by the dimension of the group and our understanding of the role solutions play is dramatically
enhanced.
