The studies of the \KS\ system
\beq
  u_t = F(u) = -{\textstyle\frac{1}{2}}(u^2)_x-u_{xx}-u_{xxxx}
    \,,\qquad   x \in [-L/2,L/2]
\ee{intro:ks}
in \refrefs{Christiansen97,lanCvit07,LanThesis} restricted the dynamics 
to the space of antisymmetric functions $u(-x,t)=-u(x,t)$ by imposing
boundary condition $u(-L/2,t)=u(L/2,t)=0$, \cf refchap{cha:KSe} for more
details. This restriction eliminated the translational symmetry \KS\ equation
has when supplied with more general periodic boundary condition $u(x+L,t)=u(x,t)$ or
in an unbounded domain $x\in(-\infty,\infty)$. Eventhough working in the antisymmetric
subspace is mathematically and computationally convenient and the dynamics are far
from trivial for sufficiently large system size, many of the interesting solutions,
such as traveling waves are eliminated. We do not need to try hard to motivate the 
study of traveling solutions, they are present in all fluid simulations and experiments
mentioned in \refsec{s:hopf} and ubiquous in physics.

Going on with the discussion in the context of \KS\ system we observe that as soon as 
we relax the antisymmetric boundary conditions and choose to work with periodic boundaries
\KS\ equation becomes invariant under the 1-$d$ Lie group of $O(2)$ rotations: if
$u(x,t)$ is a solution, then $u(x+d,t)$ is an equivalent
solution for any $-L/2 < d \leq L/2$.
As a result,
KS can have \rpo\ solutions with nonzero shift
\beq
u(x+d,\period{}) = u(x,0)
\,.
\ee{KSrpos}
where $\period{}$ is the period and recurrence becomes relative: The periodic orbits that organized phase space in \refrefs{Christiansen97,lanCvit07,LanThesis} are now not the only generic solutions, we are
also faced with relative periodic orbits, solutions that repeat them self up to a translation. 
were introduced by Poincar\'e in his study of the 3-body problem\rf{ChencinerLink,rtb}.
In PDE's they are also known as modulated traveling waves and they have been found and studied, for example
in \KS\ equation\rf{BrKevr96}, Complex Ginzburg-Landau equation\rf{lop05rel}, plane Couette flow\rf{Visw07b}.
A striking recent application of \rpo s has been the discovery
of ``choreographies" of $N$-body problems\rf{McC7,McC8,McC}.

