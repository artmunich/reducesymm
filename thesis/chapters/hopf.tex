% hopf.tex, poetry from nsf06am
% $Author$ $Date$

% Predrag                   oct 10 2006
% Predrag                   sep 13 2005

% \section{Dynamicist's vision of turbulence}
% \label{s:hopf}
% \file{rescued from nsf06am/TEX/hopf.tex Oct 10 2006}

%%%%%%%%%%%%% bring these from y-lan/bibtex or?
% "cycprl"
% "Ezr91"
% "crete03"
% "LanDescent"
%
%%%%%%%%%%%%% bring these form gibson/bibtex
% `hopf48' on page 858 undefined on input line 26.
% `FNSTks85' on page 858 undefined on input line 45.
% `Aubry88' on page 859 undefined on input line 68.
% `ku' on page 859 undefined on input line 71.
% `ZG96' on page 859 undefined on input line 74.
% `Lan:Thesis' on page 860 undefined on input line 172.



% We will focus on:
% (1)
% {\em elucidation of the {\statesp} flow topology and the symbolic
% dynamics for intermediate system dimensions.}

This thesis is part of a wider effort\rf{chfield} to describe
turbulence from a dynamical systems perspective that goes back
to the seminal paper of Hopf\rf{hopf48}. The relation of
dynamics to turbulence underlies many fundamental developments
in dynamical systems theory, from the very (re)discovery of
chaos by Lorenz\rf{lorenz} to the Ruelle-Takens\rf{ruell71}
view of turbulence, to the work on inertial
manifolds\rf{constantin_integral_1989} of partial differential
equations (PDEs). The emphasis here is not on the transition to
turbulence or on the derivation of reduced models of a partial
differential equation. On the contrary we ask: For a given
system, with given boundary conditions, which we are able to
numerically simulate to sufficient accuracy
to resolve its finest features, how do we develop a dynamical
description? Hopf's answer\rf{hopf48}
is to consider the dynamics of a PDE not as the evolution of
snapshots of the underlying field but as dynamics on an
$\infty$-dimensional \statesp\ in which every point
corresponds to a state of the system. In this space a generic turbulent
trajectory visits neighborhood of a ``regular'' solution for a while, then
switches to another one, and so on. For any given system,
parameter values and boundary conditions there are two
ingredients to implementing this vision: (a) the geometry of
the \statesp\ and (b) the associated natural measure, \ie, the
likelihood that asymptotic dynamics visits a given \statesp\
region.

To explain what we mean by geometry of \statesp\ of a dynamical
system, let us consider a finite dimensional system of coupled
ordinary differential equations of the form
\beq
	\frac{d x}{d t} = v(x)\,,
\label{eq:diffEqIntro}
\eeq
where $x\,,\, v \in \Rls{N}$. The trajectory $f^{t}(x_o)$ of an initial
condition $x_o$ is obtained by integrating \refeq{eq:diffEqIntro}.

The simplest solution that might exist in such a system is an
\emph{equilibrium} point that is left invariant by the flow,
$f^t(x_o)=x_o$ for all times. When we examine the
neighborhood of the equilibrium we find that it can be
decomposed into a (local) \emph{stable subspace} along which
points converge towards the equilibrium and a (local)
\emph{unstable subspace} along which points stray away from
the equilibrium under time evolution (for the time being
\emph{center subspaces} along which neither happens will be
ignored). The global continuation of the stable (unstable)
subspace under backward (forward) time evolution is the
\emph{stable (unstable) manifold} of the equilibrium. Stable
and unstable manifolds are \emph{flow-invariant}: the
trajectory through any point on the manifold stays on it for
all times. Invariant manifolds provide topological
obstructions for any other solution: as a trajectory cannot
cross an invariant manifold, it is forced to follow its
stretching and folding. In nonlinear systems studied in this
thesis, the unstable manifolds are stretched away from
an equilibrium until nonlinearity causes them to fold sharply
back. This provides a basic mechanism for
\emph{recurrence}: trajectories of points in the \emph{non-wandering
set} return arbitrarily close to the initial point.
This set of non-wandering orbits,
which for dynamics that are locally expanding (there are directions along
which we depart away from any solution) and globally mixing (we are always forced
to fold back) we will loosely identify with the \emph{chaotic attractor},
contains the ``regular'' solutions in Hopf's vision: periodic orbits
that close after finite time,
\beq
	x(t+T)=x(t)\,,
\eeq
where $T$ the period. Periodic orbits provide a complete
characterization of the topology of a chaotic attractor in
low dimensional systems\rf{gilmore_topology_2003}, and most importantly
they can be used to quantitatively approximate the natural
measure and calculate ``observable'' quantities, such as
Lyapunov exponents and escape rates, within the framework of
periodic orbit theory\rf{DasBuch}, briefly summarized here in
\refappe{chap:POT}. The reason we refer here to the totality
of \statesp\ relations between invariant solutions as
``geometry'' rather than ``topology'' is that we are not only
interested in elucidating the topological mechanisms that
result in recurrences but also in the exact \statesp\
positions of invariant objects, such as periodic orbits, and
metric distances between different solutions.

The first successful quantitative implementation of Hopf's
vision for a spatially extended system, to the best of the
author's knowledge, can be found in
Christiansen~\etal\rf{Christiansen97}. The object of study
was \KS\ system, a dissipative PDE in one spatial dimension,
as one of the simplest systems that exhibits features
reminiscent of fluid turbulence (see \refchap{chap:KSe} for
details). A large set of periodic orbits, embedded into the
attractor and ordered hierarchically was located. Shorter orbits provided
the basic building blocks of the attractor, while longer ones
contributed quantitative corrections to periodic orbit averages.
This investigation was continued
for a ``more turbulent'' \KS\ system by Y.~Lan and
Cvitanovi\'c\rf{LanThesis,lanCvit07}.

Recently, \statesp\ of moderate Reynolds number wall bounded
shear flows became experimentally\rf{science04} and
computationally\rf{KawKida01,FE03,WK04,Visw07b,GHCW07}
accessible. The charting of Navier-Stokes \statesp, for
specific boundary conditions, with \eqva, \reqva\
and heteroclinic connections has provided the basic elements of
the geometry of the turbulent flow and there is hope that it
will eventually lead to approximation of the natural measure
using a set of ``regular'' solutions.
