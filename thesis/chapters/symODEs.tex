\renewcommand{\inputfile}{\version\ - edited 2007-03-11 symODEs}
% section: Symmetries of dynamical systems
% $Author$ $Date$

We consider a system of \ode's of the form
\beq
	\dot{x} = \vf(x,\lambda)
	\label{eq:dynsys}
\eeq
where $\vf: \Rls{n}\times\Rls{r}$ a $C^\infty$ mapping. When
not important we will suppress the $r$-dimensional vector of parameters
$\lambda$ in the notation.

As is well known any compact Lie group acting on $\Rls{n}$ can be identified
with a subgroup of $\On{n}$, \cf\ for example \refref{golubII}
for a sketch of the proof. Therefore, without loss of generality
we will concentrate on subgroups $\Gamma\subseteq\On{n}$ in the following.

\begin{definition}
We call a group element $\gamma\in\On{n}$ a symmetry of \refeq{eq:dynsys} if for every solution
$x(t)$, $\gamma x(t)$ is also a solution.
\end{definition}

It is easy to see that this definition leads to the following condition for $\gamma$ to
be a symmetry of \refeq{eq:dynsys}:
\beq
	\vf(\gamma x) =\gamma \vf(x)
	\label{eq:equiv}
\eeq
for all $x\in\Rls{n}$. We say that $\vf$ \emph{commutes} with $\gamma$ or that $\vf$ is $\gamma$-\emph{equivariant}.
When $\vf$ commutes with all $\gamma\in\Gamma$ we say that $\vf$ is $\Gamma$-equivariant.
% When $f$ is $\gamma$-equivariant
% the \ode~ \ref{eq:dynsys} remains invariant under the action of $\gamma$.

For example, the vector field in Lorenz equations  \refneq{eq:lorenz} is equivariant under the group
$\Zn{2}\cong\Dn{1}$ acting on \Rls{3} by
\[
	\Rot{\pi}(x,y,z) = (-x,-y,z)\,.
\]
Notice that this transformation can be considered as either as rotation by $\pi$ around the $z$ axis (hence the
group \Zn{2}) or as a reflection about the origin in a plane perpendicular to the $z$-axis (hence the group \Dn{1}).

As another example, the vector field in \CLe~ is equivariant under the group \SOn{2} acting on $\Rls{5}\cong \Clx{2}\times \Rls{}$
by
\[
 \Rot{\theta} (X,Y,Z) = (e^{i\theta} X, e^{i\theta} Y, Z)\,,\ \ \  \theta\in[0,2\pi)\,.
\]

Finally, the symmetry group of \AGHe~ is \On{2} acting by
\begin{eqnarray*}
  \Rot{\theta}(z_1,z_2) &=& (e^{i\theta} z_1, e^{i 2\theta} z_2)\,,\ \ \  \theta\in[0,2\pi)\,,\\
  \Refl(z_1,z_2) &=& (\conj{z}_1,\conj{z}_2)\,.
\end{eqnarray*}


The \emph{group orbit} of $x\in\Rls{n}$ is the set
\beq
	\Gamma x = \{\gamma x: \gamma\in\Gamma\}\,.
\eeq

Define the \emph{isotropy subgroup} or \emph{stabilizer} of $x\in\Rls{n}$ as
\beq
	\Sigma_x=\{\gamma\in\Gamma:\gamma x=x\}\,.
\eeq
Thus the isotropy subgroup describes the symmetries of a point $x$. The following usefull lemma
relates the isotropy subgroups of points on the same group orbit.

\begin{lemma}
\label{lm:stabGorbit}
Points on the same group orbit of $\Gamma$ have conjugate isotropy subgroups:
\beq
	\Sigma_{\gamma x}=\gamma \Sigma_x \gamma^{-1}\,.
\eeq
\end{lemma}
See \refref{golubII} for the proof.

\refLem{lm:stabGorbit} implies that we can characterize a group orbit by its \emph{type}, defined
as the conjugacy class of its isotropy subgroups.

\begin{proposition}
 Let $\Gamma$ be a compact Lie group acting on \Rls{n}. Then
 \begin{enumerate}
  \item If $\Gamma$ is finite then $|\Gamma|=|\Sigma_x||\Gamma x|$.
  \item If $\Gamma$ is continuous then $\dim \Gamma = \dim \Sigma_x+\dim \Gamma x$.
 \end{enumerate}
\end{proposition}
The proof can be found in \refref{golubII}. We note a usefull relation from the proof: $\dim\Gamma x =\dim(\Gamma/\Sigma_x)$, where the \emph{coset space} of a subgroup $\Sigma$  of $\Gamma$ is defined as $\Gamma/\Sigma=\{\gamma\Sigma|\gamma\in\Gamma\}$. Also recall that the (left) \emph{cosets} of $\Sigma$ in $\Gamma$ are the sets $\gamma\Sigma=\{\gamma\sigma|\sigma\in\Sigma\}$.

\index{stratum}
Therefore, when $\Gamma$ is continuous each group orbit is a smooth compact manifold of dimension
$\dim \Gamma x=\dim \Gamma-\dim \Sigma_x$. The union of orbits of the same type is called a \emph{stratum}
and is itself a smooth manifold. Thus \Rls{n} is stratified by the action of $\Gamma$ into
a disjoint union of strata which are in an $1-1$ correspondance to the group orbit types. Notice that in general
the strata do not have the same dimension.



% The notion of equivariance is related to the symmetries of the \ode\ \refeq{eq:dynsys}. We will also
% be interested on the symmetries of solutions.
