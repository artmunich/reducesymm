% section: Symmetries of dynamical systems

We consider a system of \ode's of the form
\beq
	\dot{x} = f(x,\lambda)
	\label{eq:dynsys}
\eeq
where $f: \Rls{n}\times\Rls{r}$ a $C^\infty$ mapping. When
not important we will suppress the $r$-dimensional vector of parameters
$\lambda$ in the notation. 

As is well known any compact Lie group acting on $\Rls{n}$ can be identified
with a subgroup of $\Og{n}$, \cf\ for example \refref{golubII} 
for a sketch of the proof. Therefore, without loss of generality
we will concentrate on subgroups $\Gamma\subseteq\Og{n}$ in the following.

\begin{definition}
We call a group element $\gamma\in\Og{n}$ a symmetry of \refeq{eq:dynsys} if for every solution
$x(t)$, $\gamma x(t)$ is also a solution.
\end{definition}

It is easy to see that this definition leads to the following condition for $\gamma$ to
be a symmetry of \refeq{eq:dynsys}:  
\beq
	f(\gamma x) =\gamma f(x)
	\label{eq:equiv}
\eeq
for all $x\in\Rls{n}$. We say that $f$ \emph{commutes} with $\gamma$ or that $f$ is $\gamma$-\emph{equivariant}.
When $f$ commutes with all $\gamma\in\Gamma$ we say that $f$ is $\Gamma$-equivariant. 
% When $f$ is $\gamma$-equivariant
% the \ode~ \ref{eq:dynsys} remains invariant under the action of $\gamma$.

% For example, Lorenz equations \ref{eq:lorenz} are 

The \emph{group orbit} of $x\in\Rls{n}$ is the set 
\beq
	\Gamma x = \{\gamma x: \gamma\in\Gamma\}\,.
\eeq

Define the \emph{isotropy subgroup} or \emph{stabilizer} of $x\in\Rls{n}$ as
\beq
	\Sigma_x=\{\gamma\in\Gamma:\gamma x=x\}\,.
\eeq
Thus the isotropy subgroup describes the symmetries of a point $x$. The following usefull lemma
relates the isotropy subgroups of points on the same group orbit:

\begin{lemma}
Points on the same group orbit of $\Gamma$ have conjugate isotropy subgroups:
\beq
	\Sigma_{\gamma x}=\gamma \Sigma_x \gamma^{-1}\,.
\eeq
\end{lemma}
See \refref{golubII} for the proof.

\begin{proposition}
 Let $\Gamma$ be a compact Lie group acting on \Rls{n}. Then
 \begin{enumerate}
  \item If $\Gamma$ is finite then $|\Gamma|=|\Sigma_x||\Gamma x|$.
  \item If $\Gamma$ is compact then $\dim \Gamma = \dim \Sigma_x+\dim \Gamma x$.
 \end{enumerate}
\end{proposition}
The proof can be found in \refref{golubII}. We note a usefull relation from the proof: $\dim\Gamma x =\dim(\Gamma/\Sigma_x)$, where the \emph{coset space} of a subgroup $\Sigma$  of $\Gamma$ is defined as $\Gamma/\Sigma=\{\gamma\Sigma|\gamma\in\Gamma\}$. Also recall that the (left) \emph{cosets} of $\Sigma$ in $\Gamma$ are the sets $\gamma\Sigma=\{\gamma\sigma|\sigma\in\Sigma\}$.

% The notion of equivariance is related to the symmetries of the \ode\ \refeq{eq:dynsys}. We will also
% be interested on the symmetries of solutions.


