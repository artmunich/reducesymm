% newton.tex
% $Author$ $Date$

%\section{Newton's method for determining \reqva}
% Predrag			jun 20 2006
% Vaggelis			may 20 2006



 Our task is to find \reqva\ solutions of \refeq{eq:KS}.
Although one can easilly see that this problem can be reduced to that of
 finding periodic orbits of a 4-dimensional ODE, here we prefer to consider our system in phase space and search for solutions of
 \beq
	\dot{b}_k=\dot{c}_k=0\,,
 \eeq
 for every $k$. The reason to do this is just getting experience before pursuing the more difficult task of locating POs and RPOs.
 Expanding $\dot{b}_k(a)$ and $\dot{c}_k(a)$ around our initial guess $a_o$ and demanding that they satisfy the equilibrium
 condition, we get
 \bea
	\dot{b}_k(a) & = & \dot{b}_k(a_o)+\left.\frac{\partial \dot{b}_k}{\partial b_j}\right|_{a_o}\delta b_j + \left.\frac{\partial \dot{b}_k}{\partial c_j}\right|_{a_o}\delta c_j = 0 \continue
	\dot{c}_k(a) & = & \dot{c}_k(a_o)+\left.\frac{\partial \dot{c}_k}{\partial b_j}\right|_{a_o}\delta b_j + \left.\frac{\partial \dot{c}_k}{\partial c_j}\right|_{a_o}\delta c_j = 0
 \eea
 or in matrix form
 \beq
    \left( \begin{array}{cc}
        \frac{\partial \dot{b}}{\partial b} & \frac{\partial \dot{b}}{\partial c} \\
        \frac{\partial \dot{c}}{\partial b}	& \frac{\partial \dot{c}}{\partial c}
     \end{array}
     \right)_{a_o}
     \left(\begin{array}{c}
       \delta b  \\
       \delta c
     \end{array}\right)
     =
     \left(\begin{array}{c}
       -\dot{b}(a_o) \\
       -\dot{c}(a_o)
     \end{array}\right)\,,
     \label{eq:NewtonEquil}
\eeq
where $\partial{\dot{b}} / \partial{b}$ \etc are $d \times d$ submatrices. Solving this
system of equations for the corrections $\delta b$ and  $\delta c$ and using the refined solution
as an initial guess yields  an approximation to the solution of the system.



\subsection{Implementing Newton's method  for RPOs}
\label{sec:NewtRPOs}

The relative periodic condition
\beq
	u(x+\kappa,t+T)=u(x,t) \,
\eeq
translates in Fourier space into
\beq	
	\sum_{k=-\infty}^{+\infty} a_k (t+T) e^{ i k (x+\kappa) / \tildeL}
		= \sum_{k=-\infty}^{+\infty} a_k (t) e^{ i k x / \tildeL} \,
\eeq
or
\beq
	e^{ik\kappa/\tildeL}a_k(t+T)=a_k(t) \,,\ \forall k \in \mathds{Z}\ \ \ \mathrm{(no\ summation)}.
	\label{eq:RPOcondition}
\eeq
We see that a relative periodic orbit returns after time $T$ to a point in
phase space with components $a_k(t+T)$ rotated in the complex plane by an
angle $-k\kappa/\tildeL$ with respect to $a_k(t)$. In matrix notation, we write \refeq{eq:RPOcondition} as
\beq
	R(\kappa)a(t+T)=a(t)\,,
	\label{eq:RPO}
\eeq
where we have defined
\beq
	R(\kappa) \equiv Diag[e^{ik\kappa/\tildeL}]\,.
\eeq
%We notice that $R(\kappa)$ is not a rotation operator..

Consider an initial guess $a'$ for a point on a relative periodic orbit and assume that it lies on
a \Poincare section $\mathcal{P}$ at $t=0$. Suppose that $\mathcal{P}$ is a hyperplane in
$\mathds{R}^{2d}$. The flow $f^t$ defined by \refeq{eq:Fcoef} transports
this point after time $T'$ into $a'(T')=f^{T'}(a')$. Suppose that this point is such that $R(\kappa')f^{T'}(a')$
is a point on $\mathcal{P}$. Consider next a point $a$ lying on $\mathcal{P}$ and in the neighborhood of $a'$,
thus satisfying
\beq
	q \cdot (a'-a) = 0\,,
	\label{eq:cond a}
\eeq
with $q$ a vector normal to $\mathcal{P}$. Point $a$ will be finally identified with the improved
approximation of a point on the periodic orbit.
The flow transports $a$ to $f^{T'}(a)$, but now $R(\kappa')f^{T'}(a)$ is not in general on $\mathcal{P}$.
Moreover we would like to have the freedom to adjust the guesses for $T'$ and $\kappa'$ into new values
$T=T'+\Delta T$ and $\kappa=\kappa'+\Delta \kappa$ to improve their accuracy.
Let as consider such slightly different values $T$ and $\kappa$ such that $R(\kappa)f^{T}(a)$ lies on
$\mathcal{P}$. Then we have the condition
\beq
	q \cdot(R(\kappa')f^{T'}(a')-R(\kappa)f^{T}(a)) = 0\,.
	\label{eq:cond Rf(a)}
\eeq

 We now can require that $a$ is a point on a relative periodic orbit and thus satisfies \refeq{eq:RPO}
\beq
	a=R(\kappa)f^{T}(a)\,,
	\label{eq:RPOcond}
\eeq
Taylor expanding $f^{T}(a)$ around $a'$ to linear order in the small quantities
$\Delta a=a-a'$ and $\Delta T$, we get
\bea
	f^{T}(a)& \simeq & f^{T}(a')+\J^T(a') \Delta a \label{eq:fTaylorl1} \\
		& \simeq & f^{T'}(a') + v \Delta T + \J^{T'}(a') \Delta a \label{eq:fTaylorl2} \,,
\eea
where $v$ is evaluated at $f^{T'}(a')$. Here $\J^t(x)$ is the Jacobian matrix, defined for a general flow through
\beq
   	J^t_{ij}(x_o)=\left.\frac{\partial x_i(t)}{\partial x_j}\right|_{x=x_0}\,.
\eeq
The Jacobian matrix is obtained by integrating the equation:
\beq
   	\dot{\mathbf{J}}^t=\mathbf{A J}^t \, ,
	\label{eq:Adef}
\eeq
subject to the initial condition:
\beq
   	\mathbf{J}^0=\mathbf{1} \, ,
\eeq
Here $\mathbf{A}$ is the matrix of variations defined as:
\beq
	A_{kj}=\frac{\partial \dot{x}_k}{\partial x_j}\,.
\eeq

In passing from \refeq{eq:fTaylorl1} to \refeq{eq:fTaylorl2} we have used the multiplicative
structure of the Jacobian, $\mathbf{J}^{T'+\delta T}(a')=\mathbf{J}^{\delta T}(f^{T'}(a'))\mathbf{J}^{T'}(a')$,
noticed that $\mathbf{J}^{\delta T}(f^{T'}(a'))=e^{\mathbf{A}\delta T}=\mathbf{1}+\mathbf{A}\delta T+\ldots$
and dropped second order terms in the small quantities.

On the other hand, we have
\bea
	R(\kappa'+\Delta\kappa) & = & R(\kappa')R(\Delta\kappa) \continue
				& \simeq & R(\kappa')(\mathbf{1}+iDiag[k]\Delta\kappa/\tildeL)\,.
	\label{eq:TaylorR}	
\eea

Substituting \refeq{eq:fTaylorl2},\refeq{eq:TaylorR} into \refeq{eq:RPOcond} and keeping only first
order terms in the small quantities, we get
\beq
	a'+\Delta a \simeq R(\kappa')f^{T'}(a') + \frac{i}{\tildeL}R(\kappa')Diag[k]f^{T'}(a')\Delta\kappa
				+ R(\kappa')v \Delta T + R(\kappa')\J^{T'}(a') \Delta a\,,
\eeq
or
\bea
	\left(1-R(\kappa')\J^{T'}(a')\right) \Delta a - R(\kappa')v \Delta T
							- \frac{i}{\tildeL}R(\kappa')Diag[k]f^{T'}(a')\Delta\kappa
					& \simeq & -\left(a'-R(\kappa')f^{T'}(a') \right) \continue
					& \equiv & -F(a') \,,
	\label{eq:NewtonBasicCond}			
\eea
where $F(a')$ is the function which zero we want to find.

Taylor expanding $R(\kappa)f^{T}(a)$ in \refeq{eq:cond Rf(a)} around $a'$ we get
\bea
	q \cdot \lefteqn{\left(R(\kappa')f^{T'}(a')-R(\kappa)f^{T}(a)\right) \simeq } \continue
%	 & & \left[R(\kappa')f^{T'}(a')-R(\kappa')(1+\frac{i}{\tildeL}Diag[k]\Delta\kappa)
%			\left(f^{T'}(a') + v(a') \Delta T + \J^{T'}(a')\Delta a\right)\right] \cdot q \continue
	 & & -q \cdot \left(R(\kappa')v\Delta T +R(\kappa')\J^{T'}(a')\Delta a
	 			+\frac{i}{\tildeL}R(\kappa')Diag[k]f^{T'}(a')\Delta\kappa \right)  = 0 \,.
	\label{eq:Taylor cond Rf(a)}
\eea

Equations \refeq{eq:cond a}, \refeq{eq:NewtonBasicCond} and \refeq{eq:Taylor cond Rf(a)}
can be compactly represented in a single matrix equation:
\beq
    \left( \begin{array}{ccc}
       1-R(\kappa')\mathbf{J}^{T'}(a') 	& -R(\kappa')v	  & - \frac{i}{\tildeL}R(\kappa')Diag[k]f^{T'}(a') \\
       q^{\dagger}R(\kappa')\J^{T'}(a') & q^{\dagger}R(\kappa')v & \frac{i}{\tildeL}q^{\dagger}R(\kappa')Diag[k]f^{T'}(a') \\
       q^{\dagger} 			& 0 	& 0
     \end{array}
     \right)
     \left(\begin{array}{c}
       \Delta a \\
       \Delta T \\
       \Delta \kappa
     \end{array}\right)
     =
     \left(\begin{array}{c}
       -F(a') \\
       0     \\
       0
     \end{array}\right)\,.
     \label{eq:NewtonScheme}
\eeq
Solving this equation for the corrections $\Delta a,\ \Delta T$ and $\Delta\kappa$ yields
an improved approximation to (a point of) the relative periodic orbit.

%The situation is similar to the one encountered when trying to identify
%periodic orbits with Newton's method.
