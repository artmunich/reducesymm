% Siminos thesis summary
% $Author$ $Date$


Recent experimental and theoretical advances\rf{science04}
support a dynamical vision of turbulence:
For any finite  spatial resolution, a turbulent flow follows approximately for a finite time
a pattern belonging to a { finite alphabet} of admissible patterns.
The long term dynamics is a {walk through the space of these unstable patterns}.
The question is how to characterize and classify such patterns?
Here we follow the seminal Hopf paper\rf{hopf48}, and  visualize
turbulence not as  a sequence of spatial snapshots in turbulent evolution,
but as a trajectory in an  $\infty$-$d$ \statesp\ in which an
instant in turbulent evolution is a {unique} point. In the dynamical systems approach,
theory of turbulence for a given system, with given boundary conditions,
is given by (a) the geometry of the \statesp\ and (b) the associated natural measure,
\ie, the likelihood that asymptotic dynamics visits a given \statesp\ region.

In this thesis this vision is pursued in the context of \KS\
equation, one of the simplest physically interesting spatially
extended nonlinear systems. Relaxing the restriction of
previous studies\rf{LanThesis,Christiansen97} to discrete
symmetry invariant subspace a new obstacle confronts our
attempt to elucidate the \statesp\ geometry. Continuous
symmetry, in the form of a subgroup of \On{n} acting on
\statesp, endorses \statesp\ with additional structure.
Symmetry dictates the bifurcation structure and the type of
observed solutions. At the same time recurrence becomes
relative: Not only do we have to identify points along the
orbit of the nonlinear group of time evolution but also points
along the orbit of the linear symmetry group. The latter
identification, termed symmetry reduction, although
conceptually simple as the group action is linear, is hard to
implement in practice. The high dimensionality of \statesp\
that results from truncation of a PDE is one of the reasons.
The second reason is that, for most interesting group actions,
the process introduces singularities in the structure of the
reduced \statesp.

Here we propose a procedure to efficiently project dynamics of
solutions computed in the original space to a reduced \statesp.
This is done in conjunction with eliminating time-translational
invariance with suitably chosen Poincar\'e sections to avoid
singularities while setting the stage to compute return maps
that describe how unstable manifolds of solutions organize the
flow. The procedure gives as insight, through simplifying
visualization, on which solutions of \KSe\ are
important in describing the geometry of \statesp. Moreover,
for a low-dimensional, strongly-contracting flow its
application leads to a one-dimensional return map that
encapsulates all the information about the dynamics.
