% Halcrow thesis summary
% $Author$ $Date$


The study of turbulence has been dominated historically by a ``bottom-up"
approach, with a much stronger emphasis on the physical structure of flows
than on that of the dynamical \statesp.  Turbulence has traditionally
been described in terms of various visually recognizable physical
features, such as waves and vortices.  Thanks to recent theoretical as
well as experimental advancements, it is now possible to take a more
``top-down" approach to turbulence.  Recent work has uncovered non-trivial
equilibria as well as relative periodic orbits in several turbulent
systems.  Furthermore, it is now possible to verify theoretical results at
a high degree of precision, thanks to an experimental technique known as
Particle Image Velocimetry.  These results squarely frame moderate
Reynolds number turbulence
in boundary shear flows as a
tractable dynamical systems problem.

In this thesis, I intend to elucidate
the finer structure of the \statesp\ of moderate Reynolds number
wall-bounded turbulent flows in hope of providing a more accurate and
precise description of this complex phenomenon.  Determination of
previously unknown
equilibria, \reqva, and their heteroclinic connections discovered
in course of this exploration
provides a skeleton upon which a numerically accurate description of
turbulence can be framed.  The behavior of the \eqva\ under variation of
Reynolds number and cell aspect ratios is also examined.
It is hoped that this description of the
\statesp\ will provide new avenues for research into nonlinear control systems
for shear flows as well as quantitative predictions of transport properties
of moderate Reynolds number fluid flows.
    %
    \PC{play with alternative titles: ``State space geometry of a spatio-temporally chaotic
Kuramoto-Sivashinsky flow'' continuous symmetries, etc..
    }
