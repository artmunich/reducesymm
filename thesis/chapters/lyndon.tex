
Consider a $k$-symbol alphabet $\mathcal{A}$. Let
$\alpha=uv$ and $\beta=vu$, were $u$ and $v$ any word. The words $\alpha$
and $\beta$ belong to an equivalnce class known as a \emph{necklace} and 
represented by its 
lexicographically lesser member. 
For instance, for the binary alphabet consisting of letters $0$ and $1$
the words $\{0001,\, 0010,\, 0100,\, 1000\}$ form a necklace 
represented by $0001$. An aperiodic necklace is called a \emph{Lyndon word}.

Lyndon words are of relevance to us because we are often interested 
in listing all symbolic itineraries that correspond to prime cycles up to
length $n$. For complete k-ary symbolic dynamics this is equivalent to the problem
of generating all Lyndon words of length at most $n$. As $n$ increases the efficiency
of the algorithm becomes quickly an issue. An algorithm for generating Lyndon
words has been found by Duval \refref{Duval88} and shown to be efficient in \refref{BerstelPocchiola94}.
Efficient here means that the running time is proportional to the number of words generated. 
In \reftab{duval} we provide pseudocode for Duval's
algorithm. For an explanation of the algorithm the reader is referred to 
\refref{Duval88}. For efficient algorithms to generate necklaces \cf~\refref{Cattell2000} 
and references within.

\begin{table}
	\caption{Duval's algorithm for efficient generation of Lyndon words up to length $n$. 
Here $\alpha$
and $\omega$ are the first and last letters of the alphabet $\mathcal{A}$ and the function $s(x)$
returns the next letter in the alphabet for every letter $x \neq \omega$. The auxiliary variable $\mathrm{w}$ is considered
a list of length $n$.}
\begin{tabbing}
   set i to 1 \\
   set $\mathrm{w}[1]$ to $\alpha$ \\
   while \= $i \neq 0$ \\
    	 \>  for \= $j=1$ to $n-i$ \\
         \>	\>  set $\mathrm{w}[i+j]$ to $\mathrm{w}[j]$ \\
	 \>  end for  \\
         \>  append $\mathrm{w}[1,...,i-1]$ to list of Lyndon words \\ 
         \>  set i = n \\
         \>  while \=  $i > 0$ and $\mathrm{w}[i]$ = $\omega$ \\
         \>       \> set $i$ to $i - 1$ \\
     	 \>  end while	\\	
         \>  if $i > 0$ then set $\mathrm{w}[i]$ to $s(\mathrm{w}[i])$ \\
   end while 
\end{tabbing}

\end{table}


%\begin{table}
%        \caption{Sample output of Duval's algorithm for $n=4$. We list the consecutive values of $i$ when a Lyndon word is generated.} 
%\end{table}

