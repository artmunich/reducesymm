% thesis/chapter/symRmrks.tex
% extracted from dasbuch \Chapter{discrete}{20apr2008}{World in a mirror}
% $Author: predrag $ $Date: 2008-07-30 16:02:46 -0400 (Wed, 30 Jul 2008) $

% Predrag                                       14aug2008
% extracted from \Chapter{knead} and \Chapter{symm}{{Discrete symmetries}

  \Remarks

\remark{Symmetries of the Lorenz equation:}{ \label{rem:LorenzSymm}
After having studied  \refexam{exmp:LorenzD1}
you will appreciate why {\tt ChaosBook.org}
starts out with the  symmetry-less
R\"ossler flow \refeq{Roesl_eq}, instead of the
better known Lorenz flow \refeq{Lorenz}
(indeed, getting rid of symmetry
was one of R\"ossler's motivations).
He threw the baby out with the water; for Lorenz flow
dimensionalities of stable/unstable manifolds
make possible a
robust heteroclinic connection absent from R\"ossler flow,
with unstable manifolds of an \eqv\ flowing into the
stable manifold of another \eqva.
How such connections are forced upon us is
best grasped by perusing the chapter 13 ``Heteroclinic tangles''
of the inimitable
Abraham and Shaw illustrated classic\rf{abraham:shaw}.
Their beautiful hand-drawn sketches elucidate the origin
of heteroclinic connections in the Lorenz flow (and its high-dimensional
Navier-Stokes relatives) better than any computer simulation.
Miranda and Stone\rf{GL-Mir93} were first to
quotient the $D_1$ symmetry and explicitly construct
the desymmetrized, ``proto-Lorenz system,''
by a nonlinear coordinate transformation into the Hilbert-Weyl
polynomial basis
invariant under the action of the symmetry group%
\rf{CoLiSh96}.
For in-depth discussion of symmetry-reduced (``images'')
and symmetry-extended (``covers'')
topology, symbolic dynamics, periodic orbits,
invariant polynomial bases \etc, of
Lorenz, R\"ossler and many other low-dimensional systems
there is
no better reference than the
Gilmore and Letellier monograph\rf{GL-Gil07b,GL-Let01}.
%
They interpret the proto-Lorenz and its ``double
cover'' Lorenz as ``intensities'' being
the squares of ``amplitudes,'' and call quotiented
flows such as (Lorenz)/$D_1$ ``images.''
Our ``doubled-polar angle'' visualization
\reffig{fig:PoincLorenz}
is a proto-Lorenz in disguise, with the difference: we
integrate the flow and construct Poincar\'e sections and
return maps in the Lorenz $[x,y,z]$ coordinates, without
any nonlinear coordinate transformations.
The Poincar\'e
return map \reffig{fig:RetMapLorenz} is reminiscent
in shape both of the one given by Lorenz
in his original paper, and the one plotted in
a radial coordinate by Gilmore and Letellier.
Nevertheless, it is profoundly different:
our return maps are
from unstable manifold $\to$ itself\rf{CCP96},
and thus intrinsic and coordinate independent. This is necessary
in high-dimensional flows
to avoid problems such as double-valuedness of return map projections
on arbitrary 1\dmn\ coordinates encountered already in
the R\"ossler example. More importantly, as we know the embedding
of the unstable manifold into the full \statesp, a cycle point
of our return map \emph{is} - regardless of the
length of the cycle -  the cycle point in the full  \statesp,
so no additional Newton searches are needed.

\index{Lorenz, E.N.}
\index{Cartwright, M.L.}
\index{Gilmore, R.}
\index{Letellier, C.}
    } %end \remark{Lorenz equation:}{ \label{rem:LorenzSymm}


\remark{Examples of systems with discrete symmetries.}{
One has
a $D_1$ symmetry in the Lorenz system (\refrem{rem:Lorenz}), % \rf{lorenz,GO},
the Ising model,
and in the 3\dmn\ anisotropic Kepler
potential\rf{gut82,TW,CC92},
a $D_3=C_{3v}$ symmetry in H\'enon-Heiles type potentials\rf{HH,JS,rich,laur},
a $D_4=C_{4v}$ symmetry in quartic oscillators\rf{EHP,MWR},
in the pure $x^2 y^2$ potential\rf{Mat,CP} and
in hydrogen in a magnetic field\rf{EW1},
and a $D_2=C_{2v} = V_4 =C_2\times C_2$ symmetry
in the stadium billiard\rf{robbDisc}.
A very nice application of desymmetrization is
carried out in \refref{BVdisc}.
\index{stadium billiard}
\index{billiard!stadium}
        }

\remark{H\'enon-Heiles potential.}{ \label{rem:HenHeil}
An example of a system with $D_3 = C_{3v}$ symmetry is provided by
the motion of a particle in the H\'enon-Heiles potential\rf{HH}
\[
V(r,\theta ) = {1 \over 2} r^2 + {1\over 3} r^3 \sin(3\theta )
\,\, .
\]\noindent
\index{symbolic dynamics!H\'enon-Heiles}
\index{Henon@H\'enon-Heiles!symbolic dynamics}
Our 3-disk coding is insufficient for this system because of the existence
of elliptic islands and because the three orbits that run along the symmetry
axis cannot be labeled in our code. As these orbits run along the
boundary of the fundamental domain,
they require the special treatment\rf{laurDisc}
discussed in \refsect{dscr:bound-o}.
    } %end \remark{H\'enon-Heiles potential.


    \PublicPrivate{
    }{ % switch \PublicPrivate{
\remark{Literature}{
A summary of discrete groups\rf{stevenj}.
\PC{Idea: a conserved quantity $\to$ continuous symmetry.
    Can one use Noether theorem?
\ESedit{ES: I think that we unfortunatelly cannot use it for dissipative
  systems. Noether's theorem applies provided
  we can formulate the system as a variational problem. It has caused
  me a great deal of confusion tending to think of symmetries in 
  this noetherian way even for dissipative systems.
  }
    }

Amazon.com reviewers rave about \refref{ITO96}.

Check out \refref{ITO96} (looks nice, get it)
and
 \refrefs{ChePiWa02,Jaco05,Kett95,Hall67}.

\ESedit{I like a lot Ian Melbourne's work on symmetric attractors:
http://personal.maths.surrey.ac.uk/st/I.Melbourne/symmetric_attractors.html

I find the distinction between instantaneous symmetry and symmetry
on average very meaningful and potential useful for us (it will appear
in the thesis soon.)}

% A book(but elementary, not useful):
% http://www.sst.ph.ic.ac.uk/people/d.vvedensky/courses.html

    } %end\remark{Literature}{
    } % end \PublicPrivate{

\RemarksEnd
