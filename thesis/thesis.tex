% thesis.tex
% $Author$ $Date$

\input type                     %% type: public/editing version
\input ../inputs/setupThesis    %% logical chores (nothing to edit)
\input ../inputs/layoutThesis   %% copy to here from dasbuch/book/layout.tex environments
\input ../inputs/editsDasbuch   %% editing comments, DasBuch style
\input ../inputs/def            %% do not edit; update from dasbuch/book/inputs/def.tex
\input ../inputs/defsThesis     %% all Vaggelis edits: \renewcommand, etc
% version.tex
% $Author$ $Date$

            \renewcommand{\version}{% \booktype\ %
version 2.1		June 18 2009
% Predrag: rewriting `cross-section' method
% version 2.0		April 6 2009
% Vaggelis: Georgia Tech copy
% version 1.2, 		Mar 11 2009
% Predrag, Vaggelis: post-delivery, pre-defense edits
            }
% version 1.11, 		Mar 10 2009
% Vaggelis: delivered to committee
% version 1.1,		Mar 10 2009
% Vaggelis redoing it: no globstab
% version 1.0,		  Mar 9 2009
% Vaggelis thought he did it
% version 0.4,            Mar 8 2009
% Vaggelis keeps on doing it
% version 0.3,            Mar 2 2009
% Vaggelis just does it
% version 0.2,            Jun  25 2008
% Predrag: harmonized with the new ChaosBook macros layout
% version 0.1,            Jun 10 2007
% Predrag, Vaggelis: initial setup
                 %% version      keep updating this!


\title{Recurrence in nonlinear partial differential equations with continuous symmetry}
% \title{Just do it:    \KS\  \statesp}
\author{Evangelos Siminos}

\principaladviser{Professor Predrag Cvitanovi\'c}
%\committeechair{}
\firstreader{Professor Roman Grigoriev}
\secondreader{Professor Kurt Wiesenfeld}
\thirdreader{Professor Michael Schatz}
\fourthreader{Professor Luca Dieci}[School of Mathematics]

\department{School of Physics}
\degree{Doctor of Philosophy}
\copyrightyear{2008}
    \ifpaper               %% GT SUBMISSION
\submitdate{6 April 2009}
\approveddate{6 April 2009}
%\figurespagefalse          %% FIX THIS!
    \else                   %% Predrag WHILE EDITING
\submitdate{\version}
\signaturepagefalse
\approveddate{[not yet]}
\figurespagefalse    %% drop from th2e web version
\tablespagefalse     %% drop from the web version
    \fi
%% The following are the defaults
%%    \titlepagetrue
%%    \signaturepagetrue
%%    \copyrightfalse
%%    \figurespagetrue
%%    \tablespagetrue
%%    \contentspagetrue
%%    \dedicationheadingfalse
%%    \thesisproposalfalse
%%    \strictmarginstrue

\bibfiles{../bibtex/siminos,../bibtex/ziminoz}
% \glossfiles{../bibtex/gloss} % ``List of Symbols or Abbreviations''
%                % explained in gatech-thesis-gloss.sty.
\bibpagetrue


\begin{document}
\bibliographystyle{inputs/gatech-thesis} %% must place right after \begin{document}

%% Preliminary Stuff
\begin{preliminary}

    \ifpaper               %% GT SUBMISSION
    \begin{dedication}
        \input chapters/dedication
    \end{dedication}
    \fi

%    \begin{preface}
%        \input chapters/preface
%    \end{preface}

    \begin{acknowledgements}
        \input chapters/acknowledgements
    \end{acknowledgements}

% print table of contents, figures and tables here.
    \contents

% if you need a "List of Symbols or Abbreviations" look into
% gatech-thesis-gloss.sty.
    \begin{summary}
        \input chapters/summary
    \end{summary}

\end{preliminary}

%%%%%%%%%%%%%%%%%%%%%%%%%%%%%%%%%%%%%%%%%%%%%%%%%%%%%%%%%%%%%%%%%%%%%%%%%%%%
%                          BEGIN MAIN BODY OF THESIS
%%%%%%%%%%%%%%%%%%%%%%%%%%%%%%%%%%%%%%%%%%%%%%%%%%%%%%%%%%%%%%%%%%%%%%%%%%%%

\ifboyscout \pagestyle{headings} \fi  %% Predrag: while editing

\chapter{Introduction}
%\input chapters/intro

% Hopf's last hope type stuff
    \section{Dynamicist's vision of turbulence}
    \label{s:hopf}
    \input chapters/hopf

    \section{Contribution of this thesis}
    \label{s:thesisIntro}
	\input chapters/thesisIntro
    \PublicPrivate{}{
    \section{The role of symmetry}
    \label{s:symIntro}
        \input chapters/rpoHistory
    }%end \PublicPrivate
    \PublicPrivate{
          }{ % switch to Private
    \section{Example dynamical systems used throughout the thesis}
    \label{s:exampleIntro}
    \input chapters/examples
    }%end \PublicPrivate

    \PublicPrivate{
\chapter{The role of symmetry}\label{chap:Symmetry}
                    }{
\chapter{Desymmetrization and its discontents}
\label{chap:Symmetry}

(from \HREF{http://en.wikipedia.org/wiki/Civilization_and_Its_Discontents}
{Civilization and its discontents})
                    }%end \PublicPrivate

In this chapter we provide a brief overview of the role of
symmetries in differential equations. We will restrict our
attention to finite dimensional groups acting linearly and
globally on \Rls{n}. The subject of symmetries of dynamical
systems is vast and covered in many monographs and review articles.
In this chapter we summarize the results from the literature
that will be needed in applications to the problem at hand, without
attempting to present proofs of well established results but refering
the reader to the literature.
% For more
% details the reader is referred to the monographs of Golubitsky and
% Stewart\rf{}, Chossat and Lauterbach\rf{}, Hoyle\rf{} and Olver\rf{}.

 \section{Symmetries of dynamical systems}
        \label{sec:symIntro}
        \input chapters/symODEs

\PublicPrivate{}{
  \section{Symmetries imply possible existence of \rpo s}
        \label{sec:SymRPO}
        \input chapters/symRLD
        \input chapters/symShooting
}%end \PublicPrivate

\PublicPrivate{}{
    \section{Literature survey}
        \label{sec:symmLit}
        \input chapters/symmLit
    }%end \PublicPrivate

\chapter{Desymmetrizaton of Lorenz equations}
\label{chap:Lorenz}
	In this chapter, which has been worked as an example for 
	\wwwcb~\rf{DasBuch} with Jonathan Halcrow, we present how discrete symmetry reduction in
	the case of Lorenz flow. It can be ommited in first reading
	as the method is rather pedagogical and the results are not
	needed in the rest of the thesis, but it helps illustrate the
	ideas in as simple a setting as possible. 
    \input chapters/Lorenz

\PublicPrivate{}{
    \section{Literature survey}
        \label{sec:LorenzLit}
        \input chapters/LorenzLit
    }%end \PublicPrivate 

\chapter{Desymmetrizaton of laser equations}
\label{chap:lasers}

 \section{Complex Lorenz equations}
    \label{sec:CLe}
    \input chapters/lasersSym

\chapter{\KSe}
\label{chap:KSe}
    \section{\KSe}
    \label{sec:KSe}
    \input chapters/KSe
\PublicPrivate{}{
    \section{Literature survey}
        \label{sec:KSlit}
        \input chapters/KSlit
}%end \PublicPrivate

% PART II: Implementation

\chapter{Simulating the \KSe}
\label{chap:Numerics}
    %\input chapters/numerics
    \section{Newton's method for determining \reqva}
        \input chapters/newton

% PART III Numerical results:

\chapter{\KS\ \statesp}
\label{chap:kseStSp}
In this chapter we explore the phase space of \KSe numerically for $L=22$.
The results presented here are part of Cvitanovi\'c, Davidchack and Siminos\rf{SCD07}.
    \section{Geometry of $L=22$ state space}
    \label{sec:L22}
    \input chapters/kseStSp

\chapter{\KS\ reduced \statesp}
    \label{chap:kseRedStSp}	
    \input chapters/kseRedStSp



\chapter{Periodic Orbit Theory}
\label{chap:POT}
    \section{Averaging}
    \input chapters/average

    \section{Cycle expansions}
        \label{sec:cycExp}
        \input chapters/cycExp

    \section{Flow conservation sum rules}
        \label{s-Cons-m-flow}
        \input chapters/flowCons

    \section{Stability ordering of cycle expansions}
        \label{s-StabOrd}
         \input chapters/stabOrder

% \chapter{Periodic orbit theory for \KS\ flow}
    % \input chapters/statesp

\PublicPrivate{}{
    \section{Literature survey}
        \label{sec:POsLit}
        \input chapters/POsLit
    }%end \PublicPrivate

\chapter{Speculation and Future Work}
   \label{chap:tobedone}
   \input chapters/tobedone.tex

\chapter{Conclusion}
   \label{chap:concl}
   \input chapters/conclusion


%%%%%%%%%%%%%%%%%%%%%%%%%%%%%%%%%%%%%%%%%%%%%%%%%%%%%%%%%%%%%%%%%%%%%%%%%%%%%%%%%%%%%%%%%%%%
%%%%                                CLOSING STUFF                                     %%%%%%
%%%%%%%%%%%%%%%%%%%%%%%%%%%%%%%%%%%%%%%%%%%%%%%%%%%%%%%%%%%%%%%%%%%%%%%%%%%%%%%%%%%%%%%%%%%%
\appendix

\chapter{Lyndon words}
\input chapters/lyndon

\chapter{\KS, real representation}

\section{\KSe\ according to Evangelos}
\input chapters/appeKSeDetails


\PublicPrivate{}{ % switch to Private
\chapter{A variational method}
\input chapters/appePenalizing
          } % end \PublicPrivate}

\chapter{Numerical Results}

          \PublicPrivate{
          }{ % switch to Private
\chapter{Flotsam}
\input chapters/flotsam      %% PC moved from blog/      26 jun 2008
\input chapters/bronski-2005 %% PC moved from blog/      26 jun 2008
\renewcommand{\inputfile}{\version\ - \today\ references}
          } % end \PublicPrivate}

\begin{postliminary}
\references
\gtindex
    \ifpaper               %% GT SUBMISSION
\begin{vita}
    \input chapters/vita
\end{vita}
    \fi
\end{postliminary}

\end{document}
