% thesis.tex
% $Author$ $Date$

\input type                     %% type: public/editing version
\input ../inputs/setupThesis    %% logical chores (nothing to edit)
\input ../inputs/layoutThesis   %% copy to here from dasbuch/book/layout.tex environments
\input ../inputs/editsDasbuch   %% editing comments, DasBuch style
\input ../inputs/def            %% do not edit; update from dasbuch/book/inputs/def.tex
\input ../inputs/defsThesis     %% all Vaggelis edits: \renewcommand, etc
% version.tex
% $Author$ $Date$

            \renewcommand{\version}{% \booktype\ %
version 2.1		June 18 2009
% Predrag: rewriting `cross-section' method
% version 2.0		April 6 2009
% Vaggelis: Georgia Tech copy
% version 1.2, 		Mar 11 2009
% Predrag, Vaggelis: post-delivery, pre-defense edits
            }
% version 1.11, 		Mar 10 2009
% Vaggelis: delivered to committee
% version 1.1,		Mar 10 2009
% Vaggelis redoing it: no globstab
% version 1.0,		  Mar 9 2009
% Vaggelis thought he did it
% version 0.4,            Mar 8 2009
% Vaggelis keeps on doing it
% version 0.3,            Mar 2 2009
% Vaggelis just does it
% version 0.2,            Jun  25 2008
% Predrag: harmonized with the new ChaosBook macros layout
% version 0.1,            Jun 10 2007
% Predrag, Vaggelis: initial setup
                 %% version      keep updating this!


\title{Recurrent spatio-temporal structures
       in presence of continuous symmetries}
% \title 2: Recurrence in nonlinear partial differential equations
%           with continuous symmetry
% \title 1: Just do it:    \KS\  \statesp
\author{Evangelos Siminos}

\principaladviser{Professor Predrag Cvitanovi\'c}
%\committeechair{}
\firstreader{Professor Roman Grigoriev}
\secondreader{Professor Kurt Wiesenfeld}
\thirdreader{Professor Michael Schatz}
\fourthreader{Professor Luca Dieci}[School of Mathematics]

\department{School of Physics}
\degree{Doctor of Philosophy}
\copyrightyear{2008}
    \ifpaper               %% GT SUBMISSION
\submitdate{May 2009}
\approveddate{March 23 2009}
%\figurespagefalse          %% FIX THIS!
    \else                   %% Predrag WHILE EDITING
\submitdate{\version}
\signaturepagefalse
\approveddate{[not yet]}
\figurespagefalse    %% drop from th2e web version
\tablespagefalse     %% drop from the web version
    \fi
%% The following are the defaults
%%    \titlepagetrue
%%    \signaturepagetrue
%%    \copyrightfalse
%%    \figurespagetrue
%%    \tablespagetrue
%%    \contentspagetrue
%%    \dedicationheadingfalse
%%    \thesisproposalfalse
%%    \strictmarginstrue

\bibfiles{../bibtex/siminos,../bibtex/ziminoz}
% \glossfiles{../bibtex/gloss} % ``List of Symbols or Abbreviations''
%                % explained in gatech-thesis-gloss.sty.
\bibpagetrue


\begin{document}
\bibliographystyle{inputs/gatech-thesis} %% must place right after \begin{document}

%% Preliminary Stuff
\begin{preliminary}

    \ifpaper               %% GT SUBMISSION
    \begin{dedication}
        \input chapters/dedication
    \end{dedication}
    \fi

%    \begin{preface}
%        \input chapters/preface
%    \end{preface}

    \begin{acknowledgements}
        \input chapters/acknowledgements
    \end{acknowledgements}

% print table of contents, figures and tables here.
    \contents

% if you need a "List of Symbols or Abbreviations" look into
% gatech-thesis-gloss.sty.
    \begin{summary}
        \input chapters/summaryNref
    \end{summary}

\end{preliminary}

%%%%%%%%%%%%%%%%%%%%%%%%%%%%%%%%%%%%%%%%%%%%%%%%%%%%%%%%%%%%%%%%%%%%%%%%%%%%
%                          BEGIN MAIN BODY OF THESIS
%%%%%%%%%%%%%%%%%%%%%%%%%%%%%%%%%%%%%%%%%%%%%%%%%%%%%%%%%%%%%%%%%%%%%%%%%%%%

\ifboyscout \pagestyle{headings} \fi  %% Predrag: while editing

\chapter{Introduction}
%\input chapters/intro

% Hopf's last hope type stuff
    \section{Dynamicist's vision of turbulence}
    \label{s:hopf}
    \input chapters/hopf

    \section{Contribution of this thesis}
    \label{s:thesisIntro}
	\input chapters/thesisIntro
    \PublicPrivate{}{
    \section{The role of symmetry}
    \label{s:symIntro}
        \input chapters/rpoHistory
    }%end \PublicPrivate
    \PublicPrivate{
          }{ % switch to Private
    \section{Example dynamical systems used throughout the thesis}
    \label{s:exampleIntro}
    \input chapters/examples
    }%end \PublicPrivate

\chapter{The role of symmetry}\label{chap:Symmetry}

In this chapter we provide a brief overview of the role of
symmetries in differential equations, restricted to finite
dimensional groups acting linearly and globally on \Rls{n}. The
subject of symmetries of dynamical systems is vast and covered
in many monographs and review articles. We summarize the
results from the literature that will be needed in applications
to the problem at hand, referring the reader to the literature
for proofs of well established results.

 \section{Symmetries of dynamical systems}
        \label{sec:symIntro}
        \input chapters/symODEs

%\PublicPrivate{}{
%  \section{Symmetries imply possible existence of \rpo s}
%        \label{sec:SymRPO}
%        \input chapters/symRLD
%}%end \PublicPrivate

%\PublicPrivate{}{
%    \section{Literature survey}
%        \label{sec:symmLit}
%        \input chapters/symmLit
%    }%end \PublicPrivate

\chapter{Desymmetrizaton of Lorenz equations}
\label{chap:Lorenz}
    \input chapters/Lorenz

\chapter{Desymmetrizaton of laser equations}
\label{chap:lasers}

 \section{Complex Lorenz equations}
    \label{sec:CLe}
    \input chapters/lasersSym

\chapter{\KS\ system}
\label{chap:KSe}
    \section{\KSe}
    \label{sec:KSe}
    \input chapters/KSe
%\PublicPrivate{}{
%    \section{Literature survey}
%        \label{sec:KSlit}
%        \input chapters/KSlit
%}%end \PublicPrivate

% PART II: Implementation

\chapter{Simulating the \KS\ system}
\label{chap:Numerics}
\section{Numerical integration}
	\input chapters/appeKSeDetails
    %\input chapters/numerics
\section{Shooting for \rpo s}
        \input chapters/newton
        \input chapters/symShooting

% PART III Numerical results:

\chapter{\KS\ \statesp}
\label{chap:kseStSp}

In this chapter we explore  numerically the \statesp\ of the \KS\ system
for $L=22$ system size.
The results presented are a
collaborative effort with P.~Cvitanovi\'c and R.L.~Davidchack\rf{SCD07}.

    \section{Geometry of $L=22$ \statesp}
    \label{sec:L22}
    \input chapters/kseStSp

\chapter{\KS\ reduced \statesp}
    \label{chap:kseRedStSp}	
    \input chapters/kseRedStSp


\chapter{Conclusion and future work}
   \label{chap:tobedone}
   \input chapters/conclusion
   \input chapters/tobedone.tex

% \chapter{Conclusion}
%    \label{chap:concl}


%%%%%%%%%%%%%%%%%%%%%%%%%%%%%%%%%%%%%%%%%%%%%%%%%%%%%%%%%%%%%%%%%%%%%%%%%%%%%%%%%%%%%%%%%%%%
%%%%                                CLOSING STUFF                                     %%%%%%
%%%%%%%%%%%%%%%%%%%%%%%%%%%%%%%%%%%%%%%%%%%%%%%%%%%%%%%%%%%%%%%%%%%%%%%%%%%%%%%%%%%%%%%%%%%%
\appendix

\chapter{Lyndon words}
\input chapters/lyndon

% \chapter{Periodic orbit theory for \KS\ flow}
\chapter{Stability ordering for \KS\ cycles}
\label{chap:POT} 	
In this chapter we describe a failed attempt to extract
quantitative information from \KSe\ cycles by
organizing them according to their stability.
Details on cycle expansions can be found in \rf{DasBuch}.
Here we only provide some sketchy background to stability ordering
by piecing together excerpts from \rf{DasBuch}.
	
\PublicPrivate{}{
    \section{Averaging}
    \input chapters/average

    \section{Cycle expansions}
        \label{sec:cycExp}
        \input chapters/cycExp

    \section{Flow conservation sum rules}
        \label{s-Cons-m-flow}
        \input chapters/flowCons
} % end \PublicPrivate}
%     \section{Stability ordering of cycle expansions}
        \label{s-StabOrd}
         \input chapters/stabOrder
%\PublicPrivate{}{
%    \section{Literature survey}
%        \label{sec:POsLit}
%        \input chapters/POsLit
%    }%end \PublicPrivate


%          \PublicPrivate{
%          }{ % switch to Private
%\chapter{Flotsam}
%% \input chapters/flotsam      %% PC moved back to blog/      3 oct 2009
%% \input chapters/bronski-2005 %% PC moved back to blog/      3 oct 2009
%\renewcommand{\inputfile}{\version\ - \today\ references}
%          } % end \PublicPrivate}

\begin{postliminary}
\references
% \gtindex
%     \ifpaper               %% GT SUBMISSION
% \begin{vita}
%     \input chapters/vita
% \end{vita}
%     \fi
\end{postliminary}

\end{document}
