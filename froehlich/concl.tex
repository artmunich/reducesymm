%           %experimenting with svn-multi

\svnkwsave{$RepoFile: siminos/froehlich/concl.tex $}
\svnidlong {$HeadURL$}
{$LastChangedDate$}
{$LastChangedRevision$} {$LastChangedBy$}
\svnid{$Id$}


\section{Conclusion}

Many systems in fluid dynamics exhibit a continuous symmetry. Systems
such as the \KS\ flow\rf{ku,siv},
{\pCf}\rf{Visw07b,GHCW07,HGC08,HalcrowThesis}, and flow through an
cylindrical pipe\rf{Wk04,Kerswell05} demonstrate a simple product of
$\SOn{2}$ symmetries.

In this paper we have investigated using linear subspaces in
\mslices\ to replace a dynamical system with an equivalent
lower dimensional system. The two main obstacles to using the \mslices\
are that every point must be rotatable into the subspace and it can
introduce singularities into the flow. Locally linear slices are
guaranteed to intersect each group orbit only once, but it was shown that
linear slices will intersect every group orbit of a compact Lie group in
the {\statesp}, though it will do so multiple times. The \mslices\ can
introduce singularities into the flow that did not exist in the full
space. We demonstrated that as long as the group action is well behaved
(which it is for any general $\SOn{2}$ symmetry) then a trajectory
passing through a singularity corresponds to a simple shift in the
trajectory and does not cause any difficulties. In addition we
demonstrated that the problem of dealing with singularities of a product
of $\SOn{2}$ groups acting on different coordinates of the {\statesp}
(as is the case for the \KS\rf{ku,siv},
{\pCf}\rf{Visw07b,GHCW07,HGC08,HalcrowThesis}, and
pipe flows\rf{Wk04,Kerswell05}) is equivalent to dealing with the
symmetries of each \SOn{2}\ symmetry independently.

Linear slices are a very simplistic condition to use for a subspace and
are practical to implement. They provide a practical alternative to
Hilbert bases for high dimensional flows. They also provide a
computationally simple method for finding invariants of a symmetry group;
what the Hilbert bases strive to do but are very costly to implement for
high dimensional flows.

While we have demonstrated that linear slices can be implemented in the
\mslices, work remains to be done before they can be used in more general
systems. We were able to impose a simple restriction for the
\cLe\ to choose a unique representative from the
hyperplane, but unfortunately it does not generalize to more advanced
systems. In addition, fluid flows, such as the {\pCf} and cylindrical
pipe flow, can also exhibit a discrete symmetry that the \mslices\ does
not handle. More has to be done to reduce a system with a discrete
symmetry before the \mslices\ can be used for the continuous symmetry.
