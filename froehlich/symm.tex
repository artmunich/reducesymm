%           %experimenting with svn-multi

\svnkwsave{$RepoFile: siminos/froehlich/symm.tex $}
\svnidlong {$HeadURL$}
{$LastChangedDate$}
{$LastChangedRevision$} {$LastChangedBy$}
\svnid{$Id$}


\section{Symmetries of dynamics}
\label{sect:SymmDyn}

Before we can investigate symmetries of a dynamical system, we must first develop a working definition of a what a 'symmetry' is.

We begin by defining the notion of `equivariance.'
A flow $\dot{x}= \vel(x)$ is \emph{equivariant} under an operation $\LieEl$ if
\beq
\vel(x)=\LieEl^{-1}\vel(\LieEl \, x)
\,.
\ee{eq:FiniteRot}
If $\ssp(\tau)$ is a solution to the dynamical
equations, then this implies $\LieEl\,\ssp(\tau)$ is also a solution.

The equivariant operations of dynamical system form a group under composition, and it this group that we call the symmetry of the dynamics.

In many flows (the \cLe\ are a particularly simple example), this symmetry group will form a Lie group. When the symmetry group is a Lie group, \mslices\ (see \refsect{sect:reducedStateSp}) can be applied to the system to replace it with an equivalent lower dimensional system.


\subsection{Lie groups}

A theory of Lie groups is a vast subject. This report follows the notational conventions of ChaosBook.org\rf{DasBuch}. We found Roger Penrose\rf{Penr04} introduction to the subject both enjoyable and understandable.
	\toCB
    \PC{to PC - remember to copy \refref{Penr04} to ChaosBook.org.}

A Lie group is a group with properties: (1) it is a differential manifold and (2) the composition map $G \times G \rightarrow G : (g,h) \rightarrow g h^{-1}$ is $\mathbb{C}^\infty$. We will only be considering Lie groups which are compact.
In studies of the \KS\ equations\rf{ku,siv,SiminosThesis}, {\pCf}\rf{Visw07b,GHCW07,HGC08,HalcrowThesis}, and cylindrical pipe flow\rf{Wk04,Kerswell05} periodic boundary conditions are imposed in  order to enforce that the symmetry group be compact.

An element of a compact Lie group can be parameterized in exponential form\rf{SiminosThesis,DasBuch}.
    \PC{Stefan wrote: ``By assumption the Lie group of symmetries is compact and connected;'' then ``we don't need connected, we just need a representation of the group elements that is differentiable, and there is one for $\On{n}$. For a group like $\On{n}$ you have two connected components, proof presumably still works? {\bf PC}: Agreed. 'Compact' is not needed, either}
For example, an element of a compact Lie group that is continuously connected to the identity can be expressed as
\beq
\LieEl(\gSpace)=e^{{\gSpace} \cdot \Lg }
    \,,\qquad
\gSpace \cdot \Lg = \sum_a \gSpace_a \Lg_a
\,,
\ee{FiniteRot}
where $\gSpace \cdot \Lg $ is a \emph{Lie algebra} element, the $\gSpace_a$ are the parameters of the transformation, and the $\Lg_a$ are a set of N linearly independent $[d\!\times\!d]$ antihermitian matrices acting linearly on the {\statesp} vectors\rf{DasBuch}.
    \PC{need to generalize to groups like \On{n}}
                                                    \exerbox{exer:FinRot2d}
%

A rotation by an infinitesimal amount, $|\delta \gSpace| \ll 1$, can be  expressed as\rf{DasBuch}
\beq
\LieEl(\delta \gSpace) \simeq 1 + \delta \gSpace \cdot \Lg.
\ee{eq:infinitesimal}
The $\Lg_a$  are called the \emph{generators} of infinitesimal rotations. To see why, define the group action tangent at $\ssp$,
\beq
 \groupTan_{a}(\ssp) = \Lg _{a} \ssp
    \,,\qquad
 a=1,2,\cdots,N,
\ee{PC:groupTan}
and consider a transformation induced by an infinitesimal
time-dependent variation of group `phases'
$\delta \gSpace_a = \timeStep \, \dot{\gSpace_a}$,
\[
\delta \ssp = \timeStep \, \dot{\gSpace} \cdot \groupTan(\ssp)
\,.
\]
So $\dot{\gSpace} \cdot \groupTan(\ssp)$ is the velocity
of the flow along the group orbit of \ssp.
We shall use $\groupTan_a(\sspRed)$ notation (rather than
$\Lg_{a}\sspRed$) to emphasize that the group action
induces a \emph{tangent field} at $\sspRed$.

The statement of equivariance
\refeq{eq:FiniteRot} for infinitesimal rotations is:
\[
\dot{x}=(1-\gSpace \cdot \Lg ) \vel(x+\gSpace \cdot \Lg  x)
       =\vel(x)-\gSpace \cdot \left(
            \Lg \vel(x) - \frac{d\vel}{dx} \Lg x
                     \right)
\,.
\]
We can now state the {\em infinitesimal
rotations} version of the equivariance condition
\refeq{eq:FiniteRot} as:
\beq
0 = - \groupTan_{a}(\vel)+\Mvar \groupTan_{a}(\ssp)
\,,
\label{eq:InfnmslRot}
\eeq
where $\Mvar$ is the \stabmat\ \refeq{SF:stabMat}.
% \refeq{5x5stabMat}.
    \PC{Stefan, you might want to learn about Jacobi derivatives,
        explain this statement geometrically}

%We have used both this infinitesimal rotation condition and
%the finite angle rotation condition \refeq{eq:FiniteRot}, to
%verify that the \cLe\ are rotationally equivariant.
%    \PC{Stefan, have you done this? Yes or no.}
%                                                    \exerbox{exer:InfinRotInvari}
%                                                    \exerbox{exer:FinRotInvarCmplx}
%                                                    \exerbox{exer:FinRotInvari}

When dealing with symmetry groups, certain subsets of the {\statesp} play an important role in understanding the action of the group.

\begin{definition}
\label{def:grouporbit}
\textbf{Group orbit.}
The orbit of a point $\ssp$ under the group $\Group$ is the set of all points that $\ssp$ are mapped to under the groups actions
\beq
\pS_\ssp=\{{g} \, \ssp: g \in \Group\}.
\eeq
The points in the fixed-point subspace $\pS_\Group$ are those points whose group orbits consist of only itself ($\pS_\ssp=\{\ssp\}$).
\end{definition}

%\paragraph{Definition:           Fixed-point subspace}
\begin{definition}
\label{def:centralizer}
\index{slice}
\textbf{Fixed-point subspace.}
\index{centralizer}\index{fixed-point subspace}
\index{G-fixed@\Group-fixed}\index{fixed point!under \Group}
$\pS_H$ or a `centralizer' of a subgroup $H \subset \Group$,
$\Group$ a symmetry of dynamics, is the set of all \statesp\
points left \emph{$H$-fixed}, \emph{point-wise} invariant
under action of the subgroup
\beq
\pS_H = \Fix{H} =
   \{ \ssp \in \pS : {h} \, \ssp = \ssp \mbox{ for all } h \in H \}
\,.
\ee{dscr:FPsubsp}
% \paragraph{Definition:         Invariants.}
Points in the \emph{fixed-point subspace}  $\pS_\Group$ are fixed
points of the full group action. They are called \emph{invariant
points},
	\index{invariant!points}
\beq
\pS_\Group = \Fix{\Group} =
   \{ \ssp \in \pS : {g} \, \ssp = \ssp \mbox{ for all } g \in \Group \}
\,.
\ee{dscr:InvPoints}
\end{definition}
		                                                  \toCB

If a point is an invariant point of the symmetry group, by the definition of equivariance \refeq{eq:FiniteRot} the velocity at that point is also in $\pS_\Group$, so the trajectory through that point will remain in $\pS_\Group$. $\pS_\Group$ is disjoint from the rest of the {\statesp} since no trajectory can ever enter or leave it. The fixed-point subspace of the $\SOn{2}$ symmetry group of the \cLe\ is the $z$-axis (see \refexam{exam:FinRot}). The velocity \refeq{eq:CLeR} at a point on the $z$-axis points only in the $z$-direction and so the trajectory remains on the $z$-axis for all times, as expected.

\begin{definition}
\label{def:SO2}
\textbf{\SOn{2}.} The special orthogonal group, \SOn{2}, is a group of length-preserving rotations. `Special' refers to requirement that $det \, g = 1$, in contradistinction to the orthogonal group \On{n} which allows for $det \, g = \pm 1$. \SOn{2}\ is a compact Lie group with a single infinitesimal generator.
\end{definition}

$\SOn{2}$ symmetries are common amongst fluid flows. The \KS\ flow\rf{ku,siv}, {\pCf}\rf{Visw07b,GHCW07,HGC08,HalcrowThesis}, and flow through a cylindrical pipe\rf{Wk04,Kerswell05} all have symmetry groups which are products (see \refdef{def:productGroup}) of $\SOn{2}$ symmetries along with discrete symmetries.

%
%%%%%%%%%%%%%%%%%%%%%%%%%%%%%%%%%%%%%%%%%%%%%%%%%
\example{\SOn{2} irreducible representations.}{
    \label{exam:SO2irrepst}
    \index{SO(2)@\SOn{2}!irreducible representation}
%DB% (continued from \refexam{exmp:contSO2rot})~~
Expand a smooth periodic function $u(\gSpace + 2\pi) =
u(\gSpace)$ as a Fourier series
\beq
u(\gSpace) = \frac{a_0}{2} + \sum_{m=1}^\infty \left(
a_m \cos m \gSpace + b_m \sin m \gSpace
                               \right)
\,.
\ee{FourierExp}
The matrix representation of the \SOn{2}\ action
%DB% \refeq{SO2LieFunct}
on the $m$th Fourier coefficient pair
$(a_m,b_m)$ is                                       \toCB
\beq
\LieEl^{(m)}(\gSpace') \,=\,  \left(\barr{cc}
 ~\cos m \gSpace'  & \sin m \gSpace' \\
 -\sin m \gSpace'  & \cos m \gSpace'
    \earr\right)
= \cos m \gSpace' \id^{(m)}
  + \sin m \gSpace'\, \frac{1}{m} \Lg^{(m)}
\,,
\ee{SO2irrepAlg-m}
with the Lie group generator
    \index{generator!anti-hermitian}
    \index{anti-hermitian!generator}
\beq
 \Lg^{(m)} \,=\,   \left(\barr{cc}
    0  &  m  \\
   -m  &  0
    \earr\right)
\,.
\ee{SO2irrepAlg-Lg}
$\Lg^{(m)}$ is the Lie
algebra generator and $\id^{(m)}$ is the identity on the $m$-irreducible
subspace, 0 elsewhere.
The \SOn{2}\ group tangent $\groupTan(u)$
%DB% \refeq{GroupTangField}
to \statesp\ point $u(\gSpace)$ is the sum over invariant subspace
contributions                            \toCB
\beq
 \groupTan(u) = \sum_{m=1}^\infty \groupTan^{(m)}(u)
    \,,\qquad
 \groupTan^{(m)}(u)
\,=\, m \,\left(\barr{c}
   ~b_m  \\
   -a_m
    \earr\right)
\,.
\ee{u:x:tang}
The $L^2$ norm of $\groupTan(u)$ is weighted by
the quadratic Casimir \refeq{QuadCasimir}. For \SOn{2} this is
$C_2^{(m)} = m^2$,
\beq
\oint \frac{d\gSpace}{2\pi}
     \, (\Lg u(\gSpace))^T \Lg u(2\pi-\gSpace)
= \sum_{m=1}^\infty m^2 \left(a_m^2 + b_m^2\right)
\,.
\ee{tangL2norm}
It converges only for sufficiently smooth $u(\gSpace)$. What
does that mean?
%DB% We saw in \refeq{RPOtrans_gen} that
$\Lg$ generates translations, and by \refeq{SO2irrepAlg-Lg} the velocity of the $m$th Fourier mode is $m$ times higher than for the $m=1$ component. If $| u^{(m)}| = (a_m^2+b^2_m)^{1/2}$ does not fall off faster than $1/m$, the action of \SOn{2}\ is overwhelmed by the high Fourier modes.
    } % end \example{\SOn{2} irreps
%%%%%%%%%%%%%%%%%%%%%%%%%%%%%%%%%%%%%%%%%%%%%%%%%
%

%
%%%%%%%%%%%%%%%%%%%%%%%%%%%%%%%%%%%%%%%%%%%%%%%%%
\example{\SOn{2} rotations for \cLe.}{\label{exam:FinRot}
    \index{SO(2)@\SOn{2}}
The \SOn{2} symmetry group of \cLe\ acts on the
5-dim\-ens\-ion\-al space \refeq{eq:CLeR}
by a finite angle \SOn{2} rotation:
\index{generator!anti-hermitian}
\index{anti-hermitian!generator}
\beq
\LieEl(\gSpace) \,=\,  \left(\barr{ccccc}
  \cos \gSpace  & \sin \gSpace  & 0 & 0 & 0 \\
 -\sin \gSpace  & \cos \gSpace  & 0 & 0 & 0 \\
 0 & 0 &  \cos \gSpace & \sin \gSpace   & 0 \\
 0 & 0 & -\sin \gSpace & \cos \gSpace   & 0 \\
 0 & 0 & 0             & 0              & 1
    \earr\right)
\,.
\ee{CLfRots}
The corresponding Lie algebra generator is
%    \PC{is the sign standard?
%        Oct 2010: yes, as in Arfken and Webber}
    \toCB
\beq
 \Lg \,=\,   \left(\barr{ccccc}
    0  &  1 & 0  &  0 & 0  \\
   -1  &  0 & 0  &  0 & 0 \\
    0  &  0 & 0  &  1 & 0  \\
    0  &  0 &-1  &  0 & 0 \\
    0  &  0 & 0  &  0 & 0
    \earr\right)
\,.
\ee{CLfLieGen}
%From \refeq{SO2irrepAlg-m} we see that
The action of \SOn{2}\ on the \cLe\ \statesp\ thus decomposes into $m=0$ \Group-invariant subspace ($z$-axis) and  $m=1$ subspace with multiplicity 2.

The generator $\Lg$ is indeed anti-hermitian,
$\Lg^\dagger = - \Lg$, and the group is compact, its
elements parameterized by $\gSpace \mbox{ mod } 2\pi$. Locally, at
$\ssp \in \pS$, the infinitesimal action of the group is
given by the group tangent field $\groupTan(\ssp) = \Lg \ssp
= (x_2,-x_1,y_2,-y_1,0)$. In other words, the flow induced by
the group action is normal to the radial direction in the
$(x_1,x_2)$ and $(y_1,y_2)$ planes, while the $z$-axis is left
invariant.
    } % end \example{Finite angle \SOn{2} rotations:}{
%%%%%%%%%%%%%%%%%%%%%%%%%%%%%%%%%%%%%%%%%%%%%%%%%
%

\subsection{\Reqva\ and \rpo s}

A continuous symmetry implies existence of `relative' solutions that are generalizations of \eqva\ \refdef{def:eqva},  and \po s \refdef{def:po}.

\begin{definition}
\textbf{\Reqva.}
A \reqv\ is a trajectory which stays in a single group orbit,
\beq
\ssp(\tau)=\LieEl(\tau) \ssp(0).
\eeq
The  velocity field at every point along a \reqv\ must point in the same direction as group tangent of the flow. It does so with a constant `angular' velocity $c$, \ie\ $\vel(\ssp)=c \cdot \groupTan(\ssp)$ everywhere along the \reqv. Up to the action of the group, a \reqv\ is the same as an \eqv.
\end{definition}

\begin{definition}
\textbf{\Rpo} $p$ 
is a trajectory that periodically returns to a point on the group orbit of its initial point, \ie, a set of relative periodic
points $\pS_p$ which exactly recur
    \toCB
\beq
\ssp (0) = \LieEl_p \ssp (\period{p} )
    \,,\qquad
\ssp (\tau) \in \pS_p
    \,,
\label{RPOrelper1}
\eeq
at a fixed {\em relative period} $\period{p}$, but
shifted by a fixed group action ${\LieEl_p}$
which brings the endpoint $\ssp (\period{p} ) $
back into the initial point $\ssp (0) $.
The group action ${\LieEl_p}=\LieEl(\gSpace_p)$ parameters   \toCB
$\gSpace = (\gSpace_1,\gSpace_2,\cdots\gSpace_N)$
are referred to as ``phases,'' or ``shifts.''
In general these phase are irrational, and the trajectory   \toCB
sweeps out ergodically the group orbit without ever closing
into a \po.
\end{definition}

\paragraph{Definition:
           \Rpo}
is an orbit  in {\statesp} $\pS$, 

 

The \reqva\ and \rpo s of the original system are mapped to \eqva\ and \po s in the reduced system. We can then apply the same wealth of knowledge to studying these trajectories in the \reducedsp\ that we have for \eqva\ and \po s\rf{DasBuch}.
    \ifarticle
    \else


\subsection{Inner products}
\label{def:innerProduct}
\index{slice}

One final piece of notation must be introduced before we can discuss the \mslices. The notion of an inner product on the {\statesp} is necessary to define a linear slice of the {\statesp}. We will therefore assume from now on that the {\statesp} has an inner product.

As we shall use here several inner products:
over group manifolds, over real and complex finite-dimensional coordinates, and over function spaces, it is convenient to introduce a compact notation that subsumes them all as special cases.
    \PC{We need to define these properly. When we get to fluids, it's not trivial - we will use `energy norm.' }

Here the dot product $\gSpace \cdot \Lg$ shall refer to the sum over
the Lie algebra generators of an $N$-dimensional Lie group \Group,
\beq
\gSpace \cdot \Lg = \sum_{a=1}^N \gSpace_a \Lg_a
%    \,,\qquad a = 1,2,\cdots,N
\,.
\ee{dotGroup}

The inner product of two \statesp\ vectors $x, y \in \pS$ will be denoted by $\braket{x}{y}$. If the \statesp\ is $\mathbb{R}^d$, then by the inner product we usually mean the Euclidian product of two vectors $x,y$,
\beq
\braket{x}{y} = \sum_i^d {x}_i y_i
    \,,\qquad \pS \subset \reals
\,.
\ee{innerR}
If the \statesp\ is finite dimensional and complex,
\beq
\braket{x}{y} = \sum_i^d \dual{x}_i y_i
    \,,\qquad \pS \subset \complex
\,,
\ee{innerC}
where $\dual{x}$ is the complex conjugate transpose of vector $x$, or, more generally, the hermitian conjugate $\dual{M}$ of matrix $M$.
    \PC{add a paragraph on the length as integral along a
        geodesic on a curved surface with a metric tensor}

In an inner product, a matrix $M$ acts as
    \PC{This should probably be extended to non-selfadjoint
        actions, explain the adjoint {\jacobianM} $\dual{\jMps}$.}
\beq
\braket{x}{M\,y} =
  \braket{\dual{M}\,x}{y}
\,.
\ee{adjointG}
where $\dual{M}$ is the hermitian conjugate of $M$.

If the \statesp\ is a normed function space (Banach, Hilbert, Sobolev, ...),
the inner product is given by the integral
\beq
\braket{g}{f} = \int dx \, \dual{g}(x) f(x)
\,.
\ee{innerL2}
The associated $L^2$ norm is
$|\ssp|^2 = \braket{\ssp}{\ssp} \neq 0$, unless $\ssp = 0$.

In computations the functions are expressed in terms of
complete orthonormal basis sets $\{u_n\}$,
\bea
f(x) &=& \sum_{n=0}^{\infty} a_n u_n(x)
    \continue
\braket{u_n}{u_m} &=& \delta_{nm}
\,.
\label{basisL2}
\eea

Unitary and orthogonal groups (as well as their subgroups) are defined as groups that preserve these `length' norms, $\braket{\LieEl x}{\LieEl x} =  \braket{x}{x}$, and infinitesimally their generators \refeq{eq:infinitesimal} induce no change in the norm,
\[
\braket{ \Lg_a\ssp}{\ssp}
  +\braket{\ssp}{\Lg_a\ssp} =0
\,,
\]
hence the Lie algebra generators
$\Lg$ are antisymmetric for orthogonal (sub)groups,
and antihermitian for unitary ones,
\beq
\dual{\Lg} = - \Lg
\,.
\ee{antiHerm}
This antisymmetry of generators
implies that the action of the group on vector $\ssp$ is
locally normal to it,
\beq
\braket{\groupTan_{a}(\ssp)}{\ssp} =0
\,.
\ee{TtimesX}

A group tangent \refeq{PC:groupTan} is a vector both in the group
tangent space and in the \statesp.
We shall indicate by $\braket{\groupTan_{a}(x)}{\groupTan_{b}(y)}$  the sum over \statesp\ inner product only, and by
\beq
\braket{\groupTan(x)}{\groupTan(y)} =
    \sum_{a=1}^N \braket{\groupTan_{a}(x)}{\groupTan_{a}(y)} =
  \braket{x}{\dual{\Lg} \cdot {\Lg}\,y}
\ee{innerGdot}
the sum over both group and spatial dimensions.

Any representation of a compact group $\Group$ is fully
reducible, and for a Lie group
the invariant tensors constructed by contractions
of $\Lg_a$ are useful for identifying irreducible
representations. The simplest such invariant is
\beq
\dual{\Lg} \cdot \Lg = \sum_\alpha C_2^{(\alpha)} \, \id^{(\alpha)}
\,,
\ee{QuadCasimir}
where $C_2^{(\alpha)}$ is the quadratic Casimir for
irreducible representation labeled $\alpha$, and
$\id^{(\alpha)}$ is the identity on the $\alpha$-irreducible
subspace, 0 elsewhere. $ C_2^{(\alpha)} =0$ if $\alpha$
is an invariant subspace.
The dot product of two tangent fields
\refeq{innerGdot} is thus a sum of inner products
weighted by Casimirs,
\beq
\braket{\groupTan(x)}{\groupTan(y)}
   = \sum_\alpha C_2^{(\alpha)} \dual{x}_i\, \delta_{ij}^{(\alpha)} y_j
\,.
\ee{braket}
An example is the Fourier series \refeq{tangL2norm}.
For compact groups $C_2^{(\alpha)}$ are strictly nonnegative by
the antihermiticity \refeq{antiHerm} of Lie algebra generators.

    \fi %end of article switch

