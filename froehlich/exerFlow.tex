\svnkwsave{$RepoFile: siminos/froehlich/exerFlow.tex $}
\svnidlong {$HeadURL$}
{$LastChangedDate$}
{$LastChangedRevision$} {$LastChangedBy$}
\svnid{$Id$}


\chapter{Exercises}
\label{sect:exerFlow}

\section{\CLf}

\exercise{Visualizations of the 5-dimensional \cLf:}{ \label{exer:PlotCLf}
Plot \cLf\ projected on any three of the five $\{x_1, x_2,
y_1, y_2, z\}$ axes. Experiment with different
visualizations.
    %
    \Private{ % subversion label pages
$\footnotemark\footnotetext{{\tt \svnkw{RepoFile}}, rev. \svnfilerev:
 last edit by \svnFullAuthor{\svnfileauthor},
 \svnfilemonth/\svnfileday/\svnfileyear}$
    } % end \Private{
    } % end \exercise{Visualizations of the 5-dimensional \cLf

\solution{exer:PlotCLf}{Visualizations of the 5-dimensional \cLf.}{
A numerical solution of the set of ODEs \refeq{eq:CLeR} is
obtained by using the \texttt{Mathematica} \texttt{NDSolve} function,
for $t$ from 0 to 100 and setting the initial $(x_1, x_2,
y_1, y_2, z)$ to some arbitrary  value. We set
\texttt{MaxSteps} $\to$ \texttt{Infinity} in order to resolve the fine
structure of the flow. The solution is then plotted by means
of \texttt{ParametricPlot3D}, in any three of the five
$\{x_1, x_2, y_1, y_2, z\}$ axes. \refFig{fig:CLEx1x2z}
illustrates the shape of the attractor projected onto the
$\{x_1,x_2,z\}$ subspace, with parameters set to the Siminos
 values \refeq{SiminosPrmts}.
Projections onto $\{y_1,y_2,z\}$ and other subspaces are
visually similar, and seem not to offer additional insights into dynamics
of this system.

Here is a \texttt{Mathematica} program that generates a
long-time plot of \cLe, such as \reffig{fig:CLEx1x2z},
initiated from a point on the attractor (after this
integration, the initial condition ic is the final point of
the preceding integration):

\begin{quote}
\texttt{
v[t$\_$]=\{-$\sigma$ x1[t] + $\sigma$ y1[t], -$\sigma$ x2[t] + $\sigma$ y2[t],\\
           (r1 - z[t])x1[t] - r2x2[t] - y1[t] - ey2[t],\\
           {r2}{x1}[t] + ({r1} - z[t]){x2}[t] + e{y1}[t] - {y2}[t],\\
           -b z[t] + {x1}[t]{y1}[t] + {x2}[t]{y2}[t]\};\\
x[{t$\_$}] = \{{x1}[t], {x2}[t], {y1}[t], {y2}[t], z[t]\};\\
d={Length}[x[t]];\\
{eqns}={Table}[D[x[t][[i]],t]==v[t][[i]],\{i,1,d\}];\\
{xde}=\{{x1}, {x2}, {y1}, {y2}, z\};\\
\\
\noindent
r1= 28; r2=0; b=8/3; e=1/10; $\sigma$=10;\\
tf= 80;\\
ic=\{x1[0]==0.867, x2[0]==-0.455, y1[0]==-0.552, y2[0]==0.453, z[0]==22.4\};\\
\\\\
sol=NDSolve[\{eqns,ic\}, {xde}, \{t,0,tf\}, {MaxSteps} $\to  \infty$]{//}{Flatten};\\
traj[{t\_}]=x[t]/.sol;\\
p1=ParametricPlot3D[\{traj[t][[1]],\ traj[t][[2]],\ traj[t][[5]]\},\ \{t,0,tf\},\ PlotPoints $\to$
 400,\ PlotRange $\to$  All]\\
ic=Table[x[0][[i]]==traj[tf][[i]],\ \{i,1,d\}];
}
\end{quote}

\authorSF{Jun 17 2009}
    } %end solution{exer:PlotCLf}

\section{Linear stability}

\exercise{\Stabmat\ of \cLf:}{ \label{exer:StabmatCLf}
Find the \stabmat\ of \cLe\ \refeq{eq:CLeR}.
    }

\solution{exer:StabmatCLf}{\Stabmat\ of \cLf.}{
Substituting  \refeq{eq:CLeR} into \refeq{5x5stabMat}, \texttt{Mathematica} program

     \begin{quote}
$
\text{v}[t] = \left\{\text{x1}'[t],\text{x2}'[t],\text{y1}'[t],\text{y2}'[t],z'[t]\right\};
\\
x[t] = \{\text{x1}[t],\text{x2}[t],\text{y1}[t],\text{y2}[t],z[t]\};
\\
\Mvar = D[\text{v}[t],\{x[t],1\}]
    $
    \end{quote}
we find
  \beq
  \Mvar =\left(\barr{ccccc}
    -\sigma    	& 0 		& \sigma & 0    &  0 \\
	0 	& -\sigma       & 0      & \sigma   &  0 \\
	r_1-z  &     -r_2      & -1     & -e & -x_1 \\
	r_2     & r_1-z       	& e  	& -1       & -x_2 \\
	y_1     & y_2           & x_1    & x_2      & -b
    \earr\right)
\,.
  \ee{CLeStabMat}
\authorSFPC{Jun 19 2009}
    }

\section{\Eqva}

Note to PC: remember to transfer solution to \refexer{exer:EquiCLe} to
ChaosBook.org.

\exercise{\Eqva\ of \cLe:}{\label{exer:EquiCLe}
Find all \eqva\ of \cLe.
    }

\solution{exer:EquiCLe}{\Eqva\ of \cLe.}{
Either the {\eqv} point is in $Fix(G)$ or it is not.

If it is in $Fix(G)$, then it is on the $z$-axis so we have $x_1 = x_2 = y_1 = y_2 = 0$. These values already give us $\dot{x}_1 = \dot{x}_2 = \dot{y}_1 = \dot{y}_2 = 0$, so we only need to check when $\dot{z} = 0$. This happens when $x_1 y_1 + x_2 y_2 - b z = - b z = 0$, so $z = 0$, and the origin is an {\eqv} point.

If an {\eqv} point is not in $Fix(G)$ then by the $SO(2)$ equivariance of the equations, there exists a corresponding {\eqv} point with $x_1 = 0$. The equations $\dot{x}_1 = 0$ and $\dot{x}_2 = 0$ then give us that $y_1 = x_1 = 0$ and $x_2 = y_2$ (assuming $\sigma$ is nonzero). Next, look at $\dot{y}_1 = -r_2 x_2 - e y_2 = 0$. If $x_2 = 0$ then we must also have $y_2 = 0$ and we are then on the z-axis and in $Fix(G)$. So we can assume $x_2$ is nonzero. We then have $- r_2 x_2 - e y_2 = 0$, $x_2 = y_2$ nonzero so we can divide through by $-x_2$ and get $r_2 + e = 0$. So in order for there to be an {\eqv} point other than the origin, we must have $e + r_2 = 0$. Looking at $\dot{y}_2 = 0$ we get that $y_2 / x_2 = r_1 - z$, so $z = r_1 - 1$. Finally, $\dot{z} = x_2 y_2 - b z = 0$, so $x_2 y_2 = b (r_1 - 1)$, and $x_2^2 = b (r_1 - 1)$. A solution to this will exist only if $r_1 > 1$.

Assuming $\sigma$ nonzero (Note: if $\sigma = 0$ then x is constant and the system is now linear), we therefore find that the origin is always an {\eqv} point, and it will be the only {\eqv} point unless $e + r_2 = 0$ and $r_1 > 1$. In this case, the system has the additional {\eqv} points $(0, 0, +/- \sqrt{b (r_1 - 1)}, +/- \sqrt{b (r_1 - 1)}, r_1 - 1)$ and all the points in their group orbits.
\authorSF{May 14 2010}
    } %end \solution{exer:EquiCLe}{\Eqva\ of \cLe.}

    % \subsection{Eigenvalues and eigenvectors of the \stabmat}


\exercise{\Eqva\ of \cLe:}{\label{exer:allEquiCLe}
In \refexer{exer:EquiCLe} we found only one \eqv\ of \cLe.
The Ning and Haken\rf{NingHakenCLE90} version of \cLe\
(a
truncation of Maxwell-Bloch equations describing a single
mode ring laser) sets $e+\ImrCLor=0$ so that a detuned
\eqv\ exists. Test your routines on 2 cases: (a)  $e=0$, $\ImrCLor=0$.
As discussed by Siminos\rf{SiminosThesis}, reality of
parameters $a$, $\rho$ in \refeq{eq:CLe} implies existence of
a discrete \Ztwo\ symmetry. (b) $e+\ImrCLor=0$, $e \neq 0$.
You might want to compare results with those of Ning and Haken\rf{NingHakenCLE90}.
    }

\solution{exer:allEquiCLe}{\Eqva\ of \cLe.}{
The alternate solution to the previous exercise showed that the origin was the only {\eqv} point, unless $e + \rho_2 = 0$. In this case if $r_1 > 1$ then the points $(0, 0, +/- \sqrt{b (r_1 - 1)}, +/- \sqrt{b (r_1 - 1)}, r_1 - 1)$ and all the points in their group orbits are also {\eqv} points.
\authorSF{May 14 2010}
    } %end \solution{exer:EquiCLe}{\Eqva\ of \cLe.}


\exercise{Eigenvalues and eigenvectors of $\EQV{0}$ \stabmat:}
         {\label{exer:EigenE0}
Find the eigenvalues and the eigenvectors of the \stabmat\
$\Mvar$ at $\EQV{0}=(0,0,0,0,0)$ determined in
\refexer{exer:EquiCLe}.
}

\solution{exer:EigenE0}{Eigenvalues and eigenvectors of $\EQV{0}$ \stabmat.}{
Using the Mathematica function \texttt{Eigensytem} and
setting $(x_1, x_2, y_1, y_2, z)=(0,0,0,0,0)$ at $t=0$ we
obtain
\beq
\begin{split}
\lambda_{1,2} &=11.8277 \pm 0.062985 i\\
\lambda_{3,4} &=-22.8277 \pm 0.037015 i\\
\lambda_5 &=-2.66667\\
\end{split}
\eeq
as the eigenvalues of the system, with the associated eigenvectors:
    \PC{replace the complex eigenvectors by their Re, Im parts.}
\beq
\begin{split}
e_{1} &= e_2^* =(0.00132196+0.458131 i, 0.458131-0.00132196 i, i, 1, 0)
\\
e_3 &= e_4^* = (0.00224947-0.779559 i, -0.779559-0.00224947 i, 2.842171+i, 1, 0)
\\
e_5 &=(0, 0, 0, 0, 1)
\,.
\end{split}
\eeq
\authorSF{Jun 26 2009}
} % end \solution{exer:EigenE0}{eigenvectors of $\EQV{0}$ \stabmat

\exercise{The eigen-system of the \stabmat\ at $\EQV{0}$:}
{\label{exer:PlotEigenE0}
Plot the eigenvectors of $\Mvar$ at $\EQV{0}$ and the \cLf\
at values very close to $\EQV{0}$.
}

\solution{exer:PlotEigenE0}
         {The eigen-system of the \stabmat\ at $\EQV{0}$.}{
In order to plot the complex eigenvectors, we split them into
their real and complex parts and plot each separately (so
that one complex eigenvector becomes two, defining a plane).
Using the same method as in the previous plotting exercises
(\refexer{exer:PlotCLf}, \refexer{exer:PlotPolEqu}, \refexer{exer:PlotRelEqu}
) we arrive at \reffig{fig:CLEE0}.
As the eigenvalues are nearly real, the stable/unstable 2$d$
manifolds barely spiral, and are not very illuminating.
\authorSF{Jul 01 2009}
} % end \solution{exer:PlotEigenE0}{The eigen-system of the stability matrix at $\EQV{0}$

\section{Symmetries of dynamics}

% \subsection{Rotational equivariance of \cLe}

\exercise{\SOn{2} rotations in a plane:}{ \label{exer:FinRot2d}
Show by exponentiation \refeq{FiniteRot} that the \SOn{2} Lie algebra
element $\Lg$ generates rotation $\mathbb{G}$ in a plane,
\beq
\mathbb{G}(\theta) = e^{\Lg \theta}
 = \cos\theta
   \left(\barr{cc}
    1  &  0   \\
    0  &  1
    \earr\right)
 + \sin\theta
   \left(\barr{cc}
    0  &  1   \\
   -1  &  0
    \earr\right)
 =   \left(\barr{cc}
    \cos\theta  &  \sin\theta   \\
   -\sin\theta  &  \cos\theta
    \earr\right)
 \,.
\label{SO2gener2d}
\eeq
    } %end \exercise{\SOn{2} rotations in a plane}{

\solution{exer:FinRot2d}{\SOn{2} rotations in a plane.}{
    \PC{(note to Predrag)
    also refer to example exmp:InOutSprl, if this exercise makes it
    to ChaosBook.org}
Expand $\exp(\theta\Lg)$ as a power series, noting that
\[
\Lg^2 = {\MatrixII{ 0}{ 1}
                 {-1}{ 0}
         }^2 = - \unit
\,.
\]
\authorPC{Jul 10 2009}
    } %end \solution{exer:FinRot2d}{\SOn{2} rotations in a plane

\exercise{Rotational equivariance of \cLe\ for infinitesimal angles.}
   { \label{exer:InfinRotInvari}
Show that \cLe\ are equivariant under infinitesimal \SOn{2} rotations.
    }

\solution{exer:InfinRotInvari}
         {Rotational equivariance of \cLe\ for infinitesimal angles.}{
Now that we have the \stabmat\ \refeq{5x5stabMat}, we can
 check the equivariance condition \refeq{eq:InfnmslRot},
$0=- \Lg v(x)+\Mvar \Lg x$,
where $\Mvar=\frac{\pde v}{\pde x}$ is the \stabmat, by explicit substitution.
The matrix $ \Lg $ is
\beq
 \Lg =\left(\barr{ccccc}
0&1&0&0&0\\
-1&0&0&0&0\\
0&0&0&1&0\\
0&0&-1&0&0\\
0&0&0&0&0
\earr\right)
\,.
\ee{SO2generCLe}

Plugging these into \refeq{eq:InfnmslRot}, as well as using
\refeq{eq:CLeR} for $v(x)$, the result is indeed 0, as expected. Then
the system is rotationally equivariant for infinitesimal
angles.
\authorSF{Jun 21 2009}
    } % end \solution{exer:InfinRotInvari}

\exercise{\Un{1} equivariance of \cLe\ for finite angles:}{ \label{exer:FinRotInvarCmplx}
Show that \cLe\ \refeq{eq:CLe} are equivariant under finite angle rotation
\refeq{FiniteRot}.
    }

\solution{exer:FinRotInvarCmplx}{\Un{1} equivariance of \cLe\ for finite angles.}{
Multiply the coordinates by a complex phase:
$x \to e^{\theta}x$, $y \to e^{\theta}y$. Equivariance of \refeq{eq:CLe} follows
by inspection. If all coefficients are real, there is also a discrete
\Zn{1} symmetry under complex conjugation of the three equations. However, we shall
consider here the cases where one or both of the parameters $r$ and $a$ are complex,
breaking this discrete symmetry.
\authorPC{Jul 9 2009}
    } % end \solution{exer:FinRotInvarCmplx}{\Un{1} equivariance

\exercise{\SOn{2} equivariance of \cLe\ for finite angles:}{ \label{exer:FinRotInvari}
Show that \cLe\ \refeq{eq:CLeR} are equivariant under rotation for finite angles.
    }

\solution{exer:FinRotInvari}{\SOn{2} equivariance of \cLe\ for finite angles.}{
For this problem, the operation is rotation and $v(x)$ is
given by \refeq{eq:CLeR}. Rotation can be defined using the
matrix
\beq
\begin{split}
\mathbb{G}(\theta) = \left(
\begin{array}{ccccc}
 \cos (\theta ) & \sin (\theta ) & 0 & 0 & 0 \\
 -\sin (\theta ) & \cos (\theta ) & 0 & 0 & 0 \\
 0 & 0 & \cos (\theta ) & \sin (\theta ) & 0 \\
 0 & 0 & -\sin (\theta ) & \cos (\theta ) & 0 \\
 0 & 0 & 0 & 0 & 1
\end{array}
\right)
\label{eq:rotate}
\end{split}
\eeq
where $\theta$ is the angle of rotation. We need to verify that
\beq
v(x)=\mathbb{G}^{-1} \cdot v(\mathbb{G} \cdot x)
\,.
\eeq
First, the system is rotated as
\beq
\left(
\begin{array}{ccccc}
 \cos (\theta ) & \sin (\theta ) & 0 & 0 & 0 \\
 -\sin (\theta ) & \cos (\theta ) & 0 & 0 & 0 \\
 0 & 0 & \cos (\theta ) & \sin (\theta ) & 0 \\
 0 & 0 & -\sin (\theta ) & \cos (\theta ) & 0 \\
 0 & 0 & 0 & 0 & 1
\end{array}
\right).
\left(
\begin{array}{c}
 {x}_1(t) \\
 {x}_2(t) \\
 {y}_1(t) \\
 {y}_2(t) \\
 {z}(t)
\end{array}
\right)
\,.
\eeq
The time derivative of the resulting matrix is taken, and then it
is multiplied on the left by the inverse of the rotation
matrix, giving
\beq
\mathbb{G}^{-1} \cdot v(\mathbb{G} \cdot x) = \left(
\begin{array}{c}
 \sigma  y_1(t)-\sigma  x_1(t) \\
 \sigma  y_2(t)-\sigma  x_2(t) \\
 -r_2 x_2(t)-y_1(t)-e y_2(t)+x_1(t) (r_1-z(t)) \\
 r_2 x_1(t)+e y_1(t)-y_2(t)+x_2(t) (r_1-z(t)) \\
 x_1(t) y_1(t)+x_2(t) y_2(t)-b z(t)
\end{array}
\right)
\,,
\eeq
which is the same as the original system of ODEs, so that the
system is rotationally equivariant for all finite angle
rotations.

When doing this in \texttt{Mathematica}, I
could not find how to set $\frac{d\theta}{dt}$ equal to 0, so
that the \texttt{Mathematica} output did not exactly match the
expected output. Knowing that $\frac{d\theta}{dt}$ does, in
fact, equal 0, one can ignore the additional terms I found.
\authorSF{Jun 19 2009}
    }

\section{\Reqva}

% \subsection{Coordinate changes}

\exercise{Harmonic oscillator in polar coordinates:}{\label{exer:HarmOscPolar}
Given a harmonic oscillator that follows $\dot{p}=-q$ and
$\dot{q}=p$, rewrite the system in polar coordinates
and find equations for $r$ and $\theta$.
}
\solution{exer:HarmOscPolar}{Harmonic oscillator in polar coordinates.}{
Harmonic oscillator equations in
Cartesian coordinates are
\beq
\dot{p}=-q\,,\qquad
\dot{q}=p \,.
\label{eq:HarmOscDE}
\eeq
In polar form, we  write
\beq
q=r\cos \theta \,,\qquad
p=r\sin \theta
\,.
\eeq
The inverse of the Jacobian
$\frac{\partial\{q,p\}}{\partial\{r,\theta\}}$ of this
transformation is
\beq
\frac{\partial\{r,\theta\}}{\partial\{q,p\}} =
\left(
\begin{array}{cc}\medskip
\cos \theta &\sin \theta \\
-\frac{1}{r}\sin \theta &\frac{1}{r} \cos \theta
\end{array}
\right)
\,,
\eeq
leading to
\beq
\left(
\begin{array}{c}\medskip
\dot{r}\\
\dot{\theta}
\end{array}
\right)=
\left(
\begin{array}{cc}\medskip
\cos \theta &\sin \theta \\
-\frac{1}{r}\sin \theta &\frac{1}{r} \cos \theta
\end{array}
\right) \cdot
\left(
\begin{array}{c}\medskip
\dot{q}\\
\dot{p}
\end{array}
\right)
=\left(
\begin{array}{c}\medskip
0\\
-1
\end{array}
\right)
\eeq
\authorSF{Jun 23 2009}
    } % end \solution{exer:HarmOscPolar}{Harmonic oscillator in polar coordinates

\exercise{\CLe\ in polar coordinates.}{\label{exer:polarCLe}
Rewrite \cLe\ from Cartesian to polar coordinates,
\beq
(x_1,x_2,y_1,y_2,z) =
    (\rho_1 \cos\theta_1,\rho_1\sin\theta_1,
     \rho_2\cos\theta_2,\rho_2\sin\theta_2,z)
\,,
\label{eq:CartToPol}
\eeq
where $\rho_1 \geq 0 \,,\rho_2 \geq 0$.
    }

\solution{exer:polarCLe}{\CLe\ in polar coordinates.}{
We use the same method here as in
\refexer{exer:HarmOscPolar}. The Jacobian of this
transformation can be written as
\beq
\frac{\partial\{\rho_1,\theta_1,\rho_2,\theta_2,z\}}
     {\partial\{x_1,x_2,y_1,y_2,z\}} =
\left(
\begin{array}{ccccc}
\cos(\theta_1) & \sin(\theta_1) & 0 & 0 & 0 \\
 -\frac{\sin(\theta_1)}{\rho_1} & \frac{\cos(\theta_1)}{\rho_1} & 0 & 0 & 0 \\
 0 & 0 & \cos(\theta_2) & \sin(\theta_2) & 0 \\
 0 & 0 & -\frac{\sin(\theta_2)}{\rho_2} & \frac{\cos(\theta_2)}{\rho_2} & 0 \\
 0 & 0 & 0 & 0 & 1
\end{array}
\right)
\,.
\label{eq:PolarJacob}
\eeq
Multiplying the velocity matrix on the left by the Jacobian
\refeq{eq:PolarJacob}, we get
\[ %\beq
\left(
\begin{array}{c}
\dot{\rho}_1\\
\dot{\theta}_1\\
\dot{\rho}_2\\
\dot{\theta}_2\\
\dot{z}
\end{array}
\right)
=
\left(
\begin{array}{c}
 -\sigma\left(\rho_1 - \rho_2\cos\theta\right) \\
 -\sigma\frac{\rho_2}{\rho_1}\sin \theta  \\
 -\rho_2 + \rho_1\left((r_1-z)\cos \theta - r_2 \sin\theta\right)\\
  e  + \frac{\rho_1}{\rho_2}\left((r_1-z)\sin\theta +r_2 \cos\theta\right)\\
 -b z + \rho_1\rho_2\cos\theta
\end{array}
\right)
,
\] %\ee{eq:PolarCLe}
where $\theta = \theta_1-\theta_2$. Following \refref{SiminosThesis},
we set $r_2=0$ in what follows.
It is convenient to rewrite this as 4 coupled equations
\beq
\left(
\begin{array}{c}
\dot{\rho}_1\\
\dot{\rho}_2\\
\dot{\theta}\\
\dot{z}
\end{array}
\right)
=
\left(
\begin{array}{c}
 -\sigma\left(\rho_1 - \rho_2\cos\theta\right) \\
 -\rho_2 + (r_1-z)\rho_1\cos \theta\\
  -e -\left(\sigma\frac{\rho_2}{\rho_1}
 +(r_1-z)\frac{\rho_1}{\rho_2}\right)\sin\theta\\
 -b z + \rho_1\rho_2\cos\theta
\end{array}
\right)
\,,
\label{eq:PolarCLeTheta}
\eeq
and two driven ones for the two angles,
\beq
\left(
\begin{array}{c}
\dot{\theta}_1\\
\dot{\theta}_2
\end{array}
\right)
=
\left(
\begin{array}{c}
-\sigma\frac{\rho_2}{\rho_1}\sin\theta  \\
 e + (r_1-z)\frac{\rho_1}{\rho_2}\sin\theta
\end{array}
\right)
\,,
\label{eq:PolarCLeAngles}
\eeq
in agreement with \refref{SiminosThesis}.
% PC: agrees with the Siminos thesis
\authorSFPC{Jun 23 2009}
} % end \solution{exer:polarCLe}{\CLe\ in polar coordinates

\exercise{Visualizations of the \cLf\ in polar coordinates:}{
\label{exer:PlotPolarCLf}
Plot \cLf\ projected on any three of the $\{{\rho}_1, {\rho}_2,
\theta, z\}$ coordinates. Experiment with different
visualizations. The flow \refeq{eq:PolarCLeTheta} is singular
as ${\rho}_j \to 0$, with angle $\theta_j$ going through a rapid
change there: is that a problem? (See also \refexer{exer:CLEsmall-x1x2}.)
Does it make sense to insist on $\rho_1 \geq 0 \,,\rho_2 \geq 0$,
or should one let them have either sign in order that the
$\theta$ trajectory be continuous?
    } % end {Visualizations of the \cLf\ in polar coordinates

\solution{exer:PlotPolarCLf}{Visualizations of the \cLf\ in polar coordinates.}{
(solution not available)
    } %end \solution{Visualizations of the \cLf\ in polar coordinates

% \subsection{Computing and plotting the relative {\eqv} $\REQB{1}$}

\exercise{Computing the \reqv\ $\REQB{1}$:}{\label{exer:CompRelEqu}
Find the \reqva\ of the \cLe\ by finding the \eqva\ of the
system in polar coordinates \refeq{eq:PolarCLeTheta}.
Compute the velocity of such \reqv.
}

\solution{exer:CompRelEqu}{Computing the \reqv\  $\REQB{1}$.}{
A relative {\eqv} point occurs when the derivatives
\refeq{eq:PolarCLeTheta} are equal to zero (so that the
difference between the two angles $\theta_1$ and $\theta_2$
is constant, but the angles themselves are not constant). We
use the \texttt{Solve} function (or the \texttt{Reduce}
function) in \texttt{Mathematica} to find these points.

We define the system in \texttt{Solve} by setting all time
derivatives in \refeq{eq:PolarCLeTheta} to zero.
\texttt{Mathematica} returns eight solutions of the form
\beq
\begin{split}
z(t) &\to r_1-1-\frac{e^2}{(\sigma +1)^2}\medskip\\
\rho_2(t) &\to \pm \frac{\sqrt{e^2+(\sigma +1)^2} \sqrt{-b
   \left(e^2-(r_1-1) (\sigma +1)^2\right)}}{(\sigma +1)^2}\medskip\\
\rho_1(t) &\to \pm \frac{\sqrt{-b \left(e^2-(r_1-1) (\sigma
   +1)^2\right)}}{\sqrt{(\sigma +1)^2}}\medskip\\
\theta (t) &\to \pm \cos ^{-1}\left(\pm\frac{\sqrt{(\sigma +1)^2}}
                                  {\sqrt{e^2+(\sigma +1)^2}}\right)
\end{split}
\label{eq:E1}
\eeq
The solutions differ by combinations of negative and positive
$\rho_1,\ \rho_2,\ \theta$. A negative $\rho_i$ solution
corresponds to the same group orbit of solutions, related to
the positive $\rho_i$ solution by a rotation by $\pi$. The
$\rho_1 \geq 0 \,,\rho_2 \geq 0$ condition in
\refeq{eq:CartToPol} reduces these solutions to two,
differing by the sign of $\cos ^{-1}$ term. Defining $d=1 +
{e^2}/{(\sigma +1)^2}$, we can write them compactly
as\rf{SiminosThesis}
\beq
(\rho_1,\rho_2,\theta,z) =
\left(\sqrt{b \,(r_1-d)},\sqrt{b d \,({r_1}-d}),
      \pm \cos^{-1}\left({1}/{\sqrt{d}}\right),
      r_1-d\right)
\label{eq:E1-PC}
\eeq
%    \PC{disagree with this text, removed it:
%``Only the solution given in \refeq{eq:E1} is listed in
%\refref{SiminosThesis}, and that is the only one that is used
%in further calculations. All of these eight solutions are
%valid, but because we are only looking for a relative
%{\eqv}, only one is needed to represent them all.''
%    }
%    \PC{please plot the solution \refeq{eq:E1-PC}, not
%    the one picked by Siminos\rf{SiminosThesis}; negative
%    radius in polar coordinate makes no sense
%    }
%    \ES{No negative
%	radius appears in Siminos\rf{SiminosThesis}, Predrag
%	please read the thesis! There is only one \reqv\, the
%	two choices of sign will still give you two different
%	points on the cycle of equivalent points as you observe
%	later on.
%    {\bf PC} I trusted Rebecca to compare the signs. The youth.
%    }
As will be further shown in \refexer{exer:PlotRelEqu},
these two solutions are equivalent.
For the Siminos parameter
values \refeq{SiminosPrmts}, the \reqv\ is at
\beq
\ssp_{\REQB{}1} = (\rho_1,\rho_2,\theta,z) =
     (8.48527,
      8.48562,
      0.00909066,
      26.9999)
\,.
\label{eq:Q1}
\eeq
The angular velocity of \reqv\ $\REQB{1}$ follows from
\refeq{eq:PolarCLeAngles}. Both angles move with the same
velocity
\beq
\dot{\theta}_{i}
= \frac{\sigma e}{\sigma + 1}
\,,
\label{eq:REQB1veloc}
\eeq
and period $\period{{\REQB{}1}}= 2\pi (\sigma +
1)/\sigma e$. $\period{{\REQB{}1}}= 69.115\cdots$ for the
Siminos parameter values \refeq{SiminosPrmts}. That implies that
the simulation has to be run up to time of order 35 or higher for
the strange attractor in \reffig{fig:CLEx1x2z} to start filling in.
\authorSFPC{Jul 7 2009, Aug 7 2009}
} % end \solution{exer:CompRelEqu}{Computing the Relative Equilibrium $\REQB{1}$

\exercise{\Reqv\ $\REQB{1}$ in polar coordinates:}{\label{exer:PlotPolEqu}
Plot the {\eqv} $\REQB{1}$ in polar coordinates.
}

\solution{exer:PlotPolEqu}{\Reqv\ $\REQB{1}$ in polar coordinates.}{
The same method as in exercise \refeq{exer:PlotCLf} can be
used here. First, a numerical solution is found with
\texttt{NDSolve} for $t$ going 0 to 10 and initial point
$\REQB{1}$. We again set \texttt{MaxSteps} $\to$ \texttt{Infinity}
in order to resolve the structure of the flow. Using
\texttt{ParametricPlot3D} to plot the flow in $\{\rho_1,
\rho_2, z\}$ axes, \refFig{fig:CLEPolEqui} illustrates the
shape of the flow with parameters set to the Siminos
values\rf{SiminosThesis} \refeq{SiminosPrmts}.
    \PC{$\backslash$label should be within the $\backslash$end\{figure\}
    environment, not outside of it}

\authorSF{Jun 26 2009}
} % end \solution{exer:PlotPolEqu}{\Reqv\ $\REQB{1}$ in Polar Coordinates

\exercise{\Reqv\  $\REQB{1}$ in Cartesian coordinates:}{\label{exer:PlotRelEqu}
Plot the relative {\eqv} $\REQB{1}$ in Cartesian coordinates.
}

\solution{exer:PlotRelEqu}{\Reqv\ $\REQB{1}$ in Cartesian coordinates.}{
Using the same method as in \refexer{exer:PlotPolEqu} and in
\refexer{exer:PlotCLf}, we can plot the relative {\eqv}
in Cartesian coordinates. However, the Cartesian system is
five-dimensional and our polar system has only four dimensions. To
resolve this, we set $\theta_2$ to an arbitrary value and set
$\theta_1=\theta+\theta_2$.
    \PC{what arbitrary value of $\theta_2$? Would'nt it be better
        to set $\theta_1=0$, to agree with what follows?}
With this, we can find the
numerical value of $\REQB{1}$ in Cartesian coordinates using
\refeq{eq:CartToPol}:
\beq
\ssp_{\REQB{}1} =
(x_1,x_2,y_1,y_2,z) = (8.48492,0.0771356,8.48562,0,26.9999)
\,.
\label{eq:Q1Cart}
\eeq
Using \texttt{Mathematica} to plot the system as before, but
with $t$ going from 0 to 100, \reffig{fig:CLERelEqui}\,(a) shows
the \cLf\ at $\REQB{1}$ projected onto the $\{x_1, x_2, z\}$
subspace.

Note that the for a \reqv\ the flow is along a circle,
\ie, the group-orbit of any point on it, but due to finite precision
of the initial point and the integration, the trajectory eventually
spirals away in a "horn" shape. This
circle cuts through the middle of the \cLe\ strange attractor, as
shown in \reffig{fig:CLERelEqui}\,(b).
    \PC{state the velocity of relative {\eqv},
        compare with the imaginary part of the complex
        stability eigenvalue, and explain the two time scales
        visible in the `horn', as well as the expansion rate
        per turn of the spiral.}

As noted in the solution to \refexer{exer:CompRelEqu},
\texttt{Mathematica} returns eight relative equilibria, but
they are all equivalent.
    \PC{need to rewrite this - you have implicitly chosen
    $\theta_2[0]=0$ (or similar) when you started drawing
    this, so you always get a circle, just from the freedom of
    choosing the initial angle. Perhaps mark the eight
    solutions by fat points on the circle, merge
    \reffig{fig:FourRelEquil} and \reffig{fig:CLEPolEqui} into
    one.}
\refFig{fig:FourRelEquil} shows the \cLf\ starting near four
of these eight points. Each of the points lies on the $\REQB{1}$
orbit, and each spirals away in the same way.


\authorSF{Jun 26 2009}
    } % end \solution{exer:PlotRelEqu}

\exercise{Eigenvalues and eigenvectors of $\REQB{1}$ \stabmat:}{\label{exer:EigenQ1}
Compute the eigenvalues and eigenvectors of the stability
matrix evaluated at $\REQB{1}$ and using the Siminos parameter
values \refeq{SiminosPrmts}.
}

\solution{exer:EigenQ1}{Eigenvalues and eigenvectors of $\REQB{1}$ \stabmat.}{
Using the Mathematica function \texttt{Eigensytem} and
setting $(\rho_1, \rho_2, \theta, z)(0)$ to the values in
\refeq{eq:Q1} we obtain
\[
(\lambda_{1,2},\lambda_3,\lambda_4)
= (0.0938179 \pm 10.1945 i,-11.0009,-13.8534)
\]
as the eigenvalues of the system with the associated eigenvectors:
    \PC{Rebecca, please recheck my replacement of
     the complex eigenvectors by their Re, Im parts.}
\bea
\Re e_{1} &=& (0.266121, -0.0321133, 0.00034139, 0.719222)
\continue
\Im e_{1}  &=& (0.295017, 0.569063, 0.000551886,0)
\continue
e_3 &=& (-0.0883591, -0.0851485, -0.989135, -0.0809553)
\continue
e_4 &=& (-0.855586, -0.329912, -0.00273531, -0.398902)
\label{eigVecQ1}
\eea
The spiral-out instability appears very slow, but with a
short period $\period{spiral}=0.6163$, about 112 turns for one
period of the $\REQB{1}$. Actually, as the \reqv\ velocity
\refeq{eq:REQB1veloc} is slow, an initial deviation from
$\ssp_{\REQB{}1}$ is multiplied by the factor
$\ExpaEig_{radial}=535$, and the \reqv\ is quite unstable.
This is illustrated by \reffig{fig:CLERelEqui}\,(b). It would
be sweet if we could eliminate the detuning drift time scale
$\approx 70$ and focus just on the oscillatory time scale of
$\approx 0.6$. That is one of the motivations for
reformulating dynamics in a \reducedsp.
                \authorSFPC{Jun 26 2009, Aug 7 2009}
} % end \solution{exer:EigenQ1}{Finding the eigenvalues and eigenvec

\exercise{The eigen-system of $\REQB{1}$ \stabmat\ in polar coordinates:}
{\label{exer:PlotPolEigenQ1}
Plot the eigenvectors of \Mvar\ at $\REQB{1}$ in polar
coordinates, as well as the \cLf\ at values very near
$\REQB{1}$.
}

\solution{exer:PlotPolEigenQ1}
{The eigen-system of $\REQB{1}$ \stabmat\ in polar coordinates.}{
In order to plot the complex eigenvectors, we split them into
their real and complex parts as in
\refexer{exer:PlotEigenE0}. Using the same method as in the
previous plotting exercises (\refexer{exer:PlotCLf},
\refexer{exer:PlotRelEqu}, \refexer{exer:PlotPolEqu}) we
constructed \refFig{fig:CLEQ1}\,(a) and
\reffig{fig:CLEQ1}\,(b).


\authorSF{Jul 01 2009}
} % end \solution{exer:PlotPolEigenE1}{The eigen-system

\section{\Reducedsp}

\exercise{$\SOn{2}$ or harmonic oscillator \slice:}{ \label{exer:SO2cSect}
Construct a moving frame {\slice} for action of $\SOn{2}$ on \Rls{2}
\beq
	(x,y) \mapsto (x\cos\theta -y \sin\theta,\,x\sin\theta +y \cos\theta )
\eeq
by, for instance, the positive $y$ axis: $x=0,\,y>0$. Write
out explicitly the group transformations that bring any point
back to the {\slice}. What invariant is preserved by this
construction?
\authorES{Jun 26 2009}
    } %end \exercise{$\SOn{2}$ \slice.

\solution{exer:SO2cSect}{$\SOn{2}$ or harmonic oscillator \slice:}{
We can now construct a moving frame as follows. We write out
explicitly the group transformations:
\begin{subequations}
\begin{align}
 	\overline{x} &= x \cos\theta - y \sin\theta\label{eq:explSO2stnd1}\cont
	\overline{y} &= x \sin\theta + y \cos\theta\label{eq:explSO2stnd2}\,.
\end{align}
\end{subequations}
Then set $\overline{x}=0$ and solve \refeq{eq:explSO2stnd1} for the group
parameter to obtain the moving frame
\beq
	\theta=\tan^{-1}\frac{x}{y}
	\label{eq:SO2stndMF}
\eeq
which brings any point  back to the {\slice}. Substituting
\refeq{eq:SO2stndMF} in the remaining equation, we get the
$\SOn{2}$-invariant expression
\[
	\overline{y} = \sqrt{x^2+y^2}\,.
\]
\authorES{Jun 26 2009}
    } %end \solution{exer:SO2cSect}

\exercise{\Statesp\ reduction by a \slice, finite time segments:}
{\label{exer:csectionCLe}
Replace integration of the \cLe\ by a sequence of short time steps,
each followed by a rotation such that the next segment initial point
is in the {\slice} $x_2=0,\,x_1>0$.
    }

\solution{exer:csectionCLe}{\Statesp\ reduction by a \slice.}{
We start by setting the initial point and the time step that
will be used during the integration. Using
\texttt{Mathematica}, we first remove the $z$ component of
the initial point so that we have $x(0)-x(0) \cdot
\hat{z}=\{x_1(0),x_2(0),y_1(0),y_2(0),0\}$. In order to
rotate each of the points, we must construct a matrix
\beq
\begin{split}
\mathbb{G}=
\left(
\begin{array}{ccccc}
\cos(\theta) & \sin(\theta) & 0 & 0 & 0\\
-\sin(\theta) & \cos(\theta) & 0 & 0 & 0\\
0 & 0 & \cos(\theta) & \sin(\theta) & 0\\
0 & 0 & -\sin(\theta) & \cos(\theta) & 0\\
0 & 0 & 0 & 0 & 1
\end{array}
\right)
\end{split}
\eeq
where $\theta$ is the angle between the new point (minus the
$z$ component) and the positive $x_1$ axis ($\hat{x_1}=\{1,
0, 0, 0, 0\}$). From \refexer{exer:FinRotInvarCmplx}, we know
that the \cLe\ are invariant for this kind of rotation, so
that when we reduce the \cLf\ it contains the same
information as the 5-dimensional system. We do not need to
determine the angles themselves, we need only their cosines
and sines. The cosine is found by a dot product as:
\beq
\cos(\theta)=\frac{x \cdot \hat{x}_1}{\lvert x \rvert}
\,
\eeq
Sines can be found by rotating each point by $\frac{\pi}{2}$
and then taking the dot product as with cosine. Taking the
first point, we first rotate it with $\mathbb{G}$, and
starting with this rotated point, we integrate over the
defined time step. We then take the last point from the
integration, find its $\mathbb{G}$ and rotate it, then use it
at the beginning of the next integration from $t=timestep$ to
$t=2 \cdot timestep$. This process continues (using a
\texttt{For} loop in \texttt{Mathematica}) until the end of
the integration (at an arbitrary time). The resulting list of
rotated points is then plotted. Using this method, we
produced \reffig{fig:CLEUnrotated}\,(a) and
\reffig{fig:CLEUnrotated}\,(b).
    \ES{It would be
very instructive to also include the $\{x_1, y_1, z\}$ plot and
increase the integration time to $t=200$ to discover that you
still have to deal with a two eared attractor. Do you get any kink? }
%%%%%%%%%%%%%%%%%%%%%%%%%%%%%%%%%%%%%%%%%%%%%%%%%%%%%%%%%%%%%%%%%%%
% computed by unrotate.nb
\begin{figure}[h]
\begin{center}
(a)%~\includegraphics[width=0.35\textwidth]{CLEUnrotatedPnts}
(b)%~\includegraphics[width=0.35\textwidth]{CLEUnrotatedLine}
\end{center}
\caption{
(a) $\{x_1, x_2, z\}$ plot of the \cLf\ as each point is
rotated back to the plane. The time step used to make this
plot was 0.01 and the integration went from $t=0$ to $t=50$.
(b) With the points connected into a curve.
    }
\label{fig:CLEUnrotated}
\end{figure}
%%%%%%%%%%%%%%%%%%%%%%%%%%%%%%%%%%%%%%%%%%%%%%%%%%%%%%%%%%%%%%%%%%%%

\authorSF{Jul 14 2009}
    } %end \solution{\Statesp\ reduction by a \slice}

\exercise{Probability of hitting $(x_1,x_2) =(0,0)$:}{
\label{exer:CLEsmall-x1x2}
Investigate by simulation (and - that is extra cost -
perhaps by thinking) how close does a strange attractor
trajectory of \cLf\ come to
hitting $(x_1,x_2) =(0,0)$ and/or  $(y_1,y_2) =(0,0)$?
%    \authorPC{Jul 14 2009}
    } %end\exercise{Probability of hitting $(x_1,x_2) =(0,0)$

\solution{exer:CLEsmall-x1x2}
{Probability of hitting $(x_1,x_2) =(0,0)$.}{
%
%%%%%%%%%%%%% PC generated by PCsimul.nb
\SFIG{ProblemsPill} %CLEsmall-x1x2}
{}{
A very long time ($t=5000$) simulation of \cLf. $\{x_1,x_2\}$ plot indicates
that probability of hitting $(x_1,x_2) =(0,0)$ is for all practical purposes
equal to zero, and $\sqrt{x_1^2+x_2^2} > 0.02$.
\authorPC{Jul 14 2009}
%(The initial point is on the strange attractor).
}
{fig:CLEsmall-x1x2}
%%%%%%%%%%%%%%%%%%%%%%%%%%%%%%%%%%%%%%%%%%%%%
%
A very long time ($t=5000$) simulation of \cLf, plot of
$\{x_1,x_2\}$  neighborhood in \reffig{fig:CLEsmall-x1x2},
indicates that probability of hitting $(x_1,x_2) =(0,0)$ is
for all practical purposes equal to zero, and
$\rho_1(t)=\sqrt{x_1^2+x_2^2} > 0.02$. Plots of $\{y_1,y_2\}$
neighborhood look similar.
Hence reduction of \statesp\ to 4-dimensional $x_2=0$ \reducedsp\ by rotating successive trajectory increments back
to the $x_1>0$ semi-axis should not run into an $x_1 =0$
singularity in determining the rotation angle $\cos\theta$.
\ES{I certainly agree with this and I think it is clear in
section 4.1.4.2 of the \emph{thesis no one wants to read} that
the singularity of the transformations does not pose a problem
for reduction in the case of \cLe\ as we do not encounter it
numerically (I've tried to pay the extra price and produce a proof
that the denominator cannot vanish but failed).
The reason that the singularity is annoying for me (beyond
visualization problems) is that in the \KS\ case dynamics goes
through zeros in the denominator.}
    \authorPC{Jul 14 2009}
    } %end \solution{Probability of hitting $(x_1,x_2) =(0,0)$.}{


\exercise{\Statesp\ reduction by a \slice, ODE formulation:}
{\label{exer:csectionCLeODE}
Reconsider \refeq{exer:csectionCLe} in the sequence of
infinitesimal time steps limit, each followed by an
infinitesimal rotation such that the next segment initial
point is in the {\slice} $x_2=0,\,x_1>0$. Derive the
corresponding 4$d$ \reducedsp\ ODE for the \cLf.
    } %end \solution{\Statesp\ reduction by a \slice, ODE formulation}

\solution{exer:csectionCLeODE}
         {\Statesp\ reduction by a \slice, ODE formulation:}{
Infinitesimal time version of the moving frames
symmetry reduction is attained by taking small time steps
in \reffig{fig:PCunrot} and dropping the higher order terms,
as in \refsect{sect:SymmDyn}:
\[
d\ssp^{(n)} = d t \, v(\ssp^{(n)})
                    + d \theta_1^{(n)} \, \Lg \cdot \ssp^{(n)}
\,.
\]
The infinitesimal angle is proportional to the time step,
\[
\theta_1^{(n)} \approx \sin \theta_1^{(n)}
             = - d t \, \frac{\hat{{\bf e}}_{1} \cdot v(\ssp^{(n)})}
                    {\rho_1^{(n)}}
             \approx - d t \, v_1(\ssp^{(n)})/\ssp_2^{(n)}
\,,
\]
where $(\rho_1,\theta_1)$ are polar coordinates,
$\rho_1 = \sqrt{\ssp_1^{ 2} + \ssp_2^{2}}$,
see \refeq{eq:CartToPol}. Our \slice\ condition is $x_1=0,\;x_2>0$,
so the \reducedsp\ equations are given by
\beq
\dot{\ssp} = v - \frac{v_1}{\ssp_2} \Lg \cdot \ssp
\,.
\ee{ExerMotionMovFrame}
The motion stays in the $(d\!-\!1)$-\slice, as $\dot{\ssp}_1 = 0$ due to
the orthogonal action of $\Lg$ to the direction $\ssp$.

Moving frames symmetry reduced \cLe\ are a 4-dimensional
first order ODE system
\beq
\begin{split}
	\dot{x}_2 &= -\sigma  (x_2 - y_2) \\
	\dot{y}_1 &= - y_1 + r_2 x_2 - \left(e + {\sigma y_1}/{x_2}\right) y_2\\
	\dot{y}_2 &=  - y_2 + (r_1-z) x_2 + \left(e + {\sigma y_1}/{x_2}\right) y_1 \\
	\dot{z}\; &= -b z + x_2 y_2
\,.
\end{split}
\ee{eq:rdcdCLeR}
The resulting trajectory is illustrated in
\reffig{fig:PCunrot1}. It agrees with trajectories reported
by Siminos (there the simulation is in the full \statesp, and
the \reducedsp\ dynamics is obtained by a coordinate change).
\ES{\refeq{eq:rdcdCLeR} follows from
\cLe\ expressed in the invariant coordinates obtained by the moving frame
method. Since this blog is not explicit transformation
friendly I've added this calculation in my thesis (sec. 4.1.4).
({\bf PC:} I fixed the errors in my rewrite - we agree).
We have done the same
for ZM system long time ago when we heuristically
rederived Cartan's method. It has been moved to a footnote in Jonathan's
blog (eq. 5.37) and became one the things that never found their way back to my thesis.
({\bf PC:} I never understood why ZM system was junked {\bf ES:} We've agreed to
junk it when we realized it has no relative equilibria. Also see Sec. 4.1 of my thesis.).
One also gets the same system by using invariant
polynomials and taking the syzygy into account, see discussion preceding 5.46 in
Jonathan's blog. {\bf PC:} please write this up in your thesis - will need it for
the planned publication.}

\noindent
- Checked that it agrees with finite step + rotation of \reffig{fig:PCunrot}\\
- $x_2$ and $y_2$ seem locked, oscillate the same way with amplitude up to 20\\
- $y_1$ is very small, mostly below 0.1\\
- $( x_2 , y_2 ,z)$ plot looks discontinuous in $y_2$ whenever $x_2$ small,
across the $z$-axis (indicated in red).
\authorSF{Jul 28 2009}
    } %end \solution{\Statesp\ reduction by a \slice, ODE formulation}

\exercise{Accumulated phase shift in \slice\ \reducedsp:}
{\label{exer:csectionPhase}
Derive the 1$d$ equation for the accumulated phase shift
$\theta$ associated with the 4\dmn\ \reducedsp\ ODE of
\refexer{exer:csectionCLeODE}.
    } %end \solution{Accumulated phase shift in \slice\ reduced \states

\solution{exer:csectionPhase}{Accumulated phase shift in \slice\ \reducedsp:}{
(not available)
% \authorSF{Aug ?? 2009}
    } %end \solution{Accumulated phase shift in \slice\ reduced \states

\exercise{The moving frame flow stays in the \reducedsp:}
{\label{exer:csectionReduced}
Show that the flow \refeq{EqMotionMovFramePC} stays in a
$(d\!-\!1)$-dimensional \slice.
\authorPC{Aug 10 2009}
    } %end \exercise{The moving frame flow stays

\solution{exer:csectionReduced}
{The moving frame flow stays in the \reducedsp:}
{The motion stays in the $(d\!-\!1)$-dimensional \slice, as
the flow along the group action direction vanishes,
    \ES{I don't understand this statement.
%In fact, if I try to calculate $\dot{\ssp} \cdot \Lg \cdot \ssp^{*}$
%explicitly for \cLe\ I get
%\[
% \dot{\ssp} \cdot \Lg \cdot \ssp^{*}
% = (-\sigma\,  x_2^{*} y_1+e\, y_1^{*}y_1+y_2 y_1-y_1^{*} y_2+\sigma\,
% x_1^{*} (y_2-x_2)+e\, y_2 y_2^{*}+x_2 y_2^{*} (r_1 -z)+x_1
%(\sigma  x_2^{*}+y_2^{*} (z-r_1 )))
%\]
%(note that I always take $r_2=0$) an expression that is not in general
%equal to zero.
Can you explain the general argument and/or show
that $\dot{\ssp} \cdot \Lg \cdot \ssp^{*}=0$
by explicit calculation?
    {\bf PC:} Mhm... wonders of \texttt{Mathematica}. Is this
    verification explicit enough? Used $T^2=-1$ on the
    4-dimensional subspace. {\bf ES:} This referred to the
    previous, incorrect version. Now I see the argument. Except 	
    you should use $T^{T}\cdot T=1$ instead of making sign errors
    an even number of times. {\bf PC:} No, it was also correct
    for the previous formulation. And the generators of any rotation
    group are antisymmetric, $ \Lg^{-1} =-  \Lg$, magical fact I
    used correctly even number of times.
    }
\[
\dot{\ssp} \cdot \Lg \cdot \ssp^{*} = v  \cdot \Lg \cdot \ssp^{*}
- \frac{(v \cdot \Lg \cdot \ssp^{*})}{(\ssp \cdot\ssp^{*})_4}
                 \, (\Lg \cdot \ssp) \cdot \Lg \cdot \ssp^{*} = 0
\,.
\]
\authorPC{Aug 10 2009}
    } %end \solution{exer:csectionReduced}

\exercise{Integration on a {\csection}:}{\label{exer:SiminosSlice}
Siminos\rf{SiminosThesis} replaces \cLe\ with the system
\beq
	\frac{dx}{dt}=\PperpOp(\ssp^{*}) v(x)\,,\qquad
 	\PperpOp_{ij}(\ssp^{*})=\delta_{ij}-
    \frac{(\Lg \cdot \ssp^{*})_i (\Lg \cdot \ssp^{*})_j}{(\Lg \cdot \ssp^{*})^2}
	\label{eq:difeqTransvII}
\eeq
where $x^{*}$ a point on the {\csection}, solutions will stay on
the {\csection} $\mathcal{K}$ for any initial condition on
$\mathcal{K}$ as there is no component of $\PperpOp(x_o)
v(x)$ in the direction of the continuous symmetry.
Reproduce Siminos \reffig{fig:CLEtransvII}\,(a). Explain the similarity
to the classical 2-eared Lorenz attractor (note that
dynamics on the slice are equivariant under rotations by
$\pi$).
    \PC{define, explain $\PperpOp(x_o)$ in this section}
    } %\end \exercise{Integration on the {\slice}:

\solution{exer:SiminosSlice}{Integration on a {\csection}.}{
Siminos \rf{SiminosThesis} gives the new equation as

\beq
\dot x_{\perp} = v(x)-\mathbb{T}x^{*} \frac{\mathbb{T}x^{*}\cdot v(x)}{(\mathbb{T}x^{*})^{2}}
\label{eq:IntSlice}
\eeq
Using $\Lg =\left(\barr{ccccc}
0&1&0&0&0\\
-1&0&0&0&0\\
0&0&0&1&0\\
0&0&-1&0&0\\
0&0&0&0&0
\earr\right)$
and $x^{*}=x_{\REQB{}1}$, we can reproduce \reffig{fig:CLEtransvII}\,(a) by using the integration and plotting methods described in previous exercises. Our result is shown in \reffig{fig:CLEtransvII}\,(b).


\authorSF{Aug 07 2009}
    } %end \solution{Integration on the {\slice}}

    \PublicPrivate{}{

\exercise{\Statesp\ reduction by a locally transverse \slice:}
{\label{exer:PClocSectCLe}
Decompose $\vf(\ssp)$
in $\dot{\ssp}= \vf(\ssp)$ in a part $\vf_\shortparallel$ parallel
to the group action and a part $\vf_\perp$ transverse to it,
\beq
	\vf(\ssp)=\vf_\shortparallel(\ssp)+\vf_\perp(\ssp)\,,
\ee{flowSplit}
using the projection operator
\beq
 	\PperpOp_{ij}(\ssp)=\delta_{ij}-
    \frac{(\Lg \cdot \ssp)_i (\Lg \cdot \ssp)_j}{(\Lg \cdot \ssp)^2}
\ee{transvProj}
that projects a $d$-dimensional flow $\vf(\ssp)$ onto
flow
\beq
	\dot{\ssp}_\perp = \vf_\perp(\ssp) = \vf(\ssp)
    - \frac{(\Lg\cdot\ssp)\cdot\vf(\ssp)}{(\Lg\cdot\ssp)^2} \Lg\cdot\ssp
\ee{transvFlow}
in a $(d\!-\!1)$-dimensional {\slice} transverse to the
direction fixed by the point $\ssp$.
By ignoring
the flow component that can be compensated for by an
$\SOn{2}$ rotation we quotient the flow by $\SOn{2}$. The
idea is presumably very old; for example, it is stated as the
problem 1. of Sect. 6.2 of Arnol'd {\em Ordinary Differential
Equations}\rf{arnold92}. Note, however, that a choice of
$\ssp_0$ fixes only a direction, so the \reducedsp\ flow is still
equivariant under the action of discrete cyclic group $\Ztwo
= \{e,D(\pi)\}$ on $\ssp$, $\vf(\ssp)$ and the reference
point $\ssp_0$, just as was the case for the
Lorenz flow. % \refeq{Lorenz}.
    \PC{\emph{Mea culpa}: Here I screwed up. I forgot that rotation
    moves $v$ and counter-moves $\ssp$ in $v(\ssp)$, \ie,
    acts by the Lie derivative \refeq{inftmInv}. I could never
    understand why we do not see a translational zero eigenvalue
    everywhere (the Lie group acts globally and commutatively
    right?), but only on \eqva, \reqva\ and \rpo s. Presumably
    the projection operator \refeq{transvProj} is OK for the
    \reqv\ calculation, % of \refeq{sect:StabEq},
    as the action of the group on $\ssp_{\REQB{1}}$ is trivial?
    Not sure how to rewrite the decomposition induced by
    \refeq{transvProj} correctly, in
    terms of the full Lie derivative action, and not only the $\Lg$
    action.
    }

\refeq{transvFlow} differs from \refeq{EqMotionMovFramePC}
only by replacement of the \slice\ fixed by an arbitrary
point $\ssp^{*}$ by \slice\ fixed by the instantaneous
state itself, $\ssp^{*} \to \ssp$. It does not work for us;
\rpo s do not reduce to \po s, and the strange attractor remains
embedded in 4 dimensions. Explain what is wrong with \refeq{transvFlow}.
\authorPC{Aug 7 2008}
    } %end \exercise{\Statesp\ reduction by a locally transverse \slice:}

\solution{exer:PClocSectCLe}
         {\Statesp\ reduction by a locally transverse \slice:}{
(not available)
    } %end \solution{\Statesp\ reduction by a locally transverse \csection:}

\exercise{Velocity field within \slice\ can diverge:}{
\label{exer:PCsectionCLe}
Apparent lack of singularities in \reffig{fig:CLEpcSect} appears
fortuitous or perhaps even a programming error.
This is a 4\dmn\ subspace, and indeed simulations do encounter
this subspace very, with \reducedsp\ velocity going off to infinity.
For example, take \cLf, with
(a) $\ssp^* \approx  \ssp_{\REQB{}1} = (8.48,0.077,8.48,0,26.99)$
on \reqv\ \REQB{1} group orbit,
and initial point
$\ssp(0) \approx  (4.81,0.154,7.59,0.281,15.9)$
      (taken from a long run on the strange attractor). How
far in time can you integrate the \reducedsp\ trajectory?
This, however, is perhaps a bad example, as the initial point is
not on the \slice. (b) Rotate  $\ssp(0)$ into the slice, use that
point as the initial point. How far can you integrate?
(c) Could \reducedsp\ \cLf\ flow run into a singularity
for other choices of $\ssp^*$? (d) For what
\slice s is the strange attractor of \cLf\
singularity-free?
    } %end \exercise{Velocity field within \slice\ can diverge


\solution{exer:PCsectionCLe}
{Velocity field within \slice\ can diverge:}{
(a) The denominator
$(\ssp(t) \cdot\ssp^{*})_4$ vanishes at $t \approx 1.217\cdots$.
(b) Apparently forever.
(c) The \reducedsp\ \cLf\ flow runs into a singularities in the
\slice\ choice of \reffig{fig:PCunrot1}.
(d) We have no idea...
\authorPC{Aug 7 2009}
    } %end \solution Velocity field within \slice\ can diverge

\exercise{\Statesp\ reduction by a \reqv\ $\REQB{1}$ \csection:}
{\label{exer:PCsectionCLe}
Replace integration of the \cLe\ by a sequence of short time steps,
each followed by a rotation such that the next segment initial point
is in the \reqv\ $\REQB{1}$ {\csection}
\beq
(\hat{\ssp}- \ssp_{\REQB{}1}) \cdot t_{\REQB{}1}=0
    \,,\qquad
t_{\REQB{}1} = \Lg \cdot \ssp_{\REQB{}1}
\,,
\ee{PCsect}
where for any $\ssp$, $\hat{\ssp} = \mathbb{G}(\theta) \cdot \ssp$
is the rotation that
lies in the \csection.
    } %end \exercise{\Statesp\ reduction by a \reqv\ $\REQB{1}$

\solution{exer:PCsectionCLe}
         {\Statesp\ reduction by a \reqv\ $\REQB{1}$ \csection.}{
We note that $\ssp_{\REQB{}1} \cdot t_{\REQB{}1}=0$ by the
antisymmetry of $\Lg$, so \refeq{PCsect} is a linear
condition $\hat{\ssp} \cdot \Lg \cdot \ssp_{\REQB{}1} = \ssp
\cdot \mathbb{G}(\theta) \cdot \Lg \cdot \ssp_{\REQB{}1} =0$
that determines $\theta$. Substituting \refeq{SO2gener2d} and
\refeq{SO2generCLe} yields a formula
\beq
\tan \theta = \frac{x_1 x^{\REQB{}1}_2 -x_1 x^{\REQB{}1}_2
                    +y_1 y^{\REQB{}1}_2 -y_1 y^{\REQB{}1}_2}
                   {x_1 x^{\REQB{}1}_1 + x_2 x^{\REQB{}1}_2
                   +y_1 y^{\REQB{}1}_1 + y_2 y^{\REQB{}1}_2}
\ee{PCsectTan}
for the rotation angle $\hat{\ssp} = \mathbb{G}(\theta) \cdot
\ssp$ (actually, there are two solutions, separated by $\pi$)
that rotates $\ssp$ into the \csection. In contrast to fixing one of
the polar angles as in the method of moving frames of
\refsect{sect:MovFrame}, this \csection\ introduces no
singularities, as $x_1^2 + x_2^2+y_1^2 + y_2^2>0$.
To compute $\sin \theta$, $\cos \theta$ needed by $\mathbb{G}(\theta)$
rewrite \refeq{PCsectTan} as
\bea
\cos \theta &=& (x_1 x^{\REQB{}1}_1 + x_2 x^{\REQB{}1}_2
                   +y_1 y^{\REQB{}1}_1 + y_2 y^{\REQB{}1}_2)/N
    \continue
\sin \theta &=& (x_1 x^{\REQB{}1}_2 -x_1 x^{\REQB{}1}_2
                    +y_1 y^{\REQB{}1}_2 -y_1 y^{\REQB{}1}_2)/N
\,,
\label{PCsectCos}
\eea
with $N$ fixed by $sin^2+cos^2=1$.
\authorPC{Aug 7 2009}
    } %end \solution{\Statesp\ reduction by a \reqv\ $\REQB{1}$ \csection
    } %end \PublicPrivate
