\svnkwsave{$RepoFile: siminos/froehlich/exerFlow.tex $}
\svnidlong {$HeadURL$}
{$LastChangedDate$}
{$LastChangedRevision$} {$LastChangedBy$}
\svnid{$Id$}


\chapter{Exercises}
\label{sect:exerFlow}

\section{\CLf}

\exercise{Visualizations of the 5-dimensional \cLf:}{ \label{exer:PlotCLf}
Plot \cLf\ projected on any three of the five $\{x_1, x_2,
y_1, y_2, z\}$ axes. Experiment with different
visualizations.
    %
    \Private{ % subversion label pages
$\footnotemark\footnotetext{{\tt \svnkw{RepoFile}}, rev. \svnfilerev:
 last edit by \svnFullAuthor{\svnfileauthor},
 \svnfilemonth/\svnfileday/\svnfileyear}$
    } % end \Private{
    } % end \exercise{Visualizations of the 5-dimensional \cLf

\solution{exer:PlotCLf}{Visualizations of the 5-dimensional \cLf.}{
\authorSF{Jun ?? 2010}
    } %end solution{exer:PlotCLf}

\section{Linear stability}

\exercise{\Stabmat\ of \cLf:}{ \label{exer:StabmatCLf}
Find the \stabmat\ of \cLe\ \refeq{eq:CLeR}.
    }

\solution{exer:StabmatCLf}{\Stabmat\ of \cLf.}{
\authorSF{Jun ?? 2010}
    }

\section{\Eqva}

Note to PC: remember to transfer solution to \refexer{exer:EquiCLe} to
ChaosBook.org.

\exercise{\Eqva\ of \cLe:}{\label{exer:EquiCLe}
Find all \eqva\ of \cLe.
    }

\solution{exer:EquiCLe}{\Eqva\ of \cLe.}{
Either the {\eqv} point is in $Fix(G)$ or it is not.

If it is in $Fix(G)$, then it is on the $z$-axis so we have $x_1 = x_2 = y_1 = y_2 = 0$. These values already give us $\dot{x}_1 = \dot{x}_2 = \dot{y}_1 = \dot{y}_2 = 0$, so we only need to check when $\dot{z} = 0$. This happens when $x_1 y_1 + x_2 y_2 - b z = - b z = 0$, so $z = 0$, and the origin is an {\eqv} point.

If an {\eqv} point is not in $Fix(G)$ then by the $SO(2)$ equivariance of the equations, there exists a corresponding {\eqv} point with $x_1 = 0$. The equations $\dot{x}_1 = 0$ and $\dot{x}_2 = 0$ then give us that $y_1 = x_1 = 0$ and $x_2 = y_2$ (assuming $\sigma$ is nonzero). Next, look at $\dot{y}_1 = -r_2 x_2 - e y_2 = 0$. If $x_2 = 0$ then we must also have $y_2 = 0$ and we are then on the z-axis and in $Fix(G)$. So we can assume $x_2$ is nonzero. We then have $- r_2 x_2 - e y_2 = 0$, $x_2 = y_2$ nonzero so we can divide through by $-x_2$ and get $r_2 + e = 0$. So in order for there to be an {\eqv} point other than the origin, we must have $e + r_2 = 0$. Looking at $\dot{y}_2 = 0$ we get that $y_2 / x_2 = r_1 - z$, so $z = r_1 - 1$. Finally, $\dot{z} = x_2 y_2 - b z = 0$, so $x_2 y_2 = b (r_1 - 1)$, and $x_2^2 = b (r_1 - 1)$. A solution to this will exist only if $r_1 > 1$.

Assuming $\sigma$ nonzero (Note: if $\sigma = 0$ then x is constant and the system is now linear), we therefore find that the origin is always an {\eqv} point, and it will be the only {\eqv} point unless $e + r_2 = 0$ and $r_1 > 1$. In this case, the system has the additional {\eqv} points $(0, \pm \sqrt{b (r_1 - 1)}, 0, \pm \sqrt{b (r_1 - 1)}, r_1 - 1)$ and all the points in their group orbits.
\authorSF{May 14 2010}
    } %end \solution{exer:EquiCLe}{\Eqva\ of \cLe.}

    % \subsection{Eigenvalues and eigenvectors of the \stabmat}


\exercise{\Eqva\ of \cLe:}{\label{exer:allEquiCLe}
In \refexer{exer:EquiCLe} we found only one \eqv\ of \cLe.
The Ning and Haken\rf{NingHakenCLE90} version of \cLe\
(a
truncation of Maxwell-Bloch equations describing a single
mode ring laser) sets $e+\ImrCLor=0$ so that a detuned
\eqv\ exists. Test your routines on 2 cases: (a)  $e=0$, $\ImrCLor=0$.
As discussed by Siminos\rf{SiminosThesis}, reality of
parameters $a$, $\rho$ in \refeq{eq:CLe} implies existence of
a discrete \Ztwo\ symmetry. (b) $e+\ImrCLor=0$, $e \neq 0$.
You might want to compare results with those of Ning and Haken\rf{NingHakenCLE90}.
    }

\solution{exer:allEquiCLe}{\Eqva\ of \cLe.}{
The alternate solution to the previous exercise showed that the origin was the only {\eqv} point, unless $e + r_2 = 0$. In this case if $r_1 > 1$ then the points $(0, \pm \sqrt{b (r_1 - 1)},0, \pm \sqrt{b (r_1 - 1)}, r_1 - 1)$ and all the points in their group orbits are also {\eqv} points.
\authorSF{May 14 2010}
    } %end \solution{exer:EquiCLe}{\Eqva\ of \cLe.}


\exercise{Eigenvalues and eigenvectors of $\EQV{0}$ \stabmat:}
         {\label{exer:EigenE0}
Find the eigenvalues and the eigenvectors of the \stabmat\
$\Mvar$ at $\EQV{0}=(0,0,0,0,0)$ determined in
\refexer{exer:EquiCLe}. Replace the complex eigenvectors by
their real, imaginary parts so you can plot them in (real)
\statesp.
 }

\solution{exer:EigenE0}{Eigenvalues and eigenvectors of $\EQV{0}$ \stabmat.}{
\authorSF{Jun ?? 2010}
} % end \solution{exer:EigenE0}{eigenvectors of $\EQV{0}$ \stabmat

\exercise{The eigen-system of the \stabmat\ at $\EQV{0}$:}
{\label{exer:PlotEigenE0}
Plot the eigenvectors of $\Mvar$ at $\EQV{0}$ and the \cLf\
at values very close to $\EQV{0}$.
}

\solution{exer:PlotEigenE0}
         {The eigen-system of the \stabmat\ at $\EQV{0}$.}{
\authorSF{Jun ?? 2010}
} % end \solution{exer:PlotEigenE0}{The eigen-system of the \stabmat\ at $\EQV{0}$

\section{Symmetries of dynamics}

% \subsection{Rotational equivariance of \cLe}

\exercise{\SOn{2} rotations in a plane:}{ \label{exer:FinRot2d}
Show by exponentiation \refeq{FiniteRot} that the \SOn{2} Lie algebra
element $\Lg$ generates rotation $\LieEl$ in a plane,
\beq
\LieEl(\theta) = e^{\Lg \theta}
 = \cos\theta
   \left(\barr{cc}
    1  &  0   \\
    0  &  1
    \earr\right)
 + \sin\theta
   \left(\barr{cc}
    0  &  1   \\
   -1  &  0
    \earr\right)
 =   \left(\barr{cc}
    \cos\theta  &  \sin\theta   \\
   -\sin\theta  &  \cos\theta
    \earr\right)
 \,.
\label{SO2gener2d}
\eeq
    } %end \exercise{\SOn{2} rotations in a plane}{

\solution{exer:FinRot2d}{\SOn{2} rotations in a plane.}{
    \PC{(note to Predrag)
    also refer to example exmp:InOutSprl, if this exercise makes it
    to ChaosBook.org}
Expand $\exp(\theta\Lg)$ as a power series, noting that
\[
\Lg^2 = {\MatrixII{ 0}{ 1}
                 {-1}{ 0}
         }^2 = - \unit
\,.
\]
\authorPC{Jul 10 2009}
    } %end \solution{exer:FinRot2d}{\SOn{2} rotations in a plane

\exercise{Rotational equivariance of \cLe\ for infinitesimal angles.}
   { \label{exer:InfinRotInvari}
Show that \cLe\ are equivariant under infinitesimal \SOn{2} rotations.
    }

\solution{exer:InfinRotInvari}
         {Rotational equivariance of \cLe\ for infinitesimal angles.}{


         The system is equivariant under infinitesimal angles if its Lie derivative is zero, i.e.
         \beq
         - \Lg \cdot v(x)+\Mvar \cdot \Lg \cdot x=0
         \label{SF:LieDeriv}
         \eeq
         where A is the \stabmat\ for the flow.

         For the \cLe
         \beq
            v\left(x\right) = \left(\barr{ccccc}
            -\sigma x_1+\sigma y_1 \\
            -\sigma x_2+\sigma y_2 \\
            \left(\rho_1-z\right)x_1-\rho_2 x_2-y_1-e y_2\\
            \rho_2 x_1+\left(\rho_1-z\right)x_2+e y_1-y_2\\
            -b z+x_1 y_1+x_2 y_2
            \earr\right)
         \eeq
         \beq
            \Mvar =\left(\barr{ccccc}
                -\sigma    	& 0 		& \sigma & 0    &  0 \\
	           0 	& -\sigma       & 0      & \sigma   &  0 \\
	           r_1-z  &     -r_2      & -1     & -e & -x_1 \\
	           r_2     & r_1-z       	& e  	& -1       & -x_2 \\
	           y_1     & y_2           & x_1    & x_2      & -b
                \earr\right)
        \eeq
        and the \SOn{2} rotation has
        \beq
            T = \left(\barr{ccccc}
                0 & 1 & 0 & 0 & 0\\
                -1& 0 & 0 & 0 & 0\\
                0 & 0 & 0 & 1 & 0\\
                0 & 0 & -1& 0 & 0\\
                0 & 0 & 0 & 0 & 0\\
            \earr\right)
        \eeq

        Plugging these into \refeq{SF:LieDeriv} does give you the wanted result, so the \cLe\ are rotational equivariant under infinitesimal \SOn{2} rotations.
\authorSF{Jun 14 2010}
    } % end \solution{exer:InfinRotInvari}

\exercise{\Un{1} equivariance of \cLe\ for finite angles:}{ \label{exer:FinRotInvarCmplx}
Show that \cLe\ \refeq{eq:CLe} are equivariant under finite angle rotation
\refeq{FiniteRot}.
    }

\solution{exer:FinRotInvarCmplx}{\Un{1} equivariance of \cLe\ for finite angles.}{
\authorSF{Jun ?? 2010}
    } % end \solution{exer:FinRotInvarCmplx}{\Un{1} equivariance

\exercise{\SOn{2} equivariance of \cLe\ for finite angles:}{ \label{exer:FinRotInvari}
Show that \cLe\ \refeq{eq:CLeR} are equivariant under rotation for finite angles.
    }

\solution{exer:FinRotInvari}{\SOn{2} equivariance of \cLe\ for finite angles.}{
\authorSF{Jun ?? 2010}
    }

\section{\Reqva}

\solution{exer:PlotRelEqu}{\Reqv\ $\REQB{1}$ in Cartesian coordinates.}{
\PC{Let me know if I have accidentally removed your solution here - it
can be easily recovered...}
\authorSF{Jun 4 2010}
}

\exercise{\Reqv\  $\REQB{1}$ in Cartesian coordinates:}{\label{exer:PlotRelEqu}
Plot the relative {\eqv} $\REQB{1}$ in Cartesian coordinates. State the velocity of relative {\eqv}, compare with the imaginary part of the complex stability eigenvalue, and explain the two time scales visible in the `horn', as well as the expansion rate per turn of the spiral.
}

\solution{exer:PlotRelEqu}{\Reqv\ $\REQB{1}$ in Cartesian coordinates.}{
\authorSF{Jun ?? 2010}
    } % end \solution{exer:PlotRelEqu}

\exercise{Eigenvalues and eigenvectors of $\REQB{1}$ \stabmat:}{\label{exer:EigenQ1}
Compute the eigenvalues and eigenvectors of the \stabmat\ evaluated at $\REQB{1}$ and using the Siminos parameter
values \refeq{SiminosPrmts}.
}

\solution{exer:EigenQ1}{Eigenvalues and eigenvectors of $\REQB{1}$ \stabmat.}{
Using the Mathematica function \texttt{Eigensytem} and
setting $(\rho_1, \rho_2, \theta, z)(0)$ to the values in
\refeq{eq:Q1} we obtain
\[
(\lambda_{1,2},\lambda_3,\lambda_4)
= (0.0938179 \pm 10.1945 i,-11.0009,-13.8534)
\]
as the eigenvalues of the system with the associated eigenvectors:
    \PC{Rebecca, please recheck my replacement of
     the complex eigenvectors by their Re, Im parts.}
\bea
\Re e_{1} &=& (0.266121, -0.0321133, 0.00034139, 0.719222)
\continue
\Im e_{1}  &=& (0.295017, 0.569063, 0.000551886,0)
\continue
e_3 &=& (-0.0883591, -0.0851485, -0.989135, -0.0809553)
\continue
e_4 &=& (-0.855586, -0.329912, -0.00273531, -0.398902)
\label{eigVecQ1}
\eea
The spiral-out instability appears very slow, but with a
short period $\period{spiral}=0.6163$, about 112 turns for one
period of the $\REQB{1}$. Actually, as the \reqv\ velocity
\refeq{eq:REQB1veloc} is slow, an initial deviation from
$\ssp_{\REQB{}1}$ is multiplied by the factor
$\ExpaEig_{radial}=535$, and the \reqv\ is quite unstable.
This is illustrated by \reffig{fig:CLERelEqui}\,(b). It would
be sweet if we could eliminate the detuning drift time scale
$\approx 70$ and focus just on the oscillatory time scale of
$\approx 0.6$. That is one of the motivations for
reformulating dynamics in a \reducedsp.
                \authorPC{Jun 26 2009, Aug 7 2009}
} % end \solution{exer:EigenQ1}{Finding the eigenvalues and eigenvec

\section{\Reducedsp}

\exercise{Probability of hitting $(x_1,x_2) =(0,0)$:}{
\label{exer:CLEsmall-x1x2}
Investigate by simulation (and - that is extra cost -
perhaps by thinking) how close does a strange attractor
trajectory of \cLf\ come to
hitting $(x_1,x_2) =(0,0)$ and/or  $(y_1,y_2) =(0,0)$?
%    \authorPC{Jul 14 2009}
    } %end\exercise{Probability of hitting $(x_1,x_2) =(0,0)$

\solution{exer:CLEsmall-x1x2}
{Probability of hitting $(x_1,x_2) =(0,0)$.}{
%
%%%%%%%%%%%%% PC generated by PCsimul.nb
\SFIG{ProblemsPill} %CLEsmall-x1x2}
{}{
A very long time ($t=5000$) simulation of \cLf. $\{x_1,x_2\}$ plot indicates
that probability of hitting $(x_1,x_2) =(0,0)$ is for all practical purposes
equal to zero, and $\sqrt{x_1^2+x_2^2} > 0.02$.
\authorPC{Jul 14 2009}
%(The initial point is on the strange attractor).
}
{fig:CLEsmall-x1x2}
%%%%%%%%%%%%%%%%%%%%%%%%%%%%%%%%%%%%%%%%%%%%%
%
A very long time ($t=5000$) simulation of \cLf, plot of
$\{x_1,x_2\}$  neighborhood in \reffig{fig:CLEsmall-x1x2},
indicates that probability of hitting $(x_1,x_2) =(0,0)$ is
for all practical purposes equal to zero, and
$\rho_1(t)=\sqrt{x_1^2+x_2^2} > 0.02$. Plots of $\{y_1,y_2\}$
neighborhood look similar.
Hence reduction of \statesp\ to 4-dimensional $x_2=0$ \reducedsp\ by rotating successive trajectory increments back
to the $x_1>0$ semi-axis should not run into an $x_1 =0$
singularity in determining the rotation angle $\cos\theta$.
\ES{I certainly agree with this and I think it is clear in
section 4.1.4.2 of the \emph{thesis no one wants to read} that
the singularity of the transformations does not pose a problem
for reduction in the case of \cLe\ as we do not encounter it
numerically (I've tried to pay the extra price and produce a proof
that the denominator cannot vanish but failed).
The reason that the singularity is annoying for me (beyond
visualization problems) is that in the \KS\ case dynamics goes
through zeros in the denominator.}
    \authorPC{Jul 14 2009}
    } %end \solution{Probability of hitting $(x_1,x_2) =(0,0)$.}{


\exercise{\Statesp\ reduction by a \slice, ODE formulation:}
{\label{exer:csectionCLeODE}
Reconsider \refeq{exer:csectionCLe} in the sequence of
infinitesimal time steps limit, each followed by an
infinitesimal rotation such that the next segment initial
point is in the {\slice} $x_2=0,\,x_1>0$. Derive the
corresponding 4$d$ \reducedsp\ ODE for the \cLf.
    } %end \solution{\Statesp\ reduction by a \slice, ODE formulation}

\solution{exer:csectionCLeODE}
         {\Statesp\ reduction by a \slice, ODE formulation:}{
Infinitesimal time version of the moving frames
symmetry reduction is attained by taking small time steps
in \reffig{fig:PCunrot} and dropping the higher order terms,
as in \refsect{sect:SymmDyn}:
\[
d\ssp^{(n)} = d t \, v(\ssp^{(n)})
                    + d \theta_1^{(n)} \, \Lg \cdot \ssp^{(n)}
\,.
\]
The infinitesimal angle is proportional to the time step,
\[
\theta_1^{(n)} \approx \sin \theta_1^{(n)}
             = - d t \, \frac{\hat{{\bf e}}_{1} \cdot v(\ssp^{(n)})}
                    {\rho_1^{(n)}}
             \approx - d t \, v_1(\ssp^{(n)})/\ssp_2^{(n)}
\,,
\]
where $(\rho_1,\theta_1)$ are polar coordinates,
$\rho_1 = \sqrt{\ssp_1^{ 2} + \ssp_2^{2}}$,
see \refeq{eq:CartToPol}. Our \slice\ condition is $x_1=0,\;x_2>0$,
so the \reducedsp\ equations are given by
\beq
\dot{\ssp} = v - \frac{v_1}{\ssp_2} \Lg \cdot \ssp
\,.
\ee{ExerMotionMovFrame}
The motion stays in the $(d\!-\!1)$-\slice, as $\dot{\ssp}_1 = 0$ due to
the orthogonal action of $\Lg$ to the direction $\ssp$.

Moving frames symmetry reduced \cLe\ are a 4-dimensional
first order ODE system
\beq
\begin{split}
	\dot{x}_2 &= -\sigma  (x_2 - y_2) \\
	\dot{y}_1 &= - y_1 + r_2 x_2 - \left(e + {\sigma y_1}/{x_2}\right) y_2\\
	\dot{y}_2 &=  - y_2 + (r_1-z) x_2 + \left(e + {\sigma y_1}/{x_2}\right) y_1 \\
	\dot{z}\; &= -b z + x_2 y_2
\,.
\end{split}
\ee{eq:rdcdCLeR}
The resulting trajectory is illustrated in
\reffig{fig:PCunrot1}. It agrees with trajectories reported
by Siminos (there the simulation is in the full \statesp, and
the \reducedsp\ dynamics is obtained by a coordinate change).
\ES{\refeq{eq:rdcdCLeR} follows from
\cLe\ expressed in the invariant coordinates obtained by the {\mslices}.
We have done the same
for ZM system long time ago when we heuristically
rederived Cartan's method. It has been moved to a footnote in Jonathan's
blog (eq. 5.37).
({\bf ES:} We've agreed to
junk ZM system when we realized it has no {\reqv}.
One also gets the same system by using invariant
polynomials and taking the syzygy into account, see discussion preceding (5.46) in
Jonathan's blog.
    }

\noindent
- Checked that it agrees with finite step + rotation of \reffig{fig:PCunrot}\\
- $x_2$ and $y_2$ seem locked, oscillate the same way with amplitude up to 20\\
- $y_1$ is very small, mostly below 0.1\\
- $( x_2 , y_2 ,z)$ plot looks discontinuous in $y_2$ whenever $x_2$ small,
across the $z$-axis (indicated in red).
\authorRW{Jul 28 2009}
    } %end \solution{\Statesp\ reduction by a \slice, ODE formulation}

\exercise{Accumulated phase shift in \slice\ \reducedsp:}
{\label{exer:csectionPhase}
Derive the 1$d$ equation for the accumulated phase shift
$\theta$ associated with the 4\dmn\ \reducedsp\ ODE of
\refexer{exer:csectionCLeODE}.
    } %end \solution{Accumulated phase shift in \slice\ reduced \states

\solution{exer:csectionPhase}{Accumulated phase shift in \slice\ \reducedsp:}{
(not available)
% \authorSF{Aug ?? 2009}
    } %end \solution{Accumulated phase shift in \slice\ reduced \states

\exercise{The moving frame flow stays in the \reducedsp:}
{\label{exer:csectionReduced}
Show that the flow \refeq{EqMotionMovFramePC} stays in a
$(d\!-\!1)$-dimensional \slice.
    } %end \exercise{The moving frame flow stays

\solution{exer:csectionReduced}
{The moving frame flow stays in the \reducedsp:}
{
\authorSF{Jun ?? 2010}
    } %end \solution{exer:csectionReduced}


    \PublicPrivate{}{

\exercise{\Statesp\ reduction by a locally transverse \slice:}
{\label{exer:PClocSectCLe}
Decompose $\vf(\ssp)$
in $\dot{\ssp}= \vf(\ssp)$ in a part $\vf_\shortparallel$ parallel
to the group action and a part $\vf_\perp$ transverse to it,
\beq
	\vf(\ssp)=\vf_\shortparallel(\ssp)+\vf_\perp(\ssp)\,,
\ee{flowSplit}
using the projection operator
\beq
	\vf_\perp(\ssp)=\PperpOp(\ssp) v(x)\,,\qquad
 	\PperpOp_{ij}(\ssp)=\delta_{ij}-
    \frac{\groupTan(\ssp)_i \groupTan(\ssp)_j}
         {\dotProd{\groupTan(\ssp)}{\groupTan(\ssp)}}
	\label{transvProj}
\eeq
that projects a $d$-dimensional flow $\vf(\ssp)$ onto
flow
\beq
	\dot{\ssp}_\perp = \vf_\perp(\ssp) = \vf(\ssp)
    - \frac{\groupTan(\ssp)\cdot\vf(\ssp)}
      {\dotProd{\groupTan(\ssp)}{\groupTan(\ssp)}} \groupTan(\ssp)
\ee{transvFlow}
in a $(d\!-\!1)$-dimensional {\slice} transverse to the
direction fixed by the point $\ssp$.
By ignoring
the flow component that can be compensated for by an
$\SOn{2}$ rotation we quotient the flow by $\SOn{2}$. The
idea is presumably very old; for example, it is stated as the
problem 1. of Sect. 6.2 of Arnol'd {\em Ordinary Differential
Equations}\rf{arnold92}. Note, however, that a choice of
$\ssp_0$ fixes only a direction, so the \reducedsp\ flow is still
equivariant under the action of discrete cyclic group $\Ztwo
= \{e,D(\pi)\}$ on $\ssp$, $\vf(\ssp)$ and the reference
point $\ssp_0$, just as was the case for the
Lorenz flow. % \refeq{Lorenz}.
    \PC{\emph{Mea culpa}: Here I screwed up. I forgot that rotation
    moves $v$ and counter-moves $\ssp$ in $v(\ssp)$, \ie,
    acts by the Lie derivative \refeq{inftmInv}. I could never
    understand why we do not see a translational zero eigenvalue
    everywhere (the Lie group acts globally and commutatively
    right?), but only on \eqva, \reqva\ and \rpo s. Presumably
    the projection operator \refeq{transvProj} is OK for the
    \reqv\ calculation, % of \refeq{sect:StabEq},
    as the action of the group on $\ssp_{\REQB{1}}$ is trivial?
    Not sure how to rewrite the decomposition induced by
    \refeq{transvProj} correctly, in
    terms of the full Lie derivative action, and not only the $\Lg$
    action.
    }

\refeq{transvFlow} differs from \refeq{EqMotionMovFramePC}
only by replacement of the \slice\ fixed by an arbitrary
point $\ssp^{*}$ by \slice\ fixed by the instantaneous
state itself, $\ssp^{*} \to \ssp$. It does not work for us;
\rpo s do not reduce to \po s, and the strange attractor remains
embedded in 4 dimensions. Explain what is wrong with \refeq{transvFlow}.
\authorPC{Aug 7 2008}
    } %end \exercise{\Statesp\ reduction by a locally transverse \slice:}

\solution{exer:PClocSectCLe}
         {\Statesp\ reduction by a locally transverse \slice:}{
(not available)
    } %end \solution{\Statesp\ reduction by a locally transverse \slice:}

\exercise{Velocity field within \slice\ can diverge:}{
\label{exer:PCsectionCLe}
Apparent lack of singularities in \reffig{fig:CLEpcSect} appears
fortuitous or perhaps even a programming error.
This is a 4\dmn\ subspace, and indeed simulations do encounter
this subspace very, with \reducedsp\ velocity going off to infinity.
For example, take \cLf, with
(a) $\ssp^* \approx  \ssp_{\REQB{}1} = (8.48,0.077,8.48,0,26.99)$
on \reqv\ \REQB{1} group orbit,
and initial point
$\ssp(0) \approx  (4.81,0.154,7.59,0.281,15.9)$
      (taken from a long run on the strange attractor). How
far in time can you integrate the \reducedsp\ trajectory?
This, however, is perhaps a bad example, as the initial point is
not on the \slice. (b) Rotate  $\ssp(0)$ into the slice, use that
point as the initial point. How far can you integrate?
(c) Could \reducedsp\ \cLf\ flow run into a singularity
for other choices of $\ssp^*$? (d) For what
\slice s is the strange attractor of \cLf\
singularity-free?
    } %end \exercise{Velocity field within \slice\ can diverge


\solution{exer:PCsectionCLe}
{Velocity field within \slice\ can diverge:}{
(a) The denominator
$(\ssp(t) \cdot\ssp^{*})_4$ vanishes at $t \approx 1.217\cdots$.
(b) Apparently forever.
(c) The \reducedsp\ \cLf\ flow runs into a singularities in the
\slice\ choice of \reffig{fig:PCunrot1}.
(d) We have no idea...
\authorPC{Aug 7 2009}
    } %end \solution Velocity field within \slice\ can diverge

\exercise{\Statesp\ reduction by a \reqv\ $\REQB{1}$ \slice:}
{\label{exer:PCsectionCLe}
Replace integration of the \cLe\ by a sequence of short time steps,
each followed by a rotation such that the next segment initial point
is in the \reqv\ $\REQB{1}$ {\slice}
\beq
(\hat{\ssp}- \ssp_{\REQB{}1}) \cdot t_{\REQB{}1}=0
    \,,\qquad
t_{\REQB{}1} = \Lg \cdot \ssp_{\REQB{}1}
\,,
\ee{PCsect}
where for any $\ssp$, $\hat{\ssp} = \LieEl(\theta) \cdot \ssp$
is the rotation that
lies in the \slice.
    } %end \exercise{\Statesp\ reduction by a \reqv\ $\REQB{1}$

\solution{exer:PCsectionCLe}
         {\Statesp\ reduction by a \reqv\ $\REQB{1}$ \slice.}{
We note that $\ssp_{\REQB{}1} \cdot t_{\REQB{}1}=0$ by the
antisymmetry of $\Lg$, so \refeq{PCsect} is a linear
condition $\hat{\ssp} \cdot \Lg \cdot \ssp_{\REQB{}1} = \ssp
\cdot \LieEl(\theta) \cdot \Lg \cdot \ssp_{\REQB{}1} =0$
that determines $\theta$. Substituting \refeq{SO2gener2d} and
\refeq{SO2generCLe} yields a formula
\beq
\tan \theta = \frac{x_1 x^{\REQB{}1}_2 -x_1 x^{\REQB{}1}_2
                    +y_1 y^{\REQB{}1}_2 -y_1 y^{\REQB{}1}_2}
                   {x_1 x^{\REQB{}1}_1 + x_2 x^{\REQB{}1}_2
                   +y_1 y^{\REQB{}1}_1 + y_2 y^{\REQB{}1}_2}
\ee{PCsectTan}
for the rotation angle $\hat{\ssp} = \LieEl(\theta) \cdot
\ssp$ (actually, there are two solutions, separated by $\pi$)
that rotates $\ssp$ into the \slice. In contrast to fixing one of
the polar angles as in the {\mslices} of
\refsect{sect:MovFrame}, this \slice\ introduces no
singularities, as $x_1^2 + x_2^2+y_1^2 + y_2^2>0$.
To compute $\sin \theta$, $\cos \theta$ needed by $\LieEl(\theta)$
rewrite \refeq{PCsectTan} as
\bea
\cos \theta &=& (x_1 x^{\REQB{}1}_1 + x_2 x^{\REQB{}1}_2
                   +y_1 y^{\REQB{}1}_1 + y_2 y^{\REQB{}1}_2)/N
    \continue
\sin \theta &=& (x_1 x^{\REQB{}1}_2 -x_1 x^{\REQB{}1}_2
                    +y_1 y^{\REQB{}1}_2 -y_1 y^{\REQB{}1}_2)/N
\,,
\label{PCsectCos}
\eea
with $N$ fixed by $sin^2+cos^2=1$.
\authorPC{Aug 7 2009}
    } %end \solution{\Statesp\ reduction by a \reqv\ $\REQB{1}$ \slice
    } %end \PublicPrivate


\exercise{Stability of a \reqv\ in the reduced \statesp:}{
\label{exer:Reducedstab}
Find an expression for the {\stabmat} of the system at a {\reqv} when a linear slice is used to reduce the symmetry of the flow.

    } % end \exercise{Stability in reduced \Statesp

\solution{exer:Reducedstab}{Stability of a \reqv\ in the reduced \statesp.}{
Let $\sliceTan{}$ be a vector normal to the plane of the slice. Then the dynamics within the slice are given by the equations \refeq{SF:sliceEas}.
This gives us
\beq
\begin{split}
    \frac{\partial u_i}{\partial y_j} =& \frac{\partial v_i\left(y\right)}{\partial y_j}-\frac{\partial \dot    \theta\left(y\right)}{\partial y_j} \left(T y\right)_i - \dot \theta\left(y\right)\frac{\partial \left(T y\right)_i}{\partial y_j}\\
    \frac{\partial \dot \theta\left(y\right)}{\partial y_j}
    =& \frac{\frac{\partial v}{\partial y_j}\cdot \sliceTan{}}{T y \cdot \sliceTan{}}-\frac{v\cdot \sliceTan{}}{\left(T y \cdot \sliceTan{}\right)^2}\frac{\partial T y}{\partial y_j}\\
    \frac{\partial \left(T y\right)_i}{\partial y_j} =& T_{ij}
\end{split}
\eeq
Now, using the fact that $\frac{\partial v_i}{\partial y_j} = A_{ij}$, where $A$ is the {\stabmat} of the flow in full \statesp,
and the {\reqv} condition \refeq{SF:reqvTangent}, we get the equation
    \PC{I do not see where the sum over group symmetry generators is in
    \refeq{SF:redJacobExer}?}
\beq
\MvarRed_{ij} = \Mvar_{ij}-\velRel \cdot \Lg_{ij} -(T y)_i\,(\frac{\frac{\partial v}{\partial y_i}\cdot \sliceTan{}}{T y \cdot \sliceTan{}} - c \frac{\frac{\partial (T y)}{\partial x_j}\cdot \sliceTan{}}{T y \cdot \sliceTan{}})
\ee{SF:redJacobExer}

\authorSF{Jun 21 2010}
    } %end \solution{Stability in reduced \Statesp

\exercise{Stability of a \rpo\ in the reduced \statesp:}{
\label{exer:ReducedstabRpo}
Find an expression for the {\jacobianM} (\monodromyM) of a {\rpo}
when a linear slice is used to reduce the dynamics of the flow.

    } % end \exercise{Stability in reduced \Statesp

\solution{exer:ReducedstabRpo}{Stability of a \rpo\ in the reduced \statesp.}{
\authorSF{Jun ?? 2010}
    } %end \solution{Stability of a \rpo\ in the reduced \statesp

