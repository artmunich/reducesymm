\svnkwsave{$RepoFile: froehlich/dailyBlog.tex $}
\svnidlong {$HeadURL: svn://zero.physics.gatech.edu/froehlich/blog/dailyBlog.tex $}
{$LastChangedDate: 2009-11-21 20:25:56 -0500 (Sat, 21 Nov 2009) $}
{$LastChangedRevision: 29 $} {$LastChangedBy: froehlich $}
\svnid{$Id: dailyBlog.tex 29 2009-11-22 01:25:56Z froehlich $}


\chapter{Research blog on symmetry reduction}
\label{chap:blog}

$\footnotemark\footnotetext{{\tt \svnkw{RepoFile}}, rev. \svnfilerev:
 last edit by \svnFullAuthor{\svnfileauthor},
 \svnfilemonth/\svnfileday/\svnfileyear}$

% J Hightower - former texas politician, author, speaker 1943-
\begin{bartlett}{
Only dead fish go with the flow}
\bauthor{
\HREF{http://www.brainyquote.com/quotes/authors/j/jim_hightower.html}
     {J Hightower}, Texas politician}
\end{bartlett}


\begin{description}
\item[2010/05/10 SF] Definition (9.9) of \emph{stabilizer}
in ChaosBook.org version13, \HREF{http://chaosbook.org/version13/chapters/discrete.pdf}
{Chapter 9 - World in a mirror} seems wrong.

\item[2010/05/10 PC] You are right - we have now
\HREF{http://www.flickr.com/photos/birdtracks/4259634492/}
{replaced ``stabilizer''} by $G_p$-\emph{symmetric} throughout the next chapter.
Please alert me to its occurrences in this chapter, suggest how to
fix them.

BTW, I referenced equation number in your remark with respect to
the stable ChaosBook.org version13 in order that it always links
 correctly: equation numbers etc. keep changing in the unstable,
 currently edited version.


\item[2010/05/11 SF] I started reading chapter 10 on my own.
I do not feel like I have a good grasp of Lie groups and algebras,
do you have any suggestions on what I should read to learn more
or is it not important for the research?.

\item[2010/05/12 PC] En attendant Godot (by the time KGB geniuses have
 your computer configured, the summer might be over), so I moved the
 computer that you were using on Monday to your desk for this summer.

As far as Lie groups and algebras are concerned, ignore them from now - lets
just understand what $SO(2) / U(1)$ invariance does to complex Lorenz equations.
Physicists usually learn $SOn{3}$ next and not the general theory of Lie groups, so lets
start small: invariance on a circle.

\item[2010/05/13 SF] I finished looking through chapters 9 and 10 and the paper. I am going to start looking at the problems from chapter 10 to make sure I understood the material. If any of the problems give me trouble or I'm not sure of the solution I'll post it here.

\item[2010-05-13 PC] That's good, but work within
siminos/froehlich/exerFlow.tex in your blog. Most of the exercises are
already there. Edit them as you see fit, add new ones and their solutions
 if you find that there are missing steps that you need to do in order
to simulate/analyze \cLe.

\item[2010-05-13 PC] You might want to join
\HREF{http://www.zotero.org/groups/cns}
{http://www.zotero.org/groups/cns}
in order to be able to access the papers we are saving there. Nothing
urgent, for now. Search for zotero in siminos/blog/blog.pdf to find
a bit more info about it.

\item[2010/05/24] I started thinking about the problem of when the group tangent at the point is perpendicular to the group tangent at the slice point. I think I understand what is happening but would like you to check my work to make sure I actually do.
    
    For the $x_1 = 0$ slice for the \cLe\ I think I can show that it is only possible for a trajectory to touch the slice instantaneously unless it is an equilibrium solution: Suppose a trajectory stays in the slice for some interval of time, this means $x_1 = \dot x_1 = 0$ in this interval, which gives us that $y_1 = 0$ during this time. So $y_1$ is constant for some interval of time, so $\dot y_1=0$ here also. Looking at $\dot y_1 = 0$ when $x_1=y_1 =0$ gives $x_2 = -(e/\rho_2)y_2$. Plugging these three restrictions into the equations for $\dot x_2$ and $\dot y_2$ yield the equations $\dot y_2=-\sigma y_2 (1+(\rho_2 / e))$ and $\dot y_2 = (\rho_1 - z)(-e/\rho_2)y_2-y_2$ this means that $-\sigma y_2 (1+(\rho_2 / e))=(\rho_1 - z)(-e/\rho_2)y_2-y_2$ provided $y_2$ is nonzero, then $-\sigma (1+(\rho_2 / e))=(\rho_1-z)(-e/\rho_2)-1$. z is the only variable in that equation, so it must be constant. This means that $\dot z = -b z + x_2 y_2 + x_1 y_1=-b z+ (-e/\rho_2)y_2^2=0$. Again $y_2$ is the only variable so it too must be constant, so $x_2$ is also constant. This means the solution can stay in the slice only if all its values are constant, so it is an equilibrium solution. There are possibly some divide by zero difficulties, but hopefully a similar argument will hold for these cases.
    
    So we do not run into this problem for the \cLf\.
    
    As I understand it, the reason the group tangent along the trajectory being perpendicular to the group tangent of the slice causes problem is because this means that doing an infinitesimal rotation at the point keeps it inside the slice so the point is not a unique representative (at least when doing the infinitesimal rotations), is this correct?
    
    When choosing the slice to be a hyperplane through a point, it doesn't have to be perpendicular to the group tangent at some point does it? As long as the group tangent at any point isn't contained inside the hyperplane there shouldn't be a problem locally.
    
    I thought about trying to show that this condition being true for an interval of time implies it is always true and I have no clue how to go about it so far. I think I see why it is true if the ODE is linear like in QM, but my argument is not quite complete.

=======
\end{description}
