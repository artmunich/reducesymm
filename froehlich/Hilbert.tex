%           %experimenting with svn-multi

\svnkwsave{$RepoFile: siminos/froehlich/Hilbert.tex $}
\svnidlong {$HeadURL$}
{$LastChangedDate$}
{$LastChangedRevision$} {$LastChangedBy$}
\svnid{$Id$}


\section{Constructing invariants of the symmetry}

When dealing with a system with symmetry, knowing invariants of the
symmetry group can be very useful\rf{OlverInv}. Reducing the symmetry
using a slice provides a simple means of calculating invariants of a
symmetry. The representatives in the slice are for entire group orbits,
so if you find a formula for the representative in terms of the initial
point, this equation is necessarily left invariant under the group
action. \refRef{FelsOlver98,FelsOlver99} provide a detailed explanation
of how this can be done efficiently for more general symmetry groups.

\example{Invariants for \cLe.}{
Consider the slice normal to the vector $(1,0,0,0,0)$ for the $\SOn{2}$
symmetry of the \cLe. Using equation \refeq{SL:CLEsliceRot} we find that
a point $(x_1,x_2,y_1,y_2,z)$ will be rotated to the point
\[
(0,\, r_1,\, \frac{x_2 y_1 - x_1 y_2}{r_1},\, \frac{x_1 y_1 + x_2 y_2}{r_1},\, z)
\]
where $r_1=\sqrt{x_1^2+x_2^2}$. The equations for each of these
coordinates are invariants of the flow. These differ only by a factor of
$r_1$ from the Hilbert basis \refeq{eq:ipLaser} of
\refref{SiminosThesis,DasBuch}.
}

	\ifarticle
	\else

\section{Hilbert polynomials}
\label{SF:relStab}

The polynomials \refeq{eq:ipLaser} form a Hilbert basis for the \cLe.
\beq
\begin{split}
    u_1 &= x_1^2+x_2^2 \cont
    u_2 &= y_1^2+y_2^2 \cont
    u_3 &= x_1 y_2-x_2 y_1\cont
    u_4 &= x_1 y_1+x_2 y_2\cont
    u_5 &= z\,.
    \label{eq:ipLaser}
\end{split}
\eeq
In terms of the polar coordinates for the \cLe, these polynomials are
	\PC{shouldn't the first two be $\rho_1^2,\rho_2^2 $?}
\beq
\begin{split}
    u_1 &= \rho_1 \cont
    u_2 &= \rho_2 \cont
    u_3 &= - \rho_1 \rho_2 \sin \theta \cont
    u_4 &= \rho_1 \rho_2 \cos \theta \cont
    u_5 &= z\,.
    \label{eq:hilPolar}
\end{split}
\eeq
The \cLe\ in the Hilbert basis are:
\beq
\begin{split}
  \dot{u}_1 &=2\,\sigma\,(u_4-u_1)\,,\\
  \dot{u}_2 &=-2(\,u_2 - r_2\, u_3 -\,(r_1-u_5)\,u_4)\,,\\
  \dot{u}_3 &=-(\sigma\, +1)\,u_3+r_2\, u_1+e\, u_4\,,\\
  \dot{u}_4 &=-(\sigma\, +1)\,u_4+\,(r_1-u_5)\,u_1+\sigma\, u_2-e\,u_3\,,\\
  \dot{u}_5 &=u_4-b\, u_5\,.
\end{split}
\label{eq:CLEip}
\eeq
The {\reqv} in polar coordinates is
\[
( \rho_1 , \rho_2 , \theta , z ) = (\sqrt{b (r_1 -d)},\sqrt{b d (r_1 -d)},\theta_Q, r_1 -d)
\]
where $d = 1+ \frac{e^2}{(\sigma +1 )^2}$ and $\theta_Q$ is the angle such that
\[
\cos \theta_Q = \sqrt \frac{1}{d}
    \,,\qquad
\sin \theta_Q = -\sqrt \frac{d-1}{d}
\,.
\]

\exercise{Hilbert basis singularities}{\label{exer:CLEipSyz}
% Predrag extracted from siminos/blog/CLEflotsam      Jun  5 2010
% also ChaosBook \example{Hilbert basis singularities}{\label{exam:CLEipSyz}
%
When one takes syzygies into account in rewriting a
dynamical system, singularities are introduced. For instance,
eliminate $u_2$ using the syzygy, and show that you get
the reduced set of equations,
	\PC{I removed
$  \dot{u}_2 = -2\left(\,\frac{u_3^2+u_4^2}{u_1} - \rho_2\, u_3
                -\,(\rho_1-u_5)\,u_4\right)
$
            }
\bea
  \dot{u}_1 &=& 2\,\sigma\,(u_4-u_1)
                \continue
  \dot{u}_3 &=& -(\sigma\, +1)\,u_3+\rho_2\, u_1+e\, u_4
                \continue
  \dot{u}_4 &=& -(\sigma\, +1)\,u_4+\,(\rho_1-u_5)\,u_1
                +\sigma\, {(u_3^2+u_4^2)}/{u_1}-e\,u_3
                \continue
  \dot{u}_5 &=& u_4-b\, u_5
\,,
\label{eq:CLEipSyz}
\eea
singular as $u_1\rightarrow 0$. (PC: check this - there might
be errors)
\authorES{}
    } %end \exer{Hilbert basis singularities}

The {\stabmat} in the Hilbert basis for the \cLe\ is
	\PC{This is [5$\times$5], but there are only 4 independent
            variables. How does that show up in the eigensystem of
	    {\stabmat}?}
\beq
\begin{split}
\mathbb{A}=
\left(
\begin{array}{ccccc}
-2 \sigma & 0 & 0 & 2 \sigma & 0\\
0 & -2 & 2 r_2 & 2 (r_1 - u_5) & -2 u_4\\
r_2 & 0 & -(\sigma +1 ) & e & 0\\
(r_1 -u_5 ) & \sigma & -e & -(\sigma +1 ) & -u_1\\
0 & 0 & 0 & 1 & -b
\end{array}
\right)
\end{split}
\eeq

    \fi %end of article switch
