% master file siminos/froehlich/slice/symm.tex
% $Author$ $Date$

%\section{Symmetries of dynamics}
%\label{sec:SymmDyn}

In this Appendix we review a few basic facts about dynamics and
symmetries. We follow notational conventions of
Chaosbook.org\rf{DasBuch}, to which the reader is referred to for a more
extensive discussion of dynamics and symmetries. The core of the
paper is \refsect{sec:frame}.

If a pipe is rotated around its axis or translated, the shifted and
rotated state of the fluid is also a physically equivalent solution of
the Navier-Stokes equations. Such rotations and translations of the pipe
are examples of continuous symmetries. On the level of equations of
motion, one says that a flow $\dot{x}= \vel(x)$ is \emph{equivariant}
under a coordinate transformation $\LieEl$ if
\beq
\vel(x)=\LieEl^{-1}\vel(\LieEl \, x)
\,.
\ee{eq:FiniteRot}
The totality of group elements
$\LieEl$ forms \Group, the {\em symmetry group} of the flow.
An element of a compact Lie group $\Group \subset \On{d}$ that is
continuously connected to the identity can be expressed as
\beq
\LieEl(\gSpace)=e^{{\gSpace} \cdot \Lg }
    \,,\qquad
\gSpace \cdot \Lg = \sum_{a=1}^N \gSpace_a \Lg_a
\,,
\ee{FiniteRot}
where $\gSpace \cdot \Lg $ is a \emph{Lie algebra} element, $\gSpace
= (\gSpace_1,\gSpace_2,\cdots\gSpace_N)$ are the parameters of the
transformation, and the $\Lg_a$ are a set of $N$ linearly independent
$[d\!\times\!d]$ antisymmetric matrices acting linearly on the {\statesp}
vectors. The group action parameters $\gSpace_a$ are sometimes referred to as
`phases,' `angles,' or `shifts.'
An infinitesimal group action is generated by
$
\LieEl(\delta \gSpace) \simeq 1 + \delta \gSpace \cdot \Lg
\,.
$ %\ee{eq:infinitesimal}
A corresponding spatial transformation induced by infinitesimal
variations of group `phases' $\delta \gSpace_a$ is
\beq
\delta {\ssp} = \delta \gSpace \cdot \groupTan(\ssp)
\,,
\ee{PC:groupTan0}
where the $N$ vectors
\beq
 \groupTan_{a}(\ssp) = \Lg _{a} \ssp
    \,,\qquad
 a=1,2,\cdots,N,
\ee{PC:groupTan}
span the group tangent space at $\ssp$. We use $\groupTan_a(\ssp)$
notation (rather than $\Lg_{a}\ssp$) to emphasize that the group action
induces a \emph{tangent field} at $\ssp$.
The {tangent field} is of dimension $N$, as long as the point $\ssp$ does
not belong to a fixed-point subspace.

%\noindent
%\textbf{Fixed-point subspace.}
%$\pS_H$ or a `centralizer' of a subgroup $H \subset \Group$,
%$\Group$ a symmetry of dynamics, is the set of all \statesp\
%points left \emph{$H$-fixed}, \emph{point-wise} invariant
%under action of the subgroup
%\beq
%\pS_H = \Fix{H} =
%   \{ \ssp \in \pS : {h} \, \ssp = \ssp \mbox{ for all } h \in H \}
%\,.
%\ee{dscr:FPsubsp}
Points in the \emph{fixed-point subspace}  $\pS_\Group$ are fixed
points of the full group action. They are called \emph{invariant
points},
\beq
\pS_\Group = \Fix{\Group} =
   \{ \ssp \in \pS : {g} \, \ssp = \ssp \mbox{ for all } g \in \Group \}
\,.
\ee{dscr:InvPoints}
If a point is an invariant point of the symmetry group,
by equivariance the velocity at that point is also
in $\pS_\Group$, so the trajectory through that point will remain in
$\pS_\Group$. $\pS_\Group$ is disjoint from the rest of the {\statesp}
since no trajectory can ever enter or leave it.

Any representation of a compact group $\Group$ is fully
reducible. The invariant tensors constructed by contractions
of $\Lg_a$ are useful in identifying irreducible
representations. The simplest such invariant is
\beq
{\Lg} \cdot \Lg = - \sum_m C_2^{(m)} \, \id^{(m)}
\,,
\ee{QuadCasimir}
where $C_2^{(m)}$ is the quadratic Casimir for irreducible representation
labeled $m$, and $\id^{(m)}$ is the identity on the $m$-irreducible
subspace, 0 elsewhere. For compact groups $C_2^{(m)}$ are strictly
nonnegative. $C_2^{(m)} =0$ if $m$ is an invariant subspace.

    %
% \subsection{\SOn{2} irreducible representations.}
%   \label{exam:SO2irrepst}
    %
The simplest example of a Lie group is given by the action of \SOn{2} on
a smooth function $u(\gSpace + 2\pi) = u(\gSpace)$ periodic on interval
$[-\pi,\pi]$. Expand $u$ as a Fourier series
\beq
u(\gSpace) = \frac{a_0}{2} + \sum_{m=1}^\infty \left(
a_m \cos m \gSpace + b_m \sin m \gSpace
                               \right)
\,.
\ee{FourierExp}
The matrix representation of the \SOn{2}\ action
$\LieEl(\gSpace') u(\gSpace) = u(\gSpace+\gSpace')$
on the Fourier coefficient pair
$(a_m,b_m)$ is
\bea
\LieEl^{(m)}(\gSpace')
    &=& \exp{\left({\gSpace} \cdot \Lg^{(m)}\right)}
	\,=\,
   \left(\barr{cc}
 ~\cos m \gSpace'  & \sin m \gSpace' \\
 -\sin m \gSpace'  & \cos m \gSpace'
    \earr\right)
\,=\, \cos m \gSpace' \id^{(m)}
  + \sin m \gSpace'\, \frac{1}{m} \Lg^{(m)}
\,,
\label{SO2irrepAlg-m}
\eea
Here
\beq
 \Lg^{(m)} =   \left(\barr{cc}
    0  &  m  \\
   -m  &  0
    \earr\right)
\,.
\label{SO2irrepAlg-Lg}
\eeq
is the Lie algebra generator and $\id^{(m)}$ is the identity
on the irreducible subspace labeled $m$, 0 elsewhere. The \SOn{2}\ group
tangent $\groupTan(u)$ to \statesp\ point $u$ is
\beq
 \groupTan(u) = \sum_{m=1}^\infty \groupTan^{(m)}(u)
    \,,\qquad
 \groupTan^{(m)}(u)
\,=\, m \,\left(\barr{c}
   ~b_m  \\
   -a_m
    \earr\right)
\,.
\ee{u:x:tang}

%
% ****** End of file symm.tex ******
