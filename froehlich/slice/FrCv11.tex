% master file siminos/froehlich/slice/FrCv11.tex    pdflatex FrCv11
% $Author$ $Date$

                        %% logical setup, no need to edit %%%%%%%%%%
                        \newif\ifpaper \newif\ifPDF               %%
                        \newif\ifboyscout  \newif\ifarticle       %%
                        \boyscouttrue %% commented, WWW/boyscouts %%
                        \articletrue 							  %%
                        \paperfalse\PDFtrue %% hyperlinked    %%%%%%
    % Toggle between draft and non-draft versions
%\boyscoutfalse                 % public, for hyperlinked ChaosBook/projects
%\papertrue\boyscoutfalse     % article for submission

%%   based on edits of aiptemplate.tex AIP REVTeX4 Ver. 4.1, 9 Oct 2009.
% Predrag, created the first draft				2010-11-21

\listfiles
\documentclass[%
 reprint,%					2-column
%secnumarabic,%
 amssymb, amsmath,% ,amsfonts
 aip,cha,% 					Chaos journal
 graphicx
%groupedaddress,%
%frontmatterverbose,
]{revtex4-1}
%\documentclass[aip,cha,graphicx]{revtex4-1}
%\documentclass[aip,reprint]{revtex4-1}

%\draft %obsolete, invoke option instead
% marks overfull lines with a black rule on the right

\usepackage[pdftex]{graphicx}
\usepackage[pdftex,colorlinks]{hyperref}
%\usepackage{array}
\input ../../inputs/editsDasbuch   %% editing comments, DasBuch style
\input def            %% edited, initially from dasbuch/book/inputs/def.tex
\input ../../inputs/defsFroehlich     %% all Stefan edits: \renewcommand, etc
\graphicspath{{../../figs/}{../../Fig/}}  %% directories with color graphics files
\hypersetup{
   pdfauthor=Stefan Froehlich and Predrag Cvitanovic,
   pdfkeywords=complex Lorenz flow,
   pdftitle=Reducing continuous symmetries}

\begin{document}
\title{Reduction of continuous symmetries of chaotic flows
       by the method of slices}
% earlier titles:
% "Reduction of continuous symmetries by the method of slices"

% repeat the \author .. \affiliation  etc. as needed
% \email, \thanks, \homepage, \altaffiliation all apply to the current author.
% Explanatory text should go in the []'s,
% actual e-mail address or url should go in the {}'s for \email and \homepage.
% Please use the appropriate macro for the type of information

% \affiliation command applies to all authors since the last \affiliation command.
% The \affiliation command should follow the other information.

\author{Stefan Froehlich}
%\email[]{Your e-mail address}
%\thanks{}
%\altaffiliation{}
%\affiliation{}

\author{Predrag Cvitanovi\'{c}}
\email[]{predrag@gatech.edu}
%\homepage[]{Your web page}
%\thanks{}
\affiliation{Center for Nonlinear Science,
        School of Physics, Georgia Institute of Technology,
        Atlanta, GA 30332-0430}

\date{\today}

\begin{abstract}
    \input abstract
\end{abstract}

\pacs{
02.20.-a, 05.45.-a, 05.45.Jn, 47.27.ed
% 02.20.-a  Group theory, mathematics
% 05.45.-a 	Nonlinear dynamics and chaos
% 05.45.Jn 	High-dimensional chaos
% 47.10.Fg 	Dynamical systems methods (in Fluid Mechanics)
% 47.27.ed 	Dynamical systems approaches (turbulent flows)
% 47.52.+j 	Chaos in fluid dynamics
	}

\maketitle %must follow title, authors, abstract and \pacs

% If in two-column mode, this environment will change to single-column format so that long equations can be displayed.
% Use only when necessary.
%\begin{widetext}
%$$\mbox{put long equation here}$$
%\end{widetext}

% An example of the general form of a table:
% Fill in the caption in the braces of the \caption{} command. Put the label
% that you will use with \ref{} command in the braces of the \label{} command.
% Insert the column specifiers (l, r, c, d, etc.) in the empty braces of the
% \begin{tabular}{} command.
%
% \begin{table}
% \caption{\label{} }
% \begin{tabular}{}
% \end{tabular}
% \end{table}

\section{Introduction}
    \label{sec:intro}
    \input intro

\section{\Mframes}
    \label{sec:frame}
    \input frame

\section{Dynamics in the slice}
    \label{sec:mslices}
    \input slice

\section{Traversing a slice {\sset}}
	\label{sect:singul}
    \input singul

\section{Charting the \reducedsp}
	\label{sec:chart}
    \input  chart

\section{What lies ahead} % Conclusion}
    \label{sec:concl}
    \input  concl

\begin{acknowledgments}
We sought in vain Phil Morrison's sage counsel on how
to reduce symmetries, but none was forthcoming - hence this article.
We are, however, grateful to
D.~Barkley,
W.-J.~Beyn,
C.~Chandre,
K.A.~Mitchell,
B.~Sandstede,
R.~Wilczak,
and in particular E.~Siminos and R.L.~Davidchack
for many spirited exchanges.
S.F. work was supported by the National Science Foundation
grant DMR~0820054 and a Georgia Tech President's Undergraduate
Research Award.
P.C. thanks Glen Robinson Jr. for support. 	
\end{acknowledgments}

% Create the reference section using BibTeX:
\bibliography{../../bibtex/siminos}

\PublicPrivate{}{
\newpage
\input flotsam
	} % end Private

\end{document}
%
% ****** End of file aiptemplate.tex ******
