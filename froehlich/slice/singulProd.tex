% file siminos/froehlich/slice/singulProd.tex
% $Author$ $Date$

%\section{Singularities of $\SOn{2} \times \SOn{2}$}
%\label{sec:singulProd}

Two groups \Group\ and $H$ can be combined into the {product group}
$\Group \times H$, whose elements are pairs $(\LieEl,h)$, where $\LieEl$
belongs to \Group, and $h$ belongs to $H$, with the group multiplication
rule
\[
(\LieEl_1,h_1)(\LieEl_2,h_2)=(\LieEl_1 \LieEl_2,h_1 h_2)
\,.
\]
Some important fluid-dynamical flows exhibit continuous symmetries which
are the products of $\SOn{2}$ groups, each of which acts on a subset  of
the {\statesp} coordinates. The \KS\ equations\rf{ku,siv},
{\pCf}\rf{Visw07b,GHCW07,HGC08,HalcrowThesis}, and pipe
flow\rf{Wk04,Kerswell05} all have continuous symmetries of this form.

Let $\Group \times H$ be a Lie group with two sets of infinitesimal
generators, $\Lg_1$ and $\Lg_2$, such that the $\Lg_1$ acts only on the
$(a)$ coordinates ($e^{\gSpace_1 \Lg_1} \, (a,b)=(a',b)$), or,
infinitesimally, $\Lg_1(a,b)=(\Lg_1 a,0)$, and and $\Lg_2$ acts only on
the $(b)$ coordinates, $\Lg_2(a,b)=(0,\Lg_2 b)$. Taken together, $\Lg_1
\Lg_2(a,b) = \Lg_2 \Lg_1(a,b) = (0,0) $ for all $(a,b)$, so $\Lg_1
\Lg_2=0$.

For simplicity, we now specialize to the  $\SOn{2} \times \SOn{2}$ case.
Suppose we are rotating a trajectory $\ssp(\tau)$ into the slice normal
to the group tangents at $\slicep$. Using the slice condition
\refeq{eq:slicecondition} yields
$\braket{\vel(\sspRed)}{\sliceTan{1}}-\dot{\gSpace_1} \braket{
\groupTan_1(\sspRed)}{\sliceTan{1}}-\dot{\gSpace_2} \braket{
\groupTan_2(\sspRed)}{\sliceTan{1}}=0$. We have
$\braket{\groupTan_2(\sspRed)}{\sliceTan{1}}=0$ since $\Lg_1 \Lg_2=0$,
leaving us with the equation for $\dot{\gSpace_1}$,
\beq
\dot{\gSpace_1}=     {\braket{\vel(\sspRed)}{\sliceTan{1}}} /
                     {\braket{\groupTan_1(\sspRed)}{\sliceTan{1}}}
\,,
\eeq
and similarly for $\dot{\gSpace_2}$, the same as \refeq{eq:so2reduced}
for the rotation group consisting of only the rotations generated by
either $\Lg_1$ or $\Lg_2$.  This means a point being singular depends
only on whether or not it is singular in either of the slices normal to
only one of the group tangents, breaking up the problem of determining if
a point is singular into the same problem for each of the $\SOn{2}$
separately. In \refsect{sec:singul} we described what happens to the
\reducedsp\ trajectory as it passes through singularity of a single
$\SOn{2}$ symmetry group. Using this result we can handle the
singularities for the product of arbitrarily many $\SOn{2}$.

%
% ****** End of file singulProd.tex ******
