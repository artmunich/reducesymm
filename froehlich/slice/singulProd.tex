% file siminos/froehlich/slice/singulProd.tex
% $Author$ $Date$

%\subsection{Singularities of $\SOn{2} \times \SOn{2}$}
%\label{sec:singulProd}

Any two groups \Group\ and $H$ can be combined into
the {\em product group} $\Group \times H$, whose
elements are pairs $(\LieEl,h)$, where $\LieEl$ belongs to \Group, and
$h$ belongs to $H$, with the group multiplication rule
\[
(\LieEl_1,h_1)(\LieEl_2,h_2)=(\LieEl_1 \LieEl_2,h_1 h_2)
\,.
\]
Some important fluid-dynamical flows exhibit continuous symmetries
which are the products of $\SOn{2}$ groups, each of which acts on
a subset  of the {\statesp} coordinates. The \KS\ equations\rf{ku,siv},
{\pCf}\rf{Visw07b,GHCW07,HGC08,HalcrowThesis}, and pipe
flow\rf{Wk04,Kerswell05} all have continuous symmetries of this form.

Let $\Group \times H$ be a Lie group
with two sets of infinitesimal generators, $\Lg_1$ and $\Lg_2$, such that the
$e^{\gSpace_1 \Lg_1}$ acts only on the $(a)$
coordinates ($e^{\gSpace_1 \Lg_1} \, (a,b)=(a',b)$) and
$e^{\gSpace_2 \Lg_2}$ acts only on the $(b)$
coordinates.

We begin by looking at the action of infinitesimal rotations of $\Lg_1$.
$e^{\delta \gSpace_1 \Lg_1}(a,b)=(1+\delta \gSpace_1 \Lg_1)
(a,b)$. $e^{\delta \gSpace \Lg_1}$ fixes the
$\mathbb{B}$ coordinates, so $e^{\delta \gSpace \Lg_1}=(a',b)$ for some
$a'$.
	\PC{{\bf to Stefan}:
	While `fixes' is a correctly used math lingo, I tend to avoid it
	as it violets common usage; $\Lg_1$ does not (transitively) `fix' anything,
	it simply does not act upon $b$ coordinates.
   }

This results in $(a',b)=(a,b)+\delta \gSpace_1 \Lg_1(a,b)$ so
$\delta \gSpace_1 \Lg_1(a,b)=(a'-a,0)$. This is true for any $\delta
\gSpace_1$, so $\Lg_1$ maps the $\mathbb{B}$ coordinates to 0. The same
argument gives us that $\Lg_2$ maps the $\mathbb{A}$ coordinates to 0.
Looking at $\Lg_1 \Lg_2$ we find $\Lg_1 \Lg_2 (a,b)=\Lg_1 (0,b')=(0,0)$
for any $(a,b) \in \pS$, which is only possible if $\Lg_1
\Lg_2=0$. The same argument tells us that $\Lg_2 \Lg_1=0$ %(this also
follows from $\Lg_1 \Lg_2=0$ and the fact that the $\Lg_i$ are
antisymmetric).

We can now use $\Lg_1 \Lg_2=\Lg_2 \Lg_1=0$ to describe the
{\sset s} of product symmetry groups. For simplicity,
we specialize to the  $\SOn{2} \times \SOn{2}$ case.

Suppose we are rotating a trajectory $\ssp(\tau)$ into the slice normal
to the group tangents at $\slicep$. Using
$\sliceTan{a}=\groupTan_1(\slicep)$ for equation
\refeq{eq:slicecondition} tells us that
$\braket{\vel(\sspRed)}{\groupTan_1(\slicep)}-\dot{\gSpace_1} \braket{
\groupTan_1(\sspRed)}{\groupTan_1(\slicep)}-\dot{\gSpace_2} \braket{
\groupTan_2(\sspRed)}{\groupTan_1(\slicep)}=0$.
We have $\braket{\groupTan_2(\sspRed)}{\groupTan_1(\slicep)}=0$ since
$\Lg_1 \Lg_2=0$, leaving us with
$\braket{\vel(\sspRed)}{\groupTan_1(\slicep)}
-\dot{\gSpace_1}\braket{\groupTan_1(\sspRed)}{\groupTan_1(\slicep)}=0$.
This gives us the equation for $\dot{\gSpace_1}$,
\beq
\dot{\gSpace_1}=\frac{\braket{\vel(\sspRed)}{\groupTan_1(\slicep)}}
                     {\braket{\groupTan_1(\sspRed)}{\groupTan_1(\slicep)}}.
\eeq
This is the same as \refeq{eq:so2reduced} for the rotation group
consisting of only the rotations generated by $\Lg_1$. This equation is
independent of the action of $\Lg_2$. This means a point being singular
depends only on whether or not it is singular in either of the slices
normal to only one of the group tangents. This permits us to break up the
problem of determining if a point is singular and how this affects the
\reducedsp\ trajectory for the entire group into the same problem for
each of the $\SOn{2}$ groups generated by one of the infinitesimal
generators individually, a situation we are comfortable with.

In \refsect{sec:sliceSing} we found a simple
description for what happens to the \reducedsp\ trajectory as it passes
through singularity of a single $\SOn{2}$ symmetry group (it is rotated
by a finite amount), so using this result we can handle the singularities
for the product of arbitrarily many $\SOn{2}$.

%
% ****** End of file singulProd.tex ******
