% file siminos/froehlich/slice/slice.tex
% $Author$ $Date$

% \section{Dynamics in the slice}
% \section{\Mslices} % OLD
%       \label{sec:mslices}

Any \statesp\ trajectory can be written in a factorized
form $\ssp(\tau)=\LieEl(\tau)
\sspRed(\tau)$. Differentiating both sides with respect to time and
setting $\velRed={d\sspRed}/{d\tau}$ we find
\(
\vel(\ssp)=\dot{\LieEl} \, \sspRed+\LieEl \, \velRed(\sspRed)
\,.
\)
By the equivariance \refeq{eq:FiniteRot}
\[
\vel=\velRed + \LieEl^{-1} \, \dot{\LieEl} \, \sspRed
\,.
\]
Noting that $\LieEl^{-1}\dot{\LieEl}=e^{-\gSpace \cdot \Lg} \,
\frac{d ~~}{d \, \tau} e^{\gSpace \cdot \Lg}=\dot{\gSpace}\cdot \Lg$,
we obtain the equation for the velocity of the reduced flow:
\beq
\velRed(\sspRed)=\vel(\sspRed)-\dot{\gSpace}(\sspRed)\cdot \groupTan(\sspRed)
\,.
\ee{eq:redVel}
A transformation induced by infinitesimal
time-dependent variations \refeq{PC:groupTan0} of group `phases'
$\delta \gSpace_a = \timeStep \, \dot{\gSpace_a}$ is
\beq
\dot{\ssp} = \dot{\gSpace} \cdot \groupTan(\ssp)
\,.
\ee{PC:groupTan1}
So $\dot{\gSpace} \cdot \groupTan(\ssp)$ is the velocity
of the flow along the group orbit of \ssp, and
the velocity $\vel$ in the full \statesp\ is the sum of
the velocity normal along the group tangent and the remainder $\velRed$.

This equation is true for any factorization $\ssp(\tau)=\LieEl(\tau)
\sspRed(\tau)$, and by itself provides no information on how to calculate
$\dot{\gSpace}$.
That is attained by demanding that the reduced trajectory
stays within a slice.
Let $\sliceTan{a}$ be the group tangents at the slice
fixing point. Requirement \refeq{PCsectQ0} that the flow is confined to
the slice yields
\beq
\braket{\vel(\sspRed)}{\sliceTan{a}}
 -\braket{\dot{\gSpace}\cdot \groupTan(\sspRed)}{\sliceTan{a}}=0
\,.
\label{eq:slicecondition}
\eeq
% for each group tangent $\sliceTan{a}$ at the slice fixing point.
In principle\rf{FiSaScWu96}, this is a matrix equation in
$\braket{\groupTan_b(\sspRed)}{\sliceTan{a}}$ that one can solve
for $\dot{\gSpace}_a$. Here we shall consider only the
$\SOn{2}$ case, which has a single group tangent:
\bea
\velRed(\sspRed) &=& \vel(\sspRed)
   -\dot{\gSpace}(\sspRed) \groupTan(\sspRed)
\continue
\dot{\gSpace}(\sspRed) &=& {\braket{\vel(\sspRed)}{\sliceTan{}}}/
               {\braket{\groupTan(\sspRed)}{\sliceTan{}}}
\,.
\label{eq:so2reduced}
\eea
The first equation defines the flow confined to the slice
(see \reffig{fig:Fullspace}\,(b)), and
integration of the second, `reconstruction'
equation\rf{Marsd92,MarsdRat94} enables us to track the
corresponding trajectory in the full \statesp, so no information
is lost about the physical flow: if we know one point on the
trajectory, we can hop at will back and forth between the
reduced and the full \statesp\ trajectories.
%\item[2010-12-06 PC] in Siminos blog
Perhaps the insightful way to think about reduction of a flow to a slice
is in terms of Lagrange multipliers (see {Stone and Goldbart}\rf{StGo09},
Sect 1.5 for intuitive, geometrical interpretation of Lagrange multipliers).


%
% ****** End of file slice.tex ******
