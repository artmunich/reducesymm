% file siminos/froehlich/slice/frame.tex
% $Author$ $Date$

% \section{\Mframes}
%       \label{sec:frame}

Suppose you are observing turbulence in a pipe flow, or your
defibrillator has a mesh of sensors measuring electrical currents that
cross your heart, or you have a precomputed pattern, and are sifting
through the data set of observed patterns for something like it. Here you
see a pattern, and there you see a pattern that seems much like the first
one. How `much like the first one?' Think of the first pattern
(represented by a point {\slicep} in the \statesp\  \pS) as a
`template'\rf{rowley_reconstruction_2000,rowley_reduction_2003} or a
`reference state' and use the symmetries of the flow to slide and rotate
the `{\template}' until it overlays the second pattern (a point $\ssp$ in
the \statesp), \ie, act with elements of the symmetry group \Group\ on
the {\template} ${\slicep} \to {\LieEl}{\slicep}$ until the distance between
the two patterns
\beq
|\ssp - {\LieEl}{\slicep}|
    = |\sspRed - \slicep|
\label{minDistance0}
\eeq
is minimized. Here $\sspRed$ is the point on the group orbit of $\ssp$
(the set of all points that $\ssp$ is mapped to under the groups
actions),
\beq
\ssp=\LieEl \sspRed
	\,,\qquad
\LieEl \in \Group
\,,
\ee{sspOrbit}
closest to the {\template} {\slicep}.
This distance needs to be an invariant of the symmetry group: here
we shall assume \Group\ is a subgroup of the group of Euclidean
transformations, and measure distance
$|\ssp|^2=\braket{\ssp}{\ssp}$ in terms of the Euclidean inner product
\( %beq
\braket{x}{y} = \sum_i^d {x}_i y_i
%    \,,\; %\qquad
% x, y \in \pS \subset \reals^d
	\,,
\) %\ee{innerR}
or, if the \statesp\ is a normed function space,
\( %\beq
\braket{g}{f} = \int dx \, {g}(x) f(x)
\,,
\) %\ee{innerL2}
in terms of the $L^2$ norm $|f|^2 = \braket{f}{f}$. In practice, we
represent such spatially extended functions by discrete meshes or finite
basis sets, within a (possibly large) finite-dimensional \statesp\  $\pS
\subset \reals^d$. An example is a representation of a dissipative PDE by
truncating the Fourier basis \refeq{FourierExp} to a finite number of
modes.

If we parameterize a Lie group element $\LieEl=\LieEl(\gSpace)$ by
parameters $\gSpace = (\gSpace_1,\gSpace_2,\cdots\gSpace_N)$, the minimal
distance satisfies the extremum conditions
\beq
\frac{\partial ~~}{\partial \gSpace_a} |\ssp - \LieEl\slicep|^2
   =
\braket{\sspRed - \slicep}{\sliceTan{a}}
   = 0
    \,,\qquad
	  \sliceTan{a} = \Lg_a \slicep
\,.
\label{PCsectQ}
\eeq
Thus the minimum distance condition, combined with the Euclidean norm, says
that the point $\sspRed$ in the group orbit of \statesp\ point $\ssp$
closest to the {\template} $\slicep$ lies in a $(d\!-\!N)$\dmn\ hyperplane
through the point $\slicep$, normal to the group action tangent space
$\sliceTan{}$. In what follows we shall refer to this hyperplane as a
\emph{slice.} Slice so defined is a particular case of symmetry reduction
by transverse sections of group
orbits\rf{FelsOlver98,FelsOlver99,OlverInv} that can be traced back to
Cartan's \mframes\rf{CartanMF}.

As \Group\ is a symmetry of the system and the length $|\ssp|$ is
symmetry invariant under symmetry operations, $\Group$ is a subgroup of
the group of orthogonal transformations $\On{d}$, and the Lie algebra
{generators} $\Lg_a$ are a set of $N$ linearly independent
$[d\!\times\!d]$ antisymmetric matrices acting linearly on the {\statesp}
vectors $\ssp \in \pS \subset \reals^d$. By the antisymmetry of the Lie
algebra generators we have $\braket{\slicep}{\sliceTan{a}} =
\braket{\slicep}{\Lg_{a}\slicep}=0$ in \refeq{PCsectQ}, so the
transformation parameters $\gSpace$ for which the state $\ssp$ is closest
to the {\template} $\slicep$ are fixed by $N$ \emph{slice conditions} determined
solely by the group tangent vectors $\sliceTan{a}$,
\beq
\braket{\sspRed}{\sliceTan{a}} =0
    \,,\qquad
\sspRed = \LieEl(\gSpace) \ssp
\,.
\ee{PCsectQ0}
In the choice of $\slicep$ one should avoid solutions that belong to the
invariant or partially symmetric subspaces; for such choices some or all
$\sliceTan{a}=0$, and impose no slice conditions. The {\template}
$\slicep$ should be a generic \statesp\ point in the sense that its group
orbit has the full $N$-dimensions of the group \Group.
	\PC{comment that this is regrettable, and the reason why
	Sandstede and Fields(?) have to stand on their
	heads while bifurcating
	}

\subsection{Computing the moving frame rotation angle}
\label{exam:CLErotAngle}

The basic idea behind symmetry reduction is to define an equivalence
relation on the \statesp\ where two points are equivalent if they are in
the same group orbit. We can associate with any trajectory in the full
\statesp\, $\ssp(\tau)$ a `\reducedsp' trajectory, $\bar{\sspRed}(\tau)$,
which is the equivalence class of the full trajectory at any time (\ie\
$\ssp(\tau) \in \bar{\sspRed}(\tau)$ for each $\tau$). If a unique
representative, denoted $\sspRed$, is chosen from each equivalence class,
then the full space trajectory is recovered given $\LieEl(\tau)$, the
group action that rotates the equivalence class representative point
$\sspRed(\tau)$  to $\ssp(\tau)$ (\ie\ $\ssp(\tau)=\LieEl(\tau)
\sspRed(\tau)$). We show here how $\LieEl(\tau)$ is computed for a given
slice.

To show how the rotation into the \slice\ is computed, consider first the
\cLe. The \reducedsp\ point $\sspRed$ is given by
$\sspRed=\LieEl(\gSpace) \ssp$ where $\gSpace$ is such that
$\braket{\sspRed}{\sliceTan{}}=0$. Substituting the \SOn{2}\ Lie algebra
generator and a finite angle \SOn{2} rotation \refeq{CLfRots} acting on a
5\dmn\ space \refeq{eq:CLeR} into the slice condition \refeq{PCsectQ0}
yields the explicit formula for $\gSpace$:
\bea
0 &=&
    \braket{\ssp}{\sliceTan{}}\cos\gSpace
    +\braket{\groupTan_{}(\ssp)}{\sliceTan{}} \sin\gSpace
\label{SL:CLEsliceRot0}\\
\tan\gSpace &=&
   - \, {\braket{\ssp}{\sliceTan{}}}/
          {\braket{\groupTan_{}(\ssp)}{\sliceTan{}}}
\,.
\label{SL:CLEsliceRot}
\eea
The dot product of two tangent fields in \refeq{SL:CLEsliceRot} is a
sum of inner products weighted by Casimirs \refeq{QuadCasimir},
\beq
\braket{\groupTan(\sspRed)}{\groupTan(\slicep)}
   = \sum_m C_2^{(m)} {\sspRed}_i\, \delta_{ij}^{(m)} \slicep_j
\,.
\ee{braket}
For \cLe\
$\ssp = (x_1,x_2,y_1,y_2,z)$,
$\sspRed = (x_1',x_2',y_1',y_2',z)$,
and the moving frame condition \refeq{SL:CLEsliceRot} yields
\[
\tan\gSpace =
- \, \frac{x_1 x_2'-x_2 x_1'+y_1 y_2' -y_2 y_1'}
       {x_1 x'_1+x_2 x'_2+y_1 y'_1+y_2 y'_2}
\,.
\]
% PC 2010-12-16 - just pick the infimum.
Note that if \gSpace\ is a solution, so is $\gSpace+\pi$.
This formula is particularly simple, as in the \cLe\
example the group acts only through $m=0$ and $m=1$ representations.

%\noindent
%{\bf $\SOn{2}$ singularities.}
%\label{ex:so2singularities}
Consider next the general form \refeq{SO2irrepAlg-m} of action
of an $\SOn{2}$ symmetry on arbitrary Fourier coefficients of a spatially periodic
function \refeq{FourierExp}.
Substituting this into the slice condition \refeq{PCsectQ0} and using
\refeq{SO2irrepAlg-m} we find that
    \PC{RECHECK! and prettify this formula}
\bea
\braket{e^{\gSpace \Lg}\ssp}{\groupTan(\slicep)}
=\braket{\ssp}{\left(\sum\limits_m (\cos(-m\gSpace) \id^{(m)}
     +\sin(-m\gSpace) \frac{1}{m}\Lg^{(m)})\right) \sliceTan{}}
\continue
=\sum\limits_m
   \left(
    \braket{\ssp}{\Lg^{(m)} \slicep} \cos(m\gSpace)
  - \braket{\ssp}{\id^{(m)} \slicep} \sin(m\gSpace)
   \right)
   =0
\,.
\label{eq:so2sing}
\eea
Rewriting $\cos(m\gSpace)$, $\sin(m\gSpace)$ as
polynomials of degree $m$ in $\sin(\gSpace)$ and $\cos(\gSpace)$, we can recast
\refeq{eq:so2sing} as a polynomial equation, with coefficients
determined by $\braket{\ssp}{\Lg^{(m)} \slicep}$ and
$\braket{\ssp}{\id^{(m)}\slicep}$. The `phase' $\gSpace$ that rotates
$\ssp$ into any group-orbit traversal of the slice corresponds to a root of
this polynomial. In general these phases have to be computed
numerically.

As a generic group orbit is a smooth curved $N$\dmn\ manifold embedded in
the $d$\dmn\ \statesp, several values of $\gSpace$ might be local extrema
of the distance function \refeq{PCsectQ}. For example, group orbits of
\SOn{2}\ are topologically circles, and the distance function
\refeq{minDistance0} has maxima, minima and inflection points as extrema.
Our prescription is to pick the closest \reducedsp\ point as the unique
representative of the entire group orbit. \ie, the absolute minimum or
the \emph{infimum} of \refeq{minDistance0}. It does not matter whether
the group is compact, for example $\SOn{n}$, or noncompact, for example
the Euclidean group $E_2$ that underlies the generation of spiral
patterns\rf{Barkley94}; in either case any group orbit has one or several
locally closest passages to the {\template} state, and generically only
one that is the closest one.

So we do not have to compute all the roots of the polynomial
\refeq{eq:so2sing} - all we care about is the infimum, or the root that
corresponds to the shortest distance \refeq{minDistance0}.
While post-processing of a full \statesp\ trajectory $\ssp(\tau_j)$
requires a numerical (Newton method) determination of the
`moving frame' rotation
$\gSpace(\tau_j)$ at each time step $\tau_j$, the computation is not
as onerous as it might seem, as the knowledge of $\gSpace(\tau_j)$ and
$\groupTan(\sspRed(\tau_j))$
gives us a very good guess for $\gSpace(\tau_{j+1})$.


%
% ****** End of file frame.tex ******
