% master file siminos/froehlich/slice/abstract.tex
% $Author$ $Date$

Whenever a dynamical system has a continuous symmetry, it might be
desirable to `quotient' the symmetry and study the dynamics in the
lower-dimensional, symmetry-reduced \statesp. We study here symmetry
reduction by the \mslices\ (\mframes). We show that a `slice' hyperplane
defined by minimizing the distance to a generic `{\template}' intersects the
group orbit of every point in the full {\statesp}. Global symmetry
reduction by a single slice is, however, not natural for a chaotic /
turbulent flow; we tessellate instead the \reducedsp\ by a set of slices,
one for each dynamically prominent unstable pattern. Judiciously chosen,
such tessellation also eliminates the dynamical traversals of the \sset\
that comes along with slice slice, an artifact of using the {\template}'s
local group tangent space globally. We show that these singularities
induce explicitly computable jumps in the \reducedsp. As simple
illustration of the method we reducing the $\SOn{2}$ symmetry of the
\cLe.
  %
\PC{{\bf to Stefan}:
write this  often! this might be the only part of this text that most
people glance at.
%PC 2010-09-30: planted an error into the abstract, just to see how
%   often do you edit it.
%
%
   When you write a project report or a research article, you always
   write abstract, introduction and conclusions first, and then keep
   rewriting them often. They are the most important parts of the text,
   as that is for most people only parts they will look at.
   }

%
% ****** End of file abstract.tex ******
