% master file siminos/froehlich/slice/abstract.tex
% $Author$ $Date$

Whenever a dynamical system has a continuous symmetry, it might be
desirable to `quotient' the symmetry and study the dynamics in the
lower-dimensional, symmetry-reduced \statesp. We
study here symmetry reduction by the \mslices\ (\mframes). We show
that a slice hyperplane defined by minimizing the distance to a
generic `template' intersects the group orbit of every point in the full
{\statesp}. An artifact of using the template's local group tangent space
globally are the \sset s that come along with a given slice.
We show that these singularities induce explicitly
computable jumps in the \reducedsp. Throughout this paper we focus on
$\SOn{2}$ symmetries, using the \cLe\ as a simple example.
  %
\PC{{\bf to Stefan}:
write this  often! this might be the only part of this text that most
people glance at.
%PC 2010-09-30: planted an error into the abstract, just to see how
%   often do you edit it.
%
%
   When you write a project report or a research article, you always
   write abstract, introduction and conclusions first, and then keep
   rewriting them often. They are the most important parts of the text,
   as that is for most people only parts they will look at.
   }

%
% ****** End of file abstract.tex ******
