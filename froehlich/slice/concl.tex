% siminos/froehlich/slice/concl.tex  called by FrCv11.tex
% $Author$ $Date$

% PC 2011-01-05 incorporated concl.tex from the SF article

% \section{Conclusion}
\ifboyscout
	...
	\PC{{\bf to Stefan}:
   When you write a research article, you always
   write abstract, introduction and conclusions first, and then keep
   rewriting them often. They are the most important parts of the text,
   as that is for most people only parts they will look at.
   }
\fi

Many physically important spatially extended and fluid dynamics systems
exhibit continuous symmetries. For example,  excitable
media\rf{ZaZha70,Winfree73,Winfree1980,BaKnTu90,Barkley94}, \KS\
flow\rf{ku,siv,SCD07}, {\pCf}\rf{Visw07b,GHCW07,HGC08,GibsonMovies}, and
pipe flow\rf{Wk04,Kerswell05} are invariant (equivariant) under
combinations of translational (Euclidean), rotational and discrete
symmetries. If a physical problem has a symmetry, one should use it - one
does not want to compute the same solution over and over, all one needs
is to pick one representative solution per each symmetry related
equivalence class. Such procedure is called symmetry reduction.  In this
paper we have investigated symmetry reduction by the \mslices, a linear
procedure particularly simple and practical to implement, and answered
affirmatively the two main questions about the method:
(1) does a slice cut the group orbit of \emph{every} point in the
dynamical \statesp?
(2) can one deal with the {\sset s} that the method necessarily
introduces?

We have shown here that a symmetry-reduced trajectory passes through such
singularities through computable jumps, a nuisance numerically, but cause
to no conceptual difficulty. However, while a slice intersects each group
orbit in a neighborhood of a {\template} only once, extended globally any
slice intersects every group orbit multiple times. So even though every
slice cuts all group orbits, it makes no sense physically to use one
slice (a set of \emph{all} group orbit points that are closest to a given
{\template}) globally. We propose instead to construct a global atlas by
deploying sets of slices and linear Poincar\'e sections as charts of
neighborhoods of the most important (relative) equilibria and/or
(relative) periodic orbits.

Such global atlas should be sufficiently fine-grained so that an unstable,
ergodic trajectory never gets too close to any of the {\sset s}.
Why does this proposal have none of the elegance of, let's say,
Killing-Cartan classification of simple Lie algebras?
Why is this symmetry reduction purely a numerical procedure, rather
than an analytic change of equivariant coordinates to invariant ones?
The theory of \emph{linear} representations of compact Lie groups is a well
developed subject,
but role of symmetries in \emph{nonlinear} settings is
altogether another story. It is natural to express a dynamical system
with a symmetry in the symmetry's linear eigenfunction basis (let  us say,
Fourier modes), but for a nonlinear flow different modes are strongly coupled,
and group orbits embedded in such coordinate bases can be highly convoluted,
in ways that no single global linear slice hyperplane can handle intelligibly.

It should be emphasized that the atlas so constructed retains the dimensionality of
the original problem. The full dynamics is faithfully retained, we are \emph{not}
constructing a lower-dimensional model of the dynamics. Neighborhoods of
unstable \eqva\ and \po s are dominated by their unstable and least
contracting stable eigenvalues and are, for all practical purposes,
low-dimensional. Traversals of the ridges are, however, higher
dimensional. For example, crossing from the neighborhood of a two-rolls
state into the neighborhood of a three-rolls state entails going through
a pattern `defect,' a rapid transient whose precise description requires
many Fourier modes. Nevertheless,
the recent progress on separation of `physical' and `hyperbolically
isolated' covariant Lyapunov
vectors\rf{PoGiYaMa06,ginelli-2007-99,YaTaGiChRa08,TaGiCh09} gives us
hope that the proposed atlas could provide a systematic and controllable
framework for construction of lower-dimensional models of `turbulent'
dynamics of dissipative PDEs.

While it has been demonstrated in \refref{SiCvi10}  that the \mslices,
with a judicious choice of the {\template} and {\PoincSec}, works for a
system as simple as the \cLf, one still has to show that the method can
be implemented for a truly high-dimensional flow. In \refref{SCD07} it
was found that the coexistence of four equilibria, two \reqva\ and a
nested \fixedsp\ structure in an effectively $8$-dimensional \KS\ system
complicates matters sufficiently that no symmetry reduction has been
attempted so far. More importantly, a symmetry reduction of pipe flows, which
due to the translational symmetry have only relative (traveling)
solutions, remains an outstanding challenge\rf{ACHKW11}.
