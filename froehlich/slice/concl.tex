% siminos/froehlich/slice/concl.tex  called by FrCv11.tex
% $Author$ $Date$

% PC 2011-01-05 incorporated concl.tex from the SF article

% \section{Conclusion}
\ifboyscout
	...
	\PC{{\bf to Stefan}:
   When you write a research article, you always
   write abstract, introduction and conclusions first, and then keep
   rewriting them often. They are the most important parts of the text,
   as that is for most people only parts they will look at.
   }
\fi

Many spatially extended and fluid dynamics systems exhibit continuous
symmetries. Some of the common examples are  excitable
media\rf{ZaZha70,Winfree73,Winfree1980,BaKnTu90,Barkley94}, the \KS\
flow\rf{ku,siv,SCD07}, {\pCf}\rf{Visw07b,GHCW07,HGC08,GibsonMovies}, and
pipe flow\rf{Wk04,Kerswell05}, invariant (equivariant) under
combinations of translational (Euclidean), rotational and
discrete symmetries.

If a physical problem has a symmetry, one should use it - one does not
want to compute the same solution over and over, all one needs is to
pick one representative solution per each symmetry induced equivalence
class. Such procedure is called symmetry reduction, and the `\mslices' is
one such procedure, particularly simple and practical to implement. In this
paper we have investigated symmetry reduction by the \mslices\ and
answered affirmatively the two main questions about the method:
(1) does a slice cut the group orbit of \emph{every} point in the dynamical \statesp?
(2) can one deal with the {\sset s} that the method necessarily
introduce, artifacts of the reduction absent from the full
\statesp\ flow?

Re. (2) we have shown here that a reduced trajectory
passes through such singularities through computable jumps,
a nuisance numerically, but cause of no conceptual difficulty.

Re. (1): Locally, a slices intersect each group orbit in a neighborhood of a
template only once,
but extended globally, a slice intersects a group orbit multiple times.
So even though every slice cuts all group orbits, it makes no sense
physically to use one slice (a set of all group orbit points that are
closest to a given {\template}) globally. Instead we propose to
construct make a global atlas by
deploying sets of slices and linear Poincar\'e sections as charts of
neighborhoods of the most important (relative) equilibria and/or
(relative) periodic orbits.

Ridges (intersection boundaries between different slices) are themselves
hyperplanes of lower dimensions, easy to compute once we have decided
on the set of slices. To find what slice a given full \statesp\
trajectory point is in, one rotates with respect to each slice, and
checks whether the given group orbit belongs to it. In the \reducedsp\
the trajectory is integrated within a given slice until it hits a ridge -
then one switches to the next slice across the boundary.

Global chart should be sufficiently fine-grained so that an unstable,
ergodic trajectory never gets too close to any of the {\sset s}. That
means that the neighborhood - bounded by intersections with neighboring
slices is sufficiently small that group tangent space is nowhere within
the slice - works in smooth flows for sufficiently small neighborhoods.

While we have demonstrated that the
\mslices\ works for a system as simple as the \cLf,
one still has to the the method can be implemented for a
truly high-dimensional flow. In addition, flows such as the
\KS\ and {\pCf} also exhibit discrete symmetries that also have to be taken into account.
