% siminos/froehlich/slice/concl.tex  called by FrCv11.tex
% $Author$ $Date$

% \section{Conclusion}

Many systems in fluid dynamics exhibit a continuous symmetry. Systems
such as the \KS\ flow\rf{ku,siv},
{\pCf}\rf{Visw07b,GHCW07,HGC08,HalcrowThesis}, and flow through an
cylindrical pipe\rf{Wk04,Kerswell05} demonstrate a simple product of
$\SOn{2}$ symmetries.

In this paper we have investigated using linear subspaces in
\mslices\ to replace a dynamical system with an equivalent
lower dimensional system. The two main obstacles to using the \mslices\
are that every point must be rotatable into the subspace and it can
introduce singularities into the flow. Locally linear slices are
guaranteed to intersect each group orbit only once, but it was shown that
linear slices will intersect every group orbit of a compact Lie group in
the {\statesp}, though it will do so multiple times. The \mslices\ can
introduce singularities into the flow that did not exist in the full
space. We demonstrated that as long as the group action is well behaved
(which it is for any general $\SOn{2}$ symmetry) then a trajectory
passing through a singularity corresponds to a simple shift in the
trajectory and does not cause any difficulties. In addition we
demonstrated that the problem of dealing with singularities of a product
of $\SOn{2}$ groups acting on different coordinates of the {\statesp}
(as is the case for the \KS\rf{ku,siv},
{\pCf}\rf{Visw07b,GHCW07,HGC08,HalcrowThesis}, and
pipe flows\rf{Wk04,Kerswell05}) is equivalent to dealing with the
symmetries of each \SOn{2}\ symmetry independently.



Even every slice cuts all group orbits, it makes no sense physically to
use one slice
(a set of all group orbit points that are closest to a given `template')
globally. Instead we should do what we already do for KS Poincar\'e sections.
We need to make a global chart by deploying both linear slices and linear
Poincar\'e sections in neighborhoods of the most important (relative)
equilibria and/or (relative) periodic orbits (those are tricky, because
slice fixing points must lie in the full \statesp, and have no symmetry,
so most of the solutions we have are not good as they stand). This is the
periodic-orbit generalization of the idea of
\HREF{http://chaosbook.org/overheads/trace/Tesselate.jpg}{\statesp\ tessellation}
so dear to professional cyclist(s).


Boundaries
between hyperplanes are themselves hyperplanes of one dimension less and
should be easy compute once we have decided on the set of slices. To find
what slice a given full \statesp\ trajectory point is in, one rotates
with respect to each slice, and checks whether the given group orbit
belong to it. In the \reducedsp\ the trajectory is integrated within a
given slice until it hits a hyperplane boundary - then one switches to
the next slice across the boundary. Boundary corners are measure zero, no
way you would hit them.

Global chart should be sufficiently fine-grained that we never hit any
slice singularities. That means that the neighborhood - bounded by
intersections with neighboring slices is sufficiently small that group
tangent space is nowhere within the slice - works in smooth flows
for sufficiently small neighborhoods.


More has to be done to reduce a system with a discrete
symmetry before the \mslices\ can be used for the continuous symmetry.
