% siminos/froehlich/slice/flotsam.tex  called by FrCv11.tex
% $Author$ $Date$

\section{Flotsam}
\label{sec:flotsam}

\ifboyscout
\subsubsection{Subsubsection header}
[Just checking what the header looks like]
\else
\fi

													\toCB
symmetry reduction = lowering the order\rf{ArKoNe88} % (Arn Kozl Neist)

\noindent \textbf{PC 2010-12-11~~}
This text is clipped from 2010-06-07 siminos/CLE/CLE.tex as is. Use parts of it:

``
In refref{??} we have shown, in the $5$-dimensional \cLe\ example, that the
location of the \sset\ can be manipulated by judicious choice of the
slice fixing point, and geometrical information about the dynamics can be
extracted by constructing a return map through a {\PoincSec} that does
not intersect the {\sset}. In
higher-dimensional flows, with more involved symmetry group actions and
larger sets of stationary solutions, where a single slice and {\PoincSec}
will not suffice, we can still expect to cover the \reducedsp\ with
multiple slices, obtaining a set of discrete maps involving multiple
{\PoincSec}s. As to high-dimensional applications, it was shown in
\refref{SiminosThesis} that the coexistence of four equilibria, two
\reqva\ and a nested \fixedsp\ structure in an effectively
$8$-dimensional \KS\ system\rf{SCD07} complicates matters considerably.
''

-------------

{\bf PC}{still have to derive the general case: ``
Let $\sliceTan{}$ be a vector normal to the plane of the slice. Then the
dynamics within the slice are given by
\bea
\dot{\gSpace_a}(\sspRed) &=& \frac{\braket{\vel(\sspRed)}{\sliceTan{a}}}
               {\braket{\groupTan(\sspRed)}{\sliceTan{}}}
\continue
\velRed(\sspRed) &=& \vel(\sspRed)
   -\dot{\gSpace}(\sspRed) \cdot \groupTan(\sspRed)
\label{SF:sliceEas}
\eea
where $\velRed(\sspRed)$ is the velocity in the slice.
    ''}

-------------

{\bf PC}{draw your own reffig~{fig:CLEpcSect}:\\
        * Mark $\ssp_{\REQB{}1}$ \\
        * Draw stable eigenvector of $\ssp_{\REQB{}1}$\\
        * State value of $\ssp_{\REQB{}1}$ somewhere
        }

-------------

If $\ssp(\tau)$ is a solution to the dynamical equations, then
$\LieEl\,\ssp(\tau)$ is also a solution.

Most of the Lie groups that we will only be considering here are
compact.

$\Lg_a \Lg_b$ plays a fundamental role in the theory of Lie groups.

The $L^2$ norm of $\groupTan(u)$ is weighted by
the quadratic Casimir \refeq{QuadCasimir}. For \SOn{2} this is
$C_2^{(m)} = m^2$,
\beq
\oint \frac{d\gSpace}{2\pi}
     \, (\Lg u(\gSpace))^T \Lg u(2\pi-\gSpace)
= \sum_{m=1}^\infty m^2 \left(a_m^2 + b_m^2\right)
\,.
\ee{tangL2norm}

  %
We only care about those that are local (and preferably global) {\em
minima}, for which all the eigenvalues of the symmetric matrix
$[N\!\times\!N]$ matrix of second derivatives of distance,
\beq
\frac{\partial^2}
     {\partial \gSpace_a \partial \gSpace_b}
        |\sspRed - \slicep|^2
    =
%  - \braket{\Lg_a e^{\gSpace \cdot \Lg} \ssp}{\sliceTan{b}}=
  - \braket{\groupTan_a(\sspRed)}{\sliceTan{b}}=
  \braket{\sspRed}{\Lg_a \Lg_b\slicep}
\ee{PCinflPoint}
are positive. In practice we do not need to actually compute
this matrix, we simply pick from among the local minima
the infimum of the distance.


-------------

The distance surface $|\sspRed - \slicep|$ can have inflection points,
What role do they play? They are non-generic, but if we consider distance
to local minima at successive instants of a time-evolving trajectory,
coalescence of
nearby minima, maxima pairs cannot be avoided. At the instant of
coalescence the denominator in \refeq{SF:sliceEas} goes through a simple
pole, and the integrated trajectory within slice might jump.

The distance to the {\template} can have inflection points.

{\bf PC:}{ could it be that inflections are generic only for \SOn{2},
        but of higher codimension and thus not encountered
        by 1\dmn\ time trajectory for higher-dimensional Lie groups?}

	%
{\bf PC:}{
{\bf Tessellation of the \reducedsp\ by two or many slices:
how to implement it?}
See \HREF{http://en.wikipedia.org/wiki/Voronoi_diagram}
{wiki on Voronoi diagrams}.
We have to study Roweis  and Saul\rf{RoSa00}
\emph{``Nonlinear dimensionality reduction by locally linear embedding.''}
A sobering fact: Rowais, assistant professor at N.Y.U., a young
star in the field and universally liked, jumped out of his Washington
Square apartment earlier this year.
	}


-------------

Symmetry reduction replaces a symmetry-invariant flow by a
lower-dimensional, symmetry-reduced flow.

-------------

Shift in time to the closest passage\rf{pchaot,MFKM10,CviGib10}:
`close recurrence.'

-------------

This application of symmetry reduction to a spatially extended, PDE
system is the subject of a forthcoming publication\rf{SCD09b}.

-------------

More has to be done to reduce a system with a discrete
symmetry before the \mslices\ can be used for the continuous symmetry.

-------------

While \refref{Christiansen97} demonstrated that \po\ theory can be
applied to spatially extended systems,

-------------

Unattended to, a symmetry can induce drifts along group orbit directions
and be a great nuisance; deftly deployed
it can be a powerful tool in simplifying physical problems.

-------------

Symmetry strongly constrains the form of solutions and their
bifurcations, if appropriately implemented, it can significantly
accelerate convergence of numerical algorithms, it splits the dynamical
\statesp\ into chains of lower-dimensional flow-invariant subspaces,
dictates its invariant partitions and the symbolic dynamics.

-------------

The theory of \emph{linear} representations of symmetry groups is
a well developed subject. Role of symmetries in
\emph{nonlinear} settings is altogether another story.

-------------

 (those are tricky, because slice fixing points
must lie in the full \statesp, and have no symmetry, so most of the
solutions we have are not good as they stand)

-------------

Ridges are of Lebesgue measure zero, no way you would hit them.

-------------


-------------

-------------

-------------

-------------

-------------

-------------

-------------
