% siminos/froehlich/slice/flotsam.tex  called by FrCv11.tex
% $Author$ $Date$

\section{Flotsam}
\label{sec:flotsam}

\ifboyscout
\subsubsection{Subsubsection header}
[Just checking what the header looks like]
\else
\fi

													\toCB
symmetry reduction = lowering the order\rf{ArKoNe88} % (Arn Kozl Neist)

\noindent \textbf{PC 2010-12-11~~}
This text is clipped from 2010-06-07 siminos/CLE/CLE.tex as is. Use parts of it:

``
As to high-dimensional applications, it was shown in
\refref{SiminosThesis} that the coexistence of four equilibria, two
\reqva\ and a nested \fixedsp\ structure in an effectively
$8$-dimensional \KS\ system\rf{SCD07} complicates matters considerably.
''

-------------

{\bf PC}{still have to derive the general case: ``
Let $\sliceTan{}$ be a vector normal to the plane of the slice. Then the
dynamics within the slice are given by
\bea
\dot{\gSpace_a}(\sspRed) &=& \frac{\braket{\vel(\sspRed)}{\sliceTan{a}}}
               {\braket{\groupTan(\sspRed)}{\sliceTan{}}}
\continue
\velRed(\sspRed) &=& \vel(\sspRed)
   -\dot{\gSpace}(\sspRed) \cdot \groupTan(\sspRed)
\label{SF:sliceEas}
\eea
where $\velRed(\sspRed)$ is the velocity in the slice.
    ''}

-------------

{\bf PC}{draw your own reffig~{fig:CLEpcSect}:\\
        * Mark $\ssp_{\REQV{}1}$ \\
        * Draw stable eigenvector of $\ssp_{\REQV{}1}$\\
        * State value of $\ssp_{\REQV{}1}$ somewhere
        }

-------------

If $\ssp(\tau)$ is a solution to the dynamical equations, then
$\LieEl\,\ssp(\tau)$ is also a solution.

Most of the Lie groups that we will only be considering here are
compact.

The $L^2$ norm of $\groupTan(u)$ is weighted by
the quadratic Casimir \refeq{QuadCasimir}. For \SOn{2} this is
$C_2^{(m)} = m^2$,
\beq
\oint \frac{d\gSpace}{2\pi}
     \, (\Lg u(\gSpace))^T \Lg u(2\pi-\gSpace)
= \sum_{m=1}^\infty m^2 \left(a_m^2 + b_m^2\right)
\,.
\ee{tangL2norm}

  %
We only care about those that are local (and preferably global) {\em
minima}, for which all the eigenvalues of the symmetric matrix
$[N\!\times\!N]$ matrix of second derivatives of distance,
\beq
\frac{\partial^2}
     {\partial \gSpace_a \partial \gSpace_b}
        |\sspRed - \slicep|^2
    =
%  - \braket{\Lg_a e^{\gSpace \cdot \Lg} \ssp}{\sliceTan{b}}=
  - \braket{\groupTan_a(\sspRed)}{\sliceTan{b}}=
  \braket{\sspRed}{\Lg_a \Lg_b\slicep}
\ee{PCinflPoint}
are positive. In practice we do not need to actually compute
this matrix, we simply pick from among the local minima
the infimum of the distance.

{\bf PC:}{ could it be that inflections are generic only for \SOn{2},
        but of higher codimension and thus not encountered
        by 1\dmn\ time trajectory for higher-dimensional Lie groups?}


-------------

{\bf PC:}{
{\bf Tessellation of the \reducedsp\ by two or many slices:
how to implement it?}
See \HREF{http://en.wikipedia.org/wiki/Voronoi_diagram}
{wiki on Voronoi diagrams}.
We have to study Roweis  and Saul\rf{RoSa00}
\emph{``Nonlinear dimensionality reduction by locally linear embedding.''}
A sobering fact: Rowais, assistant professor at N.Y.U., a young
star in the field and universally liked, jumped out of his Washington
Square apartment earlier this year.
	}


-------------

Symmetry reduction replaces a symmetry-invariant flow by a
lower-dimensional, symmetry-reduced flow.

By symmetry, two points are equivalent if they are in the same group
orbit.

We can associate with any trajectory in the full \statesp, $\ssp(\tau)
\in \pS$, a \reducedsp\ trajectory, $\sspRed(\tau) \in \pS$. Conversely,
given the unique \reducedsp\ equivalence class representative
$\sspRed(\tau)$, the full space trajectory can be reconstructed given
$\LieEl(\tau)$, the group action that rotates the \reducedsp\ point
$\sspRed(\tau)$  to $\ssp(\tau)$.


-------------

Shift in time to the closest passage\rf{pchaot,MFKM10,CviGib10}:
`close recurrence.'

-------------

This application of symmetry reduction to a spatially extended, PDE
system is the subject of a forthcoming publication\rf{SCD09b}.

-------------

More has to be done to reduce a system with a discrete
symmetry before the \mslices\ can be used for the continuous symmetry.

-------------

While \refref{Christiansen97} demonstrated that \po\ theory can be
applied to spatially extended systems,

-------------

Unattended to, a symmetry can induce drifts along group orbit directions
and be a great nuisance; deftly deployed it can be a powerful tool in
simplifying physical problems.

-------------

Symmetry strongly constrains the form of solutions and their
bifurcations, if appropriately implemented, it can significantly
accelerate convergence of numerical algorithms, it splits the dynamical
\statesp\ into chains of lower-dimensional flow-invariant subspaces,
dictates its invariant partitions and the symbolic dynamics.

-------------

(those are tricky, because {\template s} must lie in the full
\statesp, and have no symmetry, so most of the solutions we have are not
good as they stand)

cannot use most \eqva\ as `templates,' as they often 	are invariants
under subgroups of the symmetry group \Group, and 	thus their group
orbits have dimensions less than $N$.

-------------

 This tessellation is akin to the
%\HREF{http://en.wikipedia.org/wiki/Vector_quantization}
{`vector quantization,'} (`block quantization,'  `pattern matching quantization'),
a computer science data compression method where sets of points are
clustered by their distance to `prototype' or `reference' points. The
method encodes values from a multidimensional vector space into a
`codebook,' a finite set of values from a discrete space of a lower
dimension; \ie, what in the theory of dynamical systems is called
`symbolic dynamics.'

Ridges are of Lebesgue measure zero, no way you would hit them.

We do not need to be very precise about the instant
where we switch, as long as we are away from either slice's singularity
subspace.

The \reducedsp\ tessellation by a set of slices
locked relative to each other by the shortest unstable manifold segments
offers a better approximation to distances between points on an attractor
of a nonlinear flow.

Ridges (intersection boundaries between different slices) are themselves
hyperplanes of lower dimensions, easy to compute once we have picked
the set of {\template s}. To find what slice a given full \statesp\
trajectory point is in, one rotates with respect to each slice, and
checks whether the given group orbit belongs to it. In the \reducedsp\
the trajectory is integrated within a given slice until it hits a ridge -
then one switches to the next slice across the boundary.

That means that the neighborhood - bounded by intersections with
neighboring slices is sufficiently small that group tangent space is
nowhere within the slice - works in smooth flows for sufficiently small
neighborhoods.

-------------

{\bf PC to Stefan}:
	While `fixes' is a correctly used math word, I tend to avoid it 	
as it violates the common usage; $\Lg_1$ does not (transitively) `fix'
anything, 	it simply does not act upon $b$ coordinates.

-------------


-------------
