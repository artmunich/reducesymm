%% ENCODING UTF-8
\documentclass[a4paper,11pt]{article}

%%%%%%%%%%%%%%%%%%%%%%%%%%%%%%%%%%%%%%%
%%%%%%%%%%%%%%%%%%%%%%%%%%%%%%%%%%%%%%%
%
% Latex Template for Marie Curie CIG
%
% NOTE this is not an official template; adjust margins, formatting
% etc as required
%
%%%%%%%%%%%%%%%%%%%%%%%%%%%%%%%%%%%%%%%
%%%%%%%%%%%%%%%%%%%%%%%%%%%%%%%%%%%%%%%

%% Arial-like fonts
\renewcommand{\rmdefault}{phv} 
\renewcommand{\sfdefault}{phv} 

\usepackage[T1]{fontenc}
\usepackage[utf8]{inputenc}
\usepackage[margin=2cm]{geometry}
\usepackage{tabularx}
\usepackage[table]{xcolor}
\usepackage{graphicx}
\usepackage{xspace}
\usepackage{fancyhdr}
\usepackage{datetime}
\usepackage[colorlinks=true,citecolor=black,urlcolor=black,linkcolor=black]{hyperref}
\usepackage{lastpage}
\usepackage{bibunits}
\usepackage{pgfgantt}

\pagestyle{fancy}

\setlength{\headheight}{14pt}

\renewcommand{\thesection}{B\arabic{section}} 
\newcommand{\content}[1]{\emph{#1}\\} 
\newif\ifMyDraft% boolean variable whether or not to show xcomments
\MyDraftfalse%    default should be false

%%%%%%%%%%%%%%%%%%%%%%%%%%%%%%%%%%%%%%%
%%%%%%%%%%%%%%%%%%%%%%%%%%%%%%%%%%%%%%%
%%%% ADJUST HERE
\newcommand{\projectname}[0]{CCLPI} 
\newcommand{\mytitle}{Chaos and complexity in ultra high intensity laser-plasma interactions}
% set MyDraft to true in order to get comments for the sections displayed, the build date, etc...
\MyDrafttrue
%%%% END OF ADJUST HERE
%%%%%%%%%%%%%%%%%%%%%%%%%%%%%%%%%%%%%%%
%%%%%%%%%%%%%%%%%%%%%%%%%%%%%%%%%%%%%%%

%%%%%%%%%%%%%
\ifMyDraft % iftrue: this is a draft
  \newenvironment{xcomment}{\em}{}
  \fancyhead{}
  \fancyfoot{}
  % \fancyhead[L]{\AppRef}
  \fancyhead[R]{\projectname{} -- version of \today{}, \currenttime}
  \fancyfoot[L]{Part B}
  \fancyfoot[R]{\upshape Page {\thepage} of \pageref{LastPage}}
  \renewcommand{\headrulewidth}{0.5pt}
  \renewcommand{\footrulewidth}{0.5pt}
\else
  \usepackage{xcomment}
  \fancyhead{}
  \fancyfoot{}
  % \fancyhead[L]{\AppRef}
  \fancyhead[R]{\projectname{}}
  \fancyfoot[L]{Part B}
  \fancyfoot[R]{\upshape Page {\thepage} of \pageref{LastPage}}
  \renewcommand{\headrulewidth}{0.5pt}
  \renewcommand{\footrulewidth}{0.5pt}
\fi

\title{\mytitle}
%%%%%%%%%%%%%

%%%%%%%%%%%%% the cover STARTPAGE and ENDPAGE macro
\newcommand{\cover}[1]{%
\newpage
\begin{center}
\vspace{4cm}
\textbf{#1}
\vspace{4cm}

PEOPLE

MARIE CURIE ACTIONS

\vspace{1cm}

% \textbf{Marie Curie International Reintegration Grants (IRG)}
\textbf{Intra-European Fellowships (IEF)}\\
\textbf{Call: FP7-PEOPLE-2013-IEF}

\vspace{4cm}

\Large{PART B}
\vspace{2cm}

\Huge{\mytitle}

\vspace{1cm}

\Large{``\projectname''}

\end{center}
\newpage
} 
%%%%%%%%%%%%% end of cover STARTPAGE and ENDPAGE macro

%%%%%%%%%%%%%%%%%%%%%%%%%%%%%%%%%%%%%%%%%%%%%%%%%%%%%%%%%%%%%%%%%%%%%%%%%%%%%%%%%%%%%%%%
%%%%%%%%%%%%%%%%%%%%%%%%%%%%%%%%%%%%%%%%%%%%%%%%%%%%%%%%%%%%%%%%%%%%%%%%%%%%%%%%%%%%%%%%
%%%%%%%%%%%%%%%%%%%%%%%%%%%%%%%%%%%%%%%%%%%%%%%%%%%%%%%%%%%%%%%%%%%%%%%%%%%%%%%%%%%%%%%%
\begin{document}

\cover{STARTPAGE}
%%%%%%%%%%%%%%%%%%%%%%%%%%%%%%%%%%%%%%%%%%%%%%%%%%%%%%%%%%%%%%%%%%%%%%%%%%%%%%%%%%%%%%%%
% \tableofcontents

\newpage
%%%%%%%%%%%%%%%%%%%%%%%%%%%%%%%%%%%%%%%%%%%%%%%%%%%%%%%%%%%%%%%%%%%%%%%%%%%%%%%%%%%%%%%%

\begin{xcomment} 
Warning: the comments are turned on. All directives and guidelines appear.

To turn them off, comment the line:\\ 
\verb|\MyDrafttrue|\\

\end{xcomment} 


%%%%%%%%%%%%%%%%%%%%%%%%%%%%%%%%%%%%%%%%%%%%%%%%%%%%%%%%%%%%%%%%%%%%%%%%%%%%%%%%%%%%%%%%
%%%%%%%%%%%%%%%%%%%%%%%%%%%%%%%%%%%%%%%%%%%%%%%%%%%%%%%%%%%%%%%%%%%%%%%%%%%%%%%%%%%%%%%%
\section{Research and Technological Quality}

\begin{bibunit}[../../../../tex/statement2012]

\begin{xcomment}  
(Maximum 8 pages)  
\end{xcomment}

%%%%%%%%%%%%%%%%%%%%%%%%%%%%%%%%%%%%%%%%%%%%%%%%%%%%%%%%%%%%%%%%%%%%%%%%%%%%%%%%%%%%%%%%
\subsection{Research and technological quality, including any interdisciplinary and multidisciplinary aspects of the proposal}

\begin{xcomment}
Give a clear description of the state-of-the-art of the research topic. Provide a clear and specific
description of the research objectives against the background of the state of the art, and the
results hoped for. The most relevant bibliographical references should be provided, and must be
included in the overall page count. If relevant, provide information on interdisciplinary /
multidisciplinary and/or inter-sectoral aspects of the proposal.
\end{xcomment}

Laser plasma interactions provide opportunities for the study of dynamics of many-body systems
driven out of equilibrium. The purpose of this proposal is to apply well developed tools of
nonlinear dynamics to the study of laser-matter interactions, but also to develop new methods.
Most importantly, we aim at a cross-fertilization of both areas, by proposing ways in which
signatures of complex dynamics could be measured in laser-plasma interactions.

There will be two levels in our study. The first concerns the chaotic dynamics of particles in optical
lattices (standing waves) at relativistic intensities, ignoring space charge effects. The
purpose here is to illustrate the effect of chaotic dynamics to mesurable quantities such as 
energy distribution functions. This will necesitate the formulation of the problem as a
chaotic scattering problem (see Sect. )

At the second level we will mover forward from the study of chaos to the study of complexity,
by taking into account the self-consistent dynamics of a plasma interacting with an ultra-intense
laser pulse. Here we will consider as an example problem the case of a solid density target
illuminated by an ultra-intense laser pulse, since there are many applications associated with
this setting [ion acceleration, high-harmonic generation (HHG), inertial confinement fusion (ICF)].

%%%%%%%%%%%%%%%%%%%%%%%%%%%%%%%%%%%%%%%%%%%%%%%%%%%%%%%%%%%%%%%%%%%%%%%%%%%%%%%%%%%%%%%%
\subsection{Appropriateness of research methodology and approach}

\begin{xcomment}
For each objective explain the methodological approach that will be employed in the project and
justify it in relation to the overall project objectives. Describe any relevant techniques, methods or
analyses that will be applied.
\end{xcomment}

%%%%%%%%%%%%%%%%%%%%%%%%%%%%%%%%%%%%%%%%%%%%%%%%%%%%%%%%%%%%%%%%%%%%%%%%%%%%%%%%%%%%%%%%
\subsection{Originality and innovative nature of the project, and relationship to the 'state of the art' of research in the field}

\begin{xcomment}
Explain the contribution that the project is expected to make to advance the state-of-the-art within
the project field. Describe any novel concepts, approaches or methods that will be employed.
\end{xcomment}

%%%%%%%%%%%%%%%%%%%%%%%%%%%%%%%%%%%%%%%%%%%%%%%%%%%%%%%%%%%%%%%%%%%%%%%%%%%%%%%%%%%%%%%%
\subsection{Timeliness and relevance of the project}

\begin{xcomment}
Describe the appropriateness of the research proposed against the state of the art and why it is
timely. Outline the benefit that will be gained from undertaking the project at European Research
Area (ERA) level and how the fellowship will contribute to enhance ERA research excellence and
reintegrate the researcher. Describe the scientific, technological, socio-economic or other
reasons for carrying out further research in the field covered by the project
\end{xcomment}

\subsection{Host research expertise in the field}

\begin{xcomment}
The host institution must explain its level of experience on the research topic proposed and
document its track record of work, including the main international collaborations. Information
provided should include participation in projects, publications, patents and any other relevant
results. 
\end{xcomment}


\subsection{Quality of the group/scientist in charge}

Similar information as above should be provided for the scientist in charge of the supervision of
the project. The host institution must demonstrate its track record of previous training
achievements especially at an advanced level within the field of research. 

\putbib[../../../../bibtex/siminos]
\end{bibunit}

%%%%%%%%%%%%%%%%%%%%%%%%%%%%%%%%%%%%%%%%%%%%%%%%%%%%%%%%%%%%%%%%%%%%%%%%%%%%%%%%%%%%%%%%
\newpage
%%%%%%%%%%%%%%%%%%%%%%%%%%%%%%%%%%%%%%%%%%%%%%%%%%%%%%%%%%%%%%%%%%%%%%%%%%%%%%%%%%%%%%%%
\section{Training}

\begin{xcomment}
 (maximum 2 pages)
\end{xcomment}

\subsection{Clarity and quality of the research training objectives for the researcher}

\begin{xcomment}
 State the training objectives and explain in detail how these can be beneficial for the (further)
development of an independent research career.
\end{xcomment}

\subsection{Relevance and quality of additional research training as well as of transferable skills offered with special attention to exposure to industry sector, where appropriate}

\begin{xcomment}
 State the training objectives and explain in detail how these can be beneficial for the (further)
development of an independent research career.
\end{xcomment}

\subsection{Measures taken by the host for providing quantitative and qualitative mentoring/tutoring}

\begin{xcomment}
 Give a short outline of the host's capacity for training, and which measures the host will
undertake for training, mentoring/tutoring the researcher.
\end{xcomment}


%%%%%%%%%%%%%%%%%%%%%%%%%%%%%%%%%%%%%%%%%%%%%%%%%%%%%%%%%%%%%%%%%%%%%%%%%%%%%%%%%%%%%%%%
\newpage
%%%%%%%%%%%%%%%%%%%%%%%%%%%%%%%%%%%%%%%%%%%%%%%%%%%%%%%%%%%%%%%%%%%%%%%%%%%%%%%%%%%%%%%%
\section{Researcher}
\begin{xcomment}  
(maximum 7 pages which includes a CV and a list of main achievements)
\end{xcomment}

%%%%%%%%%%%%%%%%%%%%%%%%%%%%%%%%%%%%%%%%%%%%%%%%%%%%%%%%%%%%%%%%%%%%%%%%%%%%%%%%%%%%%%%%
\subsection{Research experience}
\begin{xcomment}
  The applicant must present a comprehensive description of his/her research experience. A
  scientific/professional CV must be provided and should mention explicitly:
  \begin{itemize}
    \item academic achievements
    \item list of other professional activities
    \item any other relevant information.
  \end{itemize}
\end{xcomment}

\newcommand{\cventryAlt}[6]{#1 & #2, #3, #4 \\}
\newcommand{\cventryDescr}[6]{#1 & #2 #3 #4 \\}
\newcommand{\cvitem}[2]{#1 & #2\\}
\newcommand{\cvcomputer}[4]{#1 & #2 & & #3  #4 \\}
\newcommand{\emtitle}{\em}
\newcommand{\siminos}{E. Siminos}

\subsubsection{Education}
\begin{tabularx}{\linewidth}{  r  X }
  \cventryAlt{2005 -- 2009}{PhD in Physics}{Georgia Institute of Technology}{Atlanta, GA, USA}{}{adviser: Prof. P. Cvitanovi\'{c}}
  \cventryAlt{2003 -- 2005}{MS in Physics}{Georgia Institute of Technology}{Atlanta, GA, USA}{}{}%{Grade Point Average: 3.56/4} % arguments 3 to 6 are optional
  \cventryAlt{1998 -- 2003}{BS in Physics}{University of Thessaloniki}{Thessaloniki, Greece}{}{}%{Average Grade: 8.17/10}
  \cventryAlt{fall 2001}{Exchange Student}{Max Planck Institut f\"{u}r Plasmaphysik}{Greifswald, Germany}{}{}
\end{tabularx}

\subsubsection{Employment}
\begin{tabularx}{\linewidth}{  r  X }
  \cventryAlt{2011 -- { now}}{Postdoctoral Fellow}{Max Planck Institute for the Physics of Complex Systems}{Dresden, Germany}{}{}
  \cventryDescr{}{\small{Second Scientific Head of the \emph{Computational Nonlinear and Relativistic optics} group}}{}{}{}{}
  \cventryAlt{2009 -- 2011}{Postdoctoral Fellow}{Commissariat \`{a} l' \'Energie Atomique (CEA), DAM, DIF}{Arpajon (Paris area), France}{}{}
  \cventryAlt{2008 -- 2009}{Research Assistant}{Center for Nonlinear Science}{School of Physics, Georgia Tech, Atlanta, GA, USA}{support: NSF grant DMS-0807574 \& G.~Robinson~Fund}{}
  % \cventryAlt{summer 2005}{Research Assistant}{Center for Nonlinear Science}{School of Physics}{Georgia Tech}
  % 			{support: G.~Robinson~Fund}
  \cventryAlt{2003 -- 2008}{Teaching Assistant}{School of Physics, Georgia Tech}{Atlanta, GA, USA}{}{}{}{}
\end{tabularx}

\subsubsection{Teaching Experience}
\begin{tabularx}{\linewidth}{  r  X }
\cventryAlt{fall 2008}{Symmetry in dynamical systems}{School of Physics}{Georgia Tech}{USA}
 {Series of three lectures for the advanced graduate course {\em Nonlinear Dynamics} (PHYS 7224)}
\cventryAlt{2003 -- 2008}{Teaching Assistant}{School of Physics}{Georgia Tech}{USA}{}
\cventryDescr{courses}{Undergraduate Physics I \& II, Physics Laboratory I \& II, Classical Mechanics I \& II, Electromagnetism, Special Relativity, Quantum Mechanics I}
 	     {duties}{Lab sessions, recitation sessions, office hours, preparation and grading of homework \& exams}{}{}
\cventryAlt{1999 -- 2000}{Voluntary Teaching Assistant}{Department of Physics, University of Thessaloniki}{Greece}{}{}{}
\cventryDescr{fall 1999}{Lab assistant for Introductory Computer Lab}{}{}{}{}
\cventryDescr{spring 2000}{Grader for course Calculus II}{}{}{}{}
\end{tabularx}

\subsubsection{Student supervision}
\begin{tabularx}{\linewidth}{  r  X }
\cventryDescr{2012}{Advised PhD student Fabian Maucher on the use of dynamical systems methods in the study of solitons in nonlocal nonlinear media}{}{}{}{}
\cventryDescr{Fall 2008}{Advised student Dominic Kohler in his project ``Armbruster-Guckenheimer-Holmes flow''
for graduate level course ``Nonlinear Dynamics'' (School of Physics, Georgia Tech)}{}{}{}{}
\end{tabularx}

\subsubsection{\textsc{Fellowships}}
\begin{tabularx}{\linewidth}{  r  X }
	\cvitem{2007}{Gerondelis Foundation Graduate Student Fellowship, USA}
	\cvitem{2001}{Erasmus Fellowship, European Union}
\end{tabularx}

\subsubsection{\textsc{Computer skills}}
\begin{tabularx}{\linewidth}{  r  X  r X}
% {environments}{Linux, Windows}
\cvcomputer{programming}{C/C++, Fortran, Python}{libraries}{PETSc, matplotlib, channelflow}%{scripting}{Perl, bash}
\cvcomputer{markup}{\LaTeX, \textsc{html}}{other}{Mathematica, Matlab} %channelflow
\end{tabularx}

\subsubsection{\textsc{Other Activities}}
\begin{tabularx}{\linewidth}{  r  X }
\cventryDescr{2008}{Organized informal seminar for Center for Nonlinear Science, Georgia Tech.}{}{}{}{}
\cventryDescr{Referee for}{New J. Phys, J. Phys. A}{}{}{}{}
% \section{\textsc{Languages}}
% \cvcomputer{English}{fluent}{Greek}{native speaker}
% \cvcomputer{German}{fair}{French}{elementary}
\end{tabularx}

% \\ \\ \\ \\ %\\ \\ \\ \\ \\ \\ \\ \\ \\
\subsubsection{\textsc{Seminar Talks}}
\begin{tabularx}{\linewidth}{  r  X }
\cvitem{June 2012}{Helmholtz Institute Jena, Germany\newline
\emph{When does an ultra-intense laser pulse propagate in a plasma?}}% June 25 2012
% \cvitem{May 2011}{Max Planck Inst. for the Physics of Complex Systems, Dresden\newline
% \emph{Stability of nonlinear waves in collisionless plasmas}}% May 30 2011
\cvitem{March 2011}{ETH Zurich, Department of Materials\newline
\emph{Stability of nonlinear waves in collisionless plasmas}}% March 30 2011
\end{tabularx}

% \subsubsection{\textsc{Recent Conferences}}
% \begin{tabularx}{\linewidth}{  r  X }
% \cvitem{Sept. 2012\\ talk}{Dynamics Days Europe, Gothenbourg, Sweden\newline
% 	\siminos\ and P. Cvitanovi\'c, {\emtitle Continuous symmetry reduction in high-dimensional flows with the method of slices}
% }
% \cvitem{July 2012\\ poster}{EPS Conference on Plasma Physics, Stockholm, Sweden\newline
% 	\siminos,  M.~Grech, S.~Skupin, T.~Schlegel, and V.\,T.~Tikhonchuk, {\emtitle Electron heating effect on self-induced-transparency threshold in ultra-intense laser pulse interaction with overdense plasmas}
% }
% \cvitem{Sept. 2012\\ poster}{Frontiers in Intense Laser-Matter Interactions Theory Workshop, Garching, Germany\newline
% 	\siminos, M.~Grech, S.~Skupin, T.~Schlegel, and V.\,T.~Tikhonchuk, {\emtitle Electron heating effect on self-induced transparency in relativistic intensity laser-plasma interaction}
% }
% \cvitem{April 2012\\ talk}{Workshop on Laser-Plasma Interaction at Ultra-High Intensity, Dresden, Germany\newline
% \siminos, M.~Grech, S.~Skupin, T.~Schlegel, and V.\,T.~Tikhonchuk, {\emtitle Electron heating effect on self-induced-transparency threshold in ultra-intense laser pulse interaction with overdense plasmas}
% }%April 16 - 20, 2012
% \cvitem{June 2011\\ poster}{EPS Conference on Plasma Physics, Strasbourg, France\newline
% 	\siminos,  D. B\'enisti and L. Gremillet, {\emtitle A spectral method for the stability of BGK modes and application to vortex-fusion instabilities}
% }
% \cvitem{May 2011\\ talk}{Chaos, Complexity and Transport, Marseilles, France\newline
% 	\siminos,  D. B\'enisti and L. Gremillet, {\emtitle A spectral method for the stability of nonlinear Vlasov-Poisson equilibria}
% }
% \cvitem{Nov. 2010\\ talk}{Annual Meeting of the APS Division of Plasma Physics, Chicago, IL, USA\newline
% 	\siminos,  D. B\'enisti and L. Gremillet, {\emtitle Stability of nonlinear Vlasov-Poisson equilibria through spectral deformation and Fourier-Hermite expansion}
% }
% \cvitem{Sept. 2010\\ poster}{International Workshop on Laser-Matter Interaction,  Porquerolles, France\newline
% 	\siminos,  D. B\'enisti and L. Gremillet, {\emtitle Stability of nonlinear Vlasov-Poisson equilibria through Fourier-Hermite expansion}
% % 13-17 September 2010,
% }
% \cvitem{June 2009\\ poster}{Modern Challenges in Nonlinear Plasma Physics, Sani, Halkidiki, Greece\newline
%  	\siminos, P. Cvitanovi\'c and R.\,L. Davidchack,
%  	{\emtitle State-space geometry of a continuous symmetry reduced Kuramoto-Sivashinsky flow}
% }
% \cvitem{May 2009\\ talk}{SIAM Conference on Applications of Dynamical Systems, Snowbird, UT, USA\newline
% 	\siminos, P. Cvitanovi\'c and R.\,L. Davidchack,
% 	{\emtitle State-space geometry of a Kuramoto-Sivashinsky flow in
% 	terms of relative periodic orbits}\newline
% 	in Minisymposium: {\em Dynamical systems and turbulence: unstable periodic orbits}
% }
% \cvitem{Jan. 2009\\ poster}{Dynamics Days, San Diego, CA, USA\newline
%  	\siminos\ and  P.\ Cvitanovi\'c, {\emtitle Continuous symmetry reduction for high dimensional flows}
% }
% \cvitem{January 2008}{Dynamics Days, Knoxville, TN, USA\newline
%  	\siminos, P.\ Cvitanovi\'c, and R.\,L.\ Davidchack, {\emtitle State space geometry of a spatio-temporally chaotic Kuramoto-Sivashinsky flow}
% }
% \end{tabularx}

\subsection{Research results including patents, publications, teaching etc., taking into account the level of experience}
\begin{xcomment}
    Outline the major achievements of the researcher. These may also include results in the form of
  funded projects, publications, patents, reports, invited participation in conferences etc., taking into
  account the level of experience. To help the expert evaluators better understand the level of skills
  and experience it is advisable to write a short description (around 250 words) of the major
  accomplishments mentioning the purpose, results, skills acquired, derived applications etc.
\end{xcomment}

\subsection{Independent thinking and leadership qualities}
\begin{xcomment}
    Describe the activities that reflect initiative, independent thinking, project management skills and
  leadership. Describe the potential that the researcher has for increasing and reinforcing these
  qualities.
\end{xcomment}

\subsection{Match between the fellow's profile and project}
\begin{xcomment}
 Show that the applicant's skills and experience are suitable for the project proposed.
\end{xcomment}

\subsection{Potential for reaching or reinforcing a position of professional maturity}
\begin{xcomment}
 Describe the potential of the researcher to reach professional maturity through the proposed
fellowship.
\end{xcomment}


\subsection{Potential to acquire new knowledge}
\begin{xcomment}
 Describe the researcher's ability to acquire new knowledge and skills through the proposed
fellowship.
\end{xcomment}

\newpage
%%%%%%%%%%%%%%%%%%%%%%%%%%%%%%%%%%%%%%%%%%%%%%%%%%%%%%%%%%%%%%%%%%%%%%%%%%%%%%%%%%%%%%%%

%%%%%%%%%%%%%%%%%%%%%%%%%%%%%%%%%%%%%%%%%%%%%%%%%%%%%%%%%%%%%%%%%%%%%%%%%%%%%%%%%%%%%%%%
\section{Implementation}
\begin{xcomment}  
(Maximum 6 pages)
\end{xcomment}

\subsection{Quality of infrastructures/facilities and international collaborations of host}
\begin{xcomment}
  The host institution needs to specify the available infrastructures and whether these can respond
  to the needs set by the project. The host institution should further indicate to which extent the
  applicant can benefit from the host institution's participation in the international collaboration
  described in section B1.
\end{xcomment}

\subsection{Practical arrangements for the implementation and management of the research project }
\begin{xcomment}
  The applicant and the host institution must be able to provide information on how the
  implementation and management of the fellowship will be achieved. The experts will be
  examining the practical arrangements that can have an impact on the feasibility and credibility of
  the project.
\end{xcomment}

\subsection{Feasibility and credibility of the project, including work plan}\
\begin{xcomment}
 Feasibility and credibility of the project, including work plan
Provide a detailed work plan that includes the objectives and milestones that can help assess the
progress of the project. Where appropriate, describe the approach to be taken regarding the
intellectual property that may arise from the research project.
\end{xcomment}

\begin{ganttchart}[vgrid, hgrid, newline shortcut=true, bar label node/.append style={align=right}]{1}{24}
\gantttitle{Title}{24} \\
\gantttitlelist{3,6,9,12,15,18,21,24}{3} \\
\ganttbar{Literature search}{1}{24}\\
\ganttgroup{Single Particle/Chaos}{1}{10} \\
\ganttbar{Task 1 \ganttalignnewline batheh hhoh gagh}{1}{3} \\
\ganttbar{Task 2}{4}{10} \\
% \ganttmilestone{Milestone 1}{11}\\
\ganttbar{Dissesimination\ganttalignnewline publications/presentations}{9}{11}\\
\ganttgroup{Plasmas/Complexity}{11}{24} \\
\ganttbar{Separatrix tracking in PIC}{11}{13} \\
\end{ganttchart}



\subsection{Practical and administrative arrangements, and support for the hosting of the fellow  }
\begin{xcomment}
 Describe the practical arrangements in place to host a researcher coming from another country.
What support will be given to him/her to settle into their new host country (in terms of language
teaching, help with local administration, obtaining permits, accommodation, schools, childcare
etc.)?
\end{xcomment}

%%%%%%%%%%%%%%%%%%%%%%%%%%%%%%%%%%%%%%%%%%%%%%%%%%%%%%%%%%%%%%%%%%%%%%%%%%%%%%%%%%%%%%%%


%%%%%%%%%%%%%%%%%%%%%%%%%%%%%%%%%%%%%%%%%%%%%%%%%%%%%%%%%%%%%%%%%%%%%%%%%%%%%%%%%%%%%%%%
\newpage
%%%%%%%%%%%%%%%%%%%%%%%%%%%%%%%%%%%%%%%%%%%%%%%%%%%%%%%%%%%%%%%%%%%%%%%%%%%%%%%%%%%%%%%%
\section{Impact}
\begin{xcomment}  
(Maximum 4 pages)
\end{xcomment}
\subsection{Impact of competencies acquired during the fellowship on the future career prospects of the researcher, in particular through exposure to transferrable skills training with special attention to exposure to the industry sector, where appropriate}
\begin{xcomment}
 Describe the impact that competencies and skills acquired during the fellowship will have on the
prospects of reaching and/or reinforcing a position of professional maturity and/or research
independence.
\end{xcomment}

\subsection{Contribution to career development or re-establishment where relevant}
\begin{xcomment}
 Contribution to career development, or re-establishment where relevant
How will the fellowship contribute in the medium- and long-term to the development of the
fellow’s career? In the case of a fellow returning to research, how will his/her re-establishment be
helped by the fellowship?
\end{xcomment}

\subsection{Benefit of the mobility to the European Research Area}
\begin{xcomment}
 Describe how the proposed mobility is genuine and therefore beneficial to the European
Research Area. Genuine mobility is considered to allow the researcher to work in a significantly
different geographical and working environment, different from the one in which he has already
worked before.
\end{xcomment}

\subsection{Development of lasting cooperation and collaboration with other countries}
\begin{xcomment}
 What is the likelihood of creating collaboration between the host country and other countries after
the end of the fellowship?
\end{xcomment}

\subsection{Contribution to European excellence and European competitiveness regarding the expected research results}
\begin{xcomment}
Describe the extent to which the expected results of the project will increase European
excellence and ERA competitiveness and produce long-term synergies and/or structuring effects. 
\end{xcomment}

\subsection{Impact of the proposed outreach activities}
\begin{xcomment}
 Describe the outreach activities of the proposal to be implemented by the researcher during the
project duration (for examples, see box on Outreach Activities below).
\end{xcomment}

%%%%%%%%%%%%%%%%%%%%%%%%%%%%%%%%%%%%%%%%%%%%%%%%%%%%%%%%%%%%%%%%%%%%%%%%%%%%%%%%%%%%%%%%
\newpage
%%%%%%%%%%%%%%%%%%%%%%%%%%%%%%%%%%%%%%%%%%%%%%%%%%%%%%%%%%%%%%%%%%%%%%%%%%%%%%%%%%%%%%%%


% --------------------------------------------------------------- Section
\section{Ethical issues  (no maximum pages)}

MAIN ETHICAL ISSUES THAT MUST BE ADDRESSED
\begin{itemize}
\item Informed consent
\item Human embryonic stem cells
\item Privacy and data protection
\item Use of human biological samples and data
\item Research on animals
\item Research in developing countries
\item Dual use
\end{itemize}

AREAS EXCLUDED FROM FUNDING
\begin{itemize}
\item Research activity aiming at human cloning for reproductive
  purposes.
\item Research activity intended to modify the genetic heritage of
  human beings which could make such changes heritable (Research
  related to cancer treatment of the gonads can be financed).
\item Research activities intended to create human embryos solely for
  the purpose of research or for the purpose of stem cell procurement,
  including by means of somatic cell nuclear transfer.
\end{itemize}

Include the Ethical issues table below. If you indicate YES to any
issue, please identify the pages in the proposal where this ethical
issue is described. Answering 'YES' to some of these boxes does not
automatically lead to an ethical review. It enables the independent
experts to decide if an ethical review is required. If you are sure
that none of the issues apply to your proposal, simply tick the YES
box in the last row.

\clearpage

\subsection{Ethical issues table}

(Note: Research involving activities marked with an asterisk * in the
left column in the table below will be referred automatically to
Ethical Review)

% redefine vertically centered type for tabularx lines
\renewcommand{\tabularxcolumn}[1]{>{\arraybackslash}m{#1}}

\vspace{0.5cm}
%\noindent
\begin{tabularx}{\linewidth}{ | c | X | c | c | }
\rowcolor{black} & {\centering\arraybackslash \color{white} \bf Research on Human Embryo/Foetus} & {\color{white} \bf Yes} & {\color{white} \bf Page} \\ \hline
 * & Does the proposed research involve human Embryos?                                                      & & \\ \hline
 * & Does the proposed research involve human Foetal Tissues/ Cells?                                        & & \\ \hline
 * & Does the proposed research involve human Embryonic Stem Cells (hESCs)?                                 & & \\ \hline
 * & Does the proposed research on human Embryonic Stem Cells involve cells in culture?                     & & \\ \hline
 * & Does the proposed research on Human Embryonic Stem Cells involve the derivation of cells from Embryos? & & \\ \hline
   & I CONFIRM THAT NONE OF THE ABOVE ISSUES APPLY TO MY PROPOSAL                                           & & \cellcolor[gray]{0.8}\\ \hline 
\end{tabularx}

\vspace{0.5cm}
%\noindent
\begin{tabularx}{\linewidth}{ | c | X | c | c | }
\rowcolor{black} & {\centering\arraybackslash \color{white} \bf Research on Humans} & {\color{white} \bf Yes} & {\color{white} \bf Page} \\ \hline
 * & Does the proposed research involve children?                         & & \\ \hline
 * & Does the proposed research involve patients?                         & & \\ \hline
 * & Does the proposed research involve persons not able to give consent? & & \\ \hline
 * & Does the proposed research involve adult healthy volunteers?         & & \\ \hline
   & Does the proposed research involve Human genetic material?           & & \\ \hline
   & Does the proposed research involve Human biological samples?         & & \\ \hline
   & Does the proposed research involve Human data collection?            & & \\ \hline
   & I CONFIRM THAT NONE OF THE ABOVE ISSUES APPLY TO MY PROPOSAL         & & \cellcolor[gray]{0.8}\\ \hline 
\end{tabularx}

\vspace{0.5cm}
%\noindent
\begin{tabularx}{\linewidth}{ | c | X | c | c | }
\rowcolor{black} & {\centering\arraybackslash \color{white} \bf Privacy} & {\color{white} \bf Yes} & {\color{white} \bf Page} \\ \hline
   & Does the proposed research involve processing of genetic information or personal data (e.g. health, sexual lifestyle, ethnicity, political opinion, religious or philosophical conviction)? & & \\ \hline
   & Does the proposed research involve tracking the location or observation of people? & & \\ \hline
   & I CONFIRM THAT NONE OF THE ABOVE ISSUES APPLY TO MY PROPOSAL & & \cellcolor[gray]{0.8}\\ \hline
\end{tabularx}

\vspace{0.5cm}
%\noindent
\begin{tabularx}{\linewidth}{ | c | X | c | c | }
\rowcolor{black} & {\centering\arraybackslash \color{white} \bf Research on Animals} & {\color{white} \bf Yes} & {\color{white} \bf Page} \\ \hline
   & Does the proposed research involve research on animals?      & & \\ \hline
   & Are those animals transgenic small laboratory animals?       & & \\ \hline
   & Are those animals transgenic farm animals?                   & & \\ \hline
 * & Are those animals non-human primates?                        & & \\ \hline
   & Are those animals cloned farm animals?                       & & \\ \hline
   & I CONFIRM THAT NONE OF THE ABOVE ISSUES APPLY TO MY PROPOSAL & & \cellcolor[gray]{0.8}\\ \hline
\end{tabularx}

\vspace{0.5cm}
%\noindent
\begin{tabularx}{\linewidth}{ | c | X | c | c | }
\rowcolor{black} & {\centering\arraybackslash \color{white} \bf Research Involving Developing Countries} & {\color{white} \bf Yes} & {\color{white} \bf Page} \\ \hline
   & Does the proposed research involve the use of local resources (genetic, animal, plant, etc)?                             & & \\ \hline
   & Is the proposed research of benefit to local communities (e.g. capacity building, access to healthcare, education, etc)? & & \\ \hline
   & I CONFIRM THAT NONE OF THE ABOVE ISSUES APPLY TO MY PROPOSAL                                                             & & \cellcolor[gray]{0.8}\\ \hline
\end{tabularx}

\vspace{0.5cm}
%\noindent
\begin{tabularx}{\linewidth}{ | c | X | c | c | }
\rowcolor{black} & {\centering\arraybackslash \color{white} \bf Dual Use} & {\color{white} \bf Yes} & {\color{white} \bf Page} \\ \hline
   & Research having direct military use                          & & \\ \hline
   & Research having the potential for terrorist abuse            & & \\ \hline
   & I CONFIRM THAT NONE OF THE ABOVE ISSUES APPLY TO MY PROPOSAL & & \cellcolor[gray]{0.8}\\ \hline
\end{tabularx}



%%%%%%%%%%%%%%%%%%%%%%%%%%%%%%%%%%%%%%%%%%%%%%%%%%%%%%%%%%%%%%%%%%%%%%%%%%%%%%%%%%%%%%%%
%%%%%%%%%%%%%%%%%%%%%%%%%%%%%%%%%%%%%%%%%%%%%%%%%%%%%%%%%%%%%%%%%%%%%%%%%%%%%%%%%%%%%%%%
\cover{ENDPAGE}

\end{document}