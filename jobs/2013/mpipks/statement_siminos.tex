%% start of file `template_en.tex'.
%% Copyright 2007 Xavier Danaux (xdanaux@gmail.com).
%
% This work may be distributed and/or modified under the
% conditions of the LaTeX Project Public License version 1.3c,
% available at http://www.latex-project.org/lppl/.


\documentclass[11pt,a4paper,final]{moderncv}

% moderncv themes
% \moderncvtheme[blue]{casual}                 % optional argument are 'blue' (default), 'orange', 'red', 'green', 'grey' and 'roman' (for roman fonts, instead of sans serif fonts)
\moderncvtheme[grey]{classic}                % idem

% character encoding
\usepackage[utf8]{inputenc}                   % replace by the encoding you are using

% 
% \usepackage{hyperref} %ES
\usepackage{url} %ES
\usepackage{ifthen}
\usepackage{multicol} %ES
\usepackage{textcomp} %ES

\newif\ifpaper 

\papertrue % For committees, usually printed out and handed to members.
% \paperfalse % For individuals, use [doi] and [pdf] links to publications


% adjust the page margins
\usepackage[scale=0.8]{geometry}
\setlength{\hintscolumnwidth}{0.4cm}			% if you want to change the width of the column with the dates
\AtBeginDocument{\setlength{\maketitlenamewidth}{7cm}}  % only for the classic theme, if you want to change the width of your name placeholder (to leave more space for your address details
\AtBeginDocument{\recomputelengths}                     % required when changes are made to page layout lengths

\makeatletter
\renewcommand*{\bibliographyitemlabel}{\@biblabel{\arabic{enumiv}}}
\makeatother

\input ../../../CV/defs

% personal data
\firstname{\large{Evangelos}}
\familyname{\large{Siminos}}
 \title{\large{Activity Report and Research Proposal}}               % optional, remove the line if not wanted
\address{Max Planck Institute for the Physics of Complex Systems\\
	  N\"{o}thnitzer Str. 38}
	{01187 Dresden, Germany}% optional, remove the line if not wanted
% \mobile{mobile (optional)}                    % optional, remove the line if not wanted
\phone{+49 351 871 2412}                      % optional, remove the line if not wanted
% \fax{fax (optional)}                          % optional, remove the line if not wanted
\email{evangelos.siminos@gmail.com}% optional, remove the line if not wanted
% \email{evangelos.siminos@cea.fr}% optional, remove the line if not wanted
% \extrainfo{\href{http://www.cns.gatech.edu/~siminos}{\url{www.cns.gatech.edu/~siminos}}} % Does not work
 \extrainfo{\httplink[http://www.cns.gatech.edu/$\sim$siminos]{www.cns.gatech.edu/~siminos}} % optional, remove the line if not wanted
% \photo[80pt]{siminos_small.jpg}                         % '64pt' is the height the picture must be resized to and 'picture' is the name of the picture file; optional, remove the line if not wanted
% \quote{Some quote (optional)}                 % optional, remove the line if not wanted
% \nopagenumbers{}                             % uncomment to suppress automatic page numbering for CVs longer than one page



\date{\today}

\lfoot{E. Siminos - Activity Report and Research Proposal}

%----------------------------------------------------------------------------------
%            content
%----------------------------------------------------------------------------------
\begin{document}
\maketitle

% \setlength{\parindent}{0.25in} % ES: For research statement. After maketitle!

I joined the \emph{Computational Nonlinear and Relativistic Optics} group as a postdoctoral researcher in August 2011.\sep

I was able to initiate collaborations both within our group and with various workshop participants in topics which include
high intensity laser-plasma interaction, nonlinear dynamics of solitary waves in nonlocal nonlinear-media and soliton interaction
in plasmas.\sep

More recently I've been involved in a collaboration with the experimental group of Prof. Kaluza in Helmholtz Institute Jena, 
in which our role is to provide insight to experimental results by simulating the complete pump-probe setup. An incoming 
PhD student (Syed Ali Hussain) is expected to contribute to this project.\sep

At the same time, we are diverting our interests towards methods of attosecond pulse generation through laser-plasma interaction,
a topic which is both related to our previous studies but also meshes very well with the interests within the Finite Systems Division.
For this purpose a short-term postdoctoral fellow (H. Vincenti) has been recruited with the goal of establishing connections 
with ongoing experiments at CEA/Saclay.\sep

The common thread in all the above studies is provided by nonlinear dynamics, as will be explained in detail in the following.\sep


\section{Research activities at MPIPKS}
\sep
\inlinesubsect{Relativistic intensity laser-matter interaction}
Laser pulses produced by state of the art systems ionize any material they go through, 
producing a plasma, while accelerating electrons to speeds close to the speed of light within a femtosecond. 
Under such conditions, the optical properties of matter are fundamentally altered, to such an extend that a new field, that of relativistic optics, was born. 
In order to study propagation of such pulses in a medium, one has to take into account the complex interactions of a many-body system with the electromagnetic field.
The resulting nonlinear system of coupled partial differential equations (relativistic Vlasov equation coupled to Maxwell equations) is hard to solve, 
even in powerful parallel supercomputers and even simple questions are hard to answer. 
My research at the MPIPKS is centered around such a simple question: given a pulse of certain intensity, how can we determine the the critical density, 
i.e. the maximum density of a target which would allow propagation of an intense laser pulse? Numerical experiments suggest that 
an intense pulse can propagate in an, otherwise opaque, medium by heating and expelling electrons out of the medium, 'drilling' its way through it. 
To understand this effect we turned to the study of phase-space topology of a simple dynamical system describing electron motion in the laser field. 
Determining separatrices in phase space which act as transport barriers, preventing electrons to escape the potential well created by radiation pressure, 
allows a clear understanding of how pulse propagation is triggered.
\sep 

\inlinesubsect{Nonlinear dynamics of solitons in nonlocal nonlinear media}
Quasiperiodic oscillations and shape-transformations of higher-order bright solitons in nonlinear nonlocal media have been frequently observed in recent years, 
however, the origin of these phenomena was never completely elucidated. In this paper, we perform a linear stability analysis of these higher-order solitons 
by solving the Bogoliubov-de Gennes equations. This enables us to understand the emergence of a new oscillatory state 
as a growing unstable mode of a higher-order soliton. Using dynamically important states as a basis, we provide low-dimensional 
visualizations of the dynamics and identify quasiperiodic and homoclinic orbits, linking the latter to shape-transformations. 
\sep

\inlinesubsect{Relativistic solitary wave interaction} 
Numerical investigations on mutual interactions between two spatially overlapping standing electromagnetic solitons 
in a cold unmagnetized plasma are reported. It is found that an initial state comprising of two overlapping standing solitons 
evolves into different end states, depending on the amplitudes of the two solitons and the phase difference between them. 
For small amplitude solitons with zero phase difference, we observe the formation of an oscillating bound state whose 
period depends on their initial separation. These results suggest the existence of a bound state made of two solitons 
in the relativistic cold plasma fluid model.

\inlinesubsect{Periodic orbit theory of Lyapunov vectors}

\section{Proposed Research}
\sep
\inlinesubsect{Can we see the shape of a plasma wake?}
As an ultra-high-intensity laser pulse propagates through a plasma, it leaves behind it a wake (an electron plasma wave) which can act as an accelerating
structure for plasma particles. Taming the dynamics of electrons inside such wakes has been an area of active experimental and theoretical research over 
the last decade, since there is hope that laser-plasma accelerators can surpass the performance of conventional accelerators by supporting 
much higher accelerating gradients. Although experimentally obtained spectra are consistent with the so-called bubble acceleration mechanism,
there has yet been no direct observation of the accelerating wakefields within the plasma. Ongoing experiments with the POLARIS laser at the
Helmholtz Institute Jena, provide access for the first time to coherent electron structures through a pump-probe setup that utilizes shadowgraphy 
through a $6fs$ duration probe pulse. Our group participates in this study by providing theoretical support through PIC simulations in order to
gain insight into the imaging process in different regimes of interaction. This technique is expected to be useful in a variety of 
laser-plasma interaction settings beyond laser wakefield acceleration. As an example, preliminary experimental results at the same facility indicate that solitary waves 
are excited and interact with each other at the cavity left behind the wake. We plan to investigate this issue by applying the expertise we gained through
our previous studies of soliton excitation and interaction (in collaboration with V. Saxena who is applying to join our group as a postdoctoral fellow).
This would be the first time that soliton excitation and interaction is directly observed in a plasma.
\sep


\inlinesubsect{Attosecond pulse generation}
High-harmonic generation (HHG) by irradiation of solid targets with intense femtosecond pulses provides novel mechanisms for attosecond pulse
generation with high conversion efficiency. Attosecond pulses can be used as probes of processes at the atomic level, but also provide information
about processes in plasma which generate them. In collaboration with H. Vincenti, S. Skupin (MPIPKS), M. Grech (LULI) and G. Bonnaud (CEA/Saclay) 
we began to study the effect of ionization in the so-called coherent wave emmision (CWE) mechanism of attosecond pulse generation. Our goal is
to propose an experimentally realizable mechanism based on the concept of attosecond lighthouse that would allow to recover time- and spatially-resolved
information on the ionization process of an overdense plasma.
\sep


\inlinesubsect{Signatures of relativistic chaos in laser-plasma interactions}
The proposed project is devoted to the study of signatures of chaos in the dynamics of particles under ultra-high-intensity (UHI) laser pulses, with an emphasis
in the implications for applications such as particle acceleration and the generation of energetic, ultra-short electron bunches. 
We intent to develop a unified description of both laser-beam (single electron) and laser-plasma interactions, through the study of the effects of phase-space
topology to the quality and energy characteristics of the produced particle beams. Moreover, we will study the effects of radiation reaction (electron self-force), which
is expected to be important in forthcoming experimental facilities, to phase-space topology and through it detect its experimentally measurable signatures.
\\ \\ \\ 

\bibliographystyle{../../../tex/statement2011} %../../tex/poster
\bibliography{../../../siminos}

\end{document}


%% end of file `template_en.tex'.
