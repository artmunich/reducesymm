\documentclass[a4paper,10pt]{article}

\usepackage{hyperref}
\usepackage{url}
\usepackage{color}

\pagestyle{plain}

\setlength{\oddsidemargin}{0in}\setlength{\evensidemargin}{\oddsidemargin}
\setlength{\textwidth}{1.3\textwidth}
\setlength{\textheight}{1.1\textheight}
\voffset-0.3in
\hoffset-0.2in
\topmargin 0.0in
\headsep 0.0in
\headheight 0.0in

%opening
\title{Research Plan}
\author{Evangelos Siminos}
\date{November 2010}

\definecolor{darkblue}{rgb}{0.13,0.17,0.63}

\hypersetup{colorlinks=true,urlcolor=darkblue}
\definecolor{see}{rgb}{0.5,0.5,0.5}% for web links

% ES copied from cv_libertine
\newcommand{\html}[1]{\href{#1}{\color{see}\scriptsize\textsc{[html]}}}
\newcommand{\pdf}[1]{\href{#1}{\color{see}\scriptsize\textsc{[pdf]}}}
\newcommand{\doi}[1]{\href{#1}{\color{see}\scriptsize\textsc{[doi]}}}


\begin{document}

\maketitle

My research interests lie in the area of nonlinear dynamics of 
spatially extended systems. My focus is on the application of methods
of dynamical system theory in diverse physical settings including
the kinetic description of collective effects in 
systems with long-range interactions,
relativistic intensity laser-plasma interaction 
and phase turbulence.

My recent work is related to the kinetic theory description of stimulated 
Raman scattering (SRS), a process that involves the
interaction of a laser pulse and a nonlinear electrostatic wave in a plasma. SRS
is a major source of energy loss for current inertial confinement fusion efforts, while
a theoretical understanding of the saturation of its growth is still missing. It
has been recently suggested that the saturation of SRS growth may be attributed
to the loss of linear stability of the electrostatic wave. However, a general
and robust method to compute stability of nonlinear electrostatic waves did not
exist. The approach I have developed, in collaboration with D. B\'enisti and L.
Gremillet~\cite{siminos11} as a postdoc at CEA, 
involves reduction to a finite-dimensional
eigenproblem through expansion of the relevant equations, the Vlasov-Poisson
system, in a basis of orthogonal functions. Despite the huge success of
similar methods in fluid dynamics, the conservative nature of the Vlasov
equation and the coexistence of an extreme range of velocity scales inhibit
convergence of the eigenvalue calculation. As a solution to this convergence
problem we introduced spectral deformation techniques
% , most commonly employed in
% quantum mechanics, to
to separate unstable collective modes of interest from modes involving 
irrelevant velocity scales. 
This led to a very practical, fast converging scheme to compute the 
few most unstable modes of nonlinear Vlasov-Poisson equilibria.
From a theoretical perspective
this contributed to a better understanding of the `vortex fusion'
phenomenon in electrostatic waves. From a practical point of view it opens 
new possibilities for the control of systems with long range interactions 
through the selective excitation of collective modes.

My earlier PhD thesis work at the Georgia Institute of Technology
involved the study, with P. Cvitanovi\'c and R. L. Davidchack~\cite{SCD07}, 
of the Kuramoto-Sivashinsky system
as a one-dimensional 
toy-model for turbulence~\cite{SCD07}. 
In dynamical systems studies of turbulence the goal is to understand
the organization of the infinite-dimensional state space 
in terms of equilibria, periodic orbits and their
stable and unstable manifolds, which serve to connect local neighborhoods.
However, when traveling wave solutions are taken into account, the geometric
approach is obscured by the presence of equivalent, up to a continuous symmetry
transformation, trajectories. The main achievement of my
thesis~\cite{SiminosThesis,SiCvi10} was to show that continuous symmetry
reduction, i.e. the identification of symmetry related trajectories, can be
implemented in a (moderately) high-dimensional state-space and that the
geometric approach of dynamical systems can then be successfully applied.

These theoretical studies fueled my contribution to a problem of great
practical interest in the growing field of laser-matter interaction. 
The current trend is to develop compact accelerators that
exploit short, high-intensity laser pulses to accelerate particles
in much higher energies than possible with conventional accelerators 
of comparable size. Short laser pulses are hard to produce and control and
therefore an open question is whether longer pulses can be utilized for
the same purpose of particle acceleration. A first step to tackle the
problem is to connect it to the existence of solitary waves 
in a fluid-plasma electromagnetic model which could act as a mean to transfer   
energy to the particles through wavebreaking or parametric instabilities.
With G. S\'anchez-Arriaga and E. Lefebvre~\cite{SSL10}, we reduced
the problem to that of finding homoclinic and heteroclinic connections
of a Hamiltonian system of ordinary differential equations. Then,
dynamical systems theory allowed a systematic determination and classification
of solitary wave solutions. We are currently working on the problem
of stability and practical excitation of such solutions in laser-matter
interaction.

My perspective for future research is still towards cross-fertilization
between different areas of physics, through the unifying framework of
dynamical systems. In particular, I am increasingly interested in applications
of nonlinear dynamics tools in biological systems.
There is a close analogy of the Vlasov dynamics to the Kuramoto model~\cite{strogatz00} 
of coupled oscillators and my short term goal is to exploit the
framework developed for the stability of Vlasov-Poisson equilibria 
in the study of collective modes of partially synchronized systems 
of nonlinear oscillators. A possible area of subsequent generalization is the study
of systems of oscillators with intermediate range coupling, which has
received recent attention and is of direct 
relevance to biological systems~\cite{martens10,motter10}.
I believe that Nordita's active \emph{Condensed Matter, Statistical 
and Biological Physics}
group would offer me a great opportunity to realize this study and explore
its possible applications in biological physics.




\bibliography{../../bibtex/plasmas}
\bibliographystyle{unsrt}


\end{document}

