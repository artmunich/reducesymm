\documentclass[a4paper,10pt]{article}

\usepackage{hyperref}
\usepackage{url}
\usepackage{color}

\pagestyle{plain}

\setlength{\oddsidemargin}{0in}\setlength{\evensidemargin}{\oddsidemargin}
\setlength{\textwidth}{1.3\textwidth}
\setlength{\textheight}{1.5\textheight}
\voffset-0.3in
\hoffset-0.2in
\topmargin 0.0in
\headsep 0.0in
\headheight 0.0in

%opening
\title{Research Statement}
\author{Evangelos Siminos}
\date{November 2010}

\definecolor{darkblue}{rgb}{0.13,0.17,0.63}

\hypersetup{colorlinks=true,urlcolor=darkblue}
\definecolor{see}{rgb}{0.5,0.5,0.5}% for web links

% ES copied from cv_libertine
\newcommand{\html}[1]{\href{#1}{\color{see}\scriptsize\textsc{[html]}}}
\newcommand{\pdf}[1]{\href{#1}{\color{see}\scriptsize\textsc{[pdf]}}}
\newcommand{\doi}[1]{\href{#1}{\color{see}\scriptsize\textsc{[doi]}}}


\begin{document}

\maketitle

My research interests lie in the area of nonlinear dynamics of 
spatially extended systems. I focus on the application
of methods of dynamical systems theory and the development of
new computational techniques suitable for high-dimensional
systems. 

My PhD thesis work at the Georgia Institute of Technology with P.~Cvitanovi\'c
involved the study, of the
Kuramoto-Sivashinsky system as a one-dimensional 
toy-model for turbulence~\cite{SCD07}. 
In dynamical systems studies of turbulence the
organization of the infinite-dimensional state space 
can be understood in terms of equilibria, periodic orbits and their
stable and unstable manifolds, which serve to connect local neighborhoods.
However, when traveling wave solutions are taken into account, the geometric
approach is obscured by the presence of equivalent, up to a continuous symmetry
transformation, trajectories. The main achievement of my
thesis~\cite{SiminosThesis,SiCvi10} was to show that continuous symmetry
reduction, i.e. the identification of symmetry related trajectories, can be
implemented in a (moderately) high-dimensional state-space and that the
geometric approach of dynamical systems can then be successfully applied.

More recently, as a postdoc at CEA, I've been involved in the kinetic theory
description of stimulated Raman scattering (SRS), a process that involves the
interaction of a laser pulse and a nonlinear electrostatic wave in a plasma. SRS
is a major source of energy loss for current inertial confinement efforts, while
a theoretical understanding of the saturation of its growth is still missing. It
has been recently suggested that the saturation of SRS growth may be attributed
to the loss of linear stability of the electrostatic wave. However, a general
and robust method to compute stability of nonlinear electrostatic waves did not
exist. The approach I have developed, in collaboration with D. B\'enisti and L.
Gremillet~\cite{siminos11}, involves reduction to a finite-dimensional
eigenproblem through expansion of the relevant equations, the Vlasov-Poisson
system, in a basis of orthogonal functions. Despite the huge success of
similar methods in fluid dynamics, the conservative nature of the Vlasov
equation and the coexistence of an extreme range of velocity scales inhibit
convergence of the eigenvalue calculation. As a solution to this convergence
problem we introduced spectral deformation techniques
% , most commonly employed in
% quantum mechanics, to
to separate unstable collective modes of interest from modes involving 
irrelevant velocity scales. 
This led to a very practical, fast converging scheme to compute the 
few most unstable modes for the Vlasov-Poisson system.

I find chaotic advection an interesting subject to which I can contribute, as it
involves nonlinear dynamics in fluid flows. Although I have not been actively
involved in chaotic advection, I have the theoretical and computational training
required to conduct research in the field and I have always found the physics
involved very intriguing. A further reason for which I would like 
to do research in your group is your work in synchronization. 
The close analogy of the Vlasov
dynamics to the Kuramoto model of coupled oscillators has been known, and
exploited, since the nineties. I hope that our latest progress in computing
stability of Vlasov-Poisson equilibria could be of value in problems involving
coupled oscillators.

\bibliography{../../bibtex/plasmas}
\bibliographystyle{unsrt}


\end{document}

