%% start of file `template_en.tex'.
%% Copyright 2007 Xavier Danaux (xdanaux@gmail.com).
%
% This work may be distributed and/or modified under the
% conditions of the LaTeX Project Public License version 1.3c,
% available at http://www.latex-project.org/lppl/.


\documentclass[11pt,a4paper,final]{moderncv}

% moderncv themes
% \moderncvtheme[blue]{casual}                 % optional argument are 'blue' (default), 'orange', 'red', 'green', 'grey' and 'roman' (for roman fonts, instead of sans serif fonts)
\moderncvtheme[grey]{classic}                % idem

% character encoding
\usepackage[utf8]{inputenc}                   % replace by the encoding you are using

% 
% \usepackage{hyperref} %ES
\usepackage{url} %ES
\usepackage{ifthen}
\usepackage{multicol} %ES
\usepackage{textcomp} %ES

\newif\ifpaper 

% \papertrue % For committees, usually printed out and handed to members.
\paperfalse % For individuals, use [doi] and [pdf] links to publications


% adjust the page margins
\usepackage[scale=0.8, top=0.3in,bottom=0.5in]{geometry}
\setlength{\hintscolumnwidth}{0.4cm}			% if you want to change the width of the column with the dates
\AtBeginDocument{\setlength{\maketitlenamewidth}{7cm}}  % only for the classic theme, if you want to change the width of your name placeholder (to leave more space for your address details
\AtBeginDocument{\recomputelengths}                     % required when changes are made to page layout lengths

\makeatletter
\renewcommand*{\bibliographyitemlabel}{\@biblabel{\arabic{enumiv}}}
\makeatother

\input ../../vita/defs

% personal data
\firstname{\Huge{Evangelos}}
\familyname{\Huge{Siminos}}
 \title{Research Statement}               % optional, remove the line if not wanted
\address{D\'{e}partement de Physique Th\'{e}orique et Appliqu\'{e}e\\ 
Commissariat \`a l' \'Energie Atomique\\ 
CEA, DAM, DIF}{ F-91297 Arpajon, France}    % optional, remove the line if not wanted
% \mobile{mobile (optional)}                    % optional, remove the line if not wanted
\phone{+33 169 267 361}                      % optional, remove the line if not wanted
% \fax{fax (optional)}                          % optional, remove the line if not wanted
\email{siminos@gatech.edu}% optional, remove the line if not wanted
% \email{evangelos.siminos@cea.fr}% optional, remove the line if not wanted
\extrainfo{\href{http://www.cns.gatech.edu/~siminos}{\url{www.cns.gatech.edu/~siminos}}} % 
% \extrainfo{\httplink[http://www.cns.gatech.edu/~siminos]{www.cns.gatech.edu/~siminos}} % optional, remove the line if not wanted
% \photo[80pt]{siminos_small.jpg}                         % '64pt' is the height the picture must be resized to and 'picture' is the name of the picture file; optional, remove the line if not wanted
% \quote{Some quote (optional)}                 % optional, remove the line if not wanted

\nopagenumbers{}                             % uncomment to suppress automatic page numbering for CVs longer than one page



\date{\today}

% \lfoot{E. Siminos}

%----------------------------------------------------------------------------------
%            content
%----------------------------------------------------------------------------------
\begin{document}
 \maketitle

%  \setlength{\parindent}{0.25in} % ES: For research statement. After maketitle!

% My research interests lie in the area of nonlinear dynamics of 
% spatially-extended, complex systems. Recent projects include 
% the study of spatio-temporal chaos, stability and collective effects 
% in systems with long-range interactions, and soliton formation in laser-plasma 
% interaction. The unifying framework of these studies is provided 
% by dynamical systems theory.

\section{Recent and Current Research}
\sep
\inlinesubsect{Collective phenomena in plasmas}
As a postdoc at CEA, with D. B\'enisti, I've been studying kinetic 
effects in laser-plasma interaction problems. An issue of great
interest to current inertial confinement fusion efforts is the stability 
of nonlinear electrostatic plasma waves, for the study of which no well established 
and universal method was yet available. 
% Specifically, I study  
% the Vlasov-Poisson system (collisionless Boltzmann 
% equation coupled to a self-consistent electric field in a plasma) in relation to 
% collective phenomena in laser-plasma interaction problems. 
Standard numerical tools (such as projection onto a finite-dimensional basis) 
face fundamental difficulties due to the Hamiltonian structure, 
long-range interactions, extreme range of scales and continuous linear spectrum 
in the Vlasov-Poisson system, which governs the evolution of such waves. 
These difficulties were resolved in our study using 
an operator-theoretic technique (spectral deformation), 
which establishes a formal connection to dissipative systems and suppresses 
non-essential scales~\cite{siminos11}. This formulation allows the determination 
of the unstable eigenmodes of a Vlasov equilibrium and illuminates 
the physical mechanism underlying the instability.
We are currently applying our method to establish a connection between
instabilities of nonlinear electrostatic waves excited by laser-plasma
interaction and the saturation of stimulated Raman scattering, %~\cite{benisti10-1}, 
a process detrimental to inertial confinement fusion.\sep

\inlinesubsect{Relativistic solitary waves} 
A parallel effort at CEA is the study, with G.~S\'anchez-Arriaga and 
E.~Lefebvre, of relativistic solitary waves in laser-plasma interaction 
problems. 
% A current trend is to develop compact accelerators that
% exploit short, high-intensity laser pulses to accelerate particles
% in much higher energies than possible with conventional accelerators 
% of comparable size. Short laser pulses are hard to produce and control and
% therefore 
The question we wish to address is whether relatively long pulses can be utilized for
the purpose of ion acceleration. A first step to tackle the
problem is to connect it to the existence of solitary waves 
(within a fluid-plasma electromagnetic model) which could act as a means to transfer   
energy to the particles through wavebreaking or parametric instabilities. 
Mathematically, such solitary waves can be thought of as homoclinic
and heteroclinic connections of a certain Hamiltonian dynamical system.
My contribution was to deduce the existence and provide a classification of new
families of solutions, using general geometrical arguments 
involving the symmetries of the system and the dimensionality 
of invariant subspaces~\cite{SSL11}. Having located numerically a great number of such
solutions, we currently study their stability and the possibility 
of their excitation in laser-plasma interaction experiments.\sep

\inlinesubsect{Symmetry reduction in chaotic and turbulent flows}
My PhD thesis work at the Georgia Institute of Technology, 
with Prof. Cvitanovi\'c,
focused on the interplay between symmetry and nonlinear dynamics
in spatially-extended systems with coherent structures. Taking as an example
the Kuramoto-Sivashinsky system, a one-dimensional 
partial differential equation that exhibits phase-turbulence, we seek
a geometrical description as a dynamical system in a high-dimensional 
state space. This would eventually lead to quantitative prediction of 
statistical averages with tools of dynamical systems theory. 
However, when the dynamics admit a continuous
symmetry, the geometric picture is often obscured by the presence of equivalence
classes of solutions~\cite{SCD07}. To alleviate this difficulty, I worked on symmetry
reduction for Lie groups acting on high-dimensional spaces, a problem
for which classical Hilbert-basis approaches are not applicable. Our proposed 
solution is to use a geometrical approach based on Cartan's moving frames,
resulting to effective numerical schemes for symmetry reduction 
in higher-dimensional spaces. 
Application in the Kuramoto-Sivashinsky and complex Lorenz systems 
allowed an elucidation of the role unstable manifolds of certain traveling wave 
solutions play in organizing the 
global geometry~\cite{SiCvi10,SiminosThesis}.

% \section{Future Research}
% \sep
% \inlinesubsect{Collective phenomena in mesoscopic systems}
% My plan for future research is to keep visiting different areas of
% physics that I find interesting and challenging -- while maintaining 
% the dynamical systems' perspective. An area that gains momentum
% in plasma physics is the study of quantum plasmas: dense, relatively cold
% plasmas in which the quantum 
% character of the underlying many-body system cannot be ignored. Quantum
% plasmas are characteristic of extreme states of matter, such as the interiors
% of white dwarf stars or compressed targets 
% in forthcoming inertial confinement fusion experiments~\cite{shukla_spin_2009}. 
% However, what I find
% more interesting is that a quantum plasma description can be employed for
% mesoscopic scale phenomena that are currently accessible in the laboratory, 
% for instance electron dynamics in thin metal films and metal clusters, semiconductor 
% nanostructures and quantum wells~\cite{manfredi05}. Studying such systems 
% brings together fluid mechanics and solid state physics and offers the
% opportunity to reveal novel effects in both disciplines. 
% Collective, non-linear effects can be
% addressed within a kinetic framework based on Wigner equation 
% (or its semiclassical Vlasov limit), thus letting me extend my scope in a very
% natural way. At the same time, such studies mesh very well with the research 
% on mesoscopic many-body physics of MPIPKS members across divisions 
% (e.g. in Condensed Matter and Finite Systems divisions, but also in 
% several Junior Research Groups). Thus, MPIPKS would be an ideal place 
% to undertake a study of collective effects in quantum plasmas, specifically
% oriented towards mesoscopic systems.
% \\ \\ \\ 
%  At this
% point my research interests meet with those of MPIPKS members 
% across divisions (e.g. in Condensed Matter and Finite Systems divisions, 
% but also in several Junior Research Groups). A particularly useful and effective
% level of description is through a kinetic formulation, namely the 
% Wigner equation (whose classical limit is the Vlasov equation). My plan is to
% extend the method for stability calculations developed in Ref~\cite{siminos11}
% to allow the study of nonlinear collective effects in 
% quantum many-body systems, while at the same time 
% My current interests have shifted towards many-body, long-range interacting
% systems at the mesoscopic scale, at which 
% 
% quantum-plasma description of mesoscopic systems such as metal 
% films and clusters or semiconductor nanosctructures.
% 
% The approach developed for the study of collective unstable modes in the
% Vlasov-Poisson system is rather general and can be applied in transport
% equations for systems with long-range interactions that extend well 
% beyond plasma physics. Particularly interesting are systems in the mesoscopic
% scale, in which the quantum character of the underlying many-body system 
% has to be accounted for. An appropriate and useful desription is 
% through a mean-field, kinetic approximation, namely the Wigner-Poisson system,
% (whose classical limit is the Vlasov-Poisson system) and which leads to a
% quantum-plasma description of electron dynamics in many interesting 
% and diverse systems such as   [...] Nonlinear collective-effects 
% 
% 
% Not only is a study of nonlinear collective effects in the mesoscopic scale
% timely in terms of applications but furthermore meshes very 
% well with the research on mesoscopic many-body physics of MPIPKS members 
% across divisions (e.g. in Condensed Matter and Finite Systems divisions, 
% but also in several Junior Research Groups). 
% % As kinetic numerical simulations
% % are expensive (both in the classical and the quantum regimes), the
% % dynamical systems approach which characterizes my work can offer 
% At the same time it will let me expand my scope towards mesoscopic physics
% while still allowing the use of many of the tools already available to me.
% \nocite{*}
\bibliographystyle{../../tex/statement2011} %../../tex/poster
\bibliography{publications}

\end{document}


%% end of file `template_en.tex'.

