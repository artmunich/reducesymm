\documentclass[a4paper,10pt]{article}

\usepackage{hyperref}
\usepackage{url}
\usepackage{color}

\pagestyle{plain}

\setlength{\oddsidemargin}{0in}\setlength{\evensidemargin}{\oddsidemargin}
\setlength{\textwidth}{1.3\textwidth}
\setlength{\textheight}{1.1\textheight}
\voffset-0.3in
\hoffset-0.2in
\topmargin 0.0in
\headsep 0.0in
\headheight 0.0in

%opening
\title{Notes on: Kinetic approach to Magnetic Reconnection.}
\author{Evangelos Siminos}
\date{\today}

\definecolor{darkblue}{rgb}{0.13,0.17,0.63}

\hypersetup{colorlinks=true,urlcolor=darkblue}
\definecolor{see}{rgb}{0.5,0.5,0.5}% for web links

% ES copied from cv_libertine
\newcommand{\html}[1]{\href{#1}{\color{see}\scriptsize\textsc{[html]}}}
\newcommand{\pdf}[1]{\href{#1}{\color{see}\scriptsize\textsc{[pdf]}}}
\newcommand{\doi}[1]{\href{#1}{\color{see}\scriptsize\textsc{[doi]}}}


\begin{document}

\maketitle

% The proposed research will built on recent progress~\cite{siminos11} 
% by the PR and collaborators in the study of kinetic instabilities in plasmas,
% to attack the problem of magnetic reconnection from a kinetic theory perspective.

\footnote{This paragraph might make it to the proposal, but it will certainly
be made shorter.} The PR's recent work is related 
to the kinetic theory description of stimulated 
Raman scattering (SRS), a process that involves the
interaction of a laser pulse and a nonlinear electrostatic wave in a plasma. 
SRS is a major source of energy loss for current inertial confinement 
fusion efforts, while a theoretical understanding of the saturation 
of its growth is still missing. It has been recently suggested that 
the saturation of SRS growth may be attributed
to the loss of linear stability of the electrostatic wave. However, a general
and robust method to compute stability of nonlinear electrostatic waves did not
exist. The approach the PR has developed, in collaboration with D. B\'enisti and L.
Gremillet~\cite{siminos11} as a postdoc at CEA, 
involves reduction to a finite-dimensional
eigenproblem through expansion of the relevant equations, the Vlasov-Poisson
system, in a basis of Fourier-Hermite functions (in $x$ and $v$, respectively). 
Despite the huge success of
similar methods in fluid dynamics, the conservative nature of the Vlasov
equation and the coexistence of an extreme range of velocity scales inhibit
convergence of the eigenvalue calculation. As a solution to this convergence
problem we introduced spectral deformation techniques
% , most commonly employed in
% quantum mechanics, to
to separate unstable collective modes of interest from modes involving 
irrelevant velocity scales. This led to a very practical, 
fast converging scheme to compute the few most unstable modes of nonlinear 
Vlasov-Poisson equilibria.
From a theoretical perspective this contributed to a better understanding 
of the `vortex fusion' instability in electrostatic waves. 
From a practical point of view it opens new possibilities for the control 
of systems with long range interactions 
through the selective excitation of collective modes.

There are three main computational advantages of this method. 
The equilibrium under question does not need to be known in closed form, 
so that empirical data (computational or experimental) 
can actually be addressed. Furthermore, there is a direct relation between
physical moments of the probability distribution function such as the presure
and electric field, to the Hermite coefficients, so that no post-processing
of the unstable modes is required to acquire information on the moments. Most
importantly, owing to the seperation of filamentation and collective scales
that we achieved by the introduction of spectral deformation, we can achieve
a very precise computation of the growth rate and
description of the unstable modes with very small computational cost
compared to Vlasov codes. In particular the number of variables needed for
accurate representation of the unstable modes is at least an order of magnitude
smaller than with an Eulerian Vlasov code.

We therefore expect that this approach will be generalized to the Vlasov-Maxwell
system in higher phase-space dimensions and in problems relevant to magnetic
reconnection in astrophysical plasmas. 
In particular, kinetic effects were seen to be important and not accounted
for in [Onofri et al, PRL 2006] study of electron and proton acceleration 
in three-dimensional electric and magnetic fields. 




\bibliography{../../bibtex/plasmas}
\bibliographystyle{unsrt}


\end{document}

