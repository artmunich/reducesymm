\svnkwsave{$RepoFile: siminos/baroclinic/intro.tex $}
\svnidlong {$HeadURL$}
{$LastChangedDate$}
{$LastChangedRevision$} {$LastChangedBy$}
\svnid{$Id$}

\chapter{Baroclinic flows}
\label{chap:baroclinic}

\section{Introduction}
\label{s:intro}
    \SOA{Not really sure if it is the hardest one to understand. But I
    remembered it seem to distant to me the first time the concept was
    introduced to me.}
The concept of baroclinic instability is perhaps one of the harder ones
to grasp in geophysical fluid mechanics. However, it is also one of the
most fundamental concepts on this field, as it is the main driver for the
large scale circulation of both the atmosphere and the ocean. It is by this
mechanism that the atmosphere redistributes heat from low latitudes to
high latitudes, that it sustains synoptic weather systems, and by which
the ocean develops its termohaline circulation. Its importance can not be
overstated.

    \SOA{Here I am mostly speculating. However, this is the impression I
    have so far}
On the other hand, one could argue that the study of baroclinic flows
remains in its early stages. Although there exist a very well developed
linear theory for the onset of this type of instability, there still much
room to explore the nonlinear regimes. A way to approach this studies is
given by recent advances in the theory of nonlinear systems (some
references); making use of the symmetries of the system, and finding
periodic orbits and fixed points, as a way to understand the manifold of
this type of setups.

In this study we introduce the physics and present the nonlinear theory
methods that might be used to analyze baroclinic flows. We will briefly
mention some of the stability theory, but the emphasis would not be on
them. Great papers have been written about it an the reader is referred
to references [.....].

The work is dived as follows. In \refsect{s:examples} we introduce the
problem in a qualitative matter, hopping that this simple approach gives
away the underlying physical principle. In \refsect{s:vorticity} we
use Navier-Stokes equation to derive the vorticity equation and
explicitly expose the term which contribute to this instability (i.e. the
solenoidal term); we also show how can this induce circulation in the
simple setups given in \refsect{s:examples}. In \refsect{s:qg} we
introduce the QG-Equations, which are a simplified set of equations
suitable to study geophysical flows. We also present simulations for the
case were a two layer model is considered.  \refSect{s:stability}
introduces some basic setup showing how the stability of this flows might
be addressed. An we devote the remainder of this work in showing how
nonlinear techniques can be used to identify important properties of this
types of flows.

\section{Qualitative Examples}
\label{s:examples}
\SOA{Just a rough draft at this time, would continue to work soon.}
There are plenty examples which illustrate the mechanism initiating baroclinic flows, however let us start with the simpler ones, and we will progressively built up in complication. To start with, let us ignore the effects due to earths rotation and concentrate only in inertial frames. That is, let's start not by treating baroclinic instability perse, but the process by which a fluid adjust to equilibrium given a uneven distribution in density. A great example of this is Marsigli's experiment to explain undercurrent flows in the Bosphorus river from the Mediterranean to the Black Sea (see \refref{Gill82}), and we will begin with a mental experiment based on this.

Consider the situation shown in Figure \ref{f:Marsigli}.a, where two fluids of different densities, initially separated at $x_o$, are suddenly allowed to interact. The situation is clearly unstable; a pressure gradient would exist at all levels, except for the surface, going from the heavier fluid to thee lighter one. This would create both subsurface and surface currents, one due to the pressure gradient in the bottom, and the other due to mass conservation in the surface. Intuitively we can imagine that the system would finally settle to a configuration where the heavier fluid would lays on the bottom and the lighter one on top. That is, a configuration where the potential energy is minimized.

Thinking about this problem in terms of surface of constant pressure and density we can understand the instability that causes this type of behavior. In the initial configuration, the isopycnals are orthogonal to the isobars Figure \ref{f:Marsigli}.b, so that the mentioned pressure gradient is generated at $x_o$. Later, as the denser fluid starts to settle in the lower layer, this pressure gradient starts to spread out; but it will always exist as long as there is a inclination is the isopycnals. Finally, when all the transient motions are settled and equilibrium is reached, both  the isobars and the isopycnals are parallel; leaving the system in a lower potential energy state. A fundamental concept can be extracted from this: \emph{In the absence of rotation, or an external forcing, equilibrium of a fluid is reached when isopycnals and isobars are parallel to each other}. If this condition is not met, transient motions would be generated to extract the excess  potential energy, and leave the system in its lower energy state.

Mathematically this term can be quantified as the curl between the density and pressure gradients, that is:

\beq
\frac{1}{\rho^2} \nabla \rho \times \nabla p
\,,
\ee{baroclVec}

as we will show in \refsect{s:vorticity}. Obviously, this definition agrees with the intuitively notion we just developed.

Now lets add a further complication, and think about what would happen if the frame of reference were to be rotating. %To be continued....

%%%%%%%%%%%%%%%%%%%%%%%%%%%%%%%%%%%%%%%%%%%%%%%%%%%%%%%%%%%%%%%%
%INCLUDE FIGURE \ref{f:Marsigli}
%%%%%%%%%%%%%%%%%%%%%%%%%%%%%%%%%%%%%%%%%%%%%%%%%%%%%%%%%%%%%%%%%%

\section{Vorticity Equation}
\label{s:vorticity}
\section{Quasygeostrophic Equations}
\label{s:qg}
\section{Stability Theory}
\label{s:stability}
\section{Nonlinear Theory}
\label{s:nonlinear}
\section{Setup}
\label{s:setup}
