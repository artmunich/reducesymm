\svnkwsave{$RepoFile: siminos/baroclinic/intro.tex $}
\svnidlong {$HeadURL$}
{$LastChangedDate$}
{$LastChangedRevision$} {$LastChangedBy$}
\svnid{$Id$}

\chapter{Baroclinic flows}
\label{chap:baroclinic}

\section{Introduction}
\label{s:intro}
    \SOA{Not really sure if it is the hardest one to understand. But I
    remembered it seem to distant to me the first time the concept was
    introduced to me.}
The concept of baroclinic instability is perhaps one of the harder ones
to grasp in geophysical fluid mechanics. However, it is also one of the
most fundamental concepts on this field, as it is the main driver for the
large scale circulation of both the atmosphere and the ocean. It is by this
mechanism that the atmosphere redistributes heat from low latitudes to
high latitudes, that it sustains synoptic weather systems, and by which
the ocean develops its termohaline circulation. Its importance can not be
overstated.

    \SOA{Here I am mostly speculating. However, this is the impression I
    have so far}
On the other hand, one could argue that the study of baroclinic flows
remains in its early stages. Although there exist a very well developed
linear theory for the onset of this type of instability, there still much
room to explore the nonlinear regimes. A way to approach this studies is
given by recent advances in the theory of nonlinear systems (some
references); making use of the symmetries of the system, and finding
periodic orbits and fixed points, as a way to understand the manifold of
this type of setups.

In this study we introduce the physics and present the nonlinear theory
methods that might be used to analyze baroclinic flows. We will briefly
mention some of the stability theory, but the emphasis would not be on
them. Great papers have been written about it an the reader is referred
to references [.....].

The work is dived as follows. In \refsect{s:examples} we introduce the
problem in a qualitative matter, hopping that this simple approach gives
away the underlying physical principle. In \refsect{s:vorticity} we
use Navier-Stokes equation to derive the vorticity equation and
explicitly expose the term which contribute to this instability (i.e. the
solenoidal term); we also show how can this induce circulation in the
simple setups given in \refsect{s:examples}. In \refsect{s:qg} we
introduce the QG-Equations, which are a simplified set of equations
suitable to study geophysical flows. We also present simulations for the
case were a two layer model is considered.  \refSect{s:stability}
introduces some basic setup showing how the stability of this flows might
be addressed. An we devote the remainder of this work in showing how
nonlinear techniques can be used to identify important properties of this
types of flows.

\section{Qualitative Examples}
\label{s:examples}
\section{Vorticity Equation}
\label{s:vorticity}
\section{Quasygeostrophic Equations}
\label{s:qg}
\section{Stability Theory}
\label{s:stability}
\section{Nonlinear Theory}
\label{s:nonlinear}
\section{Setup}
\label{s:setup}
