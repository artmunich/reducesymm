\svnkwsave{$RepoFile: siminos/baroclinic/OrtegaBlog.tex $}
\svnidlong {$HeadURL$}
{$LastChangedDate$}
{$LastChangedRevision$} {$LastChangedBy$}
\svnid{$Id$}

\chapter{Convectively coupled waves}
\label{chap:OrtegaBlog}

\section{Introduction}
\label{sect:CCWs}

\begin{description}

\item[2012-02-08 Predrag] Please write an introduction to
the ``Convectively coupled waves'' suitable to inclusion into
your thesis: what they are, and why should one care. Add relevant references
to this repository's siminos/bibtex/siminos.bib (no separate bibliographies,
it makes updating them a pain).

\end{description}

\section{Sebastian's blog}
\label{sect:OrtegaDaily}

This is Sebastian Ortega Arango's blog for PHYS 7224,  spring 2012
\emph{Nonlinear dynamics: Chaos, and what to do about it?} course project

\begin{description}

\item[2012-02-07 Predrag] Added Sebastian Ortega Arango
<sortega@gatech.edu> project to this blog; enabled svn access, updates
(sortega  baroclinic). First task: please study \refchap{chap:baroclinic},
and improve it in any way you see fit, edit, add better references etc.

\item[2012-02-07 Sebastian]
I think I would be able to relate the project with my line of research. I
am interested in \textbf{Convectively Coupled Waves}. I read that people at NYU
has worked the mathematics in this types of phenomena, and it appears
that interesting nonlinearities arise from it. Baroclinic instability
seems to bee an important driver.

\item[2012-02-08 Predrag] I would like the project to focus on a
geophysically important model that exhibits $\SOn{2}$ invariance,
hence propose to investigate the baroclinic instability first, then apply
that to your thesis research on Convectively Coupled Waves. Advantage of
starting with the baroclinic instability is that you can hit the ground running,
as Annalisa has simulation code ready to use.

\item[2011-10-15 Annalisa] Define and explain the Rossby radius for the
atmosphere in \refsect{sect:CCWs}. Mark here [~~] when done.

\item[2012-03-27 Sebastian] I am going to follow your advice and star
working with baroclinic stability first. I have written a draft of the
introduction, the idea is that it outlines the work to be done, and is
subject to changes. I think it is a good outline to introduce baroclinic
instability, and to introduce the model used in the simulations. However,
it still not clear to me how to include the nonlinear analysis. I will
continue reading about the \KSe\ and the \pCf\ for this. I will try to
write \refsect{s:intro} today.

\item[2012-03-27 Predrag] I'm glad you are getting started - go for it.

\item[2012-03-27 Sebastian]
Also, I find a thesis that might be interesting to read Veen's thesis,
\HREF{http://igitur-archive.library.uu.nl/dissertations/2002-0801-151812/c2.pdf}
{Chapter 2}.

\item[2012-03-27 Predrag] I have not read Lennart's thesis, but he does good work. Blog
here what you find of interest as you read it.

\item[2012-04-04 Sebastian]
Finished first and second section. I am uploading to show what I have
done so far, but it is just a quick draft as far as redaction goes. I
still have to read it again and correct it. Some times I write in this
way, first I get all the ideas in paper and then iterate until I have a
coherent document. Probably not the best way out there.

In short, the point that I am trying to get trough, is that there is a
base flow given by geostrophy and the hydrostatic relation which might be
stable depending on the slope of the isopycnals. Then I introduce the
model Professor Annalisa used for her simulations (I believe is the one
described by Philips in 1951).

I think it is important for me to get focus on the nonlinear aspects of
the problem. So I will try to finish all the way to
\refsect{s:stability} by this week (but not sure if I will have the time).

I was also wondering what should I do for the nonlinear section. But I
guess I should first order my ideas and learn how to think of baroclinic
instability in terms of dynamical systems. So far I think of it from the
point of view of bifurcations (please correct me if wrong), where I have
a equilibrium point in the dynamical system which becomes unstable (or
disappears??) after some parameter increases (this is what I believe the
linear theory does). But I am not sure what happens once the flow is
unstable, my guess is that there would be some kind of strange attractor
out there in the n-dimensional space. But I guess visualization of this
would be quite hard, unless a low order spectral representation is used.
It would be very interesting to hear your vision of the instability; so
please let me know if there is a paper I can read for this, or what you
think is happening.

Also, let me know of any changes that you think might be convenient.

\item[2012-04-05 Sebastian]
A quote from Rick Salmon book (might be good for Chaos book). After
proving Ertel's Theorem: "Of course, we can prove all these results
directly from (1.1)
% \footnote{Momentum equation, continuity equation,
% thermodynamic equation and equation of state}
by pedestrian mathematical
manipulations, but that only makes it harder to appreciate their physical
significance" {\bf [2012-04-05] Predrag} added to the ChaosBook trove of
reserve quotes, thanks!

\item[2012-04-05 Sebastian]
Done with \refsect{s:stability} \emph{Stability theory} text. However,
some calculations are needed for the specific case (those for $\omega$
and $U_c$). But they might be in the literature somewhere; however there
should be easy to do (or reduce easily from the ones given by Hasha\rf{Hasha05}
and/or Vallis\rf{Vallis06}).

\item[2012-04-23 Sebastian]
I have been trying to figure out how to find the fixed points and
periodic orbits for my project. However, I am not really sure how to
implement Viswanath GMRES algorithm to Annalisa's code (maybe a simpler
one is ok, as Viswanath also look for relative periodic orbits). I think
I have to write a searching algorithm based on the method, so I have been
spending time trying to understand the solution method, although I have
not fully understood it yet. I also found Halcrow theses in the web, this
might help me figure out how to do it.

I think I will start writing down everything I have read, and what I
think it might be found for the model. And then try to do code the
searching algorithm. Let me know if you have any pointers for this, or if
I should do something differently.

\item[2012-04-23 Predrag]
I think implementing Viswanath GMRES algorithm to Annalisa's code is a
semester project. It might be easier to implement her code as a module in
channelflow.org, but that too is months of work. It is certainly worth
doing, but you cannot do it in a week. I am happy if in your project
right now you run her code, see some interesting structures, and maybe
manage to find a close recurrence in the data.

Read the end of siminos/blog/blog.tex, chapter {\em Fluids} about this.

\item[2012-04-23 Sebastian]
By the way, is there class tomorrow?

\item[2012-04-23 Predrag]
Yes, this is the last week. If people want to, they can self organize to present their
projects the coming week, but I'm not supposed to meddle, as it is exam week.

\item[2012-04-23 Sebastian]
I guess I got a little exited about the papers I read; very interesting. I will try then to limit the search to recurrent motions. But will write about what can be done in the future. I have ran Annalisa's code already, so I will try to search this by looking at dissipation and energy plots as done in Viswanath to find an initial guess for periodic orbits. But my guess is that it would be very qualitative.

I was wondering what you think the implications of periodic orbit theory are for climate and weather. I have the feeling that it must be very important. But would like to hear what you think about it. I want to study predictability for my Phd work, focusing on the tropics intraseasonal variations (MJO and such); exploring this kind of approaches seem as something important to me. Let me know when you have time to discus this and I will go by your office.

\item[2012-04-24 Sebastian]
Forgot to upload the blog yesterday. I am uploading it along with some advances in the nonlinear section (\refsect{s:nonlinear} not yet complete). Just the ideas of some important papers I have found; and what I intend in doing for Chaos project.
Currently I am running Annalisa's code for a longer period, 5 times more than before. And playing with the visualization of energies and dissipation to see if I can find close recurrences. I was also thinking in changing the parameters of the simulation. But I think I will try to find them first in the simulation as it is, and then change them if necessary.
Any suggestions or recommendations are welcome.

\item[2012-04-25 Predrag]
Here is my concrete proposal for what you can do now, for the course
project. What you have written is good. What would really help us
(Annalisa, me, you) is if you read the Chaos Gang paper (click
\HREF{http://www.cns.gatech.edu/~predrag/papers/preprints.html\#atlas12}{here}),
and implement the sliced version of Annalisa's simulations. You do not
need any invariant solutions to do this, use as a template a typical
turbulent state in the simulation.

The key physical step is choice of norm, read commentary after Eq. (1).
Slicing means that given the norm and the template, you replace ODE
integrator velocity fields by the ones in slice, Eq. (8): these videos
should be much calmer than the original simulation, as drifts have been
quotiented out. That is already enough to complete the term project.


Next keep track of phase velocity Eq (9). If that diverges it means you
are falling of the edge of your template's chart: you should use a
multiple chart atlas. If you get a few charts, ridges, and reduced flow that
encounters no singularities, we already have a publication.

Finding invariant solutions is essential, and cannot be done without
symmetry reduction, but it not necessary to illustrate symmetry
reduction.

\end{description}
