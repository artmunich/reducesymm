\svnkwsave{$RepoFile: siminos/baroclinic/OrtegaBlog.tex $}
\svnidlong {$HeadURL$}
{$LastChangedDate$}
{$LastChangedRevision$} {$LastChangedBy$}
\svnid{$Id$}

\chapter{Convectively coupled waves}
\label{chap:OrtegaBlog}

\section{Introduction}
\label{sect:CCWs}

\begin{description}

\item[2012-02-08 Predrag] Please write an introduction to
the ``Convectively coupled waves'' suitable to inclusion into
your thesis: what they are, and why should one care. Add relevant references
to this repository's siminos/bibtex/siminos.bib (no separate bibliographies,
it makes updating them a pain).

\end{description}

\section{Sebastian's blog}
\label{sect:OrtegaDaily}

This is Sebastian Ortega Arango's blog for PHYS 7224,  spring 2012
\emph{Nonlinear dynamics: Chaos, and what to do about it?} course project

\begin{description}

\item[2012-02-07 Predrag] Added Sebastian Ortega Arango
<sortega@gatech.edu> project to this blog; enabled svn access, updates
(sortega  baroclinic). First task: please study \refchap{chap:baroclinic},
and improve it in any way you see fit, edit, add better references etc.

\item[2012-02-07 Sebastian]
I think I would be able to relate the project with my line of research. I
am interested in \textbf{Convectively Coupled Waves}. I read that people at NYU
has worked the mathematics in this types of phenomena, and it appears
that interesting nonlinearities arise from it. Baroclinic instability
seems to bee an important driver.

\item[2012-02-08 Predrag] I would like the project to focus on a
geophysically important model that exhibits $\SOn{2}$ invariance,
hence propose to investigate the baroclinic instability first, then apply
that to your thesis research on Convectively Coupled Waves. Advantage of
starting with the baroclinic instability is that you can hit the ground running,
as Annalisa has simulation code ready to use.

\item[2011-10-15 Annalisa]


\end{description}
