% KSe.tex
% $Author$ $Date$

% Predrag extracted from newton.tex 		jul  3 2006
% Predrag					jun 20 2006
% Vaggelis					may 20 2006

\section{\KSe}
\label{s-KS}
% Predrag 					 4jul2006
% extracted from ~dasbuch/book/chapter/PDEs.tex  5jun2005 version
% Predrag					17sep1999

\PC{incorporate missing refs from chapter/refsPDEs.tex}

The \KS\ system\rf{ku,siv}, which
arises in the description of stability of
flame fronts and many other physical settings,
is one of the simplest nonlinear PDEs that
exhibit turbulence.
The time evolution of the ``flame front velocity'' 
$u=u(x,t)$ defined on a periodic domain
$u(x,t) = u(x+2\pi L,t)$
is given by
\beq
	u_t=(u^2)_x-u_{xx}- u_{xxxx}
	\,,\qquad	x \in [0,2\pi L]
	\,.
\ee{ks}
Here $t \geq 0$ is the time and
$x$ is the spatial coordinate.
The subscripts $x$ and $t$ denote partial derivatives with respect to
$x$ and $t$;
$u_t = du/dt$, $u_{xxxx}$ stands for the 4th spatial
derivative of 
$u=u(x,t)$ at position $x$ and time $t$.
% The ``viscosity'' parameter 
% $\nu$ 
% suppresses solutions with fast spatial variations.
% We take note, as in the Navier-Stokes equation, of the
% $u {\partial_x} u$ ``inertial'' term, the $ {\partial_x^2 } u$
% ``diffusive'' term (both with a ``wrong'' sign), etc.

% The term $(u^2)_x$ makes this a {\em nonlinear system}.
% It is one of the
% simplest conceivable nonlinear PDE, playing
% the role in the theory of spatially extended systems
% analogous to the role that
% the $x^2$ nonlinearity
% % \refeq{LogisMap}
% plays in the dynamics of iterated mappings.

% The salient feature of such
% partial differential equations is a theorem saying that
% for any finite value of the phase-space contraction
% parameter $\nu$,  the asymptotic dynamics is
% describable by a {\em finite} set of ``inertial manifold''
% ordinary differential equations. %cite{Foias88}.

% In order to find a better representation of the dynamics, we now
% turn to its topological invariants.


\subsection{Symmetries of \KSe}
% former \subsection{Scaling and symmetries}
% former symm.tex
% Predrag extracted from newton.tex 		jul  3 2006

Comparing $u_t$ and $(u^2)_x$ terms we note that $u$ has
dimensions of $[x]/[t]$, hence on dimensional grounds
it is the ``velocity'' of the flame front. 
Furthermore, the  \KSe\ is
Galilean invariant: if $u(x,t)$ is a solution, then 
$v+u(x+2vt,t)$, with $v$ an arbitrary constant velocity, is also a solution. 
Without loss of generality, in our calculations we shall set 
the mean velocity of the  front to zero,
% \index{Galilean invariance}
% \index{invariance!Galilean}
\beq
\int dx \, u = 0
\,.
\ee{GalInv}

In terms of the system size $L$, the only length scale available,
the dimensions of terms in \refeq{ks} are
$ %\[
[x]=L
$, $%\,,\quad
[t]=L^2
$, $%\,,\quad
[u]=L^{-1}
$, $%\,,\quad
[\nu]=L^2
\,.
$ %\]
% What is the non-dimensional ``Rayleigh'' number for the
% \KS\ system? 
%  Scaling out the ``viscosity'' $\nu$ 
% \[ 
% x \to x \nu^{\frac{1}{2}}
% \,,\quad
% t \to t \nu
% \,,\quad
% u \to u \nu^{-\frac{1}{2}}
% \,,
% \]
% brings the \KSe\ \refeq{ks}
% to a non-dimensional form
% \beq
% u_t=(u^2)_x-u_{xx}- u_{xxxx}
% \,,\qquad	
% 	x \in  [0,L\nu^{-\frac{1}{2}}] = [0,2\pi\tilde{L}]
% \,.
% \ee{ks-L}
% In this way the ``viscosity'' $\nu$
% and the system size $L$ are trade in for a single
% dimensionless system size parameter
% \beq
% 	\tilde{L}={L}/{(2 \pi \sqrt{\nu})}
% %	\tilde{L}=\frac{L}{2 \pi \sqrt{\nu}}
% %	\,,
% \ee{tildeL}
% which plays the role of a ``Reynolds number''
% for the \KS\ system.

% In the literature sometimes 
% $L$ is used as the system parameter, with $\nu$ fixed to $1$, and
% at other times $\nu$ is varied with $L$ fixed to either $1$ or $2\pi$.
% To minimize confusion,
% in what follows we shall state results of all 
% calculations in units of dimensionless system size $\tilde{L}$.
% \PC{motivate $2\pi$ factor by the mean wavelength,
%     refer to the equation number}
% Note that the time units also have to be
% rescaled; if $T^*_p$ is a period
% of a periodic solution of \refeq{ks} with a given
% $\nu$ and $L=2\pi$, then
% the corresponding solution of the non-dimensionalized \refeq{ks-L}
% has period 
% \beq
%  T_p= T^*_p/\nu
% \ee{Trescl}

The \KSe\  \refeq{ks} is
time translationally invariant
and 
space translationally invariant
under the 1-$d$ Lie group of $O(1)$ rotations: if
$u(x,t)$ is a solution, then $u(x+d,t)$ is an equivalent
solution for any $-L/2 < d \leq L/2$.
As a result,
KS can have \rpo\ solutions with nonzero shift
\beq
u(x+d,\period{}) = u(x,0)
\,.
\ee{KSrpos}
where $\period{}$ is the period \ES{Is this obvious to everybody? I cannot really see why it is so.} . 

The KS is also invariant under
reflection (``parity'' or ``inversion'') operation
\beq
\Refl u(x) = -u(-x)
\ee{KSparity}
and the shift symmetry operation 
\beq
\Shift u(x)=u(x+L/2)
\,. 
\ee{KSshift}
The shift symmetry is a particular case of the
above translational $O(1)$ invariance.
In the Fourier modes decomposition  this
symmetry corresponds to invariance under
\refeq{FModInvSymm}.

Relations $\Refl^2=\Shift^2=1$
induce decomposition of the space of solutions into 4 invariant
subspaces\rf{KNSks90}.

This reduction of the dynamics to the fundamental domain is particularly
useful in periodic orbit calculations, as it simplifies symbolic dynamics
and improves the convergence of cycle expansions\cite{CvitaEckardt}.

some of the \po s will
halve their period, and symmetric pairs will be eliminated.

% By symmetry there might be an equilibrium on the reflection plane that
% relates the equilibrium A and its symmetry partner SA; the 3 equilibria would
% be analogous to what you see in the Lorentz attractor pictures, crossing
% the unstable manifold of the central one throws you into the neighborhood
% of the other equilibrium.

% For discrete rotations the spectral determinants factorize
% nicely in terms of rpo's:
% read ``Discrete symmetries" chapter of ChaosBook.org - not an easy read, but
% it also uses $g\jMps$ rather than the naked $\jMps$,
% and a trace formula for irrational
% $d/L$ still puzzles me - for $d/L$ rational the determinant factorizes using
% discrete Fourier transform.
% We are fuzzy on the continuum limit of that.

