% summary.tex
% $Author$ $Date$

\section{Summary}
% \section{Conclusions}
\label{sect:rpo-sum}


We have presented a detailed investigation
of the geometry of the
{\KS} \statesp\ for $L=22$ system size.
At first glance, turbulent dynamics visualized in the \statesp\ might appear
hopelessly complex, but under a detailed examination it is
much less so than feared: it is
pieced together from low dimensional % {1-$d$ return maps}
local unstable manifolds connected by fast transient interludes.
{\KS} and \pCf\  \eqv, \reqv, \po s and 
\rpo s embody Hopf's vision:
repertoire of recurrent spatio-temporal
patterns explored by turbulent dynamics.
We used
the \eqva, \reqva\ and {\rpo}s as a probe to explore the
\statesp\  topology and chaotic dynamics.

While in general
for $\tildeL$ sufficiently large
one expects many 
coexisting attractors in the \statesp%
%Hyman and Nicolaenko
\rf{HNZks86},
in numerical studies most random initial
conditions converge to the same chaotic attractor. 

The key new feature of the full, periodic domain
KS is its continuous translational symmetry,
with attendant continuous families of
\reqva\ (traveling waves) and \rpo s.
\Rpo s, in particular, require rethinking dynamical systems
approach to constructing symbolic dynamics. 




\PC{
The real motivation for all this is that if we understand \eqva\ as
$L \to \infty$ we might have an entry into $L = \infty$ periodic orbit
theory of KS.
   }



