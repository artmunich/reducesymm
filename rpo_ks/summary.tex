% summary.tex
% $Author$ $Date$

\section{Summary}
\label{sect:rpo-sum}

We have presented a detailed investigation
of the geometry of the
{\KS} \statesp\ for $L=22$ system size, with emphasis on the role of
low-dimensional unstable manifolds of \eqva,
 and the connections between \eqva\ in organizing the flow.
A large number of unstable \rpo s and \po s has been determined
numerically.
Many of these \rpo s
\PCedit{appear} organized by the unstable manifold of $\EQV{2}$, closely
following the homoclinic loop formed between $\EQV{2}$ and $\Shift_{1/4}\EQV{2}$.


At first glance, turbulent dynamics visualized in the \statesp\ might appear
hopelessly complex, but under a detailed examination it is
much less so than feared: it is
pieced together from low dimensional % {1-$d$ return maps}
local unstable manifolds connected by fast transient interludes.
{\KS} (and \pCf, see \refref{GHCW07})  \eqva, \reqva, \po s and
\rpo s embody Hopf's vision:
a repertoire of recurrent spatio-temporal
patterns explored by turbulent dynamics.
\PC{must expand this, emphasize novelty, especially
        of the \statesp\ visualization. See \refref{GHCW07}
        for inspiration.
        }


The key new feature of the full, periodic domain
KS, with its continuous translational symmetry,
are the attendant continuous families of
\reqva\ (traveling waves) and \rpo s.
\Rpo s, in particular, will require rethinking dynamical systems
approach to constructing symbolic dynamics.




\PC{
The real motivation for all this is that if we understand \eqva\ as
$L \to \infty$ we might have an entry into $L = \infty$ periodic orbit
theory of KS.
   }
