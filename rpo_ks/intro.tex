% intro.tex
%
% Predrag                       jun 20 2006
% $Author$ $Date$

\section{Introduction}

\PC{ I. Rich History: reference Poincar\'e and onward}
%
Recent experimental and theoretical advances\rf{science04}
support a dynamical vision
of turbulence ascribed to E.~Hopf, as a walk through
a repertoire of unstable recurrent patterns:
As a turbulent flow evolves,
every so often we catch a glimpse of a familiar pattern.
For any finite  spatial resolution,
the system follows approximately for a finite time 
a pattern belonging to a 
{ finite alphabet}
of admissible patterns.
The long term dynamics is
a {  walk through the space of such unstable patterns}.

The question is how to characterize and classify such patterns?
\refRefs{MS66,Christiansen:97} have proposed (in contexts
much simpler than the full Navier-Stokes hydrodynamics) 
that the unstable spatiotemporally periodic
solutions could play that role.

Here we follow the seminal Hopf\rf{hopf48} paper, and  visualize
hydrodynamic turbulence not as  a sequence of 
$3D$  snapshots in turbulent evolution,
% with each pixel a 3-$d$ velocity field
but as a trajectory in an 
 $\infty$-$d$ state space in which an
instant in turbulent evolution is
a { unique} point. In E.~Hopf's vision, 
theory of turbulence for a given system, with given boundary conditions,
is given by the
(a) geometry of the state space and (b) the associated natural measure, 
\ie,
the likelihood that asymptotic dynamics visits a given state space region.

We explore Hopf's vision in context of
the \KSe\rf{ku,siv}.
Holmes, Lumley
and Berkooz\rf{Holmes96} offer a delightful discussion of why this system
deserves study as a staging ground for studying turbulence in 
full-fledged Navier-Stokes boundary shear flows. 
Such 
amplitude equations for interfacial instabilities arise in a variety
of contexts\rf{KNSks90} - and 
the \KS\ system is perhaps the
simplest physically interesting spatially extended nonlinear system.
% \PC{Comment om MAWs, BECS and CGLe}
% \PC{refer to Trefethen's program for fast integration}
% (see \wwwcb{/extras}).
The unstable periodic solutions of
this system have been studied in great detail%
% by Christiansen {\em et al.}
\rf{Christiansen:97,Lan:Thesis,lanCvit06}
in the antisymmetric subspace of the \KS\ dynamics.

Focus of this paper is on the role continuous symmetries
play in spatiotemporal dynamics. The new aspect is that
in presence of a continuous symmetry of dynamics solutions come
in continuous families. For the 1-$D$ \KS\ on a periodic domain
the symmetry group is the group of of all translations, and
any solution (let us say, \eqv) belongs to a circle of invariant
solutions. Furthermore, notion of exact periodicity is 
replaced by the notion of relative periodicity, and,
as we shall see, 
\reqva\ and \rpo s here play the role of \eqva\ and
\po s of the earlier studies.

\Preliminary{
%\Eqva\ and \reqva\ of \KSe\ are investigated in Kevrekidis \etal\rf{KNSks90}. 

Testing bibtex - these should exist:
\refrefs{Laurent-Polz04,lop05rel,McCordMontaldi}
\refrefs{Vanderb,Wulff00}
	}  %end \Preliminary{

