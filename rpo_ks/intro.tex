% intro.tex
% $Author$ $Date$

% Predrag                       jun 20 2006

\section{Introduction}

Recent experimental and theoretical advances\rf{science04}
support a dynamical vision of turbulence:
For any finite  spatial resolution,
a turbulent flow follows approximately for a finite time
a pattern belonging to a
{ finite alphabet}
of admissible patterns.
The long term dynamics is
a {walk through the space of these unstable patterns}.
The question is how to characterize and classify such patterns?
Here we follow the seminal Hopf\rf{hopf48} paper, and  visualize
turbulence not as  a sequence of
spatial snapshots in turbulent evolution,
% with each pixel a 3-$d$ velocity field
but as a trajectory in an
 $\infty$-$d$ \statesp\ in which an
instant in turbulent evolution is
a {unique} point. In the dynamical systems approach,
theory of turbulence for a given system, with given boundary conditions,
is given by the
(a) geometry of the \statesp\ and (b) the associated natural measure,
\ie,
the likelihood that asymptotic dynamics visits a given \statesp\ region.

Here we pursue this program in context of the \KS\ (KS) equation, %\rf{ku,siv}
one of the simplest physically interesting spatially extended
nonlinear systems.  Holmes, Lumley and Berkooz\rf{Holmes96} offer a
delightful discussion of why this system deserves study as a staging
ground for studying turbulence in full-fledged Navier-Stokes
boundary shear flows.
% \refRefs{MS66,Christiansen:97} have proposed (in contexts
% much simpler than the full Navier-Stokes hydrodynamics)
% that the unstable spatiotemporally periodic
% solutions could play that role.

Dynamical \statesp\ representation of a PDE is $\infty$-dimensional,
but the KS flow is strongly contracting and its non-wondering set,
and, within it, the set of invariant solutions investigated here, is
embedded into a finite-dimensional inertial manifold\rf{FNSTks85} in
a non-trivial, nonlinear way. `Geometry' in the title of this paper
refers to our attempt to systematically triangulate this set in
terms of a dynamically invariant solutions (\eqva, \po s, $\ldots$)
and their unstable manifolds, in a PDE representation and
DNS algorithm independent way. The goal is to describe a given
`turbulent' flow quantitatively, not model it qualitatively by a
low-dimensional model. For the case investigated here, the \statesp\
representation dimension $d \sim 10^2$ is set by requiring that the
exact invariant solutions that we compute are accurate to $\sim
10^{-5}$.
The \statesp\ is high-dimensional, the asymptotic dynamics is
confined to a low-dimensional subspace, so we are in the gray,
ill-defined overlapping zone between `turbulence'
and `spatiotemporal chaos,' the two terms that we shall use
interchangeably in what follows.

In previous work, the \statesp\ geometry and the natural measure for
this system have been
studied\rf{Christiansen:97,Lan:Thesis,LanCvi07} in terms of unstable
periodic solutions restricted to the antisymmetric subspace of the
KS dynamics.

The focus in this paper is on the role continuous symmetries play in
spatiotemporal dynamics.
    \PC{
removed ``For the 1-$d$ KS equation on a periodic
domain the symmetry group is the group of all translations, and any
solution (let us say, an \eqv ) belongs to a circle of equivalent
solutions.
        }
The notion of exact periodicity in time is
\PCedit{replaced by} the notion of relative spatiotemporal
periodicity, and \reqva\ and \rpo s here play the role the \eqva\
and \po s played in the earlier studies.

Building upon the pioneering work of \refrefs{KNSks90,ksgreene88},
we undertake here
    \PCedit{
a study of the \KS\ dynamics
for a specific system size $L = 22$, sufficiently large
to exhibit many of the features typical of `turbulent' dynamics
observed in large KS systems, but small enough to lend itself to a
detailed exploration of the  \eqva\ and \reqva,
their stable/unstable manifolds,
determination of a large number of
\rpo s, and a preliminary exploration of the relation between the
observed spatiotemporal `turbulent' patterns and the \rpo s.
% thus
% setting the stage for their systematic enumeration and application
%of the periodic orbit theory.
        }
%\PC{Check entire text, use either
%``..." or `...' consistently. My vote is for `...' }

In presence of a continuous symmetry any solution belongs to a group
manifold of equivalent solutions. The problem: If one is to
generalize the periodic orbit theory to this setting, one needs to
understand what is meant by solutions being nearby (shadowing) when
each solution belongs to a manifold of equivalent solutions. We
resolve here this puzzle by demonstrating that if one picks any
particular solution, the universe of all other solutions is rigidly
fixed through a web of heteroclinic connections between them. This
insight garnered from study of a 1-dimensional \KS\ PDE is more
remarkable still when applied to the plane Couette flow\rf{GHCW07},
with 3-$d$ velocity fields and two translational symmetries.


The main results presented here are: (a) Dynamics visualized through
physical, symmetry invariant observables, such as `energy,'
dissipation rate, \etc,
% \refsect{sec:energy},
and through
projections onto a dynamically invariant, PDE-discretization
independent \statesp\ coordinate frames, \refsect{sec:energy}. (b)
Existence of a rigid `cage' built by heteroclinic connections
between \eqva\ and \po s, \refsect{sec:L22}. (c) Preponderance of
unstable \rpo s and their likely role as the skeleton underpinning
spatiotemporal turbulence in systems with continuous symmetries,
\refsect{sec:rpos}.
