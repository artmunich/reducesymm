% intro.tex
%
% Predrag                       jun 20 2006
% $Author$ $Date$

\section{Introduction}

Recent experimental and theoretical advances\rf{science04}
support a dynamical vision
of turbulence:
\PC{please figure out how to change references from (22) to
    [22] - now one cna confuse them with eq numbers}
% ascribed to E.~Hopf:
%, as a walk through
% a repertoire of unstable recurrent patterns:
As a turbulent flow evolves,
every so often we catch a glimpse of a familiar pattern.
For any finite  spatial resolution,
the system follows approximately for a finite time 
a pattern belonging to a 
{ finite alphabet}
of admissible patterns.
The long term dynamics is
a {  walk through the space of these unstable patterns}.
The question is how to characterize and classify such patterns?
Here we follow the seminal Hopf\rf{hopf48} paper, and  visualize
turbulence not as  a sequence of 
spatial snapshots in turbulent evolution,
% with each pixel a 3-$d$ velocity field
but as a trajectory in an 
 $\infty$-$d$ \statesp\ in which an
instant in turbulent evolution is
a { unique} point. In the dynamical systems approach,  
theory of turbulence for a given system, with given boundary conditions,
is given by the
(a) topology of the \statesp\ and (b) the associated natural measure, 
\ie,
the likelihood that asymptotic dynamics visits a given \statesp\ region.

Here we pursue this program in context of
the \KSe\rf{ku,siv}.
Such 
amplitude equations for interfacial instabilities arise in a variety
of contexts\rf{KNSks90} - and 
the \KS\ system is perhaps the
simplest physically interesting spatially extended nonlinear system.
% \PC{Comment om MAWs, BECS and CGLe}
% \PC{refer to Trefethen's program for fast integration}
% (see \wwwcb{/extras}).
Holmes, Lumley
and Berkooz\rf{Holmes96} offer a delightful discussion of why this system
deserves study as a staging ground for studying turbulence in 
full-fledged Navier-Stokes boundary shear flows. 
% \refRefs{MS66,Christiansen:97} have proposed (in contexts
% much simpler than the full Navier-Stokes hydrodynamics) 
% that the unstable spatiotemporally periodic
% solutions could play that role.
So far, both the topology and the natural measure for
this system have been studied in great detail%
% by Christiansen {\em et al.}
\rf{Christiansen:97,Lan:Thesis,LanCvi07}
in terms of unstable periodic solutions
restricted to
the antisymmetric subspace of the \KS\ dynamics.

The focus in this paper is on the role continuous symmetries
play in spatiotemporal dynamics. For the 1-$D$ \KS\ on a periodic domain
the symmetry group is the group of all translations, and
any solution (let us say, an \eqv) belongs to a circle of equivalent
solutions. The notion of exact periodicity in time is 
generalized to the notion of relative, spatiotemporal periodicity, and,
as we shall see, 
\reqva\ and \rpo s here play the role the \eqva\ and
\po s played in the earlier studies. 

Building upon the pioneering work of
\refrefs{KNSks90,ksgreene88}, we undertake here a detailed study of the 
\KS\ \eqva\ and \reqva\ stable/unstable manifolds
for a system sufficiently large to exhibit many of
the features typical of turbulent dynamics observed in large \KS\ systems.
We also determine a large number of \rpo s, setting the
stage for their systematic enumeration and application 
of the periodic orbit theory.

In presence of a continuous symmetry any solution belongs to a 
group manifold of equivalent solutions. The problem: If one is to generalize
the periodic orbit theory to this setting, one needs to understand what
is meant by solutions being nearby (shadowing) when each solution is
a torus of equivalent solutions. 
We resolve here this puzzle by demonstrating that if one picks
any particular solution, the universe of all other solutions is
rigidly fixed through a web of heteroclinc connections between them.
This insight garnered from study of a
1-dimensional \KS\ PDE is more remarkable still when applied to
the plane Couette flow\rf{GHCW06}, where there
is a bagel worth of equivalent solutions for each \eqv\ numerically determined.


The main results presented here are:
(a) Existence of a rigid ``cage'' built by heteroclinic connections
between \eqva\ and \po s.
(b) Preponderance of unstable \rpo s and their likely
role as the skeleton underpinning spatiotemporal turbulence in
systems with continuous symmetries. 



