% lmderRLD.tex
% $Author$ $Date$

\section{Levenberg--Marquardt searches for \rpo s}
\label{sec:lmderRLD}

To find periodic and \rpo s of the \KSe , we use multiple shooting and
the Levenberg--Marquardt algorithm implemented in {\tt lmder} from
the MINPACK software package or function {\tt fsolve} in Matlab.

Note that the LM algorithm is able to solve underdetermined systems of
equations.  Therefore, there is no need to augment the system with
additional constraint equations, as is done in
\refrefs{lop05rel, brkevr96,Visw07b,SiminosThesis}.
For example, since L{\'o}pez {\etal}\rf{lop05rel} used {\tt lmder} to
find \rpo s in CGLE, they \edit{could have avoided augmenting} their system with the
three additional constraint equations.  In fact, we have found that,
with the additional constraint equations, the solver usually takes more
steps to converge from the same seed, or fails to converge at all. Even
thought both {\tt lmder} and {\tt fsolve} solvers require that the
number of variables does not exceed the number of equations, the
additional equations can be set identically to zero\rf{Crofts07thesis}.

We need to solve the system of $N-2$ equations
\beq
  {\bf g}(\shift_p)f^\period{p}(a_p) - a_p = 0\,,
\ee{eq:system}
with $N$ unknowns $(a_p, \period{p}, \shift_p)$, where $f^t$
is the flow map of the \KSe.

We have tried two different implementations of the multiple shooting.
The emphasis was on the simplicity of the implementations, so, even
though both implementations worked equally well, each of them had
its own minor drawbacks.

In the first implementation, we fix the total number of steps within
each shooting stage and change the numerical integrator step size
$h$ in order to adjust the total integration time to a desired value
$\period{}$.

Let $(a_0, \period{}_0, \shift_0)$ be the starting guess for a \rpo\
obtained through a close return within a chaotic attractor.  We
require that the initial step does not exceed $h_0$, so we round off the
number of integration steps to $n = \lceil \period{}_0/h_0\rceil$, where
$\lceil x \rceil$ denotes the nearest integer larger than $x$.

The integration step size is equal to $h = \period{}/n$. With the
number of shooting stages equal to $m$, the system in
\refeq{eq:system} is rewritten as follows
\bea
 F^{(1)}&\!=\!& f^\tau(a^{(1)}) - a^{(2)} = 0\,,\nonumber\\
 F^{(2)}&\!=\!& f^\tau(a^{(2)}) - a^{(3)} = 0\,,\nonumber\\
 && \cdots \\
 F^{(m-1)}&\!=\!& f^\tau(a^{(m-1)}) - a^{(m)} = 0\,,\nonumber\\
 F^{(m)}&\!=\!& {\bf g}(\shift)f^{\tau'}(a^{(m)}) - a^{(1)} = 0\,,\nonumber
\label{eq:MultShoot} \eea
where $\tau = \lfloor n/m \rfloor h$ ($\lfloor x \rfloor$ is the nearest
integer smaller than $x$),
$\tau' = nh - (m-1)\tau$, and $a^{(j)} = f^{(j-1)\tau}(a)$, $j = 1, \ldots , m$.
With the Jacobian of this system given by
\beq
  J = \left(\begin{array}{ccc}\!\!
   \displaystyle \frac{\partial F^{(j)}}{\partial a^{(k)}} &
   \displaystyle \frac{\partial F^{(j)}}{\partial \period{}} &
   \displaystyle \frac{\partial F^{(j)}}{\partial \shift}\!\!
  \end{array}\right),\quad j,k = 1,\ldots,m\,,
\eeq
the partial derivatives with respect to $a^{(k)}$ can be calculated
using the solution of \refeq{eq:KStan} as described in
Appendix~\ref{sec:stability}.  The partial derivatives
with respect to $T$ are given by
\beq
  \frac{\partial F^{(j)}}{\partial \period{}} =
  \left\{\begin{array}{ll}
    \frac{\partial f^\tau(a^{(j)})}{\partial \tau}
    \frac{\partial \tau}{\partial T} = v(f^\tau(a^{(j)}))
    \lfloor n/m \rfloor/n\,, & j = 1,\ldots, m-1\\[.5ex]
    {\bf g}(\shift) v(f^{\tau'}(a^{(j)}))
    (1 - \frac{m-1}{n} \lfloor n/m \rfloor ), & j = m\,.
  \end{array}\right.
\eeq
Note that, even though $\partial f^t(a) /\partial t = v(f^t(a))$,
it should not be evaluated using equation for the vector field.
The reason is that, since the flow $f^t$ is approximated by a
numerical solution, the derivative of the numerical solution with
respect to the step size $h$ may differ from the vector field $v$,
especially for larger step sizes.  We evaluate the derivative by
a forward difference using numerical integration with step sizes
$h$ and $h + \delta$:
\beq
  \frac{\partial f^{jh}(a)}{\partial t} = \frac{1}{j\delta}
  \left[f^{j(h+\delta)}(a) - f^{jh}(a)\right],\quad j \in
  {\mathbb Z}^{+}
\eeq with $t = jh$ and $\delta = 10^{-7}$ for double precision
calculations. Partial derivatives $\partial F^{(j)}/\partial \shift$
are all equal to zero except for $j = m$, where it is given by
\beq
  \frac{\partial F^{(m)}}{\partial \shift} =
  \frac{d{\bf g}}{d\shift}f^{\tau'}(a^{(m)})\,.
\eeq

This Jacobian is supplied to the routine {\tt lmder} or {\tt fsolve}
augmented with two rows of zeros corresponding to the two identical
zeros augmenting \refeq{eq:MultShoot} in order to make the number of
equations formally equal to the number of variables,
as discussed above.

In the second implementation, we keep $h$ and $\tau$ fixed and vary
only $\tau' = \period{} - (m-1)\tau$.  In this case, we need to be
able to determine the numerical solution of \KSe\ not only at times
$t_j = jh, j = 1, 2, \ldots$, but at any intermediate time as well.
We do this by a cubic polynomial interpolation through points
$f^{t_j}(a)$ and $f^{t_{j+1}}(a)$ with slopes $v(f^{t_j}(a))$ and
$v(f^{t_{j+1}}(a))$.  The difference from the first implementation
is in that partial derivatives $\partial F^{(j)}/\partial \period{}$
are zero for all $j = 1,\ldots,m-1$, except for
\beq
  \frac{\partial F^{(m)}}{\partial \period{}} =
  {\bf g}(\shift)v(f^{\tau'}(a^{(m)}))\,.
\eeq
which, for consistency, needs to be evaluated from the cubic
polynomial, not from the flow equation evaluated
at $f^{\tau'}(a^{(m)})$.

We found the second implementation more convenient.

For detecting periodic and \rpo s of the \KSe\ with $L = 22$, we used
$N = 32$, $h = 0.25$ (or $h_0 = 0.25$ within the first implementation),
and the number of shooting stages such that $\tau \approx 6.0$.
Once a \rpo\ is found, its existence in the \KS\ PDE is verified
 numerical approximation improved by increasing the number of
Fourier modes ($N = 64$) and reducing the step size ($h = 0.1$).
    \edit{
Among the thousands of orbits
computed in this study only two failed this higher-resolution test.
    }
