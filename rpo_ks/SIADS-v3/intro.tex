% intro.tex
% $Author$ $Date$

\section{Introduction}

Recent experimental and theoretical advances\rf{science04}
support a dynamical vision of turbulence:
For any finite  spatial resolution,
a turbulent flow follows approximately for a finite time
a pattern belonging to a
{ finite alphabet}
of admissible patterns.
The long term dynamics is
a {walk through the space of these unstable patterns}.
The question is how to characterize and classify such patterns?
Here we follow the seminal Hopf paper\rf{hopf48}, and  visualize
turbulence not as  a sequence of
spatial snapshots in turbulent evolution,
but as a trajectory in an
 infinite-dimensional \statesp\ in which an
instant in turbulent evolution is
a {unique} point. In the dynamical systems approach,
theory of turbulence for a given system, with given boundary conditions,
is given by
(a) the geometry of the \statesp\ and (b) the associated natural measure,
\ie,
the likelihood that asymptotic dynamics visits a given \statesp\ region.

We pursue this program in context of the \KS\ (KS) equation,
one of the simplest physically interesting spatially extended
nonlinear systems.  Holmes, Lumley and Berkooz\rf{Holmes96} offer a
delightful discussion of why this system deserves study as a staging
ground for studying turbulence in full-fledged Navier-Stokes
boundary shear flows.

Flows described by partial differential equations (PDEs) are
said to be infinite-dimensional because if one writes them
down as a set of ordinary differential equations (ODEs), a set
of infinitely many ODEs is needed to represent the dynamics
of one PDE. Even though their {\statesp} is thus
infinite-dimensional, the long-\-time dynamics of viscous
flows, such as Navier-Stokes, and PDEs modeling them, such as
Kuramoto-Sivashinsky, exhibits, when dissipation is high and
the system spatial extent small, apparent `low-dimensional'
dynamical behaviors. For some of these the asymptotic
dynamics is known to be confined to a finite-\-dimensional
{\em inertial manifold}, though the rigorous upper bounds on
this dimension are not of much use in the practice.

For large spatial extent the complexity of the spatial
motions also needs to be taken into account. The systems
whose spatial correlations decay sufficiently fast, and the
attractor dimension and number of positive Lyapunov exponents
diverges with system size are said\rf{HNZks86,man90b,cross93}
to be extensive, `spatio-temporally chaotic' or `weakly
turbulent.' Conversely, for small system sizes the accurate
description might require a large set\rf{GHCW07} of coupled
ODEs, but dynamics can still be `low-dimensional' in the
sense that it is characterized with one or a few positive
Lyapunov exponents. There is no wide range of scales
involved, nor decay of spatial correlations, and the system
is in this sense only `chaotic.'

For a subset of physicists and mathematicians who study
idealized `fully developed,' `homogenous' turbulence the
generally accepted usage is that the `turbulent' fluid is
characterized by a range of scales and an energy cascade
describable by statistic assumptions\rf{frisch}. What experimentalists,
engineers, geophysicists, astrophysicists actually observe
looks nothing like a `fully developed turbulence.' In the
physically driven wall-bounded shear flows, the turbulence is
dominated by unstable \emph{coherent structures}, that is,
localized recurrent vortices, rolls, streaks and like. The
statistical assumptions fail, and a dynamical systems
description from first principles is called for\rf{Holmes96}.


Dynamical \statesp\ representation of a PDE is
infinite-dimensional, but the KS flow is strongly contracting
and its non-wandering set, and, within it, the set of
invariant solutions investigated here, is embedded into a
finite-dimensional inertial manifold\rf{FNSTks85} in a
non-trivial, nonlinear way. `Geometry' in the title of this
paper refers to our attempt to systematically triangulate
this set in terms of dynamically invariant solutions (\eqva,
\po s, $\ldots$) and their unstable manifolds, in a PDE
representation and numerical simulation algorithm independent
way. The goal is to describe a given `turbulent' flow
quantitatively, not model it qualitatively by a
low-dimensional model. For the case investigated here, the
\statesp\ representation dimension $d \sim 10^2$ is set by
requiring that the exact invariant solutions that we compute
are accurate to $\sim 10^{-5}$.

Here comes our quandary. If we ban the words `turbulence' and
`spatiotemporal chaos' from our study of small extent
systems, the relevance of what we do to larger systems is
obscured. The exact unstable coherent structures we determine
pertain not only to the spatially small `chaotic' systems,
but also the spatially large `spatiotemporally chaotic' and
the spatially very large `turbulent' systems.
So, for the lack of more precise nomenclature, we take the
liberty of using the terms `chaos,' `spatiotemporal chaos,'
and `turbulence' interchangeably.


In previous work, the \statesp\ geometry and the natural measure for
this system have been
studied\rf{Christiansen97,LanThesis,lanCvit07} in terms of unstable
periodic solutions restricted to the antisymmetric subspace of the
KS dynamics.

The focus in this paper is on the role continuous symmetries
play in spatiotemporal dynamics. The notion of exact
periodicity in time is replaced by the notion of relative
spatiotemporal periodicity, and \reqva\ and \rpo s here play
the role the \eqva\ and \po s played in the earlier studies.
Our search for \rpo s in KS system was inspired by Vanessa
L{\'o}pez\rf{lop05rel} investigation of {\rpo s} of the
Complex Ginzburg-Landau equation.  However, there is a vast
literature on {\rpo s} since their first appearance, in
Poincar\'e study of the 3-body problem\rf{ChencinerLink,rtb},
where the Lagrange points are the \reqva.  They arise in
dynamics of systems with continuous symmetries, such as
motions of rigid bodies, gravitational $N$-body problems,
molecules and nonlinear waves. Recently Viswanath\rf{Visw07b}
has found both \reqva\ and \rpo s in
the plane Couette problem.
A Hopf bifurcation of a traveling
wave\rf{AGHO288,AGHks89,Krupa90} induces a small
time-dependent modulation. Brown and Kevrekidis\rf{BrKevr96}
study bifurcation branches of \po s and \rpo s in KS system
in great detail. For our system size ($\alpha=49.04$ in their
notation) they identify a periodic orbit branch. In this
context \rpo s are referred to as `modulated traveling
waves.' For fully chaotic flows we find this notion too
narrow. We compute 60,000 \po s and \rpo s that are in no
sense small `modulations' of other solutions, hence our
preference for the well established notion of a `\rpo.'
          

Building upon the pioneering work of
\refrefs{KNSks90,ksgreene88,BrKevr96}, we undertake here a
study of the \KS\ dynamics for a specific system size $L =
22$, sufficiently large to exhibit many of the features
typical of `turbulent' dynamics observed in large KS systems,
but small enough to lend itself to a detailed exploration of
the  \eqva\ and \reqva, their stable/unstable manifolds,
determination of a large number of \rpo s, and a preliminary
exploration of the relation between the observed
spatiotemporal `turbulent' patterns and the \rpo s.

In presence of a continuous symmetry any solution belongs to a group
manifold of equivalent solutions. The problem: If one is to
generalize the periodic orbit theory to this setting, one needs to
understand what is meant by solutions being nearby (shadowing) when
each solution belongs to a manifold of equivalent solutions. In a
forthcoming publication\rf{SCD09b} we resolve this puzzle by implementing
symmetry reduction. Here we demonstrate that, for \rpo s visiting the
neighborhood of equilibria, if one picks any
particular solution, the universe of all other solutions is rigidly
fixed through a web of heteroclinic connections between them. This
insight garnered from study of a 1-dimensional \KS\ PDE is more
remarkable still when applied to the plane Couette flow\rf{GHCW07},
with 3-$d$ velocity fields and two translational symmetries.


The main results presented here are: (a) Dynamics visualized through
physical, symmetry invariant observables, such as `energy,'
dissipation rate, \etc,
and through
projections onto dynamically invariant, PDE-discretization
independent \statesp\ coordinate frames, \refsect{sec:energy}. (b)
Existence of a rigid `cage' built by heteroclinic connections
between \eqva, \refsect{sec:L22}. (c) Preponderance of
unstable \rpo s and their likely role as the skeleton underpinning
spatiotemporal turbulence in systems with continuous symmetries,
\refsect{sec:rpos}.
