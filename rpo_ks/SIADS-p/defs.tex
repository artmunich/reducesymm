% defs.tex
% SIAM Journal on Applied Dynamical Systems style
% $Author$ $Date$

%%%%%%%%%%%%% SETUP          %%%%%%%%%%%%%%%%%%%
    \usepackage{amssymb}
    \usepackage{amsmath}
    \usepackage{amsfonts}
    \usepackage{graphicx}
    \usepackage{subfigure}
    \usepackage{dsfont}
    \usepackage{mathrsfs}
    \usepackage{color} % dvips allows for colors
    \bibliographystyle{hsiam}

    \usepackage[bookmarks,colorlinks]{hyperref}
    \hypersetup{
   pdfauthor={P. Cvitanovic, R. L. Davidchack and E. Siminos},
   pdfkeywords=Kuramoto-Sivashinsky chaos,
   pdftitle=State space geometry of chaotic Kuramoto-Sivashinsky flow
                }


%%%%%%%%%%%% SHORTCUTS, project specific %%%%%%%%%%
\newcommand{\po}{periodic orbit}
\newcommand{\Po}{Periodic orbit}
\newcommand{\rpo}{relative periodic orbit}
\newcommand{\Rpo}{Relative periodic orbit}

\newcommand{\eqv}{equilibrium}
\newcommand{\Eqv}{equilibrium}
\newcommand{\eqva}{equilibria}
\newcommand{\Eqva}{Equilibria}
\newcommand{\reqv}{relative equilibrium}
\newcommand{\Reqv}{Relative equilibrium}
\newcommand{\reqva}{relative equilibria}
\newcommand{\Reqva}{Relative equilibria}

\newcommand{\KS}{Kuramoto-Sivashinsky}
\newcommand{\KSe}{Kuramoto-Sivashinsky equation}
\newcommand{\pCf}{plane Couette flow}
\newcommand{\PCf}{Plane Couette flow}

\newcommand{\tildeL}{\ensuremath{\tilde{L}}}
\newcommand{\Lint}[1]{\frac{1}{L}\!\oint d#1\,}


\newcommand{\EQV}[1]{\ensuremath{\mathrm{E}_{#1}}}
\newcommand{\REQV}[2]{\ensuremath{\mathrm{TW}_{#1#2}}} 
\newcommand{\PO}[1]{\ensuremath{\mathrm{PO}_{#1}}}
\newcommand{\RPO}[1]{\ensuremath{\mathrm{RPO}_{#1}}}

\newcommand{\expctE}{\ensuremath{E}}    % E space averaged

%%%%%%%%%%%%    CROSS REFERENCING, STANDARD    %%%%%%%%%%%%%%%%%

\newcommand{\rf}      [1] {~\cite{#1}}
\newcommand{\refref}  [1] {ref.~\cite{#1}}
\newcommand{\refRef}  [1] {Ref.~\cite{#1}}
\newcommand{\refrefs} [1] {refs.~\cite{#1}}
\newcommand{\refRefs} [1] {Refs.~\cite{#1}}
\newcommand{\refeq}   [1] {(\ref{#1})}
\newcommand{\refeqs}  [2] {(\ref{#1}--\ref{#2})}
\newcommand{\refpage} [1] {page~\pageref{#1}}
    % Phys Rev style: Figure to start a sentence, else Fig.
\newcommand{\reffig}  [1] {Figure~\ref{#1}}
\newcommand{\reffigs} [2] {Figures~\ref{#1} and~\ref{#2}}
\newcommand{\refFig}  [1] {Figure~\ref{#1}}
\newcommand{\refFigs}  [2] {Figures~\ref{#1} and~\ref{#2}}
\newcommand{\refFigToFig}  [2] {Figures~\ref{#1} -~\ref{#2}}
\newcommand{\reftab}  [1] {Table~\ref{#1}}
\newcommand{\refTab}  [1] {Table~\ref{#1}}
\newcommand{\reftabs} [2] {Tables~\ref{#1} and~\ref{#2}}
\newcommand{\refsect} [1] {sect.~\ref{#1}}
\newcommand{\refsects}[2] {sects.~\ref{#1}--\ref{#2}}
\newcommand{\refSect} [1] {Sect.~\ref{#1}}
\newcommand{\refSects}[2] {Sects.~\ref{#1}--\ref{#2}}
\newcommand{\refappe} [1] {appendix~\ref{#1}}
\newcommand{\refappes}[2] {appendices~\ref{#1}--\ref{#2}}
\newcommand{\refAppe} [1] {Appendix~\ref{#1}}

%%%%%%%%%%%%%%% EQUATIONS, STANDARD %%%%%%%%%%%%%%%%%%%%%%%%%%%%%%%

\newcommand{\beq}{\begin{equation}}
\newcommand{\eeq}{\end{equation}}
\newcommand{\ee}[1]{\label{#1} \end{equation}}
\newcommand{\bea}{\begin{eqnarray}}
\newcommand{\ceq}{\nonumber \\ & & }
\newcommand{\continue}{\nonumber \\ }
\newcommand{\nnu}{\nonumber}
\newcommand{\eea}{\end{eqnarray}}



%%%%%%%%%%%%%%  Abbreviations %%%%%%%%%%%%%%%%%%%%%%%%%%%%%%%%%%%%%%%%

\newcommand{\etc}{{\em etc.}}       % etcetera in italics
\newcommand{\ie}{{that is}}     % use Latin or English?  Decide later.
\newcommand{\cf}{{\em cf.}}
\newcommand{\etal}{{\em et al.}}              % etcetera in italics

\newcommand{\jacobianM}{fundamental matrix} % standard name
\newcommand{\jacobianMs}{fundamental matrices}  %
\newcommand{\JacobianM}{Fundamental matrix} %
\newcommand{\JacobianMs}{Fundamental matrices}  %
\newcommand{\stabmat}{stability matrix}     % stability matrix
\newcommand{\Stabmat}{Stability matrix}     % Stability matrix
\newcommand{\statesp}{state space}
\newcommand{\Statesp}{State space}

%%%%%%%%%%%%%%% Sundry symbols within math eviron.: %%%%%%%%%%%%

\newcommand{\obser}{a}      % an observable from phase space to R^n
\newcommand{\Obser}{A}      % time integral of an observable
\renewcommand\Im{\ensuremath{{\rm Im\,}}}
\renewcommand\Re{\ensuremath{{\rm Re\,}}}
\renewcommand{\det}{\mbox{\rm det}\,}
\newcommand{\expct}    [1]{\left\langle {#1} \right\rangle}
\newcommand{\timeAver} [1]{\overline{#1}}
\newcommand{\matId}{\ensuremath{\mathbf 1}}       % matrix identity

\newcommand{\pVeloc}{\ensuremath{v}}    % phase-space velocity
\newcommand{\Mvar}{\ensuremath{A}}  % matrix of variations
\newcommand{\jMps}{{J}}    % jacobiam matrix, full phase space
                   % Fredholm det jacobian weight:

\newcommand{\jEigvec}[1]{\ensuremath{{\mathbf e}^{(#1)}}}   % jacobiam eigenvector
\newcommand{\ExpaEig}{\ensuremath{\Lambda}}
\newcommand{\translGen}{\ensuremath{\mathcal{L}}}

%%   optional parameter comes in [\ldots], for example
%%   \newcommand\eigRe[1][ ]{\ensuremath{\mu_{#1}}}
%%   no subscript: \eigRe\
%%   with subscript j: \eigRe[j]
%%
\newcommand{\eigExp}[1][ ]{\ensuremath{\lambda_{#1}}}   % complex eigenexponent
\newcommand{\eigRe}[1][ ]{\ensuremath{\mu_{#1}}}    % Re eigenexponent
\newcommand{\eigIm}[1][ ]{\ensuremath{\nu_{#1}}}    % Im eigenexponent


%%%%%%%%%% flows: %%%%%%%%%%%%%%%%%%%%%%%%%%%%

\newcommand\flow[2]{{f^{#1}(#2)}}
\newcommand\pSpace{x}       % phase space x=(q,p) coordinate
\newcommand\stagn{q}        %equilibrium/stagnation point suffix
\newcommand{\bbU}{\mathbb{U}}
\newcommand{\bbUsymm}{\bbU_{c}}

%%%%%%%%%% periods: %%%%%%%%%%%%%%%%%%%%%%%%%%%%

\newcommand\period[1]{{T_{#1}}}         %continuous cycle period
\newcommand{\Refl}{\ensuremath{R}}
\newcommand{\Shift}{\ensuremath{\tau}}
\newcommand{\shift}{\ensuremath{\ell}}
\newcommand{\velRel}{\ensuremath{c}}    % relative state velocity
\newcommand\Lyap{\lambda}           %Lyapunov exponent

%%%%%%%%%%%%%%%%%%%%%%%%%%%%%%%%%%%%%%%%%%%%%%%%%%%%
