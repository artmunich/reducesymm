\section{Fourier space representation} 
\label{s:FourierModes}

% derived from KSe.tex \section{{\KSe}}
%\label{sec:KSequil}
% Predrag, Lan					15dec2004
% $Author$ $Date$

% \index{Fourier!mode!truncation}
% \index{truncations!Fourier}

\noindent
Spatial periodic boundary condition $u(x,t)=u(x+2\pi\tilde{L},t)$
makes it convenient to work in the Fourier space, 
\beq
  u(x,t)=\sum_{k=-\infty}^{+\infty} a_k (t) e^{ i k x /\tilde{L} }
\, .
% \label{fseries}
\ee{eq:ksexp}
with \refeq{ks-L} replaced by an infinite set of 
ODEs for the Fourier coefficients:
\beq
% \dot{b}_k= \left( \frac{k}{\tilde{L}}\right)^2.
%      \left( 1 - \left( \frac{k}{\tilde{L}}\right)^2  \right) a_k 
%	 + i \frac{k}{\tilde{L}} \sum_{m=-\infty}^{+\infty} a_m a_{k-m}
\dot{b}_k=(k/\tilde{L})^2\left( 1 - (k/\tilde{L})^2  \right)a_k 
 	 + i (k/\tilde{L}) \sum_{m=-\infty}^{+\infty} a_m a_{k-m}
\,.
\ee{expan}
%
% This is the infinite set of ordinary differential equations promised
% in this chapter's introduction. 

% \index{infinite-dimensional flows}
% \index{flow!infinite-dimensional}

Since $u(x,t)$ is real,
$ %\[
a_k=a_{-k}^*
\,,
$ %\] %\label{cplx-b}
so we can replace the sum over $k$ in \refeq{expan} by a
sum over $k \geq 0$.
As  $\dot{a_0}=0$, $a_0$ is a conserved quantity,
in our calculations
fixed to $a_0=0$ by
the Galilean invariance condition \refeq{GalInv}.

% \index{Galilean invariance}
% \index{invariance!Galilean}

\subsection{Antisymmetric subspace} 
\label{s:AntisymmSubsp}

Fix the  Poincar\'e section to be the hyperplane
$\Re a_1=0$. We integrate \refeq{expan} with the initial
 conditions
$\Re a_1=0$, and arbitrary values of the coordinates  $a_2, \ldots, a_N$, where
$N$ is the truncation order.  When $\Re a_1$ becomes
$0$ the next time,  the coordinates  $a_2, \ldots, a_N$ are mapped
into $(a_2', \ldots a_N')=P(a_2, \ldots, a_N)$, where $P$ is the  Poincar\'e
map. $P$ defines a mapping of a $N-1$ dimensional hyperplane into itself.
Under successive iterations of  $P$, any trajectory
approaches the attractor ${\cal A}$, which itself is an invariant
set under $P$.

A trajectory of
 (\ref{expan}) can cross the plane $a_1=0$ in two possible ways:
 either when
$\dot{a_1}>0$ (``up'' intersection)
or when  $\dot{a_1}<0$ (``down'' intersection),
 with the ``down'' and ``up'' crossings
alternating.
It then makes sense to define the  Poincar\'e map $P$ as a transition between,
say, ``up'' and ``up'' crossing.
With  Poincar\'e section defined as the ``up-up'' transition,
it is natural to define a ``down-up'' transition map $\Theta$. Since
$\Theta$ describes the transition from down to up (or up to down) state,
the map $\Theta^2$ describes the transition  up-down-up, that is
$\Theta^2=P$.

Now, with the help of the 
reflection $\mathbf{I}$ and shift symmetry $ \mathbf{S}$
operations
\refeq{KSparity},
\refeq{KSshift}
the  attractor ${\cal A}_{tot}$ can be
decomposed into four pieces:
 ${\cal A}_{tot}={\cal A} \cup S {\cal A}  \cup \Theta {\cal A}
  \cup \Theta S {\cal A} $. 

A decomposition
of the attractor into four disjoint sets
is usually not possible, since sometimes $ {\cal A}$ overlaps with
$\Theta S{\cal A} $ (in this case $\Theta  {\cal A}$ will also  overlap with
$S {\cal A} $).
In any case  the set $ {\cal A}$ can be taken as
the fundamental
domain of the Poincar{\'e} map, with $S  {\cal A} $,
$\Theta  {\cal A} $ and $\Theta S  {\cal A} $ its images under the
$S$ and $\Theta$ mappings.


% The Fourier coefficients $a_k$ are in general complex numbers.
% % of time $t$.  
% We can
% isolate the antisymmetric subspace of the system \refeq{ks-L} by
% considering the case of $a_k$ pure imaginary, $a_k= i a_k$, where
% $a_k$ are real, with the evolution equations
% \beq
% % \dot{a}_k=(k^2- k^4)a_k - k \sum_{m=-\infty}^{\infty} a_m a_{k-m}
% \dot{a}_k = (k/\tilde{L})^2\left( 1 - (k/\tilde{L})^2  \right)a_k 
%  	 - (k/\tilde{L}) \sum_{m=-\infty}^{+\infty} a_m a_{k-m}
% \,.
% \ee{expan-symm}

Since \KSe\ preserves
antisymmetric solutions, one can isolate the antisymmetric
subspace 
$u(x,t)=-u(-x,t)$, or in terms of Fourier coefficients,
$a_{-k}= - a_k$. 
For Fourier coefficients which respect the $x \to -x$ symmetry of
\KSe, see discussion in \refref{Christiansen:97},
and references therein.
In \refrefs{Christiansen:97,Lan:Thesis} 
this option was used to eliminate
the continuous translational symmetry.

In the antisymmetric subspace the translational 
invariance of the full system reduces
to the invariance under discrete
translation by half a spatial period $L$.
In the Fourier representation \refeq{expan}, 
this corresponds to invariance under 
\beq
a_{2m} \to a_{2m}\,, a_{2m+1} \to -a_{2m+1}
\,, m \in \mathbb{Z}
\,.
\ee{FModInvSymm} 
The antisymmetric condition amounts to imposing
$u(0,t)=0$ boundary condition, with
the size of the system reduced to
the $[0,L/2]$ interval. In
comparing our numerical results with % other authors' 
calculations for
the full, unrestricted dynamics on $[0,L]$, we define
the non-dimensionalized system size as
$\tilde{L} = {L}/{(4 \pi \sqrt{\nu})}$,
corresponding to a system defined on the
$[0,L/2]$ domain. 

In this paper we study the full \KS\ system invariant
under continuous translations. Due to the lack of self-adjointness
(non-normality) of the linearized \KS\ flow, 
the antisymmetric subspace
is unstable under small perturbations and generic solution of 
\KSe\ belongs to the full, periodic space. Nevertheless, some of
the \eqva\ and of the shortest periodic orbits lie in this subspace
and can play important role for the topology of the flow - examples
of such solutions will be discussed in \refsect{s:L22}.
