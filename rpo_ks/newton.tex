% newton.tex
%
% Predrag           jun 20 2006
% Vaggelis          may 20 2006
% $Author$ $Date$


\subsection{Implementing Newton's method  for \rpo s}
\label{sec:NewtRPOs}

The relative periodic condition is
\[
    u(x+\shift,t+\period{})=u(x,t)
        \,.
\]
The Fourier representation diagonalizes the rotation operator, 
\beq
    \mathbf{g}(\shift)  a(t+\period{})=a(t)\,,\qquad
            \mathbf{g}_{jk}(d) = \delta_{jk} e^{ik\, \shift/\tildeL}\,,
    \label{eq:RPO}
\eeq
so the relative periodic condition
\beq
    e^{ik\, \shift /\tildeL}a_k(t+\period{})=a_k(t)  % \forall k \in \mathds{Z}
    \qquad \mbox{(no summation on $k$)}
    \label{eq:RPOcondition}
\eeq
amounts to the $k$th mode complex plane rotation by an 
angle $-k\, d /\tildeL$.
% A \rpo\ returns after time $\period{}$ to a point in 
% \statesp\ with components $a_k(t+\period{})$  with respect to $a_k(t)$. 
Starting with an initial guess $a$ for a point on a \rpo\ $p$
we use Newton's method to find an improved approximation to $a_p$ 
% of condition  \refeq{eq:RPO}:
\beq
    a_p=\mathbf{g}(\shift_p)  \flow{\period{p}}{a_p} \,,
    \label{eq:RPOcond}
\eeq
with period $\period{p}$ and shift $\shift_p$. Let 
$a$, $\period{}$ and $\shift$ 
be our guess cycle point, period, and shift, respectively. 
Taylor expand $\mathbf{g}(\shift_p)  \flow{\period{p}}{a_p}$ 
around $a_p$ to linear order in
$\delta a=a-a_p$, $\delta \period{}=\period{} - \period{p}$ and 
$\delta \shift=\shift-\shift_p$:
% Here $\jMps^t(x)$ is the Jacobian matrix, defined for a general flow through
% \beq
%       J^t_{ij}(x_o)=\left.\frac{\partial x_i(t)}{\partial x_j}\right|_{x=x_0}\,.
% \eeq
% The Jacobian matrix is obtained by integrating the equation:
% \beq
%       \dot{\mathbf{J}}^t=\mathbf{A J}^t \, ,
%   \label{eq:Adef}
% \eeq
% subject to the initial condition:
% \beq
%       \mathbf{J}^0=\mathbf{1} \, ,
% \eeq
% Here $\mathbf{A}$ is the matrix of variations defined as:
% \beq
%   A_{kj}=\frac{\partial \dot{x}_k}{\partial x_j}\,.
% \eeq
% 
% On the other hand, we have
% \bea
%   R(\kappa'+\Delta\kappa) & = & R(\kappa')R(\Delta\kappa) \continue
%               & \simeq & R(\kappa')
%           (\mathbf{1}+i\mbox{Diag}[k]\Delta\kappa/\tildeL)\,.
%   \label{eq:TaylorR}  
% \eea
% 
\beq
    \left({1}-\mathbf{g}(\shift)\jMps^{\period{}}(a)\right) \delta a 
   - \mathbf{g}(\shift)v(f^{\period{}}(a)) \delta \period{} 
                            - \mathbf{L}(\mathbf{g}(\shift)\flow{\period{}}{a})\delta \shift  
                    \,\simeq\, \mathbf{g}(\shift)\flow{\period{}}{a}-a\,,
    \label{eq:NewtonBasicCond}          
\eeq
where $\jMps^{\period{},\shift}_{ij} 
  = \mathbf{g}(\shift)\partial_i a(\period{})/\partial_j$
is the {\jacobianM}, and
$L_{kj}=\frac{ik}{\tildeL}\delta_{kj}$ is the Lie algebra translation
generator. The matrix $\mathbf{g}(d)\jMps^{\period{}}(a)$ 
has two unit eigenvalues in 
the limit $a\rightarrow a_p$, one associated with the invariance along 
the direction of the flow and the other with the
translational invariance of the system. Thus \refeq{eq:NewtonBasicCond} 
needs to be augmented by two conditions to
eliminate the indeterminacy introduced by the (close to) zero eigenvalues of 
$\mathbf{1}-\mathbf{g}(d)\jMps^{\period{}}(a)$. Following 
\refref{ViswanathPC06} we choose the conditions 
\bea
    v(a)\cdot\delta a & = & 0 \label{eq:NewtonAux1} \,\\
    (\mathbf{D[g]}a)\cdot \delta a & = & 0 \label{eq:NewtonAux2}\,.
\eea
The requirement imposed by \refeqs{eq:NewtonAux1}{eq:NewtonAux2}\ on the solution vector $\delta a$ of \refeq{eq:NewtonBasicCond} 
is that it vanishes along the directions of the flow and of infinitesimal translation of the initial condition.

Equations \refeq{eq:NewtonBasicCond} and \refeqs{eq:NewtonAux1}{eq:NewtonAux2}
can be compactly represented in a single matrix equation:
\beq
    \left( \begin{array}{ccc}
       {1}-\mathbf{g}(d)\jMps^{\period{}}(a)    & -\mathbf{g}(d)v(f^{\period{}}(a))   
                                        & -\mathbf{D[g]}(\mathbf{g}(\shift)f^{\period{}}(a))  \\
        v(a)^{\dagger}          & 0     & 0     \\
        (\mathbf{D[g]}a)^\dagger    & 0     & 0 
     \end{array}
     \right)
     \left(\begin{array}{c}
       \delta a \\
       \delta \period{} \\
       \delta \shift
     \end{array}\right)
     =
     \left(\begin{array}{c}
       \mathbf{g}(d)f^{\period{}}(a)-a \\
       0     \\
       0
     \end{array}\right)\,.
     \label{eq:NewtonScheme}
\eeq
where $v^\dagger$ denotes the adjoint of $v$. 

% In our computations the real valuedness of $u(x,t)$ 
% has been explicitly taken into account through $a_{-k} = a^*_{k}$.
% Details are given in \refappe{ap:rpo}.

