% newton.tex
%
% Predrag           jun 20 2006
% Vaggelis          may 20 2006
% $Author$ $Date$


\section{Numerical searches for \reqva\ and \rpo s}

\subsection{Newton's method for determining \reqva}
 
  Our first task is to determine all \eqva\
and  \reqva\ of the \KS\ system for a given fixed domain size
$L$.
This problem is equivalent to
finding periodic orbits of a 3-$d$ ODE system \refeq{eq:3dks} together
with determining appropriate value of the integration constant $c$.
It turns out that the  periodic solutions of  \refeq{eq:3dks} and
often difficult to find by the Newton method. 
In \refrefs{CvitLanCrete02,lanVar1,Lan:Thesis,LanCvi06} they are
determined by a the variational {\descent} method.

In the current investigation
we prefer to search for \eqv\ solutions in
the Fourier space
\beq
    \dot{b}_k=\dot{c}_k=0
\,.
\eeq
This is inefficient comapred to the above 3-$d$ ODE methods, but
it serves as a warmup for, and the first test of methods that we then use
to locate \po s and \rpo s.

%  Expanding $\dot{b}_k(a)$ and $\dot{c}_k(a)$ around our initial guess $a_o$ and demanding that they satisfy the equilibrium 
%  condition, we get
%  \bea
%   \dot{b}_k(a) & = & \dot{b}_k(a_o)+\left.\frac{\partial \dot{b}_k}{\partial b_j}\right|_{a_o}\delta b_j + \left.\frac{\partial \dot{b}_k}{\partial c_j}\right|_{a_o}\delta c_j = 0 \continue
%   \dot{c}_k(a) & = & \dot{c}_k(a_o)+\left.\frac{\partial \dot{c}_k}{\partial b_j}\right|_{a_o}\delta b_j + \left.\frac{\partial \dot{c}_k}{\partial c_j}\right|_{a_o}\delta c_j = 0
%  \eea
%  or in matrix form
%  \beq
%     \left( \begin{array}{cc}
%         \frac{\partial \dot{b}}{\partial b} & \frac{\partial \dot{b}}{\partial c} \\
%         \frac{\partial \dot{c}}{\partial b}   & \frac{\partial \dot{c}}{\partial c}
%      \end{array}
%      \right)_{a_o}
%      \left(\begin{array}{c}
%        \delta b  \\
%        \delta c
%      \end{array}\right)
%      =
%      \left(\begin{array}{c}
%        -\dot{b}(a_o) \\
%        -\dot{c}(a_o)
%      \end{array}\right)\,,
%      \label{eq:NewtonEquil}
% \eeq
% where $\partial{\dot{b}} / \partial{b}$ \etc are $d \times d$ submatrices. Solving this
% system of equations for the corrections $\delta b$ and  $\delta c$ and using the refined solution
% as an initial guess yields  an approximation to the solution of the system.
%  

%% ES Sep 11 2006 Commented out PC text. Added appendix with details of
%% Calculations to answer JFG question of Sep 7 2006.
%PC incorporated JFG Sep 7 2006 remark:
% Additional computational savings can be achieved
% for the \KS\ solutions that are restricted
% to the antisymmetric subspace of \refsect{s:AntisymmSubsp}.
% For the 3-$d$ ODE periodic solution this symmetry reduces
% the length of the loop by factor $1/2$. In the Fourier representation
% the continuous translation symmetry is eliminated by
% setting  the coefficients to purely imaginary values $i a_k$,
% thus reducing the dimensionality of the (truncated) phase space
% by factor 2. In fluid dynamics such savings are very significant, 
% and are enforced whenever possible; for small cell \KS\ systems
% studied here they are not so important, and we search for such
% \eqva\ in the full space, and use the size of the (symmetry violating)
% real Fourier coefficients as an
% additional test of the accuracy of our Newton searches.
  
% JFG Sep 7 2006: Do you then
% enforce real-valuedness in your Newton-descent via the constraint
% $a_{-k} = a^*{k}$ (the conjugates that then appear in the equations
% are nondifferentiable which is a big pain) or do you let the solutions
% go complex and then choose the real part at the end? 
% The cost of that
% is that the dimension of your search space is twice as big as it needs 
% to be. That's an unacceptable cost in fluids; perhaps in \KS\ it's not. 
% In any case, I think you should (1) either clarify that you're no 
% longer working in the antisymmetric subspace or eliminate its mention 
% earlier, and (2) explain how you ultimately arrive at real-valued 
% solutions.

\subsection{Implementing Newton's method  for \rpo s}
\label{sec:NewtRPOs}

The relative periodic condition
\beq
    u(x+d,t+T)=u(x,t) \,
\eeq
translates in Fourier space into
\beq    
    \sum_{k=-\infty}^{+\infty} a_k (t+T) e^{ i k (x+d) / \tildeL} 
        = \sum_{k=-\infty}^{+\infty} a_k (t) e^{ i k x / \tildeL} \,
\eeq
or
\beq
    e^{ik\, d /\tildeL}a_k(t+T)=a_k(t) \,,\ \forall k \in \mathds{Z}\ \ \ \mathrm{(no\ summation)}.
    \label{eq:RPOcondition}
\eeq
We see that a relative periodic orbit returns after time $T$ to a point in 
phase space with components $a_k(t+T)$ rotated in the complex plane by an 
angle $-k\, d /\tildeL$ with respect to $a_k(t)$. In matrix notation, we write \refeq{eq:RPOcondition} as
\beq
    \mathbf{g}(d)  a(t+T)=a(t)\,,
    \label{eq:RPO}
\eeq
where we have defined
\beq
    \mathbf{g}(d) \equiv \mbox{Diag}[e^{ik\, d/\tildeL}]\,.
\eeq
%We notice that $R(\kappa)$ is not a rotation operator..

% Consider an initial guess $a'$ for a point on a relative periodic orbit and assume that it lies on
% a \Poincare section $\mathcal{P}$ at $t=0$. Suppose that $\mathcal{P}$ is a hyperplane in
% $\mathds{R}^{2d}$. The flow $f^t$ defined by \refeq{eq:Fcoef} transports 
% this point after time $T'$ into $a'(T')=f^{T'}(a')$. Suppose that this point is such that $R(\kappa')f^{T'}(a')$
% is a point on $\mathcal{P}$. Consider next a point $a$ lying on $\mathcal{P}$ and in the neighborhood of $a'$,
% thus satisfying
% \beq
%   q \cdot (a'-a) = 0\,,
%   \label{eq:cond a}
% \eeq
% with $q$ a vector normal to $\mathcal{P}$. Point $a$ will be finally identified with the improved 
% approximation of a point on the periodic orbit.
% The flow transports $a$ to $f^{T'}(a)$, but now $R(\kappa')f^{T'}(a)$ is not in general on $\mathcal{P}$.
% Moreover we would like to have the freedom to adjust the guesses for $T'$ and $\kappa'$ into new values
% $T=T'+\Delta T$ and $\kappa=\kappa'+\Delta \kappa$ to improve their accuracy. 
% Let as consider such slightly different values $T$ and $\kappa$ such that $R(\kappa)f^{T}(a)$ lies on 
% $\mathcal{P}$. Then we have the condition
% \beq
%   q \cdot(R(\kappa')f^{T'}(a')-R(\kappa)f^{T}(a)) = 0\,.
%   \label{eq:cond Rf(a)}
% \eeq 

Starting with an initial guess $a$ for a point on a \rpo\ we use Newton's method to find an improved approximation to the true solution $a^*$ of condition  \refeq{eq:RPO}:
\beq
    a^*=\mathbf{g}(d^*)  f^{T^*}(a^*)\,,
    \label{eq:RPOcond}
\eeq
with period $T^*$ and shift $d^*$. Let $T$ and $d$ be our guess period and shift, respectively. 
Taylor expanding $\mathbf{g}(d^*)  f^{T^*}(a^*)$ around $a$ to linear order in the small quantities 
$\delta a=a^*-a$, $\delta T=T^*-T$ and $\delta d=d^*-d$, we get
% \bea
%   f^{T}(a)& \simeq & f^{T}(a')+\J^T(a') \Delta a \label{eq:fTaylorl1} \\ 
%       & \simeq & f^{T'}(a') + v \Delta T + \J^{T'}(a') \Delta a \label{eq:fTaylorl2} \,, 
% \eea
% where $v$ is evaluated at $f^{T'}(a')$. Here $\J^t(x)$ is the Jacobian matrix, defined for a general flow through
% \beq
%       J^t_{ij}(x_o)=\left.\frac{\partial x_i(t)}{\partial x_j}\right|_{x=x_0}\,.
% \eeq
% The Jacobian matrix is obtained by integrating the equation:
% \beq
%       \dot{\mathbf{J}}^t=\mathbf{A J}^t \, ,
%   \label{eq:Adef}
% \eeq
% subject to the initial condition:
% \beq
%       \mathbf{J}^0=\mathbf{1} \, ,
% \eeq
% Here $\mathbf{A}$ is the matrix of variations defined as:
% \beq
%   A_{kj}=\frac{\partial \dot{x}_k}{\partial x_j}\,.
% \eeq
% 
% In passing from \refeq{eq:fTaylorl1} to \refeq{eq:fTaylorl2} we have used the multiplicative 
% structure of the Jacobian, $\mathbf{J}^{T'+\delta T}(a')=\mathbf{J}^{\delta T}(f^{T'}(a'))\mathbf{J}^{T'}(a')$, 
% noticed that $\mathbf{J}^{\delta T}(f^{T'}(a'))=e^{\mathbf{A}\delta T}=\mathbf{1}+\mathbf{A}\delta T+\ldots$ 
% and dropped second order terms in the small quantities.
% 
% On the other hand, we have
% \bea
%   R(\kappa'+\Delta\kappa) & = & R(\kappa')R(\Delta\kappa) \continue
%               & \simeq & R(\kappa')
%           (\mathbf{1}+i\mbox{Diag}[k]\Delta\kappa/\tildeL)\,.
%   \label{eq:TaylorR}  
% \eea
% 
% Substituting \refeq{eq:fTaylorl2},\refeq{eq:TaylorR} into \refeq{eq:RPOcond} and keeping only first
% order terms in the small quantities, we get
% \beq
%   a+\delta a \simeq \mathbf{g}(d)  f^{T}(a) + \mathbf{D[g]}(\mathbf{g}(d) f^{T}(a))\delta d
%               + \mathbf{g} (d)v(f^{T}(a)) \delta T + \mathbf{g}(d) \J^{T}(a) \delta a\,,
% \eeq
% or
\beq
    \left(\mathbf{1}-\mathbf{g}(d)\J^{T}(a)\right) \delta a - \mathbf{g}(d)v(f^{T}(a)) \delta T 
                            - \mathbf{D[g]}(\mathbf{g}(d)f^{T}(a))\delta d  
                    \,\simeq\, \mathbf{g}(d)f^{T}(a)-a\,,
    \label{eq:NewtonBasicCond}          
\eeq
where $D[g]_{kj}=\frac{ik}{\tildeL}\delta_{kj}$. The matrix $\mathbf{g}(d)\J^{T}(a)$ has two unit eigenvalues in 
the limit $a\rightarrow a^*$, one associated with the invariance along the direction of the flow and the other with the
translational invariance of the system. Thus \refeq{eq:NewtonBasicCond} needs to be augmented by two conditions to
eliminate the indeterminacy introduced by the (close to) zero eigenvalues of $\mathbf{1}-\mathbf{g}(d)\J^{T}(a)$. Following 
\refref{ViswanathPC06} we choose the conditions 
\bea
    v(a)\cdot\delta a & = & 0 \label{eq:NewtonAux1} \,\\
    (\mathbf{D[g]}a)\cdot \delta a & = & 0 \label{eq:NewtonAux2}\,.
\eea
The requirement imposed by \refeqs{eq:NewtonAux1}{eq:NewtonAux2}\ on the solution vector $\delta a$ of \refeq{eq:NewtonBasicCond} 
is that it vanishes along the directions of the flow and of infinitesimal translation of the initial condition.

Equations \refeq{eq:NewtonBasicCond} and \refeqs{eq:NewtonAux1}{eq:NewtonAux2}
can be compactly represented in a single matrix equation:
\beq
    \left( \begin{array}{ccc}
       \mathbf{1}-\mathbf{g}(d)\mathbf{J}^{T}(a)    & -\mathbf{g}(d)v(f^{T}(a))   & -\mathbf{D[g]}(\mathbf{g}(d)f^{T}(a))  \\
        v(a)^{\dagger}          & 0     & 0     \\
        (\mathbf{D[g]}a)^\dagger    & 0     & 0 
     \end{array}
     \right)
     \left(\begin{array}{c}
       \delta a \\
       \delta T \\
       \delta d
     \end{array}\right)
     =
     \left(\begin{array}{c}
       \mathbf{g}(d)f^{T}(a)-a \\
       0     \\
       0
     \end{array}\right)\,.
     \label{eq:NewtonScheme}
\eeq
where $v^\dagger$ denotes the adjoint of $v$. 

In our computations the real valuedness of $u(x,t)$ 
has been explicitly taken into account through $a_{-k} = a^*_{k}$.
Details are given in \refappe{ap:rpo}.

