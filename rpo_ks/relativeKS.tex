% relativeKS.tex
% copied here from nsf/nsf06am/TEX/relativeKS.tex       Nov 1 2006
% $Author$ $Date$


%%%%%%%%%%%%%%%%%%%%%%%%%%%%%%%%%%%%%%%%%%%%%%%%%%%%%%%%%%%%%%%%%%
\PC{
    if there is something useful in this paragraph, incorporate into the
    text, the rest of this file goes to flotsam.tx.
    }
In \reffig{f:KS22unstM} the \eqv~\EQV{1} of
\reffig{f:KS22unstM}(a) is represented by the point~\EQV{1},
and its unstable manifold can be examined in great detail.
To each \eqv\ point corresponds a continuous family
of \eqva, and this leads to an unexpected feature of such
flows: While in dimensions higher than 2 heteroclinic connections
are a rarity (likelihood that unstable manifold of one
 \eqv\ precisely hits another \eqv\ point is zero),
for flows with continuous symmetries intersections of unstable
manifolds with continuous families of equivalent \eqva\ are common.
\refFigToFig{f:KS22E1man2}{f:KS22E3man} show
such heteroclinic connections.
% from an $\EQV{2}$~\eqv\ point to $\EQV{3}$~\eqv\ family.
These connections offer an invariant partition of the \statesp. 
%,and will be the basis of our
%{construction of symbolic dynamics}.
