% relativeKS.tex
% copied here from nsf/nsf06am/TEX/relativeKS.tex       Nov 1 2006
% $Author$ $Date$


%%%%%%%%%%%%%%%%%%%%%%%%%%%%%%%%%%%%%%%%%%%%%%%%%%%%%%%%%%%%%%%%
% former {figure}[t] \label{f:KS22cage}
\begin{figure} [t]
\begin{center}
(b) \includegraphics[width=0.3\textwidth]%,origin=c]%
        {figs/ks22E2-E3hetero.eps}
\end{center}
\caption{
(b) \EQV{2}~\eqv\ to \EQV{3}~\eqv\ heteroclinic
connection. Here we omit the unstable manifold of \EQV{2},
keeping only a few neighboring trajectories in order to indicate
the unstable manifold of \EQV{3}. The \EQV{2} and \EQV{3}
families of \eqva\ arising from the continuous translational
symmetry of KS on a periodic domain are indicated by the two circles.
\PCedit{
Edit the cage of heteroclinic connections,
xfig file /rpo\_ks/figs/ks22\_E1\_UM\_diag.fig
and rpo\_ks/figs/ks22\_E2\_UM\_diag.fig
    }
        }
\label{f:KS22unstM}
\end{figure}
%%%%%%%%%%%%%%%%%%%%%%%%%%%%%%%%%%%%%%%%%%%%%%%%%%%%%%%%%%%%%%%%%%
In \reffig{f:KS22unstM} the \eqv~\EQV{1} of
\reffig{f:KS22unstM}(a) is represented by the point~\EQV{1},
and its unstable manifold can be examined in great detail.
To each \eqv\ point corresponds a continuous family
of \eqva, and this leads to an unexpected feature of such
flows: While in dimensions higher than 2 heteroclinic connections
are a rarity (likelihood that unstable manifold of one
 \eqv\ precisely hits another \eqv\ point is zero),
for flows with continuous symmetries intersections of unstable
manifolds with continuous families of equivalent \eqva\ are common.
\refFig{f:KS22unstM} shows
such heteroclinic connections.
% from an $\EQV{2}$~\eqv\ point to $\EQV{3}$~\eqv\ family.
These connections partition the \statesp,
and will be the basis of our
{construction of symbolic dynamics}.
