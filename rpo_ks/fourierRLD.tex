% fourierRLD.tex
%
% Predrag               jun 20 2006
% $Author$ $Date$


\section{\KS\ according to Ruslan}
\label{sec:fourierRLD}

\PCedit{this section might make it as an appendix to the paper?}
\PC{move to siminos/thesis/chapters/ once rpo.tex is finalized}
%
The \KSe\ has a slightly different form:
\PC{adopted this form as the official Apr 17 2007}
\begin{equation}
  u_t=-{\textstyle\frac{1}{2}}(u^2)_x-u_{xx}- u_{xxxx} \, ,
\end{equation}
which can be reduced to Eq.~(\ref{eq:KS}) by the transformation $u
\rightarrow -2u$.

The \KSe\ in terms of Fourier modes:
\begin{equation}
  \hat{u}_k \equiv {\cal F}[u]_k = \frac{1}{L}\int_0^L u(x,t) e^{-ikx/\tildeL}dx\,,
  \qquad u(x,t) \equiv {\cal F}^{-1}[\hat{u}] = \sum_{k\in{\mathbb Z}} \hat{u}_k e^{ikx/\tildeL}
\end{equation}
 is given by
\begin{equation}
  \dot{\hat{u}}_k =[(k/\tildeL)^2-(k/\tildeL)^4]\hat{u}_k -
  \frac{ik}{2\tildeL}{\cal F}[({\cal F}^{-1}[\hat{u}])^2]_k\,.
\end{equation}
Since $u$ is real, the Fourier modes are related by $\hat{u}_{-k} =
\hat{u}^\ast_k$.

The above system is truncated as follows: The Fourier transform
${\cal F}$ is replaced by its discrete equivalent
\begin{equation}
  a_k \equiv {\cal F}_N[u]_k = \sum_{n = 0}^{N-1} u(x_n)
  e^{-ikx_n/\tildeL}\,,\qquad u(x_n) \equiv {\cal F}_N^{-1}[a]_n
  = \frac{1}{N}\sum_{k = 0}^{N-1} a_k e^{ikx_n/\tildeL}\,,
\end{equation}
where $x_n = 2\pi\tildeL/N$ and $a_{N-k} = a^\ast_k$.  Since $a_0
= 0$ due to galilean invariance and setting $a_{N/2} = 0$ (assuming
$N$ is even), the number of independent variables in the truncated
system is $N-2$.  The truncated system looks as follows:
\begin{equation}
  \dot{a}_k =[(k/\tildeL)^2-(k/\tildeL)^4]a_k -
  \frac{ik}{2\tildeL}{\cal F}_N[({\cal F}_N^{-1}[a])^2]_k\,.
\end{equation}
with $k = 1,\ldots,N/2-1$, although in the Fourier transform we need
to use $a_k$ over the full range of $k$ values from 0 to $N-1$.

The discrete Fourier transform ${\cal F}_N$ can be computed by FFT.
In Fortran and C, the routine {\tt REALFT} from Numerical Recipes
can be used.  In Matlab, it is more convenient to use complex
variables for $a_k$.  Note that Matlab function {\tt fft} is, in
fact, the inverse Fourier Transform.

To derive the equation for the matrix of variations, I use the fact
that ${\cal F}_N$ is a linear operator.  Since we need to
differentiate separately with respect to the real and imaginary
components of $a_k$, I use the notation $a_k = b_{2k-1} + ib_{2k}$.
\begin{equation}
  \frac{\partial \dot{a}_k}{\partial b_j} =
  [(k/\tildeL)^2-(k/\tildeL)^4]\delta_{kj} -
  \frac{ik}{\tildeL}{\cal F}_N[{\cal F}_N^{-1}[a]\cdot{\cal
  F}_N^{-1}[\delta_{kj}]]\,,\quad j = 1,\ldots,N-2
\end{equation}
where the dot indicates componentwise product, and the inverse
Fourier transform is applied separately to each column of
$\delta_{kj}$. Here, $\delta_{kj}$ is not a standard Kronecker
delta, but a complex valued $N\times N-2$ matrix:
\begin{equation}
  \delta_{kj} = \left(
  \begin{array}{ccccc}
  0 &  0 & 0 &  0 &\cdots\\
  1 &  i & 0 &  0 &\cdots\\
  0 &  0 & 1 &  i &\cdots\\
\multicolumn{5}{c}\dotfill \\
  0 &  0 & 0 &  0 &\cdots\\
\multicolumn{5}{c}\dotfill \\
  0 &  0 & 1 & -i &\cdots\\
  1 & -i & 0 &  0 &\cdots\\
  \end{array}  \right),
\end{equation}
with index $k$ running from 0 to $N-1$.  I admit, the notations are
a bit stretched here, but I find them convenient when coding this
equation using FFT.


