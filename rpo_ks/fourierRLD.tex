% fourierRLD.tex
%
% Predrag               jun 20 2006
% $Author$ $Date$


\section{\KS\ according to Ruslan}
\label{sec:fourierRLD}

\PCedit{this section might make it as an appendix to the paper?}
\PC{move to siminos/thesis/chapters/ once rpo.tex is finalized}
%
The \KSe\ has a slightly different form:
\PC{adopted this form as the official Apr 17 2007}
\beq
  u_t=-{\textstyle\frac{1}{2}}(u^2)_x-u_{xx}- u_{xxxx} \, ,
\eeq
which can be reduced to Eq.~(\ref{eq:KS}) by the transformation $u
\rightarrow -2u$.

The \KSe\ in terms of Fourier modes:
\beq
  \hat{u}_k = {\cal F}[u]_k = \frac{1}{L}\int_0^L u(x,t) e^{-ikx/\tildeL}dx\,,
  \qquad u(x,t) = {\cal F}^{-1}[\hat{u}] = \sum_{k\in{\mathbb Z}} \hat{u}_k e^{ikx/\tildeL}
\eeq
 is given by
\beq
  \dot{\hat{u}}_k = [(k/\tildeL)^2-(k/\tildeL)^4]\hat{u}_k -
  \frac{ik}{2\tildeL}{\cal F}[({\cal F}^{-1}[\hat{u}])^2]_k\,.
\eeq
Since $u$ is real, the Fourier modes are related by $\hat{u}_{-k} =
\hat{u}^\ast_k$.

The above system is truncated as follows: The Fourier transform
${\cal F}$ is replaced by its discrete equivalent
\beq
  a_k = {\cal F}_N[u]_k = \sum_{n = 0}^{N-1} u(x_n)
  e^{-ikx_n/\tildeL}\,,\qquad u(x_n) = {\cal F}_N^{-1}[a]_n
  = \frac{1}{N}\sum_{k = 0}^{N-1} a_k e^{ikx_n/\tildeL}\,,
\eeq
where $x_n = 2\pi\tildeL/N$ and $a_{N-k} = a^\ast_k$.  Since $a_0
= 0$ due to galilean invariance and setting $a_{N/2} = 0$ (assuming
$N$ is even), the number of independent variables in the truncated
system is $N-2$.  The truncated system looks as follows:
\beq
  \dot{a}_k = \pVeloc_k(a) = [(k/\tildeL)^2-(k/\tildeL)^4]a_k -
  \frac{ik}{2\tildeL}{\cal F}_N[({\cal F}_N^{-1}[a])^2]_k\,.
\ee{eq:KS}
where $k = 1,\ldots,N/2-1$.
%, although in the Fourier transform we need
%to use $a_k$ over the full range of $k$ values from 0 to $N-1$.
The discrete Fourier transform ${\cal F}_N$ can be computed by FFT.
In Fortran and C, the routine {\tt REALFT} from Numerical Recipes
can be used.  
% In Matlab, it is more convenient to use complex
% variables for $a_k$.  Note that Matlab function {\tt fft} is, in
% fact, the inverse Fourier Transform.

In order to find the Jacobian of the solution, or compute
Lyapunov exponents of the \KSe , one needs to solve the equation 
for a displacement vector $\delta$ in the tangent space:
\beq
  \dot{\delta} = \frac{\partial \pVeloc(a)}{\partial a} \delta\,.
\eeq
Since ${\cal F}_N$ is a linear operator, it is easy to show that
\beq
  \dot{\delta_k} = [(k/\tildeL)^2-(k/\tildeL)^4]\delta_k -
  \frac{ik}{\tildeL}{\cal F}_N[{\cal F}_N^{-1}[a]\cdot
  {\cal F}_N^{-1}[\delta]]\,,
\ee{eq:KStan}
where the dot indicates componentwise product of two vectors.  This
equation needs to be solved simultaneously with \refeq{eq:KS}.

%% To derive the equation for the matrix of variations, I use the fact
%% that ${\cal F}_N$ is a linear operator.  Since we need to
%% differentiate separately with respect to the real and imaginary
%% components of $a_k$, I use the notation $a_k = b_{2k-1} + ib_{2k}$.
%% \beq
%%   \frac{\partial \dot{a}_k}{\partial b_j} =
%%   [(k/\tildeL)^2-(k/\tildeL)^4]\delta_{kj} -
%%   \frac{ik}{\tildeL}{\cal F}_N[{\cal F}_N^{-1}[a]\cdot{\cal
%%   F}_N^{-1}[\delta_{kj}]]\,,\quad j = 1,\ldots,N-2
%% \eeq
%% where the dot indicates componentwise product, and the inverse
%% Fourier transform is applied separately to each column of
%% $\delta_{kj}$. Here, $\delta_{kj}$ is not a standard Kronecker
%% delta, but a complex valued $N\times N-2$ matrix:
%% \beq
%%   \delta_{kj} = \left(
%%   \begin{array}{ccccc}
%%   0 &  0 & 0 &  0 &\cdots\\
%%   1 &  i & 0 &  0 &\cdots\\
%%   0 &  0 & 1 &  i &\cdots\\
%% \multicolumn{5}{c}\dotfill \\
%%   0 &  0 & 0 &  0 &\cdots\\
%% \multicolumn{5}{c}\dotfill \\
%%   0 &  0 & 1 & -i &\cdots\\
%%   1 & -i & 0 &  0 &\cdots\\
%%   \end{array}  \right),
%% \eeq
%% with index $k$ running from 0 to $N-1$.  I admit, the notations are
%% a bit stretched here, but I find them convenient when coding this
%% equation using FFT.

\section{Finding \rpo s using multiple shooting and
         Levenberg--Marquardt algorithm}
\label{sec:lmderRLD}

To find periodic and \rpo s of the \KSe , we use multiple shooting and
the Levenberg--Marquardt algorithm implemented in {\tt lmder} from
the MINPACK software package or function {\tt fsolve} in Matlab.

Note that the LM algorithm is able to solve underdetermined systems of
equations.  Therefore, there is no need to augment the system with
additional equations.  For example, since Lopez \etal\ used 
{\tt lmder} to find \rpo s in CGLE, they did not need to augment 
their system with three additional equations.  In fact, we have found
that, with additional equations, the solver takes many more steps to
converge from the same seed, or fails to converge at all.
Even thought both {\tt lmder} and {\tt fsolve} solvers 
require that the number of variables does not exceed the number 
of equations, the additional equations can be set identically to 
zero. \RLD{This trick was discovered by my PhD student Jonathan 
Crofts on 30 April 2007.}

We need to solve the system of $N-2$ equations
\beq
  {\bf g}(d)f^T(a) - a = 0\,, 
\ee{eq:system}
with $N$ unknowns $(a, d, T)$, where $f^t$ is the flow map of the 
\KSe , i.e. $f^t(a) = a(t)$ is the solution of 
$\dot{a} = \pVeloc(a)$ with initial condition $a(0) = a$.

We have tried two different implementations of the multiple shooting.
The emphasis was on the simplicity of the implementations, so, even 
though both implementations worked equally well, each of them had 
its own minor drawbacks.  It is not difficult to design more 
sophisticated implementations where these drawbacks would be 
eliminated.

In the first implementation, we fix the total number of steps within 
each shooting stage and change the numerical integrator step size $h$
in order to adjust the total integration time to a desired value $T$.

Let $(a_0, d_0, T_0)$ be the starting guess for a \rpo\ obtained 
through a close return within a chaotic attractor.  We require
that the initial step does not exceed $h_0$, so we set
the number of integration steps to $n = \lceil T_0/h_0\rceil$.

The integration step size is equal to $h = T/n$. With the number of 
shooting stages equal to $m$, the system in \refeq{eq:system} is 
rewritten as follows
\bea
 F^{(1)}&\!=\!& f^\tau(a^{(1)}) - a^{(2)} = 0\,,\nonumber\\
 F^{(2)}&\!=\!& f^\tau(a^{(2)}) - a^{(3)} = 0\,,\nonumber\\
 && \cdots \\
 F^{(m-1)}&\!=\!& f^\tau(a^{(m-1)}) - a^{(m)} = 0\,,\nonumber\\
 F^{(m)}&\!=\!& {\bf g}(d)f^{\tau'}(a^{(m)}) - a^{(1)} = 0\,,\nonumber
\label{eq:MultShoot} \eea
where $\tau = \lfloor n/m \rfloor h$, $\tau' = nh - (m-1)\tau$, and
$a^{(j)} = f^{(j-1)\tau}(a)$, $j = 1, \ldots , m$.

With the Jacobian of this system given by 
\beq
  J = \left(\begin{array}{ccc}\!\!
   \displaystyle \frac{\partial F^{(j)}}{\partial a^{(k)}} & 
   \displaystyle \frac{\partial F^{(j)}}{\partial T} &
   \displaystyle \frac{\partial F^{(j)}}{\partial d}\!\!
  \end{array}\right),\quad j,k = 1,\ldots,m\,,
\eeq
the partial derivatives with respect to $a^{(k)}$ can be calculated 
using the solution of \refeq{eq:KStan}.  The partial derivatives
with respect to $T$ are given by 
\beq
  \frac{\partial F^{(j)}}{\partial T} = 
  \left\{\begin{array}{ll}
    \frac{\partial f^\tau(a^{(j)})}{\partial \tau}
    \frac{\partial \tau}{\partial T} = v(f^\tau(a^{(j)}))
    \lfloor n/m \rfloor/n\,, & j = 1,\ldots, m-1\\[.5ex]
    {\bf g}(d) v(f^{\tau'}(a^{(j)}))
    (1 - \frac{m-1}{n} \lfloor n/m \rfloor ), & j = m\,.
  \end{array}\right.
\eeq 
Note that, even though $\partial f^t(a) /\partial t = v(f^t(a))$, 
it should not be evaluated using equation for the vector field.  
The reason is that, since the flow $f^t$ is approximated by a 
numerical solution, the derivative of the numerical solution with
respect to the step size $h$ may differ from the vector field $v$, 
especially for larger step sizes.  We evaluate the derivative by
a forward difference using numerical integration with step sizes
$h$ and $h + \delta$:
\beq
  \frac{\partial f^{jh}(a)}{\partial t} = \frac{1}{j\delta}
  \left[f^{j(h+\delta)}(a) - f^{jh}(a)\right],\quad j \in 
  {\mathbb Z}^{+}
\eeq
with $t = jh$ and $\delta = 10^{-7}$ for double precision calculations.
Partial derivatives $\partial F^{(j)}/\partial d$ are all equal to
zero except for $j = m$, where it is given by
\beq
  \frac{\partial F^{(m)}}{\partial d} = \frac{d{\bf g}}{dd}
  f^{\tau'}(a^{(m)})\,.
\eeq

This Jacobian is supplied to the routine {\tt lmder} or {\tt fsolve}
augmented with two rows of zeros corresponding to the two identical
zeros augmenting \refeq{eq:MultShoot} in order to make the number of
equations formally equal to the number of variables, 
as discussed above.
\RLD{The drawback of this scheme is that the \KSe\ is integrated with 
different step sizes, so different \rpo s cannot be readily compared,
since they have been approximated using different step sizes.}

In the second implementation, we keep $h$ and $\tau$ fixed and vary
only $\tau' = T - (m-1)\tau$.  In this case, we need to be able
to determine the numerical solution of \KSe\ not only at times 
$t_j = jh, j = 1, 2, \ldots$, but at any intermediate time as well.
We do this by a cubic polynomial interpolation through points
$f^{t_j}(a)$ and $f^{t_{j+1}}(a)$ with slopes $v(f^{t_j}(a))$ and
$v(f^{t_{j+1}}(a))$.  The difference from the first implementation
is in that partial derivatives $\partial F^{(j)}/\partial T$ are
zero for all $j = 1,\ldots,m-1$, except for
\beq
  \frac{\partial F^{(m)}}{\partial T} = 
  {\bf g}(d)v(f^{\tau'}(a^{(m)}))\,.
\eeq
which, for consistency, needs to be evaluated from the cubic 
polynomial, not from the flow equation evaluated 
at $f^{\tau'}(a^{(m)})$.
\RLD{The drawback of this implementation is that the numerical 
solution of the flow is only once differentiable, so the solvers 
{\tt lmder} and {\tt fsolve} tend to sometimes slow the convergence 
down when the number of steps for integrating to $\tau'$ changes.
But the effect is rather marginal.}

We found the second implementation more convenient.

For detecting periodic and \rpo s of the \KSe\ with $L = 22$, we used
$N = 32$, $h = 0.25$ (or $h_0 = 0.25$ within the first implementation),
and the number of shooting stages such that $\tau \approx 6.0$.
Once an \rpo\ is found, its existence in the \KS\ PDE is verified 
and numerical approximation improved by increasing the number of
Fourier modes ($N = 64$) and reducing the step size ($h = 0.1$).  
