% energy.tex
% $Author$ $Date$


% \section{Representation in terms of symmetry invariant moments} % of the \KSe}
% \label{sec:energy}
% Predrag                   Apr 12 2007

In physical settings where the observation times are much longer
than the dynamical `turnover' and Lyapunov times (statistical mechanics,
quantum physics, turbulence)
periodic orbit theory\rf{DasBuch}
provides highly accurate predictions of measurable
long-time averages such as the turbulent drag\rf{GHCW07}.
Physical predictions have to be independent of a
particular choice of ODE representation of
the PDE under consideration
and, most importantly,
invariant under all symmetries of the dynamics.
In this section we discuss
a set of such physical observables for
the  1$D$ KS invariant under reflections and translations.
They offer a representation of
dynamics in which the symmetries are explicitly quotiented out.
We illustrate  this in  \refsect{sec:energyL22} by projecting
a set of explicit solutions on these coordinates.

The {space average} of a function $\obser = \obser(\pSpace,t)$  on
the interval $L$,
\beq
    \expct{\obser} = \Lint{\pSpace}\, \obser(\pSpace,t)
    % \expct{\obser} = \frac{1}{L}\int_0^{L} d\pSpace\, \obser(\pSpace,t)
    \,,
    \label{rpo:spac_ave}
\eeq
is in general time dependent.
Its mean value is given by the {time average}
\beq
\timeAver{\obser}
    =
\lim_{t\rightarrow \infty} \frac{1}{t} \int_0^t \! d\tau \, \expct{\obser}
    =
\lim_{t\rightarrow \infty} \frac{1}{t} \int_0^t \!
    \Lint{\tau}  d\pSpace\, \obser(\pSpace,\tau)
    \,.
\label{rpo:tim_ave}
\eeq
The mean value
% $\timeAver{\obser}$,
of $\obser = \obser(u_\stagn) \RLDedit{\equiv \obser_\stagn}$ evaluated on
$q$ \eqv\ or {\reqv} $u(\pSpace,t) = u_\stagn(\pSpace-ct)$ is
\beq
%         \obser_\stagn = \timeAver{\obser}_\stagn = \expct{\obser}_\stagn
\RLDedit{\timeAver{\obser}_\stagn = \expct{\obser}_\stagn = \obser_\stagn }
\,.
\label{rpo:u-eqv}
\eeq
Evaluation of the infinite time average \refeq{rpo:tim_ave}
on a function of a
\po\ or \rpo\ $u_p(\pSpace,t)=u_p(\pSpace,t+\period{p})$
 requires only a single $\period{p}$
traversal,
\beq
       \RLDedit{\timeAver{\obser}_p} = \frac{1}{\period{p}}
    \int_0^{\period{p}} \! d\tau \, \expct{\obser}
\,.
\label{rpo:u-cyc}
\eeq

Equation \refeq{ks} can be written as % in ``potential'' form
\beq
    u_t=- V_x
        \,,\qquad
    V(x,t)={\textstyle\frac{1}{2}}u^2+u_{x} + u_{xxx}
    \,.
\ee{ksPotent}
$u$ is related to the `flame-front height' $h(x,t)$ by
$u=h_x$, so \expctE \RLDedit{, defined in \refeq{eq:stdks},}
can be interpreted as the mean energy density.
% \refeq{ksEnergy}.
%
So, even though KS is a phenomenological
small-amplitude equation, the time-dependent quantity
\beq
    \expctE=
%\frac{1}{L}\int_0^{L} \!dx \,
\RLDedit{\Lint{\pSpace}}
V(x,t)=
%\frac{1}{L}\int_0^{L}\! dx \,
\RLDedit{\Lint{\pSpace}}
\frac{u^2}{2}
\label{ksEnergy}
\eeq
has a physical interpretation\rf{ksgreene88}
as the average `energy' density of the flame front.
This analogy to the corresponding definition of the
mean kinetic energy density for
the Navier-Stokes motivates what follows.

The energy \refeq{ksEnergy} is intrinsic to
the flow, independent of the particular ODE basis set
chosen to represent the PDE. However, as the Fourier
amplitudes are eigenvectors of the translation operator,
in the Fourier space the energy is a diagonalized
quadratic norm,
\beq
\expctE   % = (u,u)
          =  \sum_{k=-\infty}^{\infty} E_k
\,,\qquad
E_k = % a_{-k} a_k =
    {\textstyle\frac{1}{2}}|a_k|^2
\,,
\ee{EFourier}
and explicitly invariant term by term
under translations \refeq{eq:RPOcondFouri}.

Take time derivative of the energy density \refeq{ksEnergy},
substitute \refeq{ks} and integrate by parts. Total derivatives vanish
by the spatial periodicity on the $L$ domain:
\bea
   \dot{\expctE} &=&
     \expct{u_t \, u}
    % \frac{1}{L}\int_0^{L}u \, u_t\, dx
     = - \expct{\left({u^2}/{2} + u \, u_{x} + u \, u_{xxx}\right)_x u }
    \continue
    &=&
\expct{ u_x \, {u^2}/{2} + u_{x}{}^2 + u_x \, u_{xxx}}
    \,.
\label{rpo:ksErate}
\eea
The first term in \refeq{rpo:ksErate} vanishes by
integration by parts,
\(
3 \expct{ u_x \, u^2}= \expct{(u^3)_x} = 0
\,,
\) % {EnNonl0}
and integrating the third term by parts yet again
one gets\rf{ksgreene88} that the energy variation
\beq
   \dot{\expctE} = P - D
                \,,\qquad
      P =  \expct{u_{x}{}^2}
                \,,\quad
      D =  \expct{u_{xx}{}^2}
\ee{EnRate}
balances the power $P$ pumped in by anti-diffusion $u_{xx}$
against the energy dissipation rate $D$
by hyper-viscosity $u_{xxxx}$
in the KS equation \refeq{ks}.

The time averaged energy density  $\timeAver{E}$
computed on a typical orbit goes to a constant, so
the expectation values \refeq{rpo:EtimAve} of drive and dissipation
exactly balance each out:
\beq
    \timeAver{\dot{E}}  =
    \lim_{t\rightarrow \infty}
        \frac{1}{t} \int_0^t d\tau \, \dot{\expctE}
=
      \timeAver{P} - \timeAver{D}
%       \timeAver{(u_{x})^2} - \timeAver{(u_{xx})^2}
= 0
    \,.
\ee{rpo:EtimAve}
In particular, the \eqva\
and \reqva\ fall onto the diagonal in \reffig{f:drivedrag},
and so do time averages computed on \po s and \rpo s:
\beq
\timeAver{E}_p =
\frac{1}{\period{p}} \int_0^\period{p}d\tau \, E(\tau)
    \,,\qquad
\timeAver{P}_p =
\frac{1}{\period{p}} \int_0^\period{p} d\tau \, P(\tau)
    =
      \timeAver{D}_p
% \timeAver{(u_{x})^2}_p =
% \frac{1}{\period{p}} \int_0^\period{p} d\tau \, \expct{(u_{x})^2}
%     =
%       \timeAver{(u_{xx})^2}_p
    \,.
\label{poE}
\eeq
In the Fourier basis \refeq{EFourier} the conservation of energy on average
takes form
\beq
0 = \sum_{k=-\infty}^{\infty} ( (k/\tildeL)^2 - ( k/\tildeL)^4 )\,
    \timeAver{E}_k
\,,\qquad
E_k(t) =  {\textstyle\frac{1}{2}} |a_k(t)|^2
\,.
\ee{EFourier1}
The large $k$ convergence of this series is insensitive to the
system size $L$; $\timeAver{E_k}$ have to decrease much faster than
$(k/\tildeL)^{-4}$.
\PC{determine the decay rate, presumably exponential in $k$}
Deviation of $E_k$ from this bound for small $k$ determines the active modes.
    \PublicPrivate{%
        }{% switch to Private:
This may be useful to bound the number of equilibria, with
the upper bound given by zeros of a small number
of long wavelength modes.
        } %end \PublicPrivate{%
For \eqva\ the $L$-independent bound
    on $E$ is given by Michaelson\rf{Mks86}.
The best current bound\rf{GiacoOtto05,bronski-2005} on the long-time limit
of $E$
as a function of the system size $L$ scales as
$E \propto L^{3/2}$.

\PublicPrivate{%
        }{ %switch to Private:
Spatial representations of PDEs (such as the 3$D$
snapshots of velocity and vorticity fields in Navier-Stokes)
offer little insight into detailed dynamics of low-$Re$ flows.
Much more illuminating are the \statesp\ representations.
\PC{expand this into a visualization subsection: how we use
    $d$-dimensional vectors (stability eigenvectors, etc) to project
    from $d$-dimensions to 2 or 3 dimensions. \underline{Not}
    Fourier modes as coordinates!}
        } %end \PublicPrivate{%
