\ifsvnmulti
\svnkwsave{$RepoFile: siminos/chao/ChaosBook.tex $}
\svnidlong {$HeadURL$}
{$LastChangedDate$}
{$LastChangedRevision$} {$LastChangedBy$}
\svnid{$Id$}
\fi


\chapter{ChaosBook.org blog}
\label{chap:ChaosBook}

\begin{description}

\item[2011-08-24 PC to Chao]
I have created this chapter to collect all your ChaosBook.org
notes in one place.

%   Professor Cvitanovic, I still haven't found out why this blog cannot
%   display bibliography properly on my computer. But I think I will first
%   concentrate on those papers and come to deal with this later.

\item[2011-03-21 PC] You have the dasbuch/book/chapter/*.tex source
code, you can just clip and paste formulas to here.

\item[2011-09-12 PC] I never refer to a chapter by it's current number,
as chapter numbers change from edition to edition - latter on (years
hence) trying to figure out what ``Chapter 17'' is can be quite
confusing. Internally, each chapter is kept track off by its file name,
for example, in this blog ``smale'' refers to  \refchap{c-smale} {\em
Stretch, fold, prune}.

\end{description}

\section{ChaosBook.org broken links}
\label{c-brokenLinks}

Predrag should fix these (mark when fixed):
\begin{itemize}
  \item[[~]] 2011-09-27
\HREF{http://www.cns.gatech.edu/~predrag/papers/preprints.html\#KS}
{Vachtang's term paper} link to
\\
www.cns.gatech.edu/$\sim$predrag/papers/vachtang.ps.gz
  \item[[~]] 2011-09-27
\HREF{http://www.cns.gatech.edu/~predrag/papers/preprints.html\#KS}
{Vachtang Putkaradze PhD thesis} link to
\\
www.nbi.dk/$\sim$putkarad
  \item[[~]]
  \item[[~]]
\end{itemize}


%%-----   Flows
\section{Chapter: Go with the flow}
\label{c-flows}\noindent dasbuch/book/chapter/flows.tex

\begin{description}

\item[2011-09-11 CS]
Have been preparing for the study group tomorrow to cover this chapter. I
ran into some conceptual details need to be clarified. I will present
them to the group, and add to Chaosbooks.tex later on if necessary.



\end{description}


%%-----   Maps
\section{Chapter: Discrete time dynamics}
\label{c-maps}\noindent dasbuch/book/chapter/maps.tex
\begin{description}

\item[2011-03-19 CS to PC]
I have a question about Example 3.2. It states that
   the Birkhoff choice of the coordinates of the pinball game is smart because
   they conserve the phase space volume. I do not understand this, would
   you mind explain it more specifically?


\end{description}

%%-----   Stability
\section{Chapter: Local stability}
\label{c-stability}\noindent dasbuch/book/chapter/stability.tex
\begin{description}

\item[2011-03-19 CS to PC]
%   Hi Professor Cvitanovic, I manage to make Latex work on my laptop
%   without actually knowing what actually happened, it works anyway. As I
%   tried one bibliography in last conversation.
Yesterday I finished the Chapter but skipped the detail of the some
examples. We discussed the first half of the Chapter. Since there is lot
of mathematical detail and examples, we decided to break it into two
sessions. By the way, the notation of this chapter at the beginning is a
little bit confusing and took me a while to get the concept of what those
formulas really mean. I have some other questions but I am not used to
typing math symbols here, I will ask you later directly or write it here
if I get familiar with it.

\end{description}

%%-----   Cycle stability
\section{Chapter: Cycle stability}
\label{c-invariants}\noindent dasbuch/book/chapter/invariants.tex
\begin{description}

\item[2011-04-05 CS]
    A question about the last sentence in the first paragraph of Section
    5.4  discussed in the group study last week: Why does the
    neighborhood size scale as $1/|\Lambda_{p}|$? Wouldn't it scale as
    $|\Lambda_{p}|$?

\item[2011-04-06 PC] Mhm, clearly not written clearly enough, but
perhaps the most important property of an unstable flow
that one has to understand. The product of expanding multipliers
$|\Lambda_{p}|$ blows up exponentially with time, but the
\emph{neighborhood shrinks} exponentially with time, Detroit-like.
 Does looking
at Figure 5.1 help? Does reading Sect. 1.5.1 help? If you understand it,
can you rewrite

                                                    \toCB
Nearby points aligned along the stable
(contracting) directions  remain in the neighborhood of the
trajectory $\ssp(t)= \flow{t}{\xInit}$; the ones to keep an eye
on are the points which leave the neighborhood along the
unstable directions because all nonlinear phenomena comes
from these directions. The sub-volume $ |\pS_{\xInit}| = \prod_i^e\Delta
\ssp_i$ of the set of points which get no further away from
$\flow{t}{\xInit}$ than $L$, the typical size of the system, is
fixed by the condition that $\Delta \ssp_i \ExpaEig_i = O(L)$
in each expanding direction $i$. Hence the neighborhood size
scales as
$|\pS_{\xInit}| \propto O(L^{d_e})/|\ExpaEig_p| \propto 1/|\ExpaEig_p| $
where $\ExpaEig_p$ is the
product of expanding Floquet multipliers
(5.7) %\refeq{expVol}
only;
contracting ones play a secondary role.
''

so it makes sense to you. If you and Adam do not understand it, then
bring it up for discussion in the study group.

\item[2011-04-07 CS to PC] Got it.

\item[2011-04-11 CS]
rewrote Paragraph 1 of Section 5.4 as follows:

\CSedit{
Nearby points aligned along the stable (contracting) directions  remain
in the neighborhood of the trajectory $\ssp(t)= \flow{t}{\xInit}$; the
ones to keep an eye on are the points which leave the neighborhood along
the unstable directions because almost all nonlinear and chaotic
phenomena comes from these directions. The sub-volume $ |\pS_{\xInit}| =
\prod_i^e\Delta \ssp_i$ of the set of points which get no further away
from $\flow{t}{\xInit}$ than $L$, the typical size of the system, is
fixed by the condition that $\Delta \ssp_i \ExpaEig_i = O(L)$ in each
expanding direction $i$. Hence the neighborhood size scales as
$|\pS_{\xInit}| \propto O(L^{d_e})/|\ExpaEig_p| \propto 1/|\ExpaEig_p| $
where $\ExpaEig_p$ is the product of expanding Floquet multipliers
(5.7) %\refeq{expVol}
only(see section 1.5.1 and figure 1.9 for example);
contracting ones play a secondary role}

\item[2011-04-12 PC] Thanks, I have now rewritten the introduction to the
Chapter as well as the section 5.4 is the spirit you suggest, emphasizing
the key role the concept of 'neighborhood' will play.

\end{description}


%%-----   Smooth conjugacies
\section{Chapter: Go straight}
\label{c-conjug}\noindent dasbuch/book/chapter/conjug.tex
\begin{description}

\item[2011-04-11 CS]
Finished this Chapter. Generally able to understand it but feel like
there's whole lot more content underneath that as in the KS
transformation for example. There should be a lot tricks and methods to
construct such regularization. And I wonder what kind of singularity
could be regularized. But I guess that this is not an easy question and
should not be the emphasis to my project.


\item[2011-04-11 CS]
A trivial error: eq.~(6.13) should be ``$\sqrt{x}dx = 2dt$'',
rather than ``$\sqrt{x}dx = \sqrt{2}dt$''.

\item[2011-04-12 PC]
No error is 'trivial.' Thanks.

\end{description}


%%-----   Newton
\section{Chapter: Hamiltonian dynamics}
\label{c-newton}\noindent dasbuch/book/chapter/newton.tex
\begin{description}

\item[2011-03-19 CS to PC]
I am wondering how to choose phase space
coordinates? Do phase/state space coordinates have any requirement and
whether conservation of space volume is such a requirement? What's the
meaning of conservation of phase space requirement? In my understanding
in Hamiltonian flows, conservation of phase space volume means
conservation of energy, am I right?

\item[2011-03-25 PC]
Now, it is more subtle than that; time dependent flow can be symplectic,
but energy is not conserved; I think Percival and D.
Richards\rf{PerRich82N} (I have it in the
\HREF{http://www.cns.gatech.edu/CNS-only/LibraryCat2.htm} {CNS library})
discuss that well. Symplectic invariance is \emph{much stronger}
requirement than either either energy conservation or phase-space volume
conservation, see \HREF{http://chaosbook.org/chapters/newton.pdf}
{Section 7.4 Poincar\'e invariants} and
\HREF{http://chaosbook.org/chapters/appendStability.pdf} {Appendix D.4
Stability of Hamiltonian flows}: symplectic transformations preserve area
for each $(q,p)$ dual coordinate  pair.

\end{description}


%%-----   Billiards
\section{Chapter: Billiards}
\label{c-billiards}\noindent dasbuch/book/chapter/billiards.tex
\begin{description}

\item[2011-04-12 CS] I'll present this Chap next Friday. Since a
test on Friday is waiting for me, I'll write the solution during the weekend.

\item[2011-03-25 PC]
Glad you asked the question about choice of billiard coordinates. Please
write up the solution to \refexer{ex_birkhoff}; it is worth doing it in
class as well, a concrete example of how symplectic invariance preserves
area for each $(q,p)$ dual coordinate  pair.

\item[2011-04-19 CS] Done - I have verified in \refexer{ex_birkhoff} that
the 2-form/wedge product is conserved on the \Poincare section.

\item[2011-04-19 PC] It's a bit inelegant, no? Maybe you can discuss it with
Francois and Adam and rewrite it elegantly.

\item[2011-04-12 PC] Remember to write up your solution to
\refexer{ex_birkhoff} before you forget it. If it is good, we might
rewrite the Chapter 8 {\em Billiards} before the study group takes it up.

\item[2011-04-05 CS] Worked out \refexer{ex_birkhoff}. Finished Chapter 9
with all the details in the examples, have had exemplified pictures of
symmetry and group actions.

\item[2011-04-06 PC] Not so fast - \refexer{ex_birkhoff} is not finished
until you write down your solution.

\item[2011-04-20 PC]  I think that in \refexer{ex_birkhoff} you want to
have different radii $a_i$ for the two disks. If you do that, you have
the general map for a billiard with a smooth boundary, as you only need the
local radius of curvature. Also, while showing that
the 2-form/wedge product is conserved on the \Poincare section might make
Adam and friends happy, what you really need is the explicit $[2\!\times\!2]$
Jacobian matrix, because you will need to compute its eigenvalues - I do not see
how you do that from the wedge product alone. Once you have the \jacobianM,
the area preservation is immediate, as you will show that $|\det \jMP |= 1$.
% {\jMP}{\ensuremath{\hat{J}}}   % jacobian matrix, Poincare return

\item[2011-4-22 CS to PC 06:32am]  Generalized the proof in the exercise
to different radius and wrote out the Jacobian matrix. But what's
interesting as you can below is that as long as my previous proof when
the two circles are identical is correct, there's no way that the
Jacobian in the case of unequal radius is one. You can easily see that
the map: $(\theta_1, \sin(\phi_1))\Rightarrow(\theta_2, \sin(\phi_2))$ is
still volume-preserving in same way. Plus the fact that $s=a\theta$,
there must be a factor of ratio of two radius comes in. You can visualize
this effect when you expand or compress the circle in the righthand side.
So I guess Birkhoff coordinates $(s,p)$ just preserve phase space volume
when the radius are equal. Right?

\item[2011-4-22 PC] No.

\item[2011-4-22 CS]
You're right. I checked the proof again and found out that in the map of
$p$, I missed a factor $\frac{a_1}{a_2}$ in front of $p_1$ and other two
$\frac{1}{a_2}$ factors before the remaining two terms. Now the result is
happy. It seems that caffeine cannot perfectly replace sleep.

\item[2011-04-22 PC]
I'm feeling better, too. I'll try to unconvolute your derivation (your
`sine laws' are formulas for the impact parameters in my hand-drawn
billiard maps), but you will feel even more alert if you can
verify (or reduce your formulas to) the ChaosBook formulas
\refeq{hor}, \refeq{eq_her} and \refeq{eq_her}.


\end{description}


%%-----   Discrete symmetries
\section{Chapter: World in a mirror}
\label{c-discrete}\noindent dasbuch/book/chapter/discrete.tex
\begin{description}

\item[2011-09-11 PC to  CS] As you learn this material, keep editing
\refchap{chap:symms} and moving your notes forom here into the exposition there.
You will need this material both for the thesis and the publications.

\item[2011-03-30  CS to PC]
I have a question from the paragraph
following the definition of free action: The splitting of a group
\emph{G} into a stabilizer \emph{$G_{p}$} and \emph{m-1} coset
\emph{$cG_{p}$} relates to an orbit \emph{$M_{p}$} to \emph{m-1} other
distinct orbits \emph{$cM_{p}$}. All of them have equivalent stabilizers,
or more precisely, the points on the same group orbit have
\emph{conjugate stabilizers}: \emph{$G_{cp} = cG_{p}c^{-1}$}. For the
last sentence, does it mean that if \emph{$G_{p}$}  is a stabilizer of
\emph{$M_{p}$}, then \emph{$cG_{p}c^{-1}$} is a stabilizer of
\emph{$cM_{p}$}?

\item[2011-03-31 PC] Yes, you are right. I have now incorporated ``if
\emph{$G_{p}$}  is a stabilizer of \emph{$M_{p}$}, then
\emph{$cG_{p}c^{-1}$} is a stabilizer of \emph{$cM_{p}$}'' into
discrete.tex, thanks.

I intend to excise the dreaded word `stabilizer' from the text, just have
forgotten to do it
\HREF{http://www.flickr.com/photos/birdtracks/4259634492/in/set-72157606259014811/}
{[click]} here. Suggestion - print out the chapter, replace by hand word
`stabilizer' everywhere by 'symmetry' and let's sit together and see
whether the chapter is  easier to read.

\noindent
[ ] {\bf 2011-08-30 Predrag} mark this box once you have entered the
edits and discussed them with me.

\item[2011-04-19 CS]
BTW, any recommendation for a hands on book for Lie group and Lie
algebra? I think I need a deeper knowledge about this.

\item[2011-04-20 PC] I suggested in the siminos/blog/ (please keep reading it)
that you check out a book from the library that Meiss recommends. Do it, see
whether it eases the pain. You can also get
Gilmore and Letellier\rf{GL-Gil07b}
{\em The Symmetry of Chaos} out of the CNS library. It only does the
invariant polynomial reduction (Siminos and I believe that is useless in
higher dimensions), but it is pretty good on discrete symmetries.

\end{description}


%%-----   Continuous symmetries
\section{Chapter: Relativity for cyclists}
\label{c-continuous}
\noindent dasbuch/book/chapter/continuous.tex
\begin{description}
\item[2011-04-19 CS]
I am stuck at trying to understand eq.~(10.16):
\[
\oint\frac{d\theta}{2\pi}(\textbf{T}u(\theta))^{\emph{T}}\textbf{T}u(2\pi-\theta)
 = \sum\limits_{m=0}^{\infty}m^{2}(u_{1}^{(m)2}+u_{2}^{(m)2})
\]
First, why would there be a integral on left hand side? Second, why the
integrand is $(\textbf{T}u(\theta))^{\emph{T}}\textbf{T}u(2\pi-\theta)$?
Shouldn't it be $(\textbf{T}u(\theta))^{\emph{T}}\textbf{T}u(\theta)$
according to eq. (10.11)?

\item[2011-04-20 PC] $u(\theta)$ is a function on the interval $[0,2\pi]$,
hence integral on the left side (LHS). Compact support, hence the infinite sum
on the RHS. If I remember right (check notes or the textbook from
our Math Methods course) for Fourier transforms, a convolution on the
left hand side gives me a product on the right hand side. If I'm wrong,
let me know so I fix this; as you say, does not look like $L^2$ norm...


\end{description}


%%-----   Qualitative dynamics, pedestrian
\section{Chapter: Charting the state space}
\label{c-knead}\noindent dasbuch/book/chapter/knead.tex
\begin{description}\item[2011-09-?? CS]

\end{description}


%%-----   Qualitative dynamics, for cylists
\section{Chapter: Stretch, fold, prune}
\label{c-smale}\noindent dasbuch/book/chapter/smale.tex
\begin{description}

\item[2011-07-19 CS]
I have some questions about Chap.12 "Stretch, fold, prune" listed below.

In my understanding, in a non-wandering and chaotic set, any unstable
trajectory is ergodic. That means the trajectory will finally visit any
infinitesimally small volume in the set. Right?

\item[2011-07-20 PC]                                        \toCB
Please reread chapter ``Flows,'' alert me if the question is not answered
there (`metrically transitive,, etc...). It is trickier than
`infinitesimally small volume,' as the strange attractors tend to be
fractal.

\item[2011-07-19 CS]
I do not quite understand the last sentence of the second last paragraph
before 12.1.1 when you say ``the trick is to stop continuing an invariant
manifold while the going is still good.'' By `the going is still good,' do
you mean that we should stop when we find that the trajectory will be
attracted to an equilibrium and never get out of a certain true subset of
the whole non-wandering set?

\item[2011-07-19 CS]
About Example 12.3 H\'enon repeller complete horseshoe:
First, in figure 12.4, why after one step's evolution, point B occupies
point D's original spot and also point C and point D are in the stable
manifold? I do not think this is just a coincidence but still haven't
figure out the reason.

\item[2011-07-20 PC]                                        \toCB
Concerning the H\'enon attractor \underline{not} being symmetric across
the diagonal in general: check my
\HREF{http://chaosbook.org/version13/Maribor11.shtml}{Maribor lectures}.
In \HREF{http://chaosbook.org/overheads/dimension/dimension.pdf}{piece
\#5}: ``Dynamics in infinitely many dimensions'' slide 10 shows the
stable / unstable manifolds for the canonical H\'enon attractor - clearly
very asymmetric.

\item[2011-07-19 CS]
Secondly, in 1-dimensional symbolic dynamics, the itinerary $\gamma$ has a
simple relation with space coordinate x that $if \gamma(x_1) <
\gamma(x_2), x_1<x_2$. Does this hold in 2-dimensional Smale Horseshoe
for just one component of the symbol plane)? I ask this because I think
this is related to why do you assign the left and lower part to be 0, and
the right and higher part to be 1 (figure 12.4). Also, what if after the
first step of evolution, the lower part and the higher part is still
connected, not completely cut off by the boundary?(Is this just a special
case of incomplete Smale Horseshoe?)

I saw the spirit in symbolic dynamics approximating periodic orbits by
infinite binary sequence as we approximating an irrational number by
infinite sequence of rational number. But I still haven't see the
connection between symbolic dynamics and my work. Can you give some
instructions?

Lastly, I'm not very clear about the last three paragraphs of section
12.1.1?

\item[2011-07-20 PC] I have answered the above questions on the board,
and asked Chao to enter here answers as he understood them; we'll see
whether further discussion is needed after that.

\item[2011-07-22 CS] I read the Smale Horseshoe part in Kai's thesis. Now
I understand why the binary sequence has to be assigned this way. Only in
this way can the symbol represent the actual map. It's hard to describe
it but this picture found from wiki helped me a lot.

%%%%%%%%%%%%%%%%%%%%%%%%%%%%%%%%%%%%%%%%%%%%%%%%%%%%%%%%%%%%%%%%%%
\SFIG{SmaleHorseshoe} %{HMV}
{}{
This is a Smale horseshoe that I snitched from an unnamed wiki
without attribution. Cute, no?
    }{Fig:SmaleHorseshoe}
%%%%%%%%%%%%%%%%%%%%%%%%%%%%%%%%%%%%%%%%%%%%%%%%%%%%%%%%%%%%%%%%%

In Kai's thesis, he mentioned ``since the horseshoe is a diffeomorphism
we need to know both the future and the past.'' With the help of this
picture I kind of understand this.

Also I found something interesting that animate the procedure of H\'enon
map horseshoe:
\HREF{http://www.ibiblio.org/e-notes/Chaos/henon.htm}
{www.ibiblio.org/e-notes/Chaos/henon.htm}

\item[2011-07-23 PC]                                        \inCB
I had noticed Demidov's website before - you are right, these simulations
are very instructive, I have now added a remark about them to ChaosBook.
He uses a different definition for parameters $a$ and $b$ from H\'enon,
but unfortunately uses the same letters. His definition is natural if one
is interested in Julia sets, but unfortunately not the one H\'enon used,
and I always try to follow the foundational papers, rather than confusing
everybody with sly parameter redefinitions.

\item[2011-07-24 CS and PC]                                        \toCB
H\'enon's parametrization\rf{henon}:
\index{Henon@H\'enon map}
\index{map!H\'enon}
\bea
    x_{n+1}&=&1-ax^2_n+b y_n
        \continue
    y_{n+1}&=& x_n
\,.
\label{eq2.1a}
\eea
Demidov's parametrization\rf{DemChaos} of the H\'enon map is:
\bea
    x_{n+1}' &=& a'+ {x'}{}^2_n + b' y_n'
        \continue
    y_{n+1}' &=& x_n'
\,.
\label{DemidHen}
\eea
Dividing through by $a'$ we get
\(
\frac{x_{n+1}'}{a'} = 1 + a'\left(\frac{x_n'}{a'}\right)^2 + b'\frac{y_n'}{a'}
\,,
\)
    \CS{my formula was
\[
\frac{x''}{a'}=1+\frac{1}{a'}x^2+\frac{b'}{a'}y
\]
    }
so the two parametrizations are related by:
    \CS{my guess was
\[ %beq
x={x'}/{a'}
\,,\qquad
a=-{1}/{a'}
\,,\qquad b= {b'}/{a'}
\] %ee{DemidHenChao}
    }
\beq
x={x'}/{a'}
\,,\quad
y={y'}/{a'}
\,;\qquad
a=-{a'}
\,,\quad b= {b'}
\,.
\ee{DemidHenPar}

%%%%%%%%%%%%%%%%%%%%%%%%%%%%%%%%%%%%%%%%%%%%%%%%%%%%%%%%%%%%%%%%%%
\SFIG{Demidov_a-6_b-1}
{}{
PC: The Smale backward-forward horseshoe generated by the
Demidov\rf{DemChaos} java applets for the H\'enon parameter values
$(a,b) = (6,-1)$.
    }{Fig:Demidov}
%%%%%%%%%%%%%%%%%%%%%%%%%%%%%%%%%%%%%%%%%%%%%%%%%%%%%%%%%%%%%%%%%

\item[2011-07-24 PC]
You need the transformation between two definitions, if you are
going to use Demidov's simulations to test your ideas, and it helps greatly
if the transformation formula is the correct one. Please recheck
whether I am right in correcting your argument, and whether if you chose
\(
a'=-6
\,,\quad
b'= -1
\)
Demidov's java applets reproduce the figures in ChaosBook?

\item[2011-07-22 CS] BTW, I forgot to ask the first question on the list
yesterday "In my understanding, in a non-wandering and chaotic set, any
unstable trajectory is ergodic. That means the trajectory will finally
visit any infinitesimally small volume in the the set. Right?"

\item[2011-07-23 PC]
Mhm... I wrote a suggestion what to read above already, dated {\bf
2011-07-20 PC}. After rereading the discussion in ChaosBook, let me know
what is missing there...

\end{description}


%%-----   Finding fixed points
\section{Chapter: Fixed points, and how to get them}
\label{c-cycles}\noindent dasbuch/book/chapter/cycles.tex
\begin{description}\item[2011-09-?? CS]

\end{description}


%%-----   Walk about: Markov graphs
\section{Chapter: Walkabout: Transition graphs}
\label{c-Markov}\noindent dasbuch/book/chapter/Markov.tex
\begin{description}\item[2011-09-?? CS]

\end{description}

%%-----   Counting
\section{Chapter: Counting}
\label{c-count}\noindent dasbuch/book/chapter/count.tex
\begin{description}\item[2011-09-?? CS]

\end{description}

%%-----   Transporting densities
\section{Chapter: Transporting densities}
\label{c-measure}\noindent dasbuch/book/chapter/measure.tex
\begin{description}\item[2011-09-?? CS]

\end{description}

%%-----   Averaging
\section{Chapter: Averaging}
\label{c-average}\noindent dasbuch/book/chapter/average.tex
\begin{description}\item[2011-09-?? CS]

\end{description}

%%-----   Trace formulas
\section{Chapter: Trace formulas}
\label{c-trace}\noindent dasbuch/book/chapter/trace.tex
\begin{description}\item[2011-09-?? CS]

\end{description}


%%-----   Spectral determinants
\section{Chapter: Spectral determinants}
\label{c-det}\noindent dasbuch/book/chapter/det.tex
\begin{description}\item[2011-09-?? CS]

\end{description}

%%-----   Cycle expansions
\section{Chapter: Cycle expansions}
\label{c-recycle}\noindent dasbuch/book/chapter/recycle.tex
\begin{description}\item[2011-09-?? CS]

\end{description}

%%-----   Discrete symmetries
\section{Chapter: Discrete factorization}
\label{c-symm}\noindent dasbuch/book/chapter/symm.tex
\begin{description}\item[2011-09-?? CS]

\end{description}

%%%-----   Why cycle?
%\section{Chapter: }\label{c-flows}\noindent dasbuch/book/chapter/getused.tex
%
%
%%%-----   Why does it work?
%\section{Chapter: }\label{c-flows}\noindent dasbuch/book/chapter/converg.tex
%
%
%%%-----   Intermittency
%\section{Chapter: }\label{c-flows}\noindent dasbuch/book/chapter/inter.tex
%
%
%%%-----   Relativity for cyclists
%\section{Chapter: }\label{c-flows}\noindent dasbuch/book/chapter/rpo.tex
%
%
%%%-----   Diffusion confusion
%\section{Chapter: }\label{c-flows}\noindent dasbuch/book/chapter/diffusion.tex


%%-----   PDEs
\section{Chapter: Turbulence?}
\label{c-PDEs}\noindent dasbuch/book/chapter/PDEs.tex 30aug2011
\begin{description}


\item[2011-08-24 PC to Chao] Please read ChaosBook.org chapter
\HREF{http://ChaosBook.org/paper.shtml\#PDEs}{Turbulence?}, very
critically? It explains the relation between $L$ and the hyperviscosity
$\nu$, and it should be the fastest introduction to \KSe, for our
purposes, so blog what is unclear and what is missing.

\item[2011-08-28 Chao]

In the second paragraph of section 26.1.1, it is said ``KS equation is
Galilean invariant: if $u(x,t)$ is a solution, then $v+u(x+2vt,t)$ with
$v$ an arbitrary constant velocity is also a solution.'' So I substitute
$v+u(x+2vt,t)$ into KS equation given by 26.2, but it seems that the new
equation does not keep the original form with additional term $3vu_x$.
Would you please show me explicitly how this works?

\item[2011-08-30 Predrag] thanks for catching this typo, it should be
$v+u(x-vt,t)$. Corrected now.

\item[2011-08-28 Chao]
I do not understand the dimensional analysis in the following
paragraph. Why does time has the dimension of square of length? Why does
viscosity has the same dimension with time? Plus, where does
``viscosity'' show up in the original KS equation 26.2?

\item[2011-08-30 Predrag] thanks for catching this typo (I had not
entered the hyper-viscosity parameter $\nu$ in the defining equation).
As to the rest, we have talked about it and understood it verbally:

\noindent
 [ ] {\bf 2011-08-30 Predrag} mark this box once you have entered the
answers into the blog.

\item[2011-08-28 Chao]
In the second paragraph of page 516, we obtain two equilibrium points and
each of them has totally three dimensional manifolds and three Floquet
multipliers. Since KS equation is one-dimensional, where are the two
other dimensions from? I saw that you put the two equilibrium points in
three-dimensional space and extend the coordinate to be $c_+ =
(\sqrt{c},0,0),c_- = (-\sqrt{c},0,0)$, but this does not bring in the
dependence on the other two dimensions, right?

\noindent
[ ] {\bf 2011-08-30 Predrag} mark this box once you have entered the
answers into the blog.

\item[2011-09-10 PC]
\KSe\ is a PDE in one or two spatial dimensions, so it corresponds to an
\emph{infinity} of ODEs. \Eqv\ condition for 1 spatial dimension KS sets
the left-hand of the equation to zero, so the right hand side (in the 1
spatial dimension case \emph{only}) becomes a 4th order ODE in $d/dx$.
One integral you can do, so the system becomes a 3rd order ODE +
integration constant, or 3 first order ODEs. The best papers on this are

\noindent
[ ] Michelson\rf{Mks86}

\noindent
[ ] Lan and Cvitanovi{\'c}\rf{lanCvit07}, discussed here in
\refsect{s:lanCvit07}

\noindent
[ ] Lan\rf{LanThesis} thesis

\noindent
Chao, mark above boxes once you have read the above sources and entered
material learned from them either here, or in separate sections, one for
each source.

\item[2011-08-28 Chao]
Up to this point I am confused with another point that in Chapter 4, when
defining Floquet multipliers and stable/unstable manifold, we
introduced Jacobian matrix of continuous time-dependent map $f(x,t)$. We
expand the map f to first order and get Jacobian matrix as the
coefficient matrix. But here in KS system, we do not know the explicit
form of $f(x,t)$, which should be the solution. Then how do we deduce the
Jacobian matrix without knowing the map explicitly?

\item[2011-09-10 PC] You almost never have an explicit formula for the
\JacobianM, it is always a numerically computed linearized map. What you
do have is an explicit formula for the {\stabmat} ${\Mvar}$, in the case
of KS you have written it out in \refeq{KSstabMat2}.

\end{description}

%%-----  "Semiclassics" for noise
\section{Chapter: Noise}
\label{c-noise}\noindent dasbuch/book/chapter/noise.tex
\begin{description}\item[2011-09-?? CS]

\end{description}

%%-----   Finding cycles variationally


\section{Chapter: Relaxation for cyclists}
\label{c-relax}\noindent dasbuch/book/chapter/relax.tex
\begin{description}\item[2011-09-?? CS]

\end{description}


%%%-----   Appendices
%%\appendix
%
%
%%%-----   A brief history of chaos
%\section{Chapter: }\label{c-flows}\noindent dasbuch/book/chapter/appendHist.tex
%
%
%%%-----   Maps and billiards
%\section{Chapter: }\label{c-flows}\noindent dasbuch/book/chapter/appendB.tex
%
%
%%%-----   Linear algebra, Hamiltonian Jacobians
%\section{Chapter: }\label{c-flows}\noindent dasbuch/book/chapter/appendStability.tex
%
%
%%%-----   Cycles
%\section{Chapter: }\label{c-flows}\noindent dasbuch/book/chapter/appendCycle.tex
%
%%%-----   Symbolic dynamics techniques
%\section{Chapter: }\label{c-flows}\noindent dasbuch/book/chapter/appendSymb.tex
%
%
%%%-----   Counting
%\section{Chapter: }\label{c-flows}\noindent dasbuch/book/chapter/appendCount.tex
%
%
%%%-----   Implementing evolution
%\section{Chapter: }\label{c-flows}\noindent dasbuch/book/chapter/appendMeasure.tex
%
%%%-----   Applications
%\section{Chapter: }\label{c-flows}\noindent dasbuch/book/chapter/appendApplic.tex
%
%
%%%-----   Discrete symmetries
%\section{Chapter: }\label{c-flows}\noindent dasbuch/book/chapter/appendSymm.tex
%
%
%%%-----   Coveregence of spectral determinants
%\section{Chapter: }\label{c-flows}\noindent dasbuch/book/chapter/appendConverg.tex
%
%%%-----   Stat mech
%\section{Chapter: }\label{c-flows}\noindent dasbuch/book/chapter/appendStatM.tex
%
%
%%%-----   Infinite dimensional operators
%\section{Chapter: }\label{c-flows}\noindent dasbuch/book/chapter/appendWirzba.tex
%
%
%%%-----   Statistical Mechanics
%\section{Chapter: }\label{c-flows}\noindent dasbuch/book/chapter/statmech.tex
