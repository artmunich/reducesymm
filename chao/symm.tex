\ifsvnmulti
\svnkwsave{$RepoFile: siminos/chao/symm.tex $}
\svnidlong {$HeadURL$}
{$LastChangedDate$}
{$LastChangedRevision$} {$LastChangedBy$}
\svnid{$Id$}
\fi


\chapter{The importance of being symmetric}
\label{chap:symms}

\begin{description}

\item[2011-5-14 PC to Chao]
I have copied to this chapter the 2011-07-12 symmetry notes from
repository spieker/blog/* to help you get started with your own symmetry
notes - edit this your own way as you learn this stuff.

Keep it up-to-date - if all goes well, it can be used in forthcoming
papers and
\HREF{http://www.online-literature.com/wilde/being_earnest/}
{the thesis}. Please use the same notational conventions and
macros, makes life easier later on.

Dustin Speaker was a good grad student who flunked out because he flunked
Classical Mechanics qualifiers. Go figure...

\end{description}


\section{Basics of group theory}


Suppose that a set of elements $A,B,C,...$ has a form of group
multiplication that associates a third member of the set with
two other elements of the set; a group, $\Group =
\{\LieEl_1,\LieEl_2,...,\LieEl_n\}$,
is defined as the members of that set that
satisfy four conditions.

The first condition is called \emph{closure}, namely that
multiplication between any two elements in a group produces an
element in the same group:
\[
 \LieEl_i \circ \LieEl_j \in \Group
\]
The second condition is called \emph{associativity}:
the order in which group multiplication is performed
does not matter.  If parenthesis denote the first operation to
be performed:
\[
 (\LieEl_i \cdot \LieEl_j) \circ \LieEl_k
   = \LieEl_i \circ (\LieEl_j \circ \LieEl_k)
\,.
\]
The third condition is that there must exist an \emph{identity
element} in the group, usually denoted by $e$, such that for all
$\LieEl_i \in \Group$
\[
 e \circ \LieEl_i = \LieEl_i \circ e = \LieEl_i
\,.
\]
The fourth condition is the existence of an \emph{inverse}.  For every
element in a group $g \in \Group$, there must exist an inverse
$h = g^{-1} \in \Group$ such that
\[
 g\circ h = h\circ g = e
\,,
\]
where $e$ once again represents the identity element.  As an
example, the real line with the exception of zero, $\mathbb{R}
\cap 0$, forms an order~$\infty$ group under multiplication.
   \PC{This example is probably more confusing than helpful,
       the group is continuous but not a Lie group (no generators,
       I believe). I think you can also show that is a group/not a group
       under addition.}
It satisfies all of the above
conditions.  Including $0$ in the set would fail to satisfy the
existence of an inverse condition and the set would no longer
be a group.

Group theory is much more developed than the simple properties
listed above, but with these definitions alone, we might begin
to apply our knowledge of groups toward analysis of dynamical
systems.  First we must define few terms that will come up
frequently in our considerations.  It will be useful to
consider a simple symmetric geometric object to reference
throughout the definitions.  Consider a square that is
symmetric \PCedit{under reflection} by $\pi$ about axes
$A,B,C,D$ and rotations about the center of the square by
$\frac{\pi}{2},\pi,\frac{3\pi}{2}$ called $F,G,H$ respectively.
We also include the identity symmetry $E$.  With these 8
symmetries in place, we can work out the group multiplication
\reftab{t:squareMultTab}.
    \PC{No need to fix this now in \reftab{t:squareMultTab}, but
        the multiplication tables convention is to put identity
        into the first row/column, then other elements class by
        class, in increasing class size. This will be useful later
        on, when you write down character tables which tell you
        the weight for a given class within a given irreducible
        representation.}
\begin{table}
\caption[]
{Group multiplication table for a square.
The order of group operations is multiplying the
column by the row.}
\centering
{\small
\begin{tabular}{c|llllllll}
  & E & A & B & C & D & F & G & H
\\[1ex]
\hline\\[-0.5ex]
E & E & A & B & C & D & F & G & H \\
A & A & E & G & F & H & C & B & D \\
B & B & G & E & H & F & D & A & C \\
C & C & H & F & E & G & B & D & A \\
D & D & F & H & G & E & A & C & B \\
F & F & D & C & A & B & G & H & E \\
G & G & B & A & D & C & H & E & F \\
H & H & C & D & B & A & E & F & G \\

\end{tabular}
} %end {\small
\label{t:squareMultTab} % removed \tEQsix\
\end{table}

Now we might begin our definitions.  The first term to come up
regularly is \textit{conjugacy}.  Suppose there exists a group
$\Group = \{\LieEl_1,\LieEl_2,...,\LieEl_n\}$. Element $\mathcal{Q}$ is said to be
conjugate to $P$ if $\LieEl_iP\LieEl_i^{-1} = \mathcal{Q}$ or
$\LieEl_i^{-1}Q\LieEl_i = P$ for all elements $\LieEl_i \in \Group$.
	\PC{wrong as it stands - 'all elements' defins the class,
            conjugacy works only for specific elements, as you show in
	    \refeq{eq:class}}
For
example, in the square example, both $E$ and $G$ are conjugate
to themselves (but not to each other).

Suppose that we carry out the out calculations of the form
$\LieEl_i^{-1}A\LieEl_i$ for the group of elements of our square
symmetry.  One would find
\bea
&&\{AAA^{-1},BAB^{-1},CAC^{-1},DAD^{-1},EAE^{-1},FAF^{-1},GAG^{-1},HAH^{-1}\}
\continue
&& ~~~~~~~~~~~~~~~~~~~~ = ~~\{A,A,B,A,A,B,A,B\}
\label{eq:class}
\eea
Notice that only the elements $A$
and $B$ turn up.  If we do the same calculation for $B$, we
find that
\bea
&&\{ABA^{-1},BBB^{-1},CBC^{-1},DBD^{-1},EBE^{-1},FBF^{-1},GBG^{-1},HBH^{-1}\}
\continue
&& ~~~~~~~~~~~~~~~~~~~~ = ~~\{B,B,A,B,B,A,B,A\}
\eea
Again, note that only the elements $A$ and $B$ result.  This
property of $A$ and $B$ links them in a way that is called a
\emph{class}, which is our next definition.  A class is a
set of group elements
$\mathcal{C} \in \Group$, such that for any
$\LieEl_i \subset \Group$
\[
\LieEl_i\mathcal{C}\LieEl_i^{-1} = \mathcal{C}
\,.
\]
For our square symmetry group, we find there are 5 such
classes:
$\mathcal{C}_1 = \{E\}$,
$\mathcal{C}_2 = \{G\}$,
$\mathcal{C}_3 = \{F,H\}$,
$\mathcal{C}_4 = \{A,B\}$,
$\mathcal{C}_5 = \{C,D\}$.
Note that all of the classes, with
the exception of $\mathcal{C}_1$, do not
contain the identity element.  This means that classes, with
the exception of the class that contain the identity
as its sole element, cannot be groups.

This brings us to our next definition: \textit{subgroups}.  A
subgroup has all of the same properties as a group, but is part
of a larger group.  Just by looking at our multiplication
table, one is able to see that there are a number of subgroups:
one of order 4, 5 of order 2, and the obvious order one
identity subgroup.





%The first commonly used term is the \textbf{coordinate
%transformation}.  Suppose there exists a map $\dot{x} = f(x)$
%where $x, f(x) \in \mathcal{M}$.  For any $[d\times d]$,
%non-singular ($det(M) \neq 0$) matrix $M$, that maps the vector
%$x$


\section{Discrete symmetries of PCF}

The Dirichlet boundary condition on the top and bottom plates
leads to rotational invariance in both the spanwise and
streamwise directions, which leads to three rotational
symmetries in {\pCf}

\begin{eqnarray}
 \sigma_z[u,v,w](x,y,z) & = & [u,v,-w](x,y,-z) \\
 \sigma_x[u,v,w](x,y,z) & = & [-u,-v,w](-x,-y,z) \\
 \sigma_{xz}[u,v,w](x,y,z) & = & [-u,-v,-w](-x,-y,-z)
\end{eqnarray}




\section{Continuous symmetries}

Both the translational symmetries of the infinite-extent
pipes and planes flows, and their finite cell versions with
periodic boundary conditions imposed in both the streamwise and
spanwise directions are examples of continuous symmetries.  Factoring
out continuous symmetries in spatially extended systems can be
useful in reducing the \statesp\  and visualizing the dynamics
of infinite dimensional systems.  How this \statesp\ reduction is
implemented is discussed in detail in
\refrefs{SiminosThesis,DasBuch} for the \cLf.


\subsection{PCF symmetries and isotropy subgroups}
\label{s:symm}
% n00bs.tex 2009-06-17

    \PC{copied this section from \refref{HGC08}}
On an infinite domain and in the absence of boundary conditions, the Navier-Stokes
equations are equivariant under any $3D$~translation, $3D$~rotation, and
$\bx \to -\bx$, $\bu \to -\bu$ inversion through the origin \rf{frisch}.
In {\pCf}, the counter-moving walls restrict the rotation
symmetry to rotation by $\pi$ about the $z$-axis. We denote this rotation
by $\sigma_{x}$ and the inversion through the origin by $\sigma_{xz}$.
The suffixes indicate
which of the homogeneous directions $x,z$ change sign and simplify the
notation for the group algebra of rotation, inversion, and translations
presented in \refsects{s:flipnshift}{s:67-fold}.
The $\sigma_{xz}$ and $\sigma_x$ symmetries generate a discrete dihedral group
$D_1 \times D_1 = \{e,\sigma_x,\sigma_{z},\sigma_{xz}\}$ of order 4, where
\begin{align}
\sigma_x    \, [u,v,w](x,y,z) &= [-u,-v,w](-x,-y,z) \nnu \\
\sigma_z    \, [u,v,w](x,y,z) &= [u, v,-w](x,y,-z)  \label{sigma}\\
\sigma_{xz} \, [u,v,w](x,y,z) &= [-u,-v,-w](-x,-y,-z) \nnu
\,.
\end{align}

The walls also restrict the translation symmetry to $2D$ in-plane
translations. With periodic boundary conditions, these translations
become the $SO(2)_x \times SO(2)_z$ continuous two-parameter
group of streamwise-spanwise translations
\begin{align}
\tau(\shift_x, \shift_z) [u, v, w](x,y,z) &= [u, v, w](x+\shift_x, y, z+ \shift_z)
\,.
\label{translation}
\end{align}
The equations of {\pCf} are thus equivariant under the group
$\GPCF = O(2)_x \times O(2)_z = D_{1,x} \ltimes SO(2)_{x} \times D_{1,z}
\ltimes SO(2)_z$, where $\ltimes$ stands for a semi-direct product,
$x$  subscripts indicate streamwise translations
and sign changes in $x,y$, and $z$ subscripts indicate spanwise translations
and sign changes in $z$.

The solutions of an equivariant system can satisfy all of
the system's symmetries, a proper subgroup of them, or
have no symmetry at all. For a given solution $\bu$, the
subgroup that contains all symmetries that fix $\bu$ (that satisfy
$s \bu = \bu$) is called the isotropy (or stabilizer) subgroup of $\bu$%
\rf{hoyll06,MarRat99, golubitsky2002sp, GL-Gil07b}. For example, a typical
turbulent trajectory $\bu(\bx,t)$ has no symmetry beyond the identity,
so its isotropy group is $\{e\}$. At the other extreme is the laminar
{\eqv}, whose isotropy group is the full plane Couette symmetry
group $\GPCF$.

In between, the isotropy subgroup  of the Nagata \eqva\ and most
of the \eqva\ reported here is $S = \{e, s_1, s_2, s_3\}$, where
\begin{align}
s_1 \, [u, v, w](x,y,z) &= [u, v, -w](x+L_x/2, y, -z) \nnu \\
s_2 \, [u, v, w](x,y,z) &= [-u, -v, w](-x+L_x/2,-y,z+L_z/2) \label{shiftRot}\\
s_3 \, [u, v, w](x,y,z) &= [-u,-v,-w](-x, -y, -z+L_z/2) \nnu
\,.
\end{align}
These particular combinations of flips and shifts match the symmetries
of instabilities of streamwise-constant streaky flow\rf{W97,W03}
and are well suited to the wavy streamwise streaks observable in
simulations, %\reffig{f:bigbox},
with suitable choice of $L_x$ and $L_z$.
But $S$ is one choice among a number of intermediate isotropy
groups of $\GPCF$, and other subgroups might also play an
important role in the turbulent dynamics. In this section we
provide a partial classification of the isotropy groups of $\GPCF$, sufficient
to classify all currently known invariant solutions and to guide
the search for new solutions with other symmetries. We focus on
isotropy groups involving at most half-cell shifts. The main result is
that among these, up to conjugacy in spatial translation, there
are only  five isotropy groups in which we should expect to find {\eqva}.

\subsection{Flips and half-shifts}
\label{s:flipnshift}
% n00bs.tex 2009-06-17

    \PC{copied this section from \refref{HGC08}}
A few observations will be useful in what follows. First, we note the
key role played by the rotation and reflection symmetries $\sigma_x$
and $\sigma_z$ \refeq{sigma} in the classification of solutions and
their isotropy groups. The equivariance of {\pCf} under
continuous translations allows for {\reqvD} solutions, \ie,
solutions that are steady in a frame moving with a constant velocity
in $(x,z)$. In {\statesp}, {\reqva} either trace out
circles or wind around tori, and these sets are both
continuous-translation and time invariant. The sign changes under
$\sigma_x$, $\sigma_{z}$, and $\sigma_{xz}$, however, imply particular
centers of symmetry in $x$, $z$, and both $x$ and $z$, respectively,
and thus fix the translational phases of fields that are fixed by these
symmetries. Thus the presence of $\sigma_x$ or $\sigma_z$ in an
isotropy group prohibits {\reqva} in $x$ or $z$, and the
presence of $\sigma_{xz}$ prohibits any form of {\reqv}. Guided
by this observation, we will seek {\eqva} only for isotropy subgroups
that contain the $\sigma_{xz}$ inversion symmetry.

Second, the periodic boundary conditions impose discrete
translation symmetries of $\tau(L_x, 0)$ and $\tau(0, L_z)$ on
velocity fields. In addition to this full-period translation symmetry,
a solution can also be fixed under a rational translation, such as
$\tau(m L_x/n, 0)$ or a continuous translation $\tau(\ell_x, 0)$.
If a field is fixed under continuous translation, it is
constant along the given spatial variable. If it is fixed under rational
translation $\tau(m L_x/n, 0)$, it is fixed under $\tau(m L_x/n,
0)$ for $m \in [1, n-1]$ as well, provided that $m$ and $n$ are
relatively prime. For this reason the subgroups of the
continuous translation $SO(2)_x$ consist of the discrete cyclic groups
$C_{n,x}$ for $n=2,3,4,\ldots$ together with the trivial subgroup $\{e\}$
and the full group $SO(2)_x$ itself, and similarly for $z$. For rational
shifts $\ell_x/L_x = m/n$ we simplify the notation a bit by rewriting
\refeq{translation} as
\begin{align}
\trDiscr{x}{m/n} = \tau(m L_x/n, 0) \,,\;
\trDiscr{z}{m/n} = \tau(0,m L_z/n) \,.
\label{translDisc}
\end{align}
Since $m/n = 1/2$ will loom large in what follows, we omit exponents of $1/2$:
\beq
    \trHalf{x} = \trDiscr{x}{1/2}
    \,,\;
    \trHalf{z} = \trDiscr{z}{1/2}
    \,,\;
    \trHalf{xz} = \trHalf{x} \trHalf{z}
\,.
\label{tauHalf}
\eeq
If a field $\bu$ is fixed under a rational shift $\tau(L_x/n)$,
it is periodic on the smaller spatial domain $x \in [0,L_x/n]$.
For this reason we can exclude from our searches all \eqv\
whose isotropy subgroups contain
rational translations in favor of \eqva\ computed on smaller domains.
However, as we need to study bifurcations into
states with wavelengths longer than the initial state,
the linear stability computations
need to be carried out in the full $[L_x,2,L_z]$ cell.
For example, if \tEQ{}\ is an \eqv\ solution in the
$\bCell_{1/3}= [L_x/3,2,L_z]$ cell, we refer to the
same solution repeated thrice in $\bCell = [L_x,2,L_z]$
as ``spanwise-tripled'' or
$3 \times \tEQ{}$. Such solution is by construction fixed under the
$C_{3,x} = \{e,\trDiscr{x}{1/3},\trDiscr{x}{2/3}\}$ subgroup.


Third, some isotropy groups are conjugate to each other under
symmetries of the full group $\GPCF$. Subgroup $H'$ is conjugate to $H$
if there is an $s \in \GPCF$ for which $H' = s^{-1} H s$. In spatial terms,
two conjugate isotropy groups are equivalent to each other under a coordinate
transformation. A set of conjugate isotropy groups forms a conjugacy class.
It is necessary to consider only a single representative of each conjugacy
class; solutions belonging to conjugate isotropy groups can be generated by
applying the symmetry operation of the conjugacy.

In the present case conjugacies under spatial translation symmetries are
particularly important. Note that $O(2)$ is not an abelian group, since
reflections $\sigma$ and translations $\tau$ along the same axis do not
commute\rf{Harter93}. Instead we have $\sigma \tau  = \tau^{-1} \sigma$.
Rewriting this relation as $\sigma \tau^{2} = \tau^{-1} \sigma \tau$, we
note that
\bea
\sigma_x \tau_x(\ell_x, 0)
  &=& \tau^{-1}(\ell_x/2, 0) \, \sigma_x \, \tau(\ell_x/2, 0)
\,.
\label{origShift}
\eea
The right-hand side of \refeq{origShift} is a similarity
transformation that translates the origin of coordinate system. For
$\shift_x = L_x/2$ we have
\beq
\trDiscr{x}{-1/4} \, \sigma_{x} \, \trDiscr{x}{1/4}
        = \sigma_{x} \trHalf{x}
\label{origQuartShift}
\,,
\eeq
and similarly for the spanwise shifts / reflections. Thus for
each isotropy group containing the shift-reflect $\sigma_x \tau_x$
symmetry, there is a simpler conjugate isotropy group in which
$\sigma_x \tau_x$ is replaced by $\sigma_x$ (and similarly
for $\sigma_z \tau_z$ and $\sigma_z$). We choose as the representative
of each conjugacy class the simplest isotropy group, in which all such
reductions have been made. However, if an isotropy group contains both
$\sigma_x$ and $\sigma_x \tau_x$, it cannot be simplified this way,
since the conjugacy simply interchanges the elements.

Fourth, for $\shift_x = L_x$, we have
$\trHalf{x}^{-1} \, \sigma_{x} \, \trHalf{x} = \sigma_x \,,$
so that, in the special case of half-cell shifts,
$\sigma_x$ and $\tau_x$ commute. For the same reason,
$\sigma_z$ and $\tau_z$ commute, so the order-16 isotropy subgroup
\beq
G = D_{1,x} \times C_{2,x} \times D_{1,z} \times C_{2,z} \subset \GPCF
\ee{order16subgrp}
is abelian.


\subsection{PCF: The 67-fold path}
\label{s:67-fold}
% n00bs.tex 2009-06-17

    \PC{copied this section from \refref{HGC08}}
We now undertake a partial classification
of the lattice of isotropy subgroups of {\pCf}. We focus on
isotropy groups involving at most half-cell shifts, with $SO(2)_x \times SO(2)_z$
translations restricted to order 4 subgroup of spanwise-streamwise
translations \refeq{tauHalf} of half the cell length,
\beq
T = C_{2,x}\times C_{2,z}
  =  \{e,\trHalf{x},\trHalf{z},\trHalf{xz}\}
\,.
\label{tauD2}
\eeq
All such isotropy subgroups of $\GPCF$ are contained
in the subgroup $G$  \refeq{order16subgrp}. Within $G$, we look for the
simplest representative of each conjugacy class, as described above.


Let us first enumerate all subgroups $\isotropyG{} \subset G$.
The subgroups can be of order
$|\isotropyG{}| = \{1,2,4,8,16\}$.
A subgroup is generated by multiplication of a set of
generator elements, with the choice of
generator elements unique up to a permutation of subgroup
elements.
A subgroup of order $|\isotropyG{}| =  2$ has only one generator,
since every group element is its own inverse. There are 15
non-identity elements in $G$ to choose from, so there are 15 subgroups
of order 2.
Subgroups of order 4 are generated by multiplication of two
group elements. There are 15 choices for the first and 14
choices for the second. However, each order-4 subgroup
can be generated by $3 \cdot 2$ different choices of generators.
For example, any two of $\tau_x, \tau_z, \tau_{xz}$ in any order
generate the same group $T$. Thus there are $(15 \cdot 14)/(3 \cdot 2) = 35$
subgroups of order 4.

Subgroups of order 8 have three generators.  There are
15 choices for the first generator, 14 for the second, and 12 for the
third. There are 12 choices for the third
generator and not  13, since if it were chosen to be the product of the
first two generators, we would get a subgroup of order 4.
Each order-8 subgroup can be generated
by $7 \cdot 6 \cdot 4$ different choices of three generators, so there are
$(15 \cdot 14 \cdot 12)/(7 \cdot 6 \cdot 4) = 15$ subgroups of order 8.
In summary: there is the group $G$ itself, of order 16,
15 subgroups of order 8, 35 of order 4, 15 of
order 2, and 1 (the identity) of order 1,
or 67 subgroups in all\rf{HalcrowThesis}.
This is whole lot of isotropy subgroups to juggle; fortunately,
the observations of \refsect{s:flipnshift} show that there
are only 5 {\em distinct conjugacy classes} in which we can expect
to find \eqva.

The 15 order-2 groups fall into 8 distinct conjugacy
classes, under conjugacies between $\sigma_x \tau_x$ and $\sigma_x$
and $\sigma_z \tau_z$ and $\sigma_z$. These conjugacy classes are
represented by the 8 isotropy groups generated individually by the 8
generators
$\sigma_x,\, \sigma_z,\, \sigma_{xz},\, \sigma_x \tau_z,\,  \sigma_z \tau_x,\,
\tau_x,\, \tau_z,\,$ and $\tau_{xz}$. Of these, the latter three imply
periodicity on smaller domains. Of the remaining five,
$\sigma_x$ and $\sigma_x \tau_z$ allow {\reqva} in $z$,
$\sigma_z$ and $\sigma_z \tau_x$ allow {\reqva} in $x$.
Only a single conjugacy class, represented by the isotropy
group
\beq
  \{e,\sigma_{xz}\}
\,,
\ee{S3subgrp}
breaks both continuous translation symmetries. Thus, of all
order-2  isotropy groups, we expect only this group to have {\eqva}.
\tEQnine, \tEQten, and \tEQelev\ described below are examples of \eqva\
with isotropy group $\{e,\sigma_{xz}\}$.

Of the 35 subgroups of order 4, we need to identify those that
contain $\sigma_{xz}$ and thus support {\eqva}. We choose
as the simplest representative of each conjugacy class the isotropy
group in which $\sigma_{xz}$ appears in isolation.
Four isotropy subgroups of order 4 are generated by picking
$\sigma_{xz}$ as the first generator, and $\sigma_{z},\, \sigma_{z}
\trHalf{x},\, \sigma_{z} \trHalf{z},\,$ or $\sigma_{z} \trHalf{xz}$
as the second generator (\emph{R} for reflect-rotate):
\begin{align}
 R~~  &=  \{e, \sigma_x, \sigma_z, \sigma_{xz}\}
      ~~~~~~~\; = \{e,\sigma_{xz}\} \times \{e,\sigma_{z}\} \nnu\\
 R_x~ &=  \{e,\sigma_x \trHalf{x}, \sigma_z \trHalf{x}, \sigma_{xz}\}
      ~~ = \{e,\sigma_{xz}\} \times \{e,\sigma_{x}\trHalf{x}\}
        \label{subg4RR} \\
 R_z~ &=  \{e, \sigma_x \trHalf{z}, \sigma_z \trHalf{z}, \sigma_{xz}\}
      ~~\; = \{e,\sigma_{xz}\} \times \{e,\sigma_{z}\trHalf{z}\}
        \nnu\\
 R_{xz} &= \{e, \sigma_x \trHalf{xz}, \sigma_z \trHalf{xz}, \sigma_{xz}\}
        = \{e,\sigma_{xz}\} \times \{e,\sigma_{z}\trHalf{xz}\}
        \simeq S \,. \nnu
\end{align}
These are the only isotropy groups of order 4 containing $\sigma_{xz}$
and no isolated translation elements. Together with $\{e,\sigma_{xz}\}$,
these 5 isotropy subgroups represent the 5 conjugacy classes in
which expect to find {\eqva}.

The $R_{xz}$ isotropy subgroup is particularly important, as the\rf{N90}
{\eqva} belong to this conjugacy class\rf{W97,CB97,W03}, as do
most of the solutions reported here. The \emph{NBC} isotropy subgroup of
\refref{Schmi99} and $S$ of \refref{GHCW07} are conjugate to $R_{xz}$ under
quarter-cell coordinate transformations. In keeping with previous literature,
we often represent this conjugacy class with
$S = \{e, s_1, s_2, s_3\} = \{e, \sigma_z \tau_x, \sigma_x \tau_{xz},
\sigma_{xz} \tau_z\}$ rather than the simpler conjugate group $R_{xz}$.
Schmiegel's \emph{I} isotropy group is conjugate to our $R_{z}$; \refref{Schmi99}
contains many examples of $R_z$-isotropic \eqva. $R$-isotropic {\eqva} were found
by \rf{TuckBar03} for {\pCf} in which the translation symmetries
were broken by a streamwise ribbon. We have not searched for $R_x$-isotropic
solutions, and are not aware of any published in the literature.

The remaining subgroups of orders 4 and 8 all involve $\{e,\tau_i\}$ factors
and thus involve states that are periodic on half-domains.
For example, the isotropy subgroup of \tEQsev\ and \tEQeight\ studied below is
$S  \times \{e, \tau_{xz}\} \simeq R  \times \{e, \tau_{xz}\}$,
and thus these are doubled states of solutions on half-domains. For
the detailed count of all 67 subgroups, see \refref{HalcrowThesis}.


\exercise{Symmetries of invariant solutions for duct flows}
		 {\label{exer:pathsDuct}
{\bf PC} to {\bf CS}:
 this is optional, but it might have customers. Repeat
in this section Halcrow \etal\ classification of isotropy
groups for the duct flows (square profile, as opposed to the
circular profile of the pipe). Reason: number of groups
(Nagata, Botero, Kawahara, etc - see Marburg turbulence
conference ETC12 proceedings\rf{etc12}, in \wwwcb{/library},
and my
\HREF{http://www.channelflow.org/dokuwiki/doku.php/gtspring2009:predrag:blog\#trieste_2009_turbulence_conference}
      {channelflow.org blog})
are repeating plane Couette kind of investigations for the duct
problem. None of them have worked detailed group theory, or the
\statesp\ visualizations. For us it is simpler than the pipe,
as there is only one, streamwise continuous symmetry, and I
believe it is pure \SOn{2}, not \On{2}.
    } %end \exercise{?-path for duct flows}
