\ifsvnmulti
\svnkwsave{$RepoFile: siminos/chao/KS.tex $}
\svnidlong {$HeadURL$}
{$LastChangedDate$}
{$LastChangedRevision$} {$LastChangedBy$}
\svnid{$Id$}
\fi


\chapter{\KSe}
\label{chap:KS}

\begin{description}

\item[2011-08-24 PC to Chao]
I have created this chapter to help you get started with your own \KS\
notes - edit this your own way as you learn this stuff. You do not
need to reinvent the wheel - clip freely laTeX from Siminos papers
which are all included in this repository.

Keep it up-to-date - if all goes well, it can be used in forthcoming
papers and
\HREF{http://www.online-literature.com/wilde/being_earnest/}
{the thesis}. Please use the same notational conventions and
macros, makes life easier later on.

\end{description}

    \PC{first draft is clipped from Cvitanovi{\'c}, Davidchack and
    Siminos\rf{SCD07}, rpo.tex version of 2009-04-10, please rewrite in
    your own words.}
Recent experimental and theoretical advances\rf{science04}
support a dynamical vision of turbulence:
For any finite  spatial resolution,
a turbulent flow follows approximately for a finite time
a pattern belonging to a
{ finite alphabet}
of admissible patterns.
The long term dynamics is
a {walk through the space of these unstable patterns}.
The question is how to characterize and classify such patterns?
Here we follow the seminal Hopf paper\rf{hopf48}, and  visualize
turbulence not as  a sequence of
spatial snapshots in turbulent evolution,
but as a trajectory in an
 $\infty$-$d$ \statesp\ in which an
instant in turbulent evolution is
a {unique} point. In the dynamical systems approach,
theory of turbulence for a given system, with given boundary conditions,
is given by
(a) the geometry of the \statesp\ and (b) the associated natural measure,
\ie,
the likelihood that asymptotic dynamics visits a given \statesp\ region.

We pursue this program in context of the \KS\ (KS) equation,
one of the simplest physically interesting spatially extended
nonlinear systems.  Holmes, Lumley and Berkooz\rf{Holmes96} offer a
delightful discussion of why this system deserves study as a staging
ground for studying turbulence in full-fledged Navier-Stokes
boundary shear flows.

{
Flows described by partial differential equations (PDEs) are
said to be infinite dimensional because if one writes them
down as a set of ordinary differential equations (ODEs), a set
of infinitely many ODEs is needed to represent the dynamics
of one PDE. Even though their {\statesp} is thus
$\infty$-dimensional, the long-\-time dynamics of viscous
flows, such as Navier-Stokes, and PDEs modeling them, such as
\KS, exhibits, when dissipation is high and
the system spatial extent small, apparent `low-dimensional'
dynamical behaviors. For some of these the asymptotic
dynamics is known to be confined to a finite-\-dimensional
{\em inertial manifold}, though the rigorous upper bounds on
this dimension are not of much use in the practice.

For large spatial extent the complexity of the spatial
motions also needs to be taken into account. The systems
whose spatial correlations decay sufficiently fast, and the
attractor dimension and number of positive Lyapunov exponents
diverges with system size are said\rf{HNZks86,man90b,cross93}
to be extensive, `spatio-temporally chaotic' or `weakly
turbulent.' Conversely, for small system sizes the accurate
description might require a large set\rf{GHCW07} of coupled
ODEs, but dynamics can still be `low-dimensional' in the
sense that it is characterized with one or a few positive
Lyapunov exponents. There is no wide range of scales
involved, nor decay of spatial correlations, and the system
is in this sense only `chaotic.'

For a subset of physicists and mathematicians who study
idealized `fully developed,' `homogenous' turbulence the
generally accepted usage is that the `turbulent' fluid is
characterized by a range of scales and an energy cascade
describable by statistic assumptions\rf{frisch}. What experimentalists,
engineers, geophysicists, astrophysicists actually observe
looks nothing like a `fully developed turbulence.' In the
physically driven wall-bounded shear flows, the turbulence is
dominated by unstable \emph{coherent structures}, that is,
localized recurrent vortices, rolls, streaks and like. The
statistical assumptions fail, and a dynamical systems
description from first principles is called for\rf{Holmes96}.
} %end

Dynamical \statesp\ representation of a PDE is
$\infty$-dimensional, but the KS flow is strongly contracting
and its non-wondering set, and, within it, the set of
invariant solutions investigated here, is embedded into a
finite-dimensional inertial manifold\rf{FNSTks85} in a
non-trivial, nonlinear way. `Geometry' in the title of this
paper refers to our attempt to systematically triangulate
this set in terms of dynamically invariant solutions (\eqva,
\po s, $\ldots$) and their unstable manifolds, in a PDE
representation and numerical simulation algorithm independent
way. The goal is to describe a given `turbulent' flow
quantitatively, not model it qualitatively by a
low-dimensional model. For the case investigated here, the
\statesp\ representation dimension $d \sim 10^2$ is set by
requiring that the exact invariant solutions that we compute
are accurate to $\sim 10^{-5}$.

{
Here comes our quandary. If we ban the words `turbulence' and
`spatiotemporal chaos' from our study of small extent
systems, the relevance of what we do to larger systems is
obscured. The exact unstable coherent structures we determine
pertain not only to the spatially small `chaotic' systems,
but also the spatially large `spatiotemporally chaotic' and
the spatially very large `turbulent' systems.
So, for the lack of more precise nomenclature, we take the
liberty of using the terms `chaos,' `spatiotemporal chaos,'
and `turbulence' interchangeably.
} %end


In previous work, the \statesp\ geometry and the natural measure for
this system have been
studied\rf{Christiansen97,LanThesis,lanCvit07} in terms of unstable
periodic solutions restricted to the antisymmetric subspace of the
KS dynamics.

The focus in this paper is on the role continuous symmetries
play in spatiotemporal dynamics. The notion of exact
periodicity in time is replaced by the notion of relative
spatiotemporal periodicity, and \reqva\ and \rpo s here play
the role the \eqva\ and \po s played in the earlier studies.
Our search for \rpo s in KS system was inspired by Vanessa
L{\'o}pez\rf{lop05rel} investigation of {\rpo s} of the
Complex Ginzburg-Landau equation.  However, there is a vast
literature on {\rpo s} since their first appearance, in
Poincar\'e study of the 3-body problem\rf{ChencinerLink,rtb},
where the Lagrange points are the \reqva.  They arise in
dynamics of systems with continuous symmetries, such as
motions of rigid bodies, gravitational $N$-body problems,
molecules and nonlinear waves. Recently Viswanath\rf{Visw07b}
has found both \reqva\ and \rpo s in
the plane Couette problem.
    {
A Hopf bifurcation of a traveling
wave\rf{AGHO288,AGHks89,Krupa90} induces a small
time-dependent modulation. Brown and Kevrekidis\rf{BrKevr96}
study bifurcation branches of \po s and \rpo s in KS system
in great detail. For our system size ($\alpha=49.04$ in their
notation) they identify a \po\  branch. In this
context \rpo s are referred to as `modulated traveling
waves.' For fully chaotic flows we find this notion too
narrow. We compute 60,000 \po s and \rpo s that are in no
sense small `modulations' of other solutions, hence our
preference for the well established notion of a `\rpo.'
          }

Building upon the pioneering work of
\refrefs{KNSks90,ksgreene88,BrKevr96}, we undertake here a
study of the \KS\ dynamics for a specific system size $L =
22$, sufficiently large to exhibit many of the features
typical of `turbulent' dynamics observed in large KS systems,
but small enough to lend itself to a detailed exploration of
the  \eqva\ and \reqva, their stable/unstable manifolds,
determination of a large number of \rpo s, and a preliminary
exploration of the relation between the observed
spatiotemporal `turbulent' patterns and the \rpo s.

In presence of a continuous symmetry any solution belongs to a group
manifold of equivalent solutions. The problem: If one is to
generalize the \po\  theory to this setting, one needs to
understand what is meant by solutions being nearby (shadowing) when
each solution belongs to a manifold of equivalent solutions. {In a
forthcoming publication\rf{SCD09b} we resolve this puzzle by implementing
symmetry reduction.} Here we demonstrate that, {for \rpo s visiting the
neighborhood of equilibria,} if one picks any
particular solution, the universe of all other solutions is rigidly
fixed through a web of heteroclinic connections between them. This
insight garnered from study of a 1-dimensional \KS\ PDE is more
remarkable still when applied to the plane Couette flow\rf{GHCW07},
with 3-$d$ velocity fields and two translational symmetries.

The main results presented here are: [NONE AS YET]

\section{\KSe\ - a review}
\label{chap:KSreview}

\begin{description}

\item[2011-08-24 PC to Chao]
Please keep writing and rewriting in this section the draft of the
\KSe\ review to eventually include in your thesis.

Read also Siminos blog section \emph{ \KS: literature survey}, you might
want to copy some of that stuff into your review of this section. Other
useful sources are Lan thesis\rf{LanThesis} and Siminos
thesis\rf{SiminosThesis}.

\end{description}



\section{Reading}
\label{s:KSreading}

\subsection{Spatiotemporal chaos in terms of unstable recurrent patterns}
\label{s:Christiansen97}

\begin{description}

\item[2011-08-24 Chao]
The central work of using unstable recurrent patterns describing spatiotemporal chaos is the search of \po s, especially unstable \po s. Stable ones are not as important because they are attractive, within their respective basins of attraction,
and display no chaotic behavior.
The unstable ones are typically \emph{saddles}, with both unstable and stable eigendirections. In the unstable directions, they shoot out neighboring points while in the stable directions
they draw them in. Thus dynamics is called ``hyperbolic'' because its linearized behavior is hyperbolic. Taken together, the set of unstable \po s is the source of chaotic behavior.
While we cannot find \emph{all} unstable \po s of a system
(their number is infinite, growing exponentially
with the period of the orbit), we can decompose its dynamics as the combination of influences of the near \po s.

This paper consists of 5 parts. I mainly focused on the 4th part: one-dimensional visualization, which illustrate in detail how we find \po\  and visulize them in low dimensions.

Here is my understanding of the whole process: First we choose a \PoincSec\
in the full \statesp\ and let the dynamical system evolve for a sufficiently long time to give us enough points on the section. This enables us to define a return map for the section. Within the section, iterating the map itself may generate discrete-time \po s, which can be found out exaustively
(in finite resolution) by utilizing symbolic dynamics. Before we use symbolic dynamics, we have to build an intrinsic arclength coordinate system; for
the parameter values studied here, close to onset of chaos, this return map
is approximately unimodal. This situation is very special, due to only one direction
of any \po\ being unstable. The remaining ones (except for the single marginal
eigenvalue due to Since periodic points in a discrete time \po\ belong to a continuous \po\  in the full \statesp\, we can integrate from these periodic points to recover the full space \po.

Here are some detailed questions:

In the third paragraph of part 4, it is said
``The $i$th cycle point  $s_i$ is mapped onto its image $s_{\sigma i} = f(s_i)$ where $\sigma i$ denotes  the label of the next periodic point in the cycle.''


So that means $\sigma i$ is not meant to be $i+1$, right? Because the order of different cycle points is given in following way: ``there exists a fixed point which
is not connected to the attractor (the point
$\overline{0}$ in \reffig{returnmap}) - we
choose this fixed point
as the starting point and assign it number $1$.
Point number $2$ is the periodic point in the sample
which is closest (in the full space)
to this fixed point, and the $n$-th point is determined as the point
which has the minimum distance from the point number $n-1$ among
all the periodic points which have not yet been enumerated.
Proceeding this way, we order all the periodic points that we have found
so far.''

So from this we could know that the order of periodic points on the section is not the order of time evolution of these points. Right? But if it is true, that seems to contradict with the illustration under figure 4?

\item[2011-08-30 Predrag]
For a general orbit we  label points in map iteration by
\beq
s_{n+1} = f(s_n)
\ee{curvTime}
where $n=\{1,2,\cdots\}$ is the discrete time.
For a periodic orbit of period $n$ we use $s_a$
where $a = \Ssym{1}\Ssym{2}\cdots\Ssym{n}$, and the
dynamics acts on the label as a cyclic permutation,
$\sigma s_a = \Ssym{2}\cdots\Ssym{n}\Ssym{1}$
\beq
s_{\Ssym{2}\cdots\Ssym{n}\Ssym{1}} =
    f(s_{\Ssym{1}\Ssym{2}\cdots\Ssym{n}})
\,,
\ee{curvPO}
where the block $\Ssym{1}\Ssym{2}\cdots\Ssym{n}$ is the
cycle itenerary.
For example, 11100-cycle point $s_{11100}$ is mapped into
next cycle point $s_{11001}$ which can be far away; spatial
ordering of cycle points is given by alternating binary trees
and such...

Unfortunately, in this application we discretize the 1-dimensional unstable manifold, so there is yet another integer
suffix which refers to a spatially ordered
set of points, such that
\beq
s_{j-1} < s_{j} < s_{j+1}
\,.
\ee{curvPO}

Can you try to add text to ChaosBook.org
smale.tex, section {\em Parametrization of invariant manifolds}
which clarifies this for the next reader? Maybe a drawing with
a 3-cycle would be helpful. Not sure

[will continue this later]


\item[2011-08-24 Chao]
A prerequisite question is that from paragraph 2 to paragraph 3, the
phrase``periodic points'' should mean the points on the reduced
1-dimensional map, right?(or points on the N-1 dimensional Poincare
section?) If so, since the reduced 1-dimensional map is a projection from
higher dimension, it could be that two different periodic points have the
same projection? What to do about that?


In figure 1, it is said `` The lower arrow indicates the kink where the
invariant set A starts to overlap with $\theta SA$''. I don't understand
how the overlap of A and $\theta SA$ is connected to the kink in the
Feigenbaum tree. I can don't have the intuition of how is it like that
``A overlaps with $\theta SA$''. Can you show me some simple examples to
illustrate it pictorially so that I can have an intuitive feeling?

\end{description}

\subsection{Unstable recurrent patterns in {\KS} dynamics}
\label{s:lanCvit07}

\begin{description}

\item[2011-08-24 PC to Chao]
Please enter here your notes, questions \etc\ while studying
Lan and Cvitanovi{\'c}\rf{lanCvit07}.
\item[2011-09-03 Chao to PC]
I read this paper several days ago.

Comparing with the previous paper, this one investigated larger
\KS\ system which allows more than one unstable mode
exist. I haven't got much more new things than I did in the previous one
because the idea in this paper is the same with the previous one. I still
have some detailed questions though.

In page 4, section A, it is said ``We pick any point on a typical orbit
of (4). It corresponds to a loop in 3-d state space of (8) and so can be
used to initialize the search for an u(x) profile periodic on [0,L]''.
So, a point in the fourier space periodic orbit is mapped into a loop in
real space? Don't feel like this is correct.

For your convenience,

equation (4):
\beq
\displaystyle\dot{a}_{k}
= (k/\tilde{L})^{2}(1-(k/\tilde{L})^{2})a_{k}
- (k/\tilde{L})\sum\nolimits^{+\infty}_{m=-\infty}a_{m}a_{k-m}
\eeq

equation (8)
\beq u_x = v, v_x = w, w_x = u^2 - v -E
\eeq
For the rest part of the paper, I think I can well understand them only
if I start computation myself. I don't even know where to start to ask if
I don't go through the process. So I'll record the question here later if
I have any question in my own computation.
Plus, for further study and better understanding of this paper, I should
read references [22], [33]-[35] in particular.

\end{description}



\subsection{60,000 \rpo s and no place to go}
\label{s:SCD07}

\begin{description}

\item[2011-08-24 PC to Chao]
Please enter here your notes, questions \etc\ while studying
Cvitanovi{\'c}, Davidchack and Siminos\rf{SCD07},
\emph{On the \statesp\ geometry of the {\KS} flow}

\item[2011-09-03 Chao to PC]
This paper discussed in detail how symmetry simplifies the search for
equilibria/relative equilibria and peoriodic/relavtive periodic orbits. I
mainly focused on the first half and appendices which describes the idea
and computation methods. The rest of the paper is just details and
results of computation which I can only master by starting computing
myself.

In appendix A, I checked equation (A.6) and it is correct. But the notion
of writing equation in Fourier transform operator is not so explicit.
That's why at first I was stuck on them. Equation (A.5) is just equation 4.2 in
Chaosbook. Here I rewrite (A.5) and (A.6) as (repeated indices summed over):
\bea
\dot{b}_k &=& \frac{\partial v_k(a)}{\partial a_j}b_j
\continue
v_k(a) &=& \dot{a}_k = (q^2_k - q^4_k)a_k - \frac{iq_k}{2}\sum\nolimits_ma_ma_{k-m}
\continue
\frac{\partial v(a)_k}{\partial a_j} &=&
(q^2_k - q^4_k) - \frac{iq_k}{2}\sum\nolimits_m(\delta_{mj}a_{k-m}+a_m\delta_{j,k-m})
\continue
&=& (q^2_k - q^4_k) - {iq_k}a_{k-j}
\,.
\label{KSstabMat1}
\eea

Combining the last two equations, we obtain:
\beq
\dot{b}_k = (q^2_k - q^4_k)b_k - iq_k\sum\nolimits_ja_{k-j}b_j
\,.
\label{KSstabMat2}
\eeq
The last term is discrete convolution of vector a and b in Fourier space,
which corresponds to component-wise product of two vectors in original
space due to Fourier transform's property. That is why it has the form in
(A.6).

We have to compute vector $\dot{b}_k$ because we have to observe how a
small deviation from the periodic orbit described by ${a_k,\dot{a}_k}$
evolves and integrate segment by segment(e.g. numerical integral) to
obtain Floquet Multipliers and thus obtain Jacobian matrix and Lyapunov
exponents.

Appendix C gives the searching algorithm of relative periodic orbits. I
got the idea, waiting to embark on real computation to knock down more
blockade.

\end{description}

\subsection{Turbulence, coherent structures, dynamical systems and
symmetry}
\label{s:Holmes96}

\begin{description}

\item[2011-08-24 PC to Chao]
Please enter here your notes, questions \etc\ while studying
Holmes, Lumley and Berkooz\rf{Holmes96}.

\end{description}
