\ifsvnmulti
\svnkwsave{$RepoFile: siminos/chao/KS.tex $}
\svnidlong {$HeadURL$}
{$LastChangedDate$}
{$LastChangedRevision$} {$LastChangedBy$}
\svnid{$Id$}
\fi


\chapter{\KSe}
\label{chap:KS}

\begin{description}

\item[2011-08-24 PC to Chao]
I have created this chapter to help you get started with your own \KS\
notes - edit this your own way as you learn this stuff. You do not
need to reinvent the wheel - clip freely LaTeX from Siminos papers
which are all included in this repository.

Keep it up-to-date - if all goes well, it can be used in forthcoming
papers and
\HREF{http://www.online-literature.com/wilde/being_earnest/}
{the thesis}. Please use the same notational conventions and
macros, makes life easier later on.

\item[2011-12-10 Predrag] Ciao Chao: started returning useful text back
to \texttt{siminos/blog/}

\end{description}

    \PC{first draft is clipped from Cvitanovi{\'c}, Davidchack and
    Siminos\rf{SCD07}, rpo.tex version of 2009-04-10, please rewrite in
    your own words.}
Recent experimental and theoretical advances\rf{science04}
support a dynamical vision of turbulence:
For any finite  spatial resolution,
a turbulent flow follows approximately for a finite time
a pattern belonging to a
{ finite alphabet}
of admissible patterns.
The long term dynamics is
a {walk through the space of these unstable patterns}.
The question is how to characterize and classify such patterns?
Here we follow the seminal Hopf paper\rf{hopf48}, and  visualize
turbulence not as  a sequence of
spatial snapshots in turbulent evolution,
but as a trajectory in an
 $\infty$-$d$ \statesp\ in which an
instant in turbulent evolution is
a {unique} point. In the dynamical systems approach,
theory of turbulence for a given system, with given boundary conditions,
is given by
(a) the geometry of the \statesp\ and (b) the associated natural measure,
\ie,
the likelihood that asymptotic dynamics visits a given \statesp\ region.

We pursue this program in context of the \KS\ (KS) equation,
one of the simplest physically interesting spatially extended
nonlinear systems.  Holmes, Lumley and Berkooz\rf{Holmes96} offer a
delightful discussion of why this system deserves study as a staging
ground for studying turbulence in full-fledged Navier-Stokes
boundary shear flows.

{
Flows described by partial differential equations (PDEs) are
said to be infinite dimensional because if one writes them
down as a set of ordinary differential equations (ODEs), a set
of infinitely many ODEs is needed to represent the dynamics
of one PDE. Even though their {\statesp} is thus
$\infty$-dimensional, the long-\-time dynamics of viscous
flows, such as Navier-Stokes, and PDEs modeling them, such as
\KS, exhibits, when dissipation is high and
the system spatial extent small, apparent `low-dimensional'
dynamical behaviors. For some of these the asymptotic
dynamics is known to be confined to a finite-\-dimensional
{\em inertial manifold}, though the rigorous upper bounds on
this dimension are not of much use in the practice.

For large spatial extent the complexity of the spatial
motions also needs to be taken into account. The systems
whose spatial correlations decay sufficiently fast, and the
attractor dimension and number of positive Lyapunov exponents
diverges with system size are said\rf{HNZks86,man90b,cross93}
to be extensive, `spatio-temporally chaotic' or `weakly
turbulent.' Conversely, for small system sizes the accurate
description might require a large set\rf{GHCW07} of coupled
ODEs, but dynamics can still be `low-dimensional' in the
sense that it is characterized with one or a few positive
Lyapunov exponents. There is no wide range of scales
involved, nor decay of spatial correlations, and the system
is in this sense only `chaotic.'

For a subset of physicists and mathematicians who study
idealized `fully developed,' `homogenous' turbulence the
generally accepted usage is that the `turbulent' fluid is
characterized by a range of scales and an energy cascade
describable by statistic assumptions\rf{frisch}. What experimentalists,
engineers, geophysicists, astrophysicists actually observe
looks nothing like a `fully developed turbulence.' In the
physically driven wall-bounded shear flows, the turbulence is
dominated by unstable \emph{coherent structures}, that is,
localized recurrent vortices, rolls, streaks and like. The
statistical assumptions fail, and a dynamical systems
description from first principles is called for\rf{Holmes96}.
} %end

Dynamical \statesp\ representation of a PDE is
$\infty$-dimensional, but the KS flow is strongly contracting
and its non-wondering set, and, within it, the set of
invariant solutions investigated here, is embedded into a
finite-dimensional inertial manifold\rf{FNSTks85} in a
non-trivial, nonlinear way. `Geometry' in the title of this
paper refers to our attempt to systematically triangulate
this set in terms of dynamically invariant solutions (\eqva,
\po s, $\ldots$) and their unstable manifolds, in a PDE
representation and numerical simulation algorithm independent
way. The goal is to describe a given `turbulent' flow
quantitatively, not model it qualitatively by a
low-dimensional model. For the case investigated here, the
\statesp\ representation dimension $d \sim 10^2$ is set by
requiring that the exact invariant solutions that we compute
are accurate to $\sim 10^{-5}$.

{
Here comes our quandary. If we ban the words `turbulence' and
`spatiotemporal chaos' from our study of small extent
systems, the relevance of what we do to larger systems is
obscured. The exact unstable coherent structures we determine
pertain not only to the spatially small `chaotic' systems,
but also the spatially large `spatiotemporally chaotic' and
the spatially very large `turbulent' systems.
So, for the lack of more precise nomenclature, we take the
liberty of using the terms `chaos,' `spatiotemporal chaos,'
and `turbulence' interchangeably.
} %end


In previous work, the \statesp\ geometry and the natural measure for
this system have been
studied\rf{Christiansen97,LanThesis,lanCvit07} in terms of unstable
periodic solutions restricted to the antisymmetric subspace of the
KS dynamics.

The focus in this paper is on the role continuous symmetries
play in spatiotemporal dynamics. The notion of exact
periodicity in time is replaced by the notion of relative
spatiotemporal periodicity, and \reqva\ and \rpo s here play
the role the \eqva\ and \po s played in the earlier studies.
Our search for \rpo s in KS system was inspired by Vanessa
L{\'o}pez\rf{lop05rel} investigation of {\rpo s} of the
Complex Ginzburg-Landau equation.  However, there is a vast
literature on {\rpo s} since their first appearance, in
Poincar\'e study of the 3-body problem\rf{ChencinerLink,rtb},
where the Lagrange points are the \reqva.  They arise in
dynamics of systems with continuous symmetries, such as
motions of rigid bodies, gravitational $N$-body problems,
molecules and nonlinear waves. Recently Viswanath\rf{Visw07b}
has found both \reqva\ and \rpo s in
the plane Couette problem.
    {
A Hopf bifurcation of a traveling
wave\rf{AGHO288,AGHks89,Krupa90} induces a small
time-dependent modulation. Brown and Kevrekidis\rf{BrKevr96}
study bifurcation branches of \po s and \rpo s in KS system
in great detail. For our system size ($\alpha=49.04$ in their
notation) they identify a \po\  branch. In this
context \rpo s are referred to as `modulated traveling
waves.' For fully chaotic flows we find this notion too
narrow. We compute 60,000 \po s and \rpo s that are in no
sense small `modulations' of other solutions, hence our
preference for the well established notion of a `\rpo.'
          }

Building upon the pioneering work of
\refrefs{KNSks90,ksgreene88,BrKevr96}, we undertake here a
study of the \KS\ dynamics for a specific system size $L =
22$, sufficiently large to exhibit many of the features
typical of `turbulent' dynamics observed in large KS systems,
but small enough to lend itself to a detailed exploration of
the  \eqva\ and \reqva, their stable/unstable manifolds,
determination of a large number of \rpo s, and a preliminary
exploration of the relation between the observed
spatiotemporal `turbulent' patterns and the \rpo s.

In presence of a continuous symmetry any solution belongs to a group
manifold of equivalent solutions. The problem: If one is to
generalize the \po\  theory to this setting, one needs to
understand what is meant by solutions being nearby (shadowing) when
each solution belongs to a manifold of equivalent solutions. {In a
forthcoming publication\rf{SCD09b} we resolve this puzzle by implementing
symmetry reduction.} Here we demonstrate that, {for \rpo s visiting the
neighborhood of equilibria,} if one picks any
particular solution, the universe of all other solutions is rigidly
fixed through a web of heteroclinic connections between them. This
insight garnered from study of a 1-dimensional \KS\ PDE is more
remarkable still when applied to the plane Couette flow\rf{GHCW07},
with 3-$d$ velocity fields and two translational symmetries.

The main results presented here are: [NONE AS YET]

\section{\KSe\ - a review}
\label{chap:KSreview}

%\item[2011-9-19 CS]
\KSe\ is one of the simple non-linear partial
differential equations. It describes a system in which the transport of
energy through nonlinear mode coupling produces a balance between long
wavelength instability and short wavelength dissipation.
    \PC{Simple? In you reading I am sure you have run into more informative
    descriptions of why people study it (flame fronts, \etc)- keep expanding this
    paragraph as you read the literature}

It has many forms. One of them is \rf{ksgreene88}
    \PC{label the equations you might refer to later;
    for example, some of this is already used in \refeq{eq:KSeGreeneKim},
    improve on that rather than providing a poor cousin version here.}
\beq
y_t = -4y_{xxxx} - \alpha(y_{xx} + \frac{1}{2}y_x^2 - \frac{1}{4\pi}\int_0^{2\pi} y_x^2\,dx)
\eeq
The system is completed with given initial condition and by assuming
spatial periodicity which we take to be $2\pi$. The fourth order spatial
derivative term comes from dissipation, the second spatial derivative
term gives rise to instability, and the nonlinear term gives rise to mode
coupling. The last term is introduced to keep the level of the solution
constant, i.e., $\partial_t \oint ydx = 0$. The quantity $\alpha$ is the
bifurcation parameter.

If we take a derivative of the last equation with respect to $x$ and
substitute $\mu = y_x/2$ and rescaling as $t \to t/\alpha, x \to -x$ and
$\nu = 4/\alpha$, we recast \refeq{eq:KSeGreeneKim} in the form of \KSe\
that we shall use here:
\beq
\mu_t = (\mu^2)_x - \mu_{xx} - \nu\mu_{xxxx}
\eeq
where $\nu$ is a fourth-order ``hyper-viscosity'' damping parameter.

%to be continued...

\subsection{Energy transfer rates, \KS}
\label{sec:energy}

This section contains material copied from \refref{SCD07}.
The \KS\ [henceforth KS] system\rf{ku,siv},
which arises in the description of
stability of flame fronts, reaction-diffusion systems and many other
physical settings\rf{KNSks90}, is one of the simplest nonlinear PDEs that
exhibit spatiotemporally chaotic behavior. In the formulation
adopted here, the time evolution of the `flame front velocity'
$u=u(x,t)$ on a periodic domain $u(x,t) = u(x+L,t)$ is given by
\beq
  u_t = F(u) = -{\textstyle\frac{1}{2}}(u^2)_x-u_{xx}-u_{xxxx}
    \,,\qquad   x \in [-L/2,L/2]
    \,.
\ee{ks}
Here $t \geq 0$ is the time, and $x$ is the spatial coordinate.
The subscripts $x$ and $t$ denote partial derivatives with respect to
$x$ and $t$. In what follows
we shall state results of all calculations either in units of the
`dimensionless system size' $\tildeL$, or the system size $L = 2 \pi
\tildeL$.

In physical settings where the observation times are much
longer than the dynamical `turnover' and Lyapunov times
(statistical mechanics, quantum physics, turbulence) periodic
orbit theory\rf{DasBuch} provides highly accurate predictions
of measurable long-time averages such as the dissipation and
the turbulent drag\rf{GHCW07}. Physical predictions have to
be independent of a particular choice of ODE representation
of the PDE under consideration and, most importantly,
invariant under all symmetries of the dynamics. In this
section we discuss a set of such physical observables for the
1-$d$ KS invariant under reflections and translations. They
offer a representation of dynamics in which the symmetries
are explicitly quotiented out.
%We shall use these {observables} in \refsect{sec:energyL22} in order to
%visualize a set of solutions on these coordinates.

The {space average} of a function $\obser = \obser(\pSpace,t) = \obser(u(x,t))$  on
the interval $L$,
\beq
    \expct{\obser} = \Lint{\pSpace}\, \obser(\pSpace,t)
    \,,
    \label{rpo:spac_ave}
\eeq
is in general time dependent.
Its mean value is given by the {time average}
\beq
\timeAver{\obser}
    =
\lim_{t\rightarrow \infty} \frac{1}{t} \int_0^t \! d\tau \, \expct{\obser}
    =
\lim_{t\rightarrow \infty} \frac{1}{t} \int_0^t \!
    \Lint{\tau}  d\pSpace\, \obser(\pSpace,\tau)
    \,.
\label{rpo:tim_ave}
\eeq
The mean value of $\obser = \obser(u_\stagn) \equiv \obser_\stagn$ evaluated on $q$
\eqv\ or {\reqv} $u(\pSpace,t) = u_\stagn(\pSpace-ct)$ is
\beq
\timeAver{\obser}_\stagn = \expct{\obser}_\stagn = \obser_\stagn\,.
\label{rpo:u-eqv} \eeq Evaluation of the infinite time average
\refeq{rpo:tim_ave} on a function of a \po\ or \rpo\
$u_p(\pSpace,t)=u_p(\pSpace,t+\period{p})$ requires only a single
$\period{p}$ traversal,
\beq
  \timeAver{\obser}_p = \frac{1}{\period{p}}
    \int_0^{\period{p}} \! d\tau \, \expct{\obser}
\,.
\label{rpo:u-cyc}
\eeq

Equation \refeq{ks} can be written as
\beq
    u_t=- V_x
        \,,\qquad
    V(x,t)={\textstyle\frac{1}{2}}u^2+u_{x} + u_{xxx}
    \,.
\ee{ksPotent}
If $u$ is `flame-front velocity', then \expctE,
% defined in \refeq{eq:stdks},
\beq
{\textstyle\frac{1}{2}}u^2 - c u + u_x + u_{xxx}=\expctE
\,.
\label{eq:stdks}
\eeq
can be interpreted as the mean energy
density. So, even though KS is a phenomenological
small-amplitude equation, the time-dependent {$L^2$ norm
of $u$},
\beq
    \expctE=
  \Lint{\pSpace}
  V(x,t)=
  \Lint{\pSpace} \frac{u^2}{2}
  \,,
  \label{ksEnergy}
\eeq
has a physical interpretation\rf{ksgreene88} as the average `energy'
density of the flame front. This analogy to the mean kinetic energy
density for the Navier-Stokes motivates what follows.

The energy \refeq{ksEnergy} is intrinsic to the flow,
independent of the particular ODE basis set chosen to
represent the PDE. However, as the Fourier amplitudes are
eigenvectors of the translation operator, in the Fourier
space the energy is a diagonalized quadratic norm,
\beq
\expctE
          =  \sum_{k=-\infty}^{\infty} E_k
          = 2 \sum_{k=1}^{\infty} E_k
\,,\qquad
E_k =
    {\textstyle\frac{1}{2}}|a_k|^2
\,,
\ee{EFourier}
and explicitly invariant term by term
under KS translation
% \refeq{eq:shiftFour}
and reflection symmetries.
% \refeq{KSparity}.

Take time derivative of the energy density \refeq{ksEnergy},
substitute \refeq{ks} and integrate by parts. Total derivatives vanish
by the spatial periodicity on the $L$ domain:
\bea
   \dot{\expctE} &=&
     \expct{u_t \, u}
         = - \expct{\left({u^2}/{2} + {u_{x} + u_{xxx}}\right)_x u }
    \continue
    &=&
\expct{ u_x \, {u^2}/{2} + u_{x}{}^2 + u_x \, u_{xxx}}
    \,.
\label{rpo:ksErate}
\eea
The first term in \refeq{rpo:ksErate} vanishes by
integration by parts,
\(
3 \expct{ u_x \, u^2}= \expct{(u^3)_x} = 0
\,,
\)
and integrating the third term by parts yet again
one gets\rf{ksgreene88} that the energy variation
\beq
   \dot{\expctE} = P - D
                \,,\qquad
      P =  \expct{u_{x}{}^2}
                \,,\quad
      D =  \expct{u_{xx}{}^2}
\ee{EnRate}
balances the power $P$ pumped in by anti-diffusion $u_{xx}$
against the energy dissipation rate $D$
by hyper-viscosity $u_{xxxx}$
in the KS equation \refeq{ks}.

The time averaged energy density  $\timeAver{E}$
computed on a typical orbit goes to a constant, so
the expectation values \refeq{rpo:EtimAve} of drive and dissipation
exactly balance each out:
\beq
    \timeAver{\dot{E}}  =
    \lim_{t\rightarrow \infty}
        \frac{1}{t} \int_0^t d\tau \, \dot{\expctE}
=
      \timeAver{P} - \timeAver{D}
= 0
    \,.
\ee{rpo:EtimAve}
In particular, the \eqva\
and \reqva\ fall onto the diagonal in
%\reffig{f:drivedrag},
power input {\em vs.}
dissipation rate plots,
and so do time averages computed on \po s and \rpo s:
\beq
\timeAver{E}_p =
\frac{1}{\period{p}} \int_0^\period{p}d\tau \, E(\tau)
    \,,\qquad
\timeAver{P}_p =
\frac{1}{\period{p}} \int_0^\period{p} d\tau \, P(\tau)
    =
      \timeAver{D}_p
    \,.
\label{poE}
\eeq
In the Fourier basis \refeq{EFourier} the conservation of energy on average
takes form
\beq
0 = \sum_{k=-\infty}^{\infty} ( q_k^2 - q_k^4 )\,
    \timeAver{E}_k
\,,\qquad
E_k(t) =  {\textstyle\frac{1}{2}} |a_k(t)|^2
\,.
\ee{EFourier1}
The large $k$ convergence of this series is insensitive to the
system size $L$; $\timeAver{E_k}$ have to decrease much faster than
$q_k^{-4}$.
Deviation of $E_k$ from this bound for small $k$ determines the active modes.
For \eqva\ the $L$-independent bound
    on $E$ is given by Michaelson\rf{Mks86}.
The best current bound\rf{GiacoOtto05,bronski2005} on the long-time limit
of $E$
as a function of the system size $L$ scales as
{$E \propto L^2$}.



\subsection{{\Statesp} visualization of fluid flows}
\label{s:visualStatSp}
\renewcommand{\ssp}{a}

    \PC{2011-10-23: this entire section written by Predrag}
Hopf\rf{hopf48} envisioned the function space of {Navier-Stokes} velocity fields as
an infinite-dimensional \statesp\ $\pS$ in which each instantaneous state
of $3D$ fluid velocity field $\bu(\bx)$ is represented as a unique point
$\ssp$. In our particular application we can represent $\ssp =
(\vec{u}_{nkm})$ as a vector whose elements are the primitive
discretization variables \refeq{KSdiscr} %\refeq{pipeDiscr}
% \CS{2011-10-24:reference missing here, ``??''shows up}.
The $3D$ velocity field given
by $\vec{u}_{knm}(\zeit)$, obtained from integration of the \NS\
equations in time, can hence be seen as trajectory $\ssp(\zeit)$ in
$\approx 100,000$ dimensional space spanned by the free variables of our
numerical discretization, with the \NSe\ % \refeq{NavStokesDev}
(or \KSe)
rewritten as
\beq
   \dot{\ssp} = \vel(\ssp) ,
   \qquad
   \ssp(\zeit) = \ssp(0)
            + \int_0^\zeit \! \mathrm{d}\zeit' \, \vel(\ssp(\zeit'))
\,,
\ee{symbolicNS}
where the current state of the fluid $ \ssp(\zeit)$ is the time-$\zeit$
forward map of the initial fluid state  $\ssp(0)$.

In order to quantify whether two fluid states are close or far from each
other, one needs a notion of distance between two points in \statesp,
measured here as
%    \PC{ please read 2011-10-14 blog entries, respond. {\bf CS} I looked
%    up the full log history and don't find any 2011-10-14 blog entries.
%    {\bf Predrag} sorry, that comment was not addressed to you, forgot to
%    remove while copying into your blog }
\beq
  \Norm{\ssp-\ssp'}^2  = \braket{\ssp-\ssp'}{\ssp-\ssp'} =
%\frac{1}{V}
\int_\bCell \! d \bx \;
(\vec{u}-\vec{u}') \cdot (\vec{u}-\vec{u}')
\,.
\ee{innerproduct}
As there is no compelling reason to use this {`energy norm'}, other than
that velocity fields is what is given in a numerical computation, what
norm one uses depends very much on the application. In the study of
`optimal perturbations' that move a laminar solution to a turbulent one,
both energy\rf{TeHaHe10} and dissipation\rf{LoCaCoPeGo11} norms
have been used.  In our searches for \reqva\ and \rpo s
% (see \refsect{s:rpos})
we find it advantageous to use a `compensatory' norm.
% \refeq{compensNorm}.

Visualizations of trajectory \refeq{symbolicNS} are of necessity
projections onto two or three dimensions. A physically appealing
choice\rf{KawKida01} is to monitor the flow in terms of the
symmetry-invariant and physical energy, dissipation and power input
observables $(E(\zeit),D(\zeit),I(\zeit))$.
%as in \reffig{fig:M1Orb}\,(b).
If two fluid states are clearly separated in
such plot, they are also separated in the high-dimensional \statesp, but
converse is not true; physically distinct states might have comparable
energies, and such plots may obscure some of the most relevant features
of the flow. Furthermore, relations such as \refeq{EnRate}
% \refeq{Power=I-D}\CS{``??'',reference missing} depend on
detailed type and geometry of a given problem\rf{ksgreene88,SCD07},
and further physical observables beyond $(E(\zeit),D(\zeit),I(\zeit))$
are difficult to construct.
    \PC{add here references to \refeq{EnRate} derivations. Does
    Frisch\rf{frisch} do it?}

More straightforward are projections based on choices of several Fourier
components \refeq{KSdiscr}%\CS{reference missing}
or other primitive basis elements as
coordinate axes for projections of the flow. They are appropriate for
small perturbations off laminar state, but such coordinate axes are (i)
arbitrary and discretization-method dependent, and (ii) point in
unphysical directions, far from turbulent states which in a highly
nonlinear flows are characterized by  many strongly coupled Fourier modes
of comparable magnitude.

Recently, Gibson \etal\rf{GHCW07} have shown that the dynamics of different regions
of {\statesp} can be elucidated more profitably by a computationally
straight\-forward choice of \emph{physical} coordinates. One identifies
several prominent states of the flow $\bu_A$, $\bu_B$, $\dots$, such as
{\eqv} states and their eigenvectors, states in whose neighborhoods the
turbulent flow spends most of the time, and from them constructs, by
Gram-Schmidt or (anti)-symmetrizations, an orthonormal basis set
$\{\be_1, \be_2, \cdots, \be_n\}$. The evolving fluid state $\bu(\zeit)$
is then projected onto this basis using the inner product
\refeq{innerproduct},
\beq
\ssp(\zeit) =(\ssp_1, \ssp_2, \cdots, \ssp_n, \cdots)(\zeit)
    \,,\qquad
\ssp_n(\zeit) = \braket{\bu(\zeit)}{\be_n}
\,.
\ee{intrSspTraj}
This low-dimensional projection can be viewed in any of the $2D$ planes
$(\ssp_m, \ssp_n)$ or in $3D$ perspective views $(\ssp_{\ell},\ssp_m,
\ssp_n)$.
% see \reffig{f:MeanVelocityFrame}\CS{reference missing}.
The {\stateDsp} portraits are
{dynamically intrinsic}, since the projections are defined in terms of
intrinsic solutions of the equations of motion, and {representation
independent}, as the inner product \refeq{innerproduct} is independent of
the numerical discretization. It is worth emphasizing that the method
affords low-dimensional {\em visualization} without any low-dimensional
{\em modeling} or dimension reduction; the dynamics are computed with
fully-resolved direct numerical simulations. Although the use of
particular \reqva\ to define low-dimensional projections
% (see \refsect{s:eqbSols})\CS{``??'',reference missing}
may appear arbitrary, such a choice turns out to be
very useful when the turbulent flow is chaperoned by a few invariant
solutions and their unstable manifolds, as has been shown in other low
Reynolds number settings\rf{GHCW07}. Such visualizations are a
prerequisite to uncovering the interrelations between (the infinite
number of) invariant solutions, and constructing symbolic dynamics
partitions of \statesp\ needed for a systematic exploration of turbulent
dynamics, the key challenge that we address here for the case of turbulent
pipe flows.




\begin{description}

\item[2011-08-24 PC to Chao]
Please keep writing and rewriting in this section the draft of the
\KSe\ review to eventually include in your thesis.

% \item[2011-09-18 Predrag]  Not a line of the review has been written down
% yet, not even the \KSe?

\item[2011-10-27 PC]                                            \toCB
I've been pondering how to explain the
difference between using physical states to triangulate high-dimensional
\statesp s, and projections on a few primitive integration variables.
Maybe this helps.

\medskip

Chao showed me on the screen
his undocumented Matlab plots of \KS\ \rpo s in co-moving frames. He does
not use any variant of my $\{\be_a, \be_s, \be_m\}$ orthonormal
`physical' basis \refeq{ChaoFrame} in spite of my many attempts to
explain it.

\medskip

He plots \rpo s in `real space', meaning 3 components,
\[
\mbox{let's say } \{x_{47}, x_{48}, x_{49}\}
\,,
\]
of the 64-dimensional \KS\ discretization
vector that the integrator returns, with the mean phase speed per period
subtracted. \Rpo s do close into \po s. The shortest one looks like a
nice circle, longer ones are not a pretty sight. There is an FFT hidden in the integrator, but he
cannot plot Fourier modes. I think the problem can be understood
more clearly in the way he does it.

\medskip

It is as if a fluids experimentalist chose to look at 3 pixels of a video
image of one of a turbulent flow, and from that proceeded to reconstruct
the full turbulence. It cannot be done, even in principle; it's like
using a hot wire in a wind tunnel, in 2011. What I offer to instead to a
fluid dynamicist, experimentalist or theorist, is a full resolution $3D$
snapshot of a turbulent state (that is also a point in \statesp, but now
the basis vector goes straight through  it, and all pixels are retained),
and ask him to compare his measurement (a nearby point in \statesp) to my
snapshot. That is a powerful theory.

\medskip

Looking at a known \rpo\ is a bit misleading in this context. A circle
will close into a circle in any projection, as long the turbulent state
is global, as its \statesp\ trajectory generically has projection on all
of the $\infty$-many axes, but it is still a series of 3 pixels video.
You cannot work out the catalogue of turbulent states this way.

\medskip

Am I reaching you? If yes, can you put it in words that your colleagues
will understand? It is not about accepting superiority of Gibsonian
coordinates, it is about how to discuss this with fluid dynamicists. Even
Gibson does not understand this - since he has fallen back to his
fluid-dynamicst's roots, it's all about bifurcations, snaking and plotting
things in the parameter-energy plane.


\medskip

The whole process of communal learning and myths fascinates me. Often one
encounters group mental states where all agree on something that is
patently a nonsense. Not just string `theory', or climate change deniers.
Plumbers' `energy' has no meaning other than that numerical people can
plot it because their codes happen to spew out Eulerian velocities. Ask
an experimentalist to measure it... People click on a wiki, they exchange
a nod of agreement with a colleague and collectively psych themselves
into believing that they have understood. There are too many things, one
cannot learn them all, so communities work by consensus. It must have
been different 100 years ago when there were handful people interacting -
otherwise we still would not be using Cauchy's contour integrals and
quantum mechanics - those were inventions that really required exercising
neurons. Here we are picking axes in a linear vector space.

\end{description}
