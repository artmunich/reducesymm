\ifsvnmulti
\svnkwsave{$RepoFile: siminos/chao/dailyBlog.tex $}
\svnidlong {$HeadURL$}
{$LastChangedDate$}
{$LastChangedRevision$} {$LastChangedBy$}
\svnid{$Id$}
\fi

\chapter{Research blog}
\label{chap:blog}

$\footnotemark\footnotetext{{\tt \svnkw{RepoFile}}, rev. \svnfilerev:
 last edit by \svnFullAuthor{\svnfileauthor},
 \svnfilemonth/\svnfileday/\svnfileyear}$

\begin{description}

\item[Chao's Rules] On my
\HREF{http://chaosbook.org/~predrag/courses/PHYS-6124-10/RenminRibao.doc}
{Mao's boyscout honor} I pledge to add a substantial update on my
progress here at least twice a week, work incrementally and steadily on
drafts of my papers, and attend all physics colloquia, nonlinear physics,
math seminars, study groups, or my name be spelled the Italian way.
	\PC{the worry? Last blog entry is about Apr 19, today is July 18.
	``At least twice a week?''
		}

\item[2011-07-20 PC + CS] The above agreement has been solemnly signed
by Predrag and Chao, for the period starting after completing the qualifiers
Aug 22, and to Dec 31 2011.

\item[2011-09-15 PC] I propose an amendment to the above pledge:
``
I pledge to add a substantial update on my progress here at least twice a week
''
amended to
``
I pledge to add a substantial update (more substantial than
Predrag's weekly contributions to this blog) on my progress [...]
''

\noindent
[ ] {\bf 2011-09-15 Predrag} mark this box once you read the
proposed amendment and discussed them with me.

\item[2011-03-16 PC]
Revolution is not a dinner party, not an essay, nor a painting, nor a
piece of embroidery. \HREF{http://www.youtube.com/watch?v=BS3QOtbW4m0}
{The revolution will not be televised}, will not be televised, will
not be televised, will not be televised. The revolution will be no re-run
brothers; \HREF{http://www.gilscottheron.com/lyrevol.html}{The
revolution will be live.}

\item[2011-03-16 PC] Chao, please keep track of what goes on in
siminos/blog/ and siminos/lyapunov/ - you will be getting emails about
updates. There is some current excitement there concerning the `physical'
dimensionality of the \KS\ strange attractors.

\item[2011-03-17 CS to Ruslan, Stefan and Evangelos]
   Hi! Nice to meet you all. It is the first time I type something here.
   I am still reading Chapter 4. (and \rf{EllGaHoRa09}) I will catch up with
   you as soon as I can. Maybe I will have a lot of questions to consult
   you. Hope you won't bother:)

\item[2011-03-16 PC] Why reference to Ellis, Gay-Balmaz, Holm
                  and Ratiu\rf{EllGaHoRa09}?

\item[2011-03-22 PC to Chao]
This might be of interest to Adam, let him know about it:
\HREF{http://arxiv.org/abs/1103.3981}{arXiv:1103.3981},
\emph{Chains of rotational tori and filamentary structures close to high
 multiplicity periodic orbits in a 3D galactic potential},
 by Katsanikas, Patsis and Pinotsis.

They write
``
This paper discusses phase space structures encountered in the
neighborhood of periodic orbits with high order multiplicity in a 3D
autonomous Hamiltonian system with a potential of galactic type. We
consider 4D spaces of section and we use the method of color and rotation
[Patsis and Zachilas 1994] in order to visualize them. As examples we use
the case of two orbits, one 2-periodic and one 7-periodic. We investigate
the structure of multiple tori around them in the 4D surface of section
and in addition we study the orbital behavior in the neighborhood of the
corresponding simple unstable periodic orbits. By considering initially a
few consequents in the neighborhood of the orbits in both cases we find a
structure in the space of section, which is in direct correspondence with
what is observed in a resonance zone of a 2D autonomous Hamiltonian
system. However, in our 3D case we have instead of stability islands
rotational tori, while the chaotic zone connecting the points of the
unstable periodic orbit is replaced by filaments extending in 4D
following a smooth color variation. For more intersections, the
consequents of the orbit which started in the neighborhood of the
unstable periodic orbit, diffuse in phase space and form a cloud that
occupies a large volume surrounding the region containing the rotational
tori. In this cloud the colors of the points are mixed. The same
structures have been observed in the neighborhood of all m-periodic
orbits we have examined in the system. This indicates a generic behavior.
''

\item[2011-03-19 CS to PC]
What do you mean by "take the Hall out of library"?:) I actually do not
understand your last email.

\item[2011-03-25 PC]
Reference to Hall is in the {\em Lie police} section of Siminos
blog. It is good idea to keep reading updates of the three blogs - yours,
siminos/blog and siminos/lyapunov, they are all related to your work.

\item[2011-03-31 PC] Just curious (it is your blog, you do what you want with it):
    why did you undo the \texttt{svn-multi}? The reason why I installed
    it is that when you print paper copies of the blog it keeps track of
    the svn version, date of last edit of a given file. This is very
    useful when you have bunch of handwritten edits of earlier versions
    that you would like to keep track of. Evangelos has problems with
    system managers in France, so he introduced the switch
    \texttt{$\setminus$svnmultifalse}, which you can also comment out in
    blog.tex. But I think there should be no problem, so I have reverted
    to the version prior to your commenting out \texttt{svn-multi} lines.

\item[2011-03-31 CS] Sorry about that. I undo the \texttt{svn-multi}
because otherwise I can not compile the .tex file and generate pdf
document. I have not yet find another way to reconcile the
\texttt{svn-multi}. Before that I can make two versions, one with
\texttt{svn-multi} commented kept for my own use and one the same with
this for your convenience.

\item[2011-04-06 PC] Here is my configuration of \texttt{WinEdt}
(but that is a matter of taste)

\begin{verbatim}
[x] MikTeX 4.9 full distribution installation
[x] WinEdt, see www.cns.gatech.edu/CNS-only/WinEdt.txt
	configuration wizard, wrapping:
        [ ] disable wrapping
		[ ] remove TeX; from Conventional (Soft) Wrapping
		[ ] fixed right margin 74
		[ ] disable indent soft wrapping
	in preferences, disable
		[ ] Wrap Mode, and [ ] Line Wrapping Enabled
\end{verbatim}

In order to read \texttt{siminos/blog} you need to \texttt{LaTeX},
then \texttt{dvi -> ps} and then either read the \texttt{ps} file, or
convert \texttt{ps -> pdf}. This requires that you install these:

\begin{verbatim}
[ ] www.ghostscript.com
    [ ] gs901w32 -> C:\Program Files\gs\gs9.01\bin\gswin32.exe
[ ] ghostview pages.cs.wisc.edu/~ghost/gsview/
    manually changed 'execution modes' to
    [ ] C:\Program Files\Ghostgum\gsview\gsview32.exe
\end{verbatim}

\item[2011-09-22 PC] When you start editing, \emph{always} svn update
the repository from its home directory (in this case, siminos). If you
get a message that you have conflict in some file (red in kdesvn, letter "c"
if \texttt{svn up} in a terminal, red in tortoise), \emph{deal with it immediately}
by editing the conflicted file, and then telling the svn that the conflict is resolved.
Commit edits, the longer you wait, more likely you will run into conflicts.

At the moment you are propagating several conflicts. Search your blog for

\begin{verbatim}
<<<<<<< .mine
\end{verbatim}

and fix them.


\end{description}
