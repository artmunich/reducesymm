\ifsvnmulti
\svnkwsave{$RepoFile: siminos/chao/dailyBlog.tex $}
\svnidlong {$HeadURL$}
{$LastChangedDate$}
{$LastChangedRevision$} {$LastChangedBy$}
\svnid{$Id$}
\fi


\chapter{Research blog}
\label{chap:blog}

$\footnotemark\footnotetext{{\tt \svnkw{RepoFile}}, rev. \svnfilerev:
 last edit by \svnFullAuthor{\svnfileauthor},
 \svnfilemonth/\svnfileday/\svnfileyear}$


\begin{description}


\item[2011-03-16 PC to Ruslan, Stefan and Evangelos]
   Chao Shi,
\\ shichao116@gmail.com is starting ( rather late) to work on
   slicing and dicing Kuramoto Sivashinsky as a 1st year graduate student
   project. Hopefully we can get him to speed with your and Stefan's help
   within the next six months. At the moment he is reading parts of
   ChaosBook and our articles, but it might be wise that you teach him
   immediately how to use your KS programs and data, so we do not waste
   time on that. Chao can compute.

\item[2011-03-16 PC] Chao, please keep track of what goes on in
siminos/blog/ and siminos/lyapunov/ - you will be getting emails about
updates. There is some current excitement there concerning the `physical'
dimensionality of the \KS\ strange attractors.

\item[2011-03-17 CS to Ruslan, Stefan and Evangelos]
   Hi! Nice to meet you all. It is the first time I type something here.
   I am still reading Chapter 4. (and \rf{EllGaHoRa09}) I will catch up with
   you as soon as I can. Maybe I will have a lot of questions to consult
   you. Hope you won't bother:)

\item[2011-03-16 PC] Why reference to Ellis, Gay-Balmaz, Holm
                  and Ratiu\rf{EllGaHoRa09}?

%   Professor Cvitanovic, I still haven't found out why this blog cannot
%   display bibliography properly on my computer. But I think I will first
%   concentrate on those papers and come to deal with this later.

\item[2011-03-19 CS to PC]
%   Hi Professor Cvitanovic, I manage to make Latex work on my laptop
%   without actually knowing what actually happened, it works anyway. As I
%   tried one bibliography in last conversation.

   Yesterday I finished Chapter 4 but skipped the detail of the some
   examples. We discussed the first half of Chapter 4. Since there're
   pretty much mathematical detail and examples, we decide to break it
   into two parts. By the way, the notation of this chapter at the
   beginning is a little bit confusing and took me a while to get the
   concept of what those formulas really mean. I have some other
   questions about Chapter 4 but but now I am not used to typing math
   symbols here, I might ask you later directly or write it here if I get
   familiar with it.

\item[2011-03-21 PC] You have the dasbuch/book/chapter/*.tex source
code, you can just clip and paste formulas to here.

\item[2011-03-19 CS to PC]
Also I have a question about Example 3.2 in Chapter 3. It states that
   the choice of the coordinates of the pinball game are smart because
   they conserve the phase space volume. I don't understand this, would
   you mind explain it more specifically?

\item[2011-03-25 PC]
Glad you asked the question about choice of billiard coordinates. Please write up the solution to \refexer{ex_birkhoff}; it's worth doing it in class as well, a concrete example of how symplectic invariance preserves area for each $(q,p)$ dual coordinate  pair.

\item[2011-03-19 CS to PC]
Going from here, I am also
   wondering how to choose phase space coordinates? Does phase/state
   space coordinates have any requirement and whether conservation of
   space volume is such a requirement? What's the meaning of conservation
   of phase space requirement? In my understanding in Hamiltonian flows,
   conservation of phase space volume means conservation of energy, am I
   right?


\item[2011-03-25 PC]
Now, it is more subtle than that; time dependent flow can be symplectic, but energy is not conserved; I think Percival and D. Richards\rf{PerRich82N} (I have it in the
\HREF{http://www.cns.gatech.edu/CNS-only/LibraryCat2.htm} {CNS library}) discuss that well. Symplectic invariance is \emph{much stronger} requirement than either either energy conservation or phase-space volume conservation, see
\HREF{http://chaosbook.org/chapters/newton.pdf} {Section 7.4 Poincar\'e invariants} and \HREF{http://chaosbook.org/chapters/appendStability.pdf}
{Appendix D.4 Stability of Hamiltonian flows}:
symplectic transformations preserve area for each $(q,p)$ dual coordinate  pair.

\item[2011-03-22 PC to Chao]
This might be of interest to Adam, let him know about it:
\HREF{http://arxiv.org/abs/1103.3981}{arXiv:1103.3981},
\emph{Chains of rotational tori and filamentary structures close to high
 multiplicity periodic orbits in a 3D galactic potential},
 by Katsanikas, Patsis and Pinotsis.

They write
``
This paper discusses phase space structures encountered in the neighborhood of periodic orbits with high order multiplicity in a 3D autonomous Hamiltonian system with a potential of galactic type. We consider 4D spaces of section and we use the method of color and rotation [Patsis and Zachilas 1994] in order to visualize them. As examples we use the case of two orbits, one 2-periodic and one 7-periodic. We investigate the structure of multiple tori around them in the 4D surface of section and in addition we study the orbital behavior in the neighborhood of the corresponding simple unstable periodic orbits. By considering initially a few consequents in the neighborhood of the orbits in both cases we find a structure in the space of section, which is in direct correspondence with what is observed in a resonance zone of a 2D autonomous Hamiltonian system. However, in our 3D case we have instead of stability islands rotational tori, while the chaotic zone connecting the points of the unstable periodic orbit is replaced by filaments extending in 4D following a smooth color variation. For more intersections, the consequents of the orbit which started in the neighborhood of the unstable periodic orbit, diffuse in phase space and form a cloud that occupies a large volume surrounding the region containing the rotational tori. In this cloud the colors of the points are mixed. The same structures have been observed in the neighborhood of all m-periodic orbits we have examined in the system. This indicates a generic behavior.
''

\item[2011-03-19 CS to PC]
What do you mean by "take the Hall out of library"?:) I actually don't understand your last email.

\item[2011-03-25 PC]
Reference to Hall is in the {\em Lie police} section of Siminos
blog. It is good idea to keep reading updates of the three blogs - yours,
siminos/blog and siminos/lyapunov, they are all related to your work.

\item[2011-03-29 Ruslan]
Hi Chao.  If you want to use Matlab for your explorations of the \KS, then all my Matlab files can be found in /siminos/chao/matlab/ruslan.  File ksdupo.m is the primary file.  I'm using the cells feature of Matlab, so this file contains many sections that can be run independently.  There is not much in the way of comments, so you'll have to work it out for yourself.  I'll be happy to guide you through it if you ask specific questions.

\item[2011-03-30  CS to PC]
    Hi Professor Cvitanovi\'c, I am now reading Chapter 9. I have a question from the paragraph following the
   definition of free action: The splitting of a group \emph{G} into a stabilizer \emph{$G_{p}$} and \emph{m-1} coset
   \emph{$cG_{p}$} relates to an orbit \emph{$M_{p}$} to \emph{m-1}
   other distinct orbits \emph{$cM_{p}$}. All of them have equivalent stabilizers,
   or more precisely, the points on the same group orbit have
\emph{conjugate stabilizers}: \emph{$G_{cp} = cG_{p}c^{-1}$}.
   For the last sentence, does it mean that
   if \emph{$G_{p}$}  is a stabilizer of \emph{$M_{p}$},
   then \emph{$cG_{p}c^{-1}$} is a stabilizer of \emph{$cM_{p}$}?

\item[2011-03-31 PC] Yes, you are right. I have now incorporated ``if \emph{$G_{p}$}  is a stabilizer of \emph{$M_{p}$},
   then \emph{$cG_{p}c^{-1}$} is a stabilizer of \emph{$cM_{p}$}''
into discrete.tex, thanks.

I intend to excise the dreaded word
`stabilizer' from the text, just have forgotten to do it
\HREF{http://www.flickr.com/photos/birdtracks/4259634492/in/set-72157606259014811/}
{[click]} here. Suggestion - print out the chapter, replace by hand
word `stabilizer' everywhere by 'symmetry' and let's sit together and
see whether the chapter is  easier to read.

\item[2011-03-31 PC] Just curious (it's your blog, you do what you want with it):
    why did you undo the \texttt{svn-multi}? The reason why I
    installed it is that when you print paper copies of the blog
    it keeps track of the svn version, date of last edit of a given
    file. This is very useful when you have bunch of handwritten edits
    of earlier versions that you would like to keep track of.
    Evangelos has problems with system managers in France, so he
    introduced the switch \texttt{$\setminus$svnmultifalse}, which
    you can also comment out in blog.tex. But I think there should
    be no problem, so I have reverted to the version prior to your commenting out
    \texttt{svn-multi} lines.

\item[2011-03-31 CS] Sorry about that. I undo the \texttt{svn-multi}
because otherwise I can not compile the .tex file and generate pdf
document. I have not yet find another way to reconcile the
\texttt{svn-multi}. Before that I can make two versions, one with
\texttt{svn-multi} commented kept for my own use and one the same with
this for your convenience.

\item[2011-04-06 PC] Here is my configuration of \texttt{WinEdt}
(but that is a matter of taste)

\begin{verbatim}
[x] MikTeX 4.9 full distribution installation
[x] WinEdt, see www.cns.gatech.edu/CNS-only/WinEdt.txt
	configuration wizard, wrapping:
        [ ] disable wrapping
		[ ] remove TeX; from Conventional (Soft) Wrapping
		[ ] fixed right margin 74
		[ ] disable indent soft wrapping
	in preferences, disable
		[ ] Wrap Mode, and [ ] Line Wrapping Enabled
\end{verbatim}

In order to read \texttt{siminos/blog} you need to \texttt{LaTeX},
then \texttt{dvi -> ps} and then either read the \texttt{ps} file, or
convert \texttt{ps -> pdf}. This requires that you install these:

\begin{verbatim}
[ ] www.ghostscript.com
    [ ] gs901w32 -> C:\Program Files\gs\gs9.01\bin\gswin32.exe
[ ] ghostview pages.cs.wisc.edu/~ghost/gsview/
    manually changed 'execution modes' to
    [ ] C:\Program Files\Ghostgum\gsview\gsview32.exe
\end{verbatim}


\item[2011-04-05 CS] Worked out \refexer{ex_birkhoff}. Finished Chapter 9
with all the details in the examples, have had exemplified pictures of
symmetry and group actions.

\item[2011-04-06 PC] Not so fast - \refexer{ex_birkhoff} is not finished
until you write down your solution.

\item[2011-04-05 CS]
    A question about the last sentence in the first paragraph of Section
    5.4  discussed in the group study last week: Why does the
    neighborhood size scale as $1/|\Lambda_{p}|$? Wouldn't it scale as
    $|\Lambda_{p}|$?

\item[2011-04-06 PC] Mhm, clearly not written clearly enough, but
perhaps the most important property of an unstable flow
that one has to understand. The product of expanding multipliers
$|\Lambda_{p}|$ blows up exponentially with time, but the
\emph{neighborhood shrinks} exponentially with time, Detroit-like.
 Does looking
at Figure 5.1 help? Does reading Sect. 1.5.1 help? If you understand it,
can you rewrite

``																\toCB
Nearby points aligned along the stable
(contracting) directions  remain in the neighborhood of the
trajectory $\ssp(t)= \flow{t}{\xInit}$; the ones to keep an eye
on are the points which leave the neighborhood along the
unstable directions. The sub-volume $ |\pS_{\xInit}| = \prod_i^e\Delta
\ssp_i$ of the set of points which get no further away from
$\flow{t}{\xInit}$ than $L$, the typical size of the system, is
fixed by the condition that $\Delta \ssp_i \ExpaEig_i = O(L)$
in each expanding direction $i$. Hence the neighborhood size
scales as
$|\pS_{\xInit}| \propto O(L^{d_e})/|\ExpaEig_p| \propto 1/|\ExpaEig_p| $
where $\ExpaEig_p$ is the
product of expanding Floquet multipliers
(5.7) %\refeq{expVol}
only;
contracting ones play a secondary role.
''

so it makes sense to you. If you and Adam do not understand it, then
bring it up for discussion in the study group.

\item[2011-04-07 CS to PC] Got it.

\end{description}
