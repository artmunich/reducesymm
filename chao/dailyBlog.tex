\ifsvnmulti
\svnkwsave{$RepoFile: siminos/chao/dailyBlog.tex $}
\svnidlong {$HeadURL$}
{$LastChangedDate$}
{$LastChangedRevision$} {$LastChangedBy$}
\svnid{$Id$}
\fi


\chapter{Research blog}
\label{chap:blog}

$\footnotemark\footnotetext{{\tt \svnkw{RepoFile}}, rev. \svnfilerev:
 last edit by \svnFullAuthor{\svnfileauthor},
 \svnfilemonth/\svnfileday/\svnfileyear}$


\begin{description}


\item[2011-03-16 PC to Ruslan, Stefan and Evangelos]
   Chao Shi shichao116@gmail.com is starting ( rather late) to work on
   slicing and dicing Kuramoto Sivashinsky as a 1st year graduate student
   project. Hopefully we can get him to speed with your and Stefan's help
   within the next six months. At the moment he is reading parts of
   ChaosBook and our articles, but it might be wise that you teach him
   immediately how to use your KS programs and data, so we do not waste
   time on that. Chao can compute.

\item[2011-03-16 PC] Chao, please keep track of what goes on in
siminos/blog/ and siminos/lyapunov/ - you will be getting emails about
updates. There is some current excitement there concerning the `physical'
dimensionality of the \KS\ strange attractors.

\item[2011-03-17 CS to Ruslan, Stefan and Evangelos]
   Hi! Nice to meet you all. It is the first time I type something here.
   I am still reading Chapter 4. (and \refref{EllGaHoRa09}?) I will catch up with
   you as soon as I can. Maybe I will have a lot of questions to consult
   you. Hope you won't bother:)

\item[2011-03-16 PC] Why reference to Ellis, Gay-Balmaz, Holm
                  and Ratiu\rf{EllGaHoRa09}?

%   Professor Cvitanovic, I still haven't found out why this blog cannot
%   display bibliography properly on my computer. But I think I will first
%   concentrate on those papers and come to deal with this later.

\item[2011-03-19 CS to PC]
%   Hi Professor Cvitanovic, I manage to make Latex work on my laptop
%   without actually knowing what actually happened, it works anyway. As I
%   tried one bibliography in last conversation.

   Yesterday I finished Chapter 4 but skipped the detail of the some
   examples. We discussed the first half of Chapter 4. Since there're
   pretty much mathematical detail and examples, we decide to break it
   into two parts. By the way, the notation of this chapter at the
   beginning is a little bit confusing and took me a while to get the
   concept of what those formulas really mean. I have some other
   questions about Chapter 4 but but now I am not used to typing math
   symbols here, I might ask you later directly or write it here if I get
   familiar with it.

\item[2011-03-21 PC] You have the dasbuch/book/chapter/*.tex source
code, you can just clip and paste formulas to here.

\item[2011-03-19 CS to PC]
Also I have a question about Example 3.2 in Chapter 3. It states that
   the choice of the coordinates of the pinball game are smart because
   they conserve the phase space volume. I don't understand this, would
   you mind explain it more specifically? Going from here, I am also
   wondering how to choose phase space coordinates? Does phase/state
   space coordinates have any requirement and whether conservation of
   space volume is such a requirement? What's the meaning of conservation
   of phase space requirement? In my understanding in Hamiltonian flows,
   conservation of phase space volume means conservation of energy, am I
   right? Thanks!



\end{description}
