\ifsvnmulti
\svnkwsave{$RepoFile: siminos/chao/dailyBlog.tex $}
\svnidlong {$HeadURL$}
{$LastChangedDate$}
{$LastChangedRevision$} {$LastChangedBy$}
\svnid{$Id$}
\fi

\chapter{Research blog}
\label{chap:blog}

$\footnotemark\footnotetext{{\tt \svnkw{RepoFile}}, rev. \svnfilerev:
 last edit by \svnFullAuthor{\svnfileauthor},
 \svnfilemonth/\svnfileday/\svnfileyear}$

\begin{description}

\item[Chao's Rules] On my
\HREF{http://chaosbook.org/~predrag/courses/PHYS-6124-10/RenminRibao.doc}
{Mao's boyscout honor} I pledge to add a substantial update on my
progress (- 2011-09-26, 2011-10-04, 2011-10-07)
here at least twice a week (nothing 2011-09-19 to 2011-10-07),
work incrementally and steadily on
drafts of my papers, and attend all
physics colloquia (- 2011-09-28),
nonlinear physics  (- 2011-10-07, 2011-10-19),
math seminars (none by 2011-10-07),
study groups (- 2011-09-26),
or my name be spelled the Italian way.
	\PC{the worry? Last blog entry is about Apr 19, today is July 18.
	``At least twice a week?''
		}

\item[2011-07-20 PC + CS] The above agreement has been solemnly signed
by Predrag and Chao, for the period starting after completing the qualifiers
Aug 22, and to Dec 31 2011.

\item[2011-09-15 PC] I propose an amendment to the above pledge:
``
I pledge to add a substantial update on my progress here at least twice a week
''
amended to
``
I pledge to add a substantial update (more substantial than
Predrag's weekly contributions to this blog) on my progress [...]
''

\noindent
[ ] {\bf 2011-09-15 Predrag} mark this box once you read the
proposed amendment and discussed them with me.

\item[2011-03-16 PC]
Revolution is not a dinner party, not an essay, nor a painting, nor a
piece of embroidery. \HREF{http://www.youtube.com/watch?v=BS3QOtbW4m0}
{The revolution will not be televised}, will not be televised, will
not be televised, will not be televised. The revolution will be no re-run
brothers; \HREF{http://www.gilscottheron.com/lyrevol.html}{The
revolution will be live.}

\item[2011-03-16 PC] Chao, please keep track of what goes on in
siminos/blog/ and siminos/lyapunov/ - you will be getting emails about
updates. There is some current excitement there concerning the `physical'
dimensionality of the \KS\ strange attractors.

\item[2011-03-17 CS to Ruslan, Stefan and Evangelos]
   Hi! Nice to meet you all. It is the first time I type something here.
   I am still reading Chapter 4. (and \rf{EllGaHoRa09}) I will catch up with
   you as soon as I can. Maybe I will have a lot of questions to consult
   you. Hope you won't bother:)

\item[2011-03-16 PC] Why reference to Ellis, Gay-Balmaz, Holm
                  and Ratiu\rf{EllGaHoRa09}?

\item[2011-03-22 PC to Chao]
This might be of interest to Adam, let him know about it:
\HREF{http://arxiv.org/abs/1103.3981}{arXiv:1103.3981},
\emph{Chains of rotational tori and filamentary structures close to high
 multiplicity periodic orbits in a 3D galactic potential},
 by Katsanikas, Patsis and Pinotsis.

They write
``
This paper discusses phase space structures encountered in the
neighborhood of periodic orbits with high order multiplicity in a 3D
autonomous Hamiltonian system with a potential of galactic type. We
consider 4D spaces of section and we use the method of color and rotation
[Patsis and Zachilas 1994] in order to visualize them. As examples we use
the case of two orbits, one 2-periodic and one 7-periodic. We investigate
the structure of multiple tori around them in the 4D surface of section
and in addition we study the orbital behavior in the neighborhood of the
corresponding simple unstable periodic orbits. By considering initially a
few consequents in the neighborhood of the orbits in both cases we find a
structure in the space of section, which is in direct correspondence with
what is observed in a resonance zone of a 2D autonomous Hamiltonian
system. However, in our 3D case we have instead of stability islands
rotational tori, while the chaotic zone connecting the points of the
unstable periodic orbit is replaced by filaments extending in 4D
following a smooth color variation. For more intersections, the
consequents of the orbit which started in the neighborhood of the
unstable periodic orbit, diffuse in phase space and form a cloud that
occupies a large volume surrounding the region containing the rotational
tori. In this cloud the colors of the points are mixed. The same
structures have been observed in the neighborhood of all m-periodic
orbits we have examined in the system. This indicates a generic behavior.
''

\item[2011-03-19 CS to PC]
What do you mean by "take the Hall out of library"?:) I actually do not
understand your last email.

\item[2011-03-25 PC]
Reference to Hall is in the {\em Lie police} section of Siminos
blog. It is good idea to keep reading updates of the three blogs - yours,
siminos/blog and siminos/lyapunov, they are all related to your work.

\item[2011-03-31 PC] Just curious (it is your blog, you do what you want with it):
    why did you undo the \texttt{svn-multi}? The reason why I installed
    it is that when you print paper copies of the blog it keeps track of
    the svn version, date of last edit of a given file. This is very
    useful when you have bunch of handwritten edits of earlier versions
    that you would like to keep track of. Evangelos has problems with
    system managers in France, so he introduced the switch
    \texttt{$\setminus$svnmultifalse}, which you can also comment out in
    blog.tex. But I think there should be no problem, so I have reverted
    to the version prior to your commenting out \texttt{svn-multi} lines.

\item[2011-03-31 CS] Sorry about that. I undo the \texttt{svn-multi}
because otherwise I can not compile the .tex file and generate pdf
document. I have not yet find another way to reconcile the
\texttt{svn-multi}. Before that I can make two versions, one with
\texttt{svn-multi} commented kept for my own use and one the same with
this for your convenience.

\item[2011-04-06 PC] Here is my configuration of \texttt{WinEdt}
(but that is a matter of taste)

\begin{verbatim}
[x] MikTeX 4.9 full distribution installation
[x] WinEdt, see www.cns.gatech.edu/CNS-only/WinEdt.txt
	configuration wizard, wrapping:
        [ ] disable wrapping
		[ ] remove TeX; from Conventional (Soft) Wrapping
		[ ] fixed right margin 74
		[ ] disable indent soft wrapping
	in preferences, disable
		[ ] Wrap Mode, and [ ] Line Wrapping Enabled
\end{verbatim}

In order to read \texttt{siminos/blog} you need to \texttt{LaTeX},
then \texttt{dvi -> ps} and then either read the \texttt{ps} file, or
convert \texttt{ps -> pdf}. This requires that you install these:

\begin{verbatim}
[ ] www.ghostscript.com
    [ ] gs901w32 -> C:\Program Files\gs\gs9.01\bin\gswin32.exe
[ ] ghostview pages.cs.wisc.edu/~ghost/gsview/
    manually changed 'execution modes' to
    [ ] C:\Program Files\Ghostgum\gsview\gsview32.exe
\end{verbatim}

\item[2011-09-22 PC] When you start editing, \emph{always} svn update
the repository from its home directory (in this case, siminos). If you
get a message that you have conflict in some file (red in kdesvn, letter "c"
if \texttt{svn up} in a terminal, red in tortoise), \emph{deal with it immediately}
by editing the conflicted file, and then telling the svn that the conflict is resolved.
Commit edits, the longer you wait, more likely you will run into conflicts.

\item[2011-10-11 Predrag and Chao]
Had a long chat. Predrag said that he does not understand what Chao does
with his time - on his RA he has no obligations other than science, while
Predrag also teaches, grades problem sets, writes papers, proposals,
organizes seminars, colloquia \emph{und so weiter}. As an experiment,
Predrag stopped writing Chaos's blog on Sept 19 and nothing (in terms of
research progress, literature, things learned) has been written into it
since. Predrag had expected to have a rough draft of Davidchack et al.
paper II, with KS dynamics sliced and the infamous 40,000 (60,000?) \rpo
s organized in happy families. Predrag believes that starting computing
should be very easy, as number of people, including Predrag's
undergraduate nonlinear dynamics class, have learned it in matter of
weeks (including Predrag, working with Pukaradze and Christiansen), and
here everything is set up. Nothing that we had agreed on (see
\refpage{chap:blog}) has been carried out.

Chao explained that he feels that he must understand what he is doing,
and that he has read many papers. He has downloaded a book on
infinite-dimensional systems that seems to have a good chapter on \KS.
But he should organize his time differently, and spend 1/2 on computing,
and 1/2 on reading. This was followed by much embarrassing Predrag's
windbaggery, who even dragged in dead Feynman as a possible model on how
to do science.

Predrag's 2007 GaTech undergraduate nonlinear dynamics class
Kuramoto-Sivashinsky reports:
\HREF{http://chaosbook.org/extras/KSEproject/KSEbiffs.html}{1. A fishing expedition},
\HREF{http://chaosbook.org/extras/KSEproject/ks_sim.html}{2. Flickering flame front}.


\item[2011-10-12 Chao 2 Predrag] I plan to write my own code if I can't demystify
Simino's code in short time.

\item[2011-10-12 Predrag 2 Ruslan and Evangelos]
What do you think?

My personal feeling it would be a waste of time, as you both have well developed
codes that work and
it would be much more economical to add to them than start from the scratch.

\item[2011-10-12 Ruslan]
I agree that it's a waste of time.  I don't know about Evangelos's code,
but my Matlab code shouldn't be hard to use.  It's fairly modular and
there are just a few key function files, which most of the time can
be used as black boxes.

For initial exploration, Matlab is preferable since the code is smaller
and visualization of results directly available, but, as a rule of thumb,
if it takes about 1 hour for Matlab code to compute the result,
then it's time to start thinking about moving to Fortran or C.

Seems to me that Evangelos didn't find it hard to start changing my Matlab
functions once I explained a few things in the README file last week.
Of course I'm partial, but with Matlab you don't have to spend time worrying
about compiles, makefiles, etc.

Of course, if Chao doesn't know Matlab, it might appear difficult
for him to start using it.

\item[2011-10-12 Evangelos]
I knew virtually no Matlab when I started using Ruslan's code
- so it should be no problem for Chao to learn how to use it. My code
is written in Fortran 90 and is therefore a bit harder to compile and
use (also the user interface is not great). For what Chao is expected
to work on in the near future there should be no serious performance
penalty. So I think Ruslan's Matlab code should be the first think to
master, then if there is need for Fortran or C code he can either try
to use my code or write his own.

\item[2011-10-12 Predrag]
To summarize, continuing on what we agreed on \refpage{chap:blog}:
We advise Chao to get Ruslan's Matlab simulations running, and by the
end of this month discuss how far we have gotten, and
whether it is realistic to continue with this project.

\item[2011-10-21 PC] Sorry, I thought I had given you the reference (link
is probably on ChaosBook.org/projects homepage). Ruslan L. Davidchack's 3
Sep 2006 KS \eqva\ discussion was in siminos/blog/, but I had not copied
to your blog. Here it is:

\HREF{http://www.math.le.ac.uk/people/rld8/temp/kse22explore.html}
{www.math.le.ac.uk/people/rld8/temp/kse22explore.html}

\item[2011-10-27 Predrag and Chao] (notes written 4am on 2011-10-28). Had
a long chat, the planned follow up on the entries of {\bf 2011-10-11} and
{\bf 2011-10-12} above. Predrag said that he does not believe Chao can do
the work that is needed in this project. Nothing (in terms of research
progress, literature, things learned) has been written into this blog
since. There is nothing that could be used in the planned paper
concerning the theory, equations, \etc. Mathematical physics needed for
this project requires concentrated, precise work, sloppy is not
acceptable. Nothing that we had agreed on (see \refpage{chap:blog}) has
been carried out. He cannot support him on his grant, and advises him to
talk to graduate coordinator and look for another adviser.

Chao protested that he has worked hard this month, has studied
'Davidchack paper'\rf{SCD07} in depth, and showed Predrag his handwritten
notes. The notes were extensive, but only text, no hand calculations. He
showed Predrag on the screen his undocumented Ruslan's Matlab plots of
\rpo s in co-moving frame. He does not use any variant $\{\be_a, \be_s,
\be_m\}$ a orthonormal `physical' basis \refeq{ChaoFrame}, see {\bf
2011-10-19} above (which Predrag still suspects Chao does not
understand, until there are plots that show the contrary).

Chao plots \rpo s in `real space', meaning 3 components (let's
say $\{x_{47}, x_{48}, x_{49}\}$) of the 64-dimensional vector that
integrator returns, with the mean phase speed per period subtracted. \Rpo
s do close into \po s, but with error visible by eye that Chao blames on
Ruslan's interpolation (see {\bf 2011-10-24 Chao to Ruslan, Predrag}
above). The shortest one looks like a nice circle, longer ones are not a
pretty sight. For Predrag this is \emph{the} reason why daily research
must be documented - it enables the advisor to intervene if the project
has gone off the rails.

Chao asks to be given a chance to prove that he can do the work. His
proposal is to be a TA coming semester (he asked whether Predrag could
intervene with the graduate committee so TA obligations are not too
demanding - but that is out). Predrag advises him not to do this, but if
he does, Predrag expects to have a draft of Davidchack \etal\ paper
II, with KS dynamics sliced and the infamous 40,000 (60,000?) \rpo s
organized in happy families, ready by April 2012.

\item[2011-10-28 5am Predrag to Chao]
We better have another talk - the more I think about it the less I see
the mechanism that would make Chao learn, compute and document his
research the coming months any more than during this October.


\end{description}
