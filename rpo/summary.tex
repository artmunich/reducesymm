% summary.tex
%
% Predrag					jun 20 2006
% Predrag					dec  5 2004
% $Author$ $Date$



\section{Summary}
% \section{Conclusions}
\label{sect:rpo-sum}


We have presented a periodic orbit investigation
of the spatiotemporal dynamics of the
{\KSe}.

We used
the {\rpo}s as a probe to explore the topology and chaotic dynamics.

Steady solutions are important because they set up 
a coarse description of 
typical phase space motions.

In a long run, the hope is that the periodic orbit theory
can be applied to real-world
problems, such as moderately turbulent Navier-Stokes
flows, and use calculated results to match or predict
experimental data, or to check and modify the assumptions underlying specific
turbulence models.


At first glance, turbulant dynamics visualized in the state space might appear
hopelessly complex, but under a detailed examination it appears 
much less so than feared: it is
pieced together from essentially {1-$d$ return maps}
connected by fast transient interludes.
{\KS} and \PCf\  equilibrium, travelling and 
periodic solutions embody Hopf's vision:
repertoire of recurrent spatio-temporal
patterns explored by turbulent dynamics.


