% symm.tex
%
% Predrag created file				jul 3 2006
% $Author$ $Date$

\subsection{Scaling and symmetries}
% \section{Symmetries of \KSe}
% Predrag extracted from newton.tex 		jul  3 2006


The \KSe\ \refeq{ks} is space translationally invariant,
time translationally invariant, and invariant under
reflection
$x \to -x$, 
$u \to -u$. 

Comparing $u_t$ and $(u^2)_x$ terms we note that $u$ has
dimensions of $[x]/[t]$, so it would be more correct to
refer to it as the ``velocity'' rather than the 
``height'' of the flame front. Indeed, the  \KSe\ is
Galilean invariant: if $u(x,t)$ is a solution, then 
$v+u(x+2vt,t)$, with $v$ an arbitrary constant velocity, is also a solution. 
Without loss of generality, in our calculations we shall set 
% \index{Galilean invariance}
% \index{invariance!Galilean}
\beq
\int dx \, u = 0
\,.
\ee{GalInv}

In terms of the system size $L$, the only length scale available,
the dimensions of terms in \refeq{ks} are
$ %\[
[x]=L
$, $%\,,\quad
[t]=L^2
$, $%\,,\quad
[u]=L^{-1}
$, $%\,,\quad
[\nu]=L^2
\,.
$ %\]
What is the non-dimensional ``Rayleigh'' number for the
\KS\ system? 
 Scaling out the ``viscosity'' $\nu$ 
\[ 
x \to x \nu^{\frac{1}{2}}
\,,\quad
t \to t \nu
\,,\quad
u \to u \nu^{-\frac{1}{2}}
\,,
\]
brings the \KSe\ \refeq{ks}
to a non-dimensional form
\beq
u_t=(u^2)_x-u_{xx}- u_{xxxx}
\,,\qquad	
	x \in  [0,L\nu^{-\frac{1}{2}}] = [0,2\pi\tilde{L}]
\,.
\ee{ks-L}
In this way we trade in the ``viscosity'' $\nu$
and the system size $L$ for a single
dimensionless system size parameter
\beq
	\tilde{L}={L}/{(2 \pi \sqrt{\nu})}
%	\tilde{L}=\frac{L}{2 \pi \sqrt{\nu}}
%	\,,
\ee{tildeL}
which plays the role of a ``Reynolds number''
for the \KS\ system.

In the literature sometimes 
$L$ is used as the system parameter, with $\nu$ fixed to $1$, and
at other times $\nu$ is varied with $L$ fixed to either $1$ or $2\pi$.
To minimize confusion,
in what follows we shall state results of all 
calculations in units of dimensionless system size $\tilde{L}$.
\PC{motivate $2\pi$ factor by the mean wavelength,
    refer to the equation number}
Note that the time units also have to be
rescaled; if $T^*_p$ is a period
of a periodic solution of \refeq{ks} with a given
$\nu$ and $L=2\pi$, then
the corresponding solution of the non-dimensionalized \refeq{ks-L}
has period 
\beq
 T_p= T^*_p/\nu
\ee{Trescl}

The full KSE has a continuous symmetry: if
$u(x,t)$ is a solution, then so is $u(x+d,t)$ for any
$0 < d \leq L$.  As a result,
the KS has \rpo s with nonzero shift $-L/2 < d \leq L/2$
\[ u(x+d,\period{}) = u(x,0)
\,.
\]
where $\period{}$ is the period. 

Further down the road: we need to add this
$\mathbf{C} u(x,t) = -u(-x,t)$ symmetry
 to the continuous $O(1)$ rotation; then some of the \rpo s will
 halve their period, and symmetric pairs will be eliminated.

Fourier coefficients which respect the $x \to -x$ symmetry of
\KSe, see discussion in \refref{Christiansen:97},
and references therein.

% By symmetry there might be an equilibrium on the reflection plane that
% relates the equilibrium A and its symmetry partner SA; the 3 equilibria would
% be analogous to what you see in the Lorentz attractor pictures, crossing
% the unstable manifold of the central one throws you into the neighborhood
% of the other equilibrium.

% For discrete rotations the spectral determinants factorize
% nicely in terms of rpo's:
% read ``Discrete symmetries" chapter of ChaosBook.org - not an easy read, but
% it also uses $g\jMps$ rather than the naked $\jMps$,
% and a trace formula for irrational
% $d/L$ still puzzles me - for $d/L$ rational the determinant factorizes using
% discrete Fourier transform.
% We are fuzzy on the continuum limit of that.

