% KSe.tex
% Predrag extracted from newton.tex 		jul  3 2006
% Predrag					jun 20 2006
% Vaggelis					may 20 2006
% $Author$ $Date$

\section{\KSe\ according to Predrag}
\label{s-KS}
% Predrag 					 4jul2006
% extracted from ~dasbuch/book/chapter/PDEs.tex  5jun2005 version
% Predrag               1 jan 2000
% Predrag              17 sep 99

%  remember to incorporate missing refs from {chapter/refsPDEs}

The \KS\ system\rf{ku,siv},
arising in the description of the flame front flutter of  gas burning in
a cylindrically symmetric burner on your kitchen stove,
and many other problems of greater import,
is one of the simplest partial differential equations that
exhibit turbulence.
The time evolution of the ``height of the flame front'' 
is given by
\beq
u_t=(u^2)_x-u_{xx}- u_{xxxx}
\,,\qquad	x \in [0,L]
\,.
\ee{ks}
In this equation $t \geq 0$ is the time and
$x$ is the spatial coordinate.
The subscripts $x$ and $t$ denote partial derivatives with respect to
$x$ and $t$;
$u_t = du/dt$, $u_{xxxx}$ stands for the 4th spatial
derivative of 
$u=u(x,t)$ at position $x$ and time $t$.
% The ``viscosity'' parameter 
% $\nu$ controls the 
% suppression of solutions with fast spatial variations.
% We take note, as in the Navier-Stokes equation, of the
% $u {\partial_x} u$ ``inertial'' term, the $ {\partial_x^2 } u$
% ``diffusive'' term (both with a ``wrong'' sign), etc.

% The term $(u^2)_x$ makes this a {\em nonlinear system}.
% It is one of the
% simplest conceivable nonlinear PDE, playing
% the role in the theory of spatially extended systems
% analogous to the role that
% the $x^2$ nonlinearity
% % \refeq{LogisMap}
% plays in the dynamics of iterated mappings.
% The salient feature of such
% partial differential equations is a theorem saying that
% for any finite value of the phase-space contraction
% parameter $\nu$,  the asymptotic dynamics is
% describable by a {\em finite} set of ``inertial manifold''
% ordinary differential equations. %cite{Foias88}.


\PublicPrivate{%
		}{% switch to Private:
\subsection{Heteroclinic and homoclinic connections}
% from \Chapter{KS}{12Jun2004}{Steady solutions of KS equation}
% Lan                                            8Jun2004

KSe\ \refeq{eq:stdks} has an exact heteroclinic solution
\beq
u=a_1 \tanh(kx)+a_2 \tanh^3(kx)
\label{eq:ksexa} 
\,,
\eeq
for the specific value $c=-\frac{30k^2}{19}(304k^4-40k^2+1)$ where 
\[
a_1=60k^3-\frac{30k}{19}\,,\;a_2=-60k^3\,,\;k^2=\frac{11}{76} \mbox{ or } 
k^2=-\frac{1}{76}
\,.
\]
When $k^2=-1/76$, the hyperbolic tangent becomes ordinary tangent and there are poles on
the real axis. Near the singularity, $u$ goes like
like 
\[
u(x) \approx \frac{-60}{(x-x_0)^3}\,,\; x \to x_0 \,.
\]
All the blow-up solutions possess singularities of the same type\rf{ksham95}.
In general, the existence and structure of connections depend on parameter
$c$ in a complicated way.
		}% end \PublicPrivate



In order to find a better representation of the dynamics, we now
turn to its topological invariants.

