% KSe.tex
% Predrag extracted from newton.tex 		jul  3 2006
% Predrag					jun 20 2006
% Vaggelis					may 20 2006
% $Author$ $Date$

\section{\KSe\ according to Evangelos}

The \KSe\ % (KSe)
reads:
 \beq
  u_t=(u^2)_x-u_{xx}- u_{xxxx} \, ,
  \label{eq:KS}
 \eeq

 We assume periodic boundary conditions on the $x\in [0,2\pi \tilde{L}]$
 interval:
 \beq
   u(x+2\pi\tilde{L},t)=u(x,t) \, ,
 \eeq
 which allows a Fourier series expansion:
 \beq
  u(x,t)=\sum_{k=-\infty}^{+\infty} a_k (t) e^{ i k x / \tilde{L}} \, .
  \label{eq:Fourier}
 \eeq
 Since $u(x,t)$ is real,
 \beq
  a_{k}=a^*_{-k} \, .
  \label{eq:a*}
 \eeq
 Substituting \refeq{eq:Fourier} into \refeq{eq:KS} we get:
 \beq
  \dot{a}_k=(k/\tildeL)^2\left(1-(1/\tildeL)^2 k^2\right)a_k
        + i (k/\tildeL)  \sum_{m=-\infty}^{+\infty}a_m a_{k-m} \, .
  \label{eq:Fcoef}
 \eeq

 From \refeq{eq:Fcoef} we notice that $\dot{a}_0=0$ and thus $a_0$ is an integral
 of the equations or, from \refeq{eq:Fourier}, the average of the solution $\int dx u(x,t)$
 is a constant. Due to galilean invariance we may set $a_0=0$ without loss of generality 
 and we only have to compute $a_k$'s with $k\geq 1$. % Explain this in detail somewhere.

 Truncating the infinite tower of equations by setting $a_k=0$ for $k>d$, using the identity $a_{-k}=a^*_k$ and splitting the
 resulting equations into real and imaginary part by setting $a_k=b_k+i c_k$, we have
  
 \bea
  \dot{b}_k & = & \left(\frac{k}{\tildeL}\right)^2\left(1- \left(k/\tildeL\right)^2 \right)b_k  \continue
	& & - \frac{k}{\tildeL} \left(\sum_{m=1}^{k-1}c_m b_{k-m}+\sum_{m=k+1}^{N}c_m b_{m-k}
                    -\sum_{m=1}^{N-k}c_m b_{k+m} \right)  \continue
	& & - \frac{k}{\tildeL} \left(\sum_{m=1}^{k-1}b_m c_{k-m}-\sum_{m=k+1}^{N}b_m c_{m-k}
                    +\sum_{m=1}^{N-k}b_m c_{k+m} \right)		  
  \label{eq:tmp:b-Trunc}
 \eea
 \bea
   \dot{c}_k & = & \left(\frac{k}{\tildeL}\right)^2\left(1- \left(k/\tildeL\right)^2 \right)c_k  \continue
	& & - \frac{k}{\tildeL}\left( \sum_{m=1}^{k-1}c_m c_{k-m}-\sum_{m=k+1}^{N}c_m c_{m-k}
                    -\sum_{m=1}^{N-k}c_m c_{k+m} \right)	\continue
	& & + \frac{k}{\tildeL} \left(\sum_{m=1}^{k-1}b_m b_{k-m}+\sum_{m=k+1}^{N}b_m b_{m-k}
                    +\sum_{m=1}^{N-k}b_m b_{k+m} \right)
   \label{eq:tmp:c-Trunc}
 \eea
 where now only terms $c_{k},b_{k}$ with $0<k<d$ appear. Observe
 \beq
	\sum_{m=1}^{N-k}c_m b_{k+m} = \sum_{m=k+1}^{N}b_m c_{m-k}\,,
 \eeq
 \etc and thus \refeq{eq:tmp:b-Trunc} and \refeq{eq:tmp:c-Trunc} simplify to
  \bea
  \dot{b}_k & = & \left(\frac{k}{\tildeL}\right)^2\left(1- \left(k/\tildeL\right)^2 \right)b_k  \continue
	& & - \frac{k}{\tildeL} \left(\sum_{m=1}^{k-1}c_m b_{k-m}-2\sum_{m=1}^{N-k}c_m b_{k+m} \right)  \continue
	& & - \frac{k}{\tildeL} \left(\sum_{m=1}^{k-1}b_m c_{k-m}+2\sum_{m=1}^{N-k}b_m c_{k+m} \right)		  
  \label{eq:b-Trunc}
 \eea
 \bea
   \dot{c}_k & = & \left(\frac{k}{\tildeL}\right)^2\left(1- \left(k/\tildeL\right)^2 \right)c_k  \continue
	& & - \frac{k}{\tildeL}\left( \sum_{m=1}^{k-1}c_m c_{k-m}-2\sum_{m=1}^{N-k}c_m c_{k+m} \right)	\continue
	& &  +\frac{k}{\tildeL}\left( \sum_{m=1}^{k-1}b_m b_{k-m}+2\sum_{m=1}^{N-k}b_m b_{k+m} \right)\,.
   \label{eq:c-Trunc}
 \eea

 We begin by calculating the matrix of variations $A_{ij} \equiv \frac{\partial v_i(x)}{\partial x_j}$ for the antisymmetric
 subspace for which $b_k=0, c_{-k}=-c_{k}$ and thus
 \beq
	   \dot{c}_k =  \left(\frac{k}{\tildeL}\right)^2\left(1- \left(k/\tildeL\right)^2 \right)c_k
	 		- \frac{k}{\tildeL}\left( \sum_{m=1}^{k-1}c_m c_{k-m}
                    		-2\sum_{m=1}^{N-k}c_m c_{k+m} \right)	\,.	
 \eeq
 
 Then
 \bea
	\frac{\partial \dot{c}_k}{\partial c_{j}}  =  
		\left(\frac{k}{\tildeL}\right)^2\left(1- \left(k/\tildeL\right)^2 \right) \delta_{kj}
			- \frac{k}{\tildeL}\frac{\partial}{\partial c_j}\left( \sum_{m=1}^{k-1}c_m c_{k-m}-2\sum_{m=1}^{N-k}c_m c_{k+m} \right)	\,.	
 \eea
 Concider the second term:
 \bea
	- \frac{k}{\tildeL}\frac{\partial}{\partial c_j}\left( \sum_{m=1}^{k-1}c_m c_{k-m}-2\sum_{m=1}^{N-k}c_m c_{k+m} \right)	& = &
		- \frac{k}{\tildeL} \sum_{m=1}^{k-1} \left(\delta_{m,j} c_{k-m}+c_m \delta_{k-m,j} \right) \continue
						& & + 2 \frac{k}{\tildeL}\sum_{m=1}^{N-k} \left(\delta_{m,j} c_{k+m}+c_m \delta_{k+m,j}\right)
 \eea
 We need to consider two cases separately:
 \begin{itemize} 
	\item $k\leq j$
		\bea
			 -\frac{k}{\tildeL}\frac{\partial}{\partial c_j}\left( \sum_{m=1}^{k-1}c_m c_{k-m}-2\sum_{m=1}^{N-k}c_m c_{k+m} \right)	& = &
					-\frac{k}{\tildeL}( 0+0 ) + 2\frac{k}{\tildeL} (c_{k+j} + c_{j-k}) \continue
				& = &   2 \frac{k}{\tildeL} (c_{k+j}-c_{k-j})
		\eea
	\item $k > j$
		\bea
			 -\frac{k}{\tildeL}\frac{\partial}{\partial c_j}\left( \sum_{m=1}^{k-1}c_m c_{k-m}-2\sum_{m=1}^{N-k}c_m c_{k+m} \right)	& = &
					-\frac{k}{\tildeL}(c_{k-j} + c_{k-j} ) + 2\frac{k}{\tildeL} (c_{k+j}  + 0 ) \continue
				& = &  2 \frac{k}{\tildeL} (c_{k+j}-c_{k-j})
		\eea	
 \end{itemize}
 and thus
 \beq
	\frac{\partial \dot{c}_k}{\partial c_{j}} =  \left(\frac{k}{\tildeL}\right)^2\left(1- \left(k/\tildeL\right)^2 \right) + 2 \frac{k}{\tildeL} (c_{k+j}-c_{k-j})
 \eeq

 For the case of the full space we need to consider the four matrices $\frac{\partial \dot{b}_k}{\partial b_j},\frac{\partial \dot{b}_k}{\partial c_j},\frac{\partial \dot{c}_k}{\partial b_j},\frac{\partial \dot{c}_k}{\partial c_j}$. Following the above procedure
 \beq
	\frac{\partial \dot{c}_k}{\partial b_{j}} =  2 \frac{k}{\tildeL} ( b_{k+j}+b_{k-j} )\,,
 \eeq
 \beq
	\frac{\partial \dot{b}_k}{\partial b_{j}} =  \left(\frac{k}{\tildeL}\right)^2\left(1- \left(k/\tildeL\right)^2 \right)\delta_{kj} - 2 \frac{k}{\tildeL} (c_{k+j} + c_{k-j}) \,,
 \eeq
 \beq
	\frac{\partial \dot{b}_k}{\partial c_{j}} = 2 \frac{k}{\tildeL} (b_{k+j}-b_{k-j}) \,.
 \eeq

\section{\KSe\ according to Predrag}
\label{s-KS}
% Predrag 					 4jul2006
% extracted from ~dasbuch/book/chapter/PDEs.tex  5jun2005 version
% Predrag               1 jan 2000
% Predrag              17 sep 99

%  remember to incorporate missing refs from {chapter/refsPDEs}

The \KS\ system\rf{ku,siv},
arising in the description of the flame front flutter of  gas burning in
a cylindrically symmetric burner on your kitchen stove,
and many other problems of greater import,
is one of the simplest partial differential equations that
exhibit turbulence.
The time evolution of the ``height of the flame front'' 
is given by
\beq
u_t=(u^2)_x-u_{xx}-\nu u_{xxxx}
\,,\qquad	x \in [0,L]
\,.
\ee{ks}
In this equation $t \geq 0$ is the time and
$x$ is the spatial coordinate.
The subscripts $x$ and $t$ denote partial derivatives with respect to
$x$ and $t$;
$u_t = du/dt$, $u_{xxxx}$ stands for the 4th spatial
derivative of 
$u=u(x,t)$ at position $x$ and time $t$.
The ``viscosity'' parameter 
$\nu$ controls the 
suppression of solutions with fast spatial variations.
We take note, as in the Navier-Stokes equation, of the
$u {\partial_x} u$ ``inertial'' term, the $ {\partial_x^2 } u$
``diffusive'' term (both with a ``wrong'' sign), etc.

The term $(u^2)_x$ makes this a {\em nonlinear system}.
It is one of the
simplest conceivable nonlinear PDE, playing
the role in the theory of spatially extended systems
analogous to the role that
the $x^2$ nonlinearity
% \refeq{LogisMap}
plays in the dynamics of iterated mappings.
The salient feature of such
partial differential equations is a theorem saying that
for any finite value of the phase-space contraction
parameter $\nu$,  the asymptotic dynamics is
describable by a {\em finite} set of ``inertial manifold''
ordinary differential equations. %cite{Foias88}.


\subsection{Fourier space representation} \label{s:FourierModes}

% \index{Fourier!mode!truncation}
% \index{truncations!Fourier}

%from KSe.tex \section{{\KSe}}
%\label{sec:KSequil}
% Predrag                                       15dec2004
% Lan                                           25nov2004
%
\noindent
Spatial periodic boundary condition $u(x,t)=u(x+2\pi\tilde{L},t)$
makes it convenient to work in the Fourier space, 
\beq
  u(x,t)=\sum_{k=-\infty}^{+\infty} b_k (t) e^{ i k x /\tilde{L} }
\, .
% \label{fseries}
\ee{eq:ksexp}
with \refeq{ks-L} replaced by an infinite set of 
ODEs for the Fourier coefficients:
\beq
% \dot{b}_k= \left( \frac{k}{\tilde{L}}\right)^2.
%      \left( 1 - \left( \frac{k}{\tilde{L}}\right)^2  \right) b_k 
%	 + i \frac{k}{\tilde{L}} \sum_{m=-\infty}^{+\infty} b_m b_{k-m}
\dot{b}_k=(k/\tilde{L})^2\left( 1 - (k/\tilde{L})^2  \right)b_k 
 	 + i (k/\tilde{L}) \sum_{m=-\infty}^{+\infty} b_m b_{k-m}
\,.
\ee{expan}
%%% begin NW : added stuff to sentence and commented out old version.
%
This is the infinite set of ordinary differential equations promised
in this chapter's introduction. 
% \index{infinite-dimensional flows}
% \index{flow!infinite-dimensional}
% at the beginning of the section.
%
%%% end NW

Since $u(x,t)$ is real,
$ %\[
b_k=b_{-k}^*
\,,
$ %\] %\label{cplx-b}
so we can replace the sum over $k$ in \refeq{expan} by a
sum over $k \geq 0$.
As  $\dot{b_0}=0$, $b_0$ is a conserved quantity,
in our calculations
fixed to $b_0=0$ by
the Galilean invariance condition \refeq{GalInv}.
% \index{Galilean invariance}
% \index{invariance!Galilean}

The Fourier coefficients $b_k$ are in general complex numbers.
% of time $t$.  
We can
%simplify
isolate the antisymmetric subspace of the system \refeq{ks-L} by
considering the case of $b_k$ pure imaginary, $b_k= i a_k$, where
$a_k$ are real, with the evolution equations
\beq
% \dot{a}_k=(k^2- k^4)a_k - k \sum_{m=-\infty}^{\infty} a_m a_{k-m}
\dot{a}_k = (k/\tilde{L})^2\left( 1 - (k/\tilde{L})^2  \right)a_k 
 	 - (k/\tilde{L}) \sum_{m=-\infty}^{+\infty} a_m a_{k-m}
\,.
\ee{expan-symm}
\PC{create exercise here}
%\exerbox{}
This picks out the subspace of 
antisymmetric solutions $u(x,t)=-u(-x,t)$,
so $a_{-k}= - a_k$. By picking this subspace we eliminate
the continuous translational symmetry from our consideration;
that is probably not an option for an experimentalist,
but will do for our purposes.

In the antisymmetric subspace the translational 
invariance of the full system reduces
to the invariance under discrete
translation by half a spatial period $L$.
In the Fourier representation \refeq{expan}, 
this corresponds to invariance under 
\beq
a_{2m} \to a_{2m}\,, a_{2m+1} \to -a_{2m+1}
% \,, m \in \mathbb{Z}
\,.
\ee{FModInvSymm} 
The antisymmetric condition amounts to imposing
$u(0,t)=0$ boundary condition, with
the size of the system reduced to
the $[0,L/2]$ interval. In
comparing our numerical results with % other authors' 
calculations for
the full, unrestricted dynamics on $[0,L]$, we define
the non-dimensionalized system size as
$\tilde{L} = {L}/{(4 \pi \sqrt{\nu})}$,
corresponding to a system defined on the
$[0,L/2]$ domain. 


\PublicPrivate{%
		}{% switch to Private:
\subsection{Heteroclinic and homoclinic connections}
% from \Chapter{KS}{12Jun2004}{Steady solutions of KS equation}
% Lan                                            8Jun2004

KSe\ \refeq{eq:stdks} has an exact heteroclinic solution
\beq
u=a_1 \tanh(kx)+a_2 \tanh^3(kx)
\label{eq:ksexa} 
\,,
\eeq
for the specific value $c=-\frac{30k^2}{19}(304k^4-40k^2+1)$ where 
\[
a_1=60k^3-\frac{30k}{19}\,,\;a_2=-60k^3\,,\;k^2=\frac{11}{76} \mbox{ or } 
k^2=-\frac{1}{76}
\,.
\]
When $k^2=-1/76$, the hyperbolic tangent becomes ordinary tangent and there are poles on
the real axis. Near the singularity, $u$ goes like
like 
\[
u(x) \approx \frac{-60}{(x-x_0)^3}\,,\; x \to x_0 \,.
\]
All the blow-up solutions possess singularities of the same type\rf{ksham95}.
In general, the existence and structure of connections depend on parameter
$c$ in a complicated way.
		}% end \PublicPrivate



In order to find a better representation of the dynamics, we now
turn to its topological invariants.

