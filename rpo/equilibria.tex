% equilibria.tex
% Predrag extracted from newton.tex 		jul  3 2006
% Predrag					jun 20 2006
% Vaggelis					may 20 2006
% $Author$ $Date$

\section{\Eqva} % of the \KSe}
\label{sec:stks}

% Predrag                                       05dec2004
% Lan                                           25nov2004
% from Lan thesis                                8jun2004

\noindent
\Eqva\  (or the steady solutions)
are the simplest invariant sets in
the phase space. They,  and 
the connections between them form the
coarsest geometrical framework for organizing
phase space orbits. %\rf{ksgreene88}.

The \eqv\ condition $u_t=0$ for the {\KSe} PDE \refeq{ks-L} 
is the ODE
\[
(u^2)_x-u_{xx}- u_{xxxx}=0 
% \,.
\]
which can be analyzed as a dynamical system in its own right.
Integrating once we get
\beq
u^2-u_x- u_{xxx}=c 
\,,
\label{eq:stdks}
\eeq
where $c$ is an integration constant 
whose value strongly influences the nature of
the solutions. %of \refeq{eq:stdks}. 
Written as a 3-dimensional dynamical system
with spatial coordinate $x$ playing the role of ``time'',
%\refeq{eq:stdks} 
this is a volume preserving flow
\beq
u_x = v \,,\qquad
v_x = w \,,\qquad
w_x = u^2-v-c \,,
  \label{eq:3dks}
\eeq
with the ``time'' reversal symmetry, 
\[
x \to -x,\quad u \to -u, \quad v \to v, \quad w \to -w \,.
\]
 From \refeq{eq:3dks} we see that
\[
(u+w)_x=u^2-c \,.
\]
If $c<0$, $u+w$ increases without bound with $x \to \infty$,
and every solution escapes to infinity.
If $c=0$, the origin $(0,0,0)$ is the
only bounded solution. 

For $c>0$ there is much
$c$-dependent interesting dynamics, with
complicated fractal sets of bounded solutions.
The sets of the solutions of the \eqv\ condition 
\refeq{eq:3dks} are themselves in turn organized by the  
{\eqva} of the \eqv\ condition, and 
the connections between them.
    For $c>0$ the \eqv\ points of \refeq{eq:3dks} are
$c_{+}=(\sqrt{c},0,0)$ and $c_{-}=(-\sqrt{c},0,0)$.
Linearization of the flow around
$c_{+}$ yields 
stability eigenvalues 
\beq
 [2\lambda \,, -\lambda \pm i \theta ] 
\,,\qquad
\lambda=\frac{1}{\sqrt{3}}\sinh \phi
\,,\qquad
\theta=\cosh \phi \, ,
\label{eqvEqvEigV}
\eeq
and $\phi$ fixed by $\sinh 3\phi=3\sqrt{3c}$. 
Hence $c_{+}$ has a {1-dimensional}
unstable manifold and a 2-dimensional
stable manifold along which solutions spiral 
in. 
By the $x \to -x$ ``time reversal'' symmetry, the 
invariant manifolds of $c_{-}$ 
have reversed stability properties.

\PublicPrivate{%
		}{% switch to Private:

\subsection{Heteroclinic and homoclinic connections}
%
% from \Chapter{KS}{12Jun2004}{Steady solutions of KS equation}
% Lan                                            8Jun2004

\PC{either drop this, or credit \refref{Lan:Thesis,LanCvi06}
    if we need it
   }
\KSe\ \refeq{eq:stdks} has an exact heteroclinic solution
\beq
u=a_1 \tanh(kx)+a_2 \tanh^3(kx)
\label{eq:ksexa} 
\,,
\eeq
for the specific value $c=-\frac{30}{19}k^2(304\,k^4-40\,k^2+1)$ where 
\[
a_1=60\,k^3-\frac{30}{19}k
	\,,\qquad
a_2=-60\,k^3
	\,,\qquad
k^2=11/76 \mbox{ or } 
k^2=-1/76
\,.
\]
When $k^2=-1/76$, the hyperbolic tangent becomes ordinary tangent and there are poles on
the real axis. Near the singularity $x_0$, $u$ blows up as
\[
u(x) \approx \frac{-60~}{(x-x_0)^3}
\,.
\]
All blow-up solutions possess singularities of the same type\rf{ksham95}.
% In general, the existence and structure of connections depend on parameter
% $c$ in a complicated way.
		}% end \PublicPrivate

\subsection{\KSe\ on a periodic domain}
% However, we do not need to explore the fractal set of the 
% \KS\ {\eqva} for infinite size system here;
For a fixed system size 
$L$ with periodic boundary condition, the only surviving {\eqva}  are
those with periodicity $L$.
They satisfy 
the \eqv\ condition for \refeq{expan}
\PC{\rf{ksgreene88} to remarks}
\beq
(k/\tilde{L})^2\left( 1 - (k/\tilde{L})^2  \right)b_k 
	 + i (k/\tilde{L}) \sum_{m=-\infty}^{+\infty} b_m b_{k-m} = 0
\,.
\label{eq:stfks}
\eeq 
% We have proved in \refsect{sec:kspr} that 
Periods of spatially periodic {\eqva} are multiples of $\tildeL$.
Every time $\tilde{L}$ crosses an integer value  $\tilde{L}=n$,
$n$-cell states
are generated through pitchfork bifurcations. 
Due to the translational invariance of {\KSe},
they form invariant circles
in the full phase space.
% In the antisymmetric subspace considered here, they corresponds to two points,
% half-period translates of each other of the form
% \[
% u(x,t)=-2\sum_k b_{kn}\sin (knx) \,,
% \]
% where $b_{kn} \in \mathbb{R}$.
% % By rescaling $u,x$ and $\nu$, the $n$-cell states transform to each other.

%      With the increase of $L$ these periodic solutions 
% may bifurcate into more complicated ones.
For any fixed period $L$
%, however, 
the number 
of spatially periodic solutions is finite up to a spatial translation.
This observation can be heuristically motivated as follows. 
% \PC{this argument keeps worrying me: there are lots of solutions, like
% $u=0$, that are {\eqva}, but isolated -
% they are noplace near asymptotic dynamics.
% Do they belong to the invariant manifold?
%    }
% Equilibria are solutions valid for all times, and are thus points
% on the finite-dimensional compact inertial manifold\rf{infdymnon}.
Finite dimensionality of the inertial manifold
bounds the size of Fourier components of all solutions.
% This
% compact inertial manifold and the dynamics on it can be 
% described by analytic functions of a finite number of Fourier modes.
\PC{explain the theory; say that in practice it is useless}
On a finite-dimensional compact manifold,
an analytic function can only have a finite number
of zeros. So the {\eqva}, {\ie},
the zeros of a smooth velocity field on
the inertial manifold, are finitely many.
% The number of {\eqva} increases exponentially with $L$,
% \PC{give reference for ``exponential growth''}
% for infinite system size $L \to \infty$,
% there are infinitely many {\eqva}. 
% \PC{is there a reference where this is explained?}

For a sufficiently small $L$ 
the number of {\eqva} is small,
mostly
concentrated on the low wave number end of the Fourier spectrum.
These solutions may
be obtained by solving the truncated versions of \refeq{eq:stfks}. 
% Understanding the structure of these solutions, requires a study
% of the full phase space of the 3-dimensional dynamical system \refeq{eq:3dks},
% not attempted here.

% Locally coherent structures are observed for arbitrary system size,
% see \reffig{f:flameFlut}~(b). %\reffig{f:ksev}.


% \index{Kuramoto-Sivashinsky \eqva\}
% \index{equilibria!Kuramoto-Sivashinsky}



% \PC{say somewhere: ``
% The task of the theory is to describe this spatio-temporal
% turbulence and yield quantitative predictions for its measurable
% consequences.
%    ''}


In a periodic box of size $L$
both \eqva\ and \reqva\ are  periodic solutions 
embedded in 3-$d$ space \refeq{eq:3dks}, 
conveniently represented as loops in 
$(u,v,w)$ space whose topology is controled by
``\eqva\ of \eqva\'' stable-unstable manifold structure of
\refeq{eqvEqvEigV}.
In this representation the continuous translation symmetry
is automatic - a rotation in the $[0,L]$ periodic domain only
moves the points along the loop. For an \eqv\ the points
are stationary in time; for \reqv\ they move in time, but in
either case, the loop remains invariant.
So we do not have the problem that we encounter in the Fourier 
represenation, where from the frame of one of the \eqva\
the rest trace out circles under the action of continuous symmetry 
translations.

\Preliminary{
The $c$ integration constant also means something, we are not quite sure
what.

Perhaps there is some way of plotting the unstable manifolds too. If
there is only one unstable direction (in the full Fourier space
representation), the corresponding eigenvector is a unique loop function
$h[s] =  h(u(s),v(s),w(s))$ in the $(u,v,w)$ space, and the unstable manifold
$U$ is swept out by evolving in time the perturbed loop 
$L[s] + \delta h[s] =  L(u(s),v(s),w(s)) + \delta h(u,v,w)$
$\to$ $L[s,t]$.
Complex eigenvector defined unstable manifold plane seems
harder to visualize;  It would be interesting
to look at heteroclinic connections in the $(u,v,w)$ space, as
behavior of higher-frequency modes in Fourier reps is a bit
hard to get used to.
	    } %end \Preliminary{



\subsection{Stability of \eqva}
\label{s:StabEqui}
%%
% Predrag           5jun2005
% extracted from \Chapter{stability}{ 2apr2005}
% Predrag                  14/3-95
% taken from ks.tex

% \index{local!stability}
% \index{linear!stability}
% \index{stability!linear}
% \index{equilibrium!point}

$u_\stagn(x)$ is an \eqv\ solution of \KSe,
the linear stability matrix
${\bf \Mvar}={\bf \Mvar}(a_\stagn)$
% , its matrix of its stability exponents
% in \refeq{die}
% local expansion rate
is constant in time,
and  
the {\jacobianM}
of the \eqv\ solution is
\[
 \jMps^t(u_\stagn) = e^{{\bf \Mvar} t}
    \,\qquad
 {\bf \Mvar}={\bf \Mvar}(u_\stagn)
\,.
\]

For the \KSe\ the constant solution $u(x,t)= c^{1/2}$ is an 
\eqv\ point of \refeq{ks}. The matrix of variations
\refeq{DerMatrix}
follows from \refeq{expan}
% \index{Kuramoto-Sivashinsky system}
\beq
{\Mvar}_{kj}(a) ={\pde v_k(a)\over \pde a_j  }
=((k/\tilde{L})^2- (k/\tilde{L})^4)\delta_{kj} - 2(k/\tilde{L}) a_{k-j}
\,.
\ee{expanMvar}
For the $u(x,t)=0$ \eqv\  solution the matrix of variations
is diagonal, and -- as in \refeq{EqDyn164} -- so is the {\jacobianM}
$
\jMps^t_{kj}(0) = \delta_{kj} e^{((k/\tilde{L})^2- (k/\tilde{L})^4)t}
\,.
$

For $\tilde{L} < 1$,  $u(x,t)=0$ is the globally attractive stable 
{\eqv}.
As the system size $\tilde{L}$  is increased,
the ``flame front'' becomes increasingly unstable and turbulent,
the dynamics goes through a rich sequence of
bifurcations on which we shall not dwell here.
% studied e.g. in \refref{KNS90}. 
% , one quickly finds a
% myriad of unstable periodic solutions whose number
% grows exponentially with period.

The $|k|<??$ 
long wavelength perturbations of the flat-front {\eqv}
are linearly unstable, while all 
$|k|> ??$ short wavelength perturbations are strongly contractive.  
The high $k$ eigenvalues, corresponding to rapid variations of
the flame front, decay so fast that the corresponding eigendirections
are physically irrelevant.
% \index{Lyapunov exponent!{\eqv}}
To illustrate the rapid contraction in the non-leading eigendirections
we plot  in [MAYBE INCLUDE] % \reffig{f:eigenvalues}
the eigenvalues of the \eqv\ in the unstable regime,
for relatively small system size, % low viscosity $\nu$,
and compare them with the
stability eigenvalues of the least unstable cycle for the same 
system size.
% value of $\nu$. 
The \eqv\ solution is very unstable,
in 5 eigendirections,
the least unstable cycle only in one. 
Note that for $k>7$ the rate of contraction
is so strong that higher eigendirections are numerically meaningless for 
either solution; even though the flow is infinite-dimensional, the attracting
set must be rather thin.

While in general
for $\tilde{L}$ sufficiently large
one expects many 
coexisting attractors in the phase space%
%Hyman and Nicolaenko
\rf{HNZks86} ,
in numerical studies most random initial
conditions settle converge to the same chaotic attractor. 

From \refeq{expan} we see that the origin $u(x,t) = 0$
has Fourier modes as the  linear
stability eigenvectors. 
When $|k| \in (0,\tilde{L})$, the corresponding Fourier modes are
unstable.
The most unstable modes has $|k|=\tilde{L}/\sqrt{2}$ and defines the scale of basic building
blocks of the spatiotemporal dynamics of the {\KSe} in large system size limit,
as shown in \refsect{sec:KSnumer}. 


% \noindent
% Consider now the case of initial $a_k$ sufficiently small
% that the bilinear $ a_m a_{k-m}$ terms in \refeq{expan} can
% be neglected. Then we have a set of decoupled linear
% equations for $a_k$ whose solutions are exponentials, at most
% a finite number for  which
% $k^2 > \nu k^4$
% is growing with time, and infinitely many with
% $
% \nu k^4 > k^2
% $
% decaying in time.
% The growth of the unstable long wavelengths (low $|k|$) excites
% the short wavelengths
% through the nonlinear term in \refeq{expan}.  The excitations thus
% transferred are dissipated by the strongly damped short wavelengths,
% and a ``chaotic \eqv\'' can emerge. The very short
% wavelengths $|k| \gg 1 / \sqrt{\nu}$ remain small for all times,
% but the intermediate wavelengths of order $|k| \sim 1 / \sqrt{\nu}$
% play an important role in maintaining the dynamical {\eqv}.
% As the damping parameter decreases, the solutions increasingly take on
% % Burgers type
% shock front
% character poorly represented by the Fourier basis, and many
% higher harmonics may need to be kept
% % \rf{KNS90,GEP}
% in truncations of
% \refeq{expan}.
% 
% 
% Hence, while one may truncate the high modes in the expansion
% \refeq{expan}, care has to be exercised to ensure that no modes
% essential to the dynamics are chopped away. 
% 
Even though our starting point
\refeq{ks}
is an infinite-dimensional dynamical system, the asymptotic dynamics
unfolds on a finite-dimensional attracting manifold, and so we are back on
the familiar territory of \refsect{SecDynFlows}:
the theory of a finite number of ODEs applies to this
infinite-dimensional PDE as well.

    {\bf When is an \eqv\ important?} There are two kinds of roles
{\eqva} play:

{\em ``Hole'' in the natural measure}.
The more unstable eigendirections it has (for example, the
$u=0$ solution), the more unlikely it is  that
an orbit will recur in its neighborhood.

{\em unstable manifold of a ``least unstable''{\eqv}}.
 Asymptotic dynamics
spends  a large fraction of time in
neighborhoods of a few  {\eqva} with
only a few unstable eigendirections.


\underline{\KS\ system, truncations.}{
We describe here our criterion for reliable
truncations of the infinite ladder of 
ordinary differential equations (\ref{expan}).

Adding an extra dimension to a truncation of the system (\ref{expan})
introduces a small
perturbation, and this can (and often will) 
throw the system into a totally different asymptotic state. 
A  chaotic attractor for $N=15$ can become a period three 
window for $N=16$, and so on. 
If we compute, for example, the Lyapunov exponent
$\Lyap(\nu,N)$ for the strange attractor of the 
system (\ref{expan}), there is no reason to 
expect $\Lyap(\nu,N)$ to smoothly converge to the limit  
value $\Lyap(\nu,\infty)$ as $N \rightarrow \infty$. 
The situation is different in the periodic windows, 
where the system is structurally stable, and it makes sense to compute 
 Lyapunov exponents, escape rates, etc. for the 
{\em repeller}, \ie, the closure of the set of all 
{\em unstable} periodic orbits. 
Here the power of cycle expansions comes in: 
to compute quantities on the repeller by direct averaging methods is 
generally more difficult, because the motion quickly collapses to the 
stable cycle. 
	} %end \remark{Kuramoto-Sivashinsky system, numerical results.}


The problem one faces with high-dimensional flows is 
that their topology is hard to
visualize, and that even with a decent starting guess for a point on
a periodic orbit, methods like the Newton-Raphson method are likely to fail.
Methods that start with initial guesses for a number of points along the
cycle, such as the multipoint shooting method of \refsect{s-MultShoot},
are more robust.
The relaxation  (or variational) methods take this strategy to its
logical extreme,
and start by a guess of not a few points along a periodic orbit,
but a guess of the entire orbit.

At present the theory is in practice applicable only to systems
with a low intrinsic {\em dimension}
-- the minimum number of coordinates necessary to
capture its essential dynamics.
% \index{dimension!intrisic}
% \index{degree of freedom}
If the system is very turbulent
(a description of its long time dynamics requires a space of high
intrinsic dimension) we are out of luck. 
% \index{turbulence}

% \input{chapter/refsPDEs}

%%% end

