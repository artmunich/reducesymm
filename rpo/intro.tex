% intro.tex
%
% Predrag                       jun 20 2006
% $Author$ $Date$

\section{Introduction}

\PC{ I. Rich History: reference Poincar\'e and onward}

Recent experimental and theoretical advances  
support a dynamical vision
of turbulence ascribed to E.~Hopf by  \cite{EAS87}, as a walk through
a repertoire of unstable recurrent patterns:
As a turbulent flow evolves,
every so often we catch a glimpse of a familiar pattern.
For any finite  spatial resolution,
the system follows approximately for a finite time 
a pattern belonging to a 
{ finite alphabet}
of admissible patterns.
The long term dynamics is
a {  walk through the space of such unstable patterns}.

The question is how to characterize and classify such patterns?
\cite{MS66} and \cite{Christiansen:97} have proposed (in contexts
much simpler than the full Navier-Stokes hydrodynamics) 
that the unstable spatiotemporally periodic
solutions could play that role.

Here we follow the seminal
\cite{hopf48} paper, and  visualize
hydrodynamic turbulence not as  a sequence of 
3-$d$  instants in turbulent evolution, with
each pixel a 3-$d$ velocity field
but as a trajectory in an 
 $\infty$-$d$ state space in which an
instant in turbulent evolution is
a { unique} point. In E.~Hopf's vision, 
theory of turbulence for a given system, with given boundary conditions,
is given by the
(a) geometry of the state space and (b) the associated natural measure, 
\ie the likelihood that asymptotics dynamics visits a given state space region.

\Preliminary{
Three examples, in order of increasing complexity:
1) R\"ossler ``chaos",  3-$d$ state space;
2) Kuramoto-Sivashinsky  turbulence, $\infty$-$d$ state space;
and
3) Navier-Stokes  ``turbulence'',
 $\infty$-$d$ state space.

The theorem on finite dimensionality of inertial manifolds
of phase-space contracting PDE flows is proven in \refref{Foias88}.
% \index{inertial manifold}
% \index{Kuramoto, Y.}
% \index{Sivashinsky, G.I.}
% \index{Kuramoto-Sivashinsky system}
The Kuramoto-Sivashinsky equation was introduced in \refrefs{ku,siv}.
Holmes, Lumley
and Berkooz\rf{Holmes96} offer a delightful discussion of why this system
deserves study as a staging ground for studying turbulence in 
full-fledged Navier-Stokes equation. 
How good 
a description of a flame front this equation
is need not concern us here; suffice it to say that such model
amplitude equations for interfacial instabilities arise in a variety
of contexts - see e.g.~\refref{saddks} - and this one is perhaps the
simplest physically interesting spatially extended nonlinear system.
\PC{Comment om MAWs, BECS and CGLe}
\PC{refer to Trefethen's program for fast integration}
This chapter
% \refsect{s_extend} 
is based on V. Putkaradze's term project
(see \wwwcb{/extras}).
%a gratifying example of a successful course project which led to a
%full-fledged research paper by
and on the Christiansen {\em et al.} article \rf{Christiansen:97}. 
\PC{refer the reader to \refref{Lan:Thesis} for a
 review of the $L \to \infty$ equilibria.
 Shouldn't Michelson\rf{Mks86} be given credit here?}

\Eqva\ and \reqva\ of \KSe\ are investigated in Kevrekidis \etal\rf{saddks}. 

Testing bibtex - these should exist:
\refrefs{Laurent-Polz04,lop05rel,McCordMontaldi}
\refrefs{Vanderb,Wulff00}
	}  %end \Preliminary{

