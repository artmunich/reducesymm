% intro.tex
%
% Predrag                       jun 20 2006
% $Author$ $Date$

\section{Introduction}

\PC{ I. Rich History: reference Poincar\'e and onward}

Recent experimental and theoretical advances  
support a dynamical vision
of turbulence ascribed to E.~Hopf by  \cite{EAS87}, as a walk through
a repertoire of unstable recurrent patterns:
As a turbulent flow evolves,
every so often we catch a glimpse of a familiar pattern.
For any finite  spatial resolution,
the system follows approximately for a finite time 
a pattern belonging to a 
{ finite alphabet}
of admissible patterns.
The long term dynamics is
a {  walk through the space of such unstable patterns}.

The question is how to characterize and classify such patterns?
\cite{MS66} and \cite{Christiansen:97} have proposed (in contexts
much simpler than the full Navier-Stokes hydrodynamics) 
that the unstable spatiotemporally periodic
solutions could play that role.

Here we follow the seminal
\cite{hopf48} paper, and  visualize
hydrodynamic turbulence not as  a sequence of 
3-$d$  instants in turbulent evolution, with
each pixel a 3-$d$ velocity field
but as a trajectory in an 
 $\infty$-$d$ state space in which an
instant in turbulent evolution is
a { unique} point. In E.~Hopf's vision, 
theory of turbulence for a given system, with given boundary conditions,
is given by the
(a) geometry of the state space and (b) the associated natural measure, 
\ie the likelihood that asymptotics dynamics visits a given state space region.

\Preliminary{
Three examples, in order of increasing complexity:
1) R\"ossler ``chaos",  3-$d$ state space;
2) Kuramoto-Sivashinsky  turbulence, $\infty$-$d$ state space;
and
3) Navier-Stokes  ``turbulence'',
 $\infty$-$d$ state space.


Testing bibtex - these should exist:
\refrefs{Laurent-Polz04,lop05rel,McCordMontaldi}
\refrefs{Vanderb,Wulff00}
	}  %end \Preliminary{

