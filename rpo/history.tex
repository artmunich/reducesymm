% history.tex
%
% Predrag created file				jul 3 2006
% $Author$ $Date$


\section{{\Rpo s} - a brief history}

Chenciner\rf{ChencinerLink}
says Poincar\'e introduced them in the 3-body problem.
Consider motion of a test particle of mass
$\mu \ll 1$ in the
restricted three-body problem\rf{rtb},
under the
influence of the gravitational force of two heavy bodies with masses $1$ and
$\mu \ll 1$ fixed at $(-\mu,0)$ and $(1-\mu,0)$. \Reqv\ of this problem
(known in this context as the Lagrange points, stationary in
the co-rotating frame) are circular motions in the inertial frame,
and {\rpo s} correspond to quasiperiodic motions. 

The relative equilibria and relative periodic solutions 
respectively appear from
equilibria and periodic solutions of the Hamiltonian reduced by the symmetries.
They are in many physical situations, which include rigid bodies, gravitational
$N$-body problems, molecules and nonlinear waves.

Lan has some relative equilibria (travelling waves) for KS in his
thesis\rf{Lan:Thesis}, %http://chaosbook.org/projects/theses.html
 and for complex LG in a paper on "MAWs".
Viswanath\rf{ViswanathPC06} % arXiv.org/physics/0604062
found them in the plane Couette problem.
Siminos has for Davidchack box size $L=22$ some equilibria, and all
orbits his \rpo\ search finds are in the set you sent us.

Striking application of \rpo s has been the discovery
of ``choreographies" of $N$-body problems\rf{McC7,McC8,McC}.
\PC{some McCord references}


