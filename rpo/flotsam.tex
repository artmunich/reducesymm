% flotsam.tex
%
% Predrag created file				jun 20 2006
% $Author$ $Date$

\section{Flotsam}

\subsection{Template for KS equilibria discussion}

ES:{I observe pairs of real eigenvalues,
e.g. -58.3602685 and -58.3602681. As their absolute
value increases they differ even less.
I think it has to do with the linear part being the main
contribution to $\Mvar$ for higher modes, as well as
with treating real and imaginary components
as separate variables, which means it will appear twice.
        }

PC: {I think such contracting eigenvalues as -58.3602685 have no meaning.
Even if they are accurate eigenvalues of $\Mvar$,
what use is
$\ExpaEig_{radial} =  e^{\Lyap_p \period{p}} = e^{26\cdot58} = e^{1510}$.
        }

The \reqva (or travelling waves) appear to have a limiting propagation
velocity $c_{max} = \pm d/\period{}$.
To visualize them numerically,
start with a localized self-dual $u(x,0)$ such as
\[
u(x,0) = x e^{- x^2/2\sigma^2}
\,,
\]
with typical width $\sigma/2$ of order of typical wavelength
$\sqrt{2}$ (in $\tilde{L}$ system size units).
Time evolution of this  $u(x,t)$ is bracketed by two constant
pulses of apparently constant velocity $v=?$.
\RLD{generate figure, state $\sigma/2$, estimate $v$}
The notion of ``velocity''
is fuzzed up by the fact that the large peaks are preceeded
by smaller precursors.

PC: {Determine their velocity ANALYTICALLY?}

\underline{Stability of \KS\ equilibria:}{ 
% \label{exam:KSEquilStab}
% \index{Kuramoto-Sivashinsky equilibria}
\begin{table}
\caption[]{
Important \KS\ equilibria:
% in the antisymmetric solution 
% space of the Kuramoto-Sivashinsky equation with periodic boundary % % % % condition,
% $ \nu =1$, $L=38.5$;
% their labels,
the first few stability exponents
%, with complex pairs written together.
}

\vskip 1.5cm

{\small
%\lineup
\begin{tabular}{@{}ccccc}
\hline %\br
$~S~~~$ & $~~~~\lambda_1 \pm \,i\,\theta_1$ 
                                & $~~~~\lambda_2 \pm \,i\,\theta_2$ 
                                        & $~~~~\lambda_3 \pm \,i\,\theta_3$ 
\\ 
\hline %\mr
${C_1}$    &{0.04422 $\pm \,i\,$0.26160}   &-0.255 $\pm \,i\,$0.431 
&-0.347 $\pm \,i\,$0.463         \\
% ${C_2}$    &0.33053  & 0.097 $\pm \,i\,$0.243 
% &-0.101 $\pm \,i\,$0.233        \\
\hline %\mr
${R_1}$   &{0.01135 $\pm \,i\,$0.79651} & -0.215 $\pm \,i\,$0.549 
&-0.358 $\pm \,i\,$0.262        \\
%  ${R_2}$   &  0.33223  & -0.001 $\pm \,i\,$0.703  
%  & -0.281 $\pm \,i\,$0.399      \\
\hline %\mr
${T}$     & 0.25480  & -0.07 $\pm \,i\,$0.645 &-0.264  
\\
\hline %\br
\end{tabular}
}
\label{t:stationary}
\end{table}

{\em 
spiraling out in a plane}, all other directions contracting


{\bf
Stability of ``center'' equilibrium
	}

linearized stability exponents: 
\[ % \beq
(\lambda_{1}\pm\,i\,\theta_{1},\lambda_{2} \pm\,i\,\theta_{2}, \cdots)
	= (0.044 \pm \,i\,0.262\,,\,
		-0.255 \pm \,i\,0.431\,,\,\cdots)
\] %\eeq

The plane spanned by $\lambda_{1} \pm\,i\,\theta_{1}$ eigenvectors rotates with angular period
$\period{} ~\approx~2\pi/\theta_{1}=24.02$.
% 2*4*a(1)/0.26160 = 24.0182924586375

a trajectory 
that starts near  the $C_1$~equilibrium point spirals 
away per one rotation
with multiplier
$\ExpaEig_{\mbox{radial}}~\approx~\exp(\lambda_{1}\period{})=2.9$.
% 2*4*a(1)/0.26160*0.04422 = 1.062
% e(1.0620888) = 2.8924063421

each Poincar\' e section return, 
contracted into the stable manifold by 
factor of
{
$\ExpaEig_{2}\approx\exp(\lambda_{2}\period{})=0.002$
}
%2*4*a(1)/0.26160*(-0.255) = -6.12466
%e(-6.12) = .0022


The local Poincar\' e return map is 
{\em
in practice $1-dimensional$
}
	} %end \example{{Stability \KS\ equilibria


% Staring at the solution
% as it evolves in time we should start getting a glimpse of the
% repertoire of the spatiotemporal patterns charcterizing
% the turbulent dynamics.

Many of the \rpo s can be constructed from segments corresponding to
close approches to some of these equlibria.

\subsection{Numerical methods}

Shroff and Keller\rf{shroff:1099}
``The Recursive Projection Method" (RPM)
might be of interest to us in numerical work.
RPM stabilizes fixed-point iterative 
procedures by computing a projection onto the unstable subspace.
On this subspace a Newton or special Newton iteration is performed, 
and the fixed-point iteration is used on the complement. 
The method is effective when the dimension of the unstable subspace 
is small compared to the dimension of the system.
Examples are presented for computing unstable steady states.
The RPM can also be used to accelerate iterative procedures when 
slow convergence is due to a few slowly decaying modes.

PC also has 2 hard-copy, rather readable
old proceedings reprints by Keller\rf{Keller77,Keller79}.
There are some readable words about homotopy methods in
\refref{AllgGeorg88}.


\subsection{V. Lopez: 
	    {\Rpo s} of the Complex Ginzburg-Landau Equation}

        John, Jonathan, Vaggelis and Rytis have been pondering role of
continuous symmetries (simple ones - translations, periodic domains) in
periodic orbit theory. One needs to
generalize notion of ``periodic" to ``relative periodic" (equvariant,
travelling, running are other names I have heard), and recently
we have received results from Viswanath (who finds such in plane Couette)
and Davidchack (who finds such in \KSe\ on a periodic
domain).

As far as the discussion about ``drift" is concerned:
In presence of continuous symmetries (for PC, streamwise and
spanwise translations) one should be searching for 
{\reqva} and 
{\rpo s} - they are most likely more important than
the stationary equilibria and spatially periodic orbits. 

I like {\rpo s} paper\rf{lop05rel} by
Vanessa Lopez\rf{lopezLink},
%        http://www.cse.uiuc.edu/$\sim$vlopez:
``Relative Periodic Solutions of the Complex Ginzburg-Landau Equation" 
     
We have such systems whenever we have a translational symmetry (like plane
Couette).
{\Rpo s} were missed in earlier 
previous \KS\ investigations%
\rf{Christiansen:97,Lan:Thesis}
which focused on the antisymmetric subspace, where translational invariance
is broken.

% [From siads\@siam.org  Dec 22 2004]:
% willing to review this manuscript for Dwight Barkley.
     
A method of finding {\rpo s} for differential equations with 
continuous symmetries is described and its utility demonstrated by computing 
{\rpo s} for the one-dimensional complex Ginzburg-Landau 
equation (CGLE) with periodic boundary conditions.  A {\rpo} is 
a solution that is periodic in time, up to a transformation by an element of the 
equation's symmetry group.  With the method used, {\rpo s} are 
represented by a space-time Fourier series modified to include the symmetry 
group element and are sought as solutions to a system of nonlinear algebraic 
equations for the Fourier coefficients, group element, and time period. 
{\Rpo s} found for the CGLE exhibit a wide variety of temporal 
dynamics.
with the % sum of their positive Lyapunov exponents varying from 5.19 to 
% 60.35 and their
unstable dimensions from 3 to 8.
% Preliminary work indicates that 
% weighted averages over the collection of {\rpo s} accurately 
% approximate the value of several functionals on typical trajectories.

15 Sep 2004, Predrag Cvitanovic:
* All of your {\Rpo s} have many unstable
eigendirections. Do you know that there exist no other solutions with a
lesser number of unstable directions (let's say only one)?

--vanessa:
I do not know of any proof that shows that there are no solutions with a
lesser number of unstable directions.  I did not find any solutions with
just one or two unstable directions, but at this point I cannot say
whether that is an indication that there are none or simply that
the procedure I used only identified solutions with 3+ unstable
directions.

15 Sep 2004, Predrag Cvitanovic:
* Periodic solutions are usefull if they are embedded into a chaotic
attractor. Do you have any measure of whether the typical solutions of
CGL, your parameter values, come close to any of the solutions that you
have found? If the periodic orbit is embedded into a chaotic attractor,
typical soultion visits it infinitely often, infinitesimally close.

--vanessa:
This is something that I have not examined in detail.  I have observed (using
the ``naked eye") that the time evolution of typical trajectories (when viewed
by plotting the real versus imaginary part of A(x,t) at different times, as in
Figures 3,5,7 from the paper) sometimes resembles that of the first solution
found (i.e., the patterns displayed in Figure 3). (The first solution
found also happened to be the solution to which the solver I used
converged to most frequently.)  But at this point I do not have a
quantitative measure of this ``resemblance".

After talking to Lopez at SIAM DS05 meeting - she did something that
is useful to us (found some {\rpo s}), but I do not think
they are the dynamically important ones, so the problem remains wide open.

I read through
Vanessa Lopez's paper (clearly a summary of a PhD thesis), and while I
a like {\rpo s}, I am very worried about the general drift
of the paper.

As I cannot tell how exhaustive is her set of numerical solutions, I do
not know what to make of them - like ZG, she gets all of them with a large
number unstable dimensions. Presumably none of them are close to the
asymptotic inertial manifold. More worrisome still, she uses ZG to
``average" over periodic orbits. That is regrettable - I hoped I would
never see that stuff again. Dettmann and I tried to get Zoldi to
derive this for us, and Mainieri actually hired him at Los Alamos to learn
how this works. We are all convinced that it makes no sense whatsoever.
Regrettably Lopez used the nonsensical formulas of Zoldy, so that will
just keep confusing future readers. 
She is in computer science, so cannot
blame a graduate student for trying a formula published in Phys Rev Lett.


If she is right and CGL at her parameter values does not contain
(relative) periodic orbits with 1 or very few unstable directions, we can
kiss the ``minimal instability principle" goodbye.

The 
periodic orbits
did not captured the main dynamics of the system. In fact, in the full
space I believe that the {\rpo s} are the more commonly
encountered objects in the phase space, just as the quasi-periodic motion is a
more frequent occurrence on an invariant torus. In a system with continous
symmetry, any orbit corresponds to a continous family of orbits, so the KS
system with periodic boundary condition is born on some kind of torus. We can
either reduce the system so that the continuous symmetry does not exist, or
study {\rpo s}. {\Rpo s} correspond to periodic
orbits in the reduced system. I don't know how to reduce the KS system in a
pretty way.

      In the case of {\rpo s}, we can find a piece of orbit on
the torus and construct the whole torus by a {\em known}
global symmetry operation in the
state space. For a generic torus solution, we do not have 
a global symmetry. 

        I am confused, and on odd days of the week I
 believe that generically such solutions are quasiperiodic.

        Here we discuss these continuous symmetries as
a small-step limit of discrete symmetries:

\begin{enumerate}
\item
        symmetries of 3-disk
\item
        $C_n$ symmetry of $n$-slice pizza without reflection symmetry
\item
        Fourier analysis of periodic lattices
\item
        running modes in periodic lattice deterministic
           diffusion
\item
	celestial {\rpo s}
\end{enumerate}

The reading material for 1)-3) is on
http://www.cns.gatech.edu/courses/chaosSpring06

In my \KS\ work with Christiansen, Putkaradze and Lan we
had excluded them ``by hand", by concentrating only on the space of
antisymmetric solutions. That is not good physics, as perturbations off
them mix into the full space of asymmetric solutions. One part of our
proposed FRG would be a search for {\rpo s} of
\KSe\, with Evangelos Siminos.

\subsection{F Laurent-Polz: {\Rpo s} in point vortex systems}

We give a method to determine {\rpo s} in point vortex systems: it consists mainly into perform a symplectic reduction on a fixed point submanifold in order to obtain a two-dimensional reduced phase space. The method is applied to point vortices systems on a sphere and on the plane, but works for other surfaces with isotropy (cylinder, ellipsoid, ...). The method permits also to determine some 
{\reqva} and heteroclinic cycles connecting these {\reqva}.

    {\Reqva} are orbits of the symmetry group action which are in-
variant under the flow, this corresponds here to motions of point vortices which
are stationary in a steadily rotating frame. We first review the literature about
the point vortex system on the sphere. The existence and nonlinear stability
of {\reqva} formed of three vortices have been studied respectively in
[KN98] and [PM98]. 

    {\Rpo s} are the analogs of {\reqva}
- concerning periodic orbits, this corresponds here to motions which are
periodic in a steadily rotating frame (a precise definition is given in Section
2). Periodic orbits on the sphere were determined in [ST, To01] thanks to the
following method: they reduced the system to two-dimensional systems by a
symmetric reduction (using some finite subgroups of $SO(3)$); the computation
of the dynamics on the reduced spaces permits then to determine periodic orbits.
Our paper is devoted to transpose that method to determine {\rpo s}. To this end, we combine a symmetric reduction together with a symplectic reduction. 
The method is explained in details in Section 2, it permits
to determine periodic orbits, RPOs, and heteroclinic cycles connecting relative

[Third Referee's Report Dr Newton]:
The manuscript considers {\rpo s} in point vortex systems
that arise from splitting {\reqv} configurations. It more
comprehensively treats problems of the type considered in Souliere and
Tokieda (2002) and Tokieda (2001). It is a very nice paper.

Angel Duran,
Numerical Integration of Hamiltonian 
Relative Periodic Solutions. A First Approach
http://www.math.human.nagoya-u.ac.jp/scicade05/
% angel@mac.uva.es

    [1] B. Cano and A. Duran, A technique to improve the error propagation when inte-
grating relative equilibria, BIT 44(2004) 215-235.
    [2] A. Duran and J.M. Sanz-Serna, The numerical integration of relative equilibrium
solutions. Geometric theory, Nonlinearity 11(1998) 1547-1567.
    [3] E. Hairer, Ch. Lubich and G. Wanner, Geometric Numerical Integration. Structure-
Preserving Algorithms for Ordinary Differential Equations, Springer Series in Comput.
Mathematics, Vol. 31, Springer-Verlag 2002.

