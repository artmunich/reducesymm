% flotsam.tex
%
% Predrag                       jun 20 2006

\section{Flotsam}


\subsection{Relative periodic orbits in classical mechanics}

Alain Chenciner www.imcce.fr/Equipes/ASD/person/chenciner/chenciner.html
says Poincar\'e introduced them in the 3-body problem.


 the strategy
is to minimize an action functional defined on a space of paths $\gamma$ in the
coniguration space $\pS$ which satisfy the property
\beq
                               \gamma (t + T ) = g \cdot \gamma (t)                       
\label{McC1}
\eeq
for a fixed {\em relative period} $T$ and some fixed diffeomorphism $g$ of $\pS$, often
referred to as a {\em phase}, that leaves the Lagrangian invariant. If $g$ has order
$k$ then the corresponding orbit is periodic with period $k T$. 
Striking applications of this idea has been the discovery
of ``choreographies" of $N$-body problems\rf{McC7,McC8,McC}.
\PC{some McCord references}

     The method is still valid if $g$ does not have finite order, in which case the
trajectories are only ``periodic modulo the action of $g$". Whether or not $g$
has finite order we refer to paths in $\pS$ satisfying \refeq{McC1} as 
{\em relative loops} and
to the corresponding solutions of Lagrangian system as {relative periodic
orbits}. 

    Let $L$ be a Lagrangian on a configuration manifold $\pS$ which is invariant
under a diffeomorphism $g : \pS \to\pS$. Our aim is to determine
the critical points of the action functional
\beq
	A[\gamma] =\int_0^T  dt L(\gamma,\dot\gamma, t)
\label{McC2}
\eeq
defined on $\Lambda_g^11 (\pS)$, a Sobolev space of paths in $\pS$ satisfying 
\refeq{McC1}.


\subsection{V. Lopez: 
	    Relative Periodic Solutions of the Complex Ginzburg-Landau Equation}

I like relative periodic orbits paper\rf{lop05rel} by
Vanessa Lopez,
        http://www.cse.uiuc.edu/$\sim$vlopez:
``Relative Periodic Solutions of the Complex Ginzburg-Landau Equation" 
     
I think we need this for systems with translational symmetry (like plane
Couette).

Yueheng Lan is very enthusiastic about this
paper, he thinks relative periodic orbits is what we were missing to
be able to analyze full-space Kuramoti-Sivashinsky (so far we have
only looked at the antisymmetric subsapce, where translational invarinace
is broken)

[From siads\@siam.org  Dec 22 2004]:
willing to review this manuscript for Dwight Barkley.
     
A method of finding relative periodic orbits for differential equations with 
continuous symmetries is described and its utility demonstrated by computing 
relative periodic solutions for the one-dimensional complex Ginzburg-Landau 
equation (CGLE) with periodic boundary conditions.  A relative periodic solution is 
a solution that is periodic in time, up to a transformation by an element of the 
equation's symmetry group.  With the method used, relative periodic solutions are 
represented by a space-time Fourier series modified to include the symmetry 
group element and are sought as solutions to a system of nonlinear algebraic 
equations for the Fourier coefficients, group element, and time period. The 77 
relative periodic solutions found for the CGLE exhibit a wide variety of temporal 
dynamics, with the sum of their positive Lyapunov exponents varying from 5.19 to 
60.35 and their unstable dimensions from 3 to 8. Preliminary work indicates that 
weighted averages over the collection of relative periodic solutions accurately 
approximate the value of several functionals on typical trajectories.

15 Sep 2004, Predrag Cvitanovic:
* All of your Relative Periodic Solutions have many unstable
eigendirections. Do you know that there exist no other solutions with a
lesser number of unstable directions (let's say only one)?

--vanessa:
I do not know of any proof that shows that there are no solutions with a
lesser number of unstable directions.  I did not find any solutions with
just one or two unstable directions, but at this point I cannot say
whether that is an indication that there are none or simply that (for some
reason) the procedure I used only identified solutions with 3+ unstable
directions.

15 Sep 2004, Predrag Cvitanovic:
* Periodic solutions are usefull if they are embedded into a chaotic
attractor. Do you have any measure of whether the typical solutions of
CGL, your parameter values, come close to any of the solutions that you
have found? If the periodic orbit is embedded into a chaotic attractor,
typical soultion visits it infinitely often, infinitesimally close.

--vanessa:
This is something that I have not examined in detail.  I have observed (using
the ``naked eye") that the time evolution of typical trajectories (when viewed
by plotting the real versus imaginary part of A(x,t) at different times, as in
Figures 3,5,7 from the paper) sometimes resembles that of the first solution
found (i.e., the patterns displayed in Figure 3). (The first solution
found also happened to be the solution to which the solver I used
converged to most frequently.)  But at this point I do not have a
quantitative measure of this ``resemblance".

After talking to Lopez at SIAM DS05 meeting - she did something that
is useful to us (found some relative periodic orbits), but I do not think
they are the dynamically important ones, so the problem remains wide open.

And regrettably she used the nonsensical formulas of Zoldy, so that will
just keep confusing future readers. 
She is in computer science, so cannot
blame a graduate student for trying a formula published in Phys Rev Lett.

I like some of your speculations here, perhaps we can include them in the
proceedings version of the KS paper. That motivated me to read through
Vanessa Lopez's paper (clearly a summary of a PhD thesis), and while I
a like relative periodic orbits, I am very worried about the general drift
of the paper.

As I cannot tell how exhaustive is her set of numerical solutions, I do
not know what to make of them - like ZG, she gets all of them with a large
number unstable dimensions. Presumably none of them are close to the
asymptotic inertial manifold. More worrisome still, she uses ZG to
``average" over periodic orbits. That is regrettable - I hoped I would
never see that stuff again. Dettmann and I tried to get Zoldi to
derive this for us, and Mainieri actually hired him at Los Alamos to learn
how this works. We are all convinced that it makes no sense whatsoever.

If she is right and CGL at her parameter values does not contain
(relative) periodic orbits with 1 or very few unstable directions, we can
kiss the ``minimal instability principle" goodbye.

I did not say that their
priodic orbits captured the main dynamics of the system. In fact, in the full
space I believe that the relative periodic orbits are the more commonly
encountered objects in the phase space, just as the quasi-periodic motion is a
more frequent occurrence on an invariant torus. In a system with continous
symmetry, any orbit corresponds to a continous family of orbits, so the KS
system with periodic boundary condition is born on some kind of torus. We can
either reduce the system so that the continuous symmetry does not exist, or
study relative periodic orbits. Relative periodic orbits correspond to periodic
orbits in the reduced system. I don't know how to reduce the KS system in a
beautiful way.


However, I would like to talk to you about how to redo your calculations
as relative periodic orbits, follwoing Vanessa Lopez. I believe you do not
need to compute the whole torus. It should be much faster and more
accurate, if I am right.

      In the case of relative periodic orbits, we can find a piece of orbit on
the torus and construct the whole torus by KNOWN symmetry operations in the
PHASE SPACE. For the general torus, we don't have this symmetry. 

        John, Jonathan, Vaggelis and Rytis have been pondering role of
continuous symmetries (simple ones - translations, periodic domains) in
periodic orbit theory. One possibility seems to be that one needs to
generalize notion of ``periodic" to ``relative periodic" (equvariant,
travelling, running are other names I have heard), and recently
we have received results from Viswanath (who finds such in plane Couette)
and Davidchack (who finds such in Kuramoto-Sivshinsky on a periodic
domain).

        We are confused, and this morning I believe that generically such
solutions are quasiperiodic, which, if true, is a bit of trouble. Would
prefer not to say stupid things to Viswanath and Davidcheck on the topic.

        Unprepared as I am, I could discuss these continuous symmetries as
a small-step limit of discrete symmetries. I would describe

\begin{enumerate}
\item
        Symmetries of 3-disk
\item
        $C_n$ symmetry of $n$-slice pizza without reflection symmetry
\item
        Fourier analysis of periodic lattices
\item
        (perhaps) running modes in periodic lattice deterministic
           diffusion
\end{enumerate}

        If you fall for this, the reading material for 1)-3) is on
http://www.cns.gatech.edu/courses/chaosSpring06


As far as the discussion about ``drift" is concerned - Divakar might be
right. In presence of continuous symmetries (for PC, stremawise and
spanwise translations) one should be searching for relative (equvariant)
equilibria and periodic orbits - they are most likely more important than
the stationary equilibria and spatially periodic orbits. I do not
understand PC well enough to have an argument to exclude them.

In my Kuramoto-Sivashinsky work with Christiansen, Putkaradze and Lan we
had excluded them ``by hand", by concentrating only on the space of
antisymmetric solutions. That is not good physics, as perturbations off
them mix into the full space of asymmetric solutions. One part of our
proposed FRG would be a search for relative periodic orbits of
Kuramoto-Sivashinsky, with my student Evangelos Siminos.

\subsection{F Laurent-Polz: Relative periodic orbits in point vortex systems}

We give a method to determine relative periodic orbits in point vortex systems: it consists mainly into perform a symplectic reduction on a fixed point submanifold in order to obtain a two-dimensional reduced phase space. The method is applied to point vortices systems on a sphere and on the plane, but works for other surfaces with isotropy (cylinder, ellipsoid, ...). The method permits also to determine some relative equilibria and heteroclinic cycles connecting these relative equilibria.

    Relative equilibria are orbits of the symmetry group action which are in-
variant under the flow, this corresponds here to motions of point vortices which
are stationary in a steadily rotating frame. We first review the literature about
the point vortex system on the sphere. The existence and nonlinear stability
of relative equilibria formed of three vortices have been studied respectively in
[KN98] and [PM98]. 

    Relative periodic orbits (RPOs for short) are the analogous of relative equi-
libria concerning periodic orbits, this corresponds here to motions which are
periodic in a steadily rotating frame (a precise deinition is given in Section
2). Periodic orbits on the sphere were determined in [ST, To01] thanks to the
following method: they reduced the system to two-dimensional systems by a
symmetric reduction (using some finite subgroups of $SO(3)$); the computation
of the dynamics on the reduced spaces permits then to determine periodic orbits.
Our paper is devoted to transpose that method to determine relative periodic
orbits. To this end, we combine a symmetric reduction together with a sym-
plectic reduction. The method is explained in details in Section 2, it permits
to determine periodic orbits, RPOs, and heteroclinic cycles connecting relative

[Third Referee's Report Dr Newton]:
The manuscript considers relative periodic orbits in point vortex systems
that arise from splitting relative equilibrium configurations. It more
comprehensively treats problems of the type considered in Souliere and
Tokieda (2002) and Tokieda (2001). It is a very nice paper.

