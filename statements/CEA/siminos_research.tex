\documentclass[a4paper,10pt]{article}

\usepackage{hyperref}
\usepackage{url}
\usepackage{color}

\pagestyle{plain}

\setlength{\oddsidemargin}{0in}\setlength{\evensidemargin}{\oddsidemargin}
\setlength{\textwidth}{1.3\textwidth}
\setlength{\textheight}{1.1\textheight}
\voffset-0.3in
\hoffset-0.2in
\topmargin 0.0in
\headsep 0.0in
\headheight 0.0in

%opening
\title{Research Plan}
\author{Evangelos Siminos}
\date{January 2011}

\definecolor{darkblue}{rgb}{0.13,0.17,0.63}

\hypersetup{colorlinks=true,urlcolor=darkblue}
\definecolor{see}{rgb}{0.5,0.5,0.5}% for web links

% ES copied from cv_libertine
\newcommand{\html}[1]{\href{#1}{\color{see}\scriptsize\textsc{[html]}}}
\newcommand{\pdf}[1]{\href{#1}{\color{see}\scriptsize\textsc{[pdf]}}}
\newcommand{\doi}[1]{\href{#1}{\color{see}\scriptsize\textsc{[doi]}}}


\begin{document}

\maketitle

My research interests lie in the area of nonlinear dynamics of 
complex systems. My focus is on the application of analytical and
computational methods of dynamical systems' theory to problems
involving diverse physical settings. Recent research topics include
collective effects in systems with long-range interactions (plasmas),
relativistic intensity laser-plasma interaction and 
the state-space geometry of models of phase turbulence.

As a postdoc at CEA/Bruy\`{e}res-le-Ch\^{a}tel, I am currently studying the
kinetic theory description and nonlinear saturation mechanism of stimulated 
Raman scattering (SRS), a process that involves the interaction of a laser pulse 
and a nonlinear electrostatic wave in a plasma. SRS is a major source of energy 
loss for current inertial confinement fusion efforts, while
a theoretical understanding of the saturation of its growth is still missing.
My main contribution was to develop, in collaboration with D. B\'enisti and L.
Gremillet, a semi-analytic eigenproblem formulation which 
allowed us to reduce the problem to the study of the stability and 
collective modes of a nonlinear equilibrium configuration. Preliminary results
indicate that nonlinear SRS saturation can be understood through this approach.
At the same time, our method is general and robust enough to 
find application in the description of collective modes in other systems with
long-range interactions. %, even when there is an extreme range of scales involved.
 
As a parallel effort at CEA, I've studied
 a different problem of great 
practical interest in the field of laser-matter interaction. 
A growing trend is to develop compact accelerators that
exploit short, high-intensity laser pulses to accelerate particles
in much higher energies than possible with conventional accelerators 
of comparable size. However, short laser pulses are hard to produce 
and control and thus an open question is whether longer pulses can be utilized 
for the same purpose of particle acceleration. A first step to tackle the
problem is to connect it to the existence of solitary waves 
in a fluid-plasma model. With G. S\'anchez-Arriaga and E. Lefebvre, we reduced
the problem to that of finding homoclinic and heteroclinic connections
of a Hamiltonian system of ordinary differential equations, leading to a
systematic determination and classification
of solitary wave solutions, through very general geometrical arguments of
the theory of dynamical systems. 
We are currently working on the problem
of stability and practical excitation of such solutions in laser-matter
interaction.

My earlier PhD thesis work at the Georgia Institute of Technology
involved the study, with P. Cvitanovi\'c and R. L. Davidchack, 
of the Kuramoto-Sivashinsky system, which emerges in the description
of phase turbulence.
In dynamical systems' studies of turbulence the goal is to understand
the organization of the infinite-dimensional state space 
in terms of equilibria, periodic orbits and their
stable and unstable manifolds, which serve to connect local neighborhoods.
However, when traveling wave solutions are taken into account, the geometric
approach is obscured by the presence of equivalent (up to a continuous symmetry
transformation) trajectories. The main achievement of my
thesis was to show that continuous symmetry
reduction (the identification of symmetry related trajectories) can be
implemented in a high-dimensional state-space and that the
geometric approach of dynamical systems can then be successfully applied.

My perspective for future research is still towards cross-fertilization
between different areas of physics involving the study of complex systems, 
through the unifying framework of dynamical systems. I believe that
CEA-Saclay offers an ideal environment for such studies, given the wide
range of research topics covered, the interdisciplinarity of the research groups
and the close interaction of theorists and experimentalists. In particular
the research groups at the \textit{Service de Physique de l'Etat Condens\'{e}} are
especially attractive for someone with my research profile, featuring research
that I would be excited to get involved into.



% \bibliography{../../bibtex/plasmas}
% \bibliographystyle{unsrt}


\end{document}

