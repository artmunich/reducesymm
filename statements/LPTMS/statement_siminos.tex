%% start of file `template_en.tex'.
%% Copyright 2007 Xavier Danaux (xdanaux@gmail.com).
%
% This work may be distributed and/or modified under the
% conditions of the LaTeX Project Public License version 1.3c,
% available at http://www.latex-project.org/lppl/.


\documentclass[11pt,a4paper,final]{moderncv}

% moderncv themes
% \moderncvtheme[blue]{casual}                 % optional argument are 'blue' (default), 'orange', 'red', 'green', 'grey' and 'roman' (for roman fonts, instead of sans serif fonts)
\moderncvtheme[blue]{classic}                % idem

% character encoding
\usepackage[utf8]{inputenc}                   % replace by the encoding you are using

% 
% \usepackage{hyperref} %ES
\usepackage{url} %ES
\usepackage{multicol} %ES
\usepackage{textcomp} %ES

%% Define a new 'leo' style for the package that will use a smaller font.
\makeatletter
\def\url@leostyle{%
  \@ifundefined{selectfont}{\def\UrlFont{\sf}}{\def\UrlFont{\small\ttfamily}}}
\makeatother
%% Now actually use the newly defined style.
% \urlstyle{leo}



% adjust the page margins
\usepackage[scale=0.8]{geometry}
\recomputelengths                             % required when changes are made to page layout lengths


% ES
\newcommand{\siminos}{\textsc{E. Siminos}}
\newcommand{\emtitle}{\em}
\newcommand{\emjournal}{}
\definecolor{darkblue}{rgb}{0.13,0.17,0.63}
\newcommand*{\cvitemSC}[2]{\cvline{\textsc{\footnotesize #1}}{#2}}



% \hypersetup{colorlinks=true,urlcolor=darkblue}
\definecolor{see}{rgb}{0.13,0.17,0.63} %{0.5,0.5,0.5}% for web links

% ES copied from cv_libertine
\newcommand{\html}[1]{\href{#1}{\color{see}\scriptsize\textsc{[html]}}}
\newcommand{\pdf}[1]{\href{#1}{\color{see}\scriptsize\textsc{[pdf]}}}
\newcommand{\doi}[1]{\href{#1}{\color{see}\scriptsize\textsc{[doi]}}}


% personal data
\firstname{\Huge{Evangelos}}
\familyname{\Huge{Siminos}}
 \title{Research Statement}               % optional, remove the line if not wanted
\address{D\'{e}partement de Physique Th\'{e}orique et Appliqu\'{e}e\\ 
% Commissariat \`a l' \'Energie Atomique\\ 
CEA, DAM, DIF}{ F-91297 Arpajon, France}    % optional, remove the line if not wanted
% \mobile{mobile (optional)}                    % optional, remove the line if not wanted
\phone{+33 169 267 361}                      % optional, remove the line if not wanted
% \fax{fax (optional)}                          % optional, remove the line if not wanted
\email{siminos@gatech.edu}% optional, remove the line if not wanted
% \email{evangelos.siminos@cea.fr}% optional, remove the line if not wanted
\extrainfo{\href{http://www.cns.gatech.edu/~siminos}{\url{www.cns.gatech.edu/~siminos}}} % 
% \extrainfo{\httplink[http://www.cns.gatech.edu/~siminos]{www.cns.gatech.edu/~siminos}} % optional, remove the line if not wanted
% \photo[80pt]{siminos_small.jpg}                         % '64pt' is the height the picture must be resized to and 'picture' is the name of the picture file; optional, remove the line if not wanted
% \quote{Some quote (optional)}                 % optional, remove the line if not wanted

%\nopagenumbers{}                             % uncomment to suppress automatic page numbering for CVs longer than one page


%----------------------------------------------------------------------------------
%            content
%----------------------------------------------------------------------------------
\begin{document}
\maketitle

\setlength{\parindent}{0.25in} % ES: For research statement. After maketitle!

My research interests lie in the area of nonlinear dynamics of spatially
extended systems. Recent projects include the study of spatio-temporally 
chaotic flows, stability and collective effects in systems with long-range 
interactions, and soliton formation in laser-plasma interaction. The 
unifying framework of these studies is the geometrical perspective
of dynamical systems theory.
% interactions. 
% I use both analytical and numerical tools, with a strong  conviction that
% the numerical part should be based on (and provide additional) 
% geometrical and physical insight. 

My PhD thesis work at the Georgia Institute of Technology, with Prof. Cvitanovi\'c,
focused on the interplay between symmetry and nonlinear dynamics
in spatially extended systems. It is part of a broader research effort 
with the goal of a geometric understanding of dissipative, turbulent flows with coherent structures,
as dynamical systems in a high-dimensional state space. This would in turn
naturally lead to the quantitative prediction of statistical averages with tools
(such as trace formulas) that relate the spectrum of transfer operators to the
spectrum of periodic orbits. However, when dynamics admits a continuous
symmetry, the geometric picture is often obscured by the presence of equivalence
classes of solutions. To alleviate this difficulty, I worked on symmetry
reduction for Lie groups acting on high-dimensional spaces, a problem
for which classical Hilbert basis approaches are not applicable. One of the
main results is an explicit invariant basis for symmetry reduction in the
Kuramoto-Sivashinsky partial differential equation, which allowed an elucidation
of the unexpected role the unstable manifolds of certain traveling wave 
solutions play in organizing the global geometry. 

As a postdoc at CEA/Bruy\`{e}res-le-Ch\^{a}tel, I moved to the study of kinetic 
theory and specifically of the Vlasov-Poisson system (collisionless Boltzmann 
equation coupled to an electric field in a plasma) in relation to 
collective phenomena in laser-plasma interaction problems. 
My interest here is in transferring tools from dissipative
systems (such as Galerkin projection onto a finite dimensional dynamical system)
to the study of spatially extended systems endorsed with a Hamiltonian
structure. As a first step I've studied the stability of nonlinear plasma waves,
a problem which is of great practical interest to current inertial
confinement fusion efforts, but for which no well established and universal
method was available. In the process of developing a semi-analytic method
allowing to project the problem to a finite-dimensional basis, certain
convergence issues arose, owing to the Hamiltonian, time-reversible character of
the underlying linear operator. These were resolved using operator-theoretic
techniques (spectral deformation) and led to a fast converging and general
scheme for the computation of stability of nonlinear waves of the Vlasov-Poisson
system and the associated unstable, collective modes. We are currently working 
towards applying our method to the problem of nonlinear saturation of stimulated 
Raman scattering, a process detrimental to inertial confinement fusion.

For future research, I have developed an interest in studying collective 
effects in many body quantum systems from a semiclassical 
perspective. As the emphasis in studies of many body systems increasingly shifts 
towards the mesoscopic scale, an appropriate and useful level of description for many 
applications (e.g. electron dynamics in metal films and metal clusters, 
quantum plasmas) is within a kinetic model that accounts for the 
quantum character of the underlying system. 
Having been trained in the use of trace formulas and related
tools in classical, high-dimensional flows, 
a very natural step for me is to move from 
classical kinetic theory towards the quantum level within the semiclassical
 approximation, which utilizes the same tools.
Given the rich activity of LPTMS members on semiclassical theory 
and statistical mechanics, this would be an ideal place for me to embark
into this research area and greatly widen my scope.


%This is of interest in many different contexts in mesoscopic physics, 
%including electron dynamics in thin metal films and metal clusters, 
%quantum plasmas, .

\end{document}


%% end of file `template_en.tex'.
