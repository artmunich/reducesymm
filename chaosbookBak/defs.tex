% defs.tex                           Chaos Course project template
% Predrag                            Nov  2 2003
% Phys Rev E style 
% 
%%% Definitions file for MyNamePHYS7123.tex project

%%%%%%%%%%%%% SETUP          %%%%%%%%%%%%%%%%%%%

	\usepackage{amssymb}
	\usepackage{amsmath}
	\usepackage{graphicx}
	\usepackage{subfigure}
	\usepackage{dsfont}
	\usepackage{mathrsfs}
	\usepackage{natbib}       % required by apsrev.bst
	\usepackage[dvips]{color} % dvips allows for colors
	\bibliographystyle{apsrev}

%%%%%%%%%%%%% COMMENTS, editing specific %%%%%%%%%%%%%%%%%%%

	% in the final version, use these:
% \newcommand{\YourInitials}[1]{}
% \newcommand{\PC}[1]{}
% \newcommand{\fix}[1]{}
% \newcommand{\file}[1]{}
	%

	% while editing, keep these
\newcommand{\YourInitials}[1]{$\footnotemark\footnotetext{YL: #1}$}
\newcommand{\PC}[1]{$\footnotemark\footnotetext{PC: #1}$}
\newcommand{\fix}[1]			% purple text = needs to be rewritten
             {{\color{magenta} #1 }}
\newcommand{\file}[1]{$\footnotemark\footnotetext{{ file:} #1}$}
	%

%%%%%%%%%%%%	SHORTCUTS, project specific %%%%%%%%%%%%%

%%%%%%%%%%%%    CROSS REFERENCING, STANDARD    %%%%%%%%%%%%%%%%%

\newcommand{\rf}      [1] {~\cite{#1}}
\newcommand{\refref}  [1] {ref.~\cite{#1}}
\newcommand{\refRef}  [1] {Ref.~\cite{#1}}
\newcommand{\refrefs} [1] {refs.~\cite{#1}}
\newcommand{\refRefs} [1] {Refs.~\cite{#1}}
\newcommand{\refeq}   [1] {(\ref{#1})}
\newcommand{\refeqs}  [2] {(\ref{#1}--\ref{#2})}
\newcommand{\refpage} [1] {page~\pageref{#1}}
	% Phys Rev style: Figure to start a sentence, else Fig.
\newcommand{\reffig}  [1] {Fig.~\ref{#1}}
\newcommand{\reffigs} [2] {Figs.~\ref{#1} and~\ref{#2}}
\newcommand{\refFig}  [1] {Figure~\ref{#1}}
\newcommand{\refFigs}  [2] {Figures~\ref{#1} and~\ref{#2}}
\newcommand{\reftab}  [1] {Table~\ref{#1}}
\newcommand{\refTab}  [1] {Table~\ref{#1}}
\newcommand{\reftabs} [2] {Tables~\ref{#1} and~\ref{#2}}
\newcommand{\refsect} [1] {sect.~\ref{#1}}
\newcommand{\refsects}[2] {sects.~\ref{#1}--\ref{#2}}
\newcommand{\refSect} [1] {Sect.~\ref{#1}}
\newcommand{\refSects}[2] {Sects.~\ref{#1}--\ref{#2}}
\newcommand{\refappe} [1] {appendix~\ref{#1}}
\newcommand{\refappes}[2] {appendices~\ref{#1}--\ref{#2}}
\newcommand{\refAppe} [1] {Appendix~\ref{#1}}

%%%%%%%%%%%%%%% EQUATIONS, STANDARD %%%%%%%%%%%%%%%%%%%%%%%%%%%%%%%

\newcommand{\beq}{\begin{equation}}
\newcommand{\eeq}{\end{equation}}
\newcommand{\ee}[1] {\label{#1} \end{equation}}
\newcommand{\bea}{\begin{eqnarray}}
\newcommand{\ceq}{\nonumber \\ & & }
\newcommand{\continue}{\nonumber \\ }
\newcommand{\nnu}{\nonumber}
\newcommand{\eea}{\end{eqnarray}}


%%%%%%%%%%%%%%% VECTORS, MATRICES, STANDARD %%%%%%%%%%%%%%%%%%%
\newcommand{\MatrixII}[4]{
   \begin{pmatrix} 
             {#1}  &  {#2} \cr
             {#3}  &  {#4} \cr    
   \end{pmatrix}     
                         }
\newcommand{\MatrixIII}[9]{
   \begin{pmatrix} 
	      {#1}  &  {#2} &  {#3} \cr
             {#4}  &  {#5} &  {#6} \cr
             {#7}  &  {#8} &  {#9} \cr
   \end{pmatrix}
			      }
\newcommand{\transpVectorII}[2]{
   \begin{pmatrix} 
             {#1}  &  {#2}  \cr    
   \end{pmatrix}     
                               }
\newcommand{\VectorII}[2]{
   \begin{pmatrix} 
             {#1} \cr
             {#2} \cr    
   \end{pmatrix}     
                         }
\newcommand{\VectorIII}[3]{
   \begin{pmatrix} 
             {#1} \cr
             {#2} \cr
             {#3} \cr    
   \end{pmatrix}     
                         }


%%%%%%%%%%%%%%  Abbreviations %%%%%%%%%%%%%%%%%%%%%%%%%%%%%%%%%%%%%%%%

\newcommand{\etc}{{\em etc.}}		% etcetera in italics
\newcommand{\ie}{{that is}}		% use Latin or English?  Decide later.
\newcommand{\cf}{{\em cf.}}
					% keep homepage flexible:
\newcommand{\wwwcb}{{\tt ChaosBook.org}}

%%%%%%%%%%%%%%% Sundry symbols within math eviron.: %%%%%%%%%%%%

\newcommand{\obser}{a}		% an observable from phase space to R^n
\newcommand{\Obser}{A}		% time integral of an observable
\newcommand{\onefun}{\iota}	% the function that returns one no matter what
\newcommand{\defeq}{=}		% the different equal for a definition
\newcommand {\deff}{\stackrel{\rm def}{=}}
\newcommand{\reals}{\mathbb{R}}
\newcommand{\complex}{\mathbb{C}}
\newcommand{\integers}{\mathbb{Z}}
\newcommand{\rationals}{\mathbb{Q}}
\newcommand{\naturals}{\mathbb{N}}
\newcommand{\LieD}{{{\cal L}\!\!\llap{-}\,\,}}  % {{\pound}} % Lie Derivative 
\newcommand{\half}{{\scriptstyle{1\over2}}}
\newcommand{\pde}{\partial}
\renewcommand\Im{{\rm Im\,}}
\renewcommand\Re{{\rm Re\,}}
\renewcommand{\det}{\mbox{\rm det}\,}
\newcommand{\Det}{\mbox{\rm Det}\,}
\newcommand{\tr}{\mbox{\rm tr}\,}
\newcommand{\Tr}{\mbox{\rm tr}\,}
%\newcommand{\Tr}{\mbox{Tr}\,}
\newcommand{\sign}[1]{\sigma_{#1}}
%\newcommand{\sign}[1]{{\rm sign}(#1)}
\newcommand{\msr}{{\rho}}		        % measure 
\newcommand{\Msr}{{\mu}}		        % coarse measure 
\newcommand{\dMsr}{{d\mu}}		        % measure infinitesimal
\newcommand{\SRB}{{\rho_0}}		        % natural measure 
\newcommand{\vol}{{V}}		        	% volume of i-th tile
\newcommand{\prpgtr}[1]{\delta\negthinspace\left( {#1} \right)}
\newcommand{\zetaInv}{{1/\zeta}}
\newcommand{\expct}    [1]{\left\langle {#1} \right\rangle}
\newcommand{\spaceAver}[1]{\left\langle {#1} \right\rangle}
\newcommand{\timeAver} [1]{\overline{#1}}
\newcommand{\pS}{{\cal M}}			% symbol for phase space 
\newcommand{\NWS}{\Omega}	   % symbol for the non--wandering set
\newcommand{\intM}[1]{{\int_\pS{\!d #1}\:}}	%phasespace integral
\newcommand{\Cint}[1]{\oint\frac{d#1}{2 \pi i}\;} %Cauchy contour integral
\newcommand{\PoincS}{{\cal P}}                  % symbol for Poincare section
\newcommand{\PoincM}{{P}}			% symbol for Poincare map
\newcommand{\Lop}{{\cal L}}	   % evolution operator
\newcommand{\Aop}{{\cal A}}	   % evolution generator
\newcommand{\matId}{{\bf 1}}	   % matrix identity
\newcommand{\Mvar}{{A}}	   	   % matrix of variations
\newcommand{\jMps}{{J}}	   % loop.tex version jacobiam matrix, full phase space
% \newcommand{\jMps}{{\bf J}}	   % jacobiam matrix, full phase space
\newcommand{\jMConfig}{{\bf j}}	   % jacobiam matrix, configuration space
\newcommand{\jConfig}{j}	   % jacobian, configuration space
\newcommand{\monodromy}{{\bf J}}   % monodromy matrix, full Poincare cut
				   % Fredholm det jacobian weight:
\newcommand{\oneMinJ}[1]{\left|\det\left(\matId-\monodromy_p^{#1}\right)\right|}
\newcommand{\ExpaEig}{\Lambda}	   
\newcommand{\eigenvL}{{s}}	   
\newcommand{\inFix}[1]{{\in \mbox{\footnotesize Fix}f^{#1}}}
\newcommand{\inZero}[1]{{\in \mbox{\footnotesize Zero} \, f^{#1} }}
\newcommand{\xzero}[1]{{x_{#1}^{*}}}

%%%%%%%%%% flows: %%%%%%%%%%%%%%%%%%%%%%%%%%%%

\newcommand\flow[2]{{f^{#1}(#2)}}
\newcommand\invFlow{F}
\newcommand\hflow[2]{{\hat{f}^{#1}(#2)}}
\newcommand\timeflow{{f^t}}
\newcommand\tflow[2]{{\tilde{f}^{#1}(#2)}}
\newcommand\xInit{{x_0}}		%initial x
\newcommand\pSpace{x}		% phase space x=(q,p) coordinate
\newcommand\coord{q}		% configuration space p coordinate
\newcommand{\para}{\parallel}
\newcommand\stagn{q}		%equilibrium/stagnation point suffix

%%%%%%%%%% periods: %%%%%%%%%%%%%%%%%%%%%%%%%%%%

\newcommand\period[1]{{T_{#1}}}			%continuous cycle period
%\newcommand\period[1]{{\tau_{#1}}}
\newcommand{\cl}[1]{{n_{#1}}}	% discrete length of a cycle, Predrag
%\newcommand{\cl}[1]{|#1|}	% the length of a periodic orbit, Ronnie
\newcommand{\nCutoff}{N}	% maximal cycle length
\newcommand{\timeSegm}[1]{{\tau_{#1}}}		%billiard segment time period
\newcommand{\timeStep}{{\delta \tau}}		%integration step
\newcommand{\deltaX}{{\delta x}}		%trajectory displacement
\newcommand{\unitVec}{\hat{n}}		%unit vector
\newcommand\Lyap{\lambda}			%Lyapunov exponent
\newcommand\LyapTime{T_{\mbox{\footnotesize Lyap}}}	%Lyapunov time
\newcommand{\hatx}{{\hat{x}}}


%%%%%%%%%%%%%% Evangelos additions %%%%%%%%%%%%%%%%%

\newcommand\fp[2]{{\frac{\partial #1}{\partial #2}}}
\newcommand\fd[2]{{\frac{d #1}{d #2}}}
\newcommand\fsd[2]{{d #1/d #2}}
\newcommand\fsp[2]{{\partial #1/\partial #2}}
\newcommand\fps[3]{{\frac{\partial^2 #1}{\partial #2 \partial #3}}}
\newcommand\fsps[3]{{\partial^2 #1/\partial #2 \partial #3}}

\newcommand\Js{\mathbf{\tilde{J}}}
\newcommand\JL{\mathbf{\tilde{J}}_L}
\newcommand\Jp{\mathbf{J}_p}


%%%%%%%%%%%%%%%%%%%%%%%%%%%%%%%%%%%%%%%%%%%%%%%%%%%%





