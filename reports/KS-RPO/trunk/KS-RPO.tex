% this style for submission, web version:
%\documentclass[pre,twocolumn,groupedaddress,showpacs,showkeys]{revtex4}

% this style while editing:
\documentclass[pre,preprint]{revtex4}%,groupedaddress,showpacs,showkeys]

\input colordvi

\input defs         % all definitions in defs.tex
                \begin{document}
                \title{
                   Relative periodic orbits in Kuramoto-Sivashinsky equation
                 }
%                 \author{
%                             Evangelos Siminos
%                         }
%                 \email{gtg083n@mail.gatech.edu}
% 
% 
%                 \affiliation{
%         School of Physics\\
%         Georgia Institute of Technology, Atlanta, GA 30332-0430, U.S.A}
%                 \date{\today} % or edit manually:
%                 %\date{November 11,:}
% 
% %                \begin{abstract}
% 
% %\end{abstract}
% %\pacs{95.10.Fh, 02.70.Bf, 47.52.+j, 05.45.+a}
% \keywords{
% periodic orbits,
% chaos, turbulence
%     }
%                    \maketitle
% 
% \noindent
% {\bf Georgia Tech PHYS 4421:}\\
% \underline{\bf PHYSICS OF CONTINUOUS MATTER }\\
% {\bf course project, spring semester 2004} \\
% \& {\bf  PHYS 8901:}\\
% {\bf Special Problem, summer semester 2004}


 {\bf Relative periodic orbits of Kuramoto-Sivashinsky equation}\\ \\


 The Kuramoto-Sivashinsky equation (KSe) reads:
 \beq
  u_t=(u^2)_x-u_{xx}- u_{xxxx} \, ,
  \label{eq:KS}
 \eeq

 We assume periodic boundary conditions on the $x\in [0,2\pi \tilde{L}]$
 interval:
 \beq
   u(x+2\pi\tilde{L},t)=u(x,t) \, ,
 \eeq
 which allows a Fourier series expansion:
 \beq
  u(x,t)=\sum_{k=-\infty}^{+\infty} a_k (t) e^{ i k x / \tilde{L}} \, .
  \label{eq:Fourier}
 \eeq
 Since $u(x,t)$ is real,
 \beq
  a_{k}=a^*_{-k} \, .
  \label{eq:a*}
 \eeq
 Substituting \refeq{eq:Fourier} into \refeq{eq:KS} we get:
 \beq
  \dot{a}_k=(k/\tildeL)^2\left(1-(1/\tildeL)^2 k^2\right)a_k
        + i (k/\tildeL)  \sum_{m=-\infty}^{+\infty}a_m a_{k-m} \, .
  \label{eq:Fcoef}
 \eeq

 From \refeq{eq:Fcoef} we notice that $\dot{a}_0=0$ and thus $a_0$ is an integral
 of the equations or, from \refeq{eq:Fourier}, the average of the solution $\int dx u(x,t)$
 is a constant. Due to galilean invariance we may set $a_0=0$ without loss of generality 
 and we only have to compute $a_k$'s with $k\geq 1$. % Explain this in detail somewhere.

 Truncating the infinite tower of equations by setting $a_k=0$ for $k>d$, using the identity $a_{-k}=a^*_k$ and splitting the
 resulting equations into real and imaginary part by setting $a_k=b_k+i c_k$, we have
  
 \bea
  \dot{b}_k & = & \left(\frac{k}{\tildeL}\right)^2\left(1- \left(k/\tildeL\right)^2 \right)b_k  \continue
	& & - \frac{k}{\tildeL} \left(\sum_{m=1}^{k-1}c_m b_{k-m}+\sum_{m=k+1}^{N}c_m b_{m-k}
                    -\sum_{m=1}^{N-k}c_m b_{k+m} \right)  \continue
	& & - \frac{k}{\tildeL} \left(\sum_{m=1}^{k-1}b_m c_{k-m}-\sum_{m=k+1}^{N}b_m c_{m-k}
                    +\sum_{m=1}^{N-k}b_m c_{k+m} \right)		  
  \label{eq:tmp:b-Trunc}
 \eea
 \bea
   \dot{c}_k & = & \left(\frac{k}{\tildeL}\right)^2\left(1- \left(k/\tildeL\right)^2 \right)c_k  \continue
	& & - \frac{k}{\tildeL}\left( \sum_{m=1}^{k-1}c_m c_{k-m}-\sum_{m=k+1}^{N}c_m c_{m-k}
                    -\sum_{m=1}^{N-k}c_m c_{k+m} \right)	\continue
	& & - \frac{k}{\tildeL} \left(-\sum_{m=1}^{k-1}b_m b_{k-m}-\sum_{m=k+1}^{N}b_m b_{m-k}
                    +\sum_{m=1}^{N-k}b_m b_{k+m} \right)
   \label{eq:tmp:c-Trunc}
 \eea
 where now only terms $c_{k},b_{k}$ with $0<k<d$ appear. Observe
 \beq
	\sum_{m=1}^{N-k}c_m b_{k+m} = \sum_{m=k+1}^{N}b_m c_{m-k}\,,
 \eeq
 \etc and thus \refeq{eq:tmp:b-Trunc} and \refeq{eq:tmp:c-Trunc} simplify to
  \bea
  \dot{b}_k & = & \left(\frac{k}{\tildeL}\right)^2\left(1- \left(k/\tildeL\right)^2 \right)b_k  \continue
	& & - \frac{k}{\tildeL} \left(\sum_{m=1}^{k-1}c_m b_{k-m}-2\sum_{m=1}^{N-k}c_m b_{k+m} \right)  \continue
	& & - \frac{k}{\tildeL} \left(\sum_{m=1}^{k-1}b_m c_{k-m}+2\sum_{m=1}^{N-k}b_m c_{k+m} \right)		  
  \label{eq:b-Trunc}
 \eea
 \bea
   \dot{c}_k & = & \left(\frac{k}{\tildeL}\right)^2\left(1- \left(k/\tildeL\right)^2 \right)c_k  \continue
	& & - \frac{k}{\tildeL}\left( \sum_{m=1}^{k-1}c_m c_{k-m}-2\sum_{m=1}^{N-k}c_m c_{k+m} \right)	\continue
	& &  +\frac{k}{\tildeL} \sum_{m=1}^{k-1}b_m b_{k-m}
   \label{eq:c-Trunc}
 \eea

 We begin by calculating the matrix of variations $A_{ij} \equiv \frac{\partial v_i(x)}{\partial x_j}$ for the antisymmetric
 subspace for which $b_k=0, c_{-k}=-c_{k}$ and thus
 \beq
	   \dot{c}_k =  \left(\frac{k}{\tildeL}\right)^2\left(1- \left(k/\tildeL\right)^2 \right)c_k
	 		- \frac{k}{\tildeL}\left( \sum_{m=1}^{k-1}c_m c_{k-m}
                    		-2\sum_{m=1}^{N-k}c_m c_{k+m} \right)	\,.	
 \eeq
 
 Then
 \bea
	\frac{\partial \dot{c}_k}{\partial c_{j}}  =  
		\left(\frac{k}{\tildeL}\right)^2\left(1- \left(k/\tildeL\right)^2 \right) 
			- \frac{k}{\tildeL}\frac{\partial}{\partial c_j}\left( \sum_{m=1}^{k-1}c_m c_{k-m}-2\sum_{m=1}^{N-k}c_m c_{k+m} \right)	\,.	
 \eea
 Concider the second term:
 \bea
	- \frac{k}{\tildeL}\frac{\partial}{\partial c_j}\left( \sum_{m=1}^{k-1}c_m c_{k-m}-2\sum_{m=1}^{N-k}c_m c_{k+m} \right)	& = &
		- \frac{k}{\tildeL} \sum_{m=1}^{k-1} \left(\delta_{m,j} c_{k-m}+c_m \delta_{k-m,j} \right) \continue
						& & + 2 \frac{k}{\tildeL}\sum_{m=1}^{N-k} \left(\delta_{m,j} c_{k+m}+c_m \delta_{k+m,j}\right)
 \eea
 We need to consider two cases seperately:
 \begin{itemize} 
	\item $k\leq j$
		\bea
			 -\frac{k}{\tildeL}\frac{\partial}{\partial c_j}\left( \sum_{m=1}^{k-1}c_m c_{k-m}-2\sum_{m=1}^{N-k}c_m c_{k+m} \right)	& = &
					-\frac{k}{\tildeL}( 0+0 ) + 2\frac{k}{\tildeL} (c_{k+j} + c_{j-k}) \continue
				& = &   2 \frac{k}{\tildeL} (c_{k+j}-c_{k-j})
		\eea
	\item $k > j$
		\bea
			 -\frac{k}{\tildeL}\frac{\partial}{\partial c_j}\left( \sum_{m=1}^{k-1}c_m c_{k-m}-2\sum_{m=1}^{N-k}c_m c_{k+m} \right)	& = &
					-\frac{k}{\tildeL}(c_{k-j} + c_{k-j} ) + 2\frac{k}{\tildeL} (c_{k+j}  + 0 ) \continue
				& = &  2 \frac{k}{\tildeL} (c_{k+j}-c_{k-j})
		\eea	
 \end{itemize}
 and thus
 \beq
	\frac{\partial \dot{c}_k}{\partial c_{j}} =  \left(\frac{k}{\tildeL}\right)^2\left(1- \left(k/\tildeL\right)^2 \right) + 2 \frac{k}{\tildeL} (c_{k+j}-c_{k-j})
 \eeq

 For the case of the full space we need to consider the four matrices $\frac{\partial \dot{b}_k}{\partial b_j},\frac{\partial \dot{b}_k}{\partial c_j},\frac{\partial \dot{c}_k}{\partial b_j},\frac{\partial \dot{c}_k}{\partial c_j}$. Following the above procedure
 \beq
	\frac{\partial \dot{c}_k}{\partial b_{j}} = \left\{ 		\begin{array}{ll}
									0\,, 			&  k \leq j \\
									2 \frac{k}{\tildeL} b_{k-j}\,, 	&  k>j	\,,						
								\end{array}
							\right.
 \eeq
 \beq
	\frac{\partial \dot{b}_k}{\partial b_{j}} =  \left(\frac{k}{\tildeL}\right)^2\left(1- \left(k/\tildeL\right)^2 \right) - 2 \frac{k}{\tildeL} (c_{k+j} + c_{k-j}) \,,
 \eeq
 \beq
	\frac{\partial \dot{b}_k}{\partial c_{j}} = 2 \frac{k}{\tildeL} (b_{k+j}-b_{|k-j|}) \,.
 \eeq

\subsection{Implementing Newton's method for equilibria}

 Our task is to find equilibrium solutions of \refeq{eq:KS}. Although one can easy see that this problem can be reduced to that of
 finding periodic orbits of a two dimensional ODE, here we prefer to consider our system in phase space and search for solutions of
 \beq
	\dot{b}_k=\dot{c}_k=0\,,
 \eeq
 for every $k$. The reason to do this is just getting experience before pursuing the more difficult task of locating POs and RPOs. 
 Expanding $\dot{b}_k(a)$ and $\dot{c}_k(a)$ around our initial guess $a_o$ and demanding that they satisfy the equilibrium 
 condition, we get
 \bea
	\dot{b}_k(a) & = & \dot{b}_k(a_o)+\left.\frac{\partial \dot{b}_k}{\partial b_j}\right|_{a_o}\delta b_j + c\delta c_j = 0 \continue
	\dot{c}_k(a) & = & \dot{c}_k(a_o)+\left.\frac{\partial \dot{c}_k}{\partial b_j}\right|_{a_o}\delta b_j + \left.\frac{\partial \dot{c}_k}{\partial c_j}\right|_{a_o}\delta c_j = 0
 \eea
 or in matrix form
 \beq
    \left( \begin{array}{cc}
        \frac{\partial \dot{b}}{\partial b} & \frac{\partial \dot{b}}{\partial c} \\
        \frac{\partial \dot{c}}{\partial b}	& \frac{\partial \dot{c}}{\partial c}
     \end{array}
     \right)_{a_o}
     \left(\begin{array}{c}
       \delta b  \\
       \delta c
     \end{array}\right)
     =
     \left(\begin{array}{c}
       -\dot{b}(a_o) \\
       -\dot{c}(a_o)
     \end{array}\right)\,.
     \label{eq:NewtonEquil}
\eeq
where $\partial{\dot{b}} / \partial{b}$ \etc are $d \times d$ submatrices. Solving this
system of equations for the corrections $\delta b$ and  $\delta c$ and using the refined solution
as an initial guess yields  an approximation to the solution of the system.
 


\subsection{Implementing Newton's method \label{sec:NewtonMethod} for RPOs}

The relative periodic condition
\beq
	u(x+\kappa,t+T)=u(x,t) \,
\eeq
translates in Fourier space into
\beq	
	\sum_{k=-\infty}^{+\infty} a_k (t+T) e^{ i k (x+\kappa) / \tildeL} 
		= \sum_{k=-\infty}^{+\infty} a_k (t) e^{ i k x / \tildeL} \,
\eeq
or
\beq
	e^{ik\kappa/\tildeL}a_k(t+T)=a_k(t) \,,\ \forall k \in \mathds{Z}\ \ \ \mathrm{(no\ summation)}.
	\label{eq:RPOcondition}
\eeq
We see that a relative periodic orbit returns after time $T$ to a point in 
phase space with components $a_k(t+T)$ rotated in the complex plane by an 
angle $-k\kappa/\tildeL$ with respect to $a_k(t)$. In matrix notation, we write \refeq{eq:RPOcondition} as
\beq
	R(\kappa)a(t+T)=a(t)\,,
	\label{eq:RPO}
\eeq
where we have defined
\beq
	R(\kappa) \equiv Diag[e^{ik\kappa/\tildeL}]\,.
\eeq
%We notice that $R(\kappa)$ is not a rotation operator..

Consider an initial guess $a'$ for a point on a relative periodic orbit and assume that it lies on
a \Poincare section $\mathcal{P}$ at $t=0$. Suppose that $\mathcal{P}$ is a hyperplane in
$\mathds{R}^{2d}$. The flow $f^t$ defined by \refeq{eq:Fcoef} transports 
this point after time $T'$ into $a'(T')=f^{T'}(a')$. Suppose that this point is such that $R(\kappa')f^{T'}(a')$
is a point on $\mathcal{P}$. Consider next a point $a$ lying on $\mathcal{P}$ and in the neighborhood of $a'$,
thus satisfying
\beq
	q \cdot (a'-a) = 0\,,
	\label{eq:cond a}
\eeq
with $q$ a vector normal to $\mathcal{P}$. Point $a$ will be finally identified with the improved 
approximation of a point on the periodic orbit.
The flow transports $a$ to $f^{T'}(a)$, but now $R(\kappa')f^{T'}(a)$ is not in general on $\mathcal{P}$.
Moreover we would like to have the freedom to adjust the guesses for $T'$ and $\kappa'$ into new values
$T=T'+\Delta T$ and $\kappa=\kappa'+\Delta \kappa$ to improve their accuracy. 
Let as consider such slightly different values $T$ and $\kappa$ such that $R(\kappa)f^{T}(a)$ lies on 
$\mathcal{P}$. Then we have the condition
\beq
	q \cdot(R(\kappa')f^{T'}(a')-R(\kappa)f^{T}(a)) = 0\,.
	\label{eq:cond Rf(a)}
\eeq 

 We now can require that $a$ is a point on a relative periodic orbit and thus satisfies \refeq{eq:RPO}
\beq
	a=R(\kappa)f^{T}(a)\,,
	\label{eq:RPOcond}
\eeq
Taylor expanding $f^{T}(a)$ around $a'$ to linear order in the small quantities 
$\Delta a=a-a'$ and $\Delta T$, we get
\bea
	f^{T}(a)& \simeq & f^{T}(a')+\J^T(a') \Delta a \label{eq:fTaylorl1} \\ 
		& \simeq & f^{T'}(a') + v \Delta T + \J^{T'}(a') \Delta a \label{eq:fTaylorl2} \,, 
\eea
where $v$ is evaluated at $f^{T'}(a')$. Here $\J^t(x)$ is the Jacobian matrix, defined for a general flow through
\beq
   	J^t_{ij}(x_o)=\left.\frac{\partial x_i(t)}{\partial x_j}\right|_{x=x_0}\,.
\eeq
The Jacobian matrix is obtained by integrating the equation:
\beq
   	\dot{\mathbf{J}}^t=\mathbf{A J}^t \, ,
	\label{eq:Adef}
\eeq
subject to the initial condition:
\beq
   	\mathbf{J}^0=\mathbf{1} \, ,
\eeq
Here $\mathbf{A}$ is the matrix of variations defined as:
\beq
	A_{kj}=\frac{\partial \dot{x}_k}{\partial x_j}\,.
\eeq

In passing from \refeq{eq:fTaylorl1} to \refeq{eq:fTaylorl2} we have used the multiplicative 
structure of the Jacobian, $\mathbf{J}^{T'+\delta T}(a')=\mathbf{J}^{\delta T}(f^{T'}(a'))\mathbf{J}^{T'}(a')$, 
noticed that $\mathbf{J}^{\delta T}(f^{T'}(a'))=e^{\mathbf{A}\delta T}=\mathbf{1}+\mathbf{A}\delta T+\ldots$ 
and dropped second order terms in the small quantities.

On the other hand, we have
\bea
	R(\kappa'+\Delta\kappa) & = & R(\kappa')R(\Delta\kappa) \continue
				& \simeq & R(\kappa')(1+iDiag[k]\Delta\kappa/\tildeL)\,.
	\label{eq:TaylorR}	
\eea

Substituting \refeq{eq:fTaylorl2},\refeq{eq:TaylorR} into \refeq{eq:RPOcond} and keeping only first
order terms in the small quantities, we get
\beq
	a'+\Delta a \simeq R(\kappa')f^{T'}(a') + \frac{i}{\tildeL}R(\kappa')Diag[k]f^{T'}(a')\Delta\kappa 
				+ R(\kappa')v \Delta T + R(\kappa')\J^{T'}(a') \Delta a\,,
\eeq
or
\bea
	\left(1-R(\kappa')\J^{T'}(a')\right) \Delta a - R(\kappa')v \Delta T 
							- \frac{i}{\tildeL}R(\kappa')Diag[k]f^{T'}(a')\Delta\kappa  
					& \simeq & -\left(a'-R(\kappa')f^{T'}(a') \right) \continue
					& \equiv & -F(a') \,,
	\label{eq:NewtonBasicCond}			
\eea
where $F(a')$ is the function which zero we want to find.

Taylor expanding $R(\kappa)f^{T}(a)$ in \refeq{eq:cond Rf(a)} around $a'$ we get
\bea
	q \cdot \lefteqn{\left(R(\kappa')f^{T'}(a')-R(\kappa)f^{T}(a)\right) \simeq } \continue
%	 & & \left[R(\kappa')f^{T'}(a')-R(\kappa')(1+\frac{i}{\tildeL}Diag[k]\Delta\kappa) 
%			\left(f^{T'}(a') + v(a') \Delta T + \J^{T'}(a')\Delta a\right)\right] \cdot q \continue
	 & & -q \cdot \left(R(\kappa')v\Delta T +R(\kappa')\J^{T'}(a')\Delta a 
	 			+\frac{i}{\tildeL}R(\kappa')Diag[k]f^{T'}(a')\Delta\kappa \right)  = 0 \,.
	\label{eq:Taylor cond Rf(a)}
\eea 

Equations \refeq{eq:cond a}, \refeq{eq:NewtonBasicCond} and \refeq{eq:Taylor cond Rf(a)} 
can be compactly represented in a single matrix equation:
\beq
    \left( \begin{array}{ccc}
       1-R(\kappa')\mathbf{J}^{T'}(a') 	& -R(\kappa')v	  & - \frac{i}{\tildeL}R(\kappa')Diag[k]f^{T'}(a') \\
       q^{\dagger}R(\kappa')\J^{T'}(a') & q^{\dagger}R(\kappa')v & \frac{i}{\tildeL}q^{\dagger}R(\kappa')Diag[k]f^{T'}(a') \\
       q^{\dagger} 			& 0 	& 0 
     \end{array}
     \right)
     \left(\begin{array}{c}
       \Delta a \\
       \Delta T \\
       \Delta \kappa
     \end{array}\right)
     =
     \left(\begin{array}{c}
       -F(a') \\
       0     \\
       0
     \end{array}\right)\,.
     \label{eq:NewtonScheme}
\eeq
Solving this equation for the corrections $\Delta a,\ \Delta T$ and $\Delta\kappa$ yields 
an improved approximation to (a point of) the relative periodic orbit.

%The situation is similar to the one encountered when trying to identify 
%periodic orbits with Newton's method.


\end{document}
