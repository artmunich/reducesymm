\ifsvnmulti
 \svnkwsave{$RepoFile: lyapunov/dailyBlog.tex $}
 \svnidlong {$HeadURL$}
 {$LastChangedDate$}
 {$LastChangedRevision$} {$LastChangedBy$}
 \svnid{$Id$}
\fi

\chapter{Daily blog}
\label{c-DailyBlog}

\begin{bartlett}{
Je ne veux pas travailler\\
Je ne veux pas dejeuner\\
Je veux seulement oublier\\
Et puis je fume
            }
\bauthor{
\HREF{http://www.youtube.com/watch?v=MBoTRF2aK4s}
{China Forbes - Thomas M. Lauderdale (Pink Martini)}
    }
\end{bartlett}

\renewcommand{\ssp}{x}
\renewcommand{\vel}{\ensuremath{v}}   % state space velocity

% \section{Lyapunov vectors \KS}

\begin{description}

\item[2011-02-26 Predrag] Moved Lyapunov stuff to
    \refchap{s:LyapunovVec}.


\item[2011-02-22 Predrag]
So far the gap between us and Kazzanism way of thinking is as huge
as the distance from her to Kazzahstan.
Maybe start EVO group meetings?

\item[2011-02-23 Evangelos]
I would love to be able to use EVO but I've
    failed, both in the office and at home. I will arrange to meet
    physically with Kazz \etal, include you
    through EVO?

\item[2011-02-23 Predrag]
There is also a new multi-video service from Skype (paid service), not
sure whether it is comparable to EVO (EVO is real good for projection of
your presentation on other people's seminar screen).

Anyway, go talk to him - they'll have to learn quite a bit before the
collaboration would lead to something, but it would be great if we can
get together - potentially a serious step forward in (confined)
turbulence

\item[2011-03-02 Kaz] Let's start the discussion
 Monday 2011-03-07 at 13:30 (07:30 Atlanta time) at the Institut des Syst\`emes Complexes, 57-59 rue Lhomond, Paris.

\item[2011-03-02 Predrag] Kaz, please download
\HREF{http://evo.caltech.edu/evoGate/}{EVO.caltech.edu}, see whether you can
figure out how to use it - some notes are
\HREF{http://www.cns.gatech.edu/colloquia-seminars/AudioVisual.html}
{here}. Designed for scientists, it's better than skype if you want to
share presentations.

\item[2011-06-28 Evangelos] Met with Kazz at \HREF{http://cct11.cpt.univ-mrs.fr/}{CCT11 conference}
at Marseilles and discussed about their progress with Lyapunov vectors for
periodic orbits. To my understanding, the main points are: Kazz \etal, identify
generic orbits that come close to a periodic orbit. They use the vector defined
by the points of minimal distance along these two trajectories as a local approximation  
to the ``inertial manifold.'' However, some of the Lyapunov vectors computed along the periodic
orbit and the generic trajectory are not identical or even similar (at the points
of minimal distance). Kazz argues
that it is not possible to define physical and isolated modes for periodic orbits  
(in his numerics the number of physical dimensions inferred from a periodic orbit 
would be, in general, different from the number of physical dimensions inferred from
a generic orbit).

\item[2011-06-28 Evangelos] There are many thigs I do not understand from my discussion
with Kazz, so I will wait for his notes with more details.
 
\end{description}

\renewcommand{\ssp}{a}
