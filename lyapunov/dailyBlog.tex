\ifsvnmulti
 \svnkwsave{$RepoFile: lyapunov/dailyBlog.tex $}
 \svnidlong {$HeadURL$}
 {$LastChangedDate$}
 {$LastChangedRevision$} {$LastChangedBy$}
 \svnid{$Id$}
\fi

\chapter{Daily blog}
\label{c-DailyBlog}

\begin{bartlett}{
Je ne veux pas travailler\\
Je ne veux pas dejeuner\\
Je veux seulement oublier\\
Et puis je fume
            }
\bauthor{
\HREF{http://www.youtube.com/watch?v=MBoTRF2aK4s}
{China Forbes - Thomas M. Lauderdale (Pink Martini)}
    }
\end{bartlett}

\renewcommand{\ssp}{x}
\renewcommand{\vel}{\ensuremath{v}}   % state space velocity

% \section{Lyapunov vectors \KS}


This section of the blog is a general discussion.
\refChap{sect:LyapKS} deals specifically with the
\KS\ calculations.



\begin{description}

\item[2011-02-26 Predrag] Moved Lyapunov stuff to
    \refchap{s:LyapunovVec}.

\item[2011-10-05 Predrag] Moved H\'enon stuff to
    \refchap{c-Henon}.

\item[2011-02-22 Predrag]
So far the gap between us and Kazzanism way of thinking is as huge
as the distance from her to Kazzahstan.
Maybe start EVO group meetings?

\item[2011-02-23 Evangelos]
I would love to be able to use EVO but I've
    failed, both in the office and at home. I will arrange to meet
    physically with Kazz \etal, include you
    through EVO?

\item[2011-02-23 Predrag]
There is also a new multi-video service from Skype (paid service), not
sure whether it is comparable to EVO (EVO is real good for projection of
your presentation on other people's seminar screen).

Anyway, go talk to him - they'll have to learn quite a bit before the
collaboration would lead to something, but it would be great if we can
get together - potentially a serious step forward in (confined)
turbulence

\item[2011-03-02 Kazz] Let's start the discussion
 Monday 2011-03-07 at 13:30 (07:30 Atlanta time) at the Institut des
 Syst\`emes Complexes, 57-59 rue Lhomond, Paris.

\item[2011-03-02 Predrag] Kazz, please download
\HREF{http://evo.caltech.edu/evoGate/}{EVO.caltech.edu}, see whether you can
figure out how to use it - some notes are
\HREF{http://www.cns.gatech.edu/colloquia-seminars/AudioVisual.html}
{here}. Designed for scientists, it's better than skype if you want to
share presentations.

\item[2011-06-28 Evangelos]
There are many things I do not understand from my discussion with Kazz,
so I will wait for his notes with more details.

\item[2011-06-28 Predrag] If Kazz is really going to write a paper, it
would be very useful if he could learn subversion... Any chance you can
check svn is installed on his laptop, then walk him through svn checkout
(your GT account). If it works, I'll get his account processed by KGB,
but it is really not worth my time unless I know he can use it...

\item[2011-07-16 Predrag]
{\em Stabilization of long-period periodic orbits
    using time-delayed feedback control} by Claire
    Postlethwaite\rf{Postle09}
is perhaps of interest. She says: ``
The Pyragas method of feedback control has attracted much interest as a
method of stabilizing unstable periodic orbits in a number of situations.
We show that a time-delayed feedback control similar to the Pyragas
method can be used to stabilize periodic orbits with arbitrarily large
period, specifically those resulting from a resonant bifurcation of a
heteroclinic cycle. Our analysis reduces the infinite-dimensional
delay-equation governing the system with feedback to a three-dimensional
map, by making certain assumptions about the form of the solutions. The
stability of a fixed point in this map corresponds to the stability of
the periodic orbit in the flow, and can be computed analytically. We
compare the analytic results to a numerical example and find very good
agreement.
''

\item[2011-07-16 Predrag]
{\em Computation of finite time {Lyapunov} exponents using the
{Perron-Frobenius} operator}, by Phanindra Tallapragada,
\arXiv{1101.4338}. He/she says
``
    The problem of phase space transport which is of interest both
    theoretically and from the point of view of applications has been
    investigated extensively using geometric and probabilistic methods.
    Two of the important tools for this that emerged in the last decade
    are the finite time Lyapunov exponents (FTLE) and the
    Perron-Frobenius operator. The relationship between these approaches
    has not been clearly understood so far. In this paper a methodology
    is presented to compute the FTLE from the Perron-Frobenius operator,
    thus providing a step towards combining both the methods into a
    common framework.
''

                                                    \toCB
Reading it I have learned that the method of Lagrangian coherent
structures studies stretching and contraction around reference
trajectories and is therefore local in nature; it provides information
about invariant manifolds that determine transport in \statesp. The
Cauchy-Green tensor is given by
\[
C(\xInit; t_0; t) = \jMps(\xInit,t)^T \jMps(\xInit,t)
\,.
\]
$\jMps^t$ can be expressed in the singular value decomposition (SVD) form
\beq
\jMps = {U} {D}  {V}^T
\ee{SVD-j}
where ${D}$ is diagonal and real, and ${U}$, ${V}$ are orthogonal
matrices. The diagonal elements
$\sigma_{1}$, $\sigma_{2}$, $\dots$, $\sigma_{d}$ of ${D}$ are called the
\emph{singular values} of $\jMps$, namely the square root of the
eigenvalues of $\jMps^{T}\jMps = {V}{D}^{2}{V}^T$ (or $\jMps\jMps^{T} =
{U}{D}^{2}{U}^T$), which is a symmetric, positive semi-definite matrix
(and thus admits only real, non-negative eigenvalues).
The maximum growth of a
perturbation is given by the maximum principal stretch
\(
\sigma_{max}^2
\), i.e., by the maximum eigenvalue of $C$,
and the finite time Lyapunov exponent is given by
\[
\Lyap(\xInit; t_0; T) = \frac{1}{T} \log \sigma_{max}
\]

He then looks at how Perron-Frobenius operator transports
stability eigenvectors back and forth, and relates this to
the finite-time Lyapunov exponents. Might be worth a closer read.

\item[2011-08-09 Predrag] For comments to Takeuchi and Chat\'e\rf{TaCh11}
\emph{Can the inertial manifold be captured by unstable periodic
orbits?}, see [2011-07-21] entry, \refsect{sec:TaCh11}.

\item[2011-07-25 Kazz]
I am also a bit confused by not ordered time lines in the blog...

\item[2011-07-25 Predrag 2 Kazz]
Glad to oblige, but would be better if a question is answered pages
later, just because answer was typed at a later date? We who use subversion
always check the new edits using subversion diff, so we catch
edits anyplace in the blog...

\item[2011-10-06 Predrag] Ooops - forgot Henon.tex at home, this will
have to do until tonight:

\item[2011-10-06 Hugues] 
It seems pretty clear to me that there exists (infinitely) many "almost non-hyperbolic" \po s and all the more so than they have longer periods (remember the plots "minimum angle vs period" of Kobayashi and Saiki). But, now, how to find them? Would starting points given by points at which the CLVs of the chaotic trajectory are almost tangent be useful?

\item[2011-10-06 Predrag]
Yes, we used to call forward images of the primary H\'enon
tangencies `turnbacks' and such. My theory of 
\HREF{http://www.cns.gatech.edu/~predrag/papers/preprints.html\#GeomChaos}{pruning fronts}
says that you only have to identify non-hyperbolicities on the pruning front, the rest are just forward/backward images of it, and I like to do it best by sequences of periodic orbits.
Grassberger likes to do it it by stable/unstable manifolds, but I hope that by now you
agree with me that the cycles are the way to go. The brilliant thing about the pruning front
is that you search for non-hyperbolicities systematically,
on a (fractal) line, rather than in the whole plane. This way you systematically obtain
the grammar of admissible itineraries for the H\'enon-type maps. There are infinitely
many nearly non-hyperbolic cycles, but in practice at most one pair for each time step
in the period. The moment you find \emph{one} cycle that is stable,
you have proven that the H\'enon attractor is not a strange attractor, but rather a
strange repeller: there is roughly 50-50 chance that it is strange / not strange for
\emph{the} H\'enon parameter values.
The moment
Grassberger heard my seminar he was able to compute all cycles up to length 32 or so.
Procaccia misunderstood what it was about and just published it\rf{pre88top}, using
partially my notes and adding 1/2
Procaccian gibberish to it, so that pretty much killed the joy of writing it up
for me, it has never been written up well. An attempt is in
\HREF{http://chaosbook.org/paper.shtml\#smale}{ChaosBook.org} 
chapter {\em Stretch, fold, prune}.

\item[2011-10-06 Hugues]
About the apparent smoothness of the angle between stable and unstable manifolds: can't this information already be seen using (finite-length) approximations of the stable manifold, as in Posch et al, Fig~3.4 of the Blog? It seems to me that one can infer from such data how the angle varies all along the attractor (following a sheet) and in particular one should be able to locate the main (near-)tangencies with good accuracy.

\item[2011-10-06 Predrag]
Bit more subtle than that - turnbacks on long cycles are very sharp and very small,
so things that look smooth are not smooth; they are dense, and every place on the
strange attractor. Turnbacks are visible as singularities in
natural measure (that's always a signature of non-hyperbolicity, see Figure~16.6
in \HREF{http://chaosbook.org/paper.shtml\#measure}{ChaosBook.org}
chapter {\em Transporting densities}. The good news is the pruning front theory: you need
to only identify (near-)tangencies on the smooth primary folds, all the sick stuff then comes for free, by iterating the cycles that bracket the tangency.


\end{description}

\renewcommand{\ssp}{a}
