\ifsvnmulti
 \svnkwsave{$RepoFile: lyapunov/dailyBlog.tex $}
 \svnidlong {$HeadURL$}
 {$LastChangedDate$}
 {$LastChangedRevision$} {$LastChangedBy$}
 \svnid{$Id$}
\fi

\chapter{Daily blog}
\label{c-DailyBlog}

\begin{bartlett}{
Je ne veux pas travailler\\
Je ne veux pas dejeuner\\
Je veux seulement oublier\\
Et puis je fume
            }
\bauthor{
\HREF{http://www.youtube.com/watch?v=MBoTRF2aK4s}
{China Forbes - Thomas M. Lauderdale (Pink Martini)}
    }
\end{bartlett}

\renewcommand{\ssp}{x}
\renewcommand{\vel}{\ensuremath{v}}   % state space velocity

\section{Lyapunov vectors \KS}

\begin{description}

\item[2011-02-26 Predrag] Moved Lyapunov stuff to
    \refchap{s:LyapunovVec}.


\item[2011-02-22 Predrag]
So far the gap between us and Kaz way of thinking is huge.
Maybe start EVO group meetings?

\item[2011-02-23 Evangelos]
I would love to be able to use EVO but I've
    failed, both in the office and at home. I will arrange to meet
    physically with Kaz \etal, include you
    through EVO?

\item[2011-02-23 Predrag]
There is also a new multi-video service from Skype (paid service), not
sure whether it is comparable to EVO (EVO is real good for projection of
your presentation on other people's seminar screen).

Anyway, go talk to him - they'll have to learn quite a bit before the
collaboration would lead to something, but it would be great if we can
get together - potentially a serious step forward in (confined)
turbulence

\item[2011-03-02 Kaz] Let's start the discussion
 Monday 2011-03-07 at 13:30 (07:30 Atlanta time) at the Institut des Syst\`emes Complexes, 57-59 rue Lhomond, Paris.

\item[2011-03-02 Predrag] Kaz, please download
\HREF{http://evo.caltech.edu/evoGate/}{EVO.caltech.edu}, see whether you can
figure out how to use it - some notes are
\HREF{http://www.cns.gatech.edu/colloquia-seminars/AudioVisual.html}
{here}. Designed for scientists, it's better than skype if you want to
share presentations.

\item[2011-03-02 Evangelos] First meeting's wisdom:
\begin{itemize}
 \item We all agree that it's best to work on antisymmetric subspace KS.
 \item Evangelos will do the shooting for periodic orbits 
	(we need to decide on system size). He hopes he will find the time \ldots  
 \item Kaz uses an implicit second order Adams(-Something) method, with
	pseudospectral space discretization. He uses one of the standard
	prescriptions for anti-aliasing, in order to keep the large wavenumbers
	and corresponding isolated Lyapunov vectors uncontaminated. Uses numerical 
	recipes for FFT. (ES suggests switching to FFTW as numerical recipes FFT is
	known to be unreliable in double precision.) 
 \item Evangelos uses explicit fourth order Runge-Kutta with exponential
	time differencing (see \refref{ks05com}) and pseudospectral space
	discretization. He uses brute force anti-aliasing (increases resolution
	to the point that only modes already at round-off are affected by
	aliasing). He uses FFTW. 
 \item We agreed that a good strategy would be that Kaz implements a simple
	Newton routine, so that he can refine the orbits I give him. I've 
	pointed him to Chaosbook chapters, my own first year report on Newton's
	method for periodic orbits and my thesis.
 \item I've also agreed to give Kaz the backbone of my integrator routines
	(and also pointed him to \refref{ks05com}). Kaz is suspicious towards
	Trefethen's algorithm, he thinks that the integrating factor trick might
	lead to under-resolved large wavenumbers and problematic isolated 
	Lyapunov vectors (and that the implicit method performs better in this
	respect). I can't argue in favor of one or the other integrator.
 \item I am also thinking to try to implement anti-aliasing and see
	if agreement improves.
\end{itemize}
 


\end{description}

\renewcommand{\ssp}{a}
