\ifsvnmulti
 \svnkwsave{$RepoFile: lyapunov/dailyBlog.tex $}
 \svnidlong {$HeadURL$}
 {$LastChangedDate$}
 {$LastChangedRevision$} {$LastChangedBy$}
 \svnid{$Id$}
\fi

\chapter{Daily blog}
\label{c-DailyBlog}

\begin{bartlett}{
Je ne veux pas travailler\\
Je ne veux pas dejeuner\\
Je veux seulement oublier\\
Et puis je fume
            }
\bauthor{
\HREF{http://www.youtube.com/watch?v=MBoTRF2aK4s}
{China Forbes - Thomas M. Lauderdale (Pink Martini)}
    }
\end{bartlett}

\renewcommand{\ssp}{x}
\renewcommand{\vel}{\ensuremath{v}}   % state space velocity

% \section{Lyapunov vectors \KS}


This section of the blog is a general discussion.
\refChap{sect:LyapKS} deals specifically with the
\KS\ calculations.



\begin{description}

\item[2011-02-26 Predrag] Moved Lyapunov stuff to
    \refchap{s:LyapunovVec}.

\item[2011-10-05 Predrag] Moved H\'enon stuff to
    \refchap{c-Henon}.

\item[2011-02-22 Predrag]
So far the gap between us and Kazzanism way of thinking is as huge
as the distance from her to Kazzahstan.
Maybe start EVO group meetings?

\item[2011-02-23 Evangelos]
I would love to be able to use EVO but I've
    failed, both in the office and at home. I will arrange to meet
    physically with Kazz \etal, include you
    through EVO?

\item[2011-02-23 Predrag]
There is also a new multi-video service from Skype (paid service), not
sure whether it is comparable to EVO (EVO is real good for projection of
your presentation on other people's seminar screen).

Anyway, go talk to him - they'll have to learn quite a bit before the
collaboration would lead to something, but it would be great if we can
get together - potentially a serious step forward in (confined)
turbulence

\item[2011-03-02 Kazz] Let's start the discussion
 Monday 2011-03-07 at 13:30 (07:30 Atlanta time) at the Institut des
 Syst\`emes Complexes, 57-59 rue Lhomond, Paris.

\item[2011-03-02 Predrag] Kazz, please download
\HREF{http://evo.caltech.edu/evoGate/}{EVO.caltech.edu}, see whether you can
figure out how to use it - some notes are
\HREF{http://www.cns.gatech.edu/colloquia-seminars/AudioVisual.html}
{here}. Designed for scientists, it's better than skype if you want to
share presentations.

\item[2011-06-28 Evangelos]
There are many things I do not understand from my discussion with Kazz,
so I will wait for his notes with more details.

%\item[2011-06-28 Predrag] If Kazz is really going to write a paper, it
%would be very useful if he could learn subversion... Any chance you can
%check svn is installed on his laptop, then walk him through svn checkout
%(your GT account). If it works, I'll get his account processed by KGB,
%but it is really not worth my time unless I know he can use it...

\item[2011-07-16 Predrag]
{\em Stabilization of long-period periodic orbits
    using time-delayed feedback control} by Claire
    Postlethwaite\rf{Postle09}
is perhaps of interest. She says: ``
The Pyragas method of feedback control has attracted much interest as a
method of stabilizing unstable periodic orbits in a number of situations.
We show that a time-delayed feedback control similar to the Pyragas
method can be used to stabilize periodic orbits with arbitrarily large
period, specifically those resulting from a resonant bifurcation of a
heteroclinic cycle. Our analysis reduces the infinite-dimensional
delay-equation governing the system with feedback to a three-dimensional
map, by making certain assumptions about the form of the solutions. The
stability of a fixed point in this map corresponds to the stability of
the periodic orbit in the flow, and can be computed analytically. We
compare the analytic results to a numerical example and find very good
agreement.
''

\item[2011-08-09 Predrag] For comments to Takeuchi and Chat\'e\rf{TaCh11}
\emph{Can the inertial manifold be captured by unstable periodic
orbits?}, see [2011-07-21] entry, \refsect{sec:TaCh11}.

\item[2011-07-25 Kazz]
I am also a bit confused by not ordered time lines in the blog...

\item[2011-07-25 Predrag 2 Kazz]
Glad to oblige, but would be better if a question is answered pages
later, just because answer was typed at a later date? We who use subversion
always check the new edits using subversion diff, so we catch
edits anyplace in the blog...

\item[2011-11-05 Predrag] Gosh, I've never noticed this one:
Takeuchi, Yang, Ginelli, Radons and Chat\'{e}\rf{TaGiCh11}.
Going for 5! different permutations.

\item[2011-12-06 PC] Two more:
\emph{Geometry of inertial manifolds probed via a Lyapunov projection method}
by Hongliu Yang and Guenter Radons, \arXiv{1112.1249}:
``
A method for determining the dimension and state space geometry of
inertial manifolds of dissipative extended dynamical systems is
presented. It works by projecting vector differences between reference
states and recurrent states onto local linear subspaces spanned by the
Lyapunov vectors. A sharp characteristic transition of the projection
error occurs as soon as the number of basis vectors is increased beyond
the inertial manifold dimension. Since the method can be applied using
standard orthogonal Lyapunov vectors, it provides a simple way to
determine also experimentally inertial manifolds and their geometric
characteristics.
''

I tried out the word ``inertial manifold'' on a few mathematicians, and
they get very worked up - it is not defined, it has been shown that it
cannot be defined in 3 dimensions (million dollar question) \etc. Perhaps
safer to say ``global attractor'' or ``{\nws}''. Though neither is what
we have in mind (it has to be a connected, transitive attractor).

\emph{Dimensional collapse and fractal attractors of a system with fluctuating
 delay times}
by Jian Wang, G\"unter Radons and Hongliu Yang,
\arXiv{1112.1269}:
``
 A frequently encountered situation in the study of delay systems is that the
length of the delay time changes with time, which is of relevance in many
fields such as optics, mechanical machining, biology or physiology. A
characteristic feature of such systems is that the dimension of the system
dynamics collapses due to the fluctuations of delay times. In consequence, the
support of the long-trajectory attractors of this kind of systems is found
being fractal in contrast to the fuzzy attractors in most random systems.
''

\item[2011-12-01 PC to Jim Yorke] I agree with you - why attach any names
to equations? I like the thing referred to as ``Kaplan-Yorke dimension''.
Do you have a more descriptive name for it?
Cheers \& red socks forever

\item[2011-12-01 Jim Yorke to PC] The Kaplan Yorke Dimension is a formula
involving Lyapunov exponents. So we called the KY Dimension. No -- just
joking. We called it the Lyapunov Dimension. Logical?

\item[2011-12-01 PC to Jim Yorke] \emph{Who is the 2000th person who
invoked Lyapunov's name in vain?} Poor Lyapunov. He is a surname, not
concept, but his name gets attached to zillion things he had nothing to
do with.

There is something called ``Lyapunov equation'' which actually is in his
thesis, at least for the strictly contracting fixed point case.

Then there things like ``Lyapunov Covariant Vectors'' that he has nothing
to do with in any imaginable way which yield a ``physical dimension'' of
attractors of PDEs such as Kuramoto-Sivashinsky and Landau-Ginzburg. That
DOES NOT count Lyapunov exponents, it counts non-hyperbolically connected
``Covariant Vectors'', and seems to give a number roughly twice the KY
dimension.

So I'll stick to calling the KY dimension  ``Kaplan Yorke dimension'', and
leave poor Belorussian aristocrat out of this.


\item[2012-02-09 Predrag]
{\em Localization Properties of Covariant {Lyapunov} Vectors},
by Gary P. Morriss,
\arXiv{1202.1571}. He says
``
    The Lyapunov exponent spectrum and covariant Lyapunov vectors are studied for
a quasi-one-dimensional system of hard disks as a function of density and
system size. We characterize the system using the angle distributions between
covariant vectors and the localization properties of both Gram-Schmidt and
covariant vectors. At low density there is a {\it kinetic regime} that has
simple scaling properties for the Lyapunov exponents and the average
localization for part of the spectrum. This regime shows strong localization in
a proportion of the first Gram-Schmidt and covariant vectors and this can be
understood as highly localized configurations dominating the vector. The
distribution of angles between neighbouring covariant vectors has
characteristic shapes depending upon the difference in vector number, which
vary over the continuous region of the spectrum. At dense gas or liquid like
densities the behaviour of the covariant vectors are quite different. The
possibility of tangencies between different components of the unstable manifold
and between the stable and unstable manifolds is explored but it appears that
exact tangencies do not occur for a generic chaotic trajectory.
''

\item[2012-02-12 Predrag]
{\em Covariant hydrodynamic Lyapunov modes and strong stochasticity threshold
 in Hamiltonian lattices}
by M. Romero-Bastida, Diego Paz\'o, Juan M. L\'opez, \arXiv{1202.3476}. They scrutinize
``
the reliability of covariant and Gram-Schmidt Lyapunov vectors
for capturing hydrodynamic Lyapunov modes (HLMs) in one-dimensional Hamiltonian
lattices and show that, in contrast with previous claims, HLMs do exist for any
energy density, so that strong chaos is not essential for the appearance of
genuine (covariant) HLMs. In contrast, Gram-Schmidt Lyapunov vectors lead to
misleading results concerning the existence of HLMs in the case of weak chaos.
''

\item[2012-02-12 Predrag]
{\em Forward and adjoint sensitivity computation of chaotic dynamical
systems} by Wang\rf{Wang12} is perhaps of interest (uses Lyapunov vectors
on Lorenz attractor).

\item[2008-02-28 Predrag] \refRef{ginelli-2007-99} might offer an
intelligent way to evolve a `covariant' (?), co-moving, non-orthogonal
{\jacobianM} eigenvectors frame. See \refexer{exer:stabComoving} for
further references.

\item[2008-02-28 Predrag]
Franzosi, Poggi and Cerruti-Sola\rf{franzosi-2004},
{\em Lyapunov exponents from unstable periodic orbits
  in the {FPU}-beta model}
looks impressive -
    with analytic expressions for families of periodic orbits. They say:
    ``
Using the formulation of Newtonian dynamics in terms of Riemannian
differential geometry, we obtain analytic values of the largest
Lyapunov exponent for the Fermi-Pasta-Ulam-beta model (FPU-beta) by
computing the time averages of the metric tensor curvature and of its
fluctuations along analytically known unstable periodic orbits.
    ''



\end{description}

\renewcommand{\ssp}{a}
