\ifsvnmulti
 \svnkwsave{$RepoFile: lyapunov/students.tex $}
 \svnidlong {$HeadURL$}
 {$LastChangedDate$}
 {$LastChangedRevision$} {$LastChangedBy$}
 \svnid{$Id$}
\fi

\chapter{Log of student's projects}
\label{c-students}

\section{Kamal Sharma's project}
\label{sect:introKamal}

The goal of the summer project is that Kamal adopts the Ginelli \etal\rf{GiChLiPo12} method for computation of covariant vectors to computation of \po\ Floquet vectors,  and implements it for the \po s and \rpo s in our \KS\ database.


\begin{description}

\item[2013-06-12 Predrag] Our first goal is to learn
how to compute \emph{all} eigenvectors (CLVs) and Floquet multipliers
(in literature mistakenly often called `Lyapunov exponents') of
\JacobianMs\ for \po s computed in \refref{SCD07}. 
Please study this reference in depth. For \KS\
calculations of \refchap{sect:LyapKS}, system size $L=22$, this is 62
eigenvectors. We have at least 40,000 \rpo s.

You will first check whether you agree with Kazumasa's results for his favorite \po\ \PO{10.3} with $\period{p} = 10.253$, then the new work starts. \PO{10.3} is the prime orbit period of a pre-periodic orbit, the full period is twice that, see Fig. 6.1\,(i) \emph{Selected relative periodic and pre-periodic orbits} in \refref{SCD07}. 

Save your code in a
\texttt{siminos/sharma/} subfolder. You want to confirm the result in \refref{SCD07},
see
\HREF{http://www.cns.gatech.edu/~predrag/papers/SCD07.pdf}{sect. 4} in that paper:
    \PC{the source file is in \texttt{siminos/rpo\_ks/}}

\begin{quote}
``[...] Indeed, numerically the covariant Lyapunov
vectors\rf{ginelli-2007-99} of the $L=22$ chaotic attractor separate
into 8 ``physical'' vectors with small Lyapunov exponents
$(\Lyap_j) = (0.048,$ 0, 0, $-0.003$, $-0.189$, $-0.256$,
$-0.290$, $-0.310$),
and the remaining 54 ``hyperbolically isolated'' vectors with rapidly
decreasing exponents
$(\Lyap_j) = (-1.963$,   $-1.967$,   $-5.605$,   $-5.605$,  $-11.923$,  $-11.923$,
 $\cdots) \approx -(j/\tildeL)^4$,
in full agreement with the Yang \etal\rf{YaTaGiChRa08} investigations
of KS for large systems sizes.
 [...]''
\end{quote}

\item[2013-06-12 Predrag] 
This section is an evolving draft of Kamal Sharma's project report;
daily progress is reported in \refsect{sect:dailyBlKamal}.


Please {\color{red} write down here}, in your project's
log, the equations you are integrating (you can clip and paste
equations from elsewhere in this repository to save time). 
This will be a continuously evolving draft of the project report.
When you
write up the project report it will all be needed, so you might just
as well do it now.

If we know where are you in the project, we can help you.
Write up full fledged
derivations and explanations of what you are learning - clear enough
that some of that text can be later used in your project report, or a
publication, if we get that far. Clear exposition is 1/3 of research.



My vision of the project is sketched in \refchap{c-draft}. Studying \refchap{s:LyapunovVec} is also a good idea.
\refChap{sect:LyapKS} deals specifically with the \KS\ calculations.
Search for\\
``2012-02-06 Evangelos Talked with Hugues at MPIPKS, created a to-do list:''
\\
to see Evangelos and Hugues work plan.

\item[2013-06-12 Predrag] 
Setting up Kamal for the svn (subversion) repository access 

\texttt{svn://zero.physics.gatech.edu/siminos}

you will
need this userID, password, so save it in your secret stash of magic
words:
\\
user: sharma  password: Lyapunov

to access repositories read about subversion/svn and VPN access
on
\HREF{http://cns.gatech.edu/CNS-only} {cns.gatech.edu/CNS-only} - to access any
of the internal CNS pages you will need

cnsuser           cnsweb

\HREF{http://ChaosBook.org/library}{Click here} for internal reprints and
books collection. You will need

student           Lautrup

I do not have the patience to enter every paper in the listing on this
internal home page, so many papers are there but not listed. Their PDF file
name is the same as their bibTeX name for the publication. For example, a
book by Wassim M. Haddad\rf{HaCh08} has BibTeX name
\texttt{HaCh08}, so I put a copy of their book into the
ChaosBook.org/library as

\HREF{http://ChaosBook.org/library/HaCh08.pdf}
{ChaosBook.org/library/HaCh08.pdf}.

If you download some publications that would be nice to save for other
people, let me know, and I'll show you how to do it.

To compile the blog, 
\begin{verbatim}
 cd siminos/lyapunov/
 pdflatex blog; bibtex blog; pdflatex blog
\end{verbatim}

To help a bit with starting writing, I will
edit your text (you can see edits by using diff in Tortoise).

Here is how you can
\Kamaledit{highlight your edits}, and here is how you can add a dated
footnote\Kamal{2013-03-03}{This is a footnote. Always enter the date, so
we know how old it is.}.

In the first draft, do not worry about
English, just focus on the formulas. Write
down what a starting graduate student would need to know to be able
to set up the calculation.

If you want source files for ChaosBook.org to clip and past from, you can
\begin{verbatim}
  svn checkout svn:\\zero.physics.gatech.edu\dasbuch
\end{verbatim}

Always trim the figures, avoid lots of white
space around the edges.

If you look into directory
siminos/chaos/ and pdflatex blog.tex, you can see an example of what
a student work log looks like (after a semester of so of work).
For examples of
what final reports look like, see \wwwcb{/projects}.


Come to me or Skype me if there is something you
cannot figure out.

\end{description}


\subsection{Covariant vectors}

\begin{description}

\item[2013-06-12 Predrag]
I rewrote ChaosBook.org chapters
\HREF{http://chaosbook.org/paper.shtml\#stability} {``Linear stability''},
\HREF{http://chaosbook.org/paper.shtml\#invariants} {``Cycle
stability''}, and 
\HREF{http://chaosbook.org/paper.shtml\#Lyapunov} {``Lyapunov
exponents''}, to help graduate students. Any suggestion for improving 
these chapters is greatly appreciated.

                              \inCB
{Finite-time} {Lyapunov} or \emph{characteristic} exponents and
associated \emph{principal axes} in the theory of dynamical systems
are discussed in chapter
\HREF{http://chaosbook.org/paper.shtml\#Lyapunov} {``Lyapunov
exponents''}. Oseledec \emph{Lyapunov exponents}\rf{lyaos} are the
$\zeit\to\infty$ limit of these. \emph{Floquet multipliers} and
\emph{eigen-vectors} are property of finite-time, compact invariant
solutions, such as \po s and \rpo s; they are explained in chapter
\HREF{http://chaosbook.org/paper.shtml\#invariants} {``Cycle
stability''}. \emph{Stability exponents}\rf{GoSuOr87} are the
corresponding long-time limit, and --in one of those frustrating
historical accidents-- the corresponding
\emph{stability} eigenvectors are misnamed in recent papers
\emph{Lyapunov vectors}, even though \emph{they are not} the
eigenvectors that correspond to the Lyapunov exponents. Sorry, not my
fault. The prefix `covariant' is meant to distinguish the two kinds of
eigenvectors. That's just confusing, for no good reason - Lyapunov
has nothing to do with linear stability described by the \jacobianM\
$\jMps$, as far as I know his paper\rf{Lyap1892}
is about $(\transp{\jMps}\!\jMps)$ and the associated principal axes.

Oseledec proofs are important in mathematics, but not
sensible for computational work in dynamical systems. For me the
Goldhirsch, Sulem and Orszag\rf{GoSuOr87} exposition is the clearest
(read it \HREF{http://chaosbook.org/library/GoSuOr87.pdf}{here}).
They correctly distinguish \emph{Lyapunov} eigenvalues and eigenvectors
from the \emph{stability} eigenvalues and eigenvectors.

At that time Goldhirsch \etal\ had no proof that the long-time Lyapunov
exponents converge to the stability exponents (Kazumasa, do you have
some more recent paper that you prefer to Goldhirsch \etal?).

Trevisan and Pancotti\rf{TrePan98} \emph{Periodic orbits, Lyapunov
vectors, and singular vectors in the Lorenz system} (see above, read
it \HREF{http://chaosbook.org/library/TrePan98.pdf}{here}))
apparently need to be cited for the observation that covariant
vectors reduce to Floquet eigenvectors in the particular case of a
{\po}. Seems so obvious it is in ChaosBook without attribution, and
Ruelle and Eckmann\rf{eckerg} surely say the same (though I have not
checked). Most importantly for our project, they say:

``The leading Lyapunov vectors, as defined here, as well as the
asymptotic final singular vectors, are tangent to the attractor,
while the leading initial singular vectors, in general, point away
from it. Perturbations that are on the attractor can be found in the
subspace of the leading Lyapunov vectors.''

They have very nice pictures for Lorenz unstable orbits illustrating that.

\item[2013-06-12 Predrag]
Graduate students are brave people who immediately jump into the deep
end of the pool, without any testing. Here are two 2\dmn\ maps, the
Lozi map (for $a=1.85, b=0.3$)
% \PC{need figure of Lozi strange attractor}
\bea
   x_{n+1} &=& 1-a |x_{n}|+ b y_n  \continue
   y_{n+1} &=& x_{n}
\,,
\label{e_lozi_def}
\eea
and the H\'enon map (for $a=1.4, b=0.3$; described at length in
\refchap{c-Henon})
\bea
    x_{n+1}&=&1-ax^2_n+b y_n
        \continue
    y_{n+1}&=& x_n
\,,
\label{eq2.1a}
\eea
for which fixed points are available analytically\rf{DasBuch} (for
the Lozi map all periodic points are available analytically). So see
what Floquet exponents 
your program returns for \po s of these humble maps, and then for
ergodic trajectories, before running the programs on ergodic
trajectories in zillion dimensions.

If your programs only apply to continuous time flows (ODEs rather
than maps), ChaosBook\rf{DasBuch} has \eqva\ and \po s for 2\dmn\
Lorenz and R\"ossler flows, and Evangelos has lots of \rpo s for the
\cLe; make sure your programs work on small systems first.

\end{description}


\subsection{\KSe}

\PC{2013-03-19 Predrag copied this from \refref{SCD07}, to set up
the conventions for the \KSe\ calculations.}
%
In the formulation
adopted here, the time evolution of the `flame front velocity'
$u=u(x,t)$ on a periodic domain $u(x,t) = u(x+L,t)$ is given by
\beq
  u_t = F(u) = -{\textstyle\frac{1}{2}}(u^2)_x-u_{xx}-u_{xxxx}
    \,,\qquad   x \in [-L/2,L/2]
    \,.
\ee{ks}
Here $t \geq 0$ is the time, and $x$ is the spatial coordinate.
The subscripts $x$ and $t$ denote partial derivatives with respect to
$x$ and $t$. In what follows
we shall state results of all calculations either in units of the
`dimensionless system size' $\tildeL$, or the system size $L = 2 \pi
\tildeL$. All numerical results presented in this report
are for $\tildeL=22/2\pi \simeq 3.5014$.
Spatial periodicity $u(x,t)=u(x+L,t)$
makes it convenient to work in the Fourier space,
\beq
  u(x,t)=\sum_{k=-\infty}^{+\infty} a_k (t) e^{ i k x /\tildeL }
\,,
\ee{eq:ksexp}
with the $1$-dimensional PDE \refeq{ks}
replaced by an infinite set of
ODEs for the complex Fourier coefficients $a_k(t)$:
\beq
\dot{a}_k= \pVeloc_k(a)
     = ( q_k^2 - q_k^4 )\, a_k
    - i \frac{q_k}{2} \sum_{m=-\infty}^{+\infty} a_m a_{k-m}
\,,
\ee{expan}
where $q_k = k/\tildeL$.
Since $u(x,t)$ is real, $a_k=a_{-k}^\ast$, and we can replace the
sum by a $k > 0$ sum.


\begin{description}

\item[2013-06-12 Predrag to Kamal] 
Read this blog, study the relevant \KS\
papers\rf{SCD07,Christiansen97}

\end{description}


\section{Daily log}
\label{sect:dailyBlKamal}

This section is the daily (or at least, bi-weekly)  log of the Kamal's
work.

\begin{description}

\item[2013-06-11 Predrag] Kamal Sharma  <ksharma41@gatech.edu>, 
1. year Georgia Tech physics graduate student, joined the
collaboration as a summer research project.
Kamal will enter notes on his day-to-day progress into this log.

\item[2013-06-12 Kazumasa  to Kamal]
Great! Please read (apart from ChaosBook.org)
%Francesco 
Ginelli \etal\rf{ginelli-2007-99}
\HREF{http://prl.aps.org/abstract/PRL/v99/i13/e130601}{PRL}, and the
\HREF{http://arxiv.org/abs/1212.3961}{long follow-up paper}\rf{GiChLiPo12},
(much more detailed than the PRL).
Otherwise you will not understand anything on the codes (and the project).

My C++  code and README is in siminos/kazz/code/.
You compile and run it as C++.
Note that, though I extracted a part of my long code relevant to our project,
 it contains many functions which you probably will not need.

Hugues and I 
strongly recommend that you write your own code from zero, without
reference to mine. This is the best way to learn what's going on in
the code. Don't try to write a code with full functionality from the
beginning, but start with a minimal code, which only computes, say,
Lyapunov exponents.
Confirming that this works as you expect, you can add a code to compute
the covariant vectors from a simple forward-backward process (and
check if the exponents values computed from the covariant vectors
agree with those from the Gram-Schmidt method). For practical use,
you have to implement the "block-by-block" computation of the
vectors, to overcome memory issues (see discussions in Sec. 4.2 in 
Ginelli \etal\rf{GiChLiPo12}, \arXiv{1212.3961}). This is
the first step you should reach in this project.

For numerical integration of the KS equation, I used the
operator-splitting algorithm (Adams-Moulton method + Heun's method),
typically with time step 0.005.
For more detail, read Sect.~II~A in Takeuchi \etal\rf{TaGiCh11}.
Stiffness matters, implicit methods are common ways to overcome this problem,
 and that's why I used the Adams-Moulton method.
To further improve, it's better to split the linear and non-linear terms
 by the operator splitting method
 and use the implicit method only to the linear terms where the stiffness is.
I'm not saying that my algorithm is the best way to simulate the KS equation,
 but at least it's suited to practical use.

\item[2013-06-12 Predrag  to Kamal]
A discussion of stiffness in integrating PDEs can be found
\HREF{http://www.pvv.ntnu.no/~berland/talks/berland05expintro.pdf}
{here}. \HREF{http://www.math.ntnu.no/num/expint/} {Here} is Matlab
package for testing 47 (!) schemes.

I agree with Kazumasa: you want to write your own code,
but not necessarily reinvent the wheel. There is KS code various
places in this svn repository, for example in
\texttt{siminos/matlab/}, that you might want to compare performance
of your code with. Evangelos can help you with that.


Study the methods described in
\HREF{http://www.cns.gatech.edu/~predrag/papers/SCD07.pdf}
{Appendix A} of \refref{SCD07}.
The authors have lots of experience and they went to the
    trouble of explaining the best methods they know, so why not read
    them and implement them?

The plot of what you get for $L=22$
Lyapunov spectrum should confirm the published results (see
{\bf [2009-09-13 Ruslan]} on \refpage{sect:LyapKS},
\reffig{fig:lyapSpecCLG}, \reffig{fig:lyapSpec1},
\reffig{fig:lyapSpec}, and \reftab{tab:ks22shad}) and the
figure\PC{which one? Always state the number} in Kazumasa \etal\
paper\rf{TaGiCh11} (read it
\HREF{http://chaosbook.org/library/KoSa11.pdf} {here}).

\item[2013-06-12 Ruslan]
You can use the method developed by Cox and
Matthews\rf{cox02jcomp} and improved by Kassam and
Trefethen\rf{ks05com}, where the linear part is integrated exactly.
My Matlab code {\tt ksfmetd2.m} is based on this method and it is
stable and reasonably accurate with time step as large as 0.25.  It
also solves variational problem, so can be used to calculate Lyapunov
exponents.  See {\tt ksfmlyap.m} in {\tt siminos/matlab/ruslan/},
read {\tt 00ReadMe.txt}.  If you want to compute Lyapunov exponents
using the standard GS procedure, the relevant code is in {\tt
ksdupo.m} in the same folder (look for the cell "\%\% Compute
Lyapunov exponents of KS (using ksfmlyap and GS)").

\item[2013-06-12 Predrag]
Ruslan uses the same \KSe\ convention as Kassam and
Trefethen\refrefs{ks05com}. I have saved the Trefethen Matlab code
\HREF{http://chaosbook.org/extras/Trefethen/kursiv.m} {here}. Perhaps
you want to c++ it, see how it runs for you, or run it in Matlab and
compare with your code. There might be something else useful on
\HREF{http://chaosbook.org/extras/}{ChaosBook.org/extras} homepage,
for example the simulations by the spring 2007 GaTech chaos class.
You can search the blog for 'Trefethen' for other discussions
(Kazumasa has reservations about the Trefethen\rf{ks05com} algorithm,
but Ruslan is OK with it).
Siminos has other codes, if needed, on a different repository.


\item[2013-01-21 Evangelos] The \KS\ data you need is in \\
\texttt{siminos/matlab/ruslan/kse22orbits.mat},
\\
in a structure called eq.
Eigenvalues are in the field eq.eig and right eigenvectors are in the field eq.evec.
[e.g. eq(k).evec(:,1) is the eigenvector which corresponds to the first eigenvalue eq(k).eig(1)
of the k'th equilibrium]. However, I have not used the data for a long time so, it would
be better if Ruslan verifies how the Fourier modes are stored (I think that the numbers
in a column vector correspond to real and imaginary part of the Fourier modes
\[
 (a_1,\, b_1,\, a_2,\, b_2,\, \ldots a_N,\, b_N)
\]
and thus here there are $31+1$ complex Fourier modes (the zero'th mode is not included)).

In order to get the left eigenvectors you will need to compute the
stability matrix.

If you run into problems or have questions please email me so that we can arrange to talk through Skype.


\item[2013-06-12 Ruslan] To compute stability matrix in Matlab, use 
\\
\texttt{siminos/matlab/ruslan/ksfm.m}:
\\ {\tt [f, df] = ksfm(0,eq(1).a,22.0)}, where {\tt df} will be the stability matrix of $EQ_1$.

The idea of the method is described in \refref{SCD07} Appendix A and
B.  The Jacobian matrix is calculated from (B.1) which uses solutions
of (A.4) and (A.6).  Matlab code {\tt ksfmetd2.m} solves (A.4) and
(A.6) simultaneously.  The Jacobian matrix is output in {\tt da}.
You don't need anything else.  

\item[2013-06-1? Kamal] Here is my first blog post:

[.....]


\end{description}

\renewcommand{\ssp}{a}
