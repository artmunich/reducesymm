\ifsvnmulti
 \svnkwsave{$RepoFile: lyapunov/students.tex $}
 \svnidlong {$HeadURL$}
 {$LastChangedDate$}
 {$LastChangedRevision$} {$LastChangedBy$}
 \svnid{$Id$}
\fi

\chapter{Students log}
\label{c-students}

\section{Qi Ge's project}
\label{sect:introQG}
This section is an evolving draft of the introduction to Qi Ge's project.

Predrag's vision of what the project is sketched in \refchap{c-draft}. Studying
\refchap{s:LyapunovVec} is also a good idea.
Search for\\
``2012-02-06 Evangelos Talked with Hugues at MPIPKS, created a to-do list:''
\\
to see Evangelos and Hugues concrete work plan.

We hope that Qi Ge masters computation of covariant Lyapunov vectors for
the \po s and \rpo s in our \KS\ database.

\refChap{sect:LyapKS} deals specifically with the \KS\ calculations.

\section{Daily log}

This section is the daily (or at least, bi-weekly)  log of the students'
work.

\refChap{sect:LyapKS} deals specifically with the \KS\ calculations.


\begin{description}

\item[2013-01-11 Predrag] Qi Ge  <intertwistlet@gmail.com> joined the
collaboration as a research project student (GaTech credit 3 h/week).
He will enter notes on his day-to-day progress into this log.

\item[2013-01-12 Kazumasa]
Great! Please recommend Ge Qi to read, apart from books and reviews on
chaos and orbit expansion (Chaosbook and Eckmann-Ruelle review?),
Francesco et al.'s
\HREF{http://prl.aps.org/abstract/PRL/v99/i13/e130601}{PRL}\rf{ginelli-2007-99}, and their
\HREF{http://arxiv.org/abs/1212.3961}{long follow-up paper}\rf{GiChLiPo12}.

Otherwise he wouldn't understand anything on the codes (and the project).

Hugues will stay in Japan for 2 weeks in late February. I hope that you,
the student, and Evangelos will also have time to join the discussions
during this period.

\item[2013-01-12 Qi Ge]
Great, this will be a good start point.

\item[2013-01-23 Predrag] Please ask Jeffrey M. Heninger
<jeffrey.heninger@gatech.edu> to help you get started with this
subversion repository (come to me or skype me if there is something you
cannot do):

\texttt{svn://zero.physics.gatech.edu/siminos}

user: geqi  password: Lyapunov

\item[2013-02-14 Qi Ge 2 Kazumasa]
Can you send me the code about computing the CVLs of the KS system?
So what algorithms you use in the time step simulation, this will
help me understand the code.

\item[2013-02-14 Predrag to Qi Ge] If you look into directory
siminos/chaos/ and pdflatex blog.tex, you can see an example of what
a student work log looks like (after a semester of so of work). Chao
Shi stopped with daily / weekly updates, and I ended up writing all
of Chapter 1 {\em Kuramoto-Sivashinsky equation}, so we stopped the
project.

\item[2013-02-14 Predrag to Qi Ge] Please start writing down the
ideas we talked about today into \refsect{sect:introQG}; this will
be a continuously evolving draft of the introduction to the project
report. I have entered pointers to Evangelos, Hugues and my ideas
about what the project. If we know where are you in the project, we can
help you.

\item[2013-02-14 Kazumasa to Qi Ge]
Please find my code in C++ and README at siminos/kazz/code/.
You can compile and run it as usual C++ program.
Note that, though I extracted a part of my long code relevant to our project,
 it contains many functions which are hardly used in the usual situation.

In any case, we (Hugues and myself) (Predrag: I agree, totally)
strongly recommend that you write your own code from zero, with
reference to mine. This is the best way to learn what's going on in
the code. Don't try to write a code with full functionality from the
beginning, but start with a minimal code, which only computes, say,
Lyapunov exponents.
Confirming this works as you expect, you can add a code to compute
the covariant vectors from a simple forward-backward process (and
check if the exponents values computed from the covariant vectors
agree with those from the Gram-Schmidt method). For practical use,
you have to implement the "block-by-block" computation of the
vectors, to overcome memory issues (see discussions in Sec. 4.2 in F.
Ginelli \textit{et al.}, \arXiv{1212.3961}\rf{GiChLiPo12}). This is
the first step you should reach in this project.

\item[2013-02-14 Kazumasa to Qi Ge]
For numerical integration of the KS equation, I used the
operator-splitting algorithm (Adams-Moulton method + Heun's method),
typically with time step 0.005.
For more detail, look at Sec. II A in K. A. Takeuchi \textit{et al.},
 Phys. Rev. E \textbf{84}, 046214 (2011)\rf{TaGiCh11}.

\item[2013-02-16 Qi Ge to Predrag] So our aim is to calculate the
first seven CLVs of KS system. The solution of the equation we choose
are the hyperbolic periodic orbits.

\item[2013-02-16 Predrag] Our first goal is to learn
how to compute \emph{all} eigenvectors (CLVs) and Floquet multipliers
(mistakenly often called `Lyapunov exponents' in this text) of
\JacobianM s for \po s computed in \refref{SCD07}. For \KS\
calculations of \refchap{sect:LyapKS}, system size $L=22$, this is 62
eigenvectors. We have at least 40,000 \rpo s. You will first check
whether you agree with Kazumasa's results for his favorite orbit
favorite cycle \PO{10.25}, then the new work starts.

You will have to switch from the tweets to writing up full fledged
derivations and explanations of what you are learning - clear enough
that some of that text can be later used in your term report, or a
publication, if we get that far. Clear exposition is 1/3 of research.

\end{description}

\renewcommand{\ssp}{a}
