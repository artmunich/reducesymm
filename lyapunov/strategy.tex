\ifsvnmulti
 \svnkwsave{$RepoFile: lyapunov/strategy.tex $}
 \svnidlong {$HeadURL$}
 {$LastChangedDate$}
 {$LastChangedRevision$} {$LastChangedBy$}
 \svnid{$Id$}
\fi

\chapter{Strategy, to write up}

\section{How to read me}

The text of the paper drafts is discussed in \refsect{sect:DraftBlog}
``Draft blog.''

To read the blog, go first to the latest blog post, end
of \refchap{sect:LyapKS} for \KS\ discussion, and the end
of \refchap{c-DailyBlog} for general remarks.

\refSect{sect:intro} is so far only Predrag's introduction to the Grand
Dream of this project. \refChap{s:LyapunovVec} focuses on `covariant
Lyapunov vectors' in general. \refChap{sect:LyapKS} deals specifically
with the \KS\ calculations.
For specialized topics, consult the contents. Inner products are
discussed in \refappe{def:innerProduct} Poincar\'e sections are discussed
in \refappe{s:PoincSect}, Newton method for flows in
\refappe{s-POs-flows}, and some tentative thoughts on how to think about
co-moving Floquet vector frames are discussed in
\refappe{sect:stabComoving} (stability in a co-moving frame) and
\refappe{sect:transport} (transport of vector fields).

Throughout:  {\footnotesize inCB} on the margin                 \inCB
indicates that the text has been transferred to an
article in siminos/*/,  or to ChaosBook.org
chapters, such as
\HREF{http://ChaosBook.org/chapters/continuous.pdf}
{continuous.pdf}.
 {\footnotesize 2CB} on the margin indicates that the text
still needs to be transferred an article or ChaosBook.org.      \toCB

This \texttt{blog.pdf} file is \emph{hyperlinked}.
There is a bunch of handy links throughout,
now that we went to the trouble of downloading papers and stealing books. Brilliant.
For example, if you click on
this: \arXiv{1103.4536}, you might find a interesting paper to read.
\HREF{http://chaosbook.org/library/KoSa11.pdf}{Clicking here} will
lead you to our internal ChaosBook.org/library:
You'll need to log in as \texttt{student} and then enter \texttt{Lautrup}.
\HREF{http://www.zotero.org/groups/cns}{Zotero.org} is great,
but as only Evangelos and Predrag use it right now,
it is faster to stick stuff into ChaosBook.org library.


\section{Papers to write}

% J Hightower http://www.brainyquote.com/quotes/authors/j/jim_hightower.html
% - former texas politician, author, speaker 1943-
% Jamie Waite (?)
% 'Only dead fish swim with the stream' a quote by Malcolm Muggeridge
% from http://www.kittozutto.com/showcase/go-with-the-flow/
\begin{bartlett}{
Even a dead fish can go with the flow.}
\bauthor{
Jim Hightower, Texas politician}
\end{bartlett}


\HREF{http://www.urbandictionary.com/define.php?term=Go\%20with\%20the\%20flow}
{urbandictionary.com}:

\begin{enumerate}
   \item To not push against prevailing behavior/norms/attitudes,
   occasionally including bowing to peer pressure.

   \item To not attempt to exert a large amount of influence on the
   course of events, whether a specific series of events or events in
   general. A person who does this is often referred to as ``laidback''
   or ``easygoing''.

   \item First known to be used by the Roman Emperor Marcus Arelius in
   his `Meditations.' Marcus wrote a lot about the flow of
   thoughts and happiness and concluded that ``most things flow
   naturally'' and that it was better to ``go with the flow''
   than to try to change society.
 \end{enumerate}

See siminos/blog for our most complete listing of
PACS classification, keywords.

\subsection{\emph{Phys. Rev. Lett.} ``Can the Inertial Manifold...''}
% Be Captured by Unstable Periodic Orbits?

\begin{bartlett}
\bauthor{Bob Dylan} %: ``??''}
To live outside the law   % \\
you got to be honest.
\end{bartlett}
% \PC{track down the Bob Dylan quote}

\begin{description}
\item[2011-07-21 Kazz] wrote the first draft (Predrag's
copy is dowloadable
\HREF{http://www.scribd.com/fullscreen/60910182?access_key=key-2e56zhcvvnqo7amghf6j}
{110721Kazz.pdf} from Scribd.com).

\item[2011-09-24 Hugues]
I would like to read the  Kobayashi and Y. Saiki paper\rf{KoSa11} you
cite in this blog; can you send this to me?

\item[2011-09-24 Predrag]
A copy is ChaosBook.org/library
\HREF{http://chaosbook.org/library/KoSa11.pdf}{click here}.
You'll need to log in as \texttt{student} and then enter \texttt{Lautrup}.

\item[2011-09-24 Hugues]
Kazz and I are now struggling to produce something that really looks like
a paper on the UPO/KS/CLV story. Kazz and Francesco met Radons and Yang
(RY) at DDays Oldenburg, and discovered there that RY had been working on
connecting CLV to phase space, in other words produce evidence that the
``physical'' covariant vectors do indeed constitute a local linear
approximation of the inertial manifold everywhere in state space. They
indeed did that, without using UPOS, just by looking at recurrences of
the trajectories to randomly selected points in state space. They have a
preprint, ready to submit. Kazz told them what we had been doing, leading
to the same conclusion.

We plan to submit 2 papers to be hopefully published ``back to back''.
Kazz and I have 2-3 weeks to complete a preprint so that both papers can
be submitted simultaneously to PRL. Even though Kazz just came back to
Japan and I'm here in Dresden to lead an ``Advanced Study Group'' on
something else, we both have to hurry...

Given the situation, you  and Evangelos  will have to decide
whether you want to appear as co-authors or not. We discussed
here with Evangelos, and had nice projects/ideas... not that we will be
able to advance on these fast enough to incorporate them in the paper,
but...

\item[2011-09-24 Predrag]
I learned a new thing from your work, how to think about the relation of
\po\ Floquet eigenvaectors to the local linear tessellation of the
inertial manifold.
I really, really and truly do not care whether my name is on this PRL,
being tenured and all, as long as the true faith is propagated by our
collective work.

You also have to take in account that
I publish on glacial scales, in PRL only to help students
with getting postdocs. For example, you might not remember, but I hounded
you and Kurchan  at Inst. H. Poincar\'e in September 2008, with our
``{\optPart}'', and I am still writing up
\HREF{http://www.cns.gatech.edu/~predrag/papers/LipCvi07.pdf} {the
paper}\rf{LipCvi07} (and I \emph{mean} writing, I was at it this morning,
when your email arrived, and the associated blog is as of today 153 pages
of calculations, discussion, and literature notes). Our cigars should go
hand in hand with your physical dimensions, and together carpet the
inertial manifold like a bed of roses. But to help Domenico graduate, we
did write a PRL\rf{LipCvi08}.

\textbf{True confessions:} More seriously, I have worked on turbulent
field theories, fully focused, as a cruise missile, since
\HREF{http://www.cns.gatech.edu/~predrag/papers/preprints.html\#Trans2chaos}
{March 1976}, step by step. There is no time to screw around with
`universal' properties of turbulence, a physics tangent of 1960's as
harmful to science as the Theory of Everything and Nothing of the last few
decades. Life is too short for that. Our task is to describe given
turbulent phenomenon as is, not Lorenz it or Kolmogorov it until nothing
left to say about the phenomenon we want to describe.

My personal obsession, right or wrong: There is no way of going from the
equations of motion and boundary conditions to a description of the
turbulent flow other than in terms of invariants of the flow. So one
looooong step was going from extremal solutions of nonlinear field theory
path integrals to the periodic orbit theory. To get people off my back, I
credit Smale\rf{smale}, Ruelle\rf{ruelle} and Gutzwiller\rf{gutbook}, and
for turbulence Hopf\rf{hopf48}, but go read them and try to compute
something about turbulence; we finally did show how to recycle PDEs in
\refref{Christiansen97}, with the first demonstration that one can
compute hundreds of \po s in high-dimensional PDE \statesp s. Does
anybody cite this? No, one cites only papers which came 6 years later,
and find only (relative) \eqva\, such as Eckhardt\rf{FE03} and
Kerswell\rf{WK04}. Do I care? No, not at all - I am so happy to have a
community of colleagues who understand what is the path ahead Sure will
not find them among the super-stringers. And I do care about your
physical dimensions? Yes, very much, I do not see how to turn
Constantin\rf{constantin_integral_1989} into a computation, but your
approach seems very explicit and implementable. So it (along with our
noisy cigars) is key ingredient.


\end{description}


\subsection{\emph{Phys. Rev. E} ``The physical dimension of...''}

\begin{description}

\item[2010-05-25 Vaggelis]
Will we go for an arXiv version?

\item[2010-06-07 Predrag]
I think one should always submit
any article that is worth publishing also to arXiv;
it is open to anyone, rich or poor, and it is
more likely to reach the intended audience than only a publication through
any single journal.

\end{description}

\subsection{Reduced trace formulas?}

\begin{description}
 \item[2010-06-17 Vaggelis]
Since I have all rpo's up to level 7 for CLE I think I should try
to apply ``Continuous symmetry reduced trace formulas'' so that I get an incentive
to understand the paper. After all this group theory, it should be easier now.
 \item[2010-06-18 Predrag]
Would be nice if you did - both to understand the group theory better, and
also because I am not sure I have not missed some important detail about
invariant subspaces when I wrote the paper. Would be great to recycle KS
next, if CLE works.
\end{description}


\subsection{\emph{SIAM J. Appl. Dyn. Syst. ?}}

\begin{description}

\item[2010-06-07 Predrag] Next, the
``Dimensional reduction of Kuramoto-Sivashinsky ...'' paper:
40,000 \rpo s and noplace to go?
Can include movies and more graphics...

\end{description}
