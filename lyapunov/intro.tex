\ifsvnmulti
 \svnkwsave{$RepoFile: lyapunov/intro.tex $}
 \svnidlong {$HeadURL$}
 {$LastChangedDate$}
 {$LastChangedRevision$} {$LastChangedBy$}
 \svnid{$Id$}
\fi

Bits and pieces for a putative article,

``The physical dimension of a \KS\ flow.''

\section{Introduction}
\label{sect:intro}

Inspired by the global
studies of \refrefs{PoGiYaMa06,ginelli-2007-99,YaTaGiChRa08,TaGiCh09} of
`covariant Lyapunov vectors' we use Floquet vectors of unstable \po s to
identify the \emph{local} number of degrees of freedom that capture the
physics of a `turbulent' PDE on a compact spatial domain. That number is
proportional to the system size, for \KS\ flow roughly four times the
number of positive/marginal Floquet (or Lyapunov) exponents, and twice its
Kaplan-Yorke estimate.

The idea is to coarsely cover the \emph{continuous-symmetry reduced}
nonlinear strange attractor with a set of
hyperplanes\rf{SiCvi10,FrCvi11}, as in \reffig{fig:Tesselate}. Any
adjacent pair intersects in a `ridge' hyperplane of one less dimension.
So our task is to, for a given strange attractor, pick a set of \Poincare
section-fixing points, such that each local section is approximately tangent to the
strange attractor. For simplicity here we shall consider only the flows
with a 1 continuous parameter (the time), postponing the more general
case of continuous symmetries to a happier time. An example of such
dynamics is the \KS\ flow restricted to the antisymmetric
subspace\rf{Christiansen97,lanCvit07};
some examples of flows with continuous
symmetries are the \KS\ flow on a periodic domain\rf{SCD07}, the pipe
flow\rf{ACHKW11} and the \pCf\rf{GHCW07}.

We propose to construct a global atlas by deploying a finite
number of linear \Poincare sections in
neighborhoods of the most important equilibria and/or periodic orbits
as local charts.
This is the periodic-orbit implementation of the idea of {\statesp\
tessellation} so dear to professional cyclists, \reffig{fig:Tesselate}.

% In FrCv11.tex replace by Tesselate.png
%%%%%%%%%%%%%%%%%%%%%%%%%%%%%%%%%%%%%%%%%%%%%%%%%%
\SFIG{f_1_08_1}
{}{
Smooth dynamics  (left frame) tesselated by the skeleton of periodic
points, together with their linearized neighborhoods, (right frame).
Indicated are segments of two 1-cycles and a 2-cycle that alternates
between the neighborhoods of the two 1-cycles, shadowing first one of the
two 1-cycles, and then the other.
}{fig:Tesselate} %{Hyp} %{fig6} and {tr:fig6} in ChaosBook
%
%%%%%%%%%%%%%%%%%%%%%%%%%%%%%%%%%%%%%%%%%%%%%%%%%%
%


\section{Charting the \statesp}
	\label{sec:chart}
% extracted from siminos/froehlich/slice/FrCv11.tex   2011-02-20

Work on \KS\ suggests how to proceed: it was shown in
\refrefs{lanCvit07,SCD07} that for turbulent/chaotic systems a set of
\Poincare sections is needed to capture the dynamics. The choice of
sections should reflect the dynamically dominant patterns seen in the
solutions of nonlinear PDEs. We propose to construct a global atlas of
the dimensionally \reducedsp\ $\pSRed$ by deploying linear \Poincare
sections $\PoincS{}^{(j)}$ across neighborhoods of the qualitatively most important
patterns $\slicep{}^{(j)}$.
We shall refer to these states as \emph{\template s}, each
represented in the \statesp\ $\pS$ of the system by
a \emph{\template\ point} $\slicep$. Together with the velocity
field at this point, a template defines a linear \Poincare section,
an affine hyperplane $\sspRed \in \PoincS$,
\beq
    \vel(\slicep) \cdot (\sspRed - \slicep)= 0
\,,
\ee{locTransvVar1}
locally normal to the $\vel(\slicep)$ at the \template\ point $\slicep$.
(For a discussion of inner products, see
\refappe{def:innerProduct}.)
(For a motivation, see
\refappe{s-POs-flows} \emph{Newton method for flows}.)
Each \Poincare section $\PoincS{}^{(j)}$, provides a local chart
at $\slicep{}^{(j)}$ for a
neighborhood of an important, qualitatively distinct class of solutions
(2-rolls states, 3-rolls states, \etc); together they `Voronoi'
tessellate  the curved manifold in which the reduced strange attractor is
embedded by a finite set of hyperplane
tiles\rf{rowley_reconstruction_2000,RoSa00}.

The physical task is to, for a given dynamical flow, pick a set of
qualitatively distinct {\template s} whose \Poincare sections are locally tangent
to the strange attractor\ES{Do you mean locally transverse?}.
A \Poincare section is a
($d\!-\!1$)\dmn\ hyperplane. If we pick another {\template} point
$\slicep{}^{(2)}$, it comes along with its own \Poincare section. Any
neighboring pair of $(d\!-\!1)$\dmn\ \Poincare sections intersects in a `ridge'
(`boundary,' `edge'), a $(d\!-\!2)$\dmn\ hyperplane, easy to compute.
A global atlas so constructed should be sufficiently
fine-grained: each `chart' or `tile,' bounded by ridges to
neighboring \Poincare sections, should be sufficiently small.

Follow an ant as it traces out a trajectory
$\sspRed{}^{(1)}(\tau)$, confined to the \Poincare section $\PoincS{}^{(1)}$
The
moment $\braket{(\sspRed{}^{(1)}(\tau)-\slicep{}^{(2)})}{\vel{}{}^{(2)}}$ changes
sign, the ant has crossed the ridge and  continues its merry stroll within the
$\PoincS{}^{(2)}$ \Poincare section.

There is a rub, though - you need to know how to pick the
neighboring {\template s}. This is a reflection of the flaw inherent in use
of a \Poincare section hyperplane globally: a \Poincare section is derived from the Euclidean
notion of distance, but for nonlinear flows the distance has to be
measured curvilinearly, along unstable
manifolds\rf{Christiansen97,DasBuch}. We nevertheless have to stick with
tessellation by linearized tangent spaces, as curvilinear charts appear
computationally prohibitive. Perhaps a glance at
\reffig{fig:Tesselate} helps visualize the problem; imagine that the
tiles belong to the
\Poincare sections through {\template} points on these orbits. One could slide
{\template s} along their trajectories until the pairs of straight line
segments connecting neighboring {\template} points are minimized, but
that is not physical: one would like the dynamical trajectories to cross
ridges as continuously as possible. So how is one to orient
the {\template s} relative to each other? The choice of the first template fixes all {\em
relative phases} to the succeeding {\template s}, as was demonstrated in
\refref{SCD07}: the universe of all other solutions is rigidly fixed
through a web of heteroclinic connections between them. This insight
garnered from study of a 1-dimensional \KS\ PDE is more remarkable still
when applied to the plane Couette flow\rf{GHCW07}, with 3-$d$ velocity
fields. The {\em relative phase} between
two {\template s} is thus fixed  by the shortest heteroclinic connection,
a rigid bridge from one neighborhood to the next. Once the relative phase
between the templates {\template s} is fixed, so are their their \Poincare sections,
\ie, their tangent hyperplanes, and their intersection, \ie, the  ridge
joining them.


\section{Conclusions}
    \label{sec:concl}

If a physical flow is confined to a lower-dimensional manifold, one should
use this fact to implement a dimensionality reduction.  In this
paper we have investigated dimensionality reduction by the method
linear \Poincare sections, a linear
procedure particularly simple and practical to implement.

 However, while a \Poincare section intersects each trajectory
  in a neighborhood of a {\template} only once, extended globally any
\Poincare section intersects a longer trajectory segment multiple times. So
it makes no sense physically to use one
\Poincare section globally. We propose instead to construct a global atlas by
deploying sets of linear \Poincare sections as charts of
neighborhoods of the most important (relative) equilibria and/or
(relative) periodic orbits.

Such global atlas should be sufficiently fine-grained.

It should be emphasized that the atlas so constructed retains the
dimensionality of the original problem. The full dynamics is faithfully
retained, we are \emph{not} constructing a lower-dimensional model of the
dynamics. Neighborhoods of unstable \eqva\ and \po s are dominated by
their unstable and least contracting stable eigenvalues and are, for all
practical purposes, low-dimensional. Traversals of the ridges are,
however, higher dimensional. For example, crossing from the neighborhood
of a two-rolls state into the neighborhood of a three-rolls state entails
going through a pattern `defect,' a rapid transient whose precise
description requires many Fourier modes. Nevertheless, the recent
progress on separation of `physical' and `hyperbolically isolated'
covariant Lyapunov
vectors\rf{PoGiYaMa06,ginelli-2007-99,YaTaGiChRa08,TaGiCh09} gives us
hope that the proposed atlas could provide a systematic and controllable
framework for construction of lower-dimensional models of `turbulent'
dynamics of dissipative PDEs.

One still has to show that the method can be implemented for a truly
high-dimensional flow. In \refref{SCD07} it was found that the
coexistence of four \eqva, two \reqva\ (traveling waves) and a nested
\fixedsp\ structure in an effectively $8$-dimensional \KS. Even more
importantly, a dimensionality reduction of pipe and \pCf s remains an
outstanding challenge\rf{ACHKW11}.


	%\begin{acknowledgments}
	\medskip
	\noindent{\bf Acknowledgments}
We sought in vain Hugues Chat\'e's sage counsel on how to reduce
physical dimensionality, but none was forthcoming - hence this
confusion.
We are grateful to
??,
and in particular R.L.~Davidchack
for spirited exchanges.
P.C. work was supported by  Glen Robinson Jr.
and the National Science Foundation grant
DMR~0820054. 	
	%\end{acknowledgments}



\Remarks

\remark{Nomenclature.}{\label{rem:Lyapunov}
From ChaosBook.org: ``
											\toCB
\beq
{d \over dt} \deltaX(\xInit,t) =
{\Mvar}(\ssp) \,  \deltaX(\xInit,t)
	\,,\qquad \ssp=\ssp(\xInit,t)
\, .
\label{lin_odes}
\eeq

\beq
{\Mvar}_{ij}(\ssp) ={\pde \vel_i(\ssp)\over \pde \ssp_j  }
\ee{DerMatrix}
\beq
    \deltaX(t) = \jMps^t(\xInit) \deltaX_0
    \,, \qquad
\jMps^t_{ij}(\xInit)
  =  \left. {\pde \ssp_i(t) \over \pde \ssp_j} \right|_{\ssp=\xInit}
\, .
\label{hOdes}
\eeq
The Jacobian matrix $\jMps$ is sometimes referred to as the
{\em fundamental solution matrix} or simply
{\em fundamental matrix}, a name inherited
from the theory of linear ODEs.
It is also sometimes called the {\em Fr\'echet derivative} of the
nonlinear mapping $\flow{t}{\ssp}$. It is often denoted $Df$, but
for our needs (we shall have to sort through a plethora of
related \jacobianMs) matrix notation $\jMps$ is more
economical. $\jMps$ describes the deformation of an
infinitesimal neighborhood at
finite time $t$ in the co-moving frame of $\ssp(t)$. %,
\beq
 \jMps^t(\ssp_\stagn) = e^{{\Mvar}_\stagn t}
    \,,\qquad
 {\Mvar}_\stagn={\Mvar}(\ssp_\stagn)
\,.
\ee{eqPointStab}

The nomenclature tends to be a bit confusing.
In referring to \velgradmat) ${\Mvar}$ defined in \refeq{DerMatrix}
as the {``\stabmat''} we follow Tabor\rf{Tabor89}.
												\toCB
All too often ${\Mvar}$,
which describes the instantaneous shear of the trajectory
point $\ssp(\xInit,t)$ is referred to as the `Jacobian
matrix,'
a particularly unfortunate usage when one considers
linearized stability of an \eqv\ point \refeq{eqPointStab}.
What Jacobi had in mind in his
1841 fundamental paper\rf{Jacobi1841} on the determinants today known as
`jacobians' were transformations between different coordinate frames,
\ie, in the present context, the
$\jMps^{t} = e^{t {\Mvar}}$ transformation of the local linearized
neighborhood.
These are dimensionless quantities,
while dim\-ens\-ion\-ally ${\Mvar}_{ij}$ is 1/[time].
More unfortunate still is referring to
$\jMps^{t} = e^{t {\Mvar}}$ as an `evolution operator,' which here
% (see \refsect{s_aver_ev_op})
refers to something altogether different.
In this book
\jacobianM\ $\jMps^{t}$ always refers to
\refeq{hOdes},
the linearized deformation after a finite time $t$, either for a
continuous time flow, or a discrete time mapping.
''

Additional notes:
Manos \etal\rf{MaSkAn11} refer to \refeq{lin_odes}
as the ``variational equations'' and
$\Mvar$ in \refeq{DerMatrix} as ``the Jacobian matrix'' of  $\vel$.

(2011-03-04 Predrag: this is tedious, give up for now...)


} %end of \remark{Nomenclature

\RemarksEnd
