\ifsvnmulti
 \svnkwsave{$RepoFile: lyapunov/intro.tex $}
 \svnidlong {$HeadURL$}
 {$LastChangedDate$}
 {$LastChangedRevision$} {$LastChangedBy$}
 \svnid{$Id$}
\fi

Bits and pieces for a putative article,

``The physical dimension of a \KS\ flow.''

\section{Introduction}
\label{sect:intro}

Inspired by the global
studies of \refrefs{PoGiYaMa06,ginelli-2007-99,YaTaGiChRa08,TaGiCh09} of
`covariant Lyapunov vectors' we use Floquet vectors of unstable \po s to
identify the \emph{local} number of degrees of freedom that capture the
physics of a `turbulent' PDE on a compact spatial domain. That number is
proportional to the system size, for \KS\ flow roughly four times the
number of positive/marginal Floquet (or Lyapunov) exponents, and twice its
Kaplan-Yorke estimate.

The idea is to coarsely cover the \emph{continuous-symmetry reduced}
nonlinear strange attractor with a set of
hyperplanes\rf{SiCvi10,FrCv11}, as in \reffig{fig:Tesselate}. Any
adjacent pair intersects in a `ridge' hyperplane of one less dimension.
So our task is to, for a given strange attractor, pick a set of \Poincare
section-fixing points, such that each local section is approximately tangent to the
strange attractor. For simplicity here we shall consider only the flows
with a 1 continuous parameter (the time), postponing the more general
case of continuous symmetries to a happier time. An example of such
dynamics is the \KS\ flow restricted to the antisymmetric
subspace\rf{Christiansen97,lanCvit07};
some examples of flows with continuous
symmetries are the \KS\ flow on a periodic domain\rf{SCD07}, the pipe
flow\rf{ACHKW11} and the \pCf\rf{GHCW07}.

We propose to construct a global atlas by deploying a finite
number of linear \Poincare sections in
neighborhoods of the most important equilibria and/or periodic orbits
as local charts.
This is the periodic-orbit implementation of the idea of {\statesp\
tessellation} so dear to professional cyclists, \reffig{fig:Tesselate}.

% In FrCv11.tex replace by Tesselate.png
%%%%%%%%%%%%%%%%%%%%%%%%%%%%%%%%%%%%%%%%%%%%%%%%%%
\SFIG{f_1_08_1}
{}{
Smooth dynamics  (left frame) tesselated by the skeleton of periodic
points, together with their linearized neighborhoods, (right frame).
Indicated are segments of two 1-cycles and a 2-cycle that alternates
between the neighborhoods of the two 1-cycles, shadowing first one of the
two 1-cycles, and then the other.
}{fig:Tesselate} %{Hyp} %{fig6} and {tr:fig6} in ChaosBook
%
%%%%%%%%%%%%%%%%%%%%%%%%%%%%%%%%%%%%%%%%%%%%%%%%%%
%


\section{Charting the \statesp}
	\label{sec:chart}
% extracted from siminos/froehlich/slice/FrCv11.tex   2011-02-20

Work on \KS\ suggests how to proceed: it was shown in
\refrefs{lanCvit07,SCD07} that for turbulent/chaotic systems a set of
\Poincare sections is needed to capture the dynamics. The choice of
sections should reflect the dynamically dominant patterns seen in the
solutions of nonlinear PDEs. We propose to construct a global atlas of
the dimensionally \reducedsp\ $\pSRed$ by deploying linear \Poincare
sections $\PoincS{}^{(j)}$ across neighborhoods of the qualitatively most important
patterns $\slicep{}^{(j)}$.
We shall refer to these states as \emph{\template s}, each
represented in the \statesp\ $\pS$ of the system by
a \emph{\template\ point} $\slicep$. Together with the velocity
field at this point, a template defines a linear \Poincare section,
an affine hyperplane $\sspRed \in \PoincS$,
\beq
    \vel(\slicep) \cdot (\sspRed - \slicep)= 0
\,,
\ee{locTransvVar1}
locally normal to the $\vel(\slicep)$ at the \template\ point $\slicep$.
(For a discussion of inner products, see
\refappe{def:innerProduct}.)
(For a motivation, see
\refappe{s-POs-flows} \emph{Newton method for flows}.)
Each \Poincare section $\PoincS{}^{(j)}$, provides a local chart
at $\slicep{}^{(j)}$ for a
neighborhood of an important, qualitatively distinct class of solutions
(2-rolls states, 3-rolls states, \etc); together they `Voronoi'
tessellate  the curved manifold in which the reduced strange attractor is
embedded by a finite set of hyperplane
tiles\rf{rowley_reconstruction_2000,RoSa00}.

The physical task is to, for a given dynamical flow, pick a set of
qualitatively distinct {\template s} whose \Poincare sections are locally tangent
to the strange attractor\ES{Do you mean locally transverse?}.
A \Poincare section is a
($d\!-\!1$)\dmn\ hyperplane. If we pick another {\template} point
$\slicep{}^{(2)}$, it comes along with its own \Poincare section. Any
neighboring pair of $(d\!-\!1)$\dmn\ \Poincare sections intersects in a `ridge'
(`boundary,' `edge'), a $(d\!-\!2)$\dmn\ hyperplane, easy to compute.
A global atlas so constructed should be sufficiently
fine-grained: each `chart' or `tile,' bounded by ridges to
neighboring \Poincare sections, should be sufficiently small.

Follow an ant as it traces out a trajectory
$\sspRed{}^{(1)}(\tau)$, confined to the \Poincare section $\PoincS{}^{(1)}$
The
moment $\braket{(\sspRed{}^{(1)}(\tau)-\slicep{}^{(2)})}{\vel{}{}^{(2)}}$ changes
sign, the ant has crossed the ridge and  continues its merry stroll within the
$\PoincS{}^{(2)}$ \Poincare section.

There is a rub, though - you need to know how to pick the
neighboring {\template s}. This is a reflection of the flaw inherent in use
of a \Poincare section hyperplane globally: a \Poincare section is derived from the Euclidean
notion of distance, but for nonlinear flows the distance has to be
measured curvilinearly, along unstable
manifolds\rf{Christiansen97,DasBuch}. We nevertheless have to stick with
tessellation by linearized tangent spaces, as curvilinear charts appear
computationally prohibitive. Perhaps a glance at
\reffig{fig:Tesselate} helps visualize the problem; imagine that the
tiles belong to the
\Poincare sections through {\template} points on these orbits. One could slide
{\template s} along their trajectories until the pairs of straight line
segments connecting neighboring {\template} points are minimized, but
that is not physical: one would like the dynamical trajectories to cross
ridges as continuously as possible. So how is one to orient
the {\template s} relative to each other? The choice of the first template fixes all {\em
relative phases} to the succeeding {\template s}, as was demonstrated in
\refref{SCD07}: the universe of all other solutions is rigidly fixed
through a web of heteroclinic connections between them. This insight
garnered from study of a 1-dimensional \KS\ PDE is more remarkable still
when applied to the plane Couette flow\rf{GHCW07}, with 3-$d$ velocity
fields. The {\em relative phase} between
two {\template s} is thus fixed  by the shortest heteroclinic connection,
a rigid bridge from one neighborhood to the next. Once the relative phase
between the templates {\template s} is fixed, so are their their \Poincare sections,
\ie, their tangent hyperplanes, and their intersection, \ie, the  ridge
joining them.


\section{Conclusions}
    \label{sec:concl}

If a physical flow is confined to a lower-dimensional manifold, one should
use this fact to implement a dimensionality reduction.  In this
paper we have investigated dimensionality reduction by the method
linear \Poincare sections, a linear
procedure particularly simple and practical to implement.

 However, while a \Poincare section intersects each trajectory
  in a neighborhood of a {\template} only once, extended globally any
\Poincare section intersects a longer trajectory segment multiple times. So
it makes no sense physically to use one
\Poincare section globally. We propose instead to construct a global atlas by
deploying sets of linear \Poincare sections as charts of
neighborhoods of the most important (relative) equilibria and/or
(relative) periodic orbits.

Such global atlas should be sufficiently fine-grained.

It should be emphasized that the atlas so constructed retains the
dimensionality of the original problem. The full dynamics is faithfully
retained, we are \emph{not} constructing a lower-dimensional model of the
dynamics. Neighborhoods of unstable \eqva\ and \po s are dominated by
their unstable and least contracting stable eigenvalues and are, for all
practical purposes, low-dimensional. Traversals of the ridges are,
however, higher dimensional. For example, crossing from the neighborhood
of a two-rolls state into the neighborhood of a three-rolls state entails
going through a pattern `defect,' a rapid transient whose precise
description requires many Fourier modes. Nevertheless, the recent
progress on separation of `physical' and `hyperbolically isolated'
covariant Lyapunov
vectors\rf{PoGiYaMa06,ginelli-2007-99,YaTaGiChRa08,TaGiCh09} gives us
hope that the proposed atlas could provide a systematic and controllable
framework for construction of lower-dimensional models of `turbulent'
dynamics of dissipative PDEs.

One still has to show that the method can be implemented for a truly
high-dimensional flow. In \refref{SCD07} it was found that the
coexistence of four \eqva, two \reqva\ (traveling waves) and a nested
\fixedsp\ structure in an effectively $8$-dimensional \KS. Even more
importantly, a dimensionality reduction of pipe and \pCf s remains an
outstanding challenge\rf{ACHKW11}.


	%\begin{acknowledgments}
	\medskip
	\noindent{\bf Acknowledgments}
We sought in vain Hugues Chat\'e's sage counsel on how to reduce
physical dimensionality, but none was forthcoming - hence this
confusion.
We are grateful to
??,
and in particular R.L.~Davidchack
for spirited exchanges.
P.C. work was supported by  Glen Robinson Jr.
and the National Science Foundation grant
DMR~0820054. 	
	%\end{acknowledgments}



\Remarks

\remark{Nomenclature.}{\label{rem:Lyapunov}
From ChaosBook.org: ``
											\toCB
\beq
{d \over dt} \deltaX(\xInit,t) =
{\Mvar}(\ssp) \,  \deltaX(\xInit,t)
	\,,\qquad \ssp=\ssp(\xInit,t)
\, .
\label{lin_odes}
\eeq

\beq
{\Mvar}_{ij}(\ssp) ={\pde \vel_i(\ssp)\over \pde \ssp_j  }
\ee{DerMatrix}
\beq
    \deltaX(t) = \jMps^t(\xInit) \deltaX_0
    \,, \qquad
\jMps^t_{ij}(\xInit)
  =  \left. {\pde \ssp_i(t) \over \pde \ssp_j} \right|_{\ssp=\xInit}
\, .
\label{hOdes}
\eeq
The Jacobian matrix $\jMps$ is sometimes referred to as the
{\em fundamental solution matrix} or simply
{\em fundamental matrix}, a name inherited
from the theory of linear ODEs.
It is also sometimes called the {\em Fr\'echet derivative} of the
nonlinear mapping $\flow{t}{\ssp}$. It is often denoted $Df$, but
for our needs (we shall have to sort through a plethora of
related \jacobianMs) matrix notation $\jMps$ is more
economical. $\jMps$ describes the deformation of an
infinitesimal neighborhood at
finite time $t$ in the co-moving frame of $\ssp(t)$. %,
\beq
 \jMps^t(\ssp_\stagn) = e^{{\Mvar}_\stagn t}
    \,,\qquad
 {\Mvar}_\stagn={\Mvar}(\ssp_\stagn)
\,.
\ee{eqPointStab}

The nomenclature tends to be a bit confusing.
In referring to \velgradmat) ${\Mvar}$ defined in \refeq{DerMatrix}
as the {``\stabmat''} we follow Tabor\rf{Tabor89}.
												\toCB
All too often ${\Mvar}$,
which describes the instantaneous shear of the trajectory
point $\ssp(\xInit,t)$ is referred to as the `Jacobian
matrix,'
a particularly unfortunate usage when one considers
linearized stability of an \eqv\ point \refeq{eqPointStab}.
What Jacobi had in mind in his
1841 fundamental paper\rf{Jacobi1841} on the determinants today known as
`jacobians' were transformations between different coordinate frames,
\ie, in the present context, the
$\jMps^{t} = e^{t {\Mvar}}$ transformation of the local linearized
neighborhood.
These are dimensionless quantities,
while dim\-ens\-ion\-ally ${\Mvar}_{ij}$ is 1/[time].
More unfortunate still is referring to
$\jMps^{t} = e^{t {\Mvar}}$ as an `evolution operator,' which here
% (see \refsect{s_aver_ev_op})
refers to something altogether different.
In this book
\jacobianM\ $\jMps^{t}$ always refers to
\refeq{hOdes},
the linearized deformation after a finite time $t$, either for a
continuous time flow, or a discrete time mapping.
''

Additional notes:
Manos \etal\rf{MaSkAn11} refer to \refeq{lin_odes}
as the ``variational equations'' and
$\Mvar$ in \refeq{DerMatrix} as ``the Jacobian matrix'' of  $\vel$.

(2011-03-04 Predrag: this is tedious, give up for now...)


} %end of \remark{Nomenclature

\RemarksEnd

\section{Can the inertial manifold be captured by unstable periodic orbits?}
\label{sec:TaCh11}

\begin{description}

\item[2011-07-21 Takeuchi and Chat\'e]\rf{TaCh11}
% \arXiv{1107.2567},
gave us a draft of \emph{Phys. Rev. Lett.} to read:
\emph{Can the inertial manifold be captured by unstable periodic orbits?},
Kazz says:

`` We show numerical evidence that, as previously conjectured,
covariant Lyapunov vectors form a natural basis to describe the
chaotic solutions of spatially-extended dissipative dynamical
systems, which can be seen as a local linear approximation of the
inertial manifold. This is achieved by studying in detail the
relation between a chaotic trajectory and nearby unstable periodic
orbits, in particular with their covariant Lyapunov vectors.''

It is a nice result; they take Siminos-Davidchack $\period{p} = 10.253$
\po\ for \KS\ on $L = 22$ and run a long ergodic trajectory until it
comes very close to a periodic point $\ssp_p(t^*) $ on $p$. The nice
result is that in neighborhood of that point (forgot to ask whether they
let both the \po\ and the ergodic trajectory evolve, $\ssp(t+t^*) -
\ssp_p(t+t^*)$, or are measuring distance $\ssp(t+t^*) - \ssp_p(t^*)$ from a
static periodic point) the separation vector lies in the 9-dimensional
physical space (and not a subspace of it, for example, the unstable
manifold of $p$). That verifies that $p$ is embedded into the chaotic
attractor (it could have been an isolated but nearby cycle).
Other result to be mindful of:

Kazz uses ``the pseudospectral method with Fourier modes up to a cut off
wave number chosen so that no aliasing may occur.'' I think we do not
worry about this because it is of imprtance only for computing correctly
the high Lyapunove/Floquet eigenvectors.

Most $\jEigvec[j](\ssp_p(t^*))$, Floquet eigenvectors of $p$, do not match
up well with Lyapunov vectors $\jEigvec[j](\ssp(t^*))$. As Evangelos has
tried to get across, that means that while $\ssp(t^*)$ is close to
$\ssp_p(t^*) $ in Euclidean distance, they presumably sit on different
stable / unstable manifold sheets, and will have the corresponding
eigenvectors different.

As is my fate, Kazz resolutely refuses to do anything I ask him to do.
Instead of slicing he minimizes distance by running in circles, so
Froehlich\rf{FrCv11} apparently fails to get across that slice \emph{is
the minimal distance} symmetry reduction. He does not want to look at anything in
\statesp. They insist that \po s cannot be used to diagnose the physical dimension.
I told him that vat majority of \po s are hyperbolic and cannot exhibit angle crossings,
but the claim is still in the draft as the main result, number (2). Und so weiter.

Chat\'e says that \statesp\ visualization of regions with tangencies is a
low hanging fruit, and that they will do it (but not in this paper).

Sorry, getting too late - need to go to bed.

\end{description}


\newpage
\section{Draft of the paper(s) blog}
\label{sect:DraftBlog}

\begin{description}

\item[2011-07-21 Kazz] writes:
``
Since the last video-conference, Hugues and I have much developed our
study on the physical-spurious splitting in the KS equation using
periodic orbits.

Here I send you a summary of our latest results, which we think are
interesting and important enough to prepare a letter on that (Predrag's
copy is dowloadable
\HREF{http://www.scribd.com/fullscreen/60910182?access_key=key-2e56zhcvvnqo7amghf6j}
{110721Kazz.pdf} from Scribd.com). Attached is a summary and not a draft of
an article, and therefore words and structure are not refined enough, but
we think we can take it as a prototype of our article on this subject. We
didn't put your names as the authors simply because we don't know yet
what you think about our results.
,,

\item[2011-07-24 Siminos]
Is the spatial shift a multiple of the spatial discretization step?

\item[2011-07-25 Kazz]
No. First of all I use the quasispectral method for space discretization,
so I don't have any step in space. The spatial shift is expressed as a
shift in the phase of the Fourier modes, $e^{ik\ell}$ for a shift ${\ell}$ (which
can be any real number) for mode $k$.

However, it doesn't mean that our spatial and temporal shifts are really
the optimal ones. To determine the shift l, I look at the most weighted
Fourier mode ($k=2$ in the studied case) and match the shift with this
particular mode. Because this mode has a dominating weight in the power
spectrum, the value of l determined thereby is quite close to the best
one, but strictly speaking not. As for the resolution of the temporal
shift, it is obviously restricted by the choice of the time step.

\item[2011-07-24 Siminos]
Spatial shift in eq. (2) of 2011-07-21 draft seems to take care
of that, although this approach seems to be somewhat inefficient.

\item[2011-07-25 Kazz]
If you have a better idea, please let me know!

\item[2011-07-24 Siminos]
    Using ``the vector defined by the points of minimal distance along
    these two trajectories as a local approximation to the inertial
    manifold'' makes no sense to me. The local approximation to the
    inertial manifold is spanned by the transverse periodic orbit Floquet
    eigenvectors at that periodic point (simplest to take a local
    \Poincare section normal to $v(a)$ at $a$).

\item[2011-07-25 Kazz]
The problem about the Floquet eigenvectors is that, we cannot distinguish
physical and spurious modes for each periodic orbit (as explained in our
draft). Moreover, the goal here is to find a direct link between the
physical/spurious modes defined in the tangent space and the inertial
manifold defined in the phase space. This is why we studied the vector
connecting nearby points on the periodic orbit and on the chaotic
trajectory, both of which should be in the inertial manifold.

\item[2011-07-24 Siminos]
    I've actually raised the objection about the generic trajectory
    belonging to a different fold of the unstable manifold of the
    periodic orbit

\item[2011-07-25 Kazz]
I'm not sure about which results you are talking here and against what
I'm supposed to counter-argue. What I showed in Fig.5(c) is the numerical
observation that the angle between the difference vector (defined above;
between two nearby points in the inertial manifold) and the subspace
spanned by all the physical modes goes to zero proportionally to the
distance. This shows that for infinitesimal distances the difference
vector indeed lives in the "physical manifold" defined by all the
physical modes (NB: the unstable manifold is not the right one) in the
fold of the linear approximation. I don't make any reference to other
folds.

\item[2011-07-21 Kazz] writes in 110721Kazz.pdf:
``
the fact that for UPOs the Lyapunov exponents and the
CLVs are simply given by its Floquet multipliers (eigenvalues)
and eigenvectors
''

\item[2011-07-25 Predrag 2 Kazz]
A factoid, not correct - next public update of ChaosBook.org should fix
that. The Lyapunov exponents are logs of \emph{singular values} of
$\jMps^{T}_p\jMps_p$ which are \emph{not} Floquet exponents of
\jacobianM\ $\jMps_p$. Only in the $t_p \to \infty$ limit the Oseledec
(1968) theorem\rf{lyaos} applies (and I am not even sure of that).

\item[2011-07-21 Kazz] writes in 110721Kazz.pdf:
``
When integrating UPOs, we retake the initial condition on each cycle
(adjustment in the order of 10-13 in u for the UPO mainly studied in the
following), in order to supprss numerical error accumulated during
integration. We confirm that the values of their Lyapunov exponents
computed thereby agree with the direct calculation of the Floquet
multipliers.
''

\item[2011-07-25 Predrag 2 Kazz]
Been through this many times, but anyway: wouldn't it be swell to
generalize Ginelli \etal\ so you do not run around a \po\ more than once?
The thing that is attractive to us about your methods is that you can
compute all eigenvectors, while we can compute only those whose
contracting multipliers (divided by the most expanding one) are above the
machine precision.

\item[2011-07-21 Kazz] writes in 110721Kazz.pdf:
``
In contrast, it seems impossible to set a well-defined threshold for the
CLVs of the UPOs.
''

\item[2011-07-25 Predrag 2 Kazz]
Do you understand my ``Daily blog'' comment  [2011-06-30 Predrag 2 Kazz]?
Almost all unstable cycles are nicely hyperbolic - only a very few cycles
close to tangencies will show glimpses of non-hyperbolicity. It's OK that
you do not see small angles on hyperbolic \po s - it should be this way.
If you actually looked at \statesp\ visualizations of the
flow\rf{SCD07,GHCW07,GibsonMovies}, you would most likely see that
episodes of small inter-eigenvector angles are spatially localized, and
concentrated on the most prominent near tangencies. Read also [2011-06-30
Predrag] on Kobayashi and Saiki, and [2011-07-02 Predrag] on Inubushi.

This communication in unspeakably laborious - it's like trying to deal
with reactor in Fukushima, while sitting in my Chicao apartment. Wouldn't
it be easier if you just checked out the repository, wrote into it
yourself, checked in the draft, and I edited it directly? Guess too
computer-nerdy for the 20th century ways...

\item[2011-07-27 Kazz 2 Evangelos] As you may follow, Predrag sent Hugues
and me comments on our summary of the results. There, Predrag criticizes
our definition of the distance between the chaotic trajectory and the
periodic orbit, referring us to the Froehlich-Cvitanovi\'c paper. I
understood that he proposes using the symmetry reduction to compute the
exact distance. My question is how to do this in the KS equation. I had a
look at your thesis Chapter 8, but I am not sure if I correctly
understand it and if this is what we want.

In your thesis, you found invariant quantities under the SO(2) rotation
of the Fourier modes and plotted trajectories on these invariants, right?
Then, to obtain the exact distance, we need to know all the (N-1)
invariants (N is the number of the Fourier modes), which you did not list
all in your thesis. Am I right? Moreover, as they are nonlinear functions
of the Fourier coefficients, I don't know how to compute the geometric
distance in the phase space (i.e., the one I defined in the draft with
the optimal spatial and temporal shifts). This is essential to compare
the approach and separation rates with the Lyapunov exponents, and
moreover to reach our central conclusion shown in Fig.~5(c).

By the way, if you argue only such invariants, I don't understand why you
specify the rotation angle theta as in Eq.(138). I also wonder how you
automated the computation of the (N-1) invariants in Mathematica (or
whatever... I just wonder if I can compute all of them).

At any rate, I have the impression that our approximate distance is
accurate enough to discuss physics behind and honestly don't understand
why Predrag takes it so seriously.

\item[2011-07-29 Evangelos 2 Kazz]
I am not sure I understand what exactly you do to "match the shift"
and why is it approximate? I will therefore try to describe what I
would do and see how we differ.

First of all, I am glad someone is reading my thesis! However it
should not be necessary for your purposes. Our goal was global
state-space visualization and return maps, so we had to work with a
global (and well behaved) basis of invariant variables. This is why we
had to go beyond eq. (138) of my thesis and write out the invariants
in explicit form (to cure singularities later on).

However, to compare the distance of a point on a periodic orbit and a
point on a generic trajectory, I think it is enough to follow the steps
which led to eq. (138) of my thesis, with any convenient "slice fixing"
convention. In your case you could indeed use the k=2 mode or more
general choices discussed in our papers (Siminos and
Cvitanovi\'c\rf{SiCvi10}, Froehlich and Cvitanovi\'c\rf{FrCv11}). The
choice does not have to be a global one but could depend on the point
on the periodic orbit under consideration (e.g. take your slice to be
transverse to the group action at the given point on the periodic
orbit). Having computed (numerically) the angle that maps the points
under consideration onto the chosen slice, you can then apply the
group transformations with the angle that you compute. This would be a
rotation in all Fourier modes by the same angle. Measuring distance of
points mapped onto the slice is exactly what you need: distance of the
initial points up to translations. There is no approximation, except
in the numerical determination of the angle.

Predrag makes a great deal out of this because there are certain
annoying singularities in the resulting transformations which can lead
to trouble if you care about the global picture or global quantities.
However, as you only care to compute the distance of two points up to
translations, I do not see any problem. Any numerical difficulties can
be eliminated by redefining your slice fixing condition when problems
arise.

For me this slice fixing business is just a precise way to eliminate
the spatial shifts while ensuring meaningful Euclidean norms and
without the need to write out the explicit invariants.

\item[2011-08-01 Kazz 2 Evangelos] Hugues will be in Dresden during
the whole conference
\HREF{http://www.pks.mpg.de/~dynact11/}{http://www.pks.mpg.de/~dynact11/}
and I will be on 5-9 and 14-16 September.

Here I explain what I did to compare the chaotic trajectory and the periodic
orbit... honestly I don't think it's different from yours when we deal
with small distances, at which the two trajectories look almost the same.

Imagine a situation where the trajectory and the orbit are so close to each
other that they look practically the same, except spatial translation.
Let's say, their spatial profiles are different by length 6, with the highest
peak of the trajectory at x=7 and of the orbit at x=13, for example.
Then, to compute the distance between the two trajectories, we would like
to shift both or either of them in space in order that they overlap as well
as possible: in this case we can shift the trajectory by 6, or we can also
shift the trajectory by -7 and the orbit by -13. The former and the latter
are what I and you did, respectively, if I correctly understood your method.

Of course, in reality, the question is how to estimate the appropriate
length of the spatial shift, because the location of the highest peak is
not a good quantity to look at (it can change abruptly when there are two
equally high peaks). I used the Fourier mode k=2, which is the most dominating,
and thus the most reliable one. The location of its peaks is encoded as
the angle (argument) of its Fourier coefficient, so I shifted that of
the chaotic trajectory so that it becomes equal to that of the periodic orbit.
I of course shifted the angle of all the other Fourier modes accordingly.

I am aware that my way to determine the spatial shift is only approximative,
since I look only at the k=2 mode, but it seems that you don't have a better
solution for this, right?

A question: do you actually distinguish the "spatial shift" and the "angle"
that you explained to me? It seems to me that you do in your reply, but
I don't see how. To me this is the same thing, because the spatial shift
I'm dealing with is nothing but a SO(2) group action. It's true that
I don't fix any "slide" for the comparison, which has the highest peak
always at a fixed position, say x=0, in the example above, but it's obvious
that the two methods in the example are equivalent.

\item[2011-08-10 Evangelos 2 Kazz] I still cannot understand why you say that
what you do is approximate.
You are free to shift in space any solution by any amount you like and
therefore you are also free to fix the angle of any Fourier mode and
rotate the rest appropriately. From what you write, I understand that
it is the latter approach that you finally implement so you also work
in Fourier space. Talking about aligning the highest peaks is just a
visual way to interpret the rotation in Fourier space, right?

In our SIADS paper notation, a spatial shift by $l$ is equivalent to a
rotation in the complex plane of the kth Fourier mode by $2\pi k l /L$.

\item[2011-08-11 Kazz 2 Evangelos] To answer your question, you're in principle
right in saying that
``You are free to shift in space any solution by any amount you like.''
However, the question is how to determine the optimal shift that really yields
the minimum distance. Since I have to compute it at every time step,
I cannot use a time-consuming code here, and so I gave up computing the
optimal shift. Instead, I estimated an appropriate value of the shift by
looking only at the most dominating, k=2 Fourier mode, in place of their
entire spatial profiles. The shift is then simply the phase difference
between the k=2 Fourier coefficients of the two solutions.
In my view, this is the best compromise between accuracy and efficiency.

\item[2011-08-11 Evangelos] I do not understand. Fourier transform is
linear so it shouldn't matter which mode you use to compute the shift.
Do I miss something here?



Sorry, getting too late - need to go to bed.
\end{description}
