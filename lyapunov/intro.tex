\ifsvnmulti
 \svnkwsave{$RepoFile: lyapunov/intro.tex $}
 \svnidlong {$HeadURL$}
 {$LastChangedDate$}
 {$LastChangedRevision$} {$LastChangedBy$}
 \svnid{$Id$}
\fi

\section{The physical dimension of a \KS\ flow}
\label{sect:intro}

We use Floquet or `covariant Lyapunov'
vectors (CLV) of unstable \po s to identify the number of degrees of freedom
that capture the physics of a `turbulent' PDE on a compact spatial
domain\rf{PoGiYaMa06,ginelli-2007-99,YaTaGiChRa08,TaGiCh09}.
That number is proportional to the
system size, for \KS\ flow roughly four times the number of
 positive/marginal Floquet (or Lyapunov) exponents, twice its
Kaplan-Yorke estimate.

%\item[2010-04-07 Predrag]
The idea is to coarsely cover the nonlinear strange attractor with a set
of hyperplanes, as in \reffig{fig:Tesselate}. For any pair, they
intersect in a 'boundary' hyperplane, of one less dimension. So our task
is to, for a given strange attractor, pick a set of slice-fixing points,
such that each is approximately tangent to the strange attractor, and the
singularity hyperplanes are eliminated by requiring that they lie either
on the `wrong' side of the slice-slice intersection, or somewhere where
the strange attractor does not tread.

We need to make a global chart by deploying both linear slices and linear
Poincar\'e sections in neighborhoods of the most important (relative)
equilibria and/or (relative) periodic orbits (those are tricky, because
slice fixing points must lie in the full \statesp, and have no symmetry,
so most of the solutions we have are not good as they stand). This is the
periodic-orbit generalization of the idea of {\statesp\ tessellation} so
dear to professional cyclists, \reffig{fig:Tesselate}.

% In FrCv11.tex replace by Tesselate.png
%%%%%%%%%%%%%%%%%%%%%%%%%%%%%%%%%%%%%%%%%%%%%%%%%%
\SFIG{f_1_08_1}
{}{
Smooth dynamics  (left frame) tesselated by the skeleton of periodic
points, together with their linearized neighborhoods, (right frame).
Indicated are segments of two 1-cycles and a 2-cycle that alternates
between the neighborhoods of the two 1-cycles, shadowing first one of the
two 1-cycles, and then the other.
}{fig:Tesselate} %{Hyp} %{fig6} and {tr:fig6} in ChaosBook
%
%%%%%%%%%%%%%%%%%%%%%%%%%%%%%%%%%%%%%%%%%%%%%%%%%%
%

\Remarks

\remark{Nomenclature.}{\label{rem:Lyapunov}
From ChaosBook.org: ``
\beq
{d \over dt} \deltaX_i(\xInit,t) =
\sum_j \left.{\pde \vel_i  \over \pde \ssp_j}(\ssp)
       \right|_{\ssp=\ssp(\xInit,t)}
         \deltaX_j(\xInit,t)
\, .
\label{lin_odes}
\eeq

\beq
{\Mvar}_{ij}(\ssp) ={\pde \vel_i(\ssp)\over \pde \ssp_j  }
\ee{DerMatrix}
\beq
    \deltaX(t) = \jMps^t(\xInit) \deltaX_0
    \,, \qquad
\jMps^t_{ij}(\xInit)
  =  \left. {\pde \ssp_i(t) \over \pde \ssp_j} \right|_{\ssp=\xInit}
\, .
\label{hOdes}
\eeq
This Jacobian matrix is sometimes referred to as the
{\em fundamental solution matrix} or simply
{\em fundamental matrix}, a name inherited
from the theory of linear ODEs.
It is also sometimes called the {\em Fr\'echet derivative} of the
nonlinear mapping $\flow{t}{\ssp}$. It is often denoted $Df$, but
for our needs (we shall have to sort through a plethora of
related Jacobian matrices) matrix notation $\jMps$ is more
economical. $\jMps$ describes the deformation of an
infinitesimal neighborhood at
finite time $t$ in the co-moving frame of $\ssp(t)$. %,
\beq
 \jMps^t(\ssp_\stagn) = e^{{\Mvar}_\stagn t}
    \,,\qquad
 {\Mvar}_\stagn={\Mvar}(\ssp_\stagn)
\,.
\ee{eqPointStab}

The nomenclature tends to be a bit confusing.
In referring to \velgradmat) ${\Mvar}$ defined in \refeq{DerMatrix}
as the {``\stabmat''} we follow Tabor\rf{Tabor89}.
Sometimes ${\Mvar}$,
which describes the instantaneous shear of the trajectory
point $\ssp(\xInit,t)$ is referred to as the `Jacobian
matrix,'
a particularly unfortunate usage when one considers
linearized stability of an \eqv\ point \refeq{eqPointStab}.
What Jacobi had in mind in his
1841 fundamental paper\rf{Jacobi1841} on the determinants today known as
`jacobians' were transformations between different coordinate frames.
These are dimensionless quantities,
while dim\-ens\-ion\-ally ${\Mvar}_{ij}$ is 1/[time].
More unfortunate still is referring to
$\jMps^{t} = e^{t {\Mvar}}$ as an `evolution operator,' which here
% (see \refsect{s_aver_ev_op})
refers to something altogether different.
In this book
\jacobianM\ $\jMps^{t}$ always refers to
\refeq{hOdes},
the linearized deformation after a finite time $t$, either for a
continuous time flow, or a discrete time mapping.
''

Additional notes:
Manos \etal\rf{MaSkAn11} refer to \refeq{lin_odes}
as the ``variational equations'' and
$\Mvar$ in \refeq{DerMatrix} as ``the Jacobian matrix'' of  $\vel$.

(2011-03-04 Predrag: this is tedious, give up for now...)


} %end of \remark{Nomenclature

\RemarksEnd
